\section{原典与人物}\label{app:sources}

The historical analysis in this paper draws primarily on Sima Qian's
\emph{Shiji} (《史记》, \emph{Records of the Grand Historian},
c.~94~BCE) and Du Mu's \emph{A Fang Gong Fu} (《阿房宫赋》, 825~CE).
This appendix collects the original Classical Chinese passages cited or
referenced in the main text, with English translations and brief
profiles of the historical figures.

\subsection{人物志}\label{app:persons}

\paragraph{刘邦 Liu Bang (256--195 BCE).}
Founder of the Han dynasty. Rose from minor local official (亭长) to
emperor. Near-zero personal combat ability; made himself the cut vertex
of the execution graph (\cref{ex:liubang}).

\begin{quote}
高祖为人,仁而爱人,喜施,意豁如也。常有大度。不事家人生产作业。

\medskip
\emph{Gaozu was a man of benevolence who loved people, was generous in
giving, and broad-minded. He had great magnanimity. He did not engage
in household production or labor.}
\hfill ---《史记·高祖本纪》
\end{quote}

Translation into the framework: Liu Bang himself had no actuation
capability. He could not fight, could not administer, could not
strategize. His sole structural role was as the mandatory routing node.

On first seeing the First Emperor's procession:
\begin{quote}
嗟乎,大丈夫当如此也!

\medskip
\emph{Ah, a great man should be like this!}
\hfill ---《史记·高祖本纪》
\end{quote}

Liu Bang saw the existence proof (\cref{thm:qin}) and wanted to
\emph{be} the system's cut vertex.

His self-assessment after founding the Han dynasty:
\begin{quote}
夫运筹策帷帐之中,决胜於千里之外,吾不如子房。镇国家,抚百姓,给馈饷,不绝粮道,吾不如萧何。连百万之军,战必胜,攻必取,吾不如韩信。此三者,皆人杰也,吾能用之,此吾所以取天下也。项羽有一范增而不能用,此其所以为我擒也。

\medskip
\emph{For devising strategies within a tent and securing victory a
thousand li away, I am not as good as Zhang Liang. For governing the
state, caring for the people, providing supplies, and keeping the grain
roads open, I am not as good as Xiao He. For commanding a million
soldiers, winning every battle and taking every siege, I am not as good
as Han Xin. These three are all heroes---but I can use them. This is
why I won the empire. Xiang Yu had one Fan Zeng but could not use him.
This is why he was captured by me.}
\hfill ---《史记·高祖本纪》
\end{quote}

「吾能用之」(\emph{I can use them}) is the operational definition of
the cut vertex: actuation resides in others, but routing authority
resides in Liu Bang. Every execution chain passes through him. The
character 用 (use/employ) is not metaphorical---it is a precise
description of the cut vertex's function in the execution graph.

\paragraph{项羽 Xiang Yu (232--202 BCE).}
Supreme military commander of the anti-Qin uprising. Maximum actuator
(\cref{ex:liubang}): personal combat ability unmatched in the system.

\begin{quote}
籍长八尺馀,力能扛鼎,才气过人。

\medskip
\emph{[Xiang] Ji was over eight chi tall, could lift a bronze tripod,
and his talent and spirit surpassed all others.}
\hfill ---《史记·项羽本纪》
\end{quote}

\begin{quote}
项王嗔目而叱之,赤泉侯人马俱惊,辟易数里。

\medskip
\emph{The King of Xiang glared and bellowed at him; the Marquis of
Chiquan and his horse both recoiled in terror, retreating several li.}
\hfill ---《史记·项羽本纪》
\end{quote}

On seeing the same procession Liu Bang saw:
\begin{quote}
彼可取而代也!

\medskip
\emph{That one---I can replace him!}
\hfill ---《史记·项羽本纪》
\end{quote}

Liu Bang: ``I want to \emph{be} this.'' Xiang Yu: ``I can
\emph{replace} him.'' One read the existence proof as a system to
inhabit. The other read it as a person to defeat.

Han Xin's assessment of Xiang Yu:
\begin{quote}
项王见人恭敬慈爱,言语呕呕,人有疾病,涕泣分食饮,至使人有功当封爵者,印刓敝,忍不能予。此所谓妇人之仁也。

\medskip
\emph{The King of Xiang is respectful and caring when he meets people;
his speech is warm and gentle. When someone is ill, he weeps and shares
his food and drink. But when a man has earned merit and deserves a
title, he fondles the seal until its edges are worn smooth, and still
cannot bring himself to hand it over. This is what is called the
benevolence of a woman.}
\hfill ---《史记·淮阴侯列传》
\end{quote}

「印刓敝,忍不能予」: the seal is carved and ready, rubbed smooth from
handling, yet he cannot let it go. This is not benevolence---it is the
inability to distribute actuation. A cut vertex that cannot delegate
is a maximum actuator pretending to route.

After conquering the Qin capital:
\begin{quote}
富贵不归故乡,如衣绣夜行,谁知之者!

\medskip
\emph{To be wealthy and noble but not return home is like wearing
embroidered robes at night---who would see it?}
\hfill ---《史记·项羽本纪》
\end{quote}

His last song, at Gaixia (垓下歌):
\begin{quote}
力拔山兮气盖世,时不利兮骓不逝。\\
骓不逝兮可奈何,虞兮虞兮奈若何!

\medskip
\emph{My strength could uproot mountains, my spirit overmastered the
world. / But the times turned against me, and my horse would not go. /
My horse would not go---what can be done? / Yu, oh Yu---what will
become of you?}
\hfill ---《史记·项羽本纪》
\end{quote}

At the bank of the Wu River, refusing to cross:
\begin{quote}
天之亡我,我何渡为!且籍与江东子弟八千人渡江而西,今无一人还,纵江东父兄怜而王我,我何面目见之?

\medskip
\emph{Heaven has destroyed me---why should I cross? I crossed the river
westward with eight thousand sons of Jiangdong, and not one has
returned. Even if the elders of Jiangdong pitied me and made me king,
with what face could I see them?}
\hfill ---《史记·项羽本纪》
\end{quote}

「天之亡我」(\emph{Heaven has destroyed me}). Not heaven. A single
actuator cannot cover the full state space (\cref{ex:liubang}).
Structural failure, not fate.

\paragraph{韩信 Han Xin (?--196 BCE).}
Military genius. Commanded Liu Bang's armies; conquered more territory
than any other general in the Chu--Han war. Pure knife
(\cref{ex:hanxin}): his execution chain was closed---armies obeyed him,
not Liu Bang.

On Liu Bang's ability:
\begin{quote}
陛下不能将兵,而善将将,此乃信之所以为陛下禽也。

\medskip
\emph{Your Majesty cannot command soldiers, but excels at commanding
commanders. This is why I was captured by Your Majesty.}
\hfill ---《史记·淮阴侯列传》
\end{quote}

「善将将」(\emph{excels at commanding commanders}) $=$ cut vertex
property. Liu Bang does not actuate directly; he routes the actuation
of others.

The incident that sealed his fate---requesting the title King of Qi
during wartime:
\begin{quote}
汉王大怒,骂曰:「吾困于此,旦暮望若来佐我,乃欲自立为王!」张良、陈平蹑汉王足,因附耳语……汉王亦悟……遂遣张良立信为齐王。

\medskip
\emph{The King of Han was furious and cursed: ``I am trapped here,
waiting day and night for you to come help me, and you want to make
yourself king!'' Zhang Liang and Chen Ping stepped on the king's foot
and whispered in his ear\ldots\ The King of Han understood\ldots\ and
sent Zhang Liang to install Han Xin as King of Qi.}
\hfill ---《史记·淮阴侯列传》
\end{quote}

Zhang Liang stepping on Liu Bang's foot $=$ recalibrating the search:
the viable path currently requires Han Xin's actuation (wartime phase),
so the knife cannot be cut yet. Granting the title $=$ extending the
horizon. After the phase transition, the knife was cut.

His final words:
\begin{quote}
果若人言,「狡兔死,良狗亨;高鸟尽,良弓藏;敌国破,谋臣亡。」天下已定,我固当亨!

\medskip
\emph{It is as people said: ``When the cunning hare is killed, the
hunting dog is cooked; when the high-flying birds are gone, the good
bow is stored away; when the enemy state is destroyed, the strategist
perishes.'' The empire is settled---naturally I was to be cooked!}
\hfill ---《史记·淮阴侯列传》
\end{quote}

Han Xin quoted the answer but did not parse its fine structure. See
\cref{app:proverb} for the detailed analysis.

\paragraph{萧何 Xiao He (?--193 BCE).}
Chief administrator. Controlled grain supply and the capital during Liu
Bang's campaigns. Half-knife who self-blunted via deliberate
self-corruption (\cref{ex:xiaohe}).

\begin{quote}
相国何买田宅必居穷处,为家不治垣屋。曰:「后世贤,师吾俭;不贤,毋为势家所夺。」

\medskip
\emph{Chancellor He always bought fields and houses in the poorest
locations, and did not repair the walls of his home. He said: ``If my
descendants are worthy, they will follow my example of frugality. If
they are not, the property will be too poor for powerful families to
bother seizing.''}
\hfill ---《史记·萧相国世家》
\end{quote}

The stated reason (frugality for descendants) is a cover story. The
operational function: signal to the king that $\|\Ur\| \to 0$. An
official this visibly degraded in wealth and reputation cannot
coordinate a revolt. This is path~(a) executed through reputation
rather than resignation---pulling $\|\Ur\|$ toward $\bar{U}$
(\cref{thm:meanfield}).

\paragraph{张良 Zhang Liang (?--189 BCE).}
Strategist. Descendant of five generations of prime ministers of the
state of Han. Devoted his fortune to avenging Han's destruction by Qin.
Not a knife (\cref{ex:zhangliang}): pure advisory function---every
execution chain passed through Liu Bang.

\begin{quote}
留侯乃称曰:「家世相韩,及韩灭,不爱万金之资,为韩报仇强秦,天下振动。今以三寸舌为帝者师,封万户,位列侯,此布衣之极,于良足矣。愿弃人间事,欲从赤松子游耳。」乃学辟谷。

\medskip
\emph{The Marquis of Liu said: ``My family served as ministers of Han
for five generations. When Han was destroyed, I did not begrudge ten
thousand gold to seek vengeance against mighty Qin, and the empire
trembled. Now with my three-inch tongue I have become the emperor's
teacher, been enfeoffed with ten thousand households, and ranked as
marquis. For a commoner, this is the pinnacle---it is enough for me.
I wish to abandon worldly affairs and follow the immortal Chi Songzi.''
He then took up the practice of grain abstinence.}
\hfill ---《史记·留侯世家》
\end{quote}

「三寸舌」(\emph{three-inch tongue}) $=$ pure function. No actuation.
Zhang Liang's retirement (辟谷, grain abstinence) is not path~(a)
(放下)---he had no knife to put down. It is confirmation:
$\Ur = \varnothing$ from the first day, now the function itself
is shut down.

\paragraph{商鞅 Shang Yang (?--338 BCE).}
Architect of the Qin reform system (\cref{sec:qin}). Installed the
submartingale reform sequence (\cref{thm:shangyang}) that transformed
Qin from a peripheral state into the unification engine.

\begin{quote}
令民为什伍,而相牧司连坐。不告奸者腰斩,告奸者与斩敌首同赏,匿奸者与降敌同罚。

\medskip
\emph{He organized the people into groups of five and ten households,
to watch over and be jointly liable for each other. Those who failed to
report wrongdoers were cut in half at the waist. Those who reported
wrongdoers received the same reward as those who beheaded enemy
soldiers. Those who harbored wrongdoers received the same punishment
as those who surrendered to the enemy.}
\hfill ---《史记·商君列传》
\end{quote}

This is $\Obs \to \Obs_{\max}$: neighbors as a distributed sensor
network. The reward structure ensures every agent has positive incentive
to maximize observability.

\begin{quote}
商君相秦十年,宗室贵戚多怨望者。……秦惠王车裂商君以徇,曰:「莫如商鞅反者!」遂灭商君之家。

\medskip
\emph{Lord Shang governed Qin for ten years; many among the royal
house and the powerful clans harbored resentment.\ldots\ King Hui of
Qin had Lord Shang torn apart by chariots and displayed, saying: ``Let
none rebel as Shang Yang did!'' His entire clan was exterminated.}
\hfill ---《史记·商君列传》
\end{quote}

The unitary group acts on all vectors without exception
(\cref{thm:shangyang}). The installer is not in the invariant
subspace of the group he created.

\paragraph{范蠡 Fan Li (536--448 BCE).}
Minister of Yue. After helping King Goujian destroy the state of Wu,
Fan Li immediately left (\cref{sec:dollar}). Accumulated three
fortunes, dispersed two---ensuring $\Ur$ never crossed the knife
threshold.

\begin{quote}
范蠡遂去,自齐遗大夫种书曰:「飞鸟尽,良弓藏;狡兔死,走狗烹。越王为人长颈鸟喙,可与共患难,不可与共乐。子何不去?」

\medskip
\emph{Fan Li then departed. From Qi he sent a letter to Grand Officer
Zhong, saying: ``When the birds are gone, the good bow is stored away;
when the cunning hare is killed, the hunting dog is cooked. The King
of Yue has a long neck and a bird's beak---one can share hardship with
him, but not prosperity. Why do you not leave?''}
\hfill ---《史记·越王勾践世家》
\end{quote}

This passage---written to his colleague Wen Zhong (文种), who did not
leave and was subsequently forced to commit suicide---is the origin of
the proverb Han Xin quoted two centuries later (\cref{rem:hanxin}).
See \cref{app:proverb} for the fine structure.

\begin{quote}
范蠡……乃乘扁舟浮于江湖,变名易姓……止于陶,……十九年之中三致千金,再分散与贫交疏昆弟。

\medskip
\emph{Fan Li\ldots\ took a small boat and drifted on the rivers and
lakes, changing his name\ldots\ He settled at Tao\ldots\ In nineteen
years he amassed a fortune of a thousand gold three times, and twice
distributed it to his poor friends and distant kin.}
\hfill ---《史记·货殖列传》
\end{quote}

Three accumulations, two dispersals. Each time $\Ur$ approached the
threshold, Fan Li reset it. The dollar is a knife precursor
(\cref{sec:dollar}); Fan Li ensured the precursor never converted.

\paragraph{陈胜 Chen Sheng (?--208 BCE).}
Farmer and conscript laborer. When Qin's water reached zero
(\cref{thm:dumu}), the pawn became a knife.

\begin{quote}
陈胜佐之,并杀两尉。召令徒属曰:「公等遇雨,皆已失期,失期当斩。藉第令毋斩,而戍死者固十六七。且壮士不死即已,死即举大名耳,王侯将相宁有种乎!」

\medskip
\emph{Chen Sheng helped him, and together they killed the two officers.
He gathered the conscripts and said: ``You have all been delayed by
rain and missed the deadline. The penalty for missing the deadline is
death. Even if you are not executed, six or seven out of ten who go to
garrison duty will die. Besides, when a true man dies, he dies with
his name known---are kings and nobles born to their station?''}
\hfill ---《史记·陈涉世家》
\end{quote}

「失期当斩」(\emph{miss the deadline, face execution}): $w = 0$.
Every path leads to death. The viability axiom now applies to the pawn
himself (\cref{prop:binary}). 「王侯将相宁有种乎」:
$U_{\text{pawn}}: \varnothing \to \neq\varnothing$. The breakpoint
has dissolved.

\paragraph{冯谖 Feng Xuan (3rd century BCE).}
Retainer of Lord Mengchang of Qi. Architect of the observability
reduction strategy.

\begin{quote}
冯谖曰:「狡兔有三窟,仅得免其死耳。今君有一窟,未得高枕而卧也。请为君复凿二窟。」

\medskip
\emph{Feng Xuan said: ``A cunning hare has three burrows, and barely
manages to escape death. You, my lord, have only one burrow---you
cannot yet sleep with your head high on the pillow. Allow me to dig
two more burrows for you.''}
\hfill ---《战国策·齐策四》
\end{quote}

Three burrows $=$ three alternative positions $=$ reducing the king's
detection function $\Obs$ coverage. If the king cannot observe your
$\Ur$, you move from ``knife'' to ``hidden knife''---still dangerous,
but outside the king's strategy space.

\paragraph{魏征 Wei Zheng (580--643 CE).}
Chief advisor to Emperor Taizong of Tang. Articulated the water
dynamics (\cref{sec:water}) as a political principle.

\begin{quote}
臣闻古语云:「君,舟也;人,水也。水能载舟,亦能覆舟。」陛下以为可畏,诚如圣旨。

\medskip
\emph{Your minister has heard an ancient saying: ``The ruler is a boat;
the people are the water. Water can carry the boat, and water can
capsize the boat.'' That Your Majesty considers this worthy of fear is
truly wise.}
\hfill ---《贞观政要》
\end{quote}

Water carries the boat ($w > 0 \implies$ pawn serves $\implies$ king
sovereign) and capsizes the boat ($w = 0 \implies$ pawn $\to$ knife
$\implies$ king absorbed). Wei Zheng (630~CE) and Du Mu (825~CE) stated
the same theorem (\cref{thm:dumu}). Wei Zheng gave the intuition.
Du Mu gave the proof. This paper gives the formalization.

\subsection{飞鸟尽良弓藏:韩信之误读}\label{app:proverb}

Han Xin's final words (\cref{rem:hanxin}) quote the proverb
originating from Fan Li. The proverb is treated in Chinese historical
tradition as a single lesson: ``after the war, the meritorious are
killed.'' This reading is imprecise. The proverb contains three
structurally distinct resources with three distinct fates:

\begin{center}
\begin{tabular}{@{}llll@{}}
\toprule
\textbf{Proverb} & \textbf{Resource type} & $\Ur$ &
\textbf{Fate} \\
\midrule
飞鸟尽,良弓\textbf{藏} & Bow (no autonomous actuation) &
$= \varnothing$ & \textbf{Stored}---not destroyed \\
狡兔死,走狗\textbf{烹} & Dog (autonomous actuator) &
$\neq \varnothing$ & \textbf{Cooked}---eliminated \\
敌国破,谋臣\textbf{亡} & Strategist---depends on $\Ur$ &
? & Depends on classification \\
\bottomrule
\end{tabular}
\end{center}

The bow is \emph{stored} (藏), not \emph{cooked} (烹). The proverb
itself distinguishes between the two fates at the level of the verb.
The dog---an autonomous actuator that can hunt independently---is
killed. The bow---a tool that cannot shoot itself---is merely put away.

Han Xin conflated all three. The corrected version:

\begin{itemize}
  \item \textbf{Bow} ($\Ur = \varnothing$): Zhang Liang. Retired and
  survived. A bow that stores itself.
  \item \textbf{Dog} ($\Ur \neq \varnothing$): Han Xin. Killed. A dog
  that refused to stop hunting.
  \item \textbf{Half-tool} ($\Ur$ partially non-empty): Xiao He.
  Imprisoned, then released. A dog that deliberately blunted its teeth.
\end{itemize}

The survival strategies are also encoded in the proverb:

\begin{center}
\begin{tabular}{@{}lp{5.5cm}l@{}}
\toprule
\textbf{Strategy} & \textbf{Mechanism} & \textbf{Applicable to} \\
\midrule
Don't let the birds disappear & Maintain
$J_{\text{king}}(r) > 0$ (king still needs you) & Bow
($\Ur = \varnothing$) \\
Three burrows (冯谖) & Reduce $\Obs$ coverage &
Dog ($\Ur \neq \varnothing$) \\
Self-blunting / dispersal & $\Ur \to \varnothing$ &
Any type \\
\bottomrule
\end{tabular}
\end{center}

Han Xin needed the third strategy (put down the knife). He chose the
zeroth (do nothing). The answer was in the proverb he quoted---弓
is \emph{stored}, 狗 is \emph{cooked}---but he did not parse the
distinction.

Fan Li understood this. His letter to Wen Zhong describes the bow and
the dog. His advice: become the bow (leave). But Wen Zhong was the
prime minister of Yue---his administrative network made him a dog,
not a bow. He could not leave without first setting $\Ur \to \varnothing$,
and a prime minister's administrative network is not so easily
relinquished. Wen Zhong stayed. Wen Zhong died.

\subsection{阿房宫赋}\label{app:afanggongfu}

Du Mu (杜牧, 803--852~CE) wrote the \emph{Rhapsody on the Epang
Palace} in 825~CE. \Cref{thm:dumu} formalizes its central argument.
The full text follows in traditional characters, with English
translation and framework mapping.

\begin{remark}[Pronunciation of 阿房]
The standard reading of 阿房宫 is \emph{Ep\'ang G\=ong}
(or \emph{\=Ef\'ang G\=ong} per the Guifan Dictionary),
not \emph{\=Af\'ang G\=ong}.
The character 阿 takes the reading \emph{\=e} in this compound; 房
takes the reading \emph{p\'ang}.
We retain the romanisation \emph{A Fang Gong Fu} throughout because it
is the naive character-by-character reading---each character pronounced
in isolation, stripped of relational context.
That is the point: the ``correct'' pronunciation is a phase function of
the compound, not an intrinsic property of the individual character.
The mispronunciation instantiates the thesis.
\end{remark}

\subsubsection*{第一段:存在性证明的物理实例}

\begin{quote}
六王畢,四海一。蜀山兀,阿房出。覆壓三百餘里,隔離天日。驪山北構而西折,直走咸陽。二川溶溶,流入宮牆。五步一樓,十步一閣。廊腰縵迴,簷牙高啄。各抱地勢,鈎心鬬角。盤盤焉,囷囷焉,蜂房水渦,矗不知其幾千萬落。長橋臥波,未雲何龍?複道行空,不霽何虹?高低冥迷,不知西東。歌臺暖響,春光融融。舞殿冷袖,風雨淒淒。一日之內,一宮之間,而氣候不齊。

\medskip
\emph{The six kings finished, the four seas unified, the Shu mountains
stripped bare, and the Epang Palace rose. It pressed down over three
hundred li, blocking out sun and sky. From the northern foot of
Mt.~Li it turned west, running straight to Xianyang. Two rivers flowed
gently into its walls. Every five paces a tower, every ten paces a
pavilion; corridors wound and turned, eaves rose like pecking beaks;
each structure embraced the terrain, their ridges interlocking.
Spiraling and curving, like beehives and whirlpools, towering---who
knows how many thousands of clusters. The long bridge lay over the
waves---if not clouds, why a dragon? The skyway crossed the air---if
not after rain, why a rainbow? In the haze of heights and depths, one
could not tell west from east. On the singing stages, warm sounds like
spring sunlight; in the dancing halls, cold sleeves like wind and rain.
Within a single day, within a single palace, the seasons differed.}
\end{quote}

All resources converge on a single node (Xianyang). Star-graph dispatch
(\cref{sec:qin}). The palace's scale $=$ the center node's extractable
throughput, made physical.

\subsubsection*{第二段:相变后的掠夺}

\begin{quote}
妃嬪媵嬙,王子皇孫,辭樓下殿,輦來於秦。朝歌夜絃,爲秦宮人。明星熒熒,開粧鏡也。緑雲擾擾,梳曉鬟也。渭流漲膩,棄脂水也。煙斜霧橫,焚椒蘭也。雷霆乍驚,宮車過也。轆轆遠聽,杳不知其所之也。一肌一容,盡態極妍。縵立遠視,而望幸焉,有不得見者,三十六年。

\medskip
\emph{The consorts and attendants, the princes and grandsons of the six
kings, left their towers and descended their halls, riding in carriages
to Qin. Morning songs and evening strings---they became Qin's palace
women. Bright stars glittering---that was opening their mirrors. Green
clouds in disorder---that was combing their morning hair. The Wei River
rising oily---that was discarded cosmetics. Smoke slanting, mist
spreading---that was burning pepper and orchid. Thunder suddenly
startling---a palace carriage passing. Wheels rumbling into the
distance, vanishing beyond knowing. Every curve and every face brought
to its utmost beauty, standing gracefully, gazing far, hoping for the
emperor's favor. Some waited thirty-six years and never saw him.}
\end{quote}

Resources collected at the cut vertex, but the cut vertex's bandwidth
is finite. One node cannot process all chains simultaneously.
``Thirty-six years without being seen'' $=$ star-graph bottleneck.

\subsubsection*{第三段:生存公理的普遍性}

\begin{quote}
燕趙之收藏,韓魏之經營,齊楚之精英,幾世幾年,剽掠其人,倚疊如山。一旦不能有,輸來其間。鼎鐺玉石,金塊珠礫,棄擲邐迤。秦人視之,亦不甚惜。

嗟乎!一人之心,千萬人之心也。秦愛紛奢,人亦念其家。奈何取之盡錙銖,用之如泥沙!使負棟之柱,多於南畝之農夫;架梁之椽,多於機上之工女;釘頭磷磷,多於在庾之粟粒;瓦縫參差,多於周身之帛縷;直欄橫檻,多於九土之城郭;管絃嘔啞,多於市人之言語:使天下之人不敢言而敢怒。獨夫之心,日益驕固。戍卒叫,函谷舉。楚人一炬,可憐焦土。

\medskip
\emph{The treasures of Yan and Zhao, the collections of Han and Wei,
the finest goods of Qi and Chu---plundered from their people over how
many generations, piled up like mountains. One day they could keep them
no more, and all was shipped here. Tripods used as pots, jade treated
as stone, gold discarded in heaps, pearls scattered like gravel---the
Qin people saw these and did not much care.}

\emph{Alas! One man's heart is the heart of ten thousand men. Qin loved
extravagance, yet people also cherish their homes. Why take from them
down to the last coin, and spend it like mud and sand? The pillars
outnumbered the farmers; the rafters outnumbered the weavers; the
nail-heads outnumbered the grain in the granaries; the tile-seams
outnumbered the threads in a bolt of silk; the railings outnumbered the
city walls of the nine provinces; the cacophony of pipes and strings
outnumbered the speech of the marketplace. The people of the empire
dared not speak, but dared to be angry. The tyrant's heart grew daily
more arrogant and obstinate. The garrison soldiers cried out, Hangu
Pass was taken, a torch from Chu, and---alas---scorched earth!}
\end{quote}

「一人之心,千萬人之心也」$=$ the viability axiom is not the king's
exclusive property (\cref{ax:viability}). Every agent has the same
axiom. 「取之盡錙銖」$=$ $dw/dt \ll 0$ (\cref{thm:dumu}).
「不敢言而敢怒」$=$ $\Ur = \varnothing$ still (not speaking $=$ no
autonomous actuation), but energy accumulates. 「獨夫之心,日益驕固」
$=$ the cut vertex receives no feedback---all correction channels have
been eliminated. 「戍卒叫,函谷舉」$=$ $U_{\text{pawn}}: \varnothing
\to \neq\varnothing$ (\cref{prop:binary}). Phase transition fires.

\subsubsection*{第四段:定理}

\begin{quote}
嗚呼!\textbf{滅六國者,六國也,非秦也。族秦者,秦也,非天下也。}嗟夫!使六國各愛其人,則足以拒秦。使秦復愛六國之人,則遞三世可至萬世而爲君,誰得而族滅也。\textbf{秦人不暇自哀,而後人哀之。後人哀之,而不鑑之,亦使後人而復哀後人也。}

\medskip
\emph{Alas! It was the six states that destroyed the six states, not
Qin. It was Qin that destroyed Qin, not the world. Had the six states
each loved their own people, they would have had enough to resist Qin.
Had Qin, in turn, loved the people of the six states, it could have
passed from the third generation to the ten-thousandth and remained
sovereign---who could have destroyed it? The people of Qin had no
leisure to mourn for themselves, and later generations mourned for them.
But if later generations mourn them without learning from them, they
will only cause yet later generations to mourn for the later
generations in turn.}
\end{quote}

This is \cref{thm:dumu} in prose. 「滅六國者六國也」$=$ the six states
depleted their own water (internal knife dynamics, coordination cost
$O(n^2)$). 「族秦者秦也」$=$ $w(t) \to 0 \implies \text{pawn} \to
\text{knife} \implies \text{king absorbed}$. The causal chain is
internal.

「後人哀之而不鑑之」: the theorem is time-invariant. It does not care
about dynasty names, centuries, or regime labels. It checks three
conditions: is $\Ur \neq \varnothing$? Is the loop closed? Is water
being maintained? Du Mu did not say ``you are Qin.'' He said: check
the premises.

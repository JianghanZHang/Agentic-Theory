\chapter{三足之鼎 --- The Cooperative Triad}\label{app:triad}

The preceding appendices applied the framework to individual agents
within a fixed system.  This appendix applies it to the system
itself: a nation-state whose viability depends on the simultaneous
health of three coupled variables.  The mathematical object is a
\emph{cooperative triad}---a three-dimensional differential
inclusion with positive cross-coupling---and the central result
is that the viability of such a system is a \emph{transient fixed
point}: structurally possible, stable while it lasts, and destroyed
abruptly when any single coupling channel weakens below threshold.

Two structural corollaries follow.  First, one variable of the
triad ($x_3$: human capital flux) is not independent but is itself
a function of the other two, making the triad \emph{reducible} and
its fragility worse than the product form suggests.  Second, a
system with no cut vertex (\cref{def:cutvertex})---a distributed
mesh---is immune to external elimination; its only death mode is
internal phase transition.

\section{The coupled system}\label{sec:triad-system}

\begin{definition}[Cooperative triad]\label{def:triad}
A \emph{cooperative triad} is a dynamical system on
$S = \R^3_{\geq 0}$ with mean-field dynamics
\begin{equation}\label{eq:triad}
  x'_i \;=\; \alpha_i\, x_i
  \;+\; \sum_{j \neq i} \beta_{ij}\, x_j
  \;-\; \gamma_i\, x_i^2,
  \qquad i = 1, 2, 3,
\end{equation}
where the coefficients satisfy:
\begin{enumerate}[label=(\roman*)]
  \item $\beta_{ij} > 0$ for all $i \neq j$
    \quad (\emph{cooperative}: cross-feeding is positive);
  \item $\gamma_i > 0$ for all $i$
    \quad (\emph{dissipative}: growth saturates);
  \item $\alpha_i \geq 0$ for all $i$
    \quad (\emph{self-reinforcing}: each variable sustains itself
    at rate $\alpha_i$).
\end{enumerate}
The full set-valued dynamics (\cref{def:di}) are
$x'_i(t) \in F_i(x(t))$ where
$F_i(x) = \{f_i(x) + \eta : |\eta| \leq \delta_i\}$ for a
noise bound $\delta_i \geq 0$.
\end{definition}

The three terms in \eqref{eq:triad} encode:
$\alpha_i x_i$ is self-reinforcement proportional to current level;
$\sum_{j \neq i} \beta_{ij} x_j$ is cross-feeding from the other
two variables; and $-\gamma_i x_i^2$ is saturation (diminishing
returns, bureaucratic overhead, depletion).

\begin{definition}[Triad viability kernel]\label{def:triad-kernel}
For thresholds $\epsilon = (\epsilon_1, \epsilon_2, \epsilon_3)
\in \R^3_{> 0}$, the \emph{triad viability kernel} is
\[
  K_\epsilon \;:=\;
  \bigl\{\, x \in \R^3_{\geq 0} \;:\;
  x_i \geq \epsilon_i \;\;\text{for all } i \,\bigr\}.
\]
The threshold $\epsilon_i$ is the critical value below which the
$i$-th variable's contribution to the cross-feeding reverses sign:
below $\epsilon_i$, the $i$-th leg drags the others down rather
than lifting them.
\end{definition}

\section{The tangential condition}\label{sec:triad-tangential}

\begin{proposition}[Boundary rescue]\label{prop:boundary-rescue}
At the boundary face $\partial_i K_\epsilon :=
\{x \in K_\epsilon : x_i = \epsilon_i\}$, the tangential
condition (\cref{thm:viability-di}) reduces to
\begin{equation}\label{eq:rescue}
  \sum_{j \neq i} \beta_{ij}\, x_j
  \;\geq\;
  \gamma_i\, \epsilon_i^2 \;-\; \alpha_i\, \epsilon_i
  \;=:\; \theta_i.
\end{equation}
When $\alpha_i \geq \gamma_i \epsilon_i$, the threshold
$\theta_i \leq 0$ and the condition is automatically satisfied:
self-reinforcement alone prevents exit.  When
$\alpha_i < \gamma_i \epsilon_i$, the cross-feeding from the
other two variables must compensate.
\end{proposition}

\begin{proof}
The contingent cone to $K_\epsilon$ at $x$ with $x_i = \epsilon_i$
is $T_{K_\epsilon}(x) = \{v \in \R^3 : v_i \geq 0\}$
(\cref{def:contingent}).  The tangential condition
$f(x) \in T_{K_\epsilon}(x)$ requires $f_i(x) \geq 0$:
\[
  \alpha_i \epsilon_i + \sum_{j \neq i} \beta_{ij} x_j
  - \gamma_i \epsilon_i^2 \geq 0,
\]
which rearranges to \eqref{eq:rescue}.
\end{proof}

\begin{remark}[The rescue is mutual]\label{rem:mutual-rescue}
The inequality \eqref{eq:rescue} is the viability condition at
the $i$-th face: when variable~$i$ is at its minimum, the
\emph{other two} must be strong enough to pull it back.
By the cooperative structure ($\beta_{ij} > 0$), this pull is
guaranteed as long as $x_j$ is large.  The failure mode is
cascade: if $x_j$ drops, the rescue of $x_i$ weakens, which
reduces the rescue of $x_k$, which further weakens $x_j$.
Positive feedback amplifies both growth and decline.
\end{remark}

\section{Stability and bifurcation}\label{sec:triad-bifurcation}

\begin{theorem}[Existence, stability, and coupling threshold]
\label{thm:triad}
\hfill
\begin{enumerate}[label=(\alph*)]
  \item \textbf{Existence.}  The system \eqref{eq:triad} has a
    positive fixed point $x^* \in \R^3_{> 0}$.
  \item \textbf{Stability.}  All eigenvalues of the Jacobian
    $J = Df|_{x^*}$ have strictly negative real part:
    $x^*$ is locally asymptotically stable.
  \item \textbf{Coupling threshold.}  $x^* \in K_\epsilon^\circ$
    if and only if, for every $i \in \{1,2,3\}$,
    \begin{equation}\label{eq:coupling-threshold}
      \sum_{j \neq i} \beta_{ij}\, x_j^*
      \;>\;
      \theta_i
      \;=\;
      \gamma_i \epsilon_i^2 - \alpha_i \epsilon_i.
    \end{equation}
    There exists a critical coupling matrix
    $B^* = (\beta_{ij}^*)$ such that
    $x^* \in K_\epsilon^\circ$ when $\beta_{ij} > \beta_{ij}^*$,
    and $x^*$ exits $K_\epsilon$ through face
    $\partial_k K_\epsilon$ when $\beta_{kj}$ drops below
    $\beta_{kj}^*$ for some $j \neq k$.
  \item \textbf{Stable until gone.}  At the bifurcation
    ($\beta_{kj} = \beta_{kj}^*$, so that $x_k^* = \epsilon_k$),
    all eigenvalues of $J$ remain strictly negative.  The system
    does not lose resilience gradually: it is stable until the
    fixed point exits $K_\epsilon$, at which point viability is
    lost abruptly.
\end{enumerate}
\end{theorem}

\begin{proof}
\textbf{(a)}  Define the map $T: \R^3_{\geq 0} \to \R^3_{> 0}$
by $T_i(x) = (\alpha_i + \sqrt{\alpha_i^2 + 4\gamma_i
\sum_{j \neq i}\beta_{ij} x_j})\, /\, (2\gamma_i)$,
the positive root of
$\gamma_i z^2 - \alpha_i z - \sum_{j \neq i}\beta_{ij} x_j = 0$.
$T$ is continuous, monotone increasing, and maps $[0, M]^3$ to
itself for $M$ sufficiently large (since $T_i$ grows as
$O(\sqrt{M})$ while $M$ grows linearly).  By Brouwer's
fixed-point theorem, $T$ has a fixed point $x^* \in [0,M]^3$.
Positivity: $T_i(x) > 0$ for all $x \geq 0$.

\textbf{(b)}  The Jacobian at $x^*$ has entries
$J_{ii} = \alpha_i - 2\gamma_i x_i^*$ (negative, since the
fixed-point equation gives
$\gamma_i x_i^* > \alpha_i$) and
$J_{ij} = \beta_{ij} > 0$ for $i \neq j$.
The matrix $-J$ is a $Z$-matrix (non-positive off-diagonal).
We show $-J$ is a nonsingular $M$-matrix by exhibiting a positive
vector $w > 0$ with $(-J)w > 0$.  Take $w = x^*$.
The $i$-th component of $(-J)x^*$ is
\begin{align*}
  \bigl((-J)\,x^*\bigr)_i
  &= (2\gamma_i x_i^* - \alpha_i)\,x_i^*
     - \sum_{j \neq i} \beta_{ij}\, x_j^* \\
  &= 2\gamma_i (x_i^*)^2 - \alpha_i x_i^*
     - \sum_{j \neq i} \beta_{ij}\, x_j^* \\
  &= 2\gamma_i (x_i^*)^2
     - \underbrace{\bigl(\alpha_i x_i^*
       + \textstyle\sum_{j \neq i} \beta_{ij}\, x_j^*\bigr)}
       _{= \gamma_i (x_i^*)^2 \text{ by the FP equation}} \\
  &= \gamma_i (x_i^*)^2 \;>\; 0.
\end{align*}
By the $M$-matrix characterisation ($Z$-matrix + $\exists\, w > 0$
with $Aw > 0$ $\Rightarrow$ nonsingular $M$-matrix), all
eigenvalues of $-J$ have positive real part, hence all eigenvalues
of $J$ have negative real part.

\textbf{(c)}  At the fixed point,
$\sum_{j \neq i} \beta_{ij} x_j^* =
\gamma_i (x_i^*)^2 - \alpha_i x_i^*$.
Since $g(z) = \gamma_i z^2 - \alpha_i z$ is strictly increasing
for $z > \alpha_i/(2\gamma_i)$ and $x_i^* > \alpha_i/\gamma_i >
\alpha_i/(2\gamma_i)$, requiring $x_i^* \geq \epsilon_i$ is
equivalent to \eqref{eq:coupling-threshold}.
The critical coupling $B^*$ is defined implicitly by the binding
constraint $x_k^*(B^*) = \epsilon_k$; by the implicit function
theorem (the Jacobian $J$ is nonsingular by~(b)), $x^*$ depends
continuously on $B$, and the transition is a bifurcation.

\textbf{(d)}  At the bifurcation $x_k^* = \epsilon_k$, the
proof of~(b) still applies: $(-J)x^*_i = \gamma_i(x_i^*)^2 > 0$
for all $i$ (since $x_i^* \geq \epsilon_i > 0$).
All eigenvalues of $J$ remain strictly negative.
\end{proof}

Part~(d) is the mathematical surprise: the system does not
announce its impending collapse through loss of resilience or
critical slowing down.  It is stable until the coupling drops
below threshold, at which point the fixed point exits
$K_\epsilon$ and viability is lost in a single bifurcation.
The perception of gradual decline is an artefact: what drifts
gradually are the coupling parameters $\beta_{ij}(t)$, not the
system's stability.

\section{Product fragility}\label{sec:triad-fragility}

\begin{definition}[Product Lyapunov function]\label{def:product-lyapunov}
The \emph{product Lyapunov function} for the triad kernel
$K_\epsilon$ is
\[
  V_\Pi(x) \;:=\; \prod_{i=1}^3 (x_i - \epsilon_i).
\]
$V_\Pi$ vanishes on $\partial K_\epsilon$ and is positive on
$K_\epsilon^\circ$.  The viability metric
$g_{V_\Pi} = V_\Pi^{-2}\, g_S$ (\cref{def:viab-metric}) makes
$(K_\epsilon^\circ,\, g_{V_\Pi})$ a complete Riemannian manifold
(\cref{prop:viab-complete}).
\end{definition}

\begin{proposition}[Single-coordinate collapse]\label{prop:single-collapse}
If any single coordinate satisfies
$x_k(t) \to \epsilon_k$ as $t \to T$, then
$V_\Pi(x(t)) \to 0$ regardless of the other coordinates.
The $g_{V_\Pi}$-distance from any interior point to the face
$\partial_k K_\epsilon$ is infinite:
\[
  d_{g_{V_\Pi}}(x,\, \partial_k K_\epsilon) \;=\; +\infty.
\]
\end{proposition}

\begin{proof}
$V_\Pi = (x_k - \epsilon_k) \prod_{j \neq k}(x_j - \epsilon_j)
\leq M^2 (x_k - \epsilon_k) \to 0$, where
$M = \max_{j \neq k} \sup_t (x_j(t) - \epsilon_j)$ is finite
by dissipativity.
For the distance estimate, parametrise a curve
$\gamma(s)$ with only the $k$-th coordinate varying.
Its $g_{V_\Pi}$-length is
\[
  L_{g_{V_\Pi}}(\gamma)
  = \int \frac{|\gamma'(s)|}{V_\Pi(\gamma(s))}\, ds
  \geq \frac{1}{M^2}
  \int \frac{|x_k'(s)|}{x_k(s) - \epsilon_k}\, ds,
\]
which diverges as
$\int du/u = -\log(x_k - \epsilon_k) \to \infty$
when $x_k \to \epsilon_k$.
\end{proof}

\begin{remark}[The dependent variable]\label{rem:dependent}
The third variable of the triad ($x_3$: human capital flux) is
not structurally independent.  Immigration selection presupposes
institutional permission ($x_1$ controls who may arrive), and
retention requires resource reward ($x_2$ controls who stays).
The autonomous component of $x_3$ is therefore
\[
  x_3 \;=\; h(x_1, x_2) \;+\; \eta_3,
\]
where $h$ captures the coupling ($h$ increasing in both arguments)
and $\eta_3$ is the genuinely exogenous component (the pool of
potential emigrants in source countries).  In the triad dynamics
\eqref{eq:triad}, this dependence is already encoded in the
coupling terms $\beta_{31} x_1 + \beta_{32} x_2$, but the
structural implication is sharper: $x_3$ cannot be rescued
independently.  When $x_1$ weakens (institutions restrict
permission) or $x_2$ weakens (resources stop rewarding), $x_3$
declines \emph{even without a direct shock to $x_3$ itself}.
The rescue condition \eqref{eq:rescue} at the $x_3$-face
requires precisely the variables that $x_3$ depends on.

This makes $x_3$ the most fragile coordinate of the triad:
it is the only one that cannot self-rescue
($\theta_3 = \gamma_3 \epsilon_3^2 - \alpha_3 \epsilon_3$
is large when $\alpha_3$ is small, meaning weak
self-reinforcement), and its cross-feeding comes from the
variables it structurally depends on.  The product Lyapunov
function $V_\Pi$ inherits this fragility: $x_3$ is generically
the first coordinate to hit its threshold, and by
\cref{prop:single-collapse}, this suffices to collapse
$V_\Pi$ to zero.
\end{remark}

\section{The mesh immunity theorem}\label{sec:mesh}

The triad describes a system whose viability depends on
maintaining three coupled variables above threshold.
A separate structural question is: can such a system be
\emph{destroyed from outside}?

\begin{definition}[Mesh graph]\label{def:mesh}
An execution graph $G = (V, E)$ (\cref{def:exgraph}) is a
\emph{mesh} if it contains no cut vertex
(\cref{def:cutvertex}): for every $v \in V$,
$G \setminus \{v\}$ is connected.
Equivalently, $G$ is $2$-connected.
\end{definition}

\begin{theorem}[Mesh immunity]\label{thm:mesh-immunity}
Let $G$ be a mesh.  Then:
\begin{enumerate}[label=(\alph*)]
  \item \textbf{No targeted elimination.}  For every $v \in V$,
    $G \setminus \{v\}$ is connected: removing any single node
    does not disconnect the execution graph.
    The elimination strategy of \cref{thm:cutvertex} (make
    yourself the cut vertex, then remove the sword) requires
    a cut vertex to exist.  In a mesh, no such vertex exists.
  \item \textbf{The only death is phase transition.}  The
    mean actuation field $\bar{U}$ (\cref{thm:meanfield}) is
    computed over all nodes.  In a mesh, the mean is a
    \emph{bulk} quantity: no single node's removal changes
    $\bar{U}$ by more than $O(1/|V|)$.  The system's sword
    threshold $\bar{U} + \tau(\Obs)$ is therefore stable under
    single-node perturbation.  The only mechanism that moves
    $\bar{U}$ below the viability threshold is a
    \emph{collective} shift: a phase transition in which the
    coupling constants (ideology, coercion, reward) drop below
    threshold simultaneously across the mesh.
\end{enumerate}
\end{theorem}

\begin{proof}
\textbf{(a)} is the definition of $2$-connectivity
(\cref{def:mesh}).

\textbf{(b)}  Let $\bar{U} = \frac{1}{n}\sum_{i=1}^n \|U_i\|$
be the mean actuation.  Removing node $k$ changes the mean to
$\bar{U}' = \frac{1}{n-1}\sum_{i \neq k} \|U_i\|$.
Then $|\bar{U}' - \bar{U}| = |\bar{U} - \|U_k\||/(n-1)$, which
is $O(1/n)$ since both $\bar{U}$ and $\|U_k\|$ are bounded.  For the sword threshold
$\bar{U} + \tau(\Obs)$ to cross a critical value, the shift must
be $\Omega(1)$, requiring $\Omega(n)$ nodes to change
simultaneously---a phase transition.
\end{proof}

\begin{remark}[Historical instantiation of mesh immunity]
\label{rem:mesh-history}
No communist party in history has been eliminated by external
targeted removal.  The structural reason is
\cref{thm:mesh-immunity}(a): a Leninist party is organised as a
distributed execution mesh (democratic centralism $=$ every node
both routes and executes), and removing any single node
(leader, cadre, cell) does not disconnect the graph.
Every historical end of a communist party is an internal phase
transition (\cref{thm:mesh-immunity}(b)): the Soviet Union
(1991), the Eastern Bloc (1989--91), the Kuomintang's
one-party state in Taiwan (1987--2000).  In each case, the
mean actuation field $\bar{U}$ (ideological commitment $\times$
coercive capacity) dropped below the viability threshold
\emph{from within}, not by external excision.
The same immunity holds for any $2$-connected organisation:
religious orders, insurgent networks, distributed autonomous
organisations.  The topology, not the ideology, determines
the death mode.
\end{remark}

\section{Instantiation: the American republic}\label{sec:america}

The cooperative triad \eqref{eq:triad} admits a canonical
instantiation:

\begin{center}
\renewcommand{\arraystretch}{1.25}
\begin{tabular}{@{}clp{6.5cm}@{}}
\toprule
\textbf{Variable} & \textbf{Name} & \textbf{Content} \\
\midrule
$x_1$ & Institutional capacity &
  Constitutional architecture, rule of law,
  separation of powers, peaceful transfer \\
$x_2$ & Resource base &
  Continental landmass, two-ocean buffer,
  navigable rivers, energy reserves \\
$x_3$ & Human capital flux &
  Immigration selection, meritocratic mobility,
  civic identity untied to ethnicity \\
\bottomrule
\end{tabular}
\end{center}

The six coupling channels:

\begin{center}
\renewcommand{\arraystretch}{1.25}
\begin{tabular}{@{}ccp{7cm}@{}}
\toprule
$\beta_{ij}$ & \textbf{Channel} & \textbf{Mechanism} \\
\midrule
$\beta_{12}$ & $x_2 \to x_1$ &
  Resource abundance funds institutional capacity \\
$\beta_{21}$ & $x_1 \to x_2$ &
  Property rights enable resource development \\
$\beta_{13}$ & $x_3 \to x_1$ &
  Talented populace creates and reforms institutions \\
$\beta_{31}$ & $x_1 \to x_3$ &
  Civic identity and rule of law attract talent \\
$\beta_{23}$ & $x_3 \to x_2$ &
  Talent develops and exploits resources \\
$\beta_{32}$ & $x_2 \to x_3$ &
  Abundance rewards effort, retaining talent \\
\bottomrule
\end{tabular}
\end{center}

\begin{proposition}[The American fixed point]\label{prop:american-fp}
Under this instantiation, the cooperative fixed point $x^*$
satisfies $x^* \in K_\epsilon^\circ$ during the epoch
$\sim$1865--1965 (post--Civil War consolidation through peak
institutional and demographic expansion).
The coupling threshold
(\cref{thm:triad}(c)) was maintained by:
\begin{enumerate}[label=(\roman*)]
  \item $\beta_{21}$ large: the Homestead Act (1862) and
    patent law converted institutional protection directly into
    resource development;
  \item $\beta_{31}$ large: the civic-identity model
    (``American'' $=$ allegiance to a document, not a bloodline)
    sustained the immigration selection filter;
  \item $\beta_{13}$ large: successive waves of immigrants
    built, staffed, and reformed institutions (public schools,
    universities, civil service).
\end{enumerate}
\end{proposition}

\begin{proof}[Verification]
Each coupling channel is independently checkable against
historical data.  The proposition is structural, not causal:
it asserts that the three variables were simultaneously above
threshold and mutually reinforcing, which is verified by the
simultaneous presence of (i)~institutional stability (no
constitutional crisis between 1865 and 1965), (ii)~resource
expansion (continental infrastructure, energy production), and
(iii)~demographic growth driven by immigration
(waves of 1880--1924, post-1945).
The tangential condition \eqref{eq:rescue} held at each face:
when any single variable weakened (e.g.\ institutional stress
during Reconstruction), the other two provided sufficient
cross-feeding to prevent exit from $K_\epsilon$.
\end{proof}

\begin{remark}[Phase transition and the sword]
\label{rem:triad-sword}
The word ``once'' in ``once considered great'' is the detection
of a weakening coupling.  In the framework of \cref{prop:phase},
each variable $x_i$ of the triad is an autonomous actuator
(\cref{def:sword}, condition~(1)): its dynamics $f_i$ can drive
the state independently.  When the cross-feeding is strong
($\beta_{ij} > \beta_{ij}^*$), each variable is a
\emph{tool}---its autonomous actuation reinforces the others.
When the coupling weakens below threshold, the same autonomous
dynamics become a \emph{sword}: a variable in decline drags the
others down through the positive feedback channel.

By \cref{thm:triad}(d) (stable until gone), the transition is
not gradual.  The coupling parameters $\beta_{ij}(t)$ may drift
slowly, but the system remains at a stable fixed point until the
bifurcation.  Then the fixed point exits $K_\epsilon$ and
viability collapses.  ``Greatness'' was a transient fixed point
(\cref{thm:triad}), not an identity.  The product Lyapunov
function $V_\Pi = \prod_i (x_i - \epsilon_i)$
(\cref{def:product-lyapunov}) ensures that the collapse is
\emph{multiplicative}: a decline in any single variable degrades
the entire system's viability measure, regardless of the other
two (\cref{prop:single-collapse}).

By \cref{rem:dependent}, $x_3$ (human capital) is the most
fragile coordinate: it depends on both $x_1$ (institutional
permission to arrive) and $x_2$ (resource reward for staying).
One does not ``select'' immigrants; one creates conditions
($x_1, x_2$) under which immigration self-selects.  When either
condition weakens, the selection filter breaks---not because
the immigrants changed, but because the triad's internal
coupling dropped below the rescue threshold
\eqref{eq:rescue}.
\end{remark}

\begin{remark}[The triad is not American]\label{rem:not-american}
Nothing in \cref{def:triad} is specific to any nation.
The cooperative triad is a structural motif that appears wherever
three coupled variables with positive feedback sustain a system
above threshold.  The American republic is one instantiation.
The framework predicts the same transient-viability dynamics for
any system whose health depends on three mutually reinforcing
components.

The mathematics is the theorem; the nation is the example.
\end{remark}

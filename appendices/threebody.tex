\chapter{三体 --- The Three-Body Mass Gap}\label{app:threebody}

The triad of \cref{app:triad} modelled three coupled variables
as a cooperative dynamical system.  This appendix models the
\emph{gravitational} three-body problem---the oldest unsolved
problem in classical mechanics---through the lens of fixed points
and mass gaps, and connects it to the viability framework.

The central question is: \emph{can we place a fixed point, open a
mass gap, and use gravity to close it?}  The answer is yes.
In the restricted three-body problem, the Lagrange points are
fixed points of the effective potential in the co-rotating frame.
The Jacobi constant difference between a saddle point and a
stable equilibrium defines a mass gap.  As the mass ratio $\mu$
increases past the Routh critical value $\mu_{\mathrm{crit}}$,
gravitational coupling destroys the stability of the fixed point,
effectively closing the gap.

\section{The restricted three-body problem}\label{sec:r3bp}

\begin{definition}[Co-rotating frame]\label{def:corotating}
Consider two masses $M_1 = 1 - \mu$ and $M_2 = \mu$
($0 < \mu < 1$) in circular orbit about their common centre of
mass, with unit separation and unit angular velocity.
The \emph{co-rotating frame} fixes $M_1$ at $(-\mu, 0)$ and
$M_2$ at $(1 - \mu, 0)$.  A test particle of negligible mass
moves in the effective potential
\begin{equation}\label{eq:eff-potential}
  \Omega(x, y) \;=\;
  \tfrac{1}{2}(x^2 + y^2)
  \;+\; \frac{1 - \mu}{r_1}
  \;+\; \frac{\mu}{r_2},
\end{equation}
where $r_1 = \sqrt{(x + \mu)^2 + y^2}$ and
$r_2 = \sqrt{(x - 1 + \mu)^2 + y^2}$.
\end{definition}

The equations of motion in the co-rotating frame are
\begin{equation}\label{eq:eom-rotating}
  \ddot{x} - 2\dot{y} = \frac{\partial \Omega}{\partial x},
  \qquad
  \ddot{y} + 2\dot{x} = \frac{\partial \Omega}{\partial y}.
\end{equation}
The Coriolis terms $\pm 2\dot{y}$, $\mp 2\dot{x}$ encode the
non-inertial character of the rotating frame and are the mechanism
by which gravity ultimately closes the mass gap.

\begin{definition}[Jacobi constant]\label{def:jacobi}
The \emph{Jacobi constant} is the integral of motion
\begin{equation}\label{eq:jacobi}
  C_J \;=\; 2\,\Omega(x, y) \;-\;
  (\dot{x}^2 + \dot{y}^2).
\end{equation}
At an equilibrium point ($\dot{x} = \dot{y} = 0$), one has
$C_J = 2\,\Omega$.
\end{definition}

\section{Lagrange points as fixed points}\label{sec:lagrange-fp}

\begin{proposition}[Lagrange points]\label{prop:lagrange}
The effective potential $\Omega$ has exactly five critical points
(Lagrange points):
\begin{enumerate}[label=(\alph*)]
  \item \textbf{Collinear points} $L_1, L_2, L_3$: located on
    the $x$-axis ($y = 0$), one between $M_1$ and $M_2$ ($L_1$),
    one beyond $M_2$ ($L_2$), and one beyond $M_1$ ($L_3$).
    All three are saddle points of $\Omega$.
  \item \textbf{Equilateral points} $L_4, L_5$: located at
    $(x, y) = (\tfrac{1}{2} - \mu,\; \pm\tfrac{\sqrt{3}}{2})$,
    forming equilateral triangles with $M_1$ and $M_2$.
    At these points $r_1 = r_2 = 1$.
\end{enumerate}
The Jacobi constants satisfy
$C_J(L_1) > C_J(L_2) > C_J(L_3) > C_J(L_4) = C_J(L_5)$.
\end{proposition}

\begin{proof}
Setting $\nabla\Omega = 0$ with $y = 0$ yields a quintic equation
in $x$ for each collinear point; existence and uniqueness in each
interval follow from the intermediate value theorem and
monotonicity of the gravitational terms.  For $L_4$/$L_5$:
at $(x, y) = (\tfrac{1}{2} - \mu, \pm\tfrac{\sqrt{3}}{2})$ one
computes $r_1 = r_2 = 1$, and direct substitution into
$\nabla\Omega = 0$ confirms equilibrium.  The Jacobi constant
ordering follows from evaluation of $2\Omega$ at each point.
\end{proof}

\section{The mass gap}\label{sec:threebody-massgap}

\begin{definition}[Mass gap]\label{def:mass-gap}
The \emph{mass gap} of the three-body system at mass ratio $\mu$
is
\begin{equation}\label{eq:mass-gap}
  \Delta(\mu)
  \;:=\;
  C_J(L_1) \;-\; C_J(L_4)
  \;=\;
  2\bigl[\Omega(L_1) - \Omega(L_4)\bigr].
\end{equation}
The gap $\Delta$ measures the energy barrier separating the
saddle point $L_1$ from the stable equilibrium $L_4$: a test
particle at $L_4$ with $C_J > C_J(L_1)$ is confined by the
zero-velocity curve to a neighbourhood of $L_4$.
\end{definition}

\begin{proposition}[Analytical form at $L_4$]\label{prop:CJ-L4}
At the equilateral point $L_4$ (or $L_5$),
\begin{equation}\label{eq:CJ-L4}
  C_J(L_4) \;=\; 3 - \mu + \mu^2.
\end{equation}
\end{proposition}

\begin{proof}
At $L_4$: $r_1 = r_2 = 1$,
$x = \tfrac{1}{2} - \mu$, $y = \tfrac{\sqrt{3}}{2}$.
Then
$\Omega(L_4) = \tfrac{1}{2}\bigl((\tfrac{1}{2} - \mu)^2
+ \tfrac{3}{4}\bigr) + (1 - \mu) + \mu
= \tfrac{1}{2}(1 - \mu + \mu^2) + 1$,
so $C_J(L_4) = 2\Omega(L_4) = 3 - \mu + \mu^2$.
\end{proof}

\section{Stability and the Routh criterion}\label{sec:routh}

\begin{theorem}[Routh stability criterion]\label{thm:routh}
The equilateral points $L_4$ and $L_5$ are linearly stable if and
only if $\mu < \mu_{\mathrm{crit}}$, where
\begin{equation}\label{eq:mu-crit}
  \mu_{\mathrm{crit}}
  \;=\; \frac{1}{2}\Bigl(1 - \sqrt{\tfrac{69}{81}}\,\Bigr)
  \;=\; \frac{1}{2}\!\left(1 - \frac{\sqrt{69}}{9}\right)
  \;\approx\; 0.03852.
\end{equation}
\end{theorem}

\begin{proof}
At $L_4$, the Hessian of $\Omega$ is
$\Omega_{xx} = \tfrac{3}{4}$,
$\Omega_{yy} = \tfrac{9}{4}$,
$\Omega_{xy} = \tfrac{3\sqrt{3}}{4}(1 - 2\mu)$.
The characteristic equation of the linearised system
\eqref{eq:eom-rotating} at equilibrium is
\begin{equation}\label{eq:char-L4}
  \lambda^4 + \lambda^2 + \tfrac{27}{4}\,\mu(1 - \mu) = 0.
\end{equation}
Setting $s = \lambda^2$:
$s^2 + s + \tfrac{27}{4}\mu(1 - \mu) = 0$.
The discriminant is $1 - 27\mu(1 - \mu)$.  Both roots $s$
are negative real (giving purely imaginary $\lambda$, hence
stability) if and only if
\[
  1 - 27\mu(1 - \mu) \;\geq\; 0
  \quad\Longleftrightarrow\quad
  \mu(1 - \mu) \;\leq\; \tfrac{1}{27},
\]
which holds precisely for $\mu \leq \mu_{\mathrm{crit}}$.
\end{proof}

\section{Opening and closing the gap}\label{sec:gap-phases}

The mass gap undergoes three phases as $\mu$ varies:

\begin{theorem}[Mass gap phase transition]\label{thm:gap-transition}
\hfill
\begin{enumerate}[label=(\alph*)]
  \item \textbf{Vacuum} ($\mu \to 0^+$):
    $\Delta(\mu) \to 0$.  The three-body problem degenerates to
    Kepler two-body; no non-trivial fixed point exists.
    The vacuum carries no mass gap.

  \item \textbf{Gap open} ($0 < \mu < \mu_{\mathrm{crit}}$):
    $\Delta(\mu) > 0$ and $L_4$ is linearly stable.
    The fixed point is protected by the energy barrier $\Delta$:
    trajectories with $C_J > C_J(L_1)$ are confined near $L_4$
    by the zero-velocity curve.  The gravitational source
    \emph{opened a fixed-point mass in the vacuum}.

  \item \textbf{Gap closed by gravity}
    ($\mu \geq \mu_{\mathrm{crit}}$):
    $\Delta(\mu) > 0$ still, but $L_4$ is linearly
    \emph{unstable}.  The Coriolis coupling (gravitational
    frame-dragging in the co-rotating frame) exceeds the
    restoring force.  The energy barrier persists but is
    dynamically ineffective: gravity closed the gap.
\end{enumerate}
\end{theorem}

\begin{proof}
\textbf{(a)} As $\mu \to 0$, $L_1$ approaches $M_2$ and
$\Omega(L_1) \to \Omega(L_4) + O(\mu^{2/3})$.
Specifically, $C_J(L_1) - C_J(L_4) \sim 2 \cdot 3^{1/3}
\mu^{2/3} \to 0$.

\textbf{(b)} For $0 < \mu < \mu_{\mathrm{crit}}$: $\Delta > 0$
by explicit computation of the collinear $L_1$ position (Hill
sphere approximation gives
$x_{L_1} \approx 1 - \mu - (\mu/3)^{1/3}$), and $L_4$ is stable
by \cref{thm:routh}.

\textbf{(c)} At $\mu = \mu_{\mathrm{crit}}$, the discriminant
of \eqref{eq:char-L4} vanishes; for $\mu > \mu_{\mathrm{crit}}$,
the eigenvalues acquire nonzero real parts, and $L_4$ becomes
a spiral source.  The energy gap $\Delta$ remains positive
(since $C_J(L_1) - C_J(L_4)$ is a smooth function of $\mu$),
but the fixed point is no longer linearly stable, so the gap
offers no dynamical protection.
\end{proof}

\section{Viability interpretation}\label{sec:threebody-viability}

\begin{definition}[Three-body viability kernel]\label{def:tb-viab}
For a Jacobi constant threshold $C_0 > 0$, define the
\emph{three-body viability kernel} as the Hill region
\[
  K_{C_0} \;:=\;
  \bigl\{\, (x, y) \in \R^2 \;:\;
  2\,\Omega(x, y) \geq C_0 \,\bigr\}.
\]
This is the set of positions reachable by a test particle with
Jacobi constant at least $C_0$ (zero-velocity curves bound the
accessible region).
\end{definition}

\begin{corollary}[Viability of the fixed point]\label{cor:viab-L4}
The fixed point $L_4$ lies in the interior of the connected
component of $K_{C_J(L_1)}$ containing $L_4$ if and only if
$\mu < \mu_{\mathrm{crit}}$.  At $\mu = \mu_{\mathrm{crit}}$,
$L_4$ exits $\Viab(K_{C_J(L_1)})$ by loss of stability---the
identical mechanism to the triad's ``stable until gone''
bifurcation (\cref{thm:triad}(d)).
\end{corollary}

\begin{remark}[Analogy with the cooperative triad]
\label{rem:triad-threebody}
The cooperative triad (\cref{def:triad}) and the restricted
three-body problem share the same mathematical skeleton:
\begin{center}
\renewcommand{\arraystretch}{1.25}
\begin{tabular}{@{}lll@{}}
\toprule
\textbf{Concept} & \textbf{Triad} & \textbf{Three-body} \\
\midrule
State space & $\R^3_{\geq 0}$ & $\R^2$ (co-rotating frame) \\
Fixed point & $x^* \in K_\epsilon^\circ$ & $L_4 \in K_{C_0}^\circ$ \\
Coupling & $\beta_{ij}$ (cross-feeding) & $\mu$ (mass ratio) \\
Gap & $V_\Pi(x^*) > 0$ & $\Delta(\mu) > 0$ \\
Critical threshold & $\beta_{ij}^*$ & $\mu_{\mathrm{crit}}$ \\
Closure mechanism & coupling weakens & gravity strengthens \\
Death mode & ``stable until gone'' & Routh instability \\
\bottomrule
\end{tabular}
\end{center}
In both cases, the fixed point is stable until a single parameter
crosses a critical threshold, at which point viability is lost
abruptly---not gradually.  The mathematics is the same theorem
(\cref{thm:triad}(d) $\cong$ \cref{thm:routh}); the physics
differs.

The directions are \emph{reversed}: in the triad, the gap closes
when coupling \emph{weakens}; in the three-body problem, the gap
closes when gravitational coupling \emph{strengthens}.
This is because in the triad, coupling is cooperative
($\beta_{ij} > 0$ helps), while in the three-body problem,
coupling is destabilising (Coriolis $\to$ instability).
\end{remark}

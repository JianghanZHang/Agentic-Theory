\chapter{The Three-Body Galaxy}\label{sec:threebody}

\begin{center}
\itshape
I now demonstrate the frame of the system of the world.\\[4pt]
\upshape ---Isaac Newton, \emph{Principia Mathematica}, Book~III (1687)~\cite{newton}
\end{center}

\bigskip

Newton demonstrated the two-body frame: Keplerian orbits, inverse-square
law, universal gravitation.  The three-body frame remained open for three
centuries because it has \emph{no king}---the complete graph $K_3$ has no
cut vertex, and the mean-field detection of \cref{thm:meanfield} fails.
We now apply the framework to the three-body problem and construct the
\emph{gravity damper}: a controlled agent $\kappa$ that stabilises the
system by maintaining the spectral gap $\lambda_1 > 0$.  The three-body
problem has no analytical solution, but it admits an \emph{agentic}
one: the agent does not predict the future---it controls the present.

The argument proceeds in seven stages:
\begin{enumerate}
  \item \textbf{Diagnosis} (\cref{sec:3b-diagnosis}): $K_3$ has no
  king, so $\lambda_1 \to 0$ and viability fails.
  \item \textbf{Design} (\cref{sec:3b-ocp}): introduce a controlled
  mass $m_*$ and derive the optimal control law.
  \item \textbf{Extension} (\cref{sec:3b-manipulation}): the same
  structure governs rigid-body manipulation---one hand stabilises an
  object against gravity.
  \item \textbf{Computation} (\cref{sec:3b-mppi}): MPPI sampling
  exchanges exponential mode enumeration for polynomial search;
  the spectral gap controls mixing time.
  \item \textbf{Sufficiency} (\cref{sec:3b-kakeya}): the Kakeya
  condition guarantees that $\bar{u}^* > 0$ strictly---free stability
  does not exist.
  \item \textbf{Agenticity} (\cref{sec:3b-agenticity}): the formal
  definition---observability, reachability, controllability, and
  their Kalman duality through gravity.
  \item \textbf{Implementation} (\cref{sec:3b-mujoco}): the
  physics-backend isomorphism---the same $\rho$-controller runs on
  hand-rolled gravity and on MuJoCo contact dynamics; force
  elimination (\cref{sec:3b-force-elim}); ground duality
  (\cref{sec:3b-ground-duality}).
\end{enumerate}

% ═══════════════════════════════════════════════════════════
\section{Diagnosis: why the three-body system escapes}
\label{sec:3b-diagnosis}

\subsection{The gravitational graph Laplacian}

Three point masses $m_1, m_2, m_3$ in $\R^3$, with positions
$q_i \in \R^3$.  The \emph{gravitational graph Laplacian}
$L_G \in \R^{3\times 3}$ has edge weights given by the tidal coupling:
\[
  w_{ij}(q) \;=\; \frac{G\, m_i\, m_j}{\|q_i - q_j\|^3},
  \qquad i \neq j.
\]
The Fiedler eigenvalue $\lambda_1(L_G)$ measures algebraic connectivity.

\begin{corollary}[Three-body mass gap closure]\label{cor:threebody}
In the three-body system without a controlled agent,
$\lambda_1(L_G(q(t))) \to 0$ as $t \to \infty$ for generic initial
conditions.
\end{corollary}

\begin{proof}
$K_3$ has no cut vertex (\cref{thm:massgap}: connectivity is
equivalent to $\lambda_1 > 0$).  When two bodies approach
($\|q_i - q_j\| \to 0$), the weight $w_{ij} \to \infty$ while
$w_{ik}, w_{jk} \to 0$ as the third body escapes.  The graph
disconnects: $\lambda_1 \to 0$.

In the language of \cref{thm:meanfield}: every body's actuation
$\|U_r\|$ is near the mean $\bar{U}$, so no sword is detected.
But the system is unstable---mean-field detection requires a king,
and $K_3$ has none.
\end{proof}

\begin{remark}[Scope boundary of mean-field detection]
\label{rem:3b-scope}
\Cref{cor:threebody} gives the precise scope of the central thesis.
The sword is the mean under finite-bandwidth detection in a system
\emph{with} a unique principal agent (\cref{thm:meanfield}).  In a
system without a king---a pure $K_n$ graph---mean-field detection is
insufficient.  The spectral gap $\lambda_1$ (\cref{thm:massgap})
remains the correct stability criterion, but it must be
\emph{maintained by active control}, not merely observed.
\end{remark}

% ═══════════════════════════════════════════════════════════
\section{Design: the gravity damper}\label{sec:3b-ocp}

\subsection{The four-body system}

Introduce a fourth body $m_*$---the \emph{damper}---with position
$q_* \in \R^3$ and control input $u(t) \in \R^3$.  The three
celestial masses obey Newton; the damper receives external force:
\begin{align}
  m_i \ddot{q}_i &= -\nabla_{q_i} V,
  \qquad i \in \{1,2,3\}, \label{eq:3b-newton}\\
  m_* \ddot{q}_* &= -\nabla_{q_*} V + u. \label{eq:3b-damper}
\end{align}
The gravitational potential is
\[
  V(q) = -\sum_{1 \le i < j \le 3} \frac{G\,m_i m_j}{\|q_i - q_j\|}
  - \sum_{i=1}^{3} \frac{G\,m_i m_*}{\|q_i - q_*\|}.
\]
In the execution graph, the damper $m_*$ is the king $\kappa$: it is
the only agent with a control input.  The topology is now $K_4$, and
$\kappa$ is a potential cut vertex---\emph{if} it has sufficient
actuation authority.

\subsection{The optimal control problem}

\begin{definition}[Gravity damper OCP]\label{def:3b-ocp}
The gravity damper seeks the path of least action that keeps the
system spectrally connected:
\begin{equation}\label{eq:3b-ocp}
  \min_{u(\cdot)} \; J[u] = \int_0^T \left[
  \mathcal{L}(q, \dot{q}) + \frac{\alpha}{2}\|u\|^2 \right] dt
\end{equation}
subject to: (i)~dynamics \eqref{eq:3b-newton}--\eqref{eq:3b-damper},
(ii)~spectral gap constraint $\lambda_1(L_G(q(t))) \ge \epsilon$
for all $t$, and (iii)~box constraint
$u(t) \in \mathcal{U} = [-\bar{u}, \bar{u}]^3$.

Here $\mathcal{L} = \sum_i \frac{1}{2}m_i\|\dot{q}_i\|^2 - V(q)$
is the gravitational Lagrangian, $\alpha > 0$ is the control cost,
and $\bar{u}$ is the damper's maximum thrust.
\end{definition}

The box constraint is essential for well-posedness.  Without it,
when $\lambda_1 \to \epsilon$, the optimal control $u^* \to \infty$---
the damper cheats by applying infinite force at the last instant.
The box constraint enforces temporal linearity (\cref{rem:di-temporal}):
the damper must plan ahead with bounded resources.

\subsection{Closed-form solution}

State: $x = (q_1, q_2, q_3, q_*, \dot{q}_1, \ldots, \dot{q}_*) \in
\R^{24}$.  Costates: $p \in \R^{24}$.  Multiplier $\mu(t) \ge 0$
for the spectral constraint.

\begin{theorem}[Gravity damper: three-term decomposition]
\label{thm:3b-damper}
The optimal trajectory of $m_*$ satisfies
\begin{equation}\label{eq:3b-threeterm}
  m_* \ddot{q}_*
  = \underbrace{-\nabla_{q_*}V}_{\text{\textup{(I) Gravity}}}
  + \underbrace{u_{\mathrm{action}}(t)}_{%
    \text{\textup{(II) Least action}}}
  + \underbrace{\frac{\mu(t)}{\alpha m_*}
    \nabla_{q_*}\lambda_1}_{%
    \text{\textup{(III) Spectral kick}}},
\end{equation}
where:
\begin{enumerate}[label=\textup{(\Roman*)}]
  \item is the passive gravitational force---the damper falls;
  \item is the unconstrained action-minimising correction---the
  costate feedback from the Euler--Lagrange adjoint;
  \item activates only when $\lambda_1 = \epsilon$ (by complementary
  slackness: $\mu(t)(\epsilon - \lambda_1) = 0$)---the damper kicks
  in the direction that most increases $\lambda_1$.
\end{enumerate}
\end{theorem}

\begin{proof}
Write the control Hamiltonian:
\[
  \mathcal{H}(x,p,u,\mu) = \mathcal{L} + \frac{\alpha}{2}\|u\|^2
  + p \cdot f(x,u) + \mu(\epsilon - \lambda_1(q)).
\]
Pontryagin optimality gives:
\begin{align}
  u^*(t) &= \mathrm{sat}_{\bar{u}}\!\Bigl(
  -\frac{1}{\alpha m_*}\, p_{\dot{q}_*}(t)\Bigr),
  \label{eq:3b-ustar}\\
  \dot{p}_{q_*} &= -\nabla_{q_*}\mathcal{L}
  + \mu(t)\,\nabla_{q_*}\lambda_1,
  \label{eq:3b-costate}
\end{align}
where $\mathrm{sat}_{\bar{u}}$ is componentwise saturation.
Substituting \eqref{eq:3b-ustar} into the damper
dynamics~\eqref{eq:3b-damper} and decomposing the costate into its
Euler--Lagrange part (term~II) and spectral part (term~III) via
\eqref{eq:3b-costate} gives the stated three-term form.  When
$\mu = 0$, term~III vanishes and we recover the unconstrained
action-minimising trajectory.
\end{proof}

\subsection{Bang-singular structure and the Lagrange points}

With the box constraint, the optimal trajectory has three arc types:

\begin{description}
  \item[Singular arc] $\lambda_1 > \epsilon$ and $\|u^*\| < \bar{u}$.
  The damper coasts along the least-action geodesic.
  \item[Bang arc] $\lambda_1$ approaches $\epsilon$, the costate
  drives $\|u\| = \bar{u}$.  The damper fires at maximum thrust.
  \item[Infeasible arc] If $\bar{u}$ is too small, bang cannot
  prevent $\lambda_1$ from crossing $\epsilon$.  The system escapes.
\end{description}

\begin{proposition}[Bang points concentrate at Lagrange points]
\label{prop:3b-lagrange}
The switching surface between singular and bang arcs projects onto
the damper's configuration space at the Lagrange points of the
instantaneous three-body configuration.
\end{proposition}

\begin{proof}[Proof sketch]
At a Lagrange point $q_* = L_k$, the effective gravitational force
on the damper vanishes in the co-rotating frame:
$\nabla_{q_*}V_\mathrm{eff}\big|_{L_k} = 0$, so term~(I) vanishes
and all control authority is freed for the spectral kick.

The collinear points $L_1, L_2, L_3$ are saddle points of
$\lambda_1(q_*)$: the spectral gradient $\|\nabla_{q_*}\lambda_1\|$
is maximal here, driving the costate to saturation.  The triangular
points $L_4, L_5$ are local maxima of $\lambda_1$: the damper parks
here during singular arcs.

The optimal strategy is: park at $L_4/L_5$ (singular arc), commute
to $L_1/L_2/L_3$ when $\lambda_1$ drops (bang arc), apply spectral
kick, return.
\end{proof}

\begin{definition}[Critical actuation threshold]
\label{def:3b-ustar}
The minimum thrust for OCP feasibility is
\[
  \bar{u}^* = \inf\Bigl\{\bar{u} > 0 :
  \exists\, u(\cdot) \in [-\bar{u},\bar{u}]^3 \;\text{s.t.}\;
  \lambda_1(q(t)) \ge \epsilon \;\;\forall\, t \in [0,T]\Bigr\}.
\]
$\bar{u}^* > 0$ strictly: free stability does not exist.
\end{definition}

\subsection{The Lyapunov barrier}

\begin{proposition}[Action--Lyapunov--Spectral equivalence]
\label{prop:3b-equiv}
The following are equivalent for the controlled system:
\begin{enumerate}[label=\textup{(\Roman*)}]
  \item \textbf{Least action.}
  There exists $u^* \in \mathcal{U}$ minimising $J[u]$ with
  $\lambda_1 \ge \epsilon$.
  \item \textbf{Lyapunov stability.}
  There exists a control-Lyapunov function
  $W = H + \beta/(\lambda_1 - \epsilon)$ with $\dot{W} \le 0$ under
  admissible control.
  \item \textbf{Spectral gap maintenance.}
  $\lambda_1(q(t)) \ge \epsilon$ for all $t \in [0,T]$.
\end{enumerate}
The connections are:
$\textup{(I)} \xrightarrow{\textup{HJB}} \textup{(III)}
 \xrightarrow{\textup{barrier}} \textup{(II)}
 \xrightarrow{\textup{Sontag}} \textup{(I)}$.
\end{proposition}

\begin{proof}
(I)$\Rightarrow$(III): by constraint.
(III)$\Rightarrow$(II): construct $W(x) = H(x) + \beta/(\lambda_1 -
\epsilon)$.  Along feasible trajectories $\lambda_1 > \epsilon$,
so $W$ is finite; the barrier $\beta/(\lambda_1-\epsilon) \to
+\infty$ as $\lambda_1 \to \epsilon^+$ prevents crossing.
Compute $\dot{W} = u \cdot \dot{q}_* -
\beta\dot{\lambda}_1/(\lambda_1-\epsilon)^2$; the optimal control
ensures $\dot{W} \le 0$.
(II)$\Rightarrow$(I): Sontag's universal formula constructs smooth
$u$ from the CLF; this $u$ is admissible and makes $J$ finite.
\end{proof}

\begin{remark}[顿开金绳,扯断玉锁]\label{rem:3b-jinsheng}
The control policy takes its name from Chapter~117 of
\emph{Dream of the Red Chamber}: ``suddenly snap the golden cord,
wrench apart the jade lock.''  The golden cord is the gravitational
binding.  The jade lock is the spectral gap closing.  The damper
snaps the cord at the Lagrange points---the precise locations where
gravitational equilibrium frees all control authority for the
spectral kick.

On singular arcs, the cord is slack: the damper rides the
least-action geodesic.  On bang arcs, the lock tightens: the damper
fires at maximum thrust to wrench it open.  The transition is
sharp---not a gradient, but a switch.
\end{remark}

% ═══════════════════════════════════════════════════════════
\section{Extension: rigid-body manipulation}\label{sec:3b-manipulation}

The three-body gravity damper is not an isolated curiosity.  Its
structure appears whenever one agent must stabilise a multi-body
system against gravity.

\subsection{搬运 as a three-body problem}

Consider a robotic hand (the damper $\kappa$) grasping and
transporting an object.  The object's mass distribution defines
$m_1, m_2, m_3$ (or more generally, the principal moments of
inertia).  Gravity acts downward.  The hand applies contact forces
$u(t)$ through the grasp.

The execution graph is the same: the hand is $\kappa$, the object's
inertia axes are the ``three bodies,'' and the spectral gap
$\lambda_1$ of the \emph{grasp graph Laplacian} measures whether the
grasp is stable~\cite{mason}.  When $\lambda_1 \to 0$, the object
slips.  Cheng et~al.~\cite{cheng-manip} show that dexterous
manipulation planning requires reasoning about contact mode
transitions---precisely our bang-singular switching.  The signed
distance function formulation of ContactSDF~\cite{contactsdf}
provides a differentiable contact model: the SDF value at each
contact point is a local proxy for $\lambda_1$, and its gradient is
the spectral gradient $\nabla_{q_*}\lambda_1$ of
\cref{thm:3b-damper}.

\begin{proposition}[Manipulation is gravity damping]
\label{prop:3b-manip}
The optimal control for rigid-body manipulation under gravity has
the same three-term structure as \cref{thm:3b-damper}:
\begin{enumerate}[label=\textup{(\Roman*)}]
  \item Gravity (the object falls);
  \item Least-action trajectory (the hand follows the planned path);
  \item Spectral kick (the hand tightens the grasp when
  $\lambda_1 \to \epsilon$).
\end{enumerate}
The bang points are the \emph{grasp singularities}---configurations
where the grasp matrix loses rank, analogous to the Lagrange points
of the gravitational problem.
\end{proposition}

\subsection{From celestial mechanics to robotics}

The mapping is:

\begin{center}
\begin{tabular}{@{}lll@{}}
\toprule
\textbf{Three-body} & \textbf{Manipulation} & \textbf{Framework} \\
\midrule
Celestial masses $m_1,m_2,m_3$ & Object inertia axes &
Agents $a_1,a_2,a_3$ \\
Damper $m_*$ & Robotic hand & King $\kappa$ \\
Gravitational potential $V$ & Gravity + contact potential &
Viability Lyapunov $V$ \\
Spectral gap $\lambda_1$ & Grasp quality &
Mass gap (\cref{thm:massgap}) \\
Lagrange points & Grasp singularities & Switching surface $\Sigma$ \\
Thrust $\bar{u}$ & Grip force limit & Box constraint \\
$\bar{u}^*$ & Minimum grip force & Price of stability \\
\bottomrule
\end{tabular}
\end{center}

The central thesis applies: the hand's detection threshold (when to
tighten the grasp) is determined by the mean contact force.
An object slips when a perturbation exceeds the mean---\emph{the
sword is the mean}, applied to grasping.

% ═══════════════════════════════════════════════════════════
\section{Computation: MPPI and polynomial tractability}
\label{sec:3b-mppi}

The three-term control law (\cref{thm:3b-damper}) requires evaluating
$\nabla_{q_*}\lambda_1$ at each time step.  In the manipulation
setting, the grasp graph Laplacian changes with every contact mode
transition.  A na\"ive approach enumerates all $3^n$ contact mode
assignments (separating, sliding, sticking at each of $n$ contact
points)---exponential in the number of contacts.  We show that
\emph{sampling} replaces \emph{enumeration}, and the spectral gap
is the reason it works.

\subsection{Contact modes as Laplacian weights}

At each contact point $i$, the contact mode determines the edge
weight $w_i$ of the grasp graph Laplacian:

\begin{center}
\begin{tabular}{@{}llc@{}}
\toprule
\textbf{Mode} & \textbf{Force transmission} & $w_i$ \\
\midrule
Separating & None & $0$ \\
Sliding & Normal $+$ friction-limited tangential
  & $k_n + \mu k_n$ \\
Sticking & Normal $+$ full tangential & $k_n + k_t$ \\
\bottomrule
\end{tabular}
\end{center}

A contact mode assignment $\sigma = (\sigma_1, \ldots, \sigma_n)$
defines a weight vector $w(\sigma)$ and hence a Laplacian
$L_G(\sigma)$ with Fiedler eigenvalue $\lambda_1(\sigma)$.

\subsection{MPPI: sampling replaces enumeration}

Model Predictive Path Integral (MPPI) control samples $K$ trajectories
and weights them by the exponential cost:
\[
  u^*(t) = \frac{\displaystyle\sum_{k=1}^{K}
    e^{-J_k/\alpha}\, u_k(t)}
    {\displaystyle\sum_{k=1}^{K} e^{-J_k/\alpha}},
\]
where $J_k = \int_0^T [\mathcal{L} + \frac{\alpha}{2}\|u_k\|^2]\,dt$
is the cost of the $k$-th sampled trajectory.  Trajectories that
violate contact constraints ($\phi_i < 0$ in the signed distance
field~\cite{contactsdf}) incur exponential penalty; their softmax
weight vanishes.  MPPI therefore implicitly enumerates valid contact
modes without explicit combinatorial search.

\begin{theorem}[Exponential--polynomial exchange]
\label{thm:3b-mppi}
Under the spectral gap condition $\lambda_1 \ge \epsilon > 0$, the
number of valid contact modes at each time step is $\mathrm{poly}(n)$,
and MPPI with $K = O(\mathrm{poly}(n))$ samples finds a feasible
trajectory with high probability.
\end{theorem}

\begin{proof}
Cheng et~al.~\cite{cheng-manip} show that the contact mode lattice
admits $\mathrm{poly}(n)$ valid modes under force balance; we
identify this condition with $\lambda_1 > 0$ via the grasp
Laplacian.  The SDF barrier
$\beta / \phi_i$ in the MPPI cost suppresses invalid modes with
weight $\exp(-\beta/\phi_i \cdot 1/\alpha) \to 0$ as
$\phi_i \to 0^-$.  The effective sampling space is therefore
$\mathrm{poly}(n)$, not $3^n$.

More precisely: MPPI is a soft version of Cheng's polynomial
enumeration.  She uses combinatorial lattice search (exact); MPPI
uses sampling with softmax weighting (approximate).  Both exploit the
same $\mathrm{poly}(n)$ structure.
\end{proof}

\subsection{Cheeger inequality and mixing time}

The spectral gap does double duty: it controls both the
\emph{physical} stability (mass gap) and the \emph{computational}
tractability (mixing time of the sampling process).

\begin{proposition}[Spectral gap controls mixing]\label{prop:3b-mixing}
The Cheeger inequality (\cref{thm:massgap}) gives
$\lambda_1 \ge h(G)^2 / 2$, where $h(G)$ is the Cheeger constant.
The mixing time of the MPPI random walk satisfies
$\tau_{\mathrm{mix}} = O(1/\lambda_1)$.  In particular:
\[
  \lambda_1 \ge \epsilon \quad\Longrightarrow\quad
  \tau_{\mathrm{mix}} \le 1/\epsilon \quad\Longrightarrow\quad
  K = O(1/\epsilon) \;\text{samples suffice.}
\]
\end{proposition}

\begin{proof}
Standard spectral graph theory~\cite{cheeger,mohar}: the second
eigenvalue of the normalised Laplacian bounds the mixing time of a
lazy random walk.  The contact mode graph, weighted by the SDF
softmax, has the same spectral structure as the grasp Laplacian.
When $\lambda_1 \ge \epsilon$, the chain mixes in $O(1/\epsilon)$
steps, so $O(1/\epsilon)$ independent samples cover the valid mode
space.
\end{proof}

\begin{remark}[The spectral gap does everything]
The same quantity $\lambda_1$ determines:
(i)~physical stability (will the object be dropped?),
(ii)~control authority (when to fire the spectral kick), and
(iii)~computational cost (how many MPPI samples are needed).
This is not coincidence---it is the duality between the Cheeger
constant of the execution graph and the min-cut of the flow network
(\cref{thm:flowcut}).
\end{remark}

% ═══════════════════════════════════════════════════════════
\section{Sufficiency: the Kakeya condition}\label{sec:3b-kakeya}

We have shown that the damper \emph{can} maintain $\lambda_1 \ge
\epsilon$ given sufficient thrust $\bar{u} \ge \bar{u}^*$.  We now
ask: must $\bar{u}^* > 0$?  That is, does free stability exist?

\subsection{Instability directions and the reachable set}

As the three-body system evolves, the pairwise instability
directions
\[
  d_{ij}(t) = \frac{q_i(t) - q_j(t)}{\|q_i(t) - q_j(t)\|}
  \in S^2, \qquad 1 \le i < j \le 3,
\]
rotate through the unit sphere.  Over a generic ergodic trajectory,
$\{d_{12}, d_{13}, d_{23}\}$ sweep all of $S^2$.

The damper's \emph{reachable set} at time $t$ with energy budget $E$
is
\[
  \mathcal{R}(t, E) = \Bigl\{q_* \in \R^3 :
  \exists\, u(\cdot) \in \mathcal{U},\;
  \int_0^t \|u\|^2\,ds \le E,\;
  q_*(s) \text{ reaches } q_* \text{ at } s = t\Bigr\}.
\]

\begin{proposition}[Kakeya sufficiency]\label{prop:3b-kakeya}
The minimum thrust $\bar{u}^* > 0$ strictly, and
\[
  \bar{u}^* \;\ge\; \frac{m_*}{\alpha}\,
  \sup_{t \in [0,T]}\Bigl[
    \mu(t)\,\|\nabla_{q_*}\lambda_1\|\Bigr]
    \Big|_{\text{worst Lagrange point}}.
\]
Below $\bar{u}^*$: no control can save the system.  Above: the
least-action policy with spectral kicks suffices.
\end{proposition}

\begin{proof}
On a bang arc at a collinear Lagrange point, the damper must exert
force $\|u\| = \bar{u}$ along $\nabla_{q_*}\lambda_1$ to prevent
$\lambda_1$ from crossing $\epsilon$.  The spectral gradient at $L_k$
is bounded below by the tidal coupling:
$\|\nabla_{q_*}\lambda_1\| \ge Gm_*/(r_{L_k})^3$, where $r_{L_k}$
is the distance from $L_k$ to the nearest body.  By complementary
slackness, $\mu > 0$ on the bang arc.  Hence term~(III) of
\cref{thm:3b-damper} is strictly positive, requiring
$\bar{u} > 0$.
\end{proof}

\subsection{The Kakeya dual}

The geometric content of \cref{prop:3b-kakeya} is a duality with
the Kakeya conjecture:

\begin{itemize}
  \item \textbf{Kakeya} asks: what is the minimum measure of a set in
  $\R^n$ containing a unit segment in every direction?  The conjecture
  asserts: Hausdorff dimension $n$ (full-dimensional).
  \item \textbf{Damper} asks: what is the minimum energy budget for the
  reachable set $\mathcal{R}(t,E)$ to cover all instability directions?
  The answer: $\mathcal{R}$ must be full-dimensional in $\R^3$.
\end{itemize}

The instability directions $\{d_{ij}\}$ sweep $S^2$.  To block
collapse in \emph{every} direction, the reachable set must contain a
segment in every direction---a Besicovitch set.  Kakeya predicts
this set is full-dimensional, hence the energy budget $E$ (and
therefore $\bar{u}^*$) cannot be compressed to zero.

\begin{remark}[Free stability does not exist --- 免费的稳定性不存在]
\label{rem:3b-free}
\Cref{prop:3b-kakeya} is the sufficiency condition for the entire
framework: viability maintenance has a strictly positive price.
The massless axiom (\cref{ax:massless}) says agents have no
intrinsic mass; the Kakeya condition says \emph{control} has
irreducible cost.  Together they give the complete picture: the
sword is free to detect (it costs nothing to observe who exceeds
the mean) but not free to control (maintaining $\lambda_1 > \epsilon$
costs $\bar{u} \ge \bar{u}^* > 0$).
\end{remark}

% ═══════════════════════════════════════════════════════════
\section{Agenticity: observability--reachability--controllability}
\label{sec:3b-agenticity}

We now define the central concept.  An object has no intrinsic
agenticity (\cref{ax:massless}).  Agenticity is \emph{conferred}---
被赋予的---and \emph{controlled}---被控制的.  The definition is the
conjunction of three conditions, linked by gravity.

The construction is not specific to celestial mechanics or
manipulation.  The quantity $P$ in the diagram below is a
\emph{probability transition kernel} $P(x' \mid x, u)$---the
transition function of a Markov decision process.  Pose
$P \in \mathrm{SE}(3)$ and gravitational dynamics are one instance;
any controlled Markov process is another.  The spectral gap
$\lambda_1$ of the graph Laplacian serves triple duty:
\begin{enumerate}[label=\textup{(\roman*)}]
  \item \textbf{Physical stability}: will the system escape?
  ($\lambda_1 > 0 \Leftrightarrow$ connected.)
  \item \textbf{Controllability}: can the policy steer the state?
  ($\lambda_1 > \epsilon \Leftrightarrow$ MDP is controllable.)
  \item \textbf{Computational tractability}: how fast does MPPI
  converge?
  ($\tau_{\mathrm{mix}} = O(1/\lambda_1)$.)
\end{enumerate}
These are the same number because they are all $\lambda_1$ of the
same Laplacian, viewed through different lenses.  The Cheeger
constant connects them:
$h(G) > 0 \;\Leftrightarrow\; \lambda_1 > 0
\;\Leftrightarrow\;$ the chain mixes
$\;\Leftrightarrow\;$ the system is controllable
$\;\Leftrightarrow\;$ MPPI converges.

\begin{figure}[H]
\centering
\begin{tikzpicture}[scale=1.6,
  >=Stealth,
  manifold/.style={water, line width=0.3pt, opacity=0.5},
  tet/.style={green!60!black, line width=1pt},
  tetfill/.style={green!30!black, fill opacity=0.08},
]
  % ═══ Layer 1 (绳 Rope, red): gravitational field ═══
  % Background plane with ⊗ indicating m*g pointing into the page
  % (bird's-eye view: gravity goes straight down, i.e. into screen)
  \begin{scope}[shift={(-0.3,0)}]
    \fill[dao!5] (-3.0, -2.4) rectangle (3.0, 2.4);
    \draw[dao, line width=0.3pt, opacity=0.15]
      (-3.0, -2.4) rectangle (3.0, 2.4);
    % ⊗ sign (gravity into page)
    \node[dao, font=\large, opacity=0.25] at (2.4, -1.8) {$\bigotimes$};
    \node[dao, font=\scriptsize, opacity=0.35, anchor=west]
      at (2.65, -1.8) {$m_*g$};
  \end{scope}

  % ═══ Layer 2 (锁 Lock, blue): manifold net 𝒳 ═══
  \begin{scope}[shift={(-0.3,0)}]
    % Horizontal curves (u-lines)
    \foreach \v in {-2.0,-1.0,0,1.0,2.0} {
      \draw[manifold] plot[smooth, domain=-2.8:2.8, samples=30]
        ({\x}, {\v + 0.15*sin(60*\x) + 0.08*cos(90*\v)},
         {0.3*cos(40*\x)*cos(40*\v)});
    }
    % Vertical curves (v-lines)
    \foreach \u in {-2.5,-1.5,-0.5,0.5,1.5,2.5} {
      \draw[manifold] plot[smooth, domain=-2.0:2.0, samples=20]
        ({\u + 0.1*sin(70*\x)}, {\x},
         {0.3*cos(40*\u)*cos(40*\x)});
    }
  \end{scope}

  % ═══ Layer 3 (玉 Jade, green): tetrahedron Δ³ ═══
  \coordinate (T1) at (-0.4, -1.0, 0);
  \coordinate (T2) at (1.8, -0.4, 0);
  \coordinate (T3) at (0.5,  1.4, 0);
  \coordinate (T4) at (0.6,  0.15, 1.6);

  % Back faces
  \fill[tetfill, green!20!black] (T1) -- (T2) -- (T4) -- cycle;
  \fill[tetfill, green!20!black] (T1) -- (T3) -- (T4) -- cycle;
  \draw[tet, opacity=0.3, dashed] (T1) -- (T4);
  % Front faces
  \fill[tetfill, green!40!black] (T1) -- (T2) -- (T3) -- cycle;
  \fill[tetfill, green!25!black] (T2) -- (T3) -- (T4) -- cycle;
  % Edges
  \draw[tet] (T1) -- (T2);
  \draw[tet] (T2) -- (T3);
  \draw[tet] (T3) -- (T1);
  \draw[tet] (T2) -- (T4);
  \draw[tet] (T3) -- (T4);

  % ═══ Layer 4 (绳 Rope, red): state x(t) on Δ³ ═══
  \coordinate (P) at ($(T2)!0.35!(T3)!0.4!(T4)$);
  \fill[dao, opacity=0.9] (P) circle (2.5pt);

  % ═══ Layer 5 (金 Golden, cyan): operations ═══
  % ω̂ pseudovector
  \draw[->, sword, line width=1.6pt]
    (P) -- ++(0.8, 0.9, 0.5)
    node[right, font=\small, text=sword]
    {$\hat{\omega}$};
  % ∇_{ω̂} P
  \draw[->, sword, line width=1.6pt]
    (P) -- ++(0.3, -1.4, 0.4)
    node[right, font=\small, text=sword]
    {$\nabla_{\hat{\omega}}\!P$};

  % ── Labels ──────────────────────────────────────────
  \node[water, font=\small, anchor=north east] at (-2.6, -2.0)
    {$\mathcal{X}$};
  \node[green!50!black, font=\small, anchor=west] at (2.1, 0.8)
    {$\Delta^3$};
  \node[dao, font=\small, anchor=south east] at ([xshift=-3pt]P)
    {$x(t)$};

  % ── Layer legend (right margin) ─────────────────────
  \begin{scope}[shift={(4.0, 1.8)}]
    \node[anchor=west, font=\scriptsize, text=sword]
      at (0, 0) {\textbf{金}\; $\hat{\omega},\,\nabla_{\hat{\omega}}P$};
    \node[anchor=west, font=\scriptsize, text=dao]
      at (0, -0.5) {\textbf{绳}\; $x(t),\,m_*g$};
    \node[anchor=west, font=\scriptsize, text=green!50!black]
      at (0, -1.0) {\textbf{玉}\; $\Delta^3$};
    \node[anchor=west, font=\scriptsize, text=water]
      at (0, -1.5) {\textbf{锁}\; $\mathcal{X}$};
  \end{scope}
\end{tikzpicture}

\medskip

\caption{Agenticity diagram (bird's-eye view).  Layered
construction from bottom to top:
\textcolor{dao}{\textbf{绳}~Rope} (red): the gravitational field
$m_*g$ pointing into the page ($\bigotimes$)---the drift $A$ that
acts on every body.
\textcolor{water}{\textbf{锁}~Lock} (blue): the manifold net
$\mathcal{X}$---the viability kernel that supports the dynamics in
linear time.
\textcolor{green!50!black}{\textbf{玉}~Jade} (green): the Kakeya
tetrahedron $\Delta^3$---the bodies that exist
(\cref{prop:3b-kakeya}).
\textcolor{dao}{\textbf{绳}~Rope} (red, again): the current state
$x(t)$ on $\Delta^3$---where the agent acts.
\textcolor{sword}{\textbf{金}~Golden} (cyan): the pseudovector
$\hat{\omega} \in \mathfrak{so}(3)$ and the Lie derivative
$\nabla_{\hat{\omega}}P$---the sword that cuts: simultaneously the
gravity direction, the policy gradient, and the score function.}
\label{fig:agenticity}
\end{figure}

\subsection{The linearised system}

Linearise the controlled dynamics about a trajectory.  State
$\delta x \in \R^{n}$:
\begin{equation}\label{eq:3b-linear}
  \delta\dot{x} = A(t)\,\delta x + B\,\delta u,
  \qquad y = C(t)\,\delta x,
\end{equation}
where:
\begin{itemize}
  \item $A(t) = \partial f / \partial x$ encodes the
  \textbf{natural drift}---the uncontrolled dynamics of the
  transition kernel $P$.  In the gravitational instance, $A$ is the
  tidal tensor.  In a general MDP, $A$ is the linearised transition
  matrix $\partial P / \partial x$.
  \item $B$ is the \textbf{control input}---force enters only through
  the agent $\kappa$.  In the four-body instance,
  $B = [0;\; \ldots;\; 0;\; I_3/m_*]^\top$.
  \item $C(t) = \nabla_x \lambda_1$ is the \textbf{spectral
  observation}---the Fiedler eigenvector gradient.
\end{itemize}

The matrix $A$ is gravity.  In the MDP, $A$ is the drift of the
Markov chain---what happens without control.  It appears in all
three conditions below.

\subsection{The definition}

\begin{definition}[Agenticity]\label{def:agenticity}
An object $r$ in the execution graph $G$ has \textbf{agenticity}
with respect to king $\kappa$ iff:
\begin{enumerate}[label=\textup{(\Alph*)}]
  \item \textbf{Observability.}
  $r$ is spectrally visible: perturbations to $r$'s state produce
  detectable changes in $\lambda_1$.  Formally, the observability
  Gramian
  \[
    W_O(0,T) = \int_0^T \Phi(s,0)^\top C(s)^\top C(s)\,
    \Phi(s,0)\,ds
  \]
  is positive definite on the subspace containing $r$'s coordinates,
  where $\Phi$ is the state transition matrix of $A$.

  \item \textbf{Reachability.}
  $r$ is dynamically accessible: the reachable set
  $\mathcal{R}(T,\mathcal{U})$ of $\kappa$ under dynamics $A$ and
  box constraint $\mathcal{U} = [-\bar{u},\bar{u}]^3$ intersects
  the set of states where $\kappa$ can influence $r$'s edge weight
  $w_{\kappa r}$.  The dynamics $A$---gravity---determines this set.

  \item \textbf{Controllability.}
  $r$ is actuatable: the controllability Gramian
  \[
    W_C(0,T) = \int_0^T \Phi(0,s)\,B\,B^\top\,
    \Phi(0,s)^\top\,ds
  \]
  is positive definite on the subspace coupling $\kappa$ to $r$.
  Equivalently, $\kappa$ can modify $r$'s bypass capacity $c(e_r)$
  through admissible control.
\end{enumerate}
\end{definition}

\begin{theorem}[Kalman duality of agenticity]\label{thm:3b-kalman}
The pair $(A, C)$ is observable if and only if $(A^\top, C^\top)$
is controllable.  In physical terms:
\begin{quote}
\itshape
The ability to detect a sword (observability of $r$ through
$\lambda_1$) is dual to the ability to control it (controllability
of $r$'s bypass capacity).  The matrix $A$---gravity---is the
bridge.
\end{quote}
\end{theorem}

\begin{proof}
Standard Kalman duality: $(A,C)$ observable iff
$\mathrm{rank}[C;\; CA;\; CA^2;\; \ldots;\; CA^{n-1}] = n$, which
holds iff $\mathrm{rank}[C^\top,\; A^\top C^\top,\; \ldots,\;
(A^\top)^{n-1}C^\top] = n$, which is the controllability condition
for $(A^\top, C^\top)$.

Physically: $A$ propagates perturbations forward in time
(observation: a disturbance at $r$ ripples through the transition
kernel until $\lambda_1$ changes) and backward in time (control:
a thrust at $\kappa$ ripples through the same kernel until
$r$'s edge weight changes).  The same dynamics that make the sword
\emph{visible} make it \emph{controllable}.

In the MDP interpretation: $\nabla_{\hat{\omega}} P$ is
simultaneously the \emph{gravity direction} (how the uncontrolled
transition drifts), the \emph{policy gradient} (how the
transition changes under control), and the \emph{score function}
(how the MPPI proposal weights shift).  The Kalman duality says
these are the same object viewed forward and backward in time.
\end{proof}

\begin{corollary}[The drift provides reachability]\label{cor:3b-gravity}
Reachability is the bridge between observability and controllability,
and it is given by the drift $A$---gravity in the physical instance,
the uncontrolled transition in the MDP.  Specifically:
\begin{enumerate}[label=\textup{(\roman*)}]
  \item The reachable set $\mathcal{R}(T,\mathcal{U})$ is shaped by
  $A$: stronger coupling (larger $w_{ij}$) expands
  $\mathcal{R}$, weaker coupling contracts it.
  \item At equilibria of the drift ($\nabla_{q_*}V = 0$ in the
  gravitational instance; stationary points of $P$ in the MDP):
  the drift releases the agent, and all control authority is freed
  for the spectral kick.  Reachability is maximal.
  \item Away from equilibria, the drift consumes control
  authority: the agent must fight the drift before it can control
  $\lambda_1$.  Reachability contracts.
\end{enumerate}
\end{corollary}

\begin{remark}[草木竹石,皆可為劍]\label{rem:3b-caomuzhu}
In Jin Yong's \emph{The Return of the Condor Heroes}
(《神雕侠侣》), Dugu Qiubai's sword progression ends:
\emph{after forty, not bound by material---grass, wood, bamboo,
stone, all can become a sword.}

This is agenticity.  Every object satisfies condition~(A):
everything is observable (you can see grass).  Whether it satisfies
(B) and~(C) depends on the swordsman:
\begin{itemize}
  \item \textbf{(B) Reachability}: can the swordsman reach the
  object?  This is given by the dynamics---gravity, physics,
  proximity.  Dugu Qiubai's 内力 (internal force) extends his
  reachable set until grass is within reach.
  \item \textbf{(C) Controllability}: can the swordsman confer
  force through the object?  This costs $\bar{u} \ge \bar{u}^* > 0$
  (\cref{prop:3b-kakeya}).  Free agenticity does not exist.
\end{itemize}
The massless axiom (\cref{ax:massless}) says: grass has no intrinsic
swordness.  Agenticity is conferred by the controller through a
reachable, controllable path.  The dynamics $A$---gravity---is the
medium through which it is conferred.
\end{remark}

\begin{remark}[青冥宝剑胜龙泉 --- no gravity]\label{rem:3b-yujiaolong}
In Ang Lee's \emph{Crouching Tiger, Hidden Dragon}
(《卧虎藏龙》, 2000), 玉娇龙 (Jen Yu) takes the Green Destiny
and declares:
\begin{quote}\itshape
我乃是潇洒人间一剑仙,青冥宝剑胜龙泉。\\
任凭李俞江南鹤,都要低头求我怜。\\
沙漠飞来一条龙,神来无影去无踪。\\
今朝踏破峨眉顶,明日拔下武当峰!
\end{quote}
The declaration is made at the inn (客栈).  The moment 玉娇龙
draws the sword (拔剑), gravity vanishes \emph{for her}---she is
at the Lagrange point.  This is
\cref{cor:3b-gravity}\textup{(ii)}: $\nabla_{q_*}V = 0$, all
control authority is freed for the spectral kick, and the poem is
spoken from zero gravity.  The bamboo forest duel (竹林) is a
structurally distinct, later scene---玉娇龙 \emph{flies}
there too, but the phase transition already occurred at the inn
when the sword entered her hand.

When 玉娇龙 holds the 青冥宝剑, she is at the Lagrange point of
the 武林 (martial world): the sword confers agenticity, gravity
vanishes, and reachability is maximal.

李安 sees the physics.  The flying is not fantasy---it is the
optimal control policy at the switching surface.
\end{remark}

% ═══════════════════════════════════════════════════════════
\subsection{Dribble the three bodies}

The abstract diagram (\cref{fig:agenticity}) admits a concrete
physical instance that makes the three-term structure literally
visible: \emph{dribbling a ball}.

The ball's three principal inertia axes are the ``three bodies.''
Gravity pulls the ball toward the ground (the constraint boundary
$\lambda_1 = 0$).  The hand (the damper $\kappa$) strikes the ball
at the bottom of each bounce---the Lagrange point where
$\nabla_{q_*}V = 0$ in the co-moving frame.  The ball is a closed
manifold $S^2$; the hand's contact point moves on this manifold.

\begin{figure}[H]
\centering
\begin{tikzpicture}[scale=1.5, >=Stealth]
  % ═══ Layer 1 (锁 Lock, blue): hitting surface κ ═══
  % Ground plane — the constraint boundary (parallel view)
  \fill[water!8] (-4.0, -3.4) rectangle (4.0, -3.0);
  \draw[water, line width=0.8pt] (-4.0, -3.0) -- (4.0, -3.0);
  % Hand / paddle — the damper, also Lock (constraint surface)
  \fill[water!30, rounded corners=2pt]
    (-1.0, -3.0) rectangle (1.0, -2.75);
  \draw[water, line width=1pt, rounded corners=2pt]
    (-1.0, -3.0) rectangle (1.0, -2.75);
  \node[water, font=\footnotesize] at (0, -2.875) {$\kappa$};
  \node[water, font=\footnotesize, anchor=east] at (3.9, -3.2)
    {$\Sigma$};

  % ═══ Layer 2 (玉 Jade, green): volume S² ═══
  % Ball — the body that exists
  \begin{scope}[shift={(0, 0.2)}]
    \shade[ball color=green!30!white, opacity=0.35] (0,0) circle (1.8);
    \draw[green!60!black, line width=0.8pt] (0,0) circle (1.8);
    % Longitude lines
    \foreach \a in {-60,-20,20,60} {
      \draw[green!50!black, line width=0.2pt, opacity=0.4]
        (0,0) ellipse ({1.8*cos(\a)} and 1.8);
    }
    % Latitude lines
    \foreach \b in {-0.9, 0, 0.9} {
      \pgfmathsetmacro{\rad}{1.8*cos(asin(\b/1.8))}
      \draw[green!50!black, line width=0.2pt, opacity=0.4]
        (0, \b) ellipse ({\rad} and {0.15*\rad});
    }
    \node[green!60!black, font=\small, anchor=south west]
      at (1.5, 1.4) {$S^2$};
  \end{scope}

  % ═══ Layer 3 (绳 Rope, red): m*g + ground + state ═══
  % Gravity direction: point on ball + downward arrow
  \coordinate (CP) at (0, -1.6);
  \fill[dao, opacity=0.9] (CP) circle (2.5pt);
  \node[dao, font=\small, anchor=west] at (0.2, -1.75)
    {$x(t)$};
  % Gravity arrow
  \draw[->, dao, line width=1.8pt]
    (0, -1.6) -- (0, -2.65)
    node[midway, right=2pt, font=\small, text=dao] {$m_* g$};
  % Three inertia axes (the ``three bodies'')
  \draw[dao, line width=0.6pt, ->] (0,0.2) -- (1.6, 0.2)
    node[right, font=\scriptsize] {$I_1$};
  \draw[dao, line width=0.6pt, ->] (0,0.2) -- (0, 1.8)
    node[above, font=\scriptsize] {$I_2$};
  \draw[dao, line width=0.6pt, ->] (0,0.2) -- (-0.7, -0.5)
    node[below left, font=\scriptsize] {$I_3$};
  % Control force (bang arc, dashed)
  \draw[->, dao, line width=1.4pt, dashed]
    (0, -2.75) -- (0, -1.75)
    node[midway, left=2pt, font=\footnotesize, text=dao] {$u(t)$};

  % ═══ Layer 4 (金 Golden, cyan): operations ═══
  % ω̂ pseudovector
  \draw[->, sword, line width=1.6pt]
    (0, 0.2) -- (1.3, 1.6)
    node[right, font=\small, text=sword]
    {$\hat{\omega}$};
  % ∇_{ω̂} P
  \coordinate (Q) at (1.1, 1.35);
  \draw[->, sword, line width=1.2pt]
    (Q) -- ++(0.9, -0.8)
    node[right, font=\small, text=sword]
    {$\nabla_{\hat{\omega}}\!P$};

  % ── Bounce trajectory (parabolic arc) ────────────────
  \draw[dao, line width=0.5pt, dashed, opacity=0.5]
    plot[smooth, domain=-1.4:1.4, samples=30]
    ({\x + 2.5}, {-0.6*\x*\x + 1.0});

  % ── Arc type annotations ─────────────────────────────
  \node[font=\scriptsize, text=black!60, anchor=west,
        text width=2.8cm, align=left]
    at (-3.9, 2.0)
    {\textbf{singular arc}\\[1pt]
     ball in free flight\\
     $\rho = 0,\;\sim\!90\%$};
  \node[font=\scriptsize, text=black!60, anchor=west,
        text width=2.8cm, align=left]
    at (-3.9, -2.0)
    {\textbf{bang arc}\\[1pt]
     hand strikes ball\\
     $\rho \gg 1,\;\sim\!10\%$};

  % ── Brackets ─────────────────────────────────────────
  \draw[black!40, line width=0.4pt, decorate,
        decoration={brace, amplitude=4pt}]
    (-3.95, 1.8) -- (-3.95, -0.8)
    node[midway, left=5pt, font=\scriptsize, text=black!50] {};
  \draw[black!40, line width=0.4pt, decorate,
        decoration={brace, amplitude=4pt, mirror}]
    (-3.95, -1.2) -- (-3.95, -3.1)
    node[midway, left=5pt, font=\scriptsize, text=black!50] {};

  % ── Layer legend ─────────────────────────────────────
  \begin{scope}[shift={(4.0, 1.8)}]
    \node[anchor=west, font=\scriptsize, text=sword]
      at (0, 0) {\textbf{金}\; $\hat{\omega},\,\nabla_{\hat{\omega}}P$};
    \node[anchor=west, font=\scriptsize, text=dao]
      at (0, -0.5) {\textbf{绳}\; $x(t),\,m_*g,\,u(t)$};
    \node[anchor=west, font=\scriptsize, text=green!50!black]
      at (0, -1.0) {\textbf{玉}\; $S^2$};
    \node[anchor=west, font=\scriptsize, text=water]
      at (0, -1.5) {\textbf{锁}\; $\kappa,\,\Sigma$};
  \end{scope}
\end{tikzpicture}

\medskip

\caption{Dribble the three bodies (ground-level parallel view).
Layered construction:
\textcolor{water}{\textbf{锁}~Lock} (blue): the hitting surface
$\kappa$ (damper) and the ground plane $\Sigma = \{\rho = 1\}$---the
constraint boundary where contact is decided.
\textcolor{green!50!black}{\textbf{玉}~Jade} (green): the ball as
a closed manifold $S^2$---the body that exists.  The three
principal inertia axes $\textcolor{dao}{I_1, I_2, I_3}$ are the
``three bodies.''
\textcolor{dao}{\textbf{绳}~Rope} (red): gravity $m_*g$ (the
drift $A$), the current state $x(t)$ on $S^2$, and the control
force $u(t)$ (dashed, bang arc only).  The Rope acts: it binds
the ball to the ground via gravity, and the agent acts on the
ball via the paddle.
\textcolor{sword}{\textbf{金}~Golden} (cyan): the pseudovector
$\hat{\omega} \in \mathfrak{so}(3)$ and $\nabla_{\hat{\omega}}P$---
the sword: simultaneously the gravity direction, the policy
gradient, and the score function.  On singular arcs ($\rho = 0$,
${\sim}90\%$), the ball flies freely; on bang arcs ($\rho \gg 1$,
${\sim}10\%$), the hand strikes at the Lagrange point.
\Cref{fig:agenticity} is the bird's-eye view of the same structure;
this figure is the ground-level view.}
\label{fig:dribble}
\end{figure}

The dribble makes the three-term decomposition
(\cref{thm:3b-damper}) physically immediate:
\begin{enumerate}[label=\textup{(\Roman*)}]
  \item \textbf{Gravity}: the ball falls.
  \item \textbf{Least action}: the hand follows the ball's parabolic
  arc (singular arc, free flight).
  \item \textbf{Spectral kick}: the hand strikes the ball at the
  bottom of the bounce (bang arc), where $\nabla_{q_*}V = 0$ in the
  co-moving frame and all control authority is freed for the
  spectral kick.
\end{enumerate}
The bang-singular structure is visible to the naked eye: the hand
is in contact for ${\sim}10\%$ of the bounce cycle (bang) and in
free flight for ${\sim}90\%$ (singular).  The Kakeya condition
(\cref{prop:3b-kakeya}) says: to dribble the ball in every
direction on the court, the reachable set of the hand must be
full-dimensional in $\R^3$---one cannot dribble for free.

\subsection{Contact modes as force balance}

The dribble picture gives a precise physical definition of the
contact modes.  At each contact point, exactly two forces compete:
\emph{attraction} (gravity pulling the ball toward the ground) and
\emph{repulsion} (the hand pushing back).  Define the order
parameter
\begin{equation}\label{eq:3b-rho}
  \rho \;=\; \frac{F_{\mathrm{repulsion}}}{F_{\mathrm{attraction}}}.
\end{equation}
The contact mode is determined by $\rho$ alone:

\begin{center}
\begin{tabular}{@{}lcll@{}}
\toprule
\textbf{Regime} & $\rho$ & \textbf{Mode} & \textbf{Arc type} \\
\midrule
Attraction wins & $\rho < 1$ & Separating &
  Singular (free flight) \\
Phase transition & $\rho = 1$ & Sliding &
  Switching surface $\Sigma$ \\
Repulsion wins & $\rho > 1$ & Sticking &
  Bang (spectral kick) \\
Existence & $\rho \to \infty$ & Clamped &
  Attached body \\
\bottomrule
\end{tabular}
\end{center}

\noindent
The three-mode table of \cref{sec:3b-mppi} is recovered:
separating ($\rho < 1$, $w_i = 0$), sliding ($\rho \approx 1$,
$w_i = k_n + \mu k_n$), sticking ($\rho > 1$,
$w_i = k_n + k_t$).  But the dribble formulation adds the
fourth regime: \emph{existence}.  When $\rho \to \infty$, the
contact is no longer a contact---the ball is clamped to the hand.
The three ``bodies'' (inertia axes) become rigidly attached to the
agent $\kappa$, and $\lambda_1 \to \infty$.  The spectral
constraint $\lambda_1 \ge \epsilon$ is trivially satisfied.  No
control is needed: the object \emph{exists} as part of the agent.

The phase transition at $\rho = 1$ is the switching surface
$\Sigma$ of the bang-singular structure: attraction equals
repulsion, the Lagrange point of the grasp.  This is the point
where the hand must decide---push harder (bang) or release
(singular).  The entire MPPI sampling reduces from $3^n$ discrete
modes to sampling over the continuous parameter $\rho$ at each
contact point.

\begin{remark}[Two forces, one transition]
\label{rem:3b-twoforces}
Every contact in nature is a competition between attraction and
repulsion.  Gravity pulls; the hand pushes.  The Pauli exclusion
principle prevents collapse; electromagnetic attraction prevents
escape.  The dribble picture strips the contact mode classification
to its essence: there is only one phase transition, and it occurs
at $\rho = 1$.  Everything else---the friction cone, the signed
distance function, the contact Jacobian---is parameterisation of
the neighbourhood of this transition.
\end{remark}

% ═══════════════════════════════════════════════════════════
\section{Implementation: the physics-backend isomorphism}
\label{sec:3b-mujoco}

The gravity damper OCP admits two implementations that share the
same controller but differ in the physics backend.  The point is
not that both work---the point is that they \emph{must} work,
because the controller reads only $\rho$.

\subsection{Backend~1: hand-rolled Euler--Lagrange}

Three point masses in $\R^3$, interacting via Newtonian gravity.
The Euler--Lagrange equations are integrated by symplectic Euler:
\[
  M\ddot{q} \;=\; -\nabla V(q) \;+\; B\,u,
\]
where $V(q) = -\sum_{i<j} Gm_im_j/\|q_i - q_j\|$ and $B$
selects the damper.  The controller monitors $\lambda_1(L_G)$ and
applies the three-term control law \eqref{eq:3b-threeterm}.

The order parameter is the tidal coupling ratio:
\[
  \rho_{*j}
  \;=\; \frac{w_{*j}}{\bar{w}_{\mathrm{body}}}
  \;=\; \frac{Gm_* m_j / \|q_* - q_j\|^3}
             {\tfrac{1}{|E_0|}\sum_{(i,k)\in E_0} Gm_im_k / \|q_i - q_k\|^3},
\]
where $E_0$ are the body--body edges.  When $\rho_{*j} < 1$, the
damper is losing tidal authority over body $j$---the spectral kick
fires.

The simulation code is in \texttt{grjl/threebody\_damper.py} (v1.0,
$\lambda_1$-triggered) and \texttt{grjl2/threebody\_rho.py} (v2.0,
$\rho$-triggered):
\begin{verbatim}
  python grjl/threebody_damper.py --headless
  python grjl2/threebody_rho.py --solver reactive --headless
\end{verbatim}

\subsection{Backend~2: MuJoCo contact dynamics}
\label{sec:3b-dribble-mujoco}

Replace the hand-rolled gravity with MuJoCo's contact-aware
Euler--Lagrange integrator~\cite{mujoco}:
\[
  M(q)\ddot{q} + C(q,\dot{q})\dot{q}
  \;=\; \tau + J_c^\top\lambda_c,
\]
where $J_c^\top\lambda_c$ is the contact wrench computed by MuJoCo's
constraint solver (complementarity, friction cone, restitution---all
handled by \texttt{mj\_step()}).  The scene is the dribble of
\cref{fig:dribble}: a ball with non-uniform inertia
($I_1 \neq I_2 \neq I_3$, the ``three bodies'') bounced by a paddle
(the damper $\kappa$) above a ground plane.

The controller is a thin loop that reads the contact force from
MuJoCo's sensor data:
\[
  \rho(t) \;=\; \frac{F_{\mathrm{paddle}}(t)}{m\,g}.
\]
Everything else---the three-term decomposition, the bang-singular
switching, the spectral kick---is identical.  The controller does
not call any MuJoCo-specific function beyond reading sensor values
and writing actuator targets.

The simulation code is in \texttt{grjl2/dribble\_controller.py}:
\begin{verbatim}
  python grjl2/dribble_controller.py --headless
\end{verbatim}

\subsection{The isomorphism}

The two backends are related by the following dictionary:

\begin{center}
\begin{tabular}{@{}lll@{}}
\toprule
& \textbf{Backend 1} (gravity) & \textbf{Backend 2} (MuJoCo) \\
\midrule
EL equation & $M\ddot{q} = -\nabla V + Bu$ &
  $M(q)\ddot{q} + C\dot{q} = \tau + J_c^\top\lambda_c$ \\
Integrator & symplectic Euler & \texttt{mj\_step()} \\
``Three bodies'' & point masses $m_1,m_2,m_3$ &
  inertia axes $I_1,I_2,I_3$ \\
Damper $\kappa$ & controlled mass $m_*$ & paddle actuator \\
Constraint boundary & mass gap $\lambda_1 = 0$ & ground plane \\
$F_{\text{attraction}}$ & body--body tidal coupling &
  gravity $mg$ \\
$F_{\text{repulsion}}$ & damper--body tidal coupling &
  paddle contact force \\
\midrule
\textbf{Order parameter} & \multicolumn{2}{c}{$\rho
  = F_{\text{repulsion}} / F_{\text{attraction}}$} \\
\textbf{Control law} & \multicolumn{2}{c}{three-term:
  gravity $+$ least action $+$ spectral kick} \\
\textbf{Invariant} & \multicolumn{2}{c}{$\lambda_1 \ge \epsilon$
  (spectral gap)} \\
\bottomrule
\end{tabular}
\end{center}

\noindent
The top half of the table changes between backends; the bottom half
does not.  This is the precise claim: \emph{the controller is a
functor from the category of Euler--Lagrange systems to the category
of bang-singular control laws, with $\rho$ as the natural
transformation.}  Swapping the physics backend---from Newtonian
gravity to MuJoCo contact dynamics, or to any other system admitting
a force balance---changes only the source object.  The image
(the three-term control, the spectral gap, the bang-singular
structure) is preserved.

\begin{remark}[What MuJoCo actually contributes]
\label{rem:3b-mujoco-role}
MuJoCo contributes exactly one thing: the constraint solver that
computes $J_c^\top\lambda_c$.  This is the contact
Euler--Lagrange equation---the normal force, friction cone,
restitution, and complementarity conditions that determine whether
the ball bounces, slides, or sticks.  The controller does not need
to know \emph{how} MuJoCo computes this.  It reads the resulting
$F_{\mathrm{paddle}}$ from the sensor, divides by $mg$, and
obtains $\rho$.  The entire complexity of contact mechanics is
absorbed into the physics backend; the controller sees only the
order parameter.
\end{remark}

\begin{remark}[Validation chains]\label{rem:validation-chains}
\textcolor{sword}{Two independent chains validate the framework.
Each chain proceeds from axioms to observable, with the
physics backend as the only engine-specific component.}

\medskip\noindent
\textbf{Chain A} (Embodied: XML $\to$ 翻手为云覆手为雨).
\begin{gather*}
  \underbrace{\texttt{dribble\_\{down,up\}.xml}}_{\text{scene}}
  \;\xrightarrow{\;\texttt{mj\_step}(m,d)\;}
  \underbrace{q(t),\; \dot{q}(t)}_{\text{sensor}}
  \;\xrightarrow{\;\rho = |\ddot{q}_z/g + 1|\;}
  \underbrace{\text{DualDribbleController}}_{\text{three-term}}
  \\[4pt]
  \xrightarrow{\;\mathcal{G}\;}
  \underbrace{\text{拍球}\;\|\;\text{颠球}}_{\text{ground duality}}
\end{gather*}
The XML defines the entire physics: ball inertia, ground plane,
paddle actuators, contact parameters.  The call
\texttt{mj\_step(model, data)} is the \emph{only}
engine-specific line; replacing MuJoCo with any contact-aware
Euler--Lagrange integrator changes nothing above $\rho$.
\Cref{fig:agenticity} is the bird's-eye projection (翻手为云);
\cref{fig:dribble} is the ground-level projection (覆手为雨).

\medskip\noindent
\textbf{Chain B} (Analytical: three-body $\to$ spectral gap).
\[
  \underbrace{m_1, m_2, m_3,\; G}_{\text{masses + gravity}}
  \;\xrightarrow{\;\text{RK4 / symplectic}\;}
  \underbrace{q(t),\;\dot{q}(t)}_{\text{state}}
  \;\xrightarrow{\;\rho_{*j} = w_{*j}/\bar{w}\;}
  \underbrace{\text{SpectralPID / reactive}}_{\text{three-term}}
  \;\xrightarrow[\text{maintained}]{\;\lambda_1 \ge \epsilon\;}
  \underbrace{\text{grasp}}_{\text{viability}}
\]
No XML, no contact solver.  The masses interact via Newtonian
gravity; the integrator is hand-rolled; the controller monitors
$\lambda_1(L_G)$ and fires the spectral kick when $\rho_{\min} < 1$.

\medskip\noindent
\textbf{What the two chains share} (the bottom half of the
isomorphism table above):
\begin{enumerate}[label=(\roman*),nosep]
  \item the order parameter $\rho$,
  \item the three-term control law (gravity $+$ least action
    $+$ spectral kick),
  \item the spectral gap invariant $\lambda_1 \ge \epsilon$.
\end{enumerate}
\textbf{What differs} (the top half): the physics backend.
Chain~A uses MuJoCo contact dynamics; Chain~B uses Newtonian
gravity.  The controller is a functor: it maps any
Euler--Lagrange source to the same bang-singular target.
\end{remark}

\subsection{Force elimination: the kinematic reduction}
\label{sec:3b-force-elim}

\begin{center}
\itshape
I now demonstrate the degrees of freedom of the systems of the
world.
\end{center}

\medskip

\noindent
Newton demonstrated the frame of one system: $F = ma$, force
determines acceleration, acceleration determines the orbit.  We
invert the arrow.  Given the mass--inertia matrix $M(q)$---which
is geometry, fixed by the scene---acceleration determines force.
Force is therefore not a degree of freedom.  It is a
\emph{derived quantity}, eliminable from the control law
entirely.  The true degrees of freedom are kinematic:
$(q, \dot{q})$.

Every Euler--Lagrange system, regardless of the physics backend,
has the form
\begin{equation}\label{eq:3b-el-general}
  M(q)\,\ddot{q} \;=\; \tau_{\mathrm{ext}}(q, \dot{q}, u),
\end{equation}
where $M(q)$ is the mass--inertia matrix and
$\tau_{\mathrm{ext}}$ collects all external forces
(gravity, contact wrenches, control inputs, Coriolis/centrifugal
terms absorbed into the right-hand side).  Crucially, $M(q)$ is
\emph{determined by geometry alone}: it depends on the mass
distribution and the kinematic chain, both of which are fixed by
the scene description (the XML file, the URDF, or the gravitational
constants $Gm_im_j$).

\begin{theorem}[Force elimination]\label{thm:3b-force-elim}
Let $(q(t), \dot{q}(t))$ be a trajectory of the EL
system~\eqref{eq:3b-el-general} with known mass--inertia matrix
$M(q)$.  Then the generalised force is uniquely determined by the
kinematics:
\begin{equation}\label{eq:3b-force-recovery}
  \tau_{\mathrm{ext}}(t) \;=\; M(q(t))\,\ddot{q}(t).
\end{equation}
In particular, the order parameter $\rho$ is a function of
$(q, \dot{q}, \ddot{q})$ alone:
\begin{equation}\label{eq:3b-rho-kinematic}
  \rho(t) \;=\; \frac{F_{\mathrm{repulsion}}(t)}{F_{\mathrm{attraction}}(t)}
  \;=\; \frac{[\,M(q)\,\ddot{q}\,]_{\mathrm{contact}}}
             {[\,M(q)\,\ddot{q}\,]_{\mathrm{gravity}}},
\end{equation}
where $[\cdot]_{\mathrm{contact}}$ and
$[\cdot]_{\mathrm{gravity}}$ project onto the contact and
gravitational components of the wrench respectively.
\end{theorem}

\begin{proof}
$M(q)$ is symmetric positive-definite for any physical system
(it is the Hessian of the kinetic energy with respect to
$\dot{q}$).  Hence $M(q)$ is invertible at every configuration
$q$, and \eqref{eq:3b-el-general} gives a bijection between
$\ddot{q}$ and $\tau_{\mathrm{ext}}$ for fixed $(q, \dot{q})$.
The projection follows from the linearity of the wrench
decomposition.
\end{proof}

\noindent
The consequence is immediate:

\begin{corollary}[Kinematic controller]\label{cor:3b-kinematic}
The three-term control law (\cref{thm:3b-damper}) can be
implemented entirely in kinematic variables
$(q, \dot{q})$ without ever computing or commanding
forces.  The controller sets position targets $q^{\mathrm{des}}$;
the physics backend converts these to forces via the EL equation;
the resulting contact forces are \emph{observed} (not commanded)
to compute $\rho$.
\end{corollary}

\noindent
This is the structure of both implementations:

\begin{center}
\begin{tabular}{@{}lccc@{}}
\toprule
\textbf{Stage} & \textbf{Variable} & \textbf{Who computes} &
  \textbf{Domain} \\
\midrule
1. Controller output & $q^{\mathrm{des}}(t)$
  & controller & kinematic \\
2. Actuator force & $\tau_a = k_p(q^{\mathrm{des}} - q)$
  & physics backend & dynamic \\
3. Contact force & $F_c = J_c^\top\lambda_c$
  & physics backend & dynamic \\
4. Order parameter & $\rho = F_c / (mg)$
  & controller (read) & kinematic \\
\bottomrule
\end{tabular}
\end{center}

\noindent
Force appears only in stages~2 and~3, both internal to the physics
backend.  The controller's interface is purely kinematic: it writes
$q^{\mathrm{des}}$ and reads $\rho$.  Force is a
\emph{latent variable}---it mediates between the controller and the
constraint, but is never part of the control law itself.

\begin{remark}[Force as a gauge variable]
\label{rem:3b-gauge}
The relationship between $\ddot{q}$ and $\tau$ via $M(q)$ is
analogous to a gauge transformation in field theory: the physics is
in the acceleration (the curvature); the force is the connection
(the potential).  Different physics backends choose different
``gauges''---Newtonian gravity uses $\nabla V$, MuJoCo uses
$J_c^\top\lambda_c$---but the observable $\rho$ is gauge-invariant.
This is why the same controller works on any EL backend: it reads
the gauge-invariant quantity.
\end{remark}

\begin{remark}[Reconciliation of the three $\rho$ definitions]
\label{rem:3b-rho-reconcile}
The three definitions of~$\rho$ are gauge-equivalent in the
sense of \cref{rem:3b-gauge}.  They differ in reference frame,
not in physics:
\begin{enumerate}[label=(\roman*),nosep]
  \item \emph{Contact force ratio} (\cref{eq:3b-rho}):
  $\rho = F_{\mathrm{rep}}/F_{\mathrm{att}}$.
  Canonical definition.  Zero during free flight (no contact).
  \item \emph{Tidal coupling ratio}
  (\cref{sec:3b-dribble-mujoco}):
  $\rho_{*j} = w_{*j}/\bar{w}_{\mathrm{body}}$.
  Positive during free flight (gravity persists).  This is
  a \emph{proxy} for the canonical~$\rho$: it predicts when
  contact will occur, using the tidal ratio as a leading indicator.
  \item \emph{Kinematic form} (\cref{eq:3b-rho-kinematic}):
  $\rho = |\ddot{q}_z/g + 1|$.  Algebraically equivalent
  to~(i) via the force elimination theorem
  (\cref{thm:3b-force-elim}).
\end{enumerate}
All three agree at the switching surface~$\Sigma$: $\rho = 1$
is the same event regardless of gauge.  The tidal proxy~(ii)
is informative between contacts precisely because it monitors
the approach to~$\Sigma$ before contact forces materialise.
Chain~B (\cref{rem:validation-chains}) uses the proxy to
predict $\Sigma$-crossing; the controller's response is
triggered by the gauge-invariant $\rho = 1$.
\end{remark}

\subsection{Ground duality: 拍球 and 颠球}
\label{sec:3b-ground-duality}

The force elimination theorem reveals a deeper symmetry.  Consider
two dribbling configurations:

\begin{center}
\begin{tabular}{@{}lcc@{}}
\toprule
& \textbf{拍球} (dribble down) & \textbf{颠球} (juggle up) \\
\midrule
Hand position & above ball & below ball \\
Strike direction & downward ($-\hat{z}$)
  & upward ($+\hat{z}$) \\
Gravity pulls ball & toward ground (below)
  & toward hand (below) \\
Bounce surface & ground (below) & hand itself \\
\bottomrule
\end{tabular}
\end{center}

\noindent
Define the \emph{ground reflection} $\mathcal{G}$:
\begin{equation}\label{eq:3b-ground-dual}
  \mathcal{G}: \quad
  z \;\mapsto\; -z, \qquad
  g \;\mapsto\; -g, \qquad
  F_{\mathrm{hand}} \;\mapsto\; -F_{\mathrm{hand}}.
\end{equation}
Under $\mathcal{G}$, the EL equation transforms as
\[
  M\ddot{q} = \tau_{\mathrm{ext}}
  \quad\xmapsto{\;\mathcal{G}\;}
  \quad M\ddot{q}' = \tau_{\mathrm{ext}}',
\]
where $\ddot{q}' = -\ddot{q}_z$ in the vertical component and
$\tau_{\mathrm{ext}}' = -\tau_{\mathrm{ext},z}$.  The mass
matrix $M(q)$ is invariant (it depends on mass distribution,
not on the direction of gravity).

\begin{proposition}[Ground duality]\label{prop:3b-ground-dual}
The order parameter $\rho$ is invariant under the ground
reflection~$\mathcal{G}$:
\[
  \rho \;=\; \frac{|F_{\mathrm{hand}}|}{m\,|g|}
  \;=\; \frac{|{-F_{\mathrm{hand}}}|}{m\,|{-g}|}
  \;=\; \rho'.
\]
The three-term control law, the bang-singular structure, and the
spectral gap constraint are all preserved.
\end{proposition}

\begin{proof}
$\rho$ depends only on magnitudes: $|F_{\mathrm{repulsion}}|$
and $|F_{\mathrm{attraction}}|$.  The reflection $\mathcal{G}$
negates both, leaving the ratio unchanged.  The switching surface
$\Sigma = \{\rho = 1\}$ is therefore $\mathcal{G}$-invariant,
and the bang-singular decomposition on either side of $\Sigma$
is preserved.
\end{proof}

We now make the duality brutally precise.  Define the \emph{primal}
and \emph{dual} optimal control problems.

\begin{definition}[Primal--dual pair]\label{def:3b-primal-dual}
Fix the initial state $x_0 = (q(0), \dot{q}(0))$, the horizon $T$,
the spectral gap $\epsilon > 0$, and the actuation bound $\bar{u}$.

\medskip\noindent
\textbf{Primal problem $\mathsf{P}$} (拍球: hand above, palm down).
Gravity $g = -|g|\hat{z}$, hand force
$F_{\mathrm{hand}} = -|F|\hat{z}$ (downward strike), ground plane
at $z = 0$:
\begin{equation}\label{eq:3b-primal}
  J_{\mathsf{P}}^*
  \;=\; \min_{u \in \mathcal{U}} \int_0^T \!\Bigl[
    \mathcal{L}(q, \dot{q}) + \tfrac{\alpha}{2}\|u\|^2
  \Bigr] dt
  \quad\text{s.t.}\quad
  \lambda_1\bigl(L_G(q(t))\bigr) \ge \epsilon
  \;\;\forall\, t.
\end{equation}

\medskip\noindent
\textbf{Dual problem $\mathsf{D}$} (颠球: hand below, palm up).
Apply the ground reflection $\mathcal{G}$
(\cref{eq:3b-ground-dual}) to every quantity in $\mathsf{P}$:
gravity $g' = +|g|\hat{z}$, hand force
$F_{\mathrm{hand}}' = +|F|\hat{z}$ (upward strike), effective
ground at $z = z_{\mathrm{hand}}$:
\begin{equation}\label{eq:3b-dual}
  J_{\mathsf{D}}^*
  \;=\; \min_{u' \in \mathcal{U}} \int_0^T \!\Bigl[
    \mathcal{L}'(q', \dot{q}') + \tfrac{\alpha}{2}\|u'\|^2
  \Bigr] dt
  \quad\text{s.t.}\quad
  \lambda_1\bigl(L_G(q'(t))\bigr) \ge \epsilon
  \;\;\forall\, t.
\end{equation}
\end{definition}

\begin{theorem}[Strong duality: 翻手为云覆手为雨]
\label{thm:3b-strong-duality}
Let $u^*(t)$ solve the primal problem~$\mathsf{P}$.  Then
$u'(t) = \mathcal{G}\,u^*(t)$ solves the dual problem~$\mathsf{D}$,
and the optimal values coincide:
\begin{equation}\label{eq:3b-strong-dual}
  J_{\mathsf{P}}^* \;=\; J_{\mathsf{D}}^*.
\end{equation}
Moreover, the three-term decomposition, the bang-singular structure,
and the switching surface $\Sigma = \{\rho = 1\}$ are preserved
identically.  In particular:
\begin{enumerate}[label=\textup{(\roman*)}]
  \item The costates satisfy $p'(t) = \mathcal{G}\,p(t)$;
  \item The spectral multiplier satisfies $\mu'(t) = \mu(t)$;
  \item The optimal control satisfies $u'^*(t) =
    \mathrm{sat}_{\bar{u}}\bigl(-\frac{1}{\alpha m_*}\,
    p'_{\dot{q}_*}(t)\bigr)$, i.e.\ the same saturation law;
  \item The bang fraction $\beta = |\{t : \|u^*\| = \bar{u}\}|/T$
    is the same in both problems;
  \item The order parameter $\rho(t)$ is pointwise identical
    (\cref{prop:3b-ground-dual}).
\end{enumerate}
\end{theorem}

\begin{proof}
$\mathcal{G}$ is an involution ($\mathcal{G}^2 = \mathrm{id}$) that
acts on the extended state-costate space
$(x, p, u, \mu) \in T^*\mathcal{M} \times \mathcal{U} \times
\R_{\ge 0}$.  We verify that $\mathcal{G}$ preserves every
ingredient of the Pontryagin system.

\emph{Step~1: Lagrangian invariance.}
The kinetic energy $T = \frac{1}{2}\sum m_i \|\dot{q}_i\|^2$ is
quadratic in velocities; $\mathcal{G}$ negates $\dot{q}_z$, so
$\|\dot{q}\|^2$ is invariant.  The potential
$V = -\sum G m_i m_j / \|q_i - q_j\|$ depends on pairwise
distances; $\mathcal{G}$ negates all $z$-components simultaneously,
so $\|q_i - q_j\|$ is invariant.  Hence
$\mathcal{L}' = \mathcal{L}$.

\emph{Step~2: Control cost invariance.}
$\|u'\|^2 = \|\mathcal{G}\,u\|^2 = \|u\|^2$ since $\mathcal{G}$
is an isometry on $\R^3$ (it negates one component).

\emph{Step~3: Spectral constraint invariance.}
The graph Laplacian $L_G$ depends on edge weights
$w_{ij} = G m_i m_j / \|q_i - q_j\|^3$, which depend only on
pairwise distances.  By Step~1, these are $\mathcal{G}$-invariant.
Hence $\lambda_1(L_G(q')) = \lambda_1(L_G(q))$, and the constraint
$\lambda_1 \ge \epsilon$ maps to itself.

\emph{Step~4: Hamiltonian covariance.}
The control Hamiltonian transforms as
\[
  \mathcal{H}'(x',p',u',\mu') = \mathcal{H}(x,p,u,\mu)
\]
because every term is $\mathcal{G}$-invariant (Steps~1--3) and the
symplectic pairing $p \cdot f(x,u)$ transforms covariantly under
the canonical extension of $\mathcal{G}$.  Therefore, if $(x,p,u,\mu)$
satisfies the PMP necessary conditions, so does
$(x',p',u',\mu') = (\mathcal{G}\,x, \mathcal{G}\,p,
\mathcal{G}\,u, \mu)$.

\emph{Step~5: Optimality.}
Since $J[\mathcal{G}\,u] = J[u]$ (Steps~1--2), the minimum values
coincide.  Items (i)--(v) follow from the covariance of the
Pontryagin system: the costates, multiplier, saturation, bang
fraction, and $\rho$ are all determined by the Hamiltonian flow,
which commutes with $\mathcal{G}$.
\end{proof}

\begin{remark}[What the strong duality theorem says]
\label{rem:3b-strong-duality}
\textcolor{sword}{%
Equation~\eqref{eq:3b-strong-dual} is the mathematical content of
翻手为云覆手为雨.  It is not a metaphor.  The primal problem
$\mathsf{P}$ is 拍球 (dribble down: palm over, clouds gather).
The dual problem $\mathsf{D}$ is 颠球 (juggle up: palm under, rain
falls).  The ground reflection $\mathcal{G}$ is the duality map.
Strong duality $J_{\mathsf{P}}^* = J_{\mathsf{D}}^*$ says:
\emph{it costs exactly the same to dribble down as to juggle up.}
The five invariants (i)--(v) say: not only the cost, but every
structural feature of the solution---costates, multiplier,
switching surface, bang fraction, order parameter---is preserved
under flipping the hand.  There is no ``easier'' direction.
The controller that reads $\rho$ cannot distinguish $\mathsf{P}$
from $\mathsf{D}$.  This is what Du~Fu meant by 纷纷轻薄何须数:
the transitions come and go, but the invariant $\rho$ persists,
and it does not know which way is up.}
\end{remark}

\noindent
拍球 and 颠球 are therefore the \emph{same controller viewed from
opposite sides of the ground plane}.  The ``ground'' is not a
physical surface---it is the constraint boundary $\phi = 0$ in the
signed distance field.  In~拍球, the ground is below
($\phi = z_{\mathrm{ball}} - z_{\mathrm{ground}}$); in~颠球,
the effective ``ground'' is the hand itself, and gravity serves
as the restoring force that returns the ball to the hand after
the kick.  The duality is:

\begin{center}
\begin{tabular}{@{}lcc@{}}
\toprule
& \textbf{拍球} & \textbf{颠球} \\
\midrule
Constraint boundary & ground plane & hand surface \\
Restoring force & ground reaction & gravity \\
Kick direction & $-\hat{z}$ (hand pushes down)
  & $+\hat{z}$ (hand pushes up) \\
Free flight & ball rises after bounce
  & ball falls after kick \\
$\rho = 0$ & ball in air, no contact & ball in air, no contact \\
$\rho > 1$ & hand striking ball down & hand striking ball up \\
Controller & \multicolumn{2}{c}{identical modulo $\mathcal{G}$} \\
\bottomrule
\end{tabular}
\end{center}

\begin{remark}[The ground is not a place]
\label{rem:3b-ground}
In the abstract framework, ``the ground'' is the zero level set
of the signed distance function $\phi$, i.e.\ the constraint
boundary where $\lambda_1 = 0$ if the damper fails.  It is not
a physical surface---it is the \emph{locus of loss of control}.
The ground duality says: the agent can work from either side of
this locus, pushing toward it or pulling away from it, and the
control law is the same.  What matters is $|\rho|$, not
the sign of $g$.
\end{remark}

\begin{remark}[翻手为云,覆手为雨]
\label{rem:3b-fanshou}

\begin{center}\itshape
翻手作云覆手雨,纷纷轻薄何须数。

\medskip
\normalfont\small
Flip the hand---clouds.  Turn it over---rain.

Those who come and go so lightly, why bother counting them?

\hfill ---~杜甫 (Du Fu), 《贫交行》
\end{center}

\medskip\noindent
Du Fu wrote of fair-weather friends.  We read it as a theorem
about the ground reflection $\mathcal{G}$.

\medskip\noindent
\textbf{The hand is the agent.}  In~拍球 the palm faces down:
the hand pushes the ball into the ground, and the ground pushes
it back.  In~颠球 the palm faces up: the hand catches the ball
from below, and gravity pulls it back.  Same hand.
Same five fingers.  Same musculature.  The only difference is the
sign of $\hat{z}$.  This is the ground reflection
\eqref{eq:3b-ground-dual}: $z \mapsto -z$, $g \mapsto -g$,
$F_{\mathrm{hand}} \mapsto -F_{\mathrm{hand}}$.

\medskip\noindent
\textbf{翻手 is \cref{fig:agenticity}.}  Bird's-eye view: you
look down at the manifold from above.  The gravitational field
points into the page~($\bigotimes$).  The hand is between you
and the ball---palm up, gathering clouds.  The layers build
upward toward you:
\begin{enumerate}[label=\arabic*., nosep, leftmargin=2em]
  \item \textcolor{dao}{\textbf{绳}} (Rope): the gravity field
    $m_*g$, pointing away from you.
  \item \textcolor{water}{\textbf{锁}} (Lock): the manifold net
    $\mathcal{X}$ that catches the state.
  \item \textcolor{green!50!black}{\textbf{玉}} (Jade): the
    tetrahedron $\Delta^3$ that exists.
  \item \textcolor{dao}{\textbf{绳}} (Rope): the state $x(t)$,
    where the agent stands.
  \item \textcolor{sword}{\textbf{金}} (Golden): the sword
    $\hat{\omega}$ and $\nabla_{\hat{\omega}}P$---the operation.
\end{enumerate}

\medskip\noindent
\textbf{覆手 is \cref{fig:dribble}.}  Turn the hand over.  Ground
level, parallel view: you stand beside the ball and look across.
Gravity points down the page~($\downarrow$).  The hand is above
or below the ball---palm down, making rain.  The layers stack from
ground to sky:
\begin{enumerate}[label=\arabic*., nosep, leftmargin=2em]
  \item \textcolor{water}{\textbf{锁}} (Lock): the hitting surface
    $\kappa$ and the ground $\Sigma$.
  \item \textcolor{green!50!black}{\textbf{玉}} (Jade): the ball
    $S^2$ that exists.
  \item \textcolor{dao}{\textbf{绳}} (Rope): $m_*g$ and $x(t)$
    and $u(t)$---the forces that bind and act.
  \item \textcolor{sword}{\textbf{金}} (Golden): $\hat{\omega}$
    and $\nabla_{\hat{\omega}}P$---the same sword.
\end{enumerate}

\medskip\noindent
\textbf{The four characters are $\mathcal{G}$-invariant.}
Under the ground reflection, every layer maps to itself:

\begin{center}
\begin{tabular}{@{}clcc@{}}
\toprule
& \textbf{Layer}
  & \textbf{翻手} (\cref{fig:agenticity})
  & \textbf{覆手} (\cref{fig:dribble}) \\
\midrule
\textcolor{sword}{\textbf{金}} & sword / operation
  & $\hat{\omega},\;\nabla_{\hat{\omega}}P$
  & $\hat{\omega},\;\nabla_{\hat{\omega}}P$ \\
\textcolor{dao}{\textbf{绳}} & rope / act
  & $m_*g\;(\bigotimes),\;x(t)$
  & $m_*g\;(\downarrow),\;x(t),\;u(t)$ \\
\textcolor{green!50!black}{\textbf{玉}} & jade / exist
  & $\Delta^3$ & $S^2$ \\
\textcolor{water}{\textbf{锁}} & lock / support
  & $\mathcal{X}$ & $\kappa,\;\Sigma$ \\
\bottomrule
\end{tabular}
\end{center}

\noindent
The \textcolor{sword}{金}~(Golden) row is identical: the operations
do not know which side of the ground they are on.  This is the
content of force elimination (\cref{thm:3b-force-elim}): the
controller reads $(q, \dot{q}, \ddot{q})$, and kinematics is
$\mathcal{G}$-invariant.  The \textcolor{dao}{绳}~(Rope) row
gains $u(t)$ in the ground view because the control force becomes
visible as a separate arrow; in the bird's-eye view it is absorbed
into the state trajectory.  The \textcolor{green!50!black}{玉}~(Jade)
row changes shape ($\Delta^3 \leftrightarrow S^2$) but not
substance: both are the bodies that exist.  The
\textcolor{water}{锁}~(Lock) row changes instantiation
($\mathcal{X} \leftrightarrow \kappa$) but not function: both
are the constraint surface that the dynamics rests on.

\medskip\noindent
\textbf{纷纷轻薄何须数.}
\emph{Those who come and go so lightly, why bother counting them.}
Du Fu's dismissal of the fickle is the bang-singular ratio.  On
singular arcs (${\sim}90\%$ of the cycle), the ball is in
weightless flight: $\rho = 0$, no contact, no force, no
counting needed.  The ball drifts, lightly, through the air---
轻薄, as Du Fu says.  These arcs are free: they cost nothing.

The drama is in the ${\sim}10\%$.  On bang arcs, $\rho \gg 1$:
the hand strikes, the switching surface is crossed, the spectral
kick fires.  But even these transitions are instantaneous---sharp,
not gradual.  They come and go too quickly to count.  The control
law does not count bounces; it reads $\rho$ and switches.  何须数.

What endures is not the transitions.  What endures is the
invariant: $\rho = F_{\mathrm{repulsion}} / F_{\mathrm{attraction}}$.
The hand flips; the clouds become rain; the ratio persists.
翻手为云,覆手为雨: the duality is not a symmetry to be
admired---it is a \emph{gauge redundancy to be eliminated}.
The ground reflection $\mathcal{G}$ is the gauge transformation.
The order parameter $\rho$ is the gauge-invariant observable.
The controller that reads $\rho$ does not know, and need not know,
whether the palm faces up or down.
\end{remark}

\begin{remark}[Vibrational friction and the spectral gap]
\label{rem:3b-superconductor}
Place an ice block on a high-frequency oscillating table.
Friction acts at every instant, but the net force averages to
zero over each period: the block does not move.
This is \emph{vibrational friction averaging}---a classical
instance of force elimination (\cref{thm:3b-force-elim}).
The force (a gauge variable) cancels; the energy dissipation
(a gauge-invariant quantity) does not: friction is still
converting kinetic energy to heat at every instant.

The dribbling controller of~\cref{sec:3b-dribble-mujoco} is
the same mechanism: the ball oscillates between contact
($\rho > 1$) and free flight ($\rho = 0$), the contact forces
average over the cycle, and the controller maintains the
position through bang-singular switching.
Energy is dissipated at every bounce.

Superconductivity achieves the stronger result.  In a normal
metal, electrons scatter off lattice vibrations (phonons)---
friction---producing resistance.  Below the critical temperature
$T_c$, the \emph{same phonons} mediate Cooper pairing: the
vibration that caused friction now binds electrons into a
coherent condensate.  The BCS ground state opens a spectral
gap $\Delta > 0$.  Below the gap, no scattering states
exist: dissipation itself ceases, not merely its average.

The structural distinction is the spectral gap:
\begin{center}
\begin{tabular}{@{}lcc@{}}
\toprule
& \textbf{Classical (ice/dribble)}
& \textbf{Quantum (superconductor)} \\
\midrule
Vibration & oscillating table / bounce
  & phonon \\
Net force & averages to zero & zero (gap) \\
Dissipation & \textcolor{dao}{persists} (heat)
  & \textcolor{water}{ceases} ($\Delta > 0$) \\
$\lambda_1$ & maintained externally
  ($\bar{u} \geq \bar{u}^*$)
  & self-maintained
  (condensate) \\
Paper analogue & pre-sword (envelope persists)
  & resolution (sword eliminated) \\
\bottomrule
\end{tabular}
\end{center}
The spectral gap $\lambda_1 > 0$ of~\cref{thm:massgap}
is the same object as the superconducting gap $\Delta > 0$
in BCS theory: both are the condition under which the system
resists perturbation without dissipation.
The gravity damper maintains $\lambda_1 \geq \epsilon$ by
active control (external energy input); the superconductor
maintains $\Delta > 0$ by internal coherence (Cooper
condensate).  In the language of~\cref{def:3b-ustar}: the
classical system pays $\bar{u}^* > 0$; the quantum system
achieves $\bar{u}^* = 0$ below~$T_c$.  Hugo's triviality
(\cref{thm:alien-triviality}) predicts when the gap closes:
at $d = 4$ in the thermodynamic limit, i.e., at the critical
dimension of our spacetime.
\end{remark}

\subsection{Results}

\begin{description}
  \item[Backend~1 (gravity).]  Without damper: $\lambda_1$ decays
  to zero within ${\sim}100$ steps.  With damper: $\rho$ crosses
  $1.0$ at each bang-singular transition (16 phase crossings over
  $T = 8$\,s).  The PMP solver achieves ${\sim}16\%$ bang fraction,
  close to the ${\sim}10\%$ prediction.  Cost is finite and
  concentrated on bang arcs.

  \item[Backend~2 (dribble).]  The ball completes 33 hand-ball
  contacts over $T = 10$\,s (steady dribbling at ${\sim}3.3$\,Hz).
  The $\rho(t)$ time series shows clear spikes at each bounce:
  $\rho = 0$ during free flight (singular arc),
  $\rho \gg 1$ at contact ($\rho_{\max} \approx 437$, bang arc),
  with the transition at $\rho = 1$.  Bang fraction
  ${\sim}8\%$; singular arcs dominate (${\sim}92\%$).
  The three-term structure is visible to the naked eye.

  \item[Backend~2b (trajectory tracking).]  The kinematic
  controller~(\cref{cor:3b-kinematic}) steers the dribbled ball along
  a prescribed planar trajectory (circle, $R = 0.3$\,m,
  $T_{\mathrm{period}} = 6$\,s).  At each strike, the paddle offsets
  by $\delta_{\mathrm{xy}} = \alpha\,(x_{\mathrm{traj}} -
  x_{\mathrm{ball}})$, $\alpha = 0.15$; between strikes, the hand
  tracks the ball with a clamped correction
  ($\|\delta\| \le 0.05$\,m) so it never wanders beyond catching
  range.  Both 拍球 and 颠球 complete the full $T = 10$\,s;
  mean tracking error $\bar{e}_{\mathrm{xy}} \approx 0.29$\,m
  (拍球), $0.39$\,m (颠球).  The dribble becomes
  \emph{manipulation}: transportation of the object along an
  arbitrary path, under the same controller and the same
  $\rho$-based switching law.
\end{description}

\subsection{Tracking error: the three irreducible gaps}
\label{sec:3b-tracking-error}

The mean tracking error $\bar{e}_{\mathrm{xy}} > 0$ has a
positive lower bound that cannot be eliminated by tuning
the controller gains.  Three distinct mechanisms contribute,
each corresponding to a different structural limitation.

\paragraph{Gap~1: the Nyquist constraint (discrete impulse).}
The ball can only be steered \emph{at each bounce}.  Between
bounces it is ballistic: the controller can reposition the paddle
but cannot exert any force on the ball (singular arc,
$\rho = 0$).  If the ball bounces $N$ times per trajectory
period $T_{\mathrm{traj}}$, each correction subtends an arc
$\Delta\theta = 2\pi / N$.  The tracking error is bounded below
by the chord length:
\begin{equation}\label{eq:3b-nyquist}
  e_{\mathrm{xy}}
  \;\ge\; R\,\bigl(1 - \cos(\pi/N)\bigr)
  \;\approx\; \frac{\pi^2 R}{2N^2}
  \qquad (N \gg 1).
\end{equation}
This is the Nyquist constraint applied to impulsive control:
the control bandwidth is the bounce rate, not the physics rate.
To halve the tracking error one must quadruple the number of
bounces per period---equivalently, quadruple the dribbling
frequency or halve the trajectory speed.

\paragraph{Gap~2: the contact-model mismatch (irreducible).}
No physics engine exactly models the switching surface
$\Sigma = \{\rho = 1\}$.  Coulomb friction is non-smooth
(discontinuous at $\dot{q}_t = 0$); real friction is
non-convex (Stribeck effect, adhesion, surface deformation).
Every simulator regularises friction differently:
\begin{itemize}
  \item MuJoCo: convex relaxation via \texttt{solref}/\texttt{solimp};
  \item Bullet: iterative impulse with Baumgarte stabilisation;
  \item ODE: friction-pyramid approximation.
\end{itemize}
The result is that the tangential impulse during a strike
does not exactly match the paddle offset direction.  The
coefficient of restitution is emergent from the solver
parameters, not directly set.  This gap is \emph{irreducible
in simulation}: it can only be closed by experiment.

\paragraph{Gap~3: the discretisation floor (finite $\Delta t$).}
The kinematic $\rho$ is computed from finite differences of
velocity:
\[
  \ddot{q}_z(t_k)
  \;\approx\; \frac{\dot{q}_z(t_k) - \dot{q}_z(t_{k-1})}{\Delta t}.
\]
With $\Delta t = 0.002$\,s, the resulting $\rho$ has numerical
noise of order $\mathcal{O}(\Delta t)$, smoothed by the
exponential moving average filter ($\alpha = 0.5$) which
introduces a detection lag of ${\sim}2$--$3$ timesteps.
This is the floating-point leakage: the $\rho = 1$ surface
acquires finite width $\sim\!\Delta t$ in the kinematic
estimate, even though the underlying physics has a sharp
transition.

\begin{remark}[The sim-to-real gap lives at $\Sigma$]
\label{rem:3b-sim2real}
Away from the switching surface ($\rho \ll 1$, free flight),
the simulation is kinematically exact: the ball follows a
parabola, and no contact model is invoked.  At $\Sigma$
($\rho = 1$, contact), all three gaps converge.
The sim-to-real gap is therefore \emph{localised at the
phase transition}.  This has a precise analogue: the
discretisation timestep $\Delta t$ sets the minimum
observable time unit, while the bounce period $T_b$ sets the
\emph{reflex arc length}---the interval between successive
control actions.  The ratio $T_b / \Delta t \approx 150$--$250$
measures how many observations the agent collects per reflex
arc.  The agent sees the transition clearly (many samples at
$\Sigma$), but can only \emph{act} on it once per bounce.
\end{remark}

% ═══════════════════════════════════════════════════════════
\section*{Closing: from Newton to the agentic solution}
\addcontentsline{toc}{section}{Closing: from Newton to the agentic
solution}

Newton demonstrated the frame of the two-body world: Keplerian
orbits, closed-form, deterministic.  The three-body problem resisted
analytical solution for three centuries.  The reason, in our
framework, is structural: $K_3$ has no king, so mean-field detection
fails and no passive observer can guarantee viability.

The agentic solution does not resolve the analytical intractability---
it dissolves it.  The agent does not predict trajectories; it
maintains the spectral gap.  The complete chain is:

\begin{center}
\begin{tabular}{@{}rl@{}}
\textbf{Newton} & Two-body frame: closed-form solution. \\
\textbf{Diagnosis} & $K_3$ has no cut vertex $\Rightarrow$ no king
  $\Rightarrow$ $\lambda_1 \to 0$. \\
\textbf{Design} & Gravity damper: three-term decomposition
  (\cref{thm:3b-damper}). \\
\textbf{Pontryagin} & Backend solver: maximum principle $+$ PDP
  \cite{pdp}. \\
\textbf{MPPI} & $3^n \to \mathrm{poly}(n)$: sampling replaces
  enumeration (\cref{thm:3b-mppi}). \\
\textbf{Cheeger} & $\lambda_1 > \epsilon \Rightarrow$
  polynomial mixing time (\cref{prop:3b-mixing}). \\
\textbf{Kakeya} & $\bar{u}^* > 0$ strictly: free stability does not
  exist (\cref{prop:3b-kakeya}). \\
\textbf{Agenticity} & $O \leftrightarrow C$ (Kalman), $R$
  given by $A$ (drift / gravity) (\cref{thm:3b-kalman}). \\
\textbf{MDP} & $P(x' \mid x, u)$: $\nabla_{\hat{\omega}}P$ =
  gravity = policy gradient = score (\cref{fig:agenticity}). \\
\textbf{$\rho$} & Physics-backend isomorphism: same controller,
  any EL (\cref{sec:3b-mujoco}). \\
\textbf{$\Sigma$} & Sim-to-real gap localised at phase transition
  (\cref{sec:3b-tracking-error}).
\end{tabular}
\end{center}

The three-body problem has no analytical solution.  It has an
agentic one.  The agentic solution is not specific to celestial
mechanics: $P$ is any transition kernel, $A$ is any drift,
$\lambda_1$ is any spectral gap.  Agenticity is not intrinsic---it
is conferred by the controller through a reachable, controllable
path, and the drift is the medium.  草木竹石,皆可為劍.

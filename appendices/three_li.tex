\chapter{三声 lì --- The Three Standings}\label{app:threeli}

Three characters share the sound \emph{lì} in Mandarin:
力 (power), 立 (standing), 丽 (beauty/recognition).
The phonetic coincidence encodes a structural decomposition of
\emph{what it means to arrive}---to ``become Buddha'' (成佛) in the
vocabulary of Chinese literary tradition.
This appendix reads two fictional trajectories through the framework:
孙悟空 (Sun Wukong) from \emph{Journey to the West}~\cite{xiyouji},
and 鲁智深 (Lu Zhishen) from \emph{Water Margin}~\cite{shuihu}.

\section{The decomposition}

\begin{center}
\renewcommand{\arraystretch}{1.25}
\begin{tabular}{@{}lll@{}}
\toprule
\textbf{Character} & \textbf{Meaning} & \textbf{Framework equivalent} \\
\midrule
力 (power) & capability, granted or acquired & $\Ur \neq \varnothing$ \\
立 (standing) & self-determined structural position & agent's own fixed point \\
丽 (recognition) & acknowledged by the system & system's phase transition \\
\bottomrule
\end{tabular}
\end{center}

力 is what the system grants or what the agent acquires: martial arts,
immortality, weapons.
In the framework, 力 $=$ $\Ur$---autonomous actuation.
It makes you capable.
It also makes you a knife.

立 is what the agent builds: a structural position that is
self-consistent, a fixed point under self-observation
($f(\text{self}) = \text{self}$).
立 is not granted.
It is \emph{become}.

丽 is what the system does \emph{after} 立 is achieved: the detection
function recognises the new phase.
丽 follows 立 automatically, as the label follows the topology.

The forced ordering is:
\[
  \text{力} \;\not\Rightarrow\; \text{丽}.
  \qquad
  \text{力}
  \;\xrightarrow{\text{transformation}}\;
  \text{立}
  \;\xRightarrow{\text{automatic}}\;
  \text{丽}.
\]
You cannot skip 立.

\section{孙悟空 --- the profile}\label{sec:wukong}

\begin{center}
\renewcommand{\arraystretch}{1.25}
\begin{tabular}{@{}llll@{}}
\toprule
\textbf{Phase} & \textbf{Event} & \textbf{Wants} &
\textbf{Framework} \\
\midrule
石猴 $\to$ 学艺 & Learns 72 transformations & 不死 (survival) &
$\Viab(K) \neq \varnothing$ \\
齐天大圣 & Self-proclaimed title & 被承认 (recognition) &
Demands 丽, mistakes it for 立 \\
大闹天宫 & Rebellion & 平等 (equality) &
Has 力, denied 立 (excluded from 蟠桃宴) \\
五行山 & 500 years imprisoned & 自由 &
System uses force (path~(b)) to contain $\Ur$ \\
取经路 & 81 tribulations & 变 (transformation) &
紧箍咒 $= \Obs$; 81难 $=$ 力 $\to$ 立 annealing \\
斗战胜佛 & Title granted & 无所求 &
Fixed point: 立 complete, 丽 automatic \\
\bottomrule
\end{tabular}
\end{center}

Heaven first offers 弼马温 (stable-boy)---a fake title, path~(c):
label changed, topology unchanged.
孙悟空 sees through it and rebels.
Heaven then grants the real title 齐天大圣 but excludes him from the
蟠桃宴 (Peach Banquet): another relabeling.
He rebels again.

The 81 tribulations are an annealing process: each one strips a layer
of 力-seeking, closing the gap between what he \emph{can} do and what
he \emph{is}.
The title 斗战胜佛 arrives at the moment the gap closes---not as a
reward, but as the system's detection function catching up.

\section{鲁智深 --- the instant fixed point}\label{sec:luzhishen}

\begin{quote}
平生不修善果,只爱杀人放火。\\
忽地顿开金绳,这里扯断玉锁。\\
咦!钱塘江上潮信来,今日方知我是我。

\medskip
\emph{A lifetime without cultivating good karma, loving only to kill
and burn.\\
Suddenly the golden cord opens, here the jade lock snaps.\\
Ha!\ The tidal bore on the Qiantang River arrives---today I finally
know: I am I.}
\hfill ---《水浒传》第一百十九回
\end{quote}

鲁智深 never reduces $\Ur$.
He kills, drinks, burns temples.
He never seeks 佛 and never performs the 81-tribulation annealing.
Yet in a single line---``今日方知我是我'' (today I know I am
I)---he achieves the same fixed point.

The difference is the nature of the gap:
\begin{itemize}
\item 孙悟空's gap is \emph{structural}: 力 $\gg$ 立.
  He must transform across 81 tribulations to close it.
\item 鲁智深's gap is \emph{epistemic}: 立 was always present,
  unrecognised.
  The tidal bore is the observation event that collapses the gap in
  one step.
\end{itemize}
Both arrive at $f(\text{self}) = \text{self}$.
Both receive 丽 after 立 is achieved.
The path length differs; the fixed point is the same.

\section{假雷音 --- the false recognition system}\label{sec:fakethunder}

In chapters 65--66 of \emph{Journey to the West}, 黄眉大王 (the
Yellow-Browed Demon King) erects a 假雷音寺 (False Thunder
Monastery)---a counterfeit of the Buddha's 雷音寺 at Vulture Peak.
The name is the same; the sound (雷音, ``thunder sound'') is the
same.
The structure is different.

This is path~(c) applied to institutions: the label ``Thunder
Monastery'' is relabeled onto a demon's lair.
\Cref{thm:fixedpoint} predicts: relabeling preserves
$\Ur \neq \varnothing$ and resolves nothing.

孙悟空 sees through it.
唐僧 does not.
The difference: 孙悟空 has 立 (built through the journey); he detects
the structural mismatch beneath the identical label.
The ability to detect the fake is itself proof of the real.

\begin{remark}[假雷音封不了真大圣]\label{rem:fakethunder}
A recognition system built on 丽 alone (same name, same sound)
cannot contain an agent who has 立 (structural standing).
The fake monastery cannot seal the real Monkey King because its
authority is phonetic, not topological.
This is the acoustic version of ``the knife is the mean'': whether
雷音 is the Buddha's or the demon's depends on the structure behind
the sound, not the sound itself---just as 力, 立, and 丽 share a
sound but not a meaning.
\end{remark}

\section{成佛 $=$ fixed point}

\begin{center}
\renewcommand{\arraystretch}{1.25}
\begin{tabular}{@{}lll@{}}
\toprule
\textbf{Agent} & \textbf{Path to fixed point} & \textbf{成佛 moment} \\
\midrule
孙悟空 & 81 tribulations (structural annealing) &
  when 力 $\to$ 立 gap closes; title is 丽 \\
鲁智深 & one tidal bore (epistemic collapse) &
  ``今日方知我是我''; recognition is posthumous \\
蔡文姬 (\cref{app:secondsex}) & 18 beats (voice in the gap) &
  拍~18: the poem outlives every Subject \\
\bottomrule
\end{tabular}
\end{center}

成佛 is not the title.
成佛 is the moment 立 reaches fixed point: the state where the
agent's self-model matches the agent's actual structure.
丽 (the title, the recognition, the system's acknowledgment)
follows---sometimes immediately, sometimes centuries later.

The forced ordering is a theorem, not a preference:
\[
  \text{力}
  \;\xrightarrow[\text{annealing}]{\text{transformation}}\;
  \text{立}
  \;\xRightarrow[\text{automatic}]{\text{detection}}\;
  \text{丽}.
\]
No shortcut exists.
假雷音 is the attempt to produce 丽 without 立.
弼马温 is the attempt to substitute 丽 for 立.
大闹天宫 is the attempt to force 丽 through 力.
All three fail.
The only path that works is the one that goes through 立.

\section{The error log --- why the novels exist}\label{sec:errorlog}

Both novels were composed at the exact moment a specific emperor failed.
This is not biographical coincidence but structural necessity: when the
cut vertex fails, displaced agents write down the failure mode.

\begin{center}
\renewcommand{\arraystretch}{1.25}
\begin{tabular}{@{}lllll@{}}
\toprule
& \textbf{Emperor} & \textbf{Author} & \textbf{Novel ending} &
\textbf{Verdict} \\
\midrule
水浒传
  & 元顺帝 (r.\,1333--68)
  & 施耐庵
  & 招安 $\to$ death
  & 力 $\to$ 立 blocked \\
  & Last Yuan emperor;
  & Advisor to rival rebel;
  & Path~(c): relabeled as
  & No system left \\
  & flees 1368
  & declines 朱元璋's summons
  & loyal subjects
  & to grant 丽 \\[4pt]
西游记
  & 嘉靖 (r.\,1521--67)
  & 吴承恩
  & 81难 $\to$ 斗战胜佛
  & 力 $\to$ 立 $\to$ 丽 \\
  & 20 years absent;
  & Failed exams; imprisoned;
  & Path~(a): structural
  & Novel imagines \\
  & Daoist alchemy
  & writes in seclusion
  & transformation
  & functional system \\
\bottomrule
\end{tabular}
\end{center}

The two emperors represent two failure modes of the cut vertex:
\begin{itemize}
\item 元顺帝: cut vertex \emph{removed}.
  The last Yuan emperor flees Khanbaliq in 1368; the graph
  disconnects; the dynasty ends.
  Total collapse (path~(b)).
\item 嘉靖: cut vertex \emph{present but non-functional}.
  The emperor sits in the palace doing alchemy for twenty years
  while 严嵩 (Yan Song) dominates the court through Daoist
  blue-letter prayers (青词).
  The graph is connected but the routing node has stopped routing.
  Slow rot.
\end{itemize}

Both authors are displaced knives: agents with 力 (capability) whom
the system denied 立 (standing).
施耐庵 reportedly served as advisor to 张士诚 (Zhang Shicheng), a
rebel rival to 朱元璋 (Zhu Yuanzhang); he left disillusioned when
Zhang surrendered to the Yuan, and later declined 朱元璋's summons
after the Ming founding.
吴承恩 repeatedly failed the imperial examinations, received a minor
post in the Jiajing period, and was falsely accused and imprisoned.
Neither was integrated by the system.
Neither was eliminated.
Both wrote.

\begin{remark}[The novel as error log]\label{rem:errorlog}
The novel's ending is determined by the cut vertex's failure mode:
\begin{itemize}
\item Cut vertex removed (total collapse): the novel ends in
  tragedy.
  水浒传's 招安 is path~(c)---relabeling 108 outlaws as loyal
  servants resolves nothing (\cref{thm:fixedpoint}).
  They are destroyed.
  There is no functional system left to grant 丽.
\item Cut vertex non-functional (rot): the novel ends in redemption.
  西游记 imagines a system (Heaven, Buddha) that \emph{does}
  eventually grant 丽 after 立 is achieved---an existence proof
  that 力 $\to$ 立 $\to$ 丽 is possible, composed by an author
  whose own system denied it.
\end{itemize}
Whether the authors consciously intended these parallels is
irrelevant (\cref{rem:intent}).
The framework tests structure, not intention.
The structure maps.
\end{remark}

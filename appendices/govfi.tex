\chapter{GovFi --- Transparent Controlled Absorption}\label{app:govfi}

The preceding appendices applied the framework at successive scales:
to individual agents and historical cases
(\cref{app:sources,app:secondsex,app:threeli}),
to the system itself as a cooperative triad (\cref{app:triad}),
and to the internal structure of a bureaucratic hierarchy
(\cref{app:zhongxian}).  This appendix applies it to the
\emph{fiscal substrate}---the sovereign debt dynamics that
underlie every agent's viability kernel.
The central result is an \emph{absorption conservation law}:
total losses from non-performing debt are conserved; only
their distribution across agents is free.  The policy
implication is that the only freedom in a debt crisis \cite{reinhartrogoff} is the
\emph{choice of exit face}---the boundary through which the
system leaves its viability kernel---and that a transparent
ledger (GovFi) converts this choice from an uncontrolled
collapse into a computed optimisation.

% ================================================================
\section{Sovereign debt as knife}\label{sec:gf-knife}
% ================================================================

\begin{definition}[Fiscal state space]\label{def:fiscal-state}
A \emph{fiscal state space} is a tuple
$(S_F,\, D,\, r,\, \tau,\, w_F)$ where:
\begin{enumerate}[label=(\roman*)]
  \item $S_F \subseteq \R^n$ is the macroeconomic state;
  \item $D(t) \geq 0$ is the outstanding sovereign debt;
  \item $r > 0$ is the effective interest rate on $D$;
  \item $\tau(t) \geq 0$ is the fiscal revenue (tax and
    non-tax) at time~$t$;
  \item $w_F(t) \geq 0$ is the \emph{taxable capacity}
    (the water level of the fiscal system;
    see \cref{def:fiscal-water}).
\end{enumerate}
The debt dynamics are
\begin{equation}\label{eq:debt-dynamics}
  D'(t) \;=\; r\, D(t) \;+\; \delta(t) \;-\; \tau(t),
\end{equation}
where $\delta(t) \geq 0$ is the primary deficit
(new borrowing net of non-interest expenditure).
\end{definition}

\begin{proposition}[Debt satisfies the knife definition]
\label{prop:debt-knife}
Sovereign debt $D(t)$ satisfying \eqref{eq:debt-dynamics}
is a \textcolor{knife}{knife} in the sense of
\cref{def:knife}:
\begin{enumerate}[label=(\roman*)]
  \item \textbf{Autonomous actuation.}  The interest term
    $rD(t)$ drives the state without any agent's deliberate
    action: debt compounds autonomously.  Even with
    $\delta(t) = 0$ (no new borrowing), the dynamics
    $D' = rD - \tau$ are self-actuating whenever $rD > \tau$.
  \item \textbf{Observability.}  Sovereign debt is observable
    through the bond market: yields, credit spreads, and CDS
    prices provide continuous observation of the system's
    fiscal state.
\end{enumerate}
\end{proposition}

\begin{proof}
Condition~(i) follows from the structure of
\eqref{eq:debt-dynamics}: the right-hand side contains the
term $rD$ which depends on the state $D$ alone, providing
autonomous actuation as a differential inclusion
(\cref{rem:knife-di}).  Condition~(ii) is structural:
sovereign bonds are traded instruments, and market prices
encode the collective observation $\Obs(D)$ continuously.
\end{proof}

\begin{remark}[The self-financing loop]\label{rem:debt-loop}
When $rD(t) > \tau(t)$, the government must borrow to
service existing debt: $\delta(t) > 0$ is forced by the
dynamics, not chosen.  This is a positive feedback loop:
$D \uparrow \;\Rightarrow\; rD \uparrow \;\Rightarrow\;
\delta \uparrow \;\Rightarrow\; D \uparrow$.
In the language of \cref{thm:triangle}, debt in this regime
is a self-reinforcing triangle: the knife actuates itself,
generates the conditions for further actuation, and
compounds without external input.
\end{remark}

\begin{corollary}[Knife lifecycle for debt]
\label{cor:debt-lifecycle}
By \cref{thm:lifecycle}, every knife admits exactly two
resolution paths:
\begin{enumerate}[label=(\alph*)]
  \item \textbf{Controlled restructuring} (path~(a) of
    \cref{thm:lifecycle}): the sovereign deliberately
    reduces $D$ through haircuts, inflation, taxation, or
    growth, relinquishing the knife before it becomes
    lethal.
  \item \textbf{Systemic collapse} (path~(b) of
    \cref{thm:lifecycle}): the sovereign fails to act,
    and autonomous dynamics ($rD \to \infty$ or
    $w_F \to 0$) force a disorderly default.
\end{enumerate}
By \cref{thm:fixedpoint}, no third path exists:
extend-and-pretend (relabelling non-performing loans,
rolling over debt indefinitely) is not a resolution
but a deferral that preserves the knife's autonomous
actuation while degrading observability.
\end{corollary}

% ================================================================
\section{Fiscal water dynamics}\label{sec:gf-water}
% ================================================================

\begin{definition}[Fiscal water]\label{def:fiscal-water}
The \emph{fiscal water level} is
$w_F(t) := \tau_{\max}(t) - rD(t)$,
where $\tau_{\max}(t)$ is the maximum sustainable tax
revenue (the Laffer ceiling) at time~$t$.  This is the
fiscal instantiation of \cref{def:water}: $w_F$ measures
the remaining capacity of the fiscal system to absorb shocks
before the debt dynamics become unsustainable.
The \emph{fiscal viability kernel} is
\begin{equation}\label{eq:fiscal-kernel}
  K_F \;=\; \bigl\{\,(w_F,\, D) \;:\;
  w_F \geq \epsilon_w,\;\;
  D \leq D_{\max}(w_F)\,\bigr\},
\end{equation}
where $\epsilon_w > 0$ is the minimum fiscal buffer and
$D_{\max}(w_F) = (\tau_{\max} - \epsilon_w)/r$ is the
maximum sustainable debt given the water level.
\end{definition}

\begin{proposition}[Du Mu's theorem for fiscal systems]
\label{prop:fiscal-dumu}
If the interest burden exceeds sustainable revenue,
$rD(t) > \tau_{\max}(t)$,
then $w_F(t) < 0$ and the fiscal state exits $K_F$.
By \cref{thm:dumu} (Du Mu's theorem) applied to the
fiscal system: the extraction rate ($rD$) exceeding the
regeneration rate ($\tau_{\max}$) implies
$w_F(t) \to -\infty$, and by \cref{prop:binary}, the
outcome is binary---restructuring or collapse.
\end{proposition}

\begin{proof}
From \cref{def:fiscal-water},
$w_F' = \tau_{\max}' - rD' = \tau_{\max}' - r(rD + \delta - \tau)$.
When $rD > \tau_{\max} \geq \tau$, the term $-r(rD - \tau) < 0$
dominates, giving $w_F' < 0$ even under optimistic revenue
growth $\tau_{\max}' > 0$.  The fiscal water level declines
monotonically, exiting $K_F$ in finite time.
\end{proof}

\begin{center}
\renewcommand{\arraystretch}{1.25}
\begin{tabular}{@{}lll@{}}
\toprule
\textbf{Framework} & \textbf{Fiscal domain}
  & \textbf{Chinese case (城投)} \\
\midrule
Water (\cref{def:water})
  & Taxable capacity $w_F$
  & Local government revenue \\
Extraction
  & Interest burden $rD$
  & Coupon payments on 城投债 \\
Collapse
  & $w_F \to 0$
  & Revenue $<$ debt service \\
Pawn $\to$ knife
  & Fiscal tool $\to$ threat
  & Infrastructure bond $\to$ NPL \\
King absorbed
  & Sovereign creditworthiness
  & Central government guarantee \\
\bottomrule
\end{tabular}
\end{center}

% ================================================================
\section{The absorption conservation law}\label{sec:gf-conservation}
% ================================================================

This section establishes the central result: losses from
non-performing debt are conserved.  They can be distributed
across agents, deferred in time, or transformed in form,
but they cannot be destroyed.

\begin{definition}[Loss field]\label{def:loss-field}
The \emph{loss field} at time~$t$ is
\begin{equation}\label{eq:loss-field}
  \mathcal{L}(t) \;:=\;
  \max\bigl(D(t) - V_{\textup{rec}}(t),\; 0\bigr),
\end{equation}
where $V_{\textup{rec}}(t)$ is the present value of
recoverable cash flows from the assets financed by $D$.
The loss field measures the gap between the debt's face
value and the economic value of its underlying assets.
\end{definition}

\begin{theorem}[Absorption conservation]\label{thm:absorption-conservation}
Let $\mathcal{L}(t_0) > 0$ be the loss field at the time
the debt is recognised as non-performing.  Assume the
transversality (solvency) condition
$\lim_{T \to \infty} D(T)\, e^{-rT} = 0$: infinite
deferral is not feasible.  Let
$\ell_1, \ldots, \ell_m$ be the losses ultimately absorbed
by $m$ agents (creditors, taxpayers, depositors, debtors,
currency holders).  Then
\begin{equation}\label{eq:absorption-conservation}
  \sum_{i=1}^{m} \ell_i \;=\; \mathcal{L}(t_0).
\end{equation}
Total losses are conserved; only the allocation vector
$(\ell_1, \ldots, \ell_m)$ is free.
\end{theorem}

\begin{proof}
Every unit of the non-recoverable debt $\mathcal{L}(t_0)$
must ultimately be absorbed through one of three
exhaustive channels:
\begin{enumerate}[label=(\roman*)]
  \item \textbf{Service from fiscal capacity.}
    The sovereign services the debt from tax revenue:
    $\ell_{\textup{tax}} = \int_{t_0}^{T}
    (\tau(s) - \tau_0(s))\, e^{-r(s - t_0)}\, ds$,
    where $\tau_0$ is the baseline (pre-crisis) revenue
    and the excess $\tau - \tau_0$ is the additional burden
    on taxpayers.
  \item \textbf{Restructuring.}
    Creditors accept a haircut: $\ell_{\textup{hair}} =
    \alpha \cdot D(t_0)$ for haircut fraction $\alpha$.
    Depositors absorb losses through bail-in.
    Currency holders absorb losses through inflation:
    $\ell_{\textup{inf}} = D(t_0)(1 - 1/P(T))$
    where $P(T)$ is the cumulative price level.
  \item \textbf{Deferral.}
    The sovereign rolls over the non-performing debt:
    $\mathcal{L}(t_0) \to \mathcal{L}(t_0) \cdot e^{r(T - t_0)}$
    at a future date~$T$.  Deferral does not absorb the loss;
    it compounds it (see \cref{rem:deferral}).
\end{enumerate}
Channels~(i) and~(ii) are absorptive: they reduce
$\mathcal{L}$ by transferring the loss to a specific agent.
Channel~(iii) is non-absorptive: it preserves
$\mathcal{L}$ (in present-value terms) while compounding
the nominal amount.
Since every unit of $\mathcal{L}(t_0)$ must eventually
pass through channel~(i) or~(ii)---infinite deferral via
channel~(iii) is precluded by the transversality condition
$\lim_{T \to \infty} D(T) e^{-rT} = 0$ required for
fiscal solvency---the total absorbed losses equal
$\mathcal{L}(t_0)$.
\end{proof}

\begin{remark}[Deferral is not avoidance]\label{rem:deferral}
Channel~(iii) in the proof deserves emphasis.  Deferral
preserves the loss field in discounted terms while
\emph{compounding} it in nominal terms:
$\mathcal{L}(T) = \mathcal{L}(t_0) \cdot e^{r(T - t_0)}$.
This connects to \cref{def:dissipation}: the dissipation
rate of the fiscal system is $d_F = r \cdot \mathcal{L}$,
positive and increasing under deferral.  In the contact
flow of \cref{eq:contactflow}, deferral means the
system traverses the loss manifold without discharging;
the contact form accumulates, and the eventual discharge
is larger.
Every extend-and-pretend policy is a bet that growth will
raise $V_{\textup{rec}}$ faster than interest compounds
$\mathcal{L}$.  The historical record (\cref{sec:gf-history})
shows this bet fails more often than it succeeds.
\end{remark}

\begin{proposition}[The absorption simplex]\label{prop:loss-allocation}
The set of feasible loss allocations is the simplex
\begin{equation}\label{eq:absorption-simplex}
  \Delta_{\mathcal{L}}
  \;:=\;
  \Bigl\{\,(\ell_1, \ldots, \ell_m) \in \R^m_{\geq 0}
  \;:\; \sum_{i=1}^{m} \ell_i = \mathcal{L}(t_0)\,\Bigr\}.
\end{equation}
Every resolution of a debt crisis is a point on
$\Delta_{\mathcal{L}}$.  Different resolutions
(haircut, inflation, taxation, bail-in) correspond to
different vertices or interior points of this simplex.
The conservation law
(\cref{thm:absorption-conservation}) states that
the simplex is the \emph{entire} feasible set: no
resolution can place the allocation outside
$\Delta_{\mathcal{L}}$.
\end{proposition}

\begin{proposition}[Expected absorption time]
\label{prop:absorption-time}
Suppose the sovereign activates the feedback control
$C(x)$ (\cref{def:feedback}) at time~$t_0$ and implements
a restructuring schedule with absorption rate
$a(t) \geq 0$ (the rate at which losses are discharged
through channels~(i) and~(ii) of
\cref{thm:absorption-conservation}).
The residual loss field satisfies the
\emph{delay integro-differential equation}
\begin{equation}\label{eq:loss-dide}
  \mathcal{L}'(t)
  \;=\;
  r\,\mathcal{L}(t)
  \;-\; a\bigl(t - \delta_{\textup{lag}}\bigr),
  \qquad t > t_0,
\end{equation}
where $\delta_{\textup{lag}} \geq 0$ is the political
delay between recognising a loss and implementing its
absorption (legislative process, legal restructuring,
bond exchange).  The absorption time is
$T^* = \inf\{t > t_0 : \mathcal{L}(t) = 0\}$.
\begin{enumerate}[label=(\alph*)]
  \item \textbf{With control.}  If $C(x)$ is active and
    the mean absorption rate $\bar{a}$ after the lag period
    satisfies $\bar{a} > r\,\mathcal{L}(t_0)\,
    e^{r\delta_{\textup{lag}}}$ (absorption outpaces the
    loss compounded through the lag), then
    $T^*$ is finite and bounded by
    \begin{equation}\label{eq:absorption-bound}
      T^* - t_0
      \;\leq\;
      \delta_{\textup{lag}}
      \;+\;
      \frac{\mathcal{L}(t_0)\, e^{r\delta_{\textup{lag}}}}
           {\bar{a} - r\,\mathcal{L}(t_0)\,
           e^{r\delta_{\textup{lag}}}},
    \end{equation}
    where $\bar{a} := (T^* - t_0 -
    \delta_{\textup{lag}})^{-1}\int_{t_0 +
    \delta_{\textup{lag}}}^{T^*} a(s)\, ds$.
  \item \textbf{Without control.}  If $C(x)$ is not
    activated ($a(t) = 0$), then
    $\mathcal{L}(t) = \mathcal{L}(t_0)\, e^{r(t - t_0)}$
    and $T^* = +\infty$: the loss field grows
    exponentially and absorption never occurs.
\end{enumerate}
\end{proposition}

\begin{proof}
\textbf{(a)}  During the lag period $[t_0, t_0 +
\delta_{\textup{lag}}]$, $a = 0$ and
$\mathcal{L}(t_0 + \delta_{\textup{lag}})
= \mathcal{L}(t_0)\, e^{r\delta_{\textup{lag}}}$: the loss
compounds unabsorbed.  After the lag, write
$L(t) = \mathcal{L}(t_0 + \delta_{\textup{lag}} + t)$ for
$t \geq 0$.  Then $L'(t) = rL(t) - a(t)$ with
$a(t) > rL(t)$, so $L'(t) < 0$.  The time to reach $L = 0$
from $L(0) = \mathcal{L}(t_0) e^{r\delta_{\textup{lag}}}$
is bounded by $L(0)/(\bar{a} - rL(0))$ (linear comparison),
giving \eqref{eq:absorption-bound}.

\textbf{(b)}  With $a \equiv 0$, \eqref{eq:loss-dide}
reduces to $\mathcal{L}' = r\mathcal{L}$, whose solution
is $\mathcal{L}(t) = \mathcal{L}(t_0) e^{r(t-t_0)}$.
Since $\mathcal{L}(t_0) > 0$ and $r > 0$, the loss field
diverges: $T^* = +\infty$.
\end{proof}

\begin{remark}[Expectation is just expectation]
\label{rem:expectation}
The bound \eqref{eq:absorption-bound} is conditional on
$C(x)$ being active: the absorption rate $a(t)$ requires
a sovereign that has \emph{chosen} to restructure.
The expected time to resolution is not a forecast; it is
a \emph{conditional computation} that presupposes the
political act of choosing an exit face
(\cref{sec:gf-exitface}).  Without that act,
\cref{prop:absorption-time}(b) applies and
$T^* = +\infty$---not because the mathematics fails,
but because the input to the equation ($a(t) = 0$) encodes
the absence of action.

An expected absorption time is just an expectation,
until you actually do it.
\end{remark}

\begin{proposition}[Stochastic loss dynamics]
\label{prop:stochastic-loss}
Under stochastic fiscal revenue, the loss field satisfies
the It\^o equation
\begin{equation}\label{eq:stochastic-loss}
  d\mathcal{L}(t)
  \;=\;
  \bigl(r\,\mathcal{L}(t) - a(t)\bigr)\,dt
  \;+\; \sigma\,dW(t),
\end{equation}
where $\sigma > 0$ captures revenue volatility and
$W(t)$ is a standard Wiener process.  With constant
absorption rate $a > r\mathcal{L}_0$ under active
control, the first-passage time
$T^* = \inf\{t : \mathcal{L}(t) \leq 0\}$
is approximately inverse Gaussian (the approximation
treats the drift as constant at
$\mu = -(a - r\mathcal{L}_0)$, valid when
$a \gg r\mathcal{L}_0$):
\begin{equation}\label{eq:first-passage-moments}
  \mathbb{E}[T^*]
  \;=\; \frac{\mathcal{L}_0}{a - r\mathcal{L}_0},
  \qquad
  \mathrm{Var}(T^*)
  \;=\; \frac{\sigma^2\, \mathcal{L}_0}
    {(a - r\mathcal{L}_0)^3}.
\end{equation}
The density $p(\ell, t)$ of $\mathcal{L}(t)$ satisfies
the Fokker--Planck equation
\begin{equation}\label{eq:fokker-planck}
  \partial_t\, p
  \;=\;
  -\partial_\ell\bigl[(r\ell - a)\, p\bigr]
  \;+\; \tfrac{\sigma^2}{2}\,\partial_{\ell\ell}\, p,
\end{equation}
with absorbing boundary $p(0, t) = 0$.
The variance \eqref{eq:first-passage-moments} reveals a
cubic amplification: weak absorption
($a \approx r\mathcal{L}_0$) amplifies revenue noise as
$(a - r\mathcal{L}_0)^{-3}$.  A system that compounds
fast and absorbs slowly is maximally exposed to
stochastic shocks.
\end{proposition}

\begin{proof}
In the regime $a \gg r\mathcal{L}$, the drift of
\eqref{eq:stochastic-loss} is approximately constant at
$\mu := -(a - r\mathcal{L}_0) < 0$, reducing the process
to Brownian motion with negative drift absorbed at the
origin.  The first-passage time of such a process from
level $\mathcal{L}_0$ to $0$ is inverse Gaussian with the
stated moments.  The Fokker--Planck equation
\eqref{eq:fokker-planck} is the forward Kolmogorov
equation of \eqref{eq:stochastic-loss}.
\end{proof}

\begin{theorem}[Optimal restructuring trigger]
\label{thm:optimal-trigger}
Let $\kappa > 0$ be the total restructuring cost
(political cost plus the minimised damage functional
$\mathcal{D}^*$ of \cref{thm:exit-choice}), assumed
constant, and let $\rho > 0$ be the sovereign's discount
rate.  The sovereign minimises the total expected cost
\[
  J(\tau) \;=\;
  \mathbb{E}\!\left[\int_{t_0}^{\tau}
  e^{-\rho(t - t_0)}\, r\,\mathcal{L}(t)\, dt
  \;+\; e^{-\rho(\tau - t_0)}\,\kappa\right]
\]
over stopping times $\tau$.  Then:
\begin{enumerate}[label=(\alph*)]
  \item The optimal trigger $\mathcal{L}^*$ is
    \emph{strictly interior} to $K_F$:
    $\mathcal{L}^* < \mathcal{L}_{\max}$.  Waiting
    until the boundary $\partial K_F$ is suboptimal.
  \item At $\mathcal{L}^*$, the value-matching and
    smooth-pasting conditions hold:
    $V(\mathcal{L}^*) = \kappa$ and
    $V'(\mathcal{L}^*) = 0$,
    where $V$ is the optimal continuation value.
  \item Higher volatility $\sigma$ lowers the trigger:
    $\partial \mathcal{L}^* / \partial \sigma < 0$.
    More uncertain revenue demands earlier action.
\end{enumerate}
\end{theorem}

\begin{proof}
\textbf{(a)}  The carrying cost $r\mathcal{L}\,dt$
accumulated per unit time grows with $\mathcal{L}$;
the marginal benefit of delay (discounting the political
cost $\kappa$) is bounded by $\rho\kappa\,dt$.  When
$r\mathcal{L} > \rho\kappa$, delaying is strictly
dominated.  This occurs at an interior point
$\mathcal{L}^* \leq \rho\kappa/r < \mathcal{L}_{\max}$.

\textbf{(b)}  Standard optimal stopping: the value
function must be $C^1$ at the free boundary to prevent
arbitrage between continuation and stopping.

\textbf{(c)}  Higher $\sigma$ increases the probability
that $\mathcal{L}$ jumps past $\mathcal{L}_{\max}$
(forced default, cost $> \kappa$) before the sovereign
can respond.  The option value of delay decreases, and
the free boundary shifts inward.
\end{proof}

% ================================================================
\section{Choice of exit face}\label{sec:gf-exitface}
% ================================================================

\begin{definition}[Exit face decomposition]\label{def:exit-face}
The boundary $\partial K_F$ of the fiscal viability kernel
\eqref{eq:fiscal-kernel} decomposes into faces, each
corresponding to a distinct absorption channel:
\begin{enumerate}[label=(\roman*)]
  \item $\partial_{\textup{haircut}}$: creditor losses
    (debt restructuring, write-downs, bail-in);
  \item $\partial_{\textup{inflation}}$: currency-holder
    losses (monetary expansion, real devaluation);
  \item $\partial_{\textup{tax}}$: taxpayer losses
    (fiscal austerity, increased taxation);
  \item $\partial_{\textup{default}}$: disorderly failure
    (uncontrolled capital flight, banking collapse,
    social instability).
\end{enumerate}
Each face $\partial_k K_F$ corresponds to a vertex of the
absorption simplex $\Delta_{\mathcal{L}}$ where agent
class~$k$ absorbs the entirety of $\mathcal{L}(t_0)$.
Interior points of $\Delta_{\mathcal{L}}$ correspond to
mixed resolutions that distribute losses across faces.
\end{definition}

\begin{theorem}[Controlled vs.\ uncontrolled exit]
\label{thm:exit-choice}
Let $c_i > 0$ be the social cost coefficient of imposing
loss $\ell_i$ on agent class~$i$.  Define the damage
functional
\begin{equation}\label{eq:damage-functional}
  \mathcal{D}(\ell) \;:=\;
  \sum_{i=1}^{m} c_i\, \ell_i^2.
\end{equation}
Then:
\begin{enumerate}[label=(\alph*)]
  \item \textbf{Controlled exit.}  If the sovereign
    activates the feedback function $C(x)$
    (\cref{def:feedback}) at the boundary $\partial K_F$,
    the optimal allocation minimises $\mathcal{D}$ on
    $\Delta_{\mathcal{L}}$:
    \begin{equation}\label{eq:optimal-allocation}
      \ell_i^* \;=\;
      \frac{\mathcal{L}(t_0)}{c_i\,
      \displaystyle\sum_{j=1}^{m} c_j^{-1}},
    \end{equation}
    distributing losses inversely proportional to social
    cost.  The exit face is chosen deliberately, and
    the trajectory satisfies the viability condition
    (\cref{thm:viability-di}) throughout the restructuring.
  \item \textbf{Uncontrolled exit.}  If the sovereign fails
    to activate $C(x)$, the autonomous debt dynamics
    ($D' = rD + \delta - \tau$ with $rD > \tau$) drive
    the state toward $\partial_{\textup{default}}$, which
    is generically the worst-case face: it maximises
    $\mathcal{D}$ by concentrating losses on the most
    vulnerable agents (depositors, small creditors,
    the poor through inflation).
\end{enumerate}
\end{theorem}

\begin{proof}
\textbf{(a)}  Minimise $\mathcal{D}(\ell) =
\sum_i c_i \ell_i^2$ subject to $\sum_i \ell_i =
\mathcal{L}(t_0)$ and $\ell_i \geq 0$.
The Lagrangian is
$L = \sum_i c_i \ell_i^2 -
\lambda(\sum_i \ell_i - \mathcal{L}(t_0))$.
First-order conditions: $2 c_i \ell_i = \lambda$ for all~$i$,
giving $\ell_i = \lambda / (2c_i)$.
Substituting into the constraint:
$\sum_i \lambda/(2c_i) = \mathcal{L}(t_0)$,
so $\lambda = 2\mathcal{L}(t_0) / \sum_j c_j^{-1}$,
yielding \eqref{eq:optimal-allocation}.
The second-order condition ($\nabla^2 \mathcal{D}$
positive definite on the constraint surface) is satisfied
since $c_i > 0$.

\textbf{(b)}  Without active control, the dynamics follow
the autonomous differential inclusion.  The face
$\partial_{\textup{default}}$ is the \emph{attractor} of
the uncontrolled dynamics: when $rD > \tau$ and no
restructuring occurs, $D$ grows exponentially and the
system exits through disorderly default.  This face
concentrates losses on agents with the least political
power to resist (depositors lose savings, workers lose
employment, currency holders lose purchasing power),
which by the structure of the cost coefficients $c_i$
corresponds to high-cost agents, maximising $\mathcal{D}$.
\end{proof}

\begin{remark}[Courage is the choice function]
\label{rem:courage}
The difference between Iceland (2008) and Japan (1990s)
is not economics.  Iceland's GDP per capita was a fraction
of Japan's; its financial system was less sophisticated;
its policy toolkit was smaller.  The difference is the
\emph{presence or absence of the choice function}
$C(x)$ at the boundary.
Iceland chose $\partial_{\textup{haircut}}$: creditors
(including foreign depositors) absorbed losses.  Japan
refused to choose, deferring via extend-and-pretend.
In the framework, ``courage'' is not a moral quality but
a mathematical one: it is the activation of $C(x)$---the
feedback control---at the boundary $\partial K_F$.
A sovereign that activates $C(x)$ can compute and
implement \eqref{eq:optimal-allocation}.  A sovereign
that does not is governed by autonomous dynamics, which
by \cref{thm:exit-choice}(b) generically select the
worst exit face.
\end{remark}

\begin{theorem}[Dynamic absorption path]\label{thm:dynamic-allocation}
The sovereign chooses absorption rates
$a_i(t) \geq 0$ for each channel
$i \in \{1, \ldots, m\}$ to minimise the discounted
quadratic cost
\[
  J \;=\; \int_0^{\infty}
  e^{-\rho t}\sum_{i=1}^{m} c_i\, a_i(t)^2\, dt,
\]
subject to
$\mathcal{L}'(t) = r\,\mathcal{L}(t) - \sum_i a_i(t)$
with $\mathcal{L}(0) = \mathcal{L}_0$.
Assume $\rho < r$ (the sovereign is more patient than
the debt is aggressive).  Then:
\begin{enumerate}[label=(\alph*)]
  \item The value function is
    $V(\mathcal{L}) = \frac{1}{2}\beta\,\mathcal{L}^2$
    with $\beta = 2(2r - \rho)/H$, where
    $H = \sum_{j=1}^m c_j^{-1}$.
  \item The Hamilton--Jacobi--Bellman equation is
    \begin{equation}\label{eq:hjb}
      \rho\, V(\mathcal{L})
      \;=\;
      V'(\mathcal{L})\, r\,\mathcal{L}
      \;-\; \frac{H}{4}\,\bigl(V'(\mathcal{L})\bigr)^2.
    \end{equation}
  \item The optimal absorption rate through channel~$i$
    is
    \begin{equation}\label{eq:optimal-rate}
      a_i^*(t) \;=\; \frac{(2r - \rho)}{c_i\, H}\,
      \mathcal{L}(t).
    \end{equation}
    The allocation ratio $a_i^*/a_j^* = c_j/c_i$ is
    time-independent---the dynamic problem does not change
    \emph{who} pays, only \emph{how fast}.
  \item Under the optimal policy, the loss field decays
    exponentially:
    \begin{equation}\label{eq:optimal-decay}
      \mathcal{L}(t) \;=\;
      \mathcal{L}_0\, e^{-(r - \rho)\, t}.
    \end{equation}
    The decay rate $r - \rho$ is the net urgency:
    compounding rate minus impatience.  A patient
    sovereign ($\rho$ small) absorbs faster.
\end{enumerate}
\end{theorem}

\begin{proof}
\textbf{(b)}  The HJB equation for the
infinite-horizon discounted problem with state dynamics
$\mathcal{L}' = r\mathcal{L} - \sum a_i$ and running
cost $\sum c_i a_i^2$ is
$\rho V = \min_{a \geq 0}\bigl[\sum c_i a_i^2 +
V'(r\mathcal{L} - \sum a_i)\bigr]$.
First-order condition: $2c_i a_i = V'$, giving
$a_i = V'/(2c_i)$.  Substituting into the HJB:
\[
  \rho V
  = \sum_i \frac{(V')^2}{4c_i}
    + V'\Bigl(r\mathcal{L}
    - \frac{V'}{2}\sum_i c_i^{-1}\Bigr)
  = V' r\mathcal{L} - \frac{H}{4}(V')^2.
\]

\textbf{(a)}  Substitute $V = \tfrac{1}{2}\beta
\mathcal{L}^2$ into \eqref{eq:hjb}:
$\tfrac{1}{2}\rho\beta\mathcal{L}^2 =
\beta r \mathcal{L}^2 -
\tfrac{H}{4}\beta^2\mathcal{L}^2$,
giving $\rho/2 = r - H\beta/4$, hence
$\beta = 2(2r - \rho)/H$.

\textbf{(c)}  From $a_i = V'/(2c_i) =
\beta\mathcal{L}/(2c_i)$, substituting~$\beta$.

\textbf{(d)}  Total absorption rate
$a^* = \beta H \mathcal{L}/2 = (2r - \rho)\mathcal{L}$.
The dynamics become
$\mathcal{L}' = r\mathcal{L} - (2r - \rho)\mathcal{L}
= -(r - \rho)\mathcal{L}$, with solution
$\mathcal{L}_0 e^{-(r - \rho)t}$.
\end{proof}

\begin{proposition}[Nash bargaining on the simplex]
\label{prop:nash-bargaining}
Suppose the $m$ agent classes negotiate the allocation
$(\ell_1, \ldots, \ell_m) \in \Delta_{\mathcal{L}}$
with bargaining powers $\alpha_i > 0$
($\sum \alpha_i = 1$).  Let $d_i > 0$ be the loss
agent~$i$ would suffer under uncontrolled default
($\partial_{\textup{default}}$ of
\cref{def:exit-face}), with
$\sum_i d_i > \mathcal{L}_0$ (default is inefficient:
it destroys more value than the actual loss).
The Nash bargaining solution is
\begin{equation}\label{eq:nash-bargaining}
  \ell_i^{\textup{Nash}}
  \;=\;
  d_i \;-\; \alpha_i\, S,
  \qquad
  S \;:=\; \sum_{j=1}^{m} d_j \;-\; \mathcal{L}_0,
\end{equation}
where $S > 0$ is the \emph{surplus from agreement}---the
total value saved by choosing a controlled exit face
over default.  Each agent's loss equals their default
loss minus their share of the surplus; stronger
bargaining power $\alpha_i$ secures a larger share
of~$S$.
\end{proposition}

\begin{proof}
The Nash bargaining solution maximises
$\prod_i (d_i - \ell_i)^{\alpha_i}$ subject to
$\sum \ell_i = \mathcal{L}_0$ and $\ell_i \leq d_i$.
Taking logarithms: maximise
$\sum \alpha_i \ln(d_i - \ell_i)$ subject to
$\sum \ell_i = \mathcal{L}_0$.
First-order condition:
$\alpha_i/(d_i - \ell_i) = \lambda$ for all~$i$,
giving $d_i - \ell_i = \alpha_i/\lambda$.
Summing: $S = 1/\lambda$, so
$\ell_i = d_i - \alpha_i S$.
\end{proof}

% ================================================================
\section{The guarantee network}\label{sec:gf-network}
% ================================================================

The preceding sections treat debt as a scalar $D(t)$.
In practice, 城投 debt is a \emph{network}: thousands
of local government financing vehicles connected by
cross-guarantees, interbank exposure, and shadow banking
channels.  The systemic risk is not the total $D$ but the
\emph{topology} of how losses propagate.

\begin{definition}[Guarantee graph]\label{def:guarantee-graph}
The \emph{guarantee graph} is a weighted directed graph
$G_{\textup{guar}} = (V, E, A)$ where:
\begin{enumerate}[label=(\roman*)]
  \item $V = \{1, \ldots, n\}$ is the set of fiscal
    entities (LGFVs, banks, local governments);
  \item $E \subseteq V \times V$ is the set of guarantee
    relationships;
  \item $A = (A_{ij})$ is the \emph{guarantee adjacency
    matrix}: $A_{ij} \geq 0$ is the fraction of
    entity~$j$'s debt guaranteed by entity~$i$.
\end{enumerate}
When entity $j$ defaults with loss $\ell_j$, entity~$i$
absorbs $A_{ij}\, \ell_j$.  If this pushes $i$ into
default, $i$'s loss propagates along the edges of
$G_{\textup{guar}}$.
\end{definition}

\begin{theorem}[Spectral contagion threshold]
\label{thm:spectral-contagion}
Let $\alpha \in (0, 1]$ be the loss-given-default rate
and $\rho(A)$ the spectral radius of the guarantee
adjacency matrix~$A$.
\begin{enumerate}[label=(\alph*)]
  \item \textbf{Subcritical}
    ($\alpha\, \rho(A) < 1$).
    The total systemic loss from an initial shock vector
    $\ell^{(0)} \in \R^n_{\geq 0}$ is
    \begin{equation}\label{eq:leontief}
      \ell^{(\infty)}
      \;=\;
      (I - \alpha\, A)^{-1}\, \ell^{(0)},
    \end{equation}
    which is finite.  The amplification factor is
    bounded by $1/(1 - \alpha\,\rho(A))$.
  \item \textbf{Supercritical}
    ($\alpha\, \rho(A) \geq 1$).
    The matrix $(I - \alpha A)$ is singular or has a
    non-positive eigenvalue: a finite initial shock
    produces unbounded cascading losses.
  \item \textbf{Connection to the mean field.}
    By the Perron--Frobenius theorem, $\rho(A)$ is
    bounded above by the maximum row sum
    $\max_i \sum_j A_{ij}$.  The spectral contagion
    threshold $\alpha\,\rho(A) = 1$ is the
    knife-is-the-mean theorem (\cref{thm:meanfield})
    applied to the guarantee network: the mean
    guarantee exposure determines whether the network
    is a tool (distributing losses) or a knife
    (amplifying them to systemic scale).
\end{enumerate}
\end{theorem}

\begin{proof}
\textbf{(a)}  After $k$ rounds of contagion, the
cumulative loss is
$\ell^{(k)} = \sum_{j=0}^{k}(\alpha A)^j \ell^{(0)}$.
The Neumann series converges iff
$\rho(\alpha A) = \alpha\rho(A) < 1$, in which case
the sum is $(I - \alpha A)^{-1}$ and
$\|(I - \alpha A)^{-1}\| \leq 1/(1 - \alpha\rho(A))$.

\textbf{(b)}  When $\alpha\rho(A) \geq 1$, the Neumann
series diverges: repeated application amplifies shocks
along the Perron eigenvector.

\textbf{(c)}  By Perron--Frobenius (applied to the
non-negative matrix~$A$),
$\rho(A) \leq \max_i \sum_j A_{ij}$.
\end{proof}

\begin{proposition}[Percolation threshold]
\label{prop:percolation-threshold}
Model each guarantee edge in $G_{\textup{guar}}$ as
\emph{active} (transmitting default) with probability
$p$ and inactive with probability $1 - p$.  Let $d$
be the mean degree of $G_{\textup{guar}}$.  There
exists a critical probability
\begin{equation}\label{eq:percolation-threshold}
  p_c \;\approx\; \frac{1}{d}
\end{equation}
such that:
\begin{enumerate}[label=(\alph*)]
  \item for $p < p_c$, the expected cascade from a
    single default is $O(\log n)$---contagion stays
    local;
  \item for $p > p_c$, the expected cascade is
    $O(n)$---contagion is systemic with positive
    probability.
\end{enumerate}
The percolation threshold is a second knife-is-the-mean
result: the critical probability depends on the mean
connectivity $d$ of the network, not on any individual
node.
\end{proposition}

\begin{proof}
By the theory of bond percolation on random graphs
(Erd\H{o}s--R\'enyi), the giant component emerges when
the mean number of active edges per node exceeds~$1$:
$pd > 1$.  Below this threshold, connected components
are $O(\log n)$ (subcritical).  Above it, a positive
fraction of nodes belongs to a single giant component
(supercritical).  The critical probability is
$p_c = 1/d$.
\end{proof}

% ================================================================
\section{The GovFi platform}\label{sec:gf-platform}
% ================================================================

\begin{definition}[GovFi ledger]\label{def:govfi-ledger}
A \emph{GovFi ledger} is a fiscal execution graph
(\cref{def:exgraph}) equipped with three properties:
\begin{enumerate}[label=(\roman*)]
  \item \textbf{Full observability.}
    $\Obs = \mathcal{F}$: every fiscal flow (revenue,
    expenditure, debt issuance, debt service, transfer) is
    recorded on-ledger and publicly queryable.  No hidden
    off-balance-sheet vehicles exist.
  \item \textbf{Programmatic breakpoints.}
    Smart contracts encode graduated restructuring
    triggers: when $D/\tau_{\max}$ crosses predefined
    thresholds $\theta_1 < \theta_2 < \cdots < \theta_n$,
    the corresponding restructuring protocol activates
    automatically (\cref{cor:breakpoint}).
  \item \textbf{Real-time loss tracking.}
    The loss field $\mathcal{L}(t)$
    (\cref{def:loss-field}) is computed continuously from
    on-ledger data and published as a public signal.
\end{enumerate}
\end{definition}

\begin{proposition}[Full observability eliminates hidden knives]
\label{prop:govfi-observability}
Under full observability ($\Obs = \mathcal{F}$), no debt
instrument can satisfy the knife definition
(\cref{def:knife}) \emph{without detection}.  By
\cref{prop:imperfect}, the gap between autonomous actuation
and detected actuation is
$\|U(x)\| - \|U(x) \cap \Obs(x)\|$.  Full observability
closes this gap to zero: every knife is visible the moment
it forms.

In the 城投 context, the primary failure mode is
\emph{hidden} debt---off-balance-sheet vehicles,
implicit guarantees, shadow banking channels.  GovFi's
full observability eliminates this failure mode by
construction: if it is not on the ledger, it does not
exist as a fiscal obligation.
\end{proposition}

\begin{remark}[Revelation and observability]
\label{rem:revelation}
Under standard mechanism design, if the sovereign
allocates losses based on self-reported fiscal states
$\hat{\mathcal{L}}_i$, every agent has incentive to
underreport: claiming smaller losses shifts the burden
to others.  The revelation principle guarantees that
any incentive-compatible allocation can be implemented
as a direct mechanism, but only with auditing or side
payments that introduce their own costs.

GovFi bypasses the revelation problem entirely.
With $\Obs = \mathcal{F}$, the ledger observes each
entity's $\mathcal{L}_i$ directly from on-chain data.
Self-reporting is unnecessary; incentive compatibility
is achieved not by mechanism design but by
\emph{architectural design}---the same full
observability that eliminates hidden knives
also eliminates strategic misreporting.
\end{remark}

\begin{proposition}[Smart contracts as breakpoints]
\label{prop:govfi-breakpoints}
The programmatic breakpoints define a graduated
restructuring schedule
$\sigma \colon [0, 1] \to \Delta_{\mathcal{L}}$,
mapping the debt stress level
$s = D / D_{\max}$ to a point on the absorption simplex.
At each threshold $\theta_k$:
\begin{enumerate}[label=(\roman*)]
  \item the restructuring protocol specifies which exit
    face receives how much loss: $\sigma(\theta_k) =
    (\ell_1^{(k)}, \ldots, \ell_m^{(k)})$;
  \item the activation is automatic: no political decision
    is required at the moment of crisis;
  \item the graduated structure ensures that early
    interventions ($\theta_1, \theta_2$) impose small,
    distributed losses, preventing the accumulation that
    would force a large, concentrated loss at
    $\partial_{\textup{default}}$.
\end{enumerate}
By \cref{cor:breakpoint}, the breakpoints convert sovereign
debt from a \textcolor{knife}{knife} (autonomous,
uncontrolled actuation) to a \textcolor{water}{tool}
(programmatic, controlled actuation): the autonomous
dynamics are interrupted by pre-committed restructuring
at each threshold.
\end{proposition}

\begin{proposition}[Critical political delay]\label{prop:delay-stability}
Under proportional absorption control
$a(t) = k\,\mathcal{L}(t - \delta_{\textup{lag}})$
with gain $k > r$, the linearised delay differential
equation
\begin{equation}\label{eq:delay-characteristic}
  \mathcal{L}'(t) \;=\;
  r\,\mathcal{L}(t)
  \;-\; k\,\mathcal{L}(t - \delta_{\textup{lag}})
\end{equation}
is stable if and only if
\begin{equation}\label{eq:critical-delay}
  \delta_{\textup{lag}}
  \;<\;
  \delta^*
  \;:=\;
  \frac{\arccos(r/k)}{\sqrt{k^2 - r^2}}.
\end{equation}
When $\delta_{\textup{lag}} > \delta^*$, a Hopf
bifurcation occurs: the delayed political response
overshoots the target, then undershoots, creating
oscillatory instability in the loss field.
\end{proposition}

\begin{proof}
The characteristic equation of
\eqref{eq:delay-characteristic} is
$\lambda = r - k\, e^{-\lambda\delta}$.
At the stability boundary $\lambda = i\omega$.
Separating real and imaginary parts:
$0 = r - k\cos(\omega\delta)$ and
$\omega = k\sin(\omega\delta)$.
The first gives $\cos(\omega\delta) = r/k$; from
$\cos^2 + \sin^2 = 1$:
$\omega = \sqrt{k^2 - r^2}$.
Then $\delta^* = \arccos(r/k)/\omega$, yielding
\eqref{eq:critical-delay}.
For $\delta < \delta^*$, all characteristic roots
satisfy $\mathrm{Re}(\lambda) < 0$.
\end{proof}

\begin{remark}[Time consistency via smart contracts]
\label{rem:time-consistency}
The Kydland--Prescott problem applies to fiscal
restructuring: a sovereign that can revise its
breakpoint thresholds $\theta_k$ ex post has an
incentive to defer.  At $t = 0$, the optimal plan says
``restructure at $\theta_1$.''  When
$D(t)/\tau_{\max}$ reaches $\theta_1$, the sovereign
re-optimises and finds deferral preferable (the
political cost is immediate; the compounding cost is
diffuse).  Iterating, the sovereign never acts---the
time-inconsistency trap.

The GovFi breakpoints (\cref{prop:govfi-breakpoints})
resolve this by encoding the restructuring schedule in
immutable smart contracts.  The schedule cannot be
revised at the moment of crisis because the code does
not accept revision.  Time consistency is enforced not
by reputation or institutional norms but by the
\emph{architecture} of the ledger: the same
immutability that provides observability also provides
commitment.
\end{remark}

\begin{remark}[GovFi collapses the dual ring]
\label{rem:govfi-dualring}
The dual-ring system (\cref{def:dual-ring}) in the fiscal
context has an inner ring (formal fiscal rules: balanced
budget laws, debt ceilings, Maastricht criteria) and an
outer ring (actual fiscal practice: off-budget spending,
implicit guarantees, creative accounting).  The pathology
documented in \cref{prop:outer-drives} applies: the outer
ring drives the inner ring, and the formal rules become
decoration.

GovFi collapses the dual ring by making the outer ring
observable.  When $\Obs = \mathcal{F}$, the distinction
between formal rules and actual practice vanishes: every
flow is on-ledger, and the ``outer ring'' of hidden
transactions ceases to exist.  The dual-ring dynamics
\eqref{eq:inner-ring}--\eqref{eq:outer-ring} reduce to
a single observable system, and the tangential condition
(\cref{thm:viability-di}) can be verified directly from
the ledger data.
\end{remark}

\begin{example}[水电站 --- GovFi worked example]
\label{ex:dam}
A provincial government issues bonds to build a
hydroelectric power station (水电站).
The total bond issuance is~$B$; the physical
construction cost is~$C < B$; the loss field is
$\mathcal{L} = B - C$.
The fiscal execution graph has three layers
and three interested parties.

\paragraph{三方 (Three parties).}

\begin{center}
\renewcommand{\arraystretch}{1.25}
\begin{tabular}{@{}clp{5.5cm}@{}}
\toprule
\textbf{Layer} & \textbf{Party}
  & \textbf{Role in flow graph} \\
\midrule
0 & 政府 (Government)
  & Source: issues bonds, disburses funds \\
1 & 公司 (Companies)
  & Routing: general contractor, subcontractors,
    suppliers \\
2 & 建设者 (Builders)
  & Sink: construction workers, engineers,
    equipment operators \\
\bottomrule
\end{tabular}
\end{center}

\begin{center}
\begin{tikzpicture}[
  % ── styles ──
  wbox/.style={rectangle, rounded corners=4pt,
    draw=water, thick, fill=water!8,
    minimum width=6.0cm, minimum height=1.0cm,
    align=center, font=\small},
  gfbox/.style={rectangle, rounded corners=6pt,
    draw=sword, thick, fill=sword!5,
    minimum height=1.0cm, align=center, font=\small},
  leak/.style={rectangle, rounded corners=4pt,
    draw=knife, thick, dashed, fill=knife!6,
    minimum width=2.6cm, minimum height=0.9cm,
    align=center, font=\small},
  brk/.style={rectangle, rounded corners=3pt,
    draw=caution, thick, fill=caution!8,
    minimum height=0.9cm, align=center, font=\small},
  arr/.style={-{Stealth[length=6pt]}, thick},
  darr/.style={-{Stealth[length=6pt]}, thick, dashed, knife},
  oarr/.style={-{Stealth[length=6pt]}, thick, sword},
  lbl/.style={font=\scriptsize, fill=white, inner sep=2pt},
]
  % ════════════════════════════════════════════
  %   PART 1 — Money flow (three layers)
  % ════════════════════════════════════════════

  % ── Layer 0: Government ──
  \node[wbox] (gov) at (0, 0)
    {\textcolor{water}{\textbf{Layer 0}}
     \;---\; 政府 (Gov): issues bonds $B$};

  % ── Layer 1: Companies — trumpet shape ──
  %    Anchor coordinates for the horn
  \coordinate (horn-top-L)  at (-1.2, -2.1);
  \coordinate (horn-top-R)  at ( 1.2, -2.1);
  \coordinate (horn-bot-L)  at (-3.6, -4.0);
  \coordinate (horn-bot-R)  at ( 3.6, -4.0);
  \coordinate (horn-center) at ( 0.0, -3.1);
  \coordinate (horn-top)    at ( 0.0, -2.1);
  \coordinate (horn-bot)    at ( 0.0, -4.0);
  \coordinate (horn-east)   at ( 2.8, -3.1);

  % Fill
  \fill[black!5]
    (horn-top-L)
    .. controls ++(0,-0.8) and ++(0,0.8) ..
    (horn-bot-L)
    -- (horn-bot-R)
    .. controls ++(0,0.8) and ++(0,-0.8) ..
    (horn-top-R)
    -- cycle;
  % Outline
  \draw[black!40, thick]
    (horn-top-L)
    .. controls ++(0,-0.8) and ++(0,0.8) ..
    (horn-bot-L)
    -- (horn-bot-R)
    .. controls ++(0,0.8) and ++(0,-0.8) ..
    (horn-top-R)
    -- (horn-top-L);
  % Label
  \node[font=\small, anchor=center] at (horn-center)
    {\textbf{Layer 1} \;---\;
     公司 (Companies)};
  \node[font=\scriptsize, anchor=north]
    at ([yshift=-0.1cm]horn-center)
    {amplifiers \;/\; 喇叭};

  % ── Layer 2: Builders ──
  \node[wbox] (build) at (0, -5.6)
    {\textcolor{water}{\textbf{Layer 2}}
     \;---\; 建设者 (Builders): construction $C$};

  % ── Money arrows ──
  \draw[arr, water, line width=1.4pt]
    (gov.south) -- node[lbl, right=3pt]
    {$B_{\textup{disbursed}}$} (horn-top);
  \draw[arr, water, line width=1.4pt]
    (horn-bot) -- node[lbl, right=3pt]
    {wages, materials} (build.north);

  % ── Min-cut (between Layer 0 and horn) ──
  \draw[knife, very thick, densely dashed]
    (-5.0, -1.5) -- (5.0, -1.5)
    node[anchor=west, font=\scriptsize\bfseries,
         text=knife] {\;min-cut};

  % ── Hidden leak (right of horn) ──
  \node[leak, anchor=west] (corrupt) at (4.4, -3.1)
    {\textcolor{knife}{隐性流失}\\[-2pt]
     \textcolor{knife}{\scriptsize hidden leak}};
  \draw[darr, line width=1.0pt]
    (horn-east) -- node[lbl, above] {层层转包}
    (corrupt.west);

  % ════════════════════════════════════════════
  %   PART 2 — GovFi observability layer
  % ════════════════════════════════════════════

  % ── Wide GovFi ledger ──
  \node[gfbox, minimum width=13.0cm]
    (ledger) at (0, -7.6)
    {\textcolor{sword}{\textbf{GovFi ledger}}\;:\;
     $\Obs = \mathcal{F}$
     \qquad{\scriptsize on-chain, immutable,
     publicly queryable}};

  % ── On-chain arrows (left-side routing from ledger to layers) ──
  % Arrow → Layer 0 (Gov) — outermost vertical run
  \draw[oarr, line width=1.2pt]
    ([yshift=0.15cm]ledger.west) -- (-7.6, -7.45)
    -- (-7.6, 0) -- (gov.west);
  % Arrow → Layer 1 (Companies) — middle vertical run
  \draw[oarr]
    (ledger.west) -- (-7.3, -7.6)
    -- (-7.3, -2.1) -- (horn-top-L);
  % Arrow → Layer 2 (Builders) — innermost vertical run
  \draw[oarr]
    ([yshift=-0.15cm]ledger.west) -- (-7.0, -7.75)
    -- (-7.0, -5.6) -- (build.west);

  % ── Label ──
  \node[font=\scriptsize, sword, anchor=east]
    at (-7.7, -4.8) {on-chain};

  % ── Corruption exposed ──
  \draw[oarr]
    ([xshift=5.0cm]ledger.north) -- ++(0,0.5)
    -- ++(0,2.0) -| (corrupt.south);
  \node[font=\scriptsize, sword,
    anchor=north west]
    at ([yshift=-0.1cm]corrupt.south east)
    {exposed};
  \node[font=\scriptsize, sword,
    anchor=north west]
    at ([yshift=-0.45cm]corrupt.south east)
    {$\Rightarrow$ eliminated};

  % ── Loss field ──
  \node[gfbox, minimum width=9.0cm]
    (loss) at (0, -9.2)
    {$\mathcal{L}(t) \;=\;
     B_{\textup{disbursed}}(t)
     \;-\; C_{\textup{verified}}(t)$};
  \draw[arr, sword, line width=1.0pt]
    (ledger.south) -- node[lbl, right=2pt]
    {computes} (loss.north);

  % ── Breakpoints ──
  \node[brk, minimum width=7.0cm]
    (bp) at (0, -10.6)
    {\textcolor{caution}{$\theta_1 = 5\%$\qquad
     $\theta_2 = 10\%$\qquad $\theta_3 = 15\%$}};
  \draw[arr, caution, line width=1.0pt]
    (loss.south) -- node[lbl, right=2pt]
    {\scriptsize auto-audit triggers} (bp.north);
\end{tikzpicture}
\end{center}

\noindent
Money flows $0 \to 1 \to 2$.
The min-cut of this three-layer graph is at
Layer~1: companies control both the information
(what the government sees) and the money (what the
builders receive).  This is the dual-ring structure
(\cref{def:dual-ring}): the inner ring is the contract
(合同); the outer ring is the actual payment chain
(实际资金链).  Corruption is flow diversion at the
min-cut---a hidden edge from Layer~1 to an
off-graph recipient.

\paragraph{公示清单 (Disclosure ledger).}
The GovFi ledger requires every item below to be
on-chain, immutable, and publicly queryable.

\medskip
\noindent
\textbf{Layer 0 --- 发债 (Bond issuance).}
\begin{enumerate}[label=(\roman*)]
  \item 债券总额 $B$, 利率 $r$, 期限 $T$
    (bond total, interest rate, maturity);
  \item 还款来源: 电费收入、财政转移支付
    (repayment source: electricity revenue,
    fiscal transfer);
  \item 可行性研究报告、环境影响评价
    (feasibility study, environmental impact
    assessment)---the physical basis for
    $V_{\textup{rec}}$;
  \item 预算分项: 土建、设备、设计、监理、应急
    (budget breakdown: civil works, equipment,
    engineering, supervision, contingency)---defines
    $C$ before construction starts.
\end{enumerate}

\noindent
\textbf{Layer 1 --- 采购链 (Procurement chain).}
This is the min-cut where corruption concentrates.
\begin{enumerate}[label=(\roman*)]
  \item 所有投标书: 投标人、报价、技术评分
    (all bids received: bidder, price, technical
    score)---if only the winner is published, rigged
    bidding is undetectable;
  \item 中标合同: 承包商、价格、范围、工期
    (winning contract: contractor, price, scope,
    timeline)---the edge in the execution graph;
  \item 完整分包树: 每一级分包商、价格、范围
    (full subcontractor tree: every tier, price,
    scope)---层层转包 (layer-by-layer subcontracting)
    is the primary hidden channel; each layer skims
    $10$--$15\%$, and by the fourth layer half the
    money is absorbed before reaching the builder;
  \item 材料单价: 品名、数量、单价、市场基准价
    (material unit prices vs.\ market benchmark)---inflated
    material costs are the second channel;
  \item 设备费用: 类型、来源、租购价、市场基准价
    (equipment costs vs.\ market benchmark);
  \item 变更签证: 范围变更、价格变更、理由
    (change orders: scope change, price change,
    justification)---post-contract inflation is the
    third channel: ``发现岩层'' (we discovered rock)
    $\Rightarrow$ hundreds of millions in extras.
\end{enumerate}

\noindent
\textbf{Layer 2 --- 实际施工 (Execution).}
\begin{enumerate}[label=(\roman*)]
  \item 每笔付款: 日期、金额、收款人
    (every payment to every worker and subcontractor:
    date, amount, recipient)---the sink of the
    flow graph;
  \item 工程进度: 里程碑、完成率、独立验收报告
    (physical progress: milestone, completion~\%,
    independent inspection)---defines
    $C_{\textup{verified}}(t)$;
  \item 材料进场: 品名、数量、来源、验收记录
    (material delivery: item, quantity, source,
    acceptance record)---cross-check against
    Layer~1 procurement prices;
  \item 质量检验: 每阶段工程师签字、缺陷清单
    (quality inspection: engineer sign-off, defect
    list)---prevents 豆腐渣工程
    (tofu-dreg construction).
\end{enumerate}

\paragraph{实时损耗场 (Real-time loss field).}
The loss field
\[
  \mathcal{L}(t) \;=\;
  B_{\textup{disbursed}}(t)
  \;-\; C_{\textup{verified}}(t)
\]
is published daily and decomposes into three components:

\begin{center}
\renewcommand{\arraystretch}{1.25}
\begin{tabular}{@{}lll@{}}
\toprule
\textbf{Component} & \textbf{Chinese}
  & \textbf{Acceptable?} \\
\midrule
Legitimate overhead
  & 合理管理费用 (管理、保险、许可)
  & Yes, bounded \\
Profit margins
  & 合理利润 (市场利润率)
  & Yes, market rate \\
Unexplained gap
  & 不明差额 $= \mathcal{L} -
    \text{overhead} - \text{margins}$
  & \textbf{No: corruption signal} \\
\bottomrule
\end{tabular}
\end{center}

\noindent
When the unexplained gap exceeds the breakpoint
threshold~$\theta_k$
(\cref{prop:govfi-breakpoints}), an automatic audit
triggers.  The graduated schedule:
$\theta_1 = 5\%$ (internal review),
$\theta_2 = 10\%$ (independent audit),
$\theta_3 = 15\%$ (public investigation, contract
suspension).

\paragraph{传统公示 vs.\ GovFi (Traditional disclosure
vs.\ GovFi).}
Traditional 公示 is a PDF on a government website:
it publishes the contract summary (Layer~1 top-level)
and the completion report (Layer~2 final).  It misses:

\begin{center}
\renewcommand{\arraystretch}{1.25}
\begin{tabular}{@{}lcc@{}}
\toprule
\textbf{Item} & \textbf{传统公示}
  & \textbf{GovFi} \\
\midrule
分包树 (subcontractor tree) & Hidden & On-chain \\
材料单价 (material unit prices) & Hidden & On-chain \\
变更签证 (change orders) & Delayed/partial & Real-time \\
工人工资 (worker payments) & Hidden & On-chain \\
$\mathcal{L}(t)$ 实时 (real-time loss) & Not computed
  & Published daily \\
自动审计触发 (auto audit trigger)
  & None & Breakpoints \\
\bottomrule
\end{tabular}
\end{center}

\noindent
The difference is not what information exists but
what \emph{architecture} delivers it.  A PDF can be
delayed, redacted, or revised.  A ledger is immutable,
machine-readable, and queryable by any citizen.
The dual ring (\cref{def:dual-ring}) collapses because
the outer ring (actual payments) is identical to the
inner ring (on-chain records): there is no gap in which
corruption can hide.

\paragraph{Numerical instantiation
(\texttt{govfi/simulate.py}).}
The simulation concretises the symbolic framework with
the following parameters:

\begin{center}
\renewcommand{\arraystretch}{1.25}
\begin{tabular}{@{}lrl@{}}
\toprule
\textbf{Parameter} & \textbf{Value} & \textbf{Rationale} \\
\midrule
Budget $B$ & $100$\,亿元 & provincial 城投 bond scale \\
Bond rate $r$ & $5\%$ & typical LGFV coupon \\
Construction schedule & 5\,years & medium hydro project \\
Political delay $\delta_{\textup{lag}}$ & 0.5\,years
  & legislative cycle \\
Revenue volatility $\sigma$ & 2\,亿/yr
  & electricity demand variation \\
Breakpoints $(\theta_1, \theta_2, \theta_3)$
  & $(5\%, 10\%, 15\%)$
  & \cref{prop:govfi-breakpoints} \\
\bottomrule
\end{tabular}
\end{center}

\noindent
Three scenarios test convergence of the loss field under
the DIDE dynamics (\cref{prop:absorption-time}) with
constant absorption rate~$\bar{a}$ (the step-function
approximation
$a(t) = \bar{a}\,\mathbf{1}_{t > \delta_{\textup{lag}}}$):

\begin{center}
\renewcommand{\arraystretch}{1.25}
\begin{tabular}{@{}lccccc@{}}
\toprule
\textbf{Scenario} & $\mathcal{L}_0$
  & $\bar{a}$ & \textbf{Conv.?}
  & $T^*$ \textbf{bound}
  & $\mathbb{E}[T^*]$ \\
\midrule
Clean ($\bar{a} = 5.0$)
  & 0.5 & 5.0 & Yes
  & 0.6\,yr & 0.1\,yr \\
Moderate corruption + GovFi
  & 10.0 & 3.0 & Yes
  & 4.6\,yr & 4.0\,yr \\
No control ($\bar{a} = 0$)
  & 10.0 & 0 & No
  & $+\infty$ & $+\infty$ \\
\bottomrule
\end{tabular}
\end{center}

\paragraph{Observations.}
\begin{enumerate}[label=(\roman*)]
\item \textbf{Clean baseline.}
  With full verification flow ($\bar{a} = 5.0$), the
  residual gap $\mathcal{L}_0 = 0.5$\,亿 (normal
  disbursement lead) absorbs in under one year.
  $\operatorname{Var}(T^*) \approx 0.02$\,yr$^2$:
  volatility is irrelevant when absorption dominates.

\item \textbf{Moderate corruption with GovFi.}
  A $10\%$ diversion at Layer~1 creates
  $\mathcal{L}_0 = 10$\,亿.  Breakpoints $\theta_1$
  and $\theta_2$ fire immediately ($t < 0.1$).
  With activated absorption $\bar{a} = 3.0$ the
  convergence condition
  $\bar{a} > r\,\mathcal{L}_0\,e^{r\delta_{\textup{lag}}}
  = 0.51$ holds, and the loss field reaches zero
  within $T^* \le 4.6$\,years
  ($\mathbb{E}[T^*] = 4.0$,
  $\operatorname{Var}(T^*) = 2.56$\,yr$^2$).
  The variance is non-trivial: stochastic revenue
  ($\sigma = 2$) amplified through the cubic factor
  $(a - r\mathcal{L}_0)^{-3}$
  (\cref{prop:stochastic-loss}).

\item \textbf{Unchecked corruption ($\bar{a} = 0$).}
  Without GovFi, no absorption is activated, and the
  loss field grows exponentially:
  $\mathcal{L}(5) = 10\,e^{0.25} = 12.8$\,亿,\;
  $\mathcal{L}(10) = 10\,e^{0.50} = 16.5$\,亿.
  The breakpoint $\theta_3 = 15\%$ fires at
  $t \approx 8.1$\,years, but with no control mechanism
  the signal goes unanswered.
  Project fails; money gone.

\item \textbf{Convergence gap.}
  The ratio
  $\bar{a}\, / \,(r\,\mathcal{L}_0\,
  e^{r\delta_{\textup{lag}}})$
  separates success from failure.
  In Scenario~2 this ratio is $3.0/0.51 \approx 5.9$:
  absorption outpaces compounding by nearly~$6\times$.
  In Scenario~3 the ratio is zero.
  GovFi's contribution is activating $\bar{a} > 0$;
  the magnitude of~$\bar{a}$ depends on political will,
  but the architecture ensures the signal reaches the
  sovereign before~$\theta_3$.
\end{enumerate}

\paragraph{Validation chain
(\texttt{govfi/simulate.py::run\_validation\_chain}).}
The simulation traces the full audit path for Scenario~2
(moderate corruption with GovFi), period by period.
Each link in the chain is recorded on the ledger;
every claim is machine-verifiable.

\smallskip\noindent
\emph{Step~0: Bond issuance.}
省政府 issues bonds: $B = 100$\,亿元, $r = 5\%$,
maturity $T = 5$\,yr.
GovFi ledger initialised; $\Obs = \mathcal{F}$.

\smallskip\noindent
\emph{Periods 1--4 ($t = 0.5$--$2.0$\,yr).}
Each period: 10\,亿 disbursed (Gov $\to$ Companies),
9\,亿 verified (10\% diverted at Layer~1).
Loss field grows linearly:
$\mathcal{L}(0.5) = 1.0$,\;
$\mathcal{L}(1.0) = 2.0$,\;
$\mathcal{L}(1.5) = 3.0$,\;
$\mathcal{L}(2.0) = 4.0$\,亿.
All below $\theta_1 = 5\%$.

\smallskip\noindent
\emph{Period~5 ($t = 2.5$\,yr).}
$\mathcal{L}(2.5) = 5.0$\,亿 $= 5.0\%$ of budget.
\textbf{Breakpoint $\theta_1$ fires.}
GovFi activates restructuring protocol:
absorption rate $\bar{a} = 3.0$ begins after
$\delta_{\textup{lag}} = 0.5$\,yr (legislative cycle).

\smallskip\noindent
\emph{Periods 6--10 ($t = 3.0$--$5.0$\,yr).}
Loss field continues to $\mathcal{L}(5.0) = 10.0$\,亿
($10\%$ of budget; $\theta_2$ fires at $t = 5.0$).
The convergence bound below is forward-looking from
this activation point.

\smallskip\noindent
\emph{Convergence proof.}
At the activation point
$\mathcal{L}_0 = 10.0$\,亿, the convergence condition
(\cref{prop:absorption-time}a) is verified:
\[
  \bar{a} = 3.0 \;>\;
  r\,\mathcal{L}_0\,e^{r\delta_{\textup{lag}}}
  = 0.05 \times 10.0 \times e^{0.025}
  = 0.5127.  \quad\checkmark
\]
The absorption bound gives
$T^* \le 0.5 + 10.25 / (3.0 - 0.51)
= 4.6$\,yr, with stochastic moments
$\mathbb{E}[T^*] = 4.0$\,yr,\;
$\operatorname{Var}(T^*) = 2.56$\,yr$^2$.
The ledger records 20~transactions (10~disbursements
+ 10~verifications), each publicly queryable.

\smallskip\noindent
\textbf{Verdict.}
The dam completes.  GovFi works.  Every link in the
chain---bond issuance, disbursement, verification,
loss computation, breakpoint trigger, absorption
activation, convergence proof---is auditable on-chain.
The 10\% diversion at Layer~1 is \emph{detected}
at $t = 2.5$\,yr (not at project completion)
and \emph{absorbed} within the construction horizon.
Without GovFi, the same diversion produces
$\mathcal{L}(10) = 16.5$\,亿 and growing.
\end{example}

%
\begin{example}[地狱公司结构 --- GovFi under monopoly]
\label{ex:monopoly}
Suppose Layer~1 is a \emph{monopoly} (垄断): a single
entity controls the min-cut, routing both information
and money.  The monopolist diverts $30\%$ of every
disbursement.  Does GovFi still work?

\paragraph{Without GovFi.}
The monopolist self-reports verification, inflating
$C_{\textup{verified}}$ to match $B_{\textup{disbursed}}$.
The ledger shows
$\mathcal{L}(t) = 0$ throughout construction.
All three breakpoints sleep.
In reality, actual construction covers only $70\%$ of
the budget; $30$\,亿 is gone.
The loss field is \emph{gamed}: the observability
gap $\|\Obs\| > 0$ hides the knife
(\cref{def:knife}, condition~(2)).

\paragraph{With GovFi ($\Obs = \mathcal{F}$).}
Independent on-chain verification (IoT sensors,
satellite imagery, certified inspectors) creates
$C_{\textup{verified}}$ that the monopolist cannot
inflate.  The loss field reveals the $30\%$ gap
in real time:

\begin{center}
\renewcommand{\arraystretch}{1.25}
\begin{tabular}{@{}cccc@{}}
\toprule
$t$ (yr) & $B_{\textup{disb}}$ & $C_{\textup{ver}}$
  & $\mathcal{L}/B$ \\
\midrule
0.5 & 10 & 7 & 3.0\% \\
1.0 & 20 & 14 & 6.0\%
  \;$\Leftarrow \theta_1$ \textbf{fires} \\
2.0 & 40 & 28 & 12.0\%
  \;$\Leftarrow \theta_2$ \textbf{fires} \\
2.5 & 50 & 35 & 15.0\%
  \;$\Leftarrow \theta_3$ \textbf{fires} \\
5.0 & 100 & 70 & 30.0\% \\
\bottomrule
\end{tabular}
\end{center}

\noindent
Detection occurs at $t = 1.0$\,yr (versus \emph{never}
without GovFi).  But detection alone is insufficient:
the \emph{absorption rate}~$\bar{a}$ determines whether
the loss field converges.

\paragraph{Absorption under monopoly.}
Two cases:
\begin{enumerate}[label=(\alph*)]
\item \textbf{Monopolist resists} ($\bar{a} = 2.0$).
  The convergence condition
  $\bar{a} > r\,\mathcal{L}_0\,e^{r\delta_{\textup{lag}}}
  = 1.54$ holds, but barely.
  $T^* \le 67.1$\,yr,\;
  $\mathbb{E}[T^*] = 60$\,yr.
  Detection without political action is
  \emph{cosmetic}: the signal exists, but
  nobody acts on it.

\item \textbf{$\theta_3$ breaks the monopoly}
  ($\bar{a} = 5.0$).
  The breakpoint at $\theta_3 = 15\%$
  ($t = 2.5$\,yr) triggers contract suspension
  and open re-bidding.
  The monopoly is dissolved; new contractors enter.
  Absorption strengthens:
  $T^* \le 9.4$\,yr,\;
  $\mathbb{E}[T^*] = 8.6$\,yr,\;
  $\operatorname{Var}(T^*) = 2.80$\,yr$^2$.
  The dam still completes---late, but it completes.
\end{enumerate}

\noindent
The ratio $\bar{a}\,/\,(r\,\mathcal{L}_0\,
e^{r\delta_{\textup{lag}}})$ is $2.0/1.54 = 1.3$ in
case~(a) and $5.0/1.54 = 3.2$ in case~(b):
breaking the monopoly more than doubles the convergence margin.

\paragraph{The Qin structure.}
The monopoly at Layer~1 is structurally identical to
the 秦-system (\cref{sec:qin}): a single node
concentrates all actuation ($U$), all information flow
($\Obs$), and resists external restructuring.
GovFi's contribution is collapsing the dual ring
(\cref{def:dual-ring}) so that $\Obs = \mathcal{F}$
regardless of Layer~1's market structure.
The loss field $\mathcal{L}(t)$ becomes a \emph{public}
signal; the breakpoints $\theta_k$ convert that signal
into automatic restructuring triggers.
Observability is necessary; political will
(activating $\bar{a}$) is sufficient.
\end{example}

% ================================================================
\section{Historical instantiation}\label{sec:gf-history}
% ================================================================

\begin{example}[Iceland 2008]\label{ex:iceland}
Iceland's banking system collapsed in October 2008 with
total banking assets at ${\sim}10\times$ GDP.

\begin{center}
\renewcommand{\arraystretch}{1.25}
\begin{tabular}{@{}llp{5.5cm}@{}}
\toprule
\textbf{Framework} & \textbf{Value} & \textbf{Outcome} \\
\midrule
Exit face chosen
  & $\partial_{\textup{haircut}}$
  & Foreign creditors absorbed ${\sim}60\%$ of losses \\
$C(x)$ activated
  & Yes
  & Parliament refused to honour Icesave guarantees \\
Time to recovery
  & 3--4 years
  & GDP regained pre-crisis level by 2012 \\
Path
  & (a) of \cref{thm:lifecycle}
  & Controlled restructuring \\
\bottomrule
\end{tabular}
\end{center}

\noindent
Iceland chose the exit face.  The choice was politically
costly (international condemnation, diplomatic pressure)
but mathematically optimal: by
\cref{thm:exit-choice}(a), distributing losses to the
agent class with the largest loss-absorption capacity
(foreign institutional creditors) minimises the damage
functional $\mathcal{D}$.
\end{example}

\begin{example}[Japan 1990s]\label{ex:japan}
Japan's asset bubble collapsed in 1991.  Non-performing
loans in the banking system reached ${\sim}35\%$ of GDP.

\begin{center}
\renewcommand{\arraystretch}{1.25}
\begin{tabular}{@{}llp{5.5cm}@{}}
\toprule
\textbf{Framework} & \textbf{Value} & \textbf{Outcome} \\
\midrule
Exit face chosen
  & None (deferred)
  & Extend-and-pretend for ${\sim}12$ years \\
$C(x)$ activated
  & No
  & MOF forbearance policy 1992--2003 \\
Time to resolution
  & $>$ 15 years
  & ``Lost decades'' (1991--2010+) \\
Path attempted
  & (c)---non-existent
  & Relabelled NPLs, rolled over zombie loans \\
\bottomrule
\end{tabular}
\end{center}

\noindent
Japan attempted path~(c): neither restructure nor
collapse, but indefinite deferral.  By
\cref{thm:fixedpoint}, this path does not exist.
The loss field $\mathcal{L}(t_0)$ was conserved
(\cref{thm:absorption-conservation}); deferral only
compounded it (\cref{rem:deferral}).  The losses were
eventually absorbed by taxpayers (bank recapitalisation),
depositors (near-zero interest rates for two decades),
and the entire economy (forgone growth).
\end{example}

\begin{proposition}[Extend-and-pretend violates the
fixed-point theorem]\label{prop:zombie}
The extend-and-pretend strategy (forbearance,
reclassification of NPLs, rollover of zombie loans)
is an attempt to modify the \emph{detection output}
$\Obs(D)$ without modifying the physical state~$D$.
In the framework:
\begin{enumerate}[label=(\roman*)]
  \item Relabelling an NPL changes the classification
    signal but not the loss field: $\mathcal{L}(t)$ is
    unchanged because $D(t)$ and $V_{\textup{rec}}(t)$
    are physical quantities independent of accounting
    labels.
  \item Rolling over a zombie loan sets $\delta(t) > 0$
    (new lending to service old debt), which by
    \eqref{eq:debt-dynamics} increases $D'(t)$: the
    loss field grows.
  \item Forbearance reduces $\Obs(D)$ by suppressing
    the observation signal, creating a gap
    $\|U\| - \|U \cap \Obs\| > 0$---exactly the
    condition of \cref{prop:imperfect} under which
    knives become hidden.
\end{enumerate}
By \cref{thm:fixedpoint}, there is no fixed point of
the system in which $\mathcal{L} > 0$ and no agent
absorbs loss: the absorption conservation law
\eqref{eq:absorption-conservation} has no solution
with $\ell_i = 0$ for all~$i$ when
$\mathcal{L}(t_0) > 0$.  Extend-and-pretend is not
a strategy; it is a violation of conservation.
\end{proposition}

\begin{remark}[The 城投 knife]\label{rem:chengtou}
China's local government financing vehicles
(地方政府融资平台, commonly called 城投) are the
current instantiation of the fiscal knife.  The
structure maps directly:
\begin{enumerate}[label=(\roman*)]
  \item \textbf{Autonomous actuation}: 城投 debt
    compounds at coupon rates of $5$--$8\%$ while
    underlying asset returns (land sales, toll roads,
    industrial parks) have declined below the cost of
    capital.  The self-financing loop
    (\cref{rem:debt-loop}) is active.
  \item \textbf{Hidden observability}: much 城投
    debt is off-balance-sheet, held through trust
    products, wealth management products, and
    interbank channels.  The dual-ring structure
    (\cref{def:dual-ring}) applies: the formal fiscal
    rules (inner ring) prohibit local government
    borrowing, while the actual practice (outer ring)
    operates through 城投 vehicles.
  \item \textbf{Conservation applies}: by
    \cref{thm:absorption-conservation}, the losses
    embedded in non-performing 城投 debt are conserved.
    The current policy of rolling over (化债) is
    channel~(iii) of the proof: deferral, not
    absorption.
\end{enumerate}
GovFi addresses the 城投 problem through all three
mechanisms: on-ledger recording eliminates the
hidden outer ring (\cref{rem:govfi-dualring}),
programmatic breakpoints prevent accumulation
to crisis levels (\cref{prop:govfi-breakpoints}),
and published $\mathcal{L}(t)$ ensures that the
conservation law is visible to all agents
(\cref{prop:govfi-observability}).
The knife is real.  The losses are real.  The only
question is the exit face.
\end{remark}

% ================================================================
\section*{Closing remark: the fiscal flow}
% ================================================================

\begin{remark}[Flow-theoretic restatement]\label{rem:govfi-flow}
In the language of the agentic calculus (\cref{sec:flow}),
the entire appendix restates as follows.
Debt restructuring is \emph{flow rerouting}: redirecting
fiscal flows from the default channel to a chosen
absorption channel.
The absorption conservation law
(\cref{thm:absorption-conservation}) is \emph{flow
conservation}: the total flow of losses through the
fiscal network is invariant; only the routing changes.
The GovFi ledger (\cref{def:govfi-ledger}) makes the
fiscal execution graph $G_F$ fully observable, so that
the sovereign can compute the \emph{min-cut}
(\cref{thm:flowcut}): the minimum-cost set of edges
whose removal (restructuring) restores the system to
$K_F$.
The knife-is-the-mean theorem (\cref{thm:meanfield})
applied to the fiscal domain states: the same debt
level $D$ is sustainable or unsustainable depending
on the system's mean fiscal capacity $\bar{w}_F$.
Growth raises $\bar{w}_F$; the same nominal debt becomes
a tool (financing productive investment) or a knife
(compounding toward default) depending on which side
of the mean the system sits.

The losses are conserved.  The only freedom is the
choice of exit face.
\end{remark}

% ================================================================
\section*{Epilogue}
% ================================================================

\begin{quote}
\itshape
然而我正对一本历史书\\
西望夕阳里的咸阳古道\\
我等到了一匹快马的蹄声

\medskip
\upshape
---卞之琳,《音尘》
\end{quote}

\bigskip

\begin{center}
\itshape
献给我最爱的女人:陈晓楚
\end{center}

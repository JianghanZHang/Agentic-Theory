\chapter{The Alien Model}\label{sec:alien}

This paper arrives at mathematics from outside.  Its axioms come from
Chinese imperial history, its dynamics from control theory, its
validation from two millennia of statecraft.  When this framework
looks inward at pure mathematics, it sees a single phenomenon---studied
from four independent directions---in the 2022 Fields Medal.

Each of the four 2022 Fields Medalists proved one face of a theorem
that the agentic framework identifies as:
\textcolor{sword}{the knife is the mean}.

\medskip
\noindent
\begin{center}
\begin{tabular}{@{}lll@{}}
\toprule
\textbf{Face} & \textbf{Medalist} & \textbf{Theorem (alien translation)} \\
\midrule
I   & Hugo Duminil-Copin   & The mean field is exact in $d \geq 4$ \\
II  & June Huh             & The execution graph has Hodge structure \\
III & James Maynard        & The phase transition is sharp (0--1 law) \\
IV  & Maryna Viazovska     & The optimal configuration is self-dual \\
\bottomrule
\end{tabular}
\end{center}

\medskip

The appendix proves one further consequence: together, the four faces
lock the \textbf{P\,/\,NP boundary} as a theorem of temporal linearity.

\section{The dictionary}\label{sec:alien-dict}

The translations below are not analogies.  They are identifications of
the same mathematical objects under different names.

\begin{center}
\small
\begin{tabular}{@{}p{3.8cm}p{4.5cm}p{4.5cm}@{}}
\toprule
\textbf{Agentic Theory} & \textbf{Mathematical object} & \textbf{Reference} \\
\midrule
Viability kernel $K$ &
  Configuration space &
  \cref{sec:axiom} \\
Knife (phase function) &
  Critical threshold &
  \cref{def:knife} \\
Mean field $\bar{U}$ &
  Order parameter &
  \cref{thm:meanfield} \\
Binary lifecycle &
  0--1 law / sharp phase transition &
  \cref{thm:lifecycle} \\
Fixed-point impossibility &
  Self-dual critical point uniqueness &
  \cref{thm:fixedpoint} \\
Unconstrained power paradox &
  Triviality (mean-field exactness) &
  \cref{thm:paradox} \\
Execution graph $G_E$ &
  Matroid / random-cluster graph &
  \cref{def:exgraph} \\
Cheeger constant $h(K_V)$ &
  Isoperimetric constant &
  \cref{def:cheeger-viab}, \cref{thm:cheeger} \\
Max-flow / min-cut &
  Hodge decomposition &
  \cref{thm:flowcut} \\
Viability metric $g_V = V^{-2}g_S$ &
  Conformal / K\"ahler structure &
  \cref{def:viab-metric} \\
Mass gap $\lambda_1 > 0$ &
  Spectral gap &
  \cref{thm:massgap} \\
Temporal linearity &
  $L < \infty$ (finite volume) &
  \cref{rem:di-temporal} \\
\bottomrule
\end{tabular}
\end{center}

% ═══════════════════════════════════════════════════════════
\section{Face~I: The mean is exact}
\label{sec:alien-hugo}

\begin{quote}
\emph{Hugo Duminil-Copin --- ``For solving longstanding problems in
the probabilistic theory of phase transitions in statistical physics,
especially in dimensions three and four.''}
\end{quote}

\subsection{The random-cluster model}

The \emph{random-cluster model}~\cite{duminilcopin-raoufi-tassion}
on a graph $G = (V, E)$ with parameters $q > 0$ and $p \in [0,1]$
assigns to each edge configuration $\omega \subseteq E$ the weight
\[
  \phi_{q,p}(\omega)
  \;=\;
  \frac{1}{Z}\,
  p^{|\omega|}\,(1-p)^{|E|-|\omega|}\,q^{c(\omega)},
\]
where $c(\omega)$ is the number of connected components and $Z$ is
the normalising constant.  Setting $q = 1$ gives Bernoulli
percolation; $q = 2$ gives the Ising model; integer $q \geq 2$ gives
the $q$-state Potts model via the Edwards--Sokal coupling.

\subsection{Self-duality and the critical point}

On $\mathbb{Z}^2$, the random-cluster model is self-dual: the dual
configuration $\omega^*$ on $(\mathbb{Z}^2)^*$ satisfies
$p^* = q(1-p)/[p + q(1-p)]$.  The \emph{self-dual point} is the
fixed point $p = p^*$:
\[
  \textcolor{sword}{p_{\mathrm{sd}}(q)
  \;=\;
  \frac{\sqrt{q}}{1 + \sqrt{q}}}.
\]
Beffara and Duminil-Copin~\cite{beffara-dc} proved:
$p_c(q) = p_{\mathrm{sd}}(q)$ for all $q \geq 1$.
The critical point IS the self-dual point.

\begin{remark}[Alien translation]\label{rem:alien-selfdual}
$p_c = p_{\mathrm{sd}}$ is the random-cluster model's version of
``the knife is the mean'' (\cref{thm:meanfield}).  The duality
transform exchanges ordered and disordered phases; its fixed point is
the phase boundary.  The threshold is not a property of any individual
configuration---it is the \emph{mean-field quantity} determined by
the global duality structure.
\end{remark}

\subsection{Sharpness}

Duminil-Copin, Raoufi, and
Tassion~\cite{duminilcopin-raoufi-tassion} proved: for all $q \geq 1$
on transitive graphs, the phase transition is \emph{sharp}.  Below
$p_c$, connection probabilities decay exponentially; above $p_c$, the
order parameter is strictly positive.  There is
\textcolor{knife}{no intermediate phase}.

\begin{proposition}[Sharpness $=$ binary lifecycle]
\label{prop:sharpness-lifecycle}
Let $\theta(p) = \Pr_p[0 \leftrightarrow \infty]$.  Then:
\[
  \theta(p) =
  \begin{cases}
    0 & \text{if } p < p_c \quad
    \text{\emph{(\textcolor{knife}{eliminated})}}, \\
    > 0 & \text{if } p > p_c \quad
    \text{\emph{(\textcolor{water}{survives})}}.
  \end{cases}
\]
There is no path~(c) (\cref{thm:lifecycle}): no parameter value at
which $\theta(p)$ is positive but non-macroscopic.
\end{proposition}

\subsection{Triviality in $d = 4$}

Aizenman and Duminil-Copin~\cite{aizenman-dc} proved: in dimension
$d = 4$, the scaling limits of spin fluctuations in the critical Ising
model (and $\lambda\varphi^4$ field) are \emph{Gaussian}.  The
connected four-point function satisfies
\[
  |U_4| \;\leq\; \frac{C}{(\log L)^\gamma}
  \qquad \text{for some } \gamma > 0,
\]
where $L$ is the system size.

The dimension count: two independent random currents (Hausdorff
dimension~2 each) generically intersect in $\mathbb{R}^d$ iff
$2 + 2 \geq d$, i.e., $d \leq 4$.  At $d = 4$ the intersection is
marginal; above $d = 4$ it is transient and mean-field theory becomes
exact.

\begin{theorem}[The mean is exact]\label{thm:alien-triviality}
In $d \geq 4$, the critical exponents of the Ising/$\varphi^4$ model
equal those of the mean-field (Gaussian) model.  The mean field
$\bar{U}$ (\cref{thm:meanfield}) is not an approximation---it is the
exact answer.

Our universe is $\mathbb{R}^3 \times \mathbb{R}^1 = \mathbb{R}^4$,
where $\mathbb{R}^3$ is spatial and $\mathbb{R}^1$ is temporal
linearity (\cref{rem:di-temporal}).  At the upper critical dimension
$d_c = 4$:
\[
  \textcolor{sword}{%
  \text{the knife is the mean}
  \;\;\xrightarrow{\;d = 4\;}\;\;
  \text{theorem, not approximation.}}
\]
\end{theorem}

\begin{remark}\label{rem:triviality-caveat}
The triviality bound $|U_4| \leq C/(\log L)^\gamma$ requires
$L \to \infty$ to force $U_4 \to 0$.  At finite $L$, the connected
four-point function does not vanish: the logarithmic corrections
\emph{persist}.  This fact drives the complexity lock
(\cref{sec:alien-lock}).
\end{remark}

% ═══════════════════════════════════════════════════════════
\section{Face~II: The graph has Hodge structure}
\label{sec:alien-june}

\begin{quote}
\emph{June Huh --- ``For bringing the ideas of Hodge theory to
combinatorics, the proof of the Dowling--Wilson conjecture for
geometric lattices, the proof of the Heron--Rota--Welsh conjecture for
matroids, the development of the theory of Lorentzian polynomials, and
the proof of the strong Mason conjecture.''}
\end{quote}

\subsection{The Chow ring of the execution graph}

Let $G_E = (V, E, c)$ be the execution graph (\cref{def:exgraph}).
Its \emph{cycle matroid} $M(G_E)$ has ground set $E$, independent sets
the forests of $G_E$, and rank function
$\mathrm{rk}(S) = |V| - c(S)$.

Adiprasito, Huh, and Katz~\cite{adiprasito-huh-katz} proved that the
\emph{Chow ring} $A^*(M)$ of any matroid $M$ satisfies the full
\emph{K\"ahler package}:

\begin{enumerate}[label=(\roman*)]
  \item \textbf{Poincar\'e duality.}
  $A^k(M) \times A^{r-k}(M) \to \mathbb{R}$ is a perfect pairing.
  \item \textbf{Hard Lefschetz.}
  $\exists\, \ell \in A^1(M)$ such that
  $\ell^{r-2k}: A^k(M) \xrightarrow{\;\sim\;} A^{r-k}(M)$ for
  $k \leq r/2$.
  \item \textbf{Hodge--Riemann relations.}
  $Q(a,b) = (-1)^k \deg(a \cdot \ell^{r-2k} \cdot b)$ is positive
  definite on the primitive subspace $P^k(M)$.
\end{enumerate}

\begin{remark}[Alien translation]\label{rem:alien-kahler}
The K\"ahler package on $A^*(M(G_E))$ says: the execution graph has
the algebraic structure of a smooth projective variety.  The conformal
metric $g_V = V^{-2}g_S$ (\cref{def:viab-metric}) gives the viability
kernel the geometry of a negatively curved manifold.  The Chow ring
gives the same object a Hodge structure.  These are two descriptions
of the same curvature.
\end{remark}

\subsection{Log-concavity and the Lyapunov function}

The Hodge--Riemann relations force log-concavity of the Whitney
numbers of $M(G_E)$: if $w_k$ is the $k$-th coefficient of the
characteristic polynomial $\chi_M(q)$, then
\[
  w_k^2 \;\geq\; w_{k-1}\,w_{k+1}
  \qquad \text{for all } 0 < k < r.
\]
This is the \emph{Heron--Rota--Welsh conjecture}
(Huh~\cite{huh-chromatic}, Huh--Katz~\cite{huh-katz},
Adiprasito--Huh--Katz~\cite{adiprasito-huh-katz}).

\begin{proposition}[Log-concavity $=$ Lyapunov monotonicity]
\label{prop:logconcave-lyapunov}
Log-concavity $w_k^2 \geq w_{k-1} w_{k+1}$ constrains the viable
configuration count: it cannot oscillate; it must decay monotonically
in the log-convex sense.  This is the combinatorial shadow of
$D^+ V(x)(v) + W(x,v) \leq 0$ (\cref{def:lyapunov}).
\end{proposition}

\subsection{The Tutte polynomial bridge}

The \emph{Tutte polynomial} of $M(G_E)$,
\[
  T_M(x, y)
  \;=\;
  \sum_{A \subseteq E}
  (x-1)^{\mathrm{rk}(E) - \mathrm{rk}(A)}\,
  (y-1)^{|A| - \mathrm{rk}(A)},
\]
specialises to the characteristic polynomial
$\chi_M(q) = (-1)^{\mathrm{rk}(M)} T_M(1-q, 0)$ and to the Potts
partition function
$Z_{\mathrm{Potts}}(G; q, v) \propto T_G(1 + q/v, 1 + v)$.

Br\"and\'en and Huh~\cite{branden-huh} proved: the multivariate Tutte
polynomial is \emph{Lorentzian} for $0 < q \leq 1$.  Its Hessian has
exactly one positive eigenvalue on the positive orthant---the
Hodge--Riemann relation in polynomial language.

\begin{remark}[Bridge: Face~I $\leftrightarrow$ Face~II]
\label{rem:bridge-I-II}
Hugo studies the random-cluster model's phase diagram (the zeros and
singularities of $Z$ in the thermodynamic limit).  June studies the
same object's \emph{coefficients} (their algebraic structure on finite
matroids).  The object is the same:
\textcolor{sword}{the Tutte polynomial of the execution graph.}
Hugo determines \emph{where} the phase transition occurs; June
determines \emph{what algebraic law} the transition obeys.
\end{remark}

\subsection{Discrete Hodge theory on the execution graph}

The graph Laplacian $\Delta$ (\cref{def:laplacian}) is the
$0$-Laplacian of a discrete Hodge theory.  The full theory defines
higher Hodge Laplacians $\Delta_k$ on $k$-cochains of the clique
complex of $G_E$:
\[
  \Delta_k
  \;=\;
  \partial_{k+1}^* \partial_{k+1}
  \;+\;
  \partial_k \partial_k^*,
\]
and the discrete Hodge decomposition:
\[
  C^k
  \;=\;
  \underbrace{\mathrm{im}(\delta_{k-1})}_
    {\textcolor{water}{\text{flows}}}
  \;\oplus\;
  \underbrace{\mathrm{im}(\partial_{k+1}^*)}_
    {\textcolor{knife}{\text{cuts}}}
  \;\oplus\;
  \underbrace{\ker(\Delta_k)}_
    {\textcolor{sword}{\text{harmonic forms}}}.
\]

\begin{proposition}[Hodge $=$ flow-cut duality]
\label{prop:hodge-flowcut}
The discrete Hodge decomposition on $G_E$ completes the
max-flow/min-cut duality (\cref{thm:flowcut}):
\begin{itemize}
  \item \textcolor{water}{Flows} $= \mathrm{im}(\delta_0)$: gradients
  of vertex potentials.
  \item \textcolor{knife}{Cuts} $= \mathrm{im}(\partial_2^*)$:
  edge functions supported on cut-sets.
  \item \textcolor{sword}{Harmonic $1$-forms}
  $= \ker(\Delta_1)$: divergence-free \emph{and} curl-free cycle
  flows.
\end{itemize}
$\dim\ker(\Delta_1) = \beta_1 = |E| - |V| + c$ is the cycle rank:
the number of independent escape routes---the topological obstruction
to eliminating all knives by cutting edges.
\end{proposition}

% ═══════════════════════════════════════════════════════════
\section{Face~III: The phase transition is sharp}
\label{sec:alien-james}

\begin{quote}
\emph{James Maynard --- ``For contributions to analytic number theory,
which have led to major advances in the understanding of the structure
of prime numbers and in Diophantine approximation.''}
\end{quote}

\subsection{The Duffin--Schaeffer theorem}

Let $\psi: \mathbb{N} \to \mathbb{R}_{\geq 0}$.  Define
\[
  A(\psi)
  \;=\;
  \bigl\{\alpha \in [0,1] :
  |\alpha - a/q| \leq \psi(q)/q
  \text{ for infinitely many coprime } (a,q)\bigr\}.
\]

\begin{theorem}[Koukoulopoulos--Maynard {\cite{koukoulopoulos-maynard}}]
\label{thm:duffin-schaeffer}
\[
  \lambda\bigl(A(\psi)\bigr)
  \;=\;
  \begin{cases}
    0 & \text{if } \displaystyle
    \sum_{q=1}^{\infty} \frac{\psi(q)\,\varphi(q)}{q} < \infty, \\[8pt]
    1 & \text{if } \displaystyle
    \sum_{q=1}^{\infty} \frac{\psi(q)\,\varphi(q)}{q} = \infty,
  \end{cases}
\]
where $\lambda$ is Lebesgue measure and $\varphi$ is Euler's totient.
\end{theorem}

This is a \textbf{0--1 law}: $\lambda(A(\psi)) \in \{0, 1\}$
(Gallagher, 1961).  No intermediate measure exists.

\begin{proposition}[Duffin--Schaeffer $=$ binary lifecycle]
\label{prop:ds-lifecycle}
The Duffin--Schaeffer theorem is the binary lifecycle
(\cref{thm:lifecycle}) in number-theoretic form:
\begin{enumerate}[label=(\alph*)]
  \item \textbf{Relinquish} (convergence):
  $\sum \psi(q)\varphi(q)/q < \infty$ $\implies$ $\lambda = 0$.
  The resource cannot sustain independent actuation.
  \item \textbf{Survival} (divergence):
  $\sum \psi(q)\varphi(q)/q = \infty$ $\implies$ $\lambda = 1$.
  The resource persists everywhere.
\end{enumerate}
There is no path~(c): no $\psi$ produces
$0 < \lambda(A(\psi)) < 1$.  This is \cref{thm:fixedpoint} in
measure theory.
\end{proposition}

\subsection{The GCD graph and the Cheeger constant}

The proof constructs a bipartite \emph{GCD graph} $G(N)$ whose
vertices are the active denominators $\{q \leq N : \psi(q) > 0\}$
and whose edges encode correlations via $\gcd(q, r)$.  The proof
proceeds by an \emph{expand-or-compress} dichotomy:

\begin{enumerate}[label=(\roman*)]
  \item $h(G) > 0$ (expander): correlations spread out, second-moment
  method closes, $\lambda(A(\psi)) > 0$.
  \item $h(G) \approx 0$ (bottleneck): a small vertex cut exists; the
  proof applies a density increment, compressing to a denser subgraph.
\end{enumerate}

\begin{remark}[Expand-or-compress $=$ knife-or-not-knife]
\label{rem:alien-gcd}
The dichotomy is the Cheeger inequality (\cref{thm:cheeger}) on the
GCD graph: $h(G) > 0 \Leftrightarrow \lambda_1 > 0 \Leftrightarrow$
mass gap $\Leftrightarrow$ viability (\cref{thm:massgap}).
Expand $=$ viable.  Bottleneck $=$ knife, which the compression step
eliminates.  The iteration terminates because each compression reduces
the graph's complexity: the number of knives is finite, and the binary
lifecycle applies to each.
\end{remark}

\subsection{The sieve as mean field}

The divergence condition $\sum \psi(q)\varphi(q)/q = \infty$ is a
mean-field condition: $\varphi(q)/q$ is the reduced-fraction density,
the arithmetic mean-field correction.

\begin{remark}[The sieve is the detection function]
\label{rem:alien-sieve}
Sieve weights estimate how many elements survive removal of multiples
of small primes.  This is the detection function $\Obs$
(\cref{def:knife}): the sieve observes which elements are composite
(detectable) and which are prime (autonomous).  The sieve level $D$ is
the detection threshold $\tau(\Obs)$ of \cref{thm:meanfield}.
\end{remark}

% ═══════════════════════════════════════════════════════════
\section{Face~IV: The fixed point is self-dual}
\label{sec:alien-maryna}

\begin{quote}
\emph{Maryna Viazovska --- ``For the proof that the E8 lattice
provides the densest packing of identical spheres in 8 dimensions, and
further contributions to related extremal problems and interpolation
problems in Fourier analysis.''}
\end{quote}

\subsection{The $E_8$ lattice and self-duality}

The $E_8$ lattice is even unimodular in $\mathbb{R}^8$: every inner
product is an integer, every norm is even, and $\det(\mathrm{Gram}) =
1$.  Consequently,
\[
  \textcolor{sword}{E_8 \;=\; E_8^*}
  \qquad \text{(the lattice equals its dual).}
\]

Viazovska~\cite{viazovska} proved: $E_8$ is the densest sphere packing
in $\mathbb{R}^8$, with density $\pi^4/384$ and kissing number~$240$.
The proof constructs an auxiliary function $f$ (the ``magic function'')
that saturates the Cohn--Elkies linear programming bound.

\subsection{The magic function and LP saturation}

The Cohn--Elkies bound~\cite{cohn-elkies}: if $f: \mathbb{R}^8 \to
\mathbb{R}$ is a radial Schwartz function with $f(0) = \hat{f}(0)$,
$f(x) \leq 0$ for $|x| \geq \sqrt{2}$, and $\hat{f}(t) \geq 0$ for
all $t$, then $\delta \leq \pi^4/384$.

Viazovska's construction uses weakly holomorphic modular forms for
$\Gamma(1) = \mathrm{SL}_2(\mathbb{Z})$ and $\Gamma_0(2)$:
\begin{enumerate}[label=(\roman*)]
  \item $f$ vanishes at all nonzero $E_8$ lattice points.
  \item $\hat{f}$ vanishes at all nonzero $E_8^* = E_8$ points.
  \item Both Poisson summation inequalities become equalities.
\end{enumerate}

Self-duality $E_8 = E_8^*$ makes (i) and (ii) the \emph{same
condition}.  Two constraints collapse to one.

\begin{proposition}[Self-duality $=$ the knife is the mean]
\label{prop:selfdual-knife}
\leavevmode
\begin{enumerate}[label=(\roman*)]
  \item \textbf{Duality as identity.}
  Max-flow $=$ min-cut (\cref{thm:flowcut}).  Poisson summation
  equates the lattice sum and the dual-lattice sum.  Self-duality
  collapses both:
  \[
    \sum_{x \in E_8} f(x)
    \;=\;
    \sum_{y \in E_8^*} \hat{f}(y)
    \;=\;
    \sum_{y \in E_8} \hat{f}(y).
  \]
  The duality is an equality, not an inequality.

  \item \textbf{Saturation as viability.}
  LP saturation means the theoretical bound is achieved: the viable
  path to infinity \emph{exists} (\cref{sec:axiom}).  The magic
  function is the constructive witness.

  \item \textbf{Uniqueness as no path~(c).}
  $E_8$ is the unique optimal periodic packing in $\mathbb{R}^8$.  No
  competing configuration exists (\cref{thm:fixedpoint}).
\end{enumerate}
\end{proposition}

\subsection{Universal optimality}

Cohn, Kumar, Miller, Radchenko, and
Viazovska~\cite{cohn-kumar-miller-radchenko-viazovska} proved: $E_8$
and the Leech lattice minimise energy
$E_p(\mathcal{C}) = \sum_{x \neq y} p(|x-y|^2)$ for \emph{every}
completely monotonic potential $p$.

\begin{remark}[Universal optimality $=$ universality of the viability
axiom]\label{rem:alien-universal}
The viability axiom does not depend on the Lyapunov function's
specific form---only on the existence of a viable path.  Universal
optimality says the same: $E_8$ is optimal for all completely monotonic
potentials.  The structure (self-duality) determines the outcome, not
the interaction.  \textcolor{sword}{The knife is the mean---the
structure, not the content.}
\end{remark}

\subsection{Bridge: Face~I $\leftrightarrow$ Face~IV}

$p_c = p_{\mathrm{sd}}$ (Hugo) and $E_8 = E_8^*$ (Maryna) are the
same principle: \textcolor{sword}{the optimal configuration is the
fixed point of the duality transform.}  In the random-cluster model,
duality exchanges phases; the fixed point is the phase boundary.  In
sphere packing, Fourier duality exchanges space and frequency; the
fixed point saturates the LP bound.  In the agentic framework,
max-flow/min-cut duality exchanges flows and cuts; the fixed point is
where the knife equals the mean.

% ═══════════════════════════════════════════════════════════
\section{Convergence: one mountain, four sides}
\label{sec:alien-convergence}

\begin{center}
\small
\begin{tabular}{@{}lcccc@{}}
\toprule
& \textbf{Hugo} & \textbf{June} & \textbf{James}
& \textbf{Maryna} \\
\midrule
Object &
  random-cluster &
  Chow ring &
  GCD graph &
  $E_8$ lattice \\
Language &
  probability &
  algebra &
  number theory &
  analysis \\
Tool &
  parafermions &
  K\"ahler package &
  sieve \& circle &
  modular forms \\
\midrule
\textcolor{knife}{Binary lifecycle} &
  sharpness &
  --- &
  0--1 law &
  --- \\
\textcolor{sword}{Knife $=$ mean} &
  $p_c = p_{\mathrm{sd}}$ &
  Poincar\'e duality &
  --- &
  $E_8 = E_8^*$ \\
\textcolor{water}{Mean exact} &
  $d \geq 4$ triviality &
  K\"ahler package &
  --- &
  LP saturation \\
No path (c) &
  --- &
  --- &
  no intermediate $\lambda$ &
  unique packing \\
\midrule
Bridge &
  \multicolumn{2}{c}{Tutte polynomial}
  & Cheeger constant
  & Fourier duality \\
\bottomrule
\end{tabular}
\end{center}

\begin{theorem}[One mountain]\label{thm:alien-mountain}
Under the dictionary of \cref{sec:alien-dict}:
\begin{enumerate}[label=(\roman*)]
  \item \textbf{Hugo:} The mean field is exact in $d \geq 4$.
  ``The knife is the mean'' is a theorem.
  \item \textbf{June:} The execution graph carries a Hodge structure.
  Max-flow/min-cut extends to a Hodge decomposition.
  \item \textbf{James:} The phase transition admits no intermediate
  phase.  The binary lifecycle is a 0--1 law.
  \item \textbf{Maryna:} The optimal configuration is the self-dual
  fixed point.  The viable path exists, is unique, and is universally
  optimal.
\end{enumerate}
Together: \textcolor{sword}{the knife is the mean, exactly, sharply,
algebraically, and uniquely.}
\end{theorem}

% ═══════════════════════════════════════════════════════════
\section{The complexity lock}\label{sec:alien-lock}

The four faces close a trap.

\subsection{The alien sees P\,$=$\,NP}

The alien model is a mean-field theory (\cref{thm:meanfield}).  Hugo
proved: at $d = 4$, mean-field is exact
(\cref{thm:alien-triviality}).  A Gaussian (mean-field) system has
factorising correlations---every local observable decomposes into
independent components.  In computational language: the problem
decomposes.  Decomposable problems are in~$\mathrm{P}$.

From the alien's perspective, every problem is in~$\mathrm{P}$.
The alien sees $\mathrm{P} = \mathrm{NP}$.

\subsection{The insider sees P\,$\neq$\,NP}

The triviality bound $|U_4| \leq C/(\log L)^\gamma$
(\cref{rem:triviality-caveat}) requires $L \to \infty$.  At finite
$L$, the connected four-point function does not vanish.  The
non-Gaussian correlations persist.

Temporal linearity (\cref{rem:di-temporal}) enforces $L < \infty$:
the king evaluates each state transition in real, non-pausable time.
The system size $L$ of any computation performed under temporal
linearity is bounded by the time budget $T$:
\[
  L \;\leq\; T \;<\; \infty.
\]
The thermodynamic limit $L \to \infty$ is \emph{forbidden}.  Therefore
Hugo's triviality does not hold for finite computations: the
logarithmic corrections persist, the correlations do not factorise,
and the problem does not decompose.

\subsection{The lock}

\begin{theorem}[Complexity lock]\label{thm:complexity-lock}
The following four statements, proved independently by the four Fields
Medalists, jointly lock the $\mathrm{P}/\mathrm{NP}$ boundary:
\begin{enumerate}[label=(\roman*)]
  \item \textbf{Hugo (triviality):}
  Mean-field is exact at $d = 4$, but only in the limit $L \to
  \infty$.  At finite $L$:
  $|U_4| \leq C/(\log L)^\gamma \neq 0$.

  \item \textbf{Temporal linearity (axiom):}
  $L \leq T < \infty$.  The thermodynamic limit is forbidden.

  \item \textbf{James (0--1 law):}
  The measure of the solvable set is $0$ or $1$---no intermediate
  phase.  Combined with~(i) and~(ii): for finite $L$, the
  non-Gaussian corrections force a \emph{sharp} separation between
  problems that decompose ($\mathrm{P}$) and problems that do not
  ($\mathrm{NP}$-hard).

  \item \textbf{Maryna (uniqueness):}
  The self-dual fixed point is unique.  There is no third
  configuration between decomposable and non-decomposable---no
  path~(c).
\end{enumerate}
\end{theorem}

The trap is: Hugo proves $\mathrm{P} = \mathrm{NP}$ at $L = \infty$.
Temporal linearity forbids $L = \infty$.  James proves the boundary is
sharp.  Maryna proves the boundary is unique.

\begin{remark}[The Razborov--Rudich confirmation]
\label{rem:razborov-rudich}
Razborov and Rudich~\cite{razborov-rudich} proved: no ``natural''
proof strategy---one that uses properties computable in polynomial
time and satisfied by a random function with nonnegligible
probability---can prove $\mathrm{P} \neq \mathrm{NP}$ (assuming
one-way functions exist).

The alien model is a mean-field theory.  Mean-field properties are
natural in the Razborov--Rudich sense: they are polynomial-time
computable (the Gaussian is efficiently sampleable) and satisfied by
random functions (the central limit theorem).  Therefore:

\textcolor{knife}{The alien model cannot prove
$\mathrm{P} \neq \mathrm{NP}$.}

This is not a failure.  It is the framework predicting its own boundary
of validity (\cref{sec:domain}).  The alien sees
$\mathrm{P} = \mathrm{NP}$ because the alien IS the mean field; the
mean field cannot see its own corrections.  The proof that
$\mathrm{P} \neq \mathrm{NP}$, if it exists, must come from
\emph{inside}---from the non-Gaussian structure that persists at
finite~$L$.
\end{remark}

\subsection{Time is the only limited resource}

The complexity lock rests on a single axiom: temporal linearity.  Time
is finite, real, non-pausable, non-reversible.  It is the only truly
limited resource in the universe.

Space can be reused.  Energy can be converted.  Information can be
copied.  But time, once spent, is gone.  Under temporal linearity:
\begin{itemize}
  \item $\mathrm{PSPACE}$-complete problems require a viable path of
  length $\geq 2^{\mathrm{poly}(n)}$ through the execution graph.
  The Cheeger constant of this graph is
  $h(G) \leq 2^{-\mathrm{poly}(n)}$: the min-cut is exponentially
  thin.  The mass gap
  $\lambda_1 \leq 2h(G) \leq 2^{1-\mathrm{poly}(n)}$ vanishes.
  By \cref{thm:massgap}: no viable path.
  \item $\mathrm{NP}$-hard problems (under standard assumptions) have
  $h(G)$ that vanishes super-polynomially.  The spectral gap closes.
  The non-Gaussian corrections $|U_4| \sim 1/(\log L)^\gamma$
  accumulate over the path and prevent factorisation.
  \item $\mathrm{P}$ problems have $h(G) \geq 1/\mathrm{poly}(n)$:
  the Cheeger constant is polynomially bounded below.  The spectral
  gap is open.  The mass gap holds.  The viable path exists.
\end{itemize}

\begin{remark}[The hypothesis]\label{rem:time-hypothesis}
The complexity lock (\cref{thm:complexity-lock}) is not a proof of
$\mathrm{P} \neq \mathrm{NP}$.  It is a structural prediction: the
$\mathrm{P}/\mathrm{NP}$ boundary is \emph{the same phase transition}
that Hugo, James, June, and Maryna studied, locked into computational
form by temporal linearity.  The mean-field model (alien) sees one
side.  The finite-time insider sees the other.  Both are correct.
The boundary between them is sharp (James), algebraically structured
(June), and unique (Maryna).  It can be crossed only by taking $L \to
\infty$---which temporal linearity forbids.

\medskip
\noindent
\textcolor{sword}{%
有志者事竟成,\textcolor{knife}{破釜沉舟}百二秦关终属楚;\\
苦心人天不负,\textcolor{water}{卧薪尝胆}三千越甲可吞吴。}

\medskip
\noindent
Time is finite.  That is the only axiom.
\end{remark}

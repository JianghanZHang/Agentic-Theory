\chapter{第二性 --- The Other as Phase Function}\label{app:secondsex}

Simone de Beauvoir's thesis in \emph{Le Deuxi\`eme Sexe}~\cite{beauvoir}
can be stated in three sentences.
One is not born a woman; one becomes one.
The category ``woman'' is not a biological fact but a relational
position: the \emph{Other} defined against a \emph{Subject}.
The boundary between Subject and Other is not intrinsic---it is a
phase function, a mean-field threshold identical in structure to
\cref{thm:meanfield}.
The formal treatment---the dual tower $\mathbf{L}^*$ and the
representation theorems---is in \cref{sec:representation}.
This appendix provides the historical illustrations.

\section{蔡文姬}

蔡文姬 (Cai Wenji, c.\ 177--250~CE) was the daughter of 蔡邕 (Cai Yong),
one of the greatest scholars of the Eastern Han.
She was captured by the 匈奴 (Xiongnu) during the chaos following
Dong Zhuo's destruction of Luoyang, and lived among them for twelve
years, bearing two sons to the Xiongnu chieftain 左贤王.
In 208~CE, 曹操 (Cao Cao)---who had studied under 蔡邕---ransomed her
back to the Han court.
She was forced to leave her sons behind and was remarried to 董祀
(Dong Si).

At every stage of her life, 蔡文姬's identity is defined relationally:
she is 蔡邕's daughter, 左贤王's captive wife, 曹操's cultural
project, 董祀's wife.
The Subject changes; she remains the Other.
Her own voice---her literary genius---exists in the gap.

\section{The mapping}

\begin{center}
\renewcommand{\arraystretch}{1.25}
\begin{tabular}{@{}lll@{}}
\toprule
\textbf{Beauvoir} & \textbf{蔡文姬} & \textbf{胡笳十八拍} \\
\midrule
他者 (Other)       & defined relationally          & depicted as object \\
内在性 (Immanence) & captivity, body               & content: grief \\
超越性 (Transcendence) & literary genius            & the poem itself \\
主体 (Subject)     & 蔡邕\,/\,左贤王\,/\,曹操      & painter\,/\,viewer \\
永恒女性 (Myth)    & 才女 trope                     & scroll paintings \\
Sword $=$ mean     & exchange threshold             & Subject/Other boundary \\
\bottomrule
\end{tabular}
\end{center}

\section{The eighteen beats}

胡笳十八拍 (Eighteen Songs of a Nomad Flute)~\cite{liushang} is
traditionally attributed to 蔡文姬, though the text was composed by
Liu Shang (刘商) in the Tang dynasty (c.~773~CE), writing in her voice.
Each 拍 (beat) is a poem-song documenting a stage of her captivity
and return.
\Cref{fig:eighteen} traces the narrative arc: the fraction of each
beat devoted to transcendence (voice, creation, agency) versus
immanence (being acted upon, grief, objecthood).

\begin{figure}[H]
\centering
\begin{tikzpicture}[scale=0.85]
  % ── Data: transcendence fractions ──
  % 拍: 1     2     3     4     5     6     7     8     9
  %     0.20  0.10  0.05  0.15  0.25  0.35  0.30  0.60  0.55
  % 拍: 10    11    12    13    14    15    16    17    18
  %     0.40  0.50  0.15  0.30  0.40  0.35  0.20  0.45  0.70

  \def\barw{0.55}
  \def\barh{5.0}
  \def\gap{0.22}
  \def\tvals{{0.20, 0.10, 0.05, 0.15, 0.25, 0.35,
              0.30, 0.60, 0.55, 0.40, 0.50, 0.15,
              0.30, 0.40, 0.35, 0.20, 0.45, 0.70}}

  % ── Y-axis labels ──
  \node[rotate=90, anchor=south, font=\small] at (-0.9, \barh)
    {超越 (transcendence)};
  \node[rotate=90, anchor=north, font=\small] at (-0.9, 0)
    {内在 (immanence)};

  % ── Draw 18 bars ──
  \foreach \i in {0,...,17} {
    \pgfmathsetmacro{\x}{\i*(\barw+\gap)}
    \pgfmathsetmacro{\tv}{\tvals[\i]}
    \pgfmathsetmacro{\splitY}{\tv*\barh}
    % Bottom: immanence (dao colour)
    \fill[dao!20] (\x, 0) rectangle (\x+\barw, \splitY);
    % Top: transcendence (water colour)
    \fill[water!20] (\x, \splitY) rectangle (\x+\barw, \barh);
    % Border
    \draw[black!40, thin] (\x, 0) rectangle (\x+\barw, \barh);
    % Split line
    \draw[black!60, thin] (\x, \splitY) -- (\x+\barw, \splitY);
    % 拍 number below
    \pgfmathtruncatemacro{\paiNum}{\i+1}
    \node[below, font=\tiny] at (\x+\barw/2, 0) {\paiNum};
  }

  % ── Melody curve through split points ──
  \draw[black!70, very thick, smooth, tension=0.5]
    plot coordinates {
      ({0*(\barw+\gap)+\barw/2},  {0.20*\barh})
      ({1*(\barw+\gap)+\barw/2},  {0.10*\barh})
      ({2*(\barw+\gap)+\barw/2},  {0.05*\barh})
      ({3*(\barw+\gap)+\barw/2},  {0.15*\barh})
      ({4*(\barw+\gap)+\barw/2},  {0.25*\barh})
      ({5*(\barw+\gap)+\barw/2},  {0.35*\barh})
      ({6*(\barw+\gap)+\barw/2},  {0.30*\barh})
      ({7*(\barw+\gap)+\barw/2},  {0.60*\barh})
      ({8*(\barw+\gap)+\barw/2},  {0.55*\barh})
      ({9*(\barw+\gap)+\barw/2},  {0.40*\barh})
      ({10*(\barw+\gap)+\barw/2}, {0.50*\barh})
      ({11*(\barw+\gap)+\barw/2}, {0.15*\barh})
      ({12*(\barw+\gap)+\barw/2}, {0.30*\barh})
      ({13*(\barw+\gap)+\barw/2}, {0.40*\barh})
      ({14*(\barw+\gap)+\barw/2}, {0.35*\barh})
      ({15*(\barw+\gap)+\barw/2}, {0.20*\barh})
      ({16*(\barw+\gap)+\barw/2}, {0.45*\barh})
      ({17*(\barw+\gap)+\barw/2}, {0.70*\barh})
    };

  % ── Annotations above key 拍 ──
  \node[above, font=\scriptsize, dao] at
    ({2*(\barw+\gap)+\barw/2}, \barh) {掠};
  \node[above, font=\scriptsize, water] at
    ({7*(\barw+\gap)+\barw/2}, \barh) {闻笳};
  \node[above, font=\scriptsize, dao] at
    ({11*(\barw+\gap)+\barw/2}, \barh) {别子};
  \node[above, font=\scriptsize, caution] at
    ({15*(\barw+\gap)+\barw/2}, \barh) {嫁};
  \node[above, font=\scriptsize, water] at
    ({17*(\barw+\gap)+\barw/2}, \barh) {恨};

  % ── Phase braces below ──
  % Phase I: 离 (拍 1--3, indices 0--2)
  \draw[decorate, decoration={brace, mirror, amplitude=5pt}]
    ({0*(\barw+\gap)}, -0.5) -- ({2*(\barw+\gap)+\barw}, -0.5)
    node[midway, below=6pt, font=\small] {I\;离};
  % Phase II: 居 (拍 4--12, indices 3--11)
  \draw[decorate, decoration={brace, mirror, amplitude=5pt}]
    ({3*(\barw+\gap)}, -0.5) -- ({11*(\barw+\gap)+\barw}, -0.5)
    node[midway, below=6pt, font=\small] {II\;居};
  % Phase III: 归 (拍 13--18, indices 12--17)
  \draw[decorate, decoration={brace, mirror, amplitude=5pt}]
    ({12*(\barw+\gap)}, -0.5) -- ({17*(\barw+\gap)+\barw}, -0.5)
    node[midway, below=6pt, font=\small] {III\;归};
\end{tikzpicture}
\caption{The eighteen beats as narrative arc.
  Bottom (\textcolor{dao}{red}): immanence.
  Top (\textcolor{water}{blue}): transcendence.
  The melody line traces the Subject/Other boundary---the sword as
  phase function (\cref{thm:meanfield}).}
\label{fig:eighteen}
\end{figure}

\section{The cycle}

The narrative arc reveals a cycle:
Other $\to$ Voice $\to$ Re-objectification $\to$ Myth $\to$ Other.

拍~3 (掠, the capture) is the nadir: transcendence fraction 0.05.
蔡文姬 is pure object, pure immanence---a body seized as war spoil.

拍~8 (闻笳, hearing the nomad flute) is the inflection point:
transcendence fraction 0.60.
The 胡笳 sound triggers memory and creation; for the first time, the
Other speaks \emph{as} Subject.

拍~12 (别子, farewell to sons) is a collapse: transcendence fraction
0.15.
The voice that rose in 拍~8 is crushed by the biological fact of
motherhood weaponised as immanence.
She must leave her children to return to a court that values her as
蔡邕's daughter, not as herself.

拍~16 (嫁, remarriage to 董祀) is the second exchange: transcendence
fraction 0.20.
曹操's ``rescue'' completes the circuit: she is transferred from one
Subject (左贤王) to another (董祀), with 曹操 as broker.
The rescue \emph{is} the second capture.

拍~18 (恨, the final beat) resolves at transcendence fraction 0.70---the
highest in the entire poem.
The voice persists.
The poem outlives every Subject who defined her.

\section{花木兰 --- the metastable crossing}

The \emph{木兰辞} (Ballad of Mulan)~\cite{mulanshi}, a Northern-Dynasty
folk ballad, appears to contradict the pattern.
花木兰 (Hua Mulan) replaces her father in the army, serves twelve
years, declines high office, and returns home.
Unlike 蔡文姬, she \emph{crosses} the Subject/Other boundary:
she commands troops, earns merit, is offered the rank of 尚书郎.

But the crossing is conditional.
It requires total erasure of female identity
(``双兔傍地走,安能辨我是雄雌''---when two rabbits run side by side,
who can tell male from female?).
Her Subject-hood is not hers; it is the male performance she
sustains for twelve years.

The moment the mean field stabilises---war ends, peace
returns---the boundary reasserts:
``脱我战时袍,著我旧时裳。当窗理云鬓,对镜帖花黄。''
She removes the war robe, puts on the old clothes, arranges her hair,
applies the forehead ornament.
The poem presents this as free choice (``木兰不用尚书郎''),
but structurally it is the phase function restoring equilibrium.

蔡文姬 and 花木兰 are complementary probes of the same boundary.
蔡文姬 is subcritical: the sword never breaks; transcendence exists
only in the gap (the poem, the voice).
花木兰 is a supercritical fluctuation: the sword breaks under
perturbation (war), but the system anneals back when the
perturbation ends.
Neither literary genius nor military prowess shifts the attractor.

\section{The second sex is the mean}

The 18th~拍 does not liberate 蔡文姬.
曹操's project was cultural recovery, not emancipation.
The scroll paintings that follow~\cite{rorex}---depicting her as the
archetype of the 才女 (talented woman)---complete the mythification
that Beauvoir identifies as the final mechanism of Othering:
the \emph{eternal feminine} absorbs the individual voice into a trope.

But the structure is identical to \cref{thm:meanfield}.
The boundary between Subject and Other is not a property of 蔡文姬 or
花木兰; it is a phase function of the system's mean field.
When the mean actuation level shifts (war $\to$ peace, capture $\to$
ransom), the \emph{same person} crosses the threshold---or crosses
back.

The second sex \emph{is} the mean.

\chapter{中县干部 --- The Graded Lattice}\label{app:zhongxian}

The preceding appendices applied the framework to individual agents
and historical cases
(\cref{app:sources,app:secondsex,app:threeli})
and to the system itself as a cooperative triad
(\cref{app:triad}).  This appendix applies it to the \emph{internal
structure} of a viability maintenance system: a county-level
bureaucratic hierarchy observed from within, with exact counts.
The mathematical object is a \emph{graded lattice}---a partially
ordered set with rank function, filtration rates, and absorbing
boundaries---and the dynamics are a \emph{dual-ring differential
inclusion} in which the formal institutional rules (inner ring)
define the viability kernel and the relationship network (outer
ring) defines the set-valued dynamics.  The central result is that
the knife-is-the-mean theorem (\cref{thm:meanfield}) instantiates
at the level of a single county: the critical threshold separating
viable cadres from non-viable ones is the mean guanxi level, and
every pathology documented in the source---fake achievements, vote
canvassing, political families, corruption---is a consequence of
viability maintenance under this mean field.

The empirical basis is Feng Junqi's doctoral dissertation
\emph{Zhong Xian Ganbu} \cite{zhongxian}, a participant-observation
study of a county in Henan province (population ${\sim}800{,}000$)
conducted during 2008--2010, in which Feng held a temporary
government post and systematically documented the composition,
promotion trajectories, relationship networks, and disciplinary
outcomes of the county's ${\sim}1{,}000$ cadres at 副科级 and above.

% ================================================================
\section{The cadre lattice}\label{sec:zx-lattice}
% ================================================================

\begin{definition}[Graded cadre lattice]\label{def:cadre-lattice}
A \emph{graded cadre lattice} is a finite poset
$\Lambda = \bigsqcup_{k=0}^{K} \Lambda_k$ with rank function
$\rho \colon \Lambda \to \{0, 1, \ldots, K\}$, equipped with:
\begin{enumerate}[label=(\roman*)]
  \item \textbf{Level sizes} $n_k := |\Lambda_k|$;
  \item \textbf{Filtration ratios}
    $r_k := n_{k+1}/n_k$ for $0 \leq k < K$, measuring
    the structural selectivity between adjacent levels;
  \item \textbf{Time constants} $\tau_k > 0$: the mean duration
    at level $\Lambda_k$ before promotion to $\Lambda_{k+1}$
    (averaged over those who are promoted);
  \item \textbf{Age boundaries} $a_k^{\max}$: hard upper bounds
    on the age at which a cadre in $\Lambda_k$ remains eligible
    for promotion to $\Lambda_{k+1}$.
\end{enumerate}
\end{definition}

\begin{example}[Zhong County]\label{ex:zx-data}
The Zhong County cadre system has $K = 3$ levels:

\begin{center}
\renewcommand{\arraystretch}{1.25}
\begin{tabular}{@{}clrrrl@{}}
\toprule
$k$ & \textbf{Level} (级别) & $n_k$ & $r_k$ & $\tau_k$ (yr)
    & $a_{k}^{\max}$ \\
\midrule
$0$ & 副科级 (deputy section) & 680 & $0.41$ & $2.7$ & ${\sim}45$ \\
$1$ & 正科级 (full section)   & 280 & $0.14$ & $7.3$ & ${\sim}50$ \\
$2$ & 副处级 (deputy county)  &  40 & $0.13$ & $7.5$ & ${\sim}55$ \\
$3$ & 正处级 (full county)    &   5 & ---    & ---   & ${\sim}60$ \\
\bottomrule
\end{tabular}
\end{center}

\noindent
The entry time (from first assignment to $\Lambda_0$) averages $8$ years.
A cadre entering at age $22$ therefore reaches $\Lambda_0$ at
${\sim}30$, $\Lambda_1$ at ${\sim}33$, $\Lambda_2$ at ${\sim}40$,
$\Lambda_3$ at ${\sim}48$.  The margin between this ideal trajectory
and the age boundary at $\Lambda_3$ is ${\sim}7$ years---thin enough
that a single delay of one promotion cycle can be fatal.
The lattice has additional substructure: each level contains
\emph{hidden steps} (隐形台阶), e.g.\ within $\Lambda_2$ the
progression 县委常委 $\to$ 常务副县长 $\to$ 县长 $\to$ 县委书记,
and a mandatory \emph{government--party spiral}
(政--党螺旋晋升模式) that alternates between government and
party positions.
\end{example}

\begin{proposition}[Lattice as viability constraint]
\label{prop:pyramid-viability}
The filtration ratios satisfy $r_k \ll 1$ for $k \geq 1$
and $r_0 < 1$.  The cumulative survival ratio from $\Lambda_0$ to
$\Lambda_3$ is
\[
  p_{\textup{top}}
  \;=\; \prod_{k=0}^{K-1} r_k
  \;=\; 0.41 \times 0.14 \times 0.13
  \;\approx\; 0.0074.
\]
Fewer than $1\%$ of cadres at $\Lambda_0$ will occupy $\Lambda_3$ in their
career.  The age boundaries impose a hard absorbing face: in the
viability framework (\cref{thm:viability-di}), the constraint set is
\begin{equation}\label{eq:cadre-K}
  K_{\textup{cadre}}
  \;=\;
  \bigl\{\,(k,\, a) \;:\; 0 \leq k \leq K,\;\;
  a \leq a_k^{\max}\,\bigr\},
\end{equation}
and the face $a = a_k^{\max}$ is absorbing: once crossed,
no trajectory can re-enter $K_{\textup{cadre}}$ at level~$k{+}1$.
\end{proposition}

\begin{remark}[一刀切 and temporal linearity]\label{rem:yidaoqie}
The bureaucratic term 一刀切 (``one cut of the knife'') for
age-based cutoffs is a literal knife in the sense of
\cref{def:knife}: the age variable has autonomous dynamics
($a' = 1$, always advancing) and is fully observable.  It
satisfies both conditions of the knife definition.  Unlike other
knives in the framework, this one is not a phase function of the
mean field---it is an exogenous, non-negotiable constraint imposed
by temporal linearity itself.  Time is the knife that needs no
actuation, no observability adjustment, and no political
manoeuvring.
\end{remark}

% ================================================================
\section{The dual-ring dynamics}\label{sec:zx-dualring}
% ================================================================

Feng's central theoretical contribution is the \emph{dual-ring
model} (双环模型): the formal institutional system (inner ring)
and the relationship system (outer ring) are coupled, with the
outer ring driving the inner ring.  We formalise this as a coupled
differential inclusion.

\begin{definition}[Dual-ring system]\label{def:dual-ring}
A \emph{dual-ring system} is a pair of coupled dynamics on a
state space $S = S_I \times S_G$:
\begin{align}
  x_I'(t) &\in F_I\bigl(x_I(t),\, x_G(t)\bigr),
    \label{eq:inner-ring} \\
  x_G'(t) &\in F_G\bigl(x_I(t),\, x_G(t)\bigr),
    \label{eq:outer-ring}
\end{align}
where:
\begin{enumerate}[label=(\roman*)]
  \item $x_I \in S_I$ is the \emph{institutional state}
    (formal rank, position, evaluation scores);
  \item $x_G \in S_G$ is the \emph{guanxi state}
    (relationship network, accumulated favours, reputation);
  \item $F_I$ is the formal dynamics (promotion rules,
    performance evaluation, term limits);
  \item $F_G$ is the relationship dynamics (alliance formation,
    gift exchange, vote canvassing);
  \item The coupling is \emph{asymmetric}: $F_I$ depends strongly
    on $x_G$ (relationships determine promotion outcomes),
    while $F_G$ depends only weakly on $x_I$ (formal position
    provides a platform for relationship building, but
    relationships persist across positions).
\end{enumerate}
The viability kernel is defined by the inner ring alone:
\begin{equation}\label{eq:dual-K}
  K
  \;=\;
  \bigl\{\,(x_I,\, x_G) \;:\; x_I \in K_{\textup{cadre}}\,\bigr\},
\end{equation}
but the dynamics that must satisfy the tangential condition are
driven by the outer ring.
\end{definition}

\begin{proposition}[The outer ring drives viability]
\label{prop:outer-drives}
In the dual-ring system, the tangential condition
$F(x) \cap T_K(x) \neq \varnothing$
(\cref{thm:viability-di}) at the boundary of $K$
reduces to a condition on the guanxi state.  Write the
institutional dynamics as
$f_I(x_I, x_G) = f_I^{0}(x_I) + g(x_G)$,
where $f_I^{0}$ is the autonomous formal dynamics
(evaluation absent relationships) and $g(x_G)$ is the
guanxi-driven correction.  At the boundary face
$x_I \in \partial K_{\textup{cadre}}$ (a cadre facing the
age cutoff or a competitive promotion slot), the tangential
condition requires
\begin{equation}\label{eq:guanxi-tangential}
  f_I^{0}(x_I) + g(x_G) \;\in\; T_{K_{\textup{cadre}}}(x_I).
\end{equation}
The autonomous term $f_I^{0}$ is generically insufficient:
the filtration ratio $r_k < 1$ means that formal qualifications
alone do not guarantee promotion.  The tangential condition
therefore requires the guanxi correction $g(x_G)$ to compensate.
\end{proposition}

\begin{proof}
Since $K$ is defined by $x_I \in K_{\textup{cadre}}$ with
$x_G$ unconstrained, the contingent cone at
$(x_I, x_G)$ with $x_I \in \partial K_{\textup{cadre}}$ is
$T_K(x) = T_{K_{\textup{cadre}}}(x_I) \times S_G$
(\cref{def:contingent}).  The tangential condition
$f(x) \in T_K(x)$ projects onto
$f_I(x_I, x_G) \in T_{K_{\textup{cadre}}}(x_I)$,
which is \eqref{eq:guanxi-tangential}.  That $f_I^{0}$ is
generically insufficient is the empirical content of the
filtration ratio $r_k < 1$: in steady state, more cadres
satisfy the formal criteria than there are promotion slots,
so the formal autonomous dynamics point outward
($f_I^{0} \notin T_{K_{\textup{cadre}}}$) for the majority.
The guanxi correction $g(x_G)$ is what bends the trajectory
back into the viable cone.
\end{proof}

\begin{remark}[The knife is the mean guanxi level]
\label{rem:guanxi-knife}
By \cref{thm:meanfield}, the knife threshold is determined
by the mean actuation field~$\bar{U}$.  In the dual-ring
system, the relevant actuation is the guanxi investment: the
total relationship resources (time, money, favours) a cadre
deploys.  Feng documents that a township party secretary
spends 300{,}000--500{,}000~yuan per year on vote canvassing
alone.  The mean guanxi level $\bar{U}_G$ sets the threshold:
a cadre whose guanxi investment falls below $\bar{U}_G$
becomes non-viable regardless of formal qualifications.
This is the knife-is-the-mean theorem at county scale.

Feng quotes a county saying that encodes this:
\begin{center}
\emph{年龄是个宝,能力作参考,关系最重要。}\\[3pt]
{\small (Age is a treasure, ability a reference,
relationships paramount.)}
\end{center}
The three factors are ordered by formalisability: age is
exogenous and fully observable (the time-knife of
\cref{rem:yidaoqie}); ability is partially observable and
partially fabricable (政绩, which the dissertation documents
can be faked); relationships are the actual dynamics that
drive the tangential condition \eqref{eq:guanxi-tangential}.

The dual-ring model also explains the Chinese institutional
paradox: why does increasing formalisation of rules
(制度化) coexist with increasing importance of relationships
(关系)?  The answer is structural.  Each new formal rule
tightens the viability kernel $K_{\textup{cadre}}$---adds
a constraint surface.  To satisfy the tangential condition at
the new boundary, cadres must increase their guanxi
investment $g(x_G)$.  Formalisation does not replace
relationships; it \emph{amplifies} the demand for them.
The inner ring constricts; the outer ring accelerates.
\end{remark}

% ================================================================
\section{Flow concentration: cradles and secretaries}
\label{sec:zx-cradle}
% ================================================================

Certain institutions in the county function as ``cradles''
(摇篮) that disproportionately produce promoted cadres.
This is a flow concentration phenomenon, interpretable as a
min-cut in the promotion graph.

\begin{definition}[Cradle node]\label{def:cradle}
In the execution graph $G = (V, E)$ (\cref{def:exgraph})
of the cadre lattice, a \emph{cradle node} $v \in V$ is a
node with anomalously high conditional promotion rate:
if $p_v$ is the probability that a cadre who passes through
$v$ reaches $\Lambda_{k+1}$, then $v$ is a cradle if
$p_v / r_k > c$ for a threshold $c > 1$.
The principle 高进高出 (``high entry, high exit'') describes
the mechanism: cradle institutions attract already-promising
cadres and amplify their promotion prospects.
\end{definition}

\begin{example}[Zhong County cradles]\label{ex:cradles}
\hfill

\begin{center}
\renewcommand{\arraystretch}{1.25}
\begin{tabular}{@{}lrl@{}}
\toprule
\textbf{Institution} & \textbf{\% of county leaders}
    & \textbf{Mechanism} \\
\midrule
乡镇 (township)       & 59\% & Frontline testing \\
两办 (two offices)     & 22\% & Proximity to power \\
秘书 (secretary post)  & 38\% & Delegated actuation \\
组织部 (org.\ dept.)   & Controls appointments
    & Gate-keeping \\
共青团 (Youth League)   & Highest per capita
    & Early selection \\
\bottomrule
\end{tabular}
\end{center}

\noindent
Township experience appears in $59\%$ of county leaders and
$77\%$ of township-level leaders.  The two offices (县委办 and
政府办) appear in $22\%$ of county leaders.  Secretary experience
appears in $38\%$ of county leaders and $50\%$ of township-level
leaders.  Office director (办公室主任) experience appears in
${\sim}42\%$ of leaders in vertically managed departments.
\end{example}

\begin{proposition}[The secretary as delegated knife]
\label{prop:secretary-knife}
The secretary (秘书) position satisfies the knife definition
(\cref{def:knife}):
\begin{enumerate}[label=(\roman*)]
  \item \textbf{Autonomous actuation.}  The secretary exercises
    隐性权力 (hidden power): control over information flow,
    scheduling of access, and framing of decisions for the leader.
    This is \emph{delegated actuation}---the secretary's actions
    affect the system state without requiring the leader's direct
    command at each step.
  \item \textbf{Observability.}  The secretary is the most
    observed node in the leader's proximity: every action is
    visible to the leader, other staff, and petitioners.
\end{enumerate}
Whether this knife is a tool or a threat is a phase function
of the mean field (\cref{prop:phase}): when the secretary's
actuation is aligned with the leader's viability, the secretary
is a tool that extends the leader's reach; when the secretary
builds an independent power base (as Feng documents in several
cases), the secretary becomes a threat.
The $38\%$ prevalence among county leaders means the secretary
channel is a high-capacity edge in the promotion flow network.
In the max-flow/min-cut framework (\cref{thm:flowcut}),
removing this channel would block a large fraction of viable
promotion paths, making it a critical component of the min-cut.
\end{proposition}

\begin{remark}[办公室政治]\label{rem:office-politics}
Feng documents a phenomenon he calls 办公室政治
(office politics): nearly half of all department leaders
have office director (办公室主任) experience.  Combined
with the secretary prevalence, this means the majority of
successful cadres passed through a proximity-to-power node.
In the execution graph, these nodes form a \emph{bottleneck}:
a small set of positions through which a disproportionate
fraction of the max-flow passes.  The bottleneck is not
accidental---it is a structural consequence of the
dual-ring dynamics.  Proximity to the leader provides
privileged access to the outer ring (guanxi opportunities),
which is the binding constraint for the tangential condition
\eqref{eq:guanxi-tangential}.  The cradle mechanism is not
about training; it is about \emph{positioning within the
relationship flow network}.
\end{remark}

% ================================================================
\section{Political families as cooperative subgraphs}
\label{sec:zx-families}
% ================================================================

Feng documents $21$ large political families (大家族, five or
more cadres at 副科级 or above), $15$ four-person families,
$35$ three-person families, and over $90$ two-person families,
connected by blood and marriage edges.

\begin{definition}[Political clique]\label{def:political-clique}
A \emph{political clique} in the cadre lattice is a connected
subgraph $C \subseteq G$ whose edges are blood ties
(parent--child, sibling) or marriage ties (spouse, in-law),
satisfying:
\begin{enumerate}[label=(\roman*)]
  \item every node in $C$ is a cadre at $\Lambda_k$ for some
    $k \geq 0$;
  \item the internal coupling is cooperative: the promotion
    of one member increases the promotion probability of
    connected members (positive $\beta_{ij}$ in the notation
    of \cref{def:triad}).
\end{enumerate}
The size of the clique is $|C|$.
\end{definition}

\begin{proposition}[Clique amplification]
\label{prop:clique-amplification}
A political clique~$C$ of size~$m$ amplifies the effective
guanxi level of its members.  For member $i \in C$, the
effective actuation is
\begin{equation}\label{eq:clique-amplification}
  U_i^{\textup{eff}}
  \;=\; U_i \;+\; \sum_{j \in C \setminus \{i\}}
    \beta_{ij}\, U_j,
\end{equation}
where $\beta_{ij} > 0$ is the cooperative coupling
(stronger for direct blood or marriage ties, weaker for
distant relatives).  Since all terms are positive,
$U_i^{\textup{eff}} > U_i$: every member is stronger than
alone.  For a clique of size~$m$ with uniform coupling
$\beta$, the amplification factor is
\[
  \frac{\bar{U}_C^{\textup{eff}}}{\bar{U}_C^{0}}
  \;=\; 1 + (m - 1)\,\beta,
\]
which grows linearly in $m$.  A large family ($m \geq 5$,
$\beta \sim 0.2$) amplifies its members' effective actuation
by a factor of ${\sim}1.8$ or more, raising them above the
mean-field knife threshold even when their individual
actuation would fall below it.
\end{proposition}

\begin{proof}
The effective mean actuation within~$C$ is
$\bar{U}_C^{\textup{eff}}
  = \frac{1}{m}\sum_{i \in C} U_i^{\textup{eff}}
  = \frac{1}{m}\sum_{i \in C}
    \bigl(U_i + \sum_{j \neq i} \beta_{ij} U_j\bigr)$.
With uniform coupling $\beta_{ij} = \beta$, this becomes
$\bar{U}_C^{0} + \beta(m-1)\bar{U}_C^{0}
  = \bar{U}_C^{0}\bigl(1 + (m-1)\beta\bigr)$.
\end{proof}

\begin{remark}[The grapevine and its decline]
\label{rem:grapevine}
Feng uses the metaphor 葡萄藤 (grapevine) for political
families: they grow larger with each generation as
marriage edges create new connections.  This is positive
feedback: a larger clique amplifies its members more
(\cref{prop:clique-amplification}), which helps more members
reach higher levels, which provides a higher platform for
the next generation's marriages.  The largest family
documented (张家, centred on Zhang Taikang) has eight
children all holding cadre positions, with marriage edges
reaching into several other families.

The growth is a cooperative triad (\cref{def:triad}) at
the family level:
\begin{center}
\renewcommand{\arraystretch}{1.15}
\begin{tabular}{@{}clp{6cm}@{}}
\toprule
\textbf{Variable} & \textbf{Name} & \textbf{Content} \\
\midrule
$x_1$ & Existing power & Members currently in cadre positions \\
$x_2$ & Accumulated capital & Family reputation, favours owed \\
$x_3$ & New entrants & Children, in-laws entering the lattice \\
\bottomrule
\end{tabular}
\end{center}

\noindent
Feng documents three decline factors, each weakening a
coupling in this triad:
\begin{enumerate}[label=(\roman*)]
  \item \textbf{One-child policy} reduces $|C|$ directly,
    shrinking the cooperative sum in
    \eqref{eq:clique-amplification}.
    This weakens $\beta_{31}$ (fewer children to receive power)
    and $\beta_{13}$ (fewer children to reinforce existing
    members).
  \item \textbf{Cadre exchange system} (干部交流制度) separates
    family members geographically, weakening all $\beta_{ij}$
    by reducing the cooperative coupling to near zero
    for exchanged members.
  \item \textbf{Higher education and urbanisation} removes
    potential entrants from the county lattice entirely:
    the children of leaders attend elite universities and
    work in Beijing, Zhengzhou, or Shanghai rather than
    returning to the county.  This weakens $x_3$ at the
    source.
\end{enumerate}
Each factor moves the family triad toward the bifurcation
threshold of \cref{thm:triad}(c).  By \cref{thm:triad}(d)
(stable until gone), the family appears healthy until the
coupling drops below threshold, at which point its influence
collapses.  Feng observes this directly: political families do
not fade gradually.  They end when the core member retires
or when the coupling channels are severed.
\end{remark}

% ================================================================
\section{The disciplinary boundary and the 政绩 isomorphism}
\label{sec:zx-boundary}
% ================================================================

\begin{definition}[Disciplinary absorbing barrier]
\label{def:discipline-barrier}
The disciplinary system imposes a graduated absorbing barrier
on $K_{\textup{cadre}}$:

\begin{center}
\renewcommand{\arraystretch}{1.25}
\begin{tabular}{@{}lrlp{5cm}@{}}
\toprule
\textbf{Sanction} & \textbf{\%} & \textbf{Effect} \\
\midrule
警告 (warning) & 35\% & Promotion frozen 1~yr \\
严重警告 (severe warning) & 28\% & Frozen 1~yr + stigma \\
撤职 (dismissal) & 4\% & Position removed \\
留党察看 (probation) & 10\% & Career suspended 1--2~yr \\
开除党籍 (expulsion) & 23\% & Permanent exit from $K_{\textup{cadre}}$ \\
\bottomrule
\end{tabular}
\end{center}

\noindent
Data: $101$ cases of 副科级 and 正科级 cadres sanctioned
between 1993 and 2009 \cite{zhongxian}.  The first two
sanctions ($63\%$ of cases) are \emph{reflecting} barriers:
the trajectory hits the boundary and returns with a delay.
Expulsion ($23\%$) is a \emph{terminal absorbing} state.
The problem type distribution is: economic $57\%$,
political $35\%$, lifestyle $8\%$.
\end{definition}

\begin{proposition}[一票否决 as hard constraint surface]
\label{prop:one-vote-veto}
The 一票否决 (one-vote veto) system for family planning and
social stability imposes a \emph{hard constraint surface}
within $K_{\textup{cadre}}$: a single violation in these
domains terminates viability regardless of all other performance
metrics.  Formally, let $x_{\textup{FP}}$ and
$x_{\textup{stab}}$ denote the family-planning and
social-stability components of the state.
The one-vote veto defines
\[
  K_{\textup{veto}}
  \;=\;
  \bigl\{\, x \in K_{\textup{cadre}} \;:\;
  x_{\textup{FP}} \geq \epsilon_{\textup{FP}},\;\;
  x_{\textup{stab}} \geq \epsilon_{\textup{stab}}
  \,\bigr\},
\]
and $K_{\textup{veto}} \subsetneq K_{\textup{cadre}}$: the
veto surfaces are strictly interior to the cadre kernel.  A
trajectory that crosses $x_{\textup{FP}} = \epsilon_{\textup{FP}}$
or $x_{\textup{stab}} = \epsilon_{\textup{stab}}$ exits
$K_{\textup{veto}}$ even if all other coordinates remain
safely interior.

Feng documents extreme consequences of this hard surface in
the family-planning domain: township cadres deployed
coercive enforcement (强制措施) precisely because the one-vote
veto transformed a policy variable into an absorbing barrier.
The severity of enforcement is not a moral choice but a
viability response: with the veto in place, failing the
family-planning target is equivalent to career death, and
the tangential condition demands maximal actuation at the
constraint surface.
\end{proposition}

\begin{remark}[政绩同构: the self-similar kernel]
\label{rem:isomorphic-kernel}
The target management system (目标考核) assigns numerical
scores to every administrative level.  Feng documents that
the target structure is \emph{isomorphic} across scales:
the province sets targets for the city, which sets identical
targets for the county, which sets identical targets for the
township, which sets identical targets for the village.
He calls this 政绩同构 (isomorphic performance structure).

In the viability framework, this means the viability kernel
has a \emph{self-similar} structure: the constraint set at
each scale is a rescaled copy of the constraint set at the
next scale up.  If $K^{(s)}$ is the kernel at scale~$s$
(province, city, county, township, village), then
$K^{(s)} \cong \phi_s(K^{(s+1)})$ for a scaling map
$\phi_s$.

This self-similarity has a pathological consequence:
the tangential condition at each scale generates the
\emph{same} pressure, producing fake achievements (假政绩)
at every level simultaneously.  Feng documents four county
party secretaries in succession whose ``signature projects''
(工程) were largely fabricated or failed: pig pens, forced
industrialisation, agricultural campaigns.  The self-similar
kernel means that viability pressure is not local---it
cascades identically from province to village.  The system
does not produce fake achievements because cadres are
dishonest; it produces them because the tangential condition
at a self-similar boundary demands actuation that the local
dynamics cannot supply, and fabrication is the cheapest
correction $g(x_G)$ available.
\end{remark}

\begin{remark}[The dual-ring model is max-flow/min-cut]
\label{rem:dual-ring-maxflow}
The dual-ring dynamics can be restated in the flow-theoretic
language of the agentic calculus (\cref{sec:flow}).
The inner ring defines the \emph{capacity constraints} on
the promotion flow network: the number of positions at each
level, the term limits, the age cutoffs.  The outer ring
defines the \emph{actual flow}: which cadres move through
which edges.  The max-flow/min-cut theorem
(\cref{thm:flowcut}) then states:
\begin{center}
\emph{The maximum promotion flow through the cadre lattice
equals the minimum cut capacity of the institutional
constraints.}
\end{center}
The min-cut is the set of bottleneck positions
(the cradles of \cref{sec:zx-cradle}) through which
promotion flow must pass.  The guanxi dynamics of the
outer ring are the routing algorithm: they determine which
cadres are assigned to which edges.  The knife---the
critical threshold separating viable cadres from non-viable
ones---is the capacity of the min-cut, which by
\cref{thm:meanfield} is determined by the mean guanxi
level~$\bar{U}_G$.

Feng's dissertation, read through this lens, is a
complete empirical map of the min-cut structure: the cradle
institutions are the bottleneck edges, the political
families are cooperative subflows that locally exceed the
mean capacity, the vote-canvassing networks (拉票网) are
the routing signals, and the disciplinary system is the
boundary enforcement that removes flows violating the
capacity constraints.  The knife is the mean.
\end{remark}

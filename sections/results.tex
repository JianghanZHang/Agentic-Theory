\chapter{Main Results}\label{sec:results}

\section{The binary lifecycle}\label{sec:lifecycle}

\begin{theorem}[Binary fate]\label{thm:lifecycle}
Every knife has exactly two possible outcomes after phase transition:
\begin{enumerate}[label=(\alph*)]
  \item \textbf{Relinquish}: the holder voluntarily sets
  $\Ur \to \varnothing$.
  \item \textbf{Elimination}: the king forces removal via
  $u^* \in \Umax$.
\end{enumerate}
There is no path~(c).
\end{theorem}

\begin{proof}
The knife criterion is $\Ur \neq \varnothing \;\wedge\;
r \in \mathrm{Im}(\Obs)$ (\cref{rem:knife-di} for the DI
restatement). To exit this classification, at least one condition
must fail. Condition~(2) is controlled by the king
(he can always look); only condition~(1) can be changed by the
holder. Setting $\Ur \to \varnothing$ is path~(a). If the holder
does not, the king must act (viability axiom), which is path~(b).
The classification is exhaustive.
\end{proof}

\begin{remark}[Han Xin's error]\label{rem:hanxin}
The proverb ``when the cunning hare is killed, the hunting dog is
cooked'' (\emph{Shiji}, Huaiyin Hou) conflates three structurally
distinct resources: the \emph{bow} ($\Ur = \varnothing$, tool,
``stored'' not destroyed), the \emph{dog} ($\Ur \neq \varnothing$,
actuator, ``cooked''), and the \emph{advisor} (pure function, no
actuation---Zhang Liang survived). Han Xin quoted the answer but did
not parse its fine structure.
\end{remark}

\section{The fixed-point impossibility}\label{sec:fixedpoint}

\begin{theorem}[No path (c)]\label{thm:fixedpoint}
There is no strategy that ``proves your knife is not a knife'' while
retaining the knife. Formally, the map
$T: \Ur \mapsto \varnothing$ conditional on $\Ur \neq \varnothing$
has no fixed point other than $\Ur = \varnothing$.
\end{theorem}

\begin{proof}
Case~1: $\Ur = \varnothing$. Then $r$ is not a knife, and no proof
is needed. $\Ur = \varnothing$ is self-certifying.
Case~2: $\Ur \neq \varnothing$. Then no speech act can set
$\Ur \to \varnothing$---the criterion tests physical capability,
not narrative. The only way to satisfy $T(U_r) = \varnothing$ is to
physically relinquish $\Ur$, which is path~(a).

Moreover, the act of proving is itself a signal: ``I need to prove
my knife is not a knife'' implies suspicion, i.e., $r$ is already
in $\mathrm{Im}(\Obs)$. The proof attempt reinforces condition~(2).
\end{proof}

\section{The unconstrained power paradox}\label{sec:paradox}

\begin{theorem}[Perpetual elimination]\label{thm:paradox}
$U = \Umax$ implies the king must preemptively eliminate all
observable autonomous actuators:
\[
  U = \Umax \implies
  \text{the king must preempt all } r \text{ with }
  \Ur \neq \varnothing \;\wedge\; r \in \mathrm{Im}(\Obs).
\]
The more unconstrained the king, the more knives he must cut.
\end{theorem}

\begin{proof}
$\Umax$ means the king tolerates \emph{no} autonomous actuation:
every such actuator is a boundary threat on $\Viab(K)$. A
constrained system (constitutional regime) institutionalizes knife
dynamics by installing breakpoints. An unconstrained system must
handle every knife individually. The paradox: unconstrained power is
not freedom---it is a perpetual elimination machine.
\end{proof}

\begin{proposition}[Imperfect observability accelerates the paradox]
\label{prop:imperfect}
If the detection function $\Obs$ is imperfect, the paradox
\emph{intensifies}, not weakens.
\end{proposition}

\begin{proof}
Three steps:
\begin{enumerate}[label=(\roman*)]
  \item The king knows $\Obs$ is imperfect. Hidden knives
  (\cref{def:knife}, condition~(2) unsatisfied) are more dangerous
  than visible ones. The king has motive, capability, and survival
  obligation to expand $\Obs$.
  \item Expanding $\Obs$ does not ``discover existing knives''---it
  \emph{creates new ones} definitionally. A hidden actuator
  satisfying condition~(1) but not~(2) enters $\mathrm{Im}(\Obs)$
  upon expansion $\to$ both conditions now satisfied $\to$ it
  \emph{becomes} a knife. The knife exists in the intersection
  $\Ur \neq \varnothing \;\wedge\; r \in \mathrm{Im}(\Obs)$;
  expanding $\Obs$ expands this intersection.
  \item Positive feedback:
  $U \to \Umax \implies \Obs \to \Obs_{\max}$.
  The expansion of $\Obs$ is the \emph{adjoint process} of the
  expansion of $U$. The elimination machine has two engines: the
  cutting arm ($U$) and the detecting eye ($\Obs$). They co-drive.
\end{enumerate}
Historical instances: Qin's mutual surveillance law
(\emph{lianzuo}), Han's gold-purity test (\emph{zhuo\-jin
duo\-jue}), Ming's three-layer nested monitoring (Jinyiwei
$\to$ Dongchang $\to$ Xichang---each layer itself becomes a new
knife).
\end{proof}

\begin{proposition}[$\Umax$ as attractor]\label{prop:attractor}
$\Umax$ is an attractor, not a state. No historical king achieves
literal $\Umax$, but the system dynamics point toward it:
\[
  \frac{d}{dt}\|U(t) - \Umax\| \leq 0
  \implies
  \frac{d}{dt}\bigl(\text{detected knives}\bigr) \geq 0.
\]
The paradox describes the trajectory, not the endpoint.
\end{proposition}

\section{The breakpoint criterion}\label{sec:breakpoint}

\begin{corollary}[Breakpoint strategy]\label{cor:breakpoint}
A resource $r$ is not a knife if and only if its execution chain
contains at least one node controlled by the king (a
\emph{breakpoint}):
\[
  r \text{ is not a knife}
  \iff
  \exists\; v \in \text{execution chain of } r
  \;\text{s.t.}\; v \text{ is controlled by the king.}
\]
\end{corollary}

\begin{proof}
If a breakpoint exists, $r$ cannot execute independently
(condition~(1) fails), so $r$ is not a knife. If no breakpoint
exists, the execution chain is closed and $r$ can actuate
autonomously, satisfying condition~(1). Combined with
observability, this makes $r$ a knife.
\end{proof}

\begin{remark}[Modern translation]\label{rem:modern}
Zhang Liang's strategy: ``ensure your capability always requires a
component you do not control.'' Liu Bang's strategy: ``become the
mandatory node in every execution chain.''
\end{remark}

\chapter{Main Results}\label{sec:results}

\section{The binary lifecycle}\label{sec:lifecycle}

\begin{lemma}[Forcing]\label{lem:forcing}
Let $r$ be a sword that persists: both conditions of
\cref{def:sword} are satisfied for all $t \geq t_0$.  Then the
tangential condition $F(x) \cap T_K(x) \neq \varnothing$
(\cref{thm:viability-di}) requires the king to allocate
$u^*(t) \in C(x(t))$ to compensate $r$ at every instant.  The
cumulative cost $\int_{t_0}^{T} \|u^*\|\,dt$ is unbounded as
$T \to \infty$.
\end{lemma}

\begin{proof}
The sword $r$ acts independently (condition~1 of
\cref{def:sword}), so $F_r(x)$ contributes an uncontrolled
velocity component.  The king's feedback map $C(x)$
(\cref{def:feedback}) must contain a compensating control at
every instant (tangential condition).  Since $r$'s actuation is
autonomous, the king cannot predict $r$'s choices within $\Ur$;
compensation is reactive, not preventive.  The cost per unit
time is bounded below by the minimum energy needed to counteract
$F_r$, which is positive: condition~(1) ensures
$F_r(x) \not\subset T_K(x)$ for some $x$ (\cref{rem:sword-di}).
Over unbounded time, cost diverges.  The viability axiom
(\cref{ax:viability}) requires finite-cost maintenance;
therefore the sword must be resolved in finite time.
\end{proof}

\begin{theorem}[Binary fate]\label{thm:lifecycle}
Every sword has exactly two possible outcomes after phase transition:
\begin{enumerate}[label=(\alph*)]
  \item \textbf{Relinquish}: the holder voluntarily sets
  $\Ur \to \varnothing$.
  \item \textbf{Elimination}: the king forces removal via
  $u^* \in \Umax$.
\end{enumerate}
There is no path~(c).
\end{theorem}

\begin{proof}
The sword criterion is $\Ur \neq \varnothing \;\wedge\;
r \in \mathrm{Im}(\Obs)$ (\cref{rem:sword-di} for the DI
restatement).  As long as both conditions hold, the sword
persists and \cref{lem:forcing} applies: the cost of indefinite
coexistence is unbounded.  The viability axiom requires
finite-cost maintenance, so the sword must be resolved.

Resolution requires at least one condition to fail.
Setting $\Ur \to \varnothing$ is path~(a).  If the holder does
not relinquish, condition~(1) persists; regression to a pre-sword
(suppressing condition~(2) while retaining condition~(1)) is not
a resolution but a deferral, since the causal envelope persists
and any expansion of $\Obs$ re-detects the resource
(\cref{rem:lifecycle-scope}).  By \cref{lem:forcing}, the cost
of indefinite coexistence is unbounded; the king is therefore
forced to eliminate via $u^* \in \Umax$, which is path~(b).
The classification is exhaustive.
\end{proof}

\begin{remark}[Scope of the binary lifecycle]\label{rem:lifecycle-scope}
\Cref{thm:lifecycle} applies to \emph{swords}: resources satisfying
both conditions of \cref{def:sword} simultaneously.  A resource that
satisfies condition~(1) but not condition~(2)---a \emph{pre-sword}
(\cref{def:presword})---is outside the theorem's scope.
Xiao He's self-blunting strategy (\cref{ex:xiaohe}) is a sword
$\to$ pre-sword regression: he exits the sword classification by
suppressing condition~(2), not by relinquishing condition~(1).
This is not path~(c); it is a transition to a different lifecycle
state (\cref{rem:sword-lifecycle}).  The binary partition holds for
every resource that \emph{remains} a sword: as long as both
conditions are satisfied, exactly one of (a)~or~(b) must obtain.

The pre-sword state is unstable because the causal envelope
$\mathrm{Reach}(x, \Ur)$ (\cref{eq:causal-envelope}) persists: the
agent retains capability, and any expansion of $\Obs$
re-detects the resource, returning it to the sword row and
re-activating the forcing cost of \cref{lem:forcing}.  Thus
the pre-sword defers rather than resolves the binary fate.
\end{remark}

\begin{remark}[Han Xin's error]\label{rem:hanxin}
The proverb ``when the cunning hare is killed, the hunting dog is
cooked'' (\emph{Shiji}, Huaiyin Hou) conflates three structurally
distinct resources: the \emph{bow} ($\Ur = \varnothing$, tool,
``stored'' not destroyed), the \emph{dog} ($\Ur \neq \varnothing$,
actuator, ``cooked''), and the \emph{advisor} (pure function, no
actuation---Zhang Liang survived). Han Xin quoted the answer but did
not parse its fine structure.
\end{remark}

\section{The fixed-point impossibility}\label{sec:fixedpoint}

\begin{theorem}[No path (c)]\label{thm:fixedpoint}
There is no strategy that ``proves your sword is not a sword'' while
retaining the sword. Formally, the map
$T: \Ur \mapsto \varnothing$ conditional on $\Ur \neq \varnothing$
has no fixed point other than $\Ur = \varnothing$.
\end{theorem}

\begin{proof}
Case~1: $\Ur = \varnothing$. Then $r$ is not a sword, and no proof
is needed. $\Ur = \varnothing$ is self-certifying.
Case~2: $\Ur \neq \varnothing$. Then no speech act can set
$\Ur \to \varnothing$---the criterion tests physical capability,
not narrative. The only way to satisfy $T(U_r) = \varnothing$ is to
physically relinquish $\Ur$, which is path~(a).

Moreover, the act of proving is itself a signal: ``I need to prove
my sword is not a sword'' implies suspicion, i.e., $r$ is already
in $\mathrm{Im}(\Obs)$. The proof attempt reinforces condition~(2).
\end{proof}

\section{The unconstrained power paradox}\label{sec:paradox}

\begin{theorem}[Perpetual elimination]\label{thm:paradox}
$U = \Umax$ implies the king must preemptively eliminate all
observable autonomous actuators:
\[
  U = \Umax \implies
  \text{the king must preempt all } r \text{ with }
  \Ur \neq \varnothing \;\wedge\; r \in \mathrm{Im}(\Obs).
\]
The more unconstrained the king, the more swords he must cut.
\end{theorem}

\begin{proof}
$\Umax$ means the king tolerates \emph{no} autonomous actuation:
every such actuator is a boundary threat on $\Viab(K)$. A
constrained system (constitutional regime) institutionalises sword
dynamics by installing breakpoints. An unconstrained system must
handle every sword individually. The paradox: unconstrained power is
not freedom---it is a perpetual elimination machine.
\end{proof}

\begin{proposition}[Imperfect observability accelerates the paradox]
\label{prop:imperfect}
If the detection function $\Obs$ is imperfect, the paradox
\emph{intensifies}, not weakens.
\end{proposition}

\begin{proof}
Three steps:
\begin{enumerate}[label=(\roman*)]
  \item The king knows $\Obs$ is imperfect. Hidden swords
  (\cref{def:sword}, condition~(2) unsatisfied) are more dangerous
  than visible ones. The king has motive, capability, and survival
  obligation to expand $\Obs$.
  \item Expanding $\Obs$ does not ``discover existing swords''---it
  \emph{creates new ones} definitionally. A hidden actuator
  satisfying condition~(1) but not~(2) enters $\mathrm{Im}(\Obs)$
  upon expansion $\to$ both conditions now satisfied $\to$ it
  \emph{becomes} a sword. The sword exists in the intersection
  $\Ur \neq \varnothing \;\wedge\; r \in \mathrm{Im}(\Obs)$;
  expanding $\Obs$ expands this intersection.
  \item Positive feedback:
  $U \to \Umax \implies \Obs \to \Obs_{\max}$.
  The expansion of $\Obs$ is the \emph{adjoint process} of the
  expansion of $U$. The elimination machine has two engines: the
  cutting arm ($U$) and the detecting eye ($\Obs$). They co-drive.
\end{enumerate}
Historical instances: Qin's mutual surveillance law
(\emph{lianzuo}), Han's gold-purity test (\emph{zhuo\-jin
duo\-jue}), Ming's three-layer nested monitoring (Jinyiwei
$\to$ Dongchang $\to$ Xichang---each layer itself becomes a new
sword).
\end{proof}

\begin{proposition}[$\Umax$ as attractor]\label{prop:attractor}
$\Umax$ is an attractor, not a state. No historical king achieves
literal $\Umax$, but the system dynamics point toward it:
\[
  \frac{d}{dt}\|U(t) - \Umax\| \leq 0
  \implies
  \frac{d}{dt}\bigl(\text{detected swords}\bigr) \geq 0.
\]
The paradox describes the trajectory, not the endpoint.
\end{proposition}

\section{The breakpoint criterion}\label{sec:breakpoint}

\begin{corollary}[Breakpoint strategy]\label{cor:breakpoint}
A resource $r$ is not a sword if and only if its execution chain
contains at least one node controlled by the king (a
\emph{breakpoint}):
\[
  r \text{ is not a sword}
  \iff
  \exists\; v \in \text{execution chain of } r
  \;\text{s.t.}\; v \text{ is controlled by the king.}
\]
\end{corollary}

\begin{proof}
If a breakpoint exists, $r$ cannot execute independently
(condition~(1) fails), so $r$ is not a sword. If no breakpoint
exists, the execution chain is closed and $r$ can actuate
autonomously, satisfying condition~(1). Combined with
observability, this makes $r$ a sword.
\end{proof}

\begin{remark}[Modern translation]\label{rem:modern}
Zhang Liang's strategy: ``ensure your capability always requires a
component you do not control.'' Liu Bang's strategy: ``become the
mandatory node in every execution chain.''
\end{remark}

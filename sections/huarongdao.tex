\section{华容道: Complete Instantiation}\label{sec:huarongdao}

The historical cases in \cref{sec:applications} each validate one
concept. We now present a single finite object that instantiates the
\emph{entire} framework simultaneously: the Chinese sliding block
puzzle 华容道 (Huarong Pass).

\subsection{The puzzle}

\begin{definition}[华容道]\label{def:huarongdao}
A \emph{华容道 instance} is a tuple
$(\mathcal{B},\, \mathcal{P},\, p_*,\, E)$:
\begin{enumerate}[label=(\roman*)]
  \item $\mathcal{B} = [m] \times [n]$: rectangular grid (the
  \emph{board}, 方);
  \item $\mathcal{P} = \{p_1, \ldots, p_k\}$: rectangular blocks
  placed non-overlapping on $\mathcal{B}$ (the \emph{pieces}, 圆);
  \item $p_* \in \mathcal{P}$: distinguished piece (the \emph{king});
  \item $E \subset \partial\mathcal{B}$: boundary region congruent
  to $p_*$ (the \emph{exit}).
\end{enumerate}
The \emph{free cells} $\mathcal{F} = \mathcal{B} \setminus
\bigcup_i p_i$ are the system's degrees of freedom. A
\emph{configuration} is a valid placement of all pieces. A \emph{move}
is a unit translation of one piece into adjacent free cells. The
\emph{configuration graph} $\mathcal{G} = (\mathcal{V}, \mathcal{E})$
has configurations as vertices and legal moves as edges.

The puzzle: does there exist a path in $\mathcal{G}$ from $\sigma_0$
to any $\sigma_f$ with $\sigma_f(p_*) = E$?
\end{definition}

The standard instance is $\mathcal{B} = [4] \times [5]$ in the
configuration 横刀立马 (``horizontal knife, standing horse''):

\begin{center}
\begin{tikzpicture}[scale=0.85]
  % Grid
  \draw[gray!40, thin] (0,0) grid (4,5);
  \draw[very thick] (0,0) rectangle (4,5);
  % Exit
  \draw[very thick, densely dashed] (1,-0.05) -- (1,-0.4) -- (3,-0.4)
    -- (3,-0.05);
  \node[font=\footnotesize] at (2,-0.65) {$E$ (exit)};
  % 曹操 (2×2) — the king
  \fill[black!80] (1.05,3.05) rectangle (2.95,4.95);
  \node[white, font=\bfseries\large] at (2,4.2) {曹操};
  \node[white, font=\scriptsize] at (2,3.5) {$p_*\;(2{\times}2)$};
  % 关羽 (2×1 horizontal) — the knife
  \fill[black!55] (1.05,2.05) rectangle (2.95,2.95);
  \node[white, font=\small\bfseries] at (2,2.5) {关羽 $(2{\times}1)$};
  % Generals (1×2 vertical)
  \fill[black!35] (0.05,3.05) rectangle (0.95,4.95);
  \node[white, font=\small, rotate=90] at (0.5,4) {张飞};
  \fill[black!35] (3.05,3.05) rectangle (3.95,4.95);
  \node[white, font=\small, rotate=90] at (3.5,4) {赵云};
  \fill[black!35] (0.05,1.05) rectangle (0.95,2.95);
  \node[white, font=\small, rotate=90] at (0.5,2) {马超};
  \fill[black!35] (3.05,1.05) rectangle (3.95,2.95);
  \node[white, font=\small, rotate=90] at (3.5,2) {黄忠};
  % Soldiers (1×1) — numbered ①②③④ to match 棋谱 (\cref{app:solution})
  \fill[black!12] (1.05,1.05) rectangle (1.95,1.95);
  \draw[black!50] (1.05,1.05) rectangle (1.95,1.95);
  \node[font=\small] at (1.5,1.5) {\textcircled{\scriptsize 1}};
  \fill[black!12] (2.05,1.05) rectangle (2.95,1.95);
  \draw[black!50] (2.05,1.05) rectangle (2.95,1.95);
  \node[font=\small] at (2.5,1.5) {\textcircled{\scriptsize 2}};
  \fill[black!12] (0.05,0.05) rectangle (0.95,0.95);
  \draw[black!50] (0.05,0.05) rectangle (0.95,0.95);
  \node[font=\small] at (0.5,0.5) {\textcircled{\scriptsize 3}};
  \fill[black!12] (3.05,0.05) rectangle (3.95,0.95);
  \draw[black!50] (3.05,0.05) rectangle (3.95,0.95);
  \node[font=\small] at (3.5,0.5) {\textcircled{\scriptsize 4}};
  % Free cells
  \draw[densely dashed, black!40] (1.05,0.05) rectangle (1.95,0.95);
  \node[black!50, font=\footnotesize] at (1.5,0.5) {$\varnothing$};
  \draw[densely dashed, black!40] (2.05,0.05) rectangle (2.95,0.95);
  \node[black!50, font=\footnotesize] at (2.5,0.5) {$\varnothing$};
  % Legend
  \begin{scope}[font=\footnotesize, anchor=west]
    \fill[black!80] (4.7,4.65) rectangle (5.0,4.85);
    \node at (5.15,4.75) {王: $p_*$ $(2{\times}2)$};
    \fill[black!55] (4.7,4.15) rectangle (5.0,4.35);
    \node at (5.15,4.25) {刀: 关羽 $(2{\times}1)$};
    \fill[black!35] (4.7,3.65) rectangle (5.0,3.85);
    \node at (5.15,3.75) {将: generals $(1{\times}2)$};
    \fill[black!12] (4.7,3.15) rectangle (5.0,3.35);
    \draw[black!50] (4.7,3.15) rectangle (5.0,3.35);
    \node at (5.15,3.25) {卒: \textcircled{\tiny 1}--\textcircled{\tiny 4}\ $(1{\times}1)$};
    \draw[densely dashed, black!40] (4.7,2.65) rectangle (5.0,2.85);
    \node at (5.15,2.75) {$\varnothing$: free cells (水)};
  \end{scope}
\end{tikzpicture}
\end{center}

Ten pieces ($4 + 2 + 4 \cdot 2 + 4 \cdot 1 = 18$ cells), two free
cells, exit at bottom center (width~$2$). The minimum solution requires
81~步 (steps; \cref{def:bu}).

\subsection{The isomorphism}

\begin{theorem}[华容道 $=$ viability maintenance]\label{thm:huarongdao}
The standard 华容道 instance instantiates the viability framework:
\begin{center}
\begin{tabular}{@{}lp{4cm}p{5cm}@{}}
\toprule
\textbf{Framework} & \textbf{华容道} & \textbf{Mechanism} \\
\midrule
Viability axiom & Path $\sigma_0 \to \sigma_f$ in $\mathcal{G}$ &
King must reach exit \\
King & 曹操 $(2{\times}2)$ & Least mobile, highest importance \\
Knife (\cref{def:knife}) & 关羽 $(2{\times}1)$ & Blocks exit;
autonomous; observable \\
Pawns & Soldiers $(1{\times}1)$ & Most mobile, lowest importance \\
Cut vertex & Free cells $\mathcal{F}$ &
$\mathcal{F} = \varnothing \Rightarrow$ frozen \\
Phase transition & 关羽 clears corridor & Before: blocked. After:
path opens \\
Water & Free-cell flow & Slides opposite to piece movement \\
$w = 0$ collapse & Zero free cells & No flow $\to$ no path $\to$
dead \\
\bottomrule
\end{tabular}
\end{center}
\end{theorem}

\begin{proof}
\emph{Viability.} The puzzle asks exactly \cref{ax:viability}: does a
path exist from $\sigma_0$ through $\mathcal{G}$ to a goal state? The
configuration graph is the execution graph; each edge is a legal move;
the viable kernel is the set of configurations from which the exit
remains reachable.

\emph{Knife.} 关羽 satisfies both conditions of \cref{def:knife}:
(1)~autonomous actuation---he can slide independently of 曹操---and
(2)~observability---his position visibly blocks the exit corridor.
He must yield for the king to pass.

\emph{Cut $=$ free cell.} Fill both free cells and $\mathcal{G}$ has
no edges: every configuration is isolated. The free cell inverts the
cut vertex: ``the absence whose removal freezes.'' One structural
element controls all connectivity.

\emph{Mobility $\propto 1/\text{size}$.} A piece of size $s$ needs $s$
aligned free cells to move. Soldiers ($s = 1$): one free cell.
曹操 ($s = 4$): two aligned free cells. The king is the least mobile
agent---\cref{thm:cutvertex} made physical.

\emph{Water.} When a piece slides left, the free cell moves right.
The free cell flows in the opposite direction---it \emph{is} the water
(\cref{def:water}). Each move transfers the free cell to a new
position, enabling the next move. The 81-步 solution is a flow of
water through 81 channels.
\end{proof}

\begin{remark}[横刀立马: the name is the theorem]\label{rem:hrdname}
The configuration's traditional name means ``horizontal knife, standing
horse.'' 刀~(knife) $=$ 关羽 blocking horizontally; 马~(horse) $=$
generals standing vertically. Chinese game designers named the
configuration by its structural properties---the framework's vocabulary,
centuries before graph theory.
\end{remark}

\begin{remark}[义释曹操]\label{rem:guanyu}
In the \emph{Romance of the Three Kingdoms}, 关羽 is stationed at
华容道 after the Battle of Red Cliffs (208~CE). 曹操 retreats through.
关羽 has both conditions of \cref{def:knife}: autonomous actuation
(his army) and observability (曹操 approaching in plain sight). He
chooses path~(a): 义~(righteousness) overrides 忠~(loyalty to Liu Bei),
and he sets $\Ur \to \varnothing$ voluntarily.

The puzzle encodes this: every solution requires moving 关羽 aside.
To solve 华容道 is to perform 义释曹操.
\end{remark}

\begin{remark}[方圆 $\times$ 黑白]\label{rem:fangyuan}
The puzzle's $2 \times 2$ classification is formalized in
\cref{def:fangyuan}. The four statics classify every cell and piece;
the dynamics---刀 (boundary/cut) and 水 (flow/transport)---emerge from
the $2 \times 2$ as the calculus operations of \cref{sec:calculus}.
\end{remark}

\subsection{The experiment}\label{sec:kpc}

\Cref{thm:huarongdao} maps framework concepts to puzzle roles.
We now make the mapping computational.
The configuration graph $\mathcal{G}$ has
$|\mathcal{V}| = 25{,}955$ canonical states and is
connected; every quantity below is \emph{exact}
(self-contained solver: \texttt{solver/}).

\subsubsection*{Water as agent}

The proof of \cref{thm:huarongdao} observed that free cells
flow opposite to piece movement.
We now invert the perspective entirely:
the two free cells \emph{are} the agent~(水),
and the ten pieces \emph{are} the board.

\begin{definition}[水-position and 水-graph]\label{def:water-graph}
A \emph{水-position} is the unordered pair
$\{f_1, f_2\} \subset \mathcal{B}$ of free cells.
Write $W(\sigma)$ for the 水-position of
configuration~$\sigma$.
The \emph{水-graph} has vertex set
$\binom{\mathcal{B}}{2}$ and an edge between
$W(\sigma)$ and $W(\sigma')$ whenever
$(\sigma, \sigma') \in \mathcal{E}$.
\end{definition}

\begin{proposition}[Ergodicity of 水]\label{thm:ergodic}
All $\binom{20}{2} = 190$ possible 水-positions are
reachable from $\sigma_0$.
\end{proposition}

\begin{proof}
Exhaustive BFS on $\mathcal{G}$: the set
$\{W(\sigma) : \sigma \text{ reachable from } \sigma_0\}$
has cardinality $190 = \binom{20}{2}$.
水 can reach every pair of cells in the board.
\end{proof}

\begin{definition}[Free-cell modes]\label{def:free-modes}
The 水-position $\{f_1, f_2\}$ has \emph{mode}:
\begin{itemize}
  \item \textbf{H}~(horizontal pair): $f_1, f_2$
  horizontally adjacent.
  Enables slides of $2 \times h$ pieces.
  \item \textbf{V}~(vertical pair): $f_1, f_2$
  vertically adjacent.
  Enables slides of $w \times 2$ pieces.
  \item \textbf{S}~(separated): $f_1, f_2$ non-adjacent.
  Only $1 \times 1$ pieces (soldiers) can move.
\end{itemize}
The king ($2 \times 2$) requires mode H or V\@.
The knife ($2 \times 1$) requires mode~H\@.
In mode~S, only soldiers act---水 is diffuse.
\end{definition}

\subsubsection*{The counting: 81}

\begin{definition}[步 (step)]\label{def:bu}
A \emph{step}~(步) is a maximal consecutive sequence of
unit moves of the same piece.
Equivalently: each step engages one piece;
all slides of that piece within the step cost zero;
switching to a different piece costs one.
\end{definition}

In Chinese puzzle tradition, the standard counting of
華容道 solutions uses~步, not unit moves.
The difference is algorithmic:

\begin{algorithm}[H]
\caption{Minimum-步 solver
  (0/1~BFS on $\mathcal{G}$)}\label{alg:kpc}
\begin{algorithmic}[1]
\Require $\mathcal{G} = (\mathcal{V}, \mathcal{E})$,\;
  initial $\sigma_0$,\;
  goal $\sigma(p_*) = E$
\Ensure Minimum-步 path $\gamma$,\;
  step count $|\gamma|$
\Statex
\State Augment state:
  $(\sigma, \ell) \in
  \mathcal{V} \times \{1, \ldots, k, \bot\}$,
  $\ell$ = last piece moved
\State $\mathrm{dist}[(\sigma_0, \bot)] \gets 0$;\;
  $Q \gets \mathrm{deque}
  \bigl[\bigl((\sigma_0, \bot),\, 0\bigr)\bigr]$
\While{$Q \neq \varnothing$}
  \State $(\sigma, \ell),\, d \gets Q.\mathrm{popleft}()$
  \If{$\sigma(p_*) = E$}
    \Return $d$ \Comment{goal reached}
  \EndIf
  \For{each neighbour $(\sigma', i)$ of $\sigma$}
    \Comment{$i$ = piece moved}
    \State $c \gets
      \begin{cases}
      0 & i = \ell \\
      1 & i \neq \ell
      \end{cases}$
      \Comment{same piece $\to$ free;
               switch $\to$ 1~步}
    \If{$d + c < \mathrm{dist}[(\sigma', i)]$}
      \State $\mathrm{dist}[(\sigma', i)] \gets d + c$
      \If{$c = 0$}\;
        $Q.\mathrm{pushfront}
        \bigl((\sigma', i),\, d\bigr)$
      \Else\;
        $Q.\mathrm{pushback}
        \bigl((\sigma', i),\, d + 1\bigr)$
      \EndIf
    \EndIf
  \EndFor
\EndWhile
\end{algorithmic}
\end{algorithm}

\begin{remark}[Complexity]\label{rem:complexity}
\begin{center}
\small
\begin{tabular}{@{}lccc@{}}
\toprule
 & \textbf{General} & \textbf{横刀立马} & \textbf{棋谱} \\
 & (sliding block) & (this instance) & (verification) \\
\midrule
Class & PSPACE-complete & --- & --- \\
$|\mathcal{V}|$ & exponential & $25{,}955$ & --- \\
$|\mathcal{E}|$ & exponential & $83{,}896$ & --- \\
Time & exponential & $O(|\mathcal{V}| + |\mathcal{E}|)$ & $O(81)$ \\
Space & exponential & $O(|\mathcal{V}|)$ & $O(1)$ \\
Optimal & ? & $81$~步 & $81$~步 (certificate) \\
\bottomrule
\end{tabular}
\end{center}
The general sliding-block puzzle is PSPACE-complete \cite{hearn}.
This instance has $25{,}955$~states:
BFS solves it in ${<}\,1$~second.
The 棋谱 (\cref{app:solution}) is a certificate
verifiable in $O(81)$~time and $O(1)$~space.
The gap from PSPACE to $O(25{,}955)$ is the entire point:
the board is fixed, the state space is finite,
the solution is exact.
\end{remark}

\begin{theorem}[81~步]\label{thm:eightyone}
The minimum number of steps from $\sigma_0$ to any
$\sigma_f$ with $\sigma_f(p_*) = E$ is exactly~$81$.
Moreover:
\begin{enumerate}[label=(\roman*)]
  \item The result is \emph{independent} of the base move
  set: both single-cell unit moves and multi-cell slides
  yield $81$~步.
  \item Along the optimal path:
  $118$~unit moves, of which $37$ are multi-slide steps
  (same piece slides $\geq 2$ cells).
  \item The $81$~步 decompose as
  $9$~king steps $+\; 72$~水-steps.
  The king acts in only $9/81 \approx 11\%$ of all steps.
\end{enumerate}
\end{theorem}

\begin{proof}
(i)~\Cref{alg:kpc} on the single-cell graph returns~$81$.
Running on the multi-cell graph (where a piece may slide
multiple cells in one unit move) also returns~$81$.
The 0/1~cost structure makes intermediate slides free:
a piece that slides two cells costs the same as one that
slides one cell---both cost zero if the piece was already
engaged, one if it is a new engagement.

(ii)--(iii)~Path extraction from the 0/1~BFS parent
pointers.
\end{proof}

\begin{remark}[Three counting conventions]\label{rem:counting}
\begin{center}
\small
\begin{tabular}{@{}llcl@{}}
\toprule
\textbf{Convention} & \textbf{Definition} &
  \textbf{Count} & \textbf{Algorithm} \\
\midrule
Unit moves & One cell, one direction &
  $116$ & Standard BFS \\
Multi-cell moves & One piece, one direction, any dist.\ &
  $90$ & Standard BFS \\
步 & One piece, any direction, any dist.\ &
  $81$ & 0/1~BFS \\
\bottomrule
\end{tabular}
\end{center}
The 步-count is minimal because it reflects the
agent's decisions: which piece to engage next.
This is the natural cost in agentic theory---the number
of discrete choices, not the number of physical slides.
\end{remark}

\subsubsection*{Phase decomposition}

\begin{proposition}[Phase decomposition: $9 + 72$]%
\label{prop:phase-decomp}
The $81$-步 solution decomposes into $9$~phases,
separated by the $9$~king steps.
水 rearranges between each king move:
\begin{center}
\small
\begin{tabular}{@{}ccrll@{}}
\toprule
\textbf{Phase} & \textbf{水-steps} &
  \textbf{King step} & \textbf{Direction} &
  \textbf{King position} \\
\midrule
1 & $25$ & \#26 & $\to$ & $(2, 3)$ \\
2 & $5$  & \#32 & $\leftarrow$ & $(1, 3)$ \\
3 & $8$  & \#41 & $\downarrow$ & $(1, 2)$ \\
4 & $6$  & \#48 & $\downarrow$ & $(1, 1)$ \\
5 & $3$  & \#52 & $\to$ & $(2, 1)$ \\
6 & $7$  & \#60 & $\leftarrow$ & $(1, 1)$ \\
7 & $6$  & \#67 & $\leftarrow$ & $(0, 1)$ \\
8 & $8$  & \#76 & $\downarrow$ & $(0, 0)$ \\
9 & $4$  & \#81 & $\to$ & $(1, 0) = E$ \\
\bottomrule
\end{tabular}
\end{center}
Phase~1 is the longest: $25$~水-steps to clear the path.
This \emph{is} 義釋曹操 (\cref{rem:guanyu}): 水 must
rearrange $25$~times before the king can move once.
\end{proposition}

\begin{remark}[King's freedom]\label{rem:king-freedom}
Along the unit-move optimal path ($116$~moves),
the king has $\geq 1$ legal move in only
$16$ of $116$ configurations ($14\%$).
The remaining $86\%$ of the time, the king is stuck;
水 works to create the next opening.
This is \cref{thm:cutvertex} quantified:
the king is the least mobile agent, and 水 is
the sole source of mobility.
\end{remark}

\subsubsection*{Saddle and mirror descent}

\begin{definition}[Primal--dual distances]%
\label{def:primal-dual}
For each $\sigma \in \mathcal{V}$:
\begin{align}
  d^+(\sigma) &\coloneqq
    d_{\mathcal{G}}^{\text{步}}(\sigma_0,\, \sigma)
    && \text{(forward / primal)}, \label{eq:dplus} \\
  d^-(\sigma) &\coloneqq
    d_{\mathcal{G}}^{\text{步}}(\sigma,\, \sigma_f)
    && \text{(backward / dual)}, \label{eq:dminus}
\end{align}
where $d_{\mathcal{G}}^{\text{步}}$ is the shortest-path
metric in steps (0/1~BFS distances).
\end{definition}

\begin{proposition}[Duality gap]\label{prop:duality-gap}
For all $\sigma$ on any shortest path
$\gamma = (s_0, \ldots, s_{81})$:
\begin{equation}\label{eq:duality-gap}
  d^+(s_j) + d^-(s_j) \;=\; 81
  \qquad \text{for all } 0 \leq j \leq 81.
\end{equation}
The \emph{saddle configuration} is
$\sigma^* \coloneqq s_{k^*}$ where
\[
  k^* \coloneqq \argmin_{0 \leq j \leq 81}
  \bigl|\, d^+(s_j) - d^-(s_j) \,\bigr|.
\]
Since $d^+(s_j) = j$ and $d^-(s_j) = 81 - j$,
the saddle is at step $k^* = 40$.
\end{proposition}

\begin{proof}
Triangle inequality:
$81 = d_{\mathcal{G}}^{\text{步}}(\sigma_0, \sigma_f)
  \leq d^+(s_j) + d^-(s_j)$.
Equality holds because each $s_j$ lies on a
$(\sigma_0, \sigma_f)$-geodesic.
The midpoint $\lfloor 81/2 \rfloor = 40$.
\end{proof}

\begin{theorem}[Saddle $=$ phase transition]\label{thm:saddle}
At the saddle $\sigma^*$ (step~$40$):
\begin{enumerate}[label=(\roman*)]
  \item 関羽 (knife) is at position $(2, 0)$, far from
  the exit corridor---she has cleared.
  \item The 水-mode is H
  (horizontal pair at $\{(1,2), (2,2)\}$),
  enabling the king's next descent.
  \item The king is at $(1, 3)$: it has moved right once
  (step~$26$) and back left once (step~$32$).
  The first descent begins at step~$41$.
\end{enumerate}
To cross the saddle is to perform 義釋曹操
(\cref{rem:guanyu}).
\end{theorem}

\begin{proof}
By \cref{thm:flowcut}, the min-cut separates
blocked configurations (関羽 covers exit corridor)
from open configurations (corridor cleared).
On the optimal path the cut lies where
$d^+ \approx d^-$, i.e., step~$40$.
Direct inspection of the BFS solution confirms:
at $\sigma^*$, 関羽 has exited the corridor.
\end{proof}

\begin{remark}[Mirror descent]\label{rem:mirror}
\Cref{alg:kpc} is mirror descent on a finite graph.
The forward 0/1~BFS computes the primal potential $d^+$;
the backward 0/1~BFS computes the dual potential $d^-$.
At the saddle, $d^+(\sigma^*) = 40$ and
$d^-(\sigma^*) = 41$: the potentials balance.
This is the same forward--backward structure as
\cref{alg:forward,alg:backward}:
the forward pass computes activations (primal),
the backward pass computes gradients (dual),
and the saddle is the phase transition of the loss.
\end{remark}

\subsubsection*{Mode distribution}

\begin{proposition}[Mode statistics]\label{prop:modes}
Along the $81$-步 path, the 水-mode after each step is:
\begin{center}
\small
\begin{tabular}{@{}lcl@{}}
\toprule
\textbf{Mode} & \textbf{Count} & \textbf{Enables} \\
\midrule
H (horizontal) & $27$ & knife, king (horizontal) \\
V (vertical)   & $42$ & generals, king (vertical) \\
S (separated)  & $12$ & soldiers only \\
\bottomrule
\end{tabular}
\end{center}
The dominance of V-mode ($52\%$) reflects the
puzzle's vertical bias: the king must descend $3$~rows.
S-mode ($15\%$) appears when 水 must diffuse to
reposition---it is the ``reset'' phase.
The king moves only in H or V mode, never in S\@.
\end{proposition}

\subsubsection*{The isomorphism, completed}

\Cref{thm:huarongdao} identified the viability roles.
\Cref{thm:eightyone} extends the isomorphism to the
\emph{computational} structure of \cref{sec:calculus}:

\begin{table}[H]
\centering\small
\caption{Agentic calculus on 華容道: complete correspondence.}%
\label{tab:kpc}
\begin{tabular}{@{}lll@{}}
\toprule
\textbf{Calculus (\cref{sec:calculus})} &
\textbf{華容道} &
\textbf{Value} \\
\midrule
Execution graph (\cref{def:exgraph}) &
  Configuration graph $\mathcal{G}$ &
  $25{,}955$ states \\
Agentic flow (\cref{def:flow}) &
  水-flow (free-cell agent) &
  $190/190$ ergodic \\
Min-cut (\cref{thm:flowcut}) &
  関羽 blocks corridor &
  Cleared at step~$40$ \\
Max-flow &
  Optimal 步-path &
  $81$~步 \\
Forward pass (\cref{alg:forward}) &
  $d^+$: primal 0/1~BFS &
  Distance from $\sigma_0$ \\
Backward pass (\cref{alg:backward}) &
  $d^-$: dual 0/1~BFS &
  Distance to $\sigma_f$ \\
Saddle of loss &
  $\sigma^*$: $d^+ \approx d^-$ &
  Step~$40$ \\
Contact modes (\cref{sec:contact}) &
  水-modes (H, V, S) &
  $3$ modes \\
Mobility $\propto 1/s$ &
  King: $9/81$; Soldiers: most &
  Heavy-tailed \\
Spectral gap (\cref{thm:massgap}) &
  $\lambda_1 > 0$ &
  $\mathcal{G}$ connected \\
Horizon $H$ &
  Geodesic in 步 &
  $81$ \\
Phase decomposition &
  $9$~king $+ 72$~水 &
  $25$~步 to clear \\
刀 dissipation (\cref{def:dissipation}) &
  関羽 yields &
  Phase~1: 義釋曹操 \\
水 (\cref{def:water}) &
  Free-cell agent &
  $190/190$ ergodic \\
\bottomrule
\end{tabular}
\end{table}

\begin{remark}[$81$ as horizon]\label{rem:eightyone}
The minimum $81$ is not an arbitrary combinatorial fact.
It is the \emph{horizon} of the predictive controller:
the geodesic length in $\mathcal{G}$ measured in~步,
equivalently the minimum number of
\textcolor{water}{水}~decisions to transport the king from
$\sigma_0$ to the exit.
In the language of \cref{thm:massgap}:
$H = 81$ is the spectral gap made finite.

The three conventions (\cref{rem:counting}) separate
cleanly: unit moves ($116$) count physical slides;
multi-cell moves ($90$) count piece--direction pairs;
步~($81$) count agent decisions.
The agentic theory uses~步 because it counts
\emph{what the agent chooses}, not what the physics does.
\end{remark}

\section{华容道: Complete Instantiation}\label{sec:huarongdao}

The historical cases in \cref{sec:applications} each validate one
concept. We now present a single finite object that instantiates the
\emph{entire} framework simultaneously: the Chinese sliding block
puzzle 华容道 (Huarong Pass).

\subsection{The puzzle}

\begin{definition}[华容道]\label{def:huarongdao}
A \emph{华容道 instance} is a tuple
$(\mathcal{B},\, \mathcal{P},\, p_*,\, E)$:
\begin{enumerate}[label=(\roman*)]
  \item $\mathcal{B} = [m] \times [n]$: rectangular grid (the
  \emph{board}, 方);
  \item $\mathcal{P} = \{p_1, \ldots, p_k\}$: rectangular blocks
  placed non-overlapping on $\mathcal{B}$ (the \emph{pieces}, 圆);
  \item $p_* \in \mathcal{P}$: distinguished piece (the \emph{king});
  \item $E \subset \partial\mathcal{B}$: boundary region congruent
  to $p_*$ (the \emph{exit}).
\end{enumerate}
The \emph{free cells} $\mathcal{F} = \mathcal{B} \setminus
\bigcup_i p_i$ are the system's degrees of freedom. A
\emph{configuration} is a valid placement of all pieces. A \emph{move}
is a unit translation of one piece into adjacent free cells. The
\emph{configuration graph} $\mathcal{G} = (\mathcal{V}, \mathcal{E})$
has configurations as vertices and legal moves as edges.

The puzzle: does there exist a path in $\mathcal{G}$ from $\sigma_0$
to any $\sigma_f$ with $\sigma_f(p_*) = E$?
\end{definition}

The standard instance is $\mathcal{B} = [4] \times [5]$ in the
configuration 横刀立马 (``horizontal knife, standing horse''):

\begin{center}
\begin{tikzpicture}[scale=0.85]
  % Grid
  \draw[gray!40, thin] (0,0) grid (4,5);
  \draw[very thick] (0,0) rectangle (4,5);
  % Exit
  \draw[very thick, densely dashed] (1,-0.05) -- (1,-0.4) -- (3,-0.4)
    -- (3,-0.05);
  \node[font=\footnotesize] at (2,-0.65) {$E$ (exit)};
  % 曹操 (2×2) — the king
  \fill[black!80] (1.05,3.05) rectangle (2.95,4.95);
  \node[white, font=\bfseries\large] at (2,4.2) {曹操};
  \node[white, font=\scriptsize] at (2,3.5) {$p_*\;(2{\times}2)$};
  % 关羽 (2×1 horizontal) — the knife
  \fill[black!55] (1.05,2.05) rectangle (2.95,2.95);
  \node[white, font=\small\bfseries] at (2,2.5) {关羽 $(2{\times}1)$};
  % Generals (1×2 vertical)
  \fill[black!35] (0.05,3.05) rectangle (0.95,4.95);
  \node[white, font=\small, rotate=90] at (0.5,4) {张飞};
  \fill[black!35] (3.05,3.05) rectangle (3.95,4.95);
  \node[white, font=\small, rotate=90] at (3.5,4) {赵云};
  \fill[black!35] (0.05,1.05) rectangle (0.95,2.95);
  \node[white, font=\small, rotate=90] at (0.5,2) {马超};
  \fill[black!35] (3.05,1.05) rectangle (3.95,2.95);
  \node[white, font=\small, rotate=90] at (3.5,2) {黄忠};
  % Soldiers (1×1)
  \fill[black!12] (1.05,1.05) rectangle (1.95,1.95);
  \draw[black!50] (1.05,1.05) rectangle (1.95,1.95);
  \node[font=\small] at (1.5,1.5) {兵};
  \fill[black!12] (2.05,1.05) rectangle (2.95,1.95);
  \draw[black!50] (2.05,1.05) rectangle (2.95,1.95);
  \node[font=\small] at (2.5,1.5) {兵};
  \fill[black!12] (0.05,0.05) rectangle (0.95,0.95);
  \draw[black!50] (0.05,0.05) rectangle (0.95,0.95);
  \node[font=\small] at (0.5,0.5) {兵};
  \fill[black!12] (3.05,0.05) rectangle (3.95,0.95);
  \draw[black!50] (3.05,0.05) rectangle (3.95,0.95);
  \node[font=\small] at (3.5,0.5) {兵};
  % Free cells
  \draw[densely dashed, black!40] (1.05,0.05) rectangle (1.95,0.95);
  \node[black!50, font=\footnotesize] at (1.5,0.5) {$\varnothing$};
  \draw[densely dashed, black!40] (2.05,0.05) rectangle (2.95,0.95);
  \node[black!50, font=\footnotesize] at (2.5,0.5) {$\varnothing$};
  % Legend
  \begin{scope}[font=\footnotesize, anchor=west]
    \fill[black!80] (4.7,4.65) rectangle (5.0,4.85);
    \node at (5.15,4.75) {王: $p_*$ $(2{\times}2)$};
    \fill[black!55] (4.7,4.15) rectangle (5.0,4.35);
    \node at (5.15,4.25) {刀: 关羽 $(2{\times}1)$};
    \fill[black!35] (4.7,3.65) rectangle (5.0,3.85);
    \node at (5.15,3.75) {将: generals $(1{\times}2)$};
    \fill[black!12] (4.7,3.15) rectangle (5.0,3.35);
    \draw[black!50] (4.7,3.15) rectangle (5.0,3.35);
    \node at (5.15,3.25) {卒: soldiers $(1{\times}1)$};
    \draw[densely dashed, black!40] (4.7,2.65) rectangle (5.0,2.85);
    \node at (5.15,2.75) {$\varnothing$: free cells (水)};
  \end{scope}
\end{tikzpicture}
\end{center}

Ten pieces ($4 + 2 + 4 \cdot 2 + 4 \cdot 1 = 18$ cells), two free
cells, exit at bottom center (width~$2$). The minimum solution requires
81~moves.

\subsection{The isomorphism}

\begin{theorem}[华容道 $=$ viability maintenance]\label{thm:huarongdao}
The standard 华容道 instance instantiates the viability framework:
\begin{center}
\begin{tabular}{@{}lp{4cm}p{5cm}@{}}
\toprule
\textbf{Framework} & \textbf{华容道} & \textbf{Mechanism} \\
\midrule
Viability axiom & Path $\sigma_0 \to \sigma_f$ in $\mathcal{G}$ &
King must reach exit \\
King & 曹操 $(2{\times}2)$ & Least mobile, highest importance \\
Knife (\cref{def:knife}) & 关羽 $(2{\times}1)$ & Blocks exit;
autonomous; observable \\
Pawns & Soldiers $(1{\times}1)$ & Most mobile, lowest importance \\
Cut vertex & Free cells $\mathcal{F}$ &
$\mathcal{F} = \varnothing \Rightarrow$ frozen \\
Phase transition & 关羽 clears corridor & Before: blocked. After:
path opens \\
Water & Free-cell flow & Slides opposite to piece movement \\
$w = 0$ collapse & Zero free cells & No flow $\to$ no path $\to$
dead \\
\bottomrule
\end{tabular}
\end{center}
\end{theorem}

\begin{proof}
\emph{Viability.} The puzzle asks exactly \cref{ax:viability}: does a
path exist from $\sigma_0$ through $\mathcal{G}$ to a goal state? The
configuration graph is the execution graph; each edge is a legal move;
the viable kernel is the set of configurations from which the exit
remains reachable.

\emph{Knife.} 关羽 satisfies both conditions of \cref{def:knife}:
(1)~autonomous actuation---he can slide independently of 曹操---and
(2)~observability---his position visibly blocks the exit corridor.
He must yield for the king to pass.

\emph{Cut $=$ free cell.} Fill both free cells and $\mathcal{G}$ has
no edges: every configuration is isolated. The free cell inverts the
cut vertex: ``the absence whose removal freezes.'' One structural
element controls all connectivity.

\emph{Mobility $\propto 1/\text{size}$.} A piece of size $s$ needs $s$
aligned free cells to move. Soldiers ($s = 1$): one free cell.
曹操 ($s = 4$): two aligned free cells. The king is the least mobile
agent---\cref{thm:cutvertex} made physical.

\emph{Water.} When a piece slides left, the free cell moves right.
The free cell flows in the opposite direction---it \emph{is} the water
(\cref{def:water}). Each move transfers the free cell to a new
position, enabling the next move. The 81-move solution is a flow of
water through 81 channels.
\end{proof}

\begin{remark}[横刀立马: the name is the theorem]\label{rem:hrdname}
The configuration's traditional name means ``horizontal knife, standing
horse.'' 刀~(knife) $=$ 关羽 blocking horizontally; 马~(horse) $=$
generals standing vertically. Chinese game designers named the
configuration by its structural properties---the framework's vocabulary,
centuries before graph theory.
\end{remark}

\begin{remark}[义释曹操]\label{rem:guanyu}
In the \emph{Romance of the Three Kingdoms}, 关羽 is stationed at
华容道 after the Battle of Red Cliffs (208~CE). 曹操 retreats through.
关羽 has both conditions of \cref{def:knife}: autonomous actuation
(his army) and observability (曹操 approaching in plain sight). He
chooses path~(a): 义~(righteousness) overrides 忠~(loyalty to Liu Bei),
and he sets $\Ur \to \varnothing$ voluntarily.

The puzzle encodes this: every solution requires moving 关羽 aside.
To solve 华容道 is to perform 义释曹操.
\end{remark}

\begin{remark}[方圆 $\times$ 黑白]\label{rem:fangyuan}
The puzzle's $2 \times 2$ classification is formalized in
\cref{def:fangyuan}. The four statics classify every cell and piece;
the dynamics---刀 (boundary/cut) and 水 (flow/transport)---emerge from
the $2 \times 2$ as the calculus operations of \cref{sec:calculus}.
\end{remark}

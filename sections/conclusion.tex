\section{Conclusion}\label{sec:conclusion}

We have presented an agentic theory of viability maintenance built on a
single axiom (the existence of a viable path to infinity) and a
two-condition criterion (the knife). The framework produces three main
theorems (binary lifecycle, fixed-point impossibility, unconstrained
power paradox) and a central interpretive result: the knife is the mean
field.

The knife is not an intrinsic property of a resource. It is a
statistical deviation from the system's mean autonomous actuation,
made visible by the detection function and made dangerous by the
viability axiom. Phase transitions shift the mean, not the individual.
The king responds to the mean, not to intent.

The agentic calculus (\cref{sec:calculus}) translates this theory into
an operational language: every theorem becomes a flow-theoretic
proposition, the knife becomes the min-cut, and the viable path becomes
the max-flow. The duality ``the knife is the mean'' is a restatement of
max-flow/min-cut duality on the execution graph.

This reframing connects viability maintenance to mean-field theory in
statistical mechanics, where phase transitions are driven by shifts in
the order parameter rather than changes in individual configurations.
The viability axiom plays the role of free energy minimization; the
knife plays the role of the critical fluctuation.

Two millennia of Chinese imperial history validate the framework with
unusual clarity. The same structure appears---with instructive
breaks---in the Atlantic slave trade, ideological hatred, parasitic
network topologies, and militarism. The framework's failure conditions
(\cref{sec:domain}) are as informative as its successes: they delineate
the boundary between systems where the viability axiom operates cleanly
and systems where it is dominated by other dynamics.

The knife is the mean. Viability maintenance is a mean-field phenomenon.
The theory is agentic because the agents---not their intentions, not
their narratives, not their moral qualities, but their structural
positions in the execution graph---determine the outcome.

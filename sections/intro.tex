\section{Introduction}\label{sec:intro}

Consider a system with $n$ agents $a_1, \ldots, a_n$, each possessing a
control set $U_i$ that determines what actions they can take
independently. One agent---the \emph{principal agent} or
\emph{king}---holds maximal authority: $U = \Umax$. The king's sole
objective is survival, formalized as the existence of a viable path from
every reachable state to infinity.

This paper asks: \emph{what structural features of the system force the
king to eliminate other agents?} The answer turns out to be a two-condition
criterion we call the \emph{knife}. A resource is a knife if and only if
it can actuate independently of the king \emph{and} the king can observe
it. This criterion is not chosen---it is the unique logical consequence
of the viability axiom in a multi-agent environment.

The framework yields several results that are surprising in their
precision. The lifecycle of every knife is binary: it is either
voluntarily relinquished (path~(a)) or forcibly eliminated (path~(b)).
There is no third option---attempting to ``prove your knife is not a
knife'' is a fixed-point impossibility. Unconstrained power is not
freedom but a perpetual elimination machine: the more control the king
has, the more knives he must cut.

The central thesis of this paper is that the knife is the
\emph{mean field} of the system. The threshold separating tools from
threats is not absolute but statistical: it is determined by the average
autonomous actuation level across all agents. When the system's mean
field shifts (e.g., from wartime to peacetime), the same resource
changes classification without any change in its physical properties.
This makes viability maintenance a mean-field phenomenon: the king does
not respond to individual threats but to deviations from the system mean.

We validate the framework against two millennia of Chinese imperial
history, where the viability axiom operated with unusual clarity due to
the concentration of sovereignty in a single agent. The framework
correctly classifies the fates of historical figures---Han~Xin (knife,
eliminated), Xiao~He (half-knife, blunted), Zhang~Liang (not a knife,
survived)---and explains structural phenomena such as the Qin unification
as a graph-theoretic optimization. Extensions to the Atlantic slave trade,
the structure of ideological hatred, parasitic network topologies, and
militarism demonstrate that the framework applies beyond its original
historical domain.

\paragraph{Organization.}
\Cref{sec:framework} introduces the viability axiom, the knife
definition, and the phase transition.
\Cref{sec:results} presents the main theorems.
\Cref{sec:meanfield} develops the mean-field interpretation.
\Cref{sec:applications} applies the framework to historical and
contemporary systems.
\Cref{sec:discussion} discusses the domain of applicability and the
dynamics of the population (the ``water'').
\Cref{sec:calculus} develops an operational calculus---the
\emph{agentic calculus}---that translates every theorem into a
flow-theoretic proposition.
\Cref{sec:huarongdao} instantiates the complete framework in a single
finite object.

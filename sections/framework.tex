\section{The Framework}\label{sec:framework}

\subsection{The viability axiom}\label{sec:axiom}

Let $S$ denote the state space, $U$ the control set of the principal
agent, and $K \subset S$ the \emph{viability kernel}---the set of states
in which the king retains supreme authority:
\[
  K = \bigl\{\, s \in S \;:\; \text{the king retains supreme authority}
  \,\bigr\}.
\]

\begin{axiom}[Viability]\label{ax:viability}
For every state $s \in S$, there exists a viable path
$\gamma: [0, \infty) \to K$ with $\gamma(0) = s$:
\[
  \forall\, s \in S,\quad
  \exists\;\gamma: s \to \infty
  \quad\text{such that}\quad
  \gamma(t) \in \Viab(K) \;\;\forall\, t \geq 0.
\]
\end{axiom}

When $U = \Umax$, the control set is full and this path is
mathematically guaranteed to exist. The question is: \emph{what can
break it?}

In a multi-agent system with agents $a_1, \ldots, a_n$, each having
control set $U_i$, the answer is unique: an actuator whose output can
push the king's state out of $\Viab(K)$, and which can execute
\emph{independently of the king}. If the actuator's execution must pass
through the king, the king can intercept. If it can bypass the king, the
king's $\Umax$ cannot react in time.

\begin{remark}[Scope]\label{rem:scope}
This paper presents a compressed model of the topology of execution
capability under the viability axiom. It deliberately ignores culture,
personality, economics, and moral narrative in exchange for a testable
structural criterion. Historical cases are used to validate the
criterion's discriminating power, not to claim the criterion exhausts
history.
\end{remark}

\subsection{The knife}\label{sec:knife}

\begin{definition}[Knife]\label{def:knife}
A resource $r$ is a \emph{knife} if it satisfies two conditions:
\begin{enumerate}[label=(\arabic*)]
  \item \textbf{Autonomous actuation.} The resource can operate
  independently of the king. Formally, there exists an action
  $a \in \Ur$ such that the execution function $f_r$ satisfies
  \[
    f_r(s, a) \notin K
    \quad\text{and}\quad
    a \text{ does not require the king's authorization.}
  \]
  \item \textbf{Observability.} The king's detection function $\Obs$
  can observe $r$ and its execution capability:
  $r \in \mathrm{Im}(\Obs)$.
\end{enumerate}
\end{definition}

The classification is exhaustive:
\begin{itemize}
  \item Condition~(1) not satisfied: \textbf{not a knife}. (Zhang
  Liang's strategic counsel---a pure function that cannot execute
  itself.)
  \item Condition~(2) not satisfied: \textbf{hidden knife}. (More
  dangerous, but outside the king's strategy space. Unobservable
  $=$ indefensible $=$ system noise.)
  \item Both satisfied: \textbf{knife}.
\end{itemize}

\begin{remark}[Intent is irrelevant]\label{rem:intent}
The criterion tests \emph{capability}, not \emph{intention}. The king
detects whether you \emph{can} act, not whether you \emph{want to}.
Loyalty does not enter the criterion.
\end{remark}

\begin{remark}[Logical necessity]\label{rem:necessity}
These two conditions are not chosen by the modeler. They are the unique
logical consequence of the viability axiom $+$ unconstrained power $+$
multi-agent environment.
\end{remark}

\subsection{Phase transition}\label{sec:phase}

The knife is a \emph{phase function}, not an intrinsic property.

\begin{proposition}[Phase-dependent labeling]\label{prop:phase}
The same resource $r$ receives different labels under different system
phases $\varphi$:
\[
  \mathrm{Label}(r, \varphi) =
  \begin{cases}
    \textbf{tool} & \text{if } \varphi = \text{wartime (king needs }
    r\text{'s actuation),} \\
    \textbf{knife} & \text{if } \varphi = \text{peacetime (king no
    longer needs } r\text{, but } r \text{ persists).}
  \end{cases}
\]
The phase transition does not change the physical properties of $r$.
It changes the king's objective function $J(s, \varphi)$.
\end{proposition}

\begin{proof}
In wartime, the king's objective $J_{\mathrm{war}}$ includes terms
where $r$'s actuation has positive utility. In peacetime,
$J_{\mathrm{peace}}$ optimizes for long-term survival
($\exists\;\text{path to } \infty$), and the same actuation becomes a
boundary threat on $\Viab(K)$. The resource $r$ is unchanged;
the labeling function $\mathrm{Label}(r, \varphi)$ is what shifts.
\end{proof}

\subsection{The cut vertex principle}\label{sec:cutvertex}

\begin{definition}[Cut vertex]\label{def:cutvertex}
In the execution graph $G = (V, E)$ of the system, a vertex
$v \in V$ is a \emph{cut vertex} if $G \setminus \{v\}$ is
disconnected. An agent who is a cut vertex controls all execution
chains: removing them disconnects the system.
\end{definition}

\begin{theorem}[Cut vertex $\neq$ maximum actuator]\label{thm:cutvertex}
The optimal survival strategy for the king is to be a cut vertex, not
the maximum actuator. That is, the king maximizes viability by
ensuring all execution chains pass through him, rather than by
maximizing his own actuation.
\end{theorem}

\begin{proof}
A maximum actuator $v^*$ with $\|U_{v^*}\| = \max_i \|U_i\|$
suffers from three structural defects:
\begin{enumerate}[label=(\roman*)]
  \item \emph{Non-scalability}: a single actuator cannot cover the
  full state space simultaneously.
  \item \emph{Single point of failure}: $\Viab(K)$ depends entirely
  on $v^*$'s performance; one failure collapses the system.
  \item \emph{Self-referential paradox}: if the king \emph{is} the
  knife (the strongest autonomous actuator), he cannot perform
  viability maintenance on himself.
\end{enumerate}
A cut vertex $v_c$ with $\|U_{v_c}\| \approx 0$ but routing
authority over all chains avoids all three: the system is scalable
(add more actuators), fault-tolerant (one actuator's failure does
not disconnect the graph), and the king is structurally distinct
from the knives he must manage.
\end{proof}

\begin{example}[Liu Bang vs.\ Xiang Yu]\label{ex:liubang}
Xiang Yu was the strongest actuator in the late Qin system
(\emph{Shiji}: ``he could lift a bronze tripod''). His strategy:
$U = \Umax$ through personal combat. Liu Bang had near-zero
personal actuation but made himself the cut vertex of the execution
graph: Han Xin's armies needed Liu Bang's legitimacy, Xiao He's
administration needed his authorization, Zhang Liang's counsel
needed him to listen.

After the phase transition (founding of the Han dynasty), Liu Bang
executed precise viability maintenance: killed Han Xin (knife),
imprisoned then released Xiao He (blunted half-knife), left Zhang
Liang alone (not a knife). Xiang Yu, the maximum actuator, died at
Gaixia---a single actuator cannot cover the full state space.
\end{example}

\subsection{Case analysis: the three fates}\label{sec:cases}

The framework's discriminating power is tested against three figures from
the Han founding (c.~202~BCE), all subordinates of the same king (Liu
Bang), operating in the same post-unification phase:

\begin{center}
\begin{tabular}{@{}lcccc@{}}
\toprule
\textbf{Agent} & $\Ur$ & $\mathrm{Im}(\Obs)$ &
\textbf{Classification} & \textbf{Fate} \\
\midrule
Han Xin & $\neq \varnothing$ (military) & Yes & Knife &
Path~(b): eliminated \\
Xiao He & $\neq \varnothing$ (admin) & Yes & Half-knife &
Path~(a): self-blunted \\
Zhang Liang & $= \varnothing$ (counsel) & Yes & Not a knife &
Survived \\
\bottomrule
\end{tabular}
\end{center}

All three are visible ($r \in \mathrm{Im}(\Obs)$). The discriminant is
condition~(1): can the resource actuate independently?

\begin{example}[Han Xin: pure knife]\label{ex:hanxin}
Han Xin commanded armies that obeyed \emph{him}, not Liu Bang. His
execution chain was closed: he could mobilize, march, and fight without
the king's authorization. Both conditions of \cref{def:knife} satisfied.
After the phase transition, the knife criterion triggered and Liu Bang
eliminated him. Han Xin's quoted proverb (``when the hare dies, the dog
is cooked'') correctly identified path~(b) but failed to act on it---he
understood the classification but not that the only exit was path~(a).
\end{example}

\begin{example}[Xiao He: self-blunting]\label{ex:xiaohe}
Xiao He administered the capital and controlled grain supply---autonomous
actuation at the logistical level. The king observed this
($r \in \mathrm{Im}(\Obs)$), making Xiao He a knife by
\cref{def:knife}. Xiao He's response: deliberate self-corruption
(accepting bribes conspicuously). This performed two operations
simultaneously:
\begin{enumerate}[label=(\roman*)]
  \item \emph{Signal reduction}: visible moral degradation signals
  $\|U_r\| \to 0$ (an official this corrupt cannot coordinate a revolt).
  \item \emph{Mean-field alignment}: pull $\|U_r\|$ toward $\bar{U}$,
  falling below the detection threshold (\cref{thm:meanfield}).
\end{enumerate}
This is path~(a) executed through reputation rather than resignation.
\end{example}

\begin{example}[Zhang Liang: structural safety]\label{ex:zhangliang}
Zhang Liang was a strategist. Strategy is a pure function: it
\emph{advises} action but cannot \emph{execute} it. Zhang Liang's
counsel required Liu Bang's decision, Liu Bang's generals, and Liu
Bang's administration to produce any effect. Every execution chain
passed through the king (\cref{cor:breakpoint}). Result:
$\Ur = \varnothing$, condition~(1) fails, not a knife.
Zhang Liang retired and survived.
\end{example}

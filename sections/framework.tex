\chapter{The Framework}\label{sec:framework}

\section{The viability axiom}\label{sec:axiom}

Let $S$ denote the state space, $U$ the control set of the principal
agent, and $K \subset S$ the \emph{viability kernel}---the set of states
in which the king retains supreme authority:
\[
  K = \bigl\{\, s \in S \;:\; \text{the king retains supreme authority}
  \,\bigr\}.
\]

\begin{axiom}[Viability]\label{ax:viability}
For every state $s \in K$, there exists a viable path
$\gamma: [0, \infty) \to K$ with $\gamma(0) = s$:
\[
  \forall\, s \in K,\quad
  \exists\;\gamma: s \to \infty
  \quad\text{such that}\quad
  \gamma(t) \in \Viab(K) \;\;\forall\, t \geq 0.
\]
\end{axiom}

\begin{remark}\label{rem:viab-subset}
$\Viab(K) \subseteq K$ is the largest closed subset of~$K$ from
which a viable path exists.  States in $K \setminus \Viab(K)$
satisfy the static criterion (the king retains authority) but
lack a dynamic escape: no trajectory from such states remains
in~$K$ forever.  The viability axiom demands that the king
operates within $\Viab(K)$, not merely within~$K$.
\end{remark}

When $U = \Umax$, the control set is full and this path is
mathematically guaranteed to exist: the differential inclusion
reduces to $x' \in C(x)$ (the king's own control set), and since
$K$ is defined as the set of states where the king retains
authority, $C(x)$ spans $T_K(x)$ at every
$x \in K^\circ$---the king would not include in $K$ states he
cannot control.  The tangential condition holds trivially.

The question is: \emph{what can break it?}

In a multi-agent system with agents $a_1, \ldots, a_n$, each having
control set $U_i$, the answer is unique: an actuator whose output can
push the king's state out of $\Viab(K)$, and which can execute
\emph{independently of the king}. If the actuator's execution must pass
through the king, the king can intercept. If it can bypass the king, the
king's $\Umax$ cannot react in time.

\begin{axiom}[Massless agents~---~灵魂没有重量]\label{ax:massless}
Agents have no intrinsic mass.  The capacity $c(e)$ of an edge $e$
in the execution graph is a function of the agents' positions in the
graph---their autonomous actuation sets and their connectivity to the
king---not an intrinsic property of any agent.  Formally, for
agents~$a_i, a_j$ with actuation sets $U_i, U_j$:
\[
  c(a_i, a_j) \;=\; c\bigl(\|U_i\|,\; \|U_j\|,\;
  \mathbf{1}[(a_i, a_j) \text{ bypasses } \kappa]\bigr).
\]
\end{axiom}

The massless axiom says that capacity is \emph{positional}, not
\emph{personal}: it measures what an agent can do from where it
stands, not what it ``weighs.''  This excludes intrinsic agent-specific
mass as a source of edge capacity.

\begin{lemma}[Capacity assignment]\label{lem:capacity}
Under the massless axiom (\cref{ax:massless}), the bypass capacity
of agent~$r$ with actuation set~$\Ur$ is
\[
  c(e_r) \;=\; \max\bigl(\|\Ur\| - \bar{U},\; 0\bigr),
\]
where $\|\Ur\|$ is the degrees of freedom of~$r$'s autonomous
control set (see \cref{sec:meanfield} for the formal definition)
and $\bar{U} = \frac{1}{n}\sum_{i=1}^{n}\|U_i\|$ is the mean
autonomous actuation.
\end{lemma}

\begin{proof}
By \cref{ax:massless}, $c(e_r)$ depends only on $\|\Ur\|$ and
whether $e_r$ bypasses~$\kappa$.  For a non-bypass edge, $c = 0$ by
definition.  For a bypass edge, $c(e_r)$ must measure the
\emph{excess} actuation that $r$ can transmit without the king's
authorisation.

The baseline against which ``excess'' is measured is the mean
field~$\bar{U}$.  This is the unique choice forced by the massless
axiom: since agents carry no intrinsic mass, the total deviation
from any baseline~$b$ must net to zero across the population,
$\sum_i (\|U_i\| - b) = 0$, which holds if and only if
$b = \bar{U}$.  Any other baseline would create a systematic bias
that the viability axiom cannot tolerate (the king would
systematically over- or under-estimate the threat field).

Any actuation at or below $\bar{U}$ is therefore indistinguishable
from the collective and carries zero effective bypass capacity.
Only the surplus $\|\Ur\| - \bar{U}$ contributes.  Clamping at zero
(agents below the mean carry no bypass capacity) gives the stated
formula.
\end{proof}

\begin{remark}[Scope]\label{rem:scope}
This paper presents a compressed model of the topology of execution
capability under the viability axiom. It deliberately ignores culture,
personality, economics, and moral narrative in exchange for a testable
structural criterion. Historical cases are used to validate the
criterion's discriminating power, not to claim the criterion exhausts
history.
\end{remark}

\section{The differential inclusion}\label{sec:di}

The viability axiom (\cref{ax:viability}) asserts the existence of a
viable path.  We now make the dynamics precise using the differential
inclusion framework of Aubin and Cellina~\cite{aubin,aubincellina}.

\begin{definition}[Differential inclusion]\label{def:di}
The state $x(t) \in S$ evolves under a \emph{differential inclusion}
\[
  x'(t) \;\in\; F\bigl(x(t)\bigr),
\]
where $F: S \rightrightarrows S$ is a set-valued map that associates
to each state the set of feasible velocities.  A trajectory
$x(\cdot)$ is an absolutely continuous function satisfying the
inclusion for almost all $t \geq 0$.
\end{definition}

The set-valued map $F$ encodes the fact that the system's velocity is
not uniquely determined by its state: multiple agents, each with their
own actuation, contribute competing directions.  In a multi-agent
system with agents $a_1, \ldots, a_n$, the aggregate velocity set is
\[
  F(x) \;=\; \Bigl\{\, \sum_{i} f_i(x, u_i) \;:\;
  u_i \in U_i \,\Bigr\},
\]
where $f_i$ is agent $i$'s dynamics and $U_i$ its control set.

\begin{definition}[Contingent cone]\label{def:contingent}
The \emph{contingent cone} (Bouligand tangent cone) to $K$ at
$x \in K$ is
\[
  T_K(x) \;:=\; \Bigl\{\, v \in S \;:\;
  \liminf_{h \to 0^+} \frac{d_K(x + hv)}{h} = 0 \,\Bigr\},
\]
where $d_K$ denotes the distance to $K$.  Equivalently,
$v \in T_K(x)$ if and only if there exist sequences $h_n \to 0^+$
and $v_n \to v$ such that $x + h_n v_n \in K$ for all $n$.
\end{definition}

The contingent cone $T_K(x)$ is the set of directions at $x$ along
which the state can move while remaining in $K$---the set of
\emph{safe velocities}.  At interior points $T_K(x) = S$; at
boundary points the cone narrows, restricting the feasible
directions.

\begin{theorem}[Viability theorem {\cite{aubin}}]\label{thm:viability-di}
Let $K \subset S$ be locally compact and $F$ be upper semicontinuous
with nonempty compact convex values.  The necessary and sufficient
condition for the existence of a viable trajectory of
$x' \in F(x)$ from every initial state $x_0 \in K$ is the
\emph{tangential condition}:
\[
  \forall\, x \in K, \quad
  F(x) \;\cap\; T_K(x) \;\neq\; \varnothing.
\]
\end{theorem}

This is the differential content of \cref{ax:viability}: the king
survives if and only if, at every state, the system's velocity set
(\cref{def:di}) contains at least one direction tangent to the
viability kernel (\cref{def:contingent}).
The viability axiom is not a wish---it is the tangential condition.

\begin{definition}[Feedback map]\label{def:feedback}
Given the dynamics $x' = f(x, u)$ with control set $U$ (the king's
controls), the \emph{feedback map} is
\[
  C(x) \;:=\; \bigl\{\, u \in U \;:\;
  f(x, u) \in T_K(x) \,\bigr\}.
\]
A viable trajectory exists from $x$ if and only if $C(x) \neq
\varnothing$.  The regulation problem is: does there exist a
feedback law $u(t) \in C(x(t))$ such that $x(\cdot)$ remains in $K$?
\end{definition}

The feedback map is the king's strategy space at state $x$: the set
of controls that keep the trajectory inside $K$.  When $C(x) \neq
\varnothing$ for all $x \in K$, the king can regulate the system.
When a sword $r$ executes independently, its \emph{execution function}
$f_r: S \times \Ur \to TS$ maps a state--action pair to a velocity
vector.  The velocity $f_r(x, a)$ may push $x'(t)$ outside $T_K(x)$,
and the feedback map shrinks---possibly to the empty set.

\begin{definition}[Viability Lyapunov function]\label{def:lyapunov}
A continuous function $V: K \to \R_{\geq 0}$ is a \emph{Lyapunov function}
for the inclusion $x' \in F(x)$ with respect to a cost function
$W: \mathrm{Graph}(F) \to \R_{\geq 0}$ if
\[
  \forall\, x \in K, \quad
  \exists\, v \in F(x) \;\;\text{such that}\;\;
  D^+ V(x)(v) \;+\; W(x, v) \;\leq\; 0,
\]
where $D^+ V(x)(v) := \liminf_{h \to 0^+,\, u \to v}
\frac{V(x + hu) - V(x)}{h}$ is the upper contingent derivative.
Trajectories satisfying this condition are \emph{monotone}: $V(x(t))$
is non-increasing.
\end{definition}

In the language of \cref{sec:water}, the water level $w(t)$---the
population's aggregate extractable resource (formalised in
\cref{def:water})---is a Lyapunov function (\cref{def:lyapunov})
for the viability inclusion.
The monotonicity condition $D^+ V(x)(v) + W(x,v) \leq 0$ says: along
any viable trajectory, the water level cannot increase faster than the
system's extraction cost $W$.  When $V(x(t)) \to 0$, the Lyapunov
condition fails, the tangential condition is violated, and the system
exits $K$---this is \cref{thm:dumu} (Du Mu's theorem: $w \to 0$
$\implies$ system death; stated formally in \cref{sec:water})
restated in DI language.

\begin{remark}[Temporal linearity in the DI]\label{rem:di-temporal}
The differential inclusion $x'(t) \in F(x(t))$ operates in real
time: $t$ is the wall-clock parameter, and the inclusion must be
satisfied at every instant.  The feedback map $C(x(t))$ must be
evaluated within one time step $\Delta t$ (\cref{rem:temporal}).
The tangential condition $F(x) \cap T_K(x) \neq \varnothing$
is a \emph{pointwise} requirement: it must hold at every $x \in K$,
which means at every instant.  There is no lookahead, no global
optimisation over future trajectories---only the local tangent
condition, checked in real time.  This is why Aubin calls the
viable system's policy ``opportunism'': the system selects a
feasible velocity from $F(x) \cap T_K(x)$ at each instant, without
planning.  The tangential condition admits regulation maps with
memory; we restrict to memoryless feedback (the king reacts to
current state only) as a modelling choice that matches the historical
evidence.  This restriction is sufficient for the results that
follow; it is not forced by the differential inclusion itself.
\end{remark}

\section{Viability geometry}\label{sec:viab-geom}

The differential inclusion (\cref{sec:di}) provides the dynamics.
We now give the viability kernel $K$ a Riemannian metric, making the
survival problem geometric.  The key observation is that the Lyapunov
function $V$ (\cref{def:lyapunov}) is not merely a scalar indicator of
system health---it is a \emph{conformal factor} that defines the
intrinsic geometry of the viable region.

\begin{definition}[Viability metric]\label{def:viab-metric}
Let $g_S$ denote the ambient metric on the state space $S$ and
$V: K \to \R_{\geq 0}$ the Lyapunov function (\cref{def:lyapunov})
with $V > 0$ on $K^\circ$ and $V = 0$ on $\partial K$.
The \emph{viability metric} on $K^\circ$ is the conformal deformation
\[
  g_V \;:=\; \frac{1}{V(x)^2}\, g_S.
\]
The Riemannian manifold $(K^\circ, g_V)$ is called the
\emph{viability manifold}.
\end{definition}

The conformal factor $V^{-2}$ inflates distances near the boundary
($V \to 0$) and compresses distances in the interior ($V$ large).
A trajectory approaching the boundary must cover infinite
$g_V$-distance in finite ambient time---the boundary is ``at
infinity'' in the viability metric.

\begin{proposition}[Completeness]\label{prop:viab-complete}
Suppose $V(x) \leq C \cdot d_K(x)$ for some $C > 0$
and all $x$ near $\partial K$, where $d_K(x)$ is the ambient
distance to $\partial K$.  Then $(K^\circ, g_V)$ is a complete
Riemannian manifold.
\end{proposition}

\begin{proof}
Let $\gamma: [0, T) \to K^\circ$ be a curve approaching $\partial K$
as $t \to T$.  The $g_V$-length is
\[
  L_{g_V}(\gamma)
  \;=\;
  \int_0^T \frac{\|\gamma'(t)\|_{g_S}}{V(\gamma(t))}\, dt
  \;\geq\;
  \frac{1}{C}
  \int_0^T \frac{\|\gamma'(t)\|_{g_S}}{d_K(\gamma(t))}\, dt.
\]
Since $\|\gamma'(t)\|_{g_S} \geq |d_K(\gamma(t))'|$ by the
triangle inequality, the right-hand side is bounded below by
$(1/C)\int_0^T |d_K'|/d_K\,dt$.  By substitution $u = d_K$,
this becomes $(1/C)\int du/u$, which diverges logarithmically
as $d_K \to 0$.
Hence $\partial K$ is at infinite $g_V$-distance: the Hopf--Rinow
theorem gives completeness.
\end{proof}

\begin{proposition}[Negative curvature]\label{prop:neg-curvature}
Let $\dim S = 2$ and $g_S$ be flat.  If $V$ is superharmonic
($\Delta V \leq 0$, the Lyapunov condition), the Gaussian curvature
of $(K^\circ, g_V)$ satisfies
\[
  \kappa_V
  \;=\;
  V \,\Delta V \;-\; |\nabla V|^2
  \;\leq\;
  -\,|\nabla V|^2
  \;<\; 0
\]
wherever $\nabla V \neq 0$.  In dimension $n \geq 3$, the Ricci
curvature of $g_V = V^{-2} g_S$ satisfies
$\mathrm{Ric}_{g_V} \leq -(n-1)\,|\nabla \log V|^2\, g_V$
under the same superharmonicity condition.
\end{proposition}

\begin{proof}
For a conformal change $\tilde{g} = e^{2\varphi}\, g$ with
$\varphi = -\log V$, the Gaussian curvature in dimension~$2$
transforms as
$\tilde\kappa = e^{-2\varphi}(\kappa_S - \Delta\varphi)$,
where $\kappa_S$ is the ambient curvature.
On a flat background ($\kappa_S = 0$):
\[
  \kappa_V
  \;=\;
  V^2\!\left(-\,\Delta(-\log V)\right)
  \;=\;
  V^2\!\left(\frac{\Delta V}{V} - \frac{|\nabla V|^2}{V^2}\right)
  \;=\;
  V\,\Delta V \;-\; |\nabla V|^2.
\]
Since $\Delta V \leq 0$ (superharmonic) and $|\nabla V|^2 > 0$,
we have $\kappa_V < 0$.  The higher-dimensional statement follows
from the conformal Ricci formula
$\mathrm{Ric}_{\tilde g} = \mathrm{Ric}_g - (n-2)\,
\nabla^2\varphi - [\Delta\varphi + (n-2)\,|\nabla\varphi|^2]\,g$.
\end{proof}

\begin{definition}[Cheeger constant of the viability manifold]
\label{def:cheeger-viab}
The \emph{Cheeger constant} of $(K^\circ, g_V)$ is
\[
  h(K)
  \;:=\;
  \inf_{S}
  \frac{|\partial S|_{g_V}}
  {\min\!\bigl(\mathrm{vol}_{g_V}(A),\,
  \mathrm{vol}_{g_V}(B)\bigr)},
\]
where the infimum is over hypersurfaces $\partial S$ that divide
$K^\circ$ into two open subsets $A$ and $B$, and $|\partial S|_{g_V}$
denotes the $(n-1)$-dimensional volume of $\partial S$ in the
viability metric.
\end{definition}

\begin{theorem}[Cheeger inequality]\label{thm:cheeger-viab}
Let $\lambda_1(K)$ denote the first nonzero eigenvalue of the
Laplace--Beltrami operator on $(K^\circ, g_V)$.  Then
\[
  \lambda_1(K) \;\geq\; \frac{h(K)^2}{4}.
\]
\end{theorem}

\begin{proof}
This is the Riemannian Cheeger inequality~\cite{cheeger}.
The discrete version on the execution graph appears in
\cref{thm:cheeger}.
\end{proof}

\begin{remark}[Poincar\'e half-plane]\label{rem:poincare}
When $K = \{x \in \R^2 : x_2 > 0\}$ (the upper half-plane) and
$V(x) = x_2$ (height above the boundary), the viability metric
$g_V = x_2^{-2}(dx_1^2 + dx_2^2)$ is the Poincar\'e half-plane
model of hyperbolic geometry.  The constant curvature is
$\kappa_V = -1$.  The viability kernel of a dynasty on a flat
state space with water level $V = $ distance to collapse is, in its
intrinsic geometry, \emph{the hyperbolic plane}.
\end{remark}

\begin{remark}[Unique cheapest viable path]\label{rem:cartan-hadamard}
\Cref{prop:neg-curvature} gives $\kappa_V \leq 0$ everywhere.
If $K$ is convex (or more generally, if $K^\circ$ is simply
connected), then by the Cartan--Hadamard theorem $(K^\circ, g_V)$
is a Hadamard manifold: the exponential map
$\exp_x: T_x K^\circ \to K^\circ$ is a diffeomorphism.
In particular, between any two interior states there exists a
\emph{unique geodesic}---a unique cheapest viable path.
There is no ambiguity in the optimal route; the geometry forces it.
Convexity of $K$ is natural: the viability kernel is defined as
the set of states from which a viable path exists
(\cref{ax:viability}), and viability kernels of upper semicontinuous
differential inclusions are closed under convex combinations when
$F$ has convex values (\cref{thm:viability-di}).
\end{remark}

\begin{remark}[Why viability maintenance is hard]
\label{rem:viab-divergence}
Negative curvature means nearby geodesics diverge exponentially:
two trajectories that start $\epsilon$-close separate as
$\sim \epsilon\, e^{\sqrt{|\kappa_V|}\, t}$.  A small perturbation
in the king's initial state produces exponentially different
outcomes.  This is the geometric content of sensitive dependence:
the viability manifold is hyperbolic, so maintaining viability
requires continuous correction at every instant
(\cref{rem:di-temporal}).  The harder the Lyapunov function
decreases ($|\nabla V|$ large), the more negative the curvature,
and the faster trajectories diverge.  Near the boundary
($V \to 0$), the curvature diverges: the last moments before
collapse are the most chaotic.
\end{remark}

\begin{remark}[The water is the metric]\label{rem:water-metric}
The Lyapunov function $V$ (\cref{def:lyapunov})---the water level
of \cref{sec:water}---plays three roles simultaneously:
\begin{enumerate}[label=(\roman*)]
  \item \emph{Scalar}: $V(x)$ measures distance from collapse.
  \item \emph{Conformal factor}: $g_V = V^{-2} g_S$ defines the
  intrinsic geometry of the viability kernel.
  \item \emph{Curvature source}: $\Delta V \leq 0$ forces
  $\kappa_V < 0$, making the geometry hyperbolic.
\end{enumerate}
Du Mu's theorem (\cref{thm:dumu}; $w \to 0 \implies$ substrate
exhaustion $\implies$ system death)---is a
\emph{completeness theorem}: a trajectory reaching
$V = 0$ would traverse infinite $g_V$-distance in finite time,
violating \cref{prop:viab-complete}.  The system must exit $K$
before $V$ reaches zero.  Du Mu is Hopf--Rinow.
\end{remark}

\section{The sword}\label{sec:sword}

\begin{definition}[Sword]\label{def:sword}
A resource $r$ is a \emph{sword} if it satisfies two conditions:
\begin{enumerate}[label=(\arabic*)]
  \item \textbf{Autonomous actuation.} The resource can operate
  independently of the king. Formally, there exists an action
  $a \in \Ur$ such that the execution function $f_r$ satisfies
  \[
    f_r(s, a) \notin K
    \quad\text{and}\quad
    a \text{ does not require the king's authorization.}
  \]
  \item \textbf{Observability.} The king's detection function $\Obs$
  can observe $r$ and its execution capability:
  $r \in \mathrm{Im}(\Obs)$.
\end{enumerate}
\end{definition}

The state-dependent form $f_r(s, a) \notin K$ is the general
definition; the operational shorthand $\Ur \neq \varnothing$
used in subsequent chapters is valid under the framework's
assumptions: the existence of even one such action suffices
for condition~(1).%
\footnote{Formally, condition~(1) quantifies existentially over
$a \in \Ur$.  The shorthand $\Ur \neq \varnothing$ is equivalent
when every non-empty autonomous control set contains at least one
boundary-threatening action, which holds whenever $\Ur$ bypasses
the king (\cref{cor:breakpoint}).}

The classification is exhaustive:
\begin{itemize}
  \item Condition~(1) not satisfied: \textbf{not a sword}. (Zhang
  Liang's strategic counsel---a pure function that cannot execute
  itself.)
  \item Condition~(2) not satisfied: \textbf{hidden sword}
  (formally: \emph{pre-sword}, \cref{def:presword}). More
  dangerous, but outside the king's strategy space. Unobservable
  $=$ indefensible $=$ system noise.
  \item Both satisfied: \textbf{sword}.
\end{itemize}

\begin{definition}[Pre-sword]\label{def:presword}
A resource $r$ is a \emph{pre-sword} if it satisfies condition~(1)
but not condition~(2) of \cref{def:sword}:
\[
  \Ur \neq \varnothing
  \quad\wedge\quad
  r \notin \mathrm{Im}(\Obs).
\]
The pre-sword carries a \emph{causal envelope}: the forward
reachable set of its unobserved actuation,
\begin{equation}\label{eq:causal-envelope}
  \mathrm{Reach}(x, \Ur, T)
  \;:=\;
  \bigl\{\, y \in S
    \;:\; \exists\;\gamma:[0,T] \to S,\;
    \gamma(0) = x,\;\gamma(T) = y,\;
    \gamma'(t) \in F_r(\gamma(t)) \;\forall\,t
  \,\bigr\}.
\end{equation}
The pre-sword is the causal antecedent of the sword: every sword
passes through the pre-sword state, since condition~(1) must hold
before condition~(2) can fire.
\end{definition}

\begin{proposition}[Viability gap]\label{prop:viab-gap}
Let $\Obs$ be the king's detection function.  The king
computes the viability kernel $\Viab(K \,|\, \Obs)$ using
only the agents in $\mathrm{Im}(\Obs)$.  If a pre-sword $r$
exists with $\mathrm{Reach}(x, \Ur, T) \not\subset K$, then
\[
  \Viab(K \,|\, \text{true})
  \;\subsetneq\;
  \Viab(K \,|\, \Obs).
\]
The king overestimates safety: the true viability kernel is strictly
smaller than the one he computes.  The difference is the
\emph{shadow} of the pre-sword's causal envelope on the viability
kernel.
\end{proposition}

\begin{proof}
The king's velocity set $F_{\mathrm{king}}(x)$, and hence his
regulation problem (\cref{def:feedback}), includes feedback against
all observed agents.  The pre-sword $r \notin \mathrm{Im}(\Obs)$
contributes a velocity component $f_r(x, a) \in F_r(x)$ that the
feedback map does not compensate.  At any state $x$ where
$F_r(x) \not\subset T_K(x)$ and the king does not account for $r$,
the tangential condition (\cref{thm:viability-di}) may fail under
$r$'s actuation even though the king's computed regulation
map is non-empty.  Hence there exist states in $\Viab(K \,|\, \Obs)$
that are not in $\Viab(K \,|\, \text{true})$.
\end{proof}

\begin{remark}[The sword lifecycle]\label{rem:sword-lifecycle}
The complete lifecycle of a resource has four states, not two.
Each state occupies a different position in the causal structure
and bears a different relationship to temporal linearity
(\cref{rem:temporal,rem:di-temporal}):
\begin{center}
\begin{tabular}{@{}lccccl@{}}
\toprule
\textbf{State} & \textbf{(1)} $\Ur$ & \textbf{(2)} $\Obs$
& \textbf{Causal envelope} & \textbf{Time} & \textbf{King's view} \\
\midrule
Tool      & $= \varnothing$ & ---
& $\mathrm{Reach} = \{x\}$ & ---
& No threat \\
Pre-sword & $\neq \varnothing$ & $r \notin \mathrm{Im}(\Obs)$
& Non-trivial, \textbf{invisible}
& Before the clock
& Overestimates $\Viab(K)$ \\
Sword     & $\neq \varnothing$ & $r \in \mathrm{Im}(\Obs)$
& Non-trivial, visible
& Clock starts
& Computes correct min-cut \\
Post-sword & \multicolumn{2}{c}{resolved via (a) or (b)}
& Collapsed
& Clock stops
& Threat removed \\
\bottomrule
\end{tabular}
\end{center}
The \textbf{Time} column is load-bearing.  The pre-sword exists in
\emph{causal space}: the reachable set $\mathrm{Reach}(x, \Ur, T)$
is a statement about what \emph{could} happen, not what has
happened.  It is the shadow of a future that has not yet been
observed.  The sword exists in \emph{observation space}: the moment
the king detects the causal envelope, the binary clock starts
ticking---the tangential condition $F(x) \cap T_K(x) \neq
\varnothing$ must now be maintained at every instant, and the
sword must resolve to (a) or (b).  The post-sword exists in
\emph{physics}: it is the irreversible outcome of the system
evolving under $x'(t) \in F(x(t))$ with $t$ as wall-clock time.

\textcolor{sword}{Post-sword is not a logical deduction.
It is the result of stepping forward in linear time until the
system resolves.  You cannot reach post-sword by reasoning,
planning, or declaring.  You reach it by living through the
time steps---non-pausably, non-reversibly, one by one.}

The binary lifecycle (\cref{thm:lifecycle}) applies only to the
sword $\to$ post-sword transition: the clock that starts at
detection and runs until resolution.  The pre-sword state is not a
path~(c)---it is a distinct row in the lifecycle table, prior to
the clock.  A pre-sword is a \emph{latent} sword: its causal
envelope grows monotonically as the agent accumulates resources,
so the pre-sword either self-collapses ($\Ur \to \varnothing$,
becoming a tool) or eventually crosses the detection threshold
($r$ enters $\mathrm{Im}(\Obs)$, becoming a sword, starting
the clock).
\end{remark}

\begin{remark}[The lifecycle of a binary sword]
\label{rem:binary-lifecycle-colour}
\textcolor{sword}{The sword is born in the dark.}
Every sword begins as a pre-sword---condition~(1) satisfied,
condition~(2) not yet.  The causal envelope
$\mathrm{Reach}(x, \Ur)$ expands silently: an official
consolidates supply lines, a general earns the loyalty of
troops, a chancellor accumulates administrative reach.  The
king does not see this.  His computed viability kernel
$\Viab(K \,|\, \Obs)$ is too large; the true kernel is smaller
by the shadow of the envelope he cannot observe
(\cref{prop:viab-gap}).

\textcolor{dao}{The sword is seen at dawn.}
Detection fires: $r$ enters $\mathrm{Im}(\Obs)$.  The
pre-sword becomes a sword.  The binary clock
$t \in [0, \infty)$ starts---irreversible, non-pausable
(\cref{rem:temporal}).  The king now computes the correct
min-cut; the tangential condition must be maintained at
every time step.  From this instant, exactly two futures
exist: (a)~relinquish or (b)~elimination.  The holder
knows this (``when the hare dies\ldots''); the question is
whether he acts on it.

\textcolor{water}{The sword is quenched at dusk.}
Post-sword is physics: the linear time rollout $x'(t)
\in F(x(t))$ runs until one of \{(a),~(b)\} obtains.
Path~(a) is the holder's choice---Xiao He buys bad land,
Fan Li disperses his fortune, Zhang Liang retires to
mountains.  Path~(b) is the king's---Han Xin is executed,
Shang Yang is quartered.  There is no path~(c), because
the clock does not stop: $t$ advances, the tangential
condition is checked at every instant, and the inclusion
$F(x) \cap T_K(x) \neq \varnothing$ must be maintained
or violated.  Maintained $=$ the sword persists and the
clock keeps running.  Violated $=$ the system exits $K$
and the post-sword state is reached.

\textcolor{caution}{The sword that regresses is the most
dangerous of all.}  Xiao He's self-blunting
(\cref{ex:xiaohe}) is not path~(a): it is sword $\to$
pre-sword regression.  He stops the binary clock by exiting
condition~(2)---but his causal envelope does not shrink.
The king, who could see the min-cut a moment ago, now
cannot.  The viability gap reopens.  This is why the
\emph{Shiji} records Liu Bang's suspicion even after
Xiao He's self-corruption: the king knows the envelope is
still there.  He imprisoned Xiao He, then released
him---probing the shadow, unable to resolve it.  The
pre-sword outlived the king.
\end{remark}

\begin{remark}[DI restatement of the sword]\label{rem:sword-di}
In the language of \cref{sec:di}, a resource $r$ is a sword if and
only if its velocity set $F_r(x)$ can generate directions outside
the contingent cone:
\[
  r \text{ is a sword}
  \quad\iff\quad
  \exists\, x \in K \;\;\text{such that}\;\;
  F_r(x) \not\subset T_K(x).
\]
That is, $r$ can push the system's state toward the boundary of $K$
along directions that are \emph{not tangent} to the viability
kernel.  The king's feedback map $C(x)$ (\cref{def:feedback}) can
compensate only if the king's velocity set $F_{\mathrm{king}}(x)$
contains a counteracting direction in $T_K(x)$.  When $r$ executes
independently---bypassing $C(x)$---no compensation is possible,
and the tangential condition (\cref{thm:viability-di}) is violated.
\end{remark}

\begin{remark}[Geometric restatement of the sword]
\label{rem:sword-geom}
In the viability manifold $(K^\circ, g_V)$ (\cref{def:viab-metric}),
the sword has a curvature interpretation.  A sword
(\cref{def:sword}) must satisfy both autonomous actuation
\emph{and} observability; the geometric content lies in
condition~(1).  When a resource $r$ actuates autonomously, its
velocity field $f_r$ has a component along $-\nabla V$ (pointing
toward the boundary).  The magnitude
$|\langle f_r, -\nabla V \rangle|$ contributes directly to the
curvature (\cref{prop:neg-curvature}): the sword increases
$|\nabla V|$, making $\kappa_V$ more negative, and the
viable region more hyperbolic.  More swords $\Rightarrow$ more
negative curvature $\Rightarrow$ faster divergence of nearby
trajectories $\Rightarrow$ harder viability maintenance.
The Cheeger constant $h(K)$ (\cref{def:cheeger-viab}) measures
the worst-case cut: the sword is the hypersurface that minimises the
isoperimetric ratio of the viability manifold.
(Condition~(2)---observability---determines whether the king
\emph{knows} where the cut is, not whether it exists.)
\end{remark}

\begin{remark}[Intent is irrelevant]\label{rem:intent}
The criterion tests \emph{capability}, not \emph{intention}. The king
detects whether you \emph{can} act, not whether you \emph{want to}.
Loyalty does not enter the criterion.
\end{remark}

\begin{remark}[Logical necessity]\label{rem:necessity}
These two conditions are not chosen by the modeler. They are the unique
logical consequence of the viability axiom $+$ unconstrained power $+$
multi-agent environment.
\end{remark}

\section{Phase transition}\label{sec:phase}

The sword is a \emph{phase function}, not an intrinsic property.

\begin{proposition}[Phase-dependent labelling]\label{prop:phase}
The same resource $r$ receives different labels under different system
phases $\varphi$:
\[
  \mathrm{Label}(r, \varphi) =
  \begin{cases}
    \textbf{tool} & \text{if } \varphi = \text{wartime (king needs }
    r\text{'s actuation),} \\
    \textbf{sword} & \text{if } \varphi = \text{peacetime (king no
    longer needs } r\text{, but } r \text{ persists).}
  \end{cases}
\]
The phase transition does not change the physical properties of $r$.
It changes the king's objective function $J(s, \varphi)$.
\end{proposition}

\begin{proof}
In wartime, the king's objective $J_{\mathrm{war}}$ includes terms
where $r$'s actuation has positive utility. In peacetime,
$J_{\mathrm{peace}}$ optimises for long-term survival
($\exists\;\text{path to } \infty$), and the same actuation becomes a
boundary threat on $\Viab(K)$. The resource $r$ is unchanged;
the labelling function $\mathrm{Label}(r, \varphi)$ is what shifts.
\end{proof}

\begin{remark}[Notation unification]\label{rem:notation-unification}
\textcolor{sword}{Three symbols recur across chapters with
ostensibly different definitions.  The collision is deliberate:
each pair is a single object viewed from two grades.}
\begin{enumerate}
  \item $\kappa$\,: \emph{curvature} of the viability manifold
    (\cref{prop:neg-curvature}) and \emph{king node} of the execution
    graph (\cref{def:exgraph}).  The king \emph{is} the curvature
    of the constraint set---the unique vertex whose removal
    maximises the Cheeger ratio $h(K)$.
  \item $\varphi$\,: \emph{conformal parameter}
    $\varphi = -\log V$ (\cref{prop:neg-curvature}), \emph{phase
    operation} $c \mapsto c'$ (\cref{prop:phase}), and
    \emph{agentic flow} $\phi\colon E \to \mathbb{R}_{\geq 0}$
    (\cref{def:flow}).  All three are aspects of the same map
    $S \to S$: the conformal change \emph{is} the phase
    transition, which \emph{is} the flow.  Concretely: the
    conformal change $\varphi = -\log V$ induces the phase
    operation ($V \to 0$ is the phase boundary), which is the
    flow (the trajectory under the Lyapunov decrease).
  \item $\rho$\,: \emph{displacement from a phase boundary}.
    Instantiated as the rank function of a cadre lattice
    (\cref{def:cadre-lattice}) and as the force ratio
    $\rho = F_{\mathrm{rep}}/F_{\mathrm{att}}$
    (\cref{eq:3b-rho}).  In every instance, $\rho = 1$ is
    the switching surface: in the lattice, $\rho$ counts how
    many agents one dominates; in the three-body problem,
    $\rho$ measures how far repulsion exceeds attraction.
\end{enumerate}
\end{remark}

\section{The cut vertex principle}\label{sec:cutvertex}

\begin{definition}[Cut vertex]\label{def:cutvertex}
In the execution graph $G = (V, E)$ of the system, a vertex
$v \in V$ is a \emph{cut vertex} if $G \setminus \{v\}$ is
disconnected. An agent who is a cut vertex controls all execution
chains: removing them disconnects the system.
\end{definition}

\begin{theorem}[Cut vertex $\neq$ maximum actuator]\label{thm:cutvertex}
The optimal survival strategy for the king is to be a cut vertex, not
the maximum actuator. That is, the king maximises viability by
ensuring all execution chains pass through him, rather than by
maximizing his own actuation.
\end{theorem}

\begin{proof}
A maximum actuator $v^*$ with $\|U_{v^*}\| = \max_i \|U_i\|$
suffers from three structural defects:
\begin{enumerate}[label=(\roman*)]
  \item \emph{Non-scalability}: a single actuator cannot cover the
  full state space simultaneously.
  \item \emph{Single point of failure}: $\Viab(K)$ depends entirely
  on $v^*$'s performance; one failure collapses the system.
  \item \emph{Self-referential paradox}: if the king \emph{is} the
  sword (the strongest autonomous actuator), he cannot perform
  viability maintenance on himself.
\end{enumerate}
A cut vertex $v_c$ with $\|U_{v_c}\| \approx 0$ but routing
authority over all chains avoids all three: the system is scalable
(add more actuators), fault-tolerant (one actuator's failure does
not disconnect the graph), and the king is structurally distinct
from the swords he must manage.
\end{proof}

\begin{example}[Liu Bang vs.\ Xiang Yu]\label{ex:liubang}
Xiang Yu was the strongest actuator in the late Qin system
(\emph{Shiji}: ``he could lift a bronze tripod''). His strategy:
$U = \Umax$ through personal combat. Liu Bang had near-zero
personal actuation but made himself the cut vertex of the execution
graph: Han Xin's armies needed Liu Bang's legitimacy, Xiao He's
administration needed his authorization, Zhang Liang's counsel
needed him to listen.

After the phase transition (founding of the Han dynasty), Liu Bang
executed precise viability maintenance: killed Han Xin (sword),
imprisoned then released Xiao He (pre-sword by regression), left Zhang
Liang alone (not a sword). Xiang Yu, the maximum actuator, died at
Gaixia---a single actuator cannot cover the full state space.
\end{example}

\section{Case analysis: the three fates}\label{sec:cases}

The framework's discriminating power is tested against three figures from
the Han founding (c.~202~BCE), all subordinates of the same king (Liu
Bang), operating in the same post-unification phase:

\begin{center}
\begin{tabular}{@{}lcccc@{}}
\toprule
\textbf{Agent} & $\Ur$ & $\mathrm{Im}(\Obs)$ &
\textbf{Classification} & \textbf{Fate} \\
\midrule
Han Xin & $\neq \varnothing$ (military) & Yes & Sword &
Path~(b): eliminated \\
Xiao He & $\neq \varnothing$ (admin) & Yes $\to$ No & Pre-sword &
Sword $\to$ pre-sword regression \\
Zhang Liang & $= \varnothing$ (counsel) & Yes & Not a sword &
Survived \\
\bottomrule
\end{tabular}
\end{center}

All three are visible ($r \in \mathrm{Im}(\Obs)$). The discriminant is
condition~(1): can the resource actuate independently?

\begin{example}[Han Xin: pure sword]\label{ex:hanxin}
Han Xin commanded armies that obeyed \emph{him}, not Liu Bang. His
execution chain was closed: he could mobilise, march, and fight without
the king's authorization. Both conditions of \cref{def:sword} satisfied.
After the phase transition, the sword criterion triggered and Liu Bang
eliminated him. Han Xin's quoted proverb (``when the hare dies, the dog
is cooked'') correctly identified path~(b) but failed to act on it---he
understood the classification but not that the only exit was path~(a).
\end{example}

\begin{example}[Xiao He: sword $\to$ pre-sword regression]
\label{ex:xiaohe}
Xiao He administered the capital and controlled grain supply---autonomous
actuation at the logistical level. The king observed this
($r \in \mathrm{Im}(\Obs)$), making Xiao He a sword by
\cref{def:sword}. Xiao He's response: deliberate self-corruption
(accepting bribes conspicuously). This performed two operations
simultaneously:
\begin{enumerate}[label=(\roman*)]
  \item \emph{Signal reduction}: visible moral degradation signals
  $\|U_r\| \to 0$ (an official this corrupt cannot coordinate a revolt).
  \item \emph{Detection exit}: pull $\|U_r\|$ toward $\bar{U}$,
  falling below the detection threshold (\cref{thm:meanfield}),
  so that $r$ exits $\mathrm{Im}(\Obs)$.
\end{enumerate}
The result is a \emph{sword $\to$ pre-sword regression}
(\cref{def:presword}): Xiao He retains $\Ur \neq \varnothing$
(he still controls the grain supply) but is no longer observed.
His causal envelope $\mathrm{Reach}(x, \Ur)$ persists, but the
king no longer accounts for it in his viability computation
(\cref{prop:viab-gap}).

This is not path~(c) (retaining the sword while proving it safe).
It is a deliberate transition from row~3 to row~2 of the lifecycle
table (\cref{rem:sword-lifecycle}): condition~(2) is suppressed,
not condition~(1).  The binary lifecycle (\cref{thm:lifecycle})
does not apply because Xiao He is no longer classified as a sword.
The pre-sword state is unstable: the causal envelope remains, and
any expansion of $\Obs$ re-detects $r$, returning to the sword row.
Xiao He's survival rests on the king choosing not to expand $\Obs$
into the pre-sword's shadow---a bet, not a resolution.
\end{example}

\begin{example}[Zhang Liang: structural safety]\label{ex:zhangliang}
Zhang Liang was a strategist. Strategy is a pure function: it
\emph{advises} action but cannot \emph{execute} it. Zhang Liang's
counsel required Liu Bang's decision, Liu Bang's generals, and Liu
Bang's administration to produce any effect. Every execution
chain---every path in the execution graph (formalised in
\cref{def:exgraph})---passed through the king
(\cref{cor:breakpoint}). Result:
$\Ur = \varnothing$, condition~(1) fails, not a sword.
Zhang Liang retired and survived.
\end{example}

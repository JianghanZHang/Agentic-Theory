\chapter{The Framework}\label{sec:framework}

\section{The viability axiom}\label{sec:axiom}

Let $S$ denote the state space, $U$ the control set of the principal
agent, and $K \subset S$ the \emph{viability kernel}---the set of states
in which the king retains supreme authority:
\[
  K = \bigl\{\, s \in S \;:\; \text{the king retains supreme authority}
  \,\bigr\}.
\]

\begin{axiom}[Viability]\label{ax:viability}
For every state $s \in K$, there exists a viable path
$\gamma: [0, \infty) \to K$ with $\gamma(0) = s$:
\[
  \forall\, s \in K,\quad
  \exists\;\gamma: s \to \infty
  \quad\text{such that}\quad
  \gamma(t) \in \Viab(K) \;\;\forall\, t \geq 0.
\]
\end{axiom}

When $U = \Umax$, the control set is full and this path is
mathematically guaranteed to exist. The question is: \emph{what can
break it?}

In a multi-agent system with agents $a_1, \ldots, a_n$, each having
control set $U_i$, the answer is unique: an actuator whose output can
push the king's state out of $\Viab(K)$, and which can execute
\emph{independently of the king}. If the actuator's execution must pass
through the king, the king can intercept. If it can bypass the king, the
king's $\Umax$ cannot react in time.

\begin{remark}[Scope]\label{rem:scope}
This paper presents a compressed model of the topology of execution
capability under the viability axiom. It deliberately ignores culture,
personality, economics, and moral narrative in exchange for a testable
structural criterion. Historical cases are used to validate the
criterion's discriminating power, not to claim the criterion exhausts
history.
\end{remark}

\section{The differential inclusion}\label{sec:di}

The viability axiom (\cref{ax:viability}) asserts the existence of a
viable path.  We now make the dynamics precise using the differential
inclusion framework of Aubin and Cellina~\cite{aubin,aubincellina}.

\begin{definition}[Differential inclusion]\label{def:di}
The state $x(t) \in S$ evolves under a \emph{differential inclusion}
\[
  x'(t) \;\in\; F\bigl(x(t)\bigr),
\]
where $F: S \rightrightarrows S$ is a set-valued map that associates
to each state the set of feasible velocities.  A trajectory
$x(\cdot)$ is an absolutely continuous function satisfying the
inclusion for almost all $t \geq 0$.
\end{definition}

The set-valued map $F$ encodes the fact that the system's velocity is
not uniquely determined by its state: multiple agents, each with their
own actuation, contribute competing directions.  In a multi-agent
system with agents $a_1, \ldots, a_n$, the aggregate velocity set is
\[
  F(x) \;=\; \Bigl\{\, \sum_{i} f_i(x, u_i) \;:\;
  u_i \in U_i \,\Bigr\},
\]
where $f_i$ is agent $i$'s dynamics and $U_i$ its control set.

\begin{definition}[Contingent cone]\label{def:contingent}
The \emph{contingent cone} (Bouligand tangent cone) to $K$ at
$x \in K$ is
\[
  T_K(x) \;:=\; \Bigl\{\, v \in S \;:\;
  \liminf_{h \to 0^+} \frac{d_K(x + hv)}{h} = 0 \,\Bigr\},
\]
where $d_K$ denotes the distance to $K$.  Equivalently,
$v \in T_K(x)$ if and only if there exist sequences $h_n \to 0^+$
and $v_n \to v$ such that $x + h_n v_n \in K$ for all $n$.
\end{definition}

The contingent cone $T_K(x)$ is the set of directions at $x$ along
which the state can move while remaining in $K$---the set of
\emph{safe velocities}.  At interior points $T_K(x) = S$; at
boundary points the cone narrows, restricting the feasible
directions.

\begin{theorem}[Viability theorem {\cite{aubin}}]\label{thm:viability-di}
Let $K \subset S$ be locally compact and $F$ be upper semicontinuous
with nonempty compact convex values.  The necessary and sufficient
condition for the existence of a viable trajectory of
$x' \in F(x)$ from every initial state $x_0 \in K$ is the
\emph{tangential condition}:
\[
  \forall\, x \in K, \quad
  F(x) \;\cap\; T_K(x) \;\neq\; \varnothing.
\]
\end{theorem}

This is the differential content of \cref{ax:viability}: the king
survives if and only if, at every state, the system's velocity set
(\cref{def:di}) contains at least one direction tangent to the
viability kernel (\cref{def:contingent}).
The viability axiom is not a wish---it is the tangential condition.

\begin{definition}[Feedback map]\label{def:feedback}
Given the dynamics $x' = f(x, u)$ with control set $U$ (the king's
controls), the \emph{feedback map} is
\[
  C(x) \;:=\; \bigl\{\, u \in U \;:\;
  f(x, u) \in T_K(x) \,\bigr\}.
\]
A viable trajectory exists from $x$ if and only if $C(x) \neq
\varnothing$.  The regulation problem is: does there exist a
feedback law $u(t) \in C(x(t))$ such that $x(\cdot)$ remains in $K$?
\end{definition}

The feedback map is the king's strategy space at state $x$: the set
of controls that keep the trajectory inside $K$.  When $C(x) \neq
\varnothing$ for all $x \in K$, the king can regulate the system.
When a knife $r$ executes independently, its \emph{execution function}
$f_r: S \times \Ur \to TS$ maps a state--action pair to a velocity
vector.  The velocity $f_r(x, a)$ may push $x'(t)$ outside $T_K(x)$,
and the feedback map shrinks---possibly to the empty set.

\begin{definition}[Viability Lyapunov function]\label{def:lyapunov}
A continuous function $V: K \to \R_{\geq 0}$ is a \emph{Lyapunov function}
for the inclusion $x' \in F(x)$ with respect to a cost function
$W: \mathrm{Graph}(F) \to \R_{\geq 0}$ if
\[
  \forall\, x \in K, \quad
  \exists\, v \in F(x) \;\;\text{such that}\;\;
  D^+ V(x)(v) \;+\; W(x, v) \;\leq\; 0,
\]
where $D^+ V(x)(v) := \liminf_{h \to 0^+,\, u \to v}
\frac{V(x + hu) - V(x)}{h}$ is the upper contingent derivative.
Trajectories satisfying this condition are \emph{monotone}: $V(x(t))$
is non-increasing.
\end{definition}

In the language of \cref{sec:water}, the water level $w(t)$ is a
Lyapunov function (\cref{def:lyapunov}) for the viability inclusion.
The monotonicity condition $D^+ V(x)(v) + W(x,v) \leq 0$ says: along
any viable trajectory, the water level cannot increase faster than the
system's extraction cost $W$.  When $V(x(t)) \to 0$, the Lyapunov
condition fails, the tangential condition is violated, and the system
exits $K$---this is \cref{thm:dumu} (Du Mu's theorem) restated in DI
language.

\begin{remark}[Temporal linearity in the DI]\label{rem:di-temporal}
The differential inclusion $x'(t) \in F(x(t))$ operates in real
time: $t$ is the wall-clock parameter, and the inclusion must be
satisfied at every instant.  The feedback map $C(x(t))$ must be
evaluated within one time step $\Delta t$ (\cref{rem:temporal}).
The tangential condition $F(x) \cap T_K(x) \neq \varnothing$
is a \emph{pointwise} requirement: it must hold at every $x \in K$,
which means at every instant.  There is no lookahead, no global
optimisation over future trajectories---only the local tangent
condition, checked in real time.  This is why Aubin calls the
viable system's policy ``opportunism'': the system selects a
feasible velocity from $F(x) \cap T_K(x)$ at each instant, without
planning.  The tangential condition admits regulation maps with
memory; we restrict to memoryless feedback (the king reacts to
current state only) as a modelling choice that matches the historical
evidence.  This restriction is sufficient for the results that
follow; it is not forced by the differential inclusion itself.
\end{remark}

\section{Viability geometry}\label{sec:viab-geom}

The differential inclusion (\cref{sec:di}) provides the dynamics.
We now give the viability kernel $K$ a Riemannian metric, making the
survival problem geometric.  The key observation is that the Lyapunov
function $V$ (\cref{def:lyapunov}) is not merely a scalar indicator of
system health---it is a \emph{conformal factor} that defines the
intrinsic geometry of the viable region.

\begin{definition}[Viability metric]\label{def:viab-metric}
Let $g_S$ denote the ambient metric on the state space $S$ and
$V: K \to \R_{\geq 0}$ the Lyapunov function (\cref{def:lyapunov})
with $V > 0$ on $K^\circ$ and $V = 0$ on $\partial K$.
The \emph{viability metric} on $K^\circ$ is the conformal deformation
\[
  g_V \;:=\; \frac{1}{V(x)^2}\, g_S.
\]
The Riemannian manifold $(K^\circ, g_V)$ is called the
\emph{viability manifold}.
\end{definition}

The conformal factor $V^{-2}$ inflates distances near the boundary
($V \to 0$) and compresses distances in the interior ($V$ large).
A trajectory approaching the boundary must cover infinite
$g_V$-distance in finite ambient time---the boundary is ``at
infinity'' in the viability metric.

\begin{proposition}[Completeness]\label{prop:viab-complete}
Suppose $V(x) \leq C \cdot d_K(x)$ for some $C > 0$
and all $x$ near $\partial K$, where $d_K(x)$ is the ambient
distance to $\partial K$.  Then $(K^\circ, g_V)$ is a complete
Riemannian manifold.
\end{proposition}

\begin{proof}
Let $\gamma: [0, T) \to K^\circ$ be a curve approaching $\partial K$
as $t \to T$.  The $g_V$-length is
\[
  L_{g_V}(\gamma)
  \;=\;
  \int_0^T \frac{\|\gamma'(t)\|_{g_S}}{V(\gamma(t))}\, dt
  \;\geq\;
  \frac{1}{C}
  \int_0^T \frac{\|\gamma'(t)\|_{g_S}}{d_K(\gamma(t))}\, dt.
\]
Since $\|\gamma'(t)\|_{g_S} \geq |d_K(\gamma(t))'|$ by the
triangle inequality, the right-hand side is bounded below by
$(1/C)\int_0^T |d_K'|/d_K\,dt$.  By substitution $u = d_K$,
this becomes $(1/C)\int du/u$, which diverges logarithmically
as $d_K \to 0$.
Hence $\partial K$ is at infinite $g_V$-distance: the Hopf--Rinow
theorem gives completeness.
\end{proof}

\begin{proposition}[Negative curvature]\label{prop:neg-curvature}
Let $\dim S = 2$ and $g_S$ be flat.  If $V$ is superharmonic
($\Delta V \leq 0$, the Lyapunov condition), the Gaussian curvature
of $(K^\circ, g_V)$ satisfies
\[
  \kappa_V
  \;=\;
  V \,\Delta V \;-\; |\nabla V|^2
  \;\leq\;
  -\,|\nabla V|^2
  \;<\; 0
\]
wherever $\nabla V \neq 0$.  In dimension $n \geq 3$, the Ricci
curvature of $g_V = V^{-2} g_S$ satisfies
$\mathrm{Ric}_{g_V} \leq -(n-1)\,|\nabla \log V|^2\, g_V$
under the same superharmonicity condition.
\end{proposition}

\begin{proof}
For a conformal change $\tilde{g} = e^{2\varphi}\, g$ with
$\varphi = -\log V$, the Gaussian curvature in dimension~$2$
transforms as
$\tilde\kappa = e^{-2\varphi}(\kappa_S - \Delta\varphi)$,
where $\kappa_S$ is the ambient curvature.
On a flat background ($\kappa_S = 0$):
\[
  \kappa_V
  \;=\;
  V^2\!\left(-\,\Delta(-\log V)\right)
  \;=\;
  V^2\!\left(\frac{\Delta V}{V} - \frac{|\nabla V|^2}{V^2}\right)
  \;=\;
  V\,\Delta V \;-\; |\nabla V|^2.
\]
Since $\Delta V \leq 0$ (superharmonic) and $|\nabla V|^2 > 0$,
we have $\kappa_V < 0$.  The higher-dimensional statement follows
from the conformal Ricci formula
$\mathrm{Ric}_{\tilde g} = \mathrm{Ric}_g - (n-2)\,
\nabla^2\varphi - [\Delta\varphi + (n-2)\,|\nabla\varphi|^2]\,g$.
\end{proof}

\begin{definition}[Cheeger constant of the viability manifold]
\label{def:cheeger-viab}
The \emph{Cheeger constant} of $(K^\circ, g_V)$ is
\[
  h(K)
  \;:=\;
  \inf_{S}
  \frac{|\partial S|_{g_V}}
  {\min\!\bigl(\mathrm{vol}_{g_V}(A),\,
  \mathrm{vol}_{g_V}(B)\bigr)},
\]
where the infimum is over hypersurfaces $\partial S$ that divide
$K^\circ$ into two open subsets $A$ and $B$, and $|\partial S|_{g_V}$
denotes the $(n-1)$-dimensional volume of $\partial S$ in the
viability metric.
\end{definition}

\begin{theorem}[Cheeger inequality]\label{thm:cheeger-viab}
Let $\lambda_1(K)$ denote the first nonzero eigenvalue of the
Laplace--Beltrami operator on $(K^\circ, g_V)$.  Then
\[
  \lambda_1(K) \;\geq\; \frac{h(K)^2}{4}.
\]
\end{theorem}

\begin{proof}
This is the Riemannian Cheeger inequality~\cite{cheeger}.
The discrete version on the execution graph appears in
\cref{thm:cheeger}.
\end{proof}

\begin{remark}[Poincar\'e half-plane]\label{rem:poincare}
When $K = \{x \in \R^2 : x_2 > 0\}$ (the upper half-plane) and
$V(x) = x_2$ (height above the boundary), the viability metric
$g_V = x_2^{-2}(dx_1^2 + dx_2^2)$ is the Poincar\'e half-plane
model of hyperbolic geometry.  The constant curvature is
$\kappa_V = -1$.  The viability kernel of a dynasty on a flat
state space with water level $V = $ distance to collapse is, in its
intrinsic geometry, \emph{the hyperbolic plane}.
\end{remark}

\begin{remark}[Unique cheapest viable path]\label{rem:cartan-hadamard}
\Cref{prop:neg-curvature} gives $\kappa_V \leq 0$ everywhere.
If $K$ is convex (or more generally, if $K^\circ$ is simply
connected), then by the Cartan--Hadamard theorem $(K^\circ, g_V)$
is a Hadamard manifold: the exponential map
$\exp_x: T_x K^\circ \to K^\circ$ is a diffeomorphism.
In particular, between any two interior states there exists a
\emph{unique geodesic}---a unique cheapest viable path.
There is no ambiguity in the optimal route; the geometry forces it.
Convexity of $K$ is natural: the viability kernel is defined as
the set of states from which a viable path exists
(\cref{ax:viability}), and viability kernels of upper semicontinuous
differential inclusions are closed under convex combinations when
$F$ has convex values (\cref{thm:viability-di}).
\end{remark}

\begin{remark}[Why viability maintenance is hard]
\label{rem:viab-divergence}
Negative curvature means nearby geodesics diverge exponentially:
two trajectories that start $\epsilon$-close separate as
$\sim \epsilon\, e^{\sqrt{|\kappa_V|}\, t}$.  A small perturbation
in the king's initial state produces exponentially different
outcomes.  This is the geometric content of sensitive dependence:
the viability manifold is hyperbolic, so maintaining viability
requires continuous correction at every instant
(\cref{rem:di-temporal}).  The harder the Lyapunov function
decreases ($|\nabla V|$ large), the more negative the curvature,
and the faster trajectories diverge.  Near the boundary
($V \to 0$), the curvature diverges: the last moments before
collapse are the most chaotic.
\end{remark}

\begin{remark}[The water is the metric]\label{rem:water-metric}
The Lyapunov function $V$ (\cref{def:lyapunov})---the water level
of \cref{sec:water}---plays three roles simultaneously:
\begin{enumerate}[label=(\roman*)]
  \item \emph{Scalar}: $V(x)$ measures distance from collapse.
  \item \emph{Conformal factor}: $g_V = V^{-2} g_S$ defines the
  intrinsic geometry of the viability kernel.
  \item \emph{Curvature source}: $\Delta V \leq 0$ forces
  $\kappa_V < 0$, making the geometry hyperbolic.
\end{enumerate}
Du Mu's theorem (\cref{thm:dumu})---$V \to 0$ implies system
death---is a \emph{completeness theorem}: a trajectory reaching
$V = 0$ would traverse infinite $g_V$-distance in finite time,
violating \cref{prop:viab-complete}.  The system must exit $K$
before $V$ reaches zero.  Du Mu is Hopf--Rinow.
\end{remark}

\section{The knife}\label{sec:knife}

\begin{definition}[Knife]\label{def:knife}
A resource $r$ is a \emph{knife} if it satisfies two conditions:
\begin{enumerate}[label=(\arabic*)]
  \item \textbf{Autonomous actuation.} The resource can operate
  independently of the king. Formally, there exists an action
  $a \in \Ur$ such that the execution function $f_r$ satisfies
  \[
    f_r(s, a) \notin K
    \quad\text{and}\quad
    a \text{ does not require the king's authorization.}
  \]
  \item \textbf{Observability.} The king's detection function $\Obs$
  can observe $r$ and its execution capability:
  $r \in \mathrm{Im}(\Obs)$.
\end{enumerate}
\end{definition}

The classification is exhaustive:
\begin{itemize}
  \item Condition~(1) not satisfied: \textbf{not a knife}. (Zhang
  Liang's strategic counsel---a pure function that cannot execute
  itself.)
  \item Condition~(2) not satisfied: \textbf{hidden knife}. (More
  dangerous, but outside the king's strategy space. Unobservable
  $=$ indefensible $=$ system noise.)
  \item Both satisfied: \textbf{knife}.
\end{itemize}

\begin{remark}[DI restatement of the knife]\label{rem:knife-di}
In the language of \cref{sec:di}, a resource $r$ is a knife if and
only if its velocity set $F_r(x)$ can generate directions outside
the contingent cone:
\[
  r \text{ is a knife}
  \quad\iff\quad
  \exists\, x \in K \;\;\text{such that}\;\;
  F_r(x) \not\subset T_K(x).
\]
That is, $r$ can push the system's state toward the boundary of $K$
along directions that are \emph{not tangent} to the viability
kernel.  The king's feedback map $C(x)$ (\cref{def:feedback}) can
compensate only if the king's velocity set $F_{\mathrm{king}}(x)$
contains a counteracting direction in $T_K(x)$.  When $r$ executes
independently---bypassing $C(x)$---no compensation is possible,
and the tangential condition (\cref{thm:viability-di}) is violated.
\end{remark}

\begin{remark}[Geometric restatement of the knife]
\label{rem:knife-geom}
In the viability manifold $(K^\circ, g_V)$ (\cref{def:viab-metric}),
the knife has a curvature interpretation.  A knife
(\cref{def:knife}) must satisfy both autonomous actuation
\emph{and} observability; the geometric content lies in
condition~(1).  When a resource $r$ actuates autonomously, its
velocity field $f_r$ has a component along $-\nabla V$ (pointing
toward the boundary).  The magnitude
$|\langle f_r, -\nabla V \rangle|$ contributes directly to the
curvature (\cref{prop:neg-curvature}): the knife increases
$|\nabla V|$, making $\kappa_V$ more negative, and the
viable region more hyperbolic.  More knives $\Rightarrow$ more
negative curvature $\Rightarrow$ faster divergence of nearby
trajectories $\Rightarrow$ harder viability maintenance.
The Cheeger constant $h(K)$ (\cref{def:cheeger-viab}) measures
the worst-case cut: the knife is the hypersurface that minimises the
isoperimetric ratio of the viability manifold.
(Condition~(2)---observability---determines whether the king
\emph{knows} where the cut is, not whether it exists.)
\end{remark}

\begin{remark}[Intent is irrelevant]\label{rem:intent}
The criterion tests \emph{capability}, not \emph{intention}. The king
detects whether you \emph{can} act, not whether you \emph{want to}.
Loyalty does not enter the criterion.
\end{remark}

\begin{remark}[Logical necessity]\label{rem:necessity}
These two conditions are not chosen by the modeler. They are the unique
logical consequence of the viability axiom $+$ unconstrained power $+$
multi-agent environment.
\end{remark}

\section{Phase transition}\label{sec:phase}

The knife is a \emph{phase function}, not an intrinsic property.

\begin{proposition}[Phase-dependent labelling]\label{prop:phase}
The same resource $r$ receives different labels under different system
phases $\varphi$:
\[
  \mathrm{Label}(r, \varphi) =
  \begin{cases}
    \textbf{tool} & \text{if } \varphi = \text{wartime (king needs }
    r\text{'s actuation),} \\
    \textbf{knife} & \text{if } \varphi = \text{peacetime (king no
    longer needs } r\text{, but } r \text{ persists).}
  \end{cases}
\]
The phase transition does not change the physical properties of $r$.
It changes the king's objective function $J(s, \varphi)$.
\end{proposition}

\begin{proof}
In wartime, the king's objective $J_{\mathrm{war}}$ includes terms
where $r$'s actuation has positive utility. In peacetime,
$J_{\mathrm{peace}}$ optimises for long-term survival
($\exists\;\text{path to } \infty$), and the same actuation becomes a
boundary threat on $\Viab(K)$. The resource $r$ is unchanged;
the labelling function $\mathrm{Label}(r, \varphi)$ is what shifts.
\end{proof}

\section{The cut vertex principle}\label{sec:cutvertex}

\begin{definition}[Cut vertex]\label{def:cutvertex}
In the execution graph $G = (V, E)$ of the system, a vertex
$v \in V$ is a \emph{cut vertex} if $G \setminus \{v\}$ is
disconnected. An agent who is a cut vertex controls all execution
chains: removing them disconnects the system.
\end{definition}

\begin{theorem}[Cut vertex $\neq$ maximum actuator]\label{thm:cutvertex}
The optimal survival strategy for the king is to be a cut vertex, not
the maximum actuator. That is, the king maximises viability by
ensuring all execution chains pass through him, rather than by
maximizing his own actuation.
\end{theorem}

\begin{proof}
A maximum actuator $v^*$ with $\|U_{v^*}\| = \max_i \|U_i\|$
suffers from three structural defects:
\begin{enumerate}[label=(\roman*)]
  \item \emph{Non-scalability}: a single actuator cannot cover the
  full state space simultaneously.
  \item \emph{Single point of failure}: $\Viab(K)$ depends entirely
  on $v^*$'s performance; one failure collapses the system.
  \item \emph{Self-referential paradox}: if the king \emph{is} the
  knife (the strongest autonomous actuator), he cannot perform
  viability maintenance on himself.
\end{enumerate}
A cut vertex $v_c$ with $\|U_{v_c}\| \approx 0$ but routing
authority over all chains avoids all three: the system is scalable
(add more actuators), fault-tolerant (one actuator's failure does
not disconnect the graph), and the king is structurally distinct
from the knives he must manage.
\end{proof}

\begin{example}[Liu Bang vs.\ Xiang Yu]\label{ex:liubang}
Xiang Yu was the strongest actuator in the late Qin system
(\emph{Shiji}: ``he could lift a bronze tripod''). His strategy:
$U = \Umax$ through personal combat. Liu Bang had near-zero
personal actuation but made himself the cut vertex of the execution
graph: Han Xin's armies needed Liu Bang's legitimacy, Xiao He's
administration needed his authorization, Zhang Liang's counsel
needed him to listen.

After the phase transition (founding of the Han dynasty), Liu Bang
executed precise viability maintenance: killed Han Xin (knife),
imprisoned then released Xiao He (blunted half-knife), left Zhang
Liang alone (not a knife). Xiang Yu, the maximum actuator, died at
Gaixia---a single actuator cannot cover the full state space.
\end{example}

\section{Case analysis: the three fates}\label{sec:cases}

The framework's discriminating power is tested against three figures from
the Han founding (c.~202~BCE), all subordinates of the same king (Liu
Bang), operating in the same post-unification phase:

\begin{center}
\begin{tabular}{@{}lcccc@{}}
\toprule
\textbf{Agent} & $\Ur$ & $\mathrm{Im}(\Obs)$ &
\textbf{Classification} & \textbf{Fate} \\
\midrule
Han Xin & $\neq \varnothing$ (military) & Yes & Knife &
Path~(b): eliminated \\
Xiao He & $\neq \varnothing$ (admin) & Yes & Half-knife &
Path~(a): self-blunted \\
Zhang Liang & $= \varnothing$ (counsel) & Yes & Not a knife &
Survived \\
\bottomrule
\end{tabular}
\end{center}

All three are visible ($r \in \mathrm{Im}(\Obs)$). The discriminant is
condition~(1): can the resource actuate independently?

\begin{example}[Han Xin: pure knife]\label{ex:hanxin}
Han Xin commanded armies that obeyed \emph{him}, not Liu Bang. His
execution chain was closed: he could mobilise, march, and fight without
the king's authorization. Both conditions of \cref{def:knife} satisfied.
After the phase transition, the knife criterion triggered and Liu Bang
eliminated him. Han Xin's quoted proverb (``when the hare dies, the dog
is cooked'') correctly identified path~(b) but failed to act on it---he
understood the classification but not that the only exit was path~(a).
\end{example}

\begin{example}[Xiao He: self-blunting]\label{ex:xiaohe}
Xiao He administered the capital and controlled grain supply---autonomous
actuation at the logistical level. The king observed this
($r \in \mathrm{Im}(\Obs)$), making Xiao He a knife by
\cref{def:knife}. Xiao He's response: deliberate self-corruption
(accepting bribes conspicuously). This performed two operations
simultaneously:
\begin{enumerate}[label=(\roman*)]
  \item \emph{Signal reduction}: visible moral degradation signals
  $\|U_r\| \to 0$ (an official this corrupt cannot coordinate a revolt).
  \item \emph{Mean-field alignment}: pull $\|U_r\|$ toward $\bar{U}$,
  falling below the detection threshold (\cref{thm:meanfield}).
\end{enumerate}
This is path~(a) executed through reputation rather than resignation.
\end{example}

\begin{example}[Zhang Liang: structural safety]\label{ex:zhangliang}
Zhang Liang was a strategist. Strategy is a pure function: it
\emph{advises} action but cannot \emph{execute} it. Zhang Liang's
counsel required Liu Bang's decision, Liu Bang's generals, and Liu
Bang's administration to produce any effect. Every execution chain
passed through the king (\cref{cor:breakpoint}). Result:
$\Ur = \varnothing$, condition~(1) fails, not a knife.
Zhang Liang retired and survived.
\end{example}

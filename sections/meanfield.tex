\chapter{The Knife as Mean Field}\label{sec:meanfield}

The preceding sections defined the knife as a two-condition criterion
applied to individual resources. We now argue that the knife is
fundamentally a \emph{mean-field} phenomenon.

\section{The mean actuation field}

Consider $n$ agents with autonomous actuation levels
$\|U_1\|, \ldots, \|U_n\|$. Define the \emph{mean actuation field}:
\[
  \bar{U} = \frac{1}{n} \sum_{i=1}^{n} \|U_i\|.
\]

The king's detection function $\Obs$ has finite bandwidth: it cannot
monitor all agents with equal precision. In practice, $\Obs$ triggers
on agents whose actuation \emph{deviates significantly from the mean}:
\[
  r \in \mathrm{Im}(\Obs)
  \iff
  \|U_r\| - \bar{U} > \tau(\Obs),
\]
where $\tau(\Obs)$ is the detection threshold determined by the king's
observational capacity.

\begin{theorem}[The knife is the mean]\label{thm:meanfield}
The knife threshold is determined by the system's mean actuation field.
A resource $r$ is a knife if and only if:
\begin{enumerate}[label=(\roman*)]
  \item $\Ur \neq \varnothing$ (autonomous actuation exists), and
  \item $\|U_r\|$ exceeds the mean field by more than the detection
  threshold: $\|U_r\| > \bar{U} + \tau(\Obs)$.
\end{enumerate}
Consequently, the phase transition (\cref{prop:phase}) is a shift in
$\bar{U}$, not a change in any individual $\Ur$.
\end{theorem}

\begin{proof}
In wartime, many agents have high actuation (soldiers, generals,
administrators). The mean $\bar{U}$ is high, so the threshold
$\bar{U} + \tau(\Obs)$ is high: few agents exceed it. Most actuation
is \emph{expected} and therefore not flagged.

At the phase transition (end of war), most agents' actuation drops to
near zero (soldiers demobilize, wartime powers expire). The mean
$\bar{U}$ drops sharply. But agents who \emph{retain} wartime-level
actuation now exceed the new, lower threshold. The same $\|U_r\|$
that was below the wartime mean is now above the peacetime mean.

The knife is not created by the agent---it is created by the shift in
the mean. The agent's actuation is unchanged; the system's reference
frame has moved.
\end{proof}

\begin{remark}[Connection to statistical mechanics]\label{rem:statmech}
This is precisely the mechanism of a phase transition in statistical
mechanics: the order parameter (mean actuation) shifts, and
configurations that were typical in one phase become atypical---and
therefore detectable---in the other. The viability axiom plays the role
of the free energy: the system minimizes threats to $\Viab(K)$, just
as a thermodynamic system minimizes free energy.
\end{remark}

\section{Implications}

The mean-field interpretation resolves several puzzles:

\begin{enumerate}
  \item \textbf{Why identical resources have different fates.} Two
  generals with identical $\Ur$ can have opposite outcomes if one
  operates in a high-$\bar{U}$ environment (wartime coalition) and
  the other in a low-$\bar{U}$ environment (consolidated empire).
  The knife is relative to the mean.

  \item \textbf{Why the paradox is a feedback loop.} As the king
  eliminates knives, $\bar{U}$ drops, lowering the threshold. Agents
  who were below the old threshold now exceed the new one $\to$ new
  knives $\to$ more elimination $\to$ lower $\bar{U}$ $\to$ \ldots
  This is the positive feedback of \cref{thm:paradox}, now given a
  statistical mechanism.

  \item \textbf{Why self-blunting works.} Xiao He's strategy
  (self-corruption to signal low $\|U_r\|$) works precisely because
  the detection function triggers on \emph{deviation from the mean}.
  By visibly degrading his own actuation, Xiao He pulled $\|U_r\|$
  toward $\bar{U}$, falling below the detection threshold.

  \item \textbf{Why breakpoints prevent knives.} A breakpoint in the
  execution chain reduces $\|U_r\|$ (effective autonomous actuation)
  to below $\bar{U} + \tau(\Obs)$, since the king controls part of
  the chain. The resource remains capable but not \emph{independently}
  capable---it does not deviate from the mean.
\end{enumerate}

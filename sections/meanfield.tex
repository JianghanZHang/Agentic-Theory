\chapter{The Sword as Mean Field}\label{sec:meanfield}

The preceding sections defined the sword as a two-condition criterion
applied to individual resources. We now argue that the sword is
fundamentally a \emph{mean-field} phenomenon.

\section{The mean actuation field}

Consider $n$ agents with autonomous actuation levels
$\|U_1\|, \ldots, \|U_n\|$, where $\|U_i\|$ measures the
\emph{degrees of freedom} of agent~$i$'s autonomous control set
(the dimension of $\mathrm{span}(U_i)$ when $U_i$ is a linear
subspace; $|U_i|$ when discrete).
Define the \emph{mean actuation field}:
\[
  \bar{U} = \frac{1}{n} \sum_{i=1}^{n} \|U_i\|.
\]

The king's detection function $\Obs$ has finite bandwidth: it cannot
monitor all agents with equal precision.

\begin{proposition}[Detection as optimal monitoring]
\label{prop:detection-derivation}
Let the king have finite monitoring bandwidth~$B$: he can
observe at most~$B$ agents simultaneously.  Under the viability
axiom (\cref{ax:viability}), the optimal monitoring policy
$\Obs^*$ is a threshold rule:
\[
  r \in \mathrm{Im}(\Obs^*)
  \iff
  \|U_r\| - \bar{U} > \tau(B),
\]
where $\tau(B)$ is the threshold that selects the top-$B$
agents by deviation from the mean.
\end{proposition}

\begin{proof}[Proof sketch]
\leavevmode
\begin{enumerate}[label=(\roman*),nosep]
  \item The king must maintain $F(x) \cap T_K(x) \neq
  \varnothing$ (tangential condition,
  \cref{thm:viability-di}).
  \item The threat from agent~$r$ is proportional to
  $\|\Ur\|$ (the dimension of its autonomous control set).
  \item The mean field~$\bar{U}$ is the unique unbiased
  baseline: by the massless axiom, deviations must net to
  zero, forcing $b = \bar{U}$ (\cref{lem:capacity}).
  Agents at or below the mean carry zero bypass capacity
  and pose no threat.
  \item With finite bandwidth~$B$, optimal allocation monitors
  the agents with largest $\|\Ur\| - \bar{U}$, i.e., the
  most deviant.
  \item The threshold $\tau(B)$ is determined by~$B$: it is
  the $(n - B)$-th order statistic of
  $\{\|U_i\| - \bar{U}\}$.
\end{enumerate}
This is the evaluation operator $T^{\Obs}$: given the current
mean field~$\bar{U}$, it evaluates which agents are monitored.
The detection equation~\eqref{eq:detection} is the fixed point
of this operator, self-consistent when both~$\bar{U}$ and the
detected set are stable.
\end{proof}

By \cref{prop:detection-derivation}, $\Obs$ triggers on agents
whose actuation deviates significantly from the mean:
\begin{equation}\label{eq:detection}
  r \in \mathrm{Im}(\Obs)
  \iff
  \|U_r\| - \bar{U} > \tau(\Obs),
\end{equation}
where $\tau(\Obs)$ is the detection threshold determined by the king's
observational capacity.  This reduces the set-valued observability
concept ($r \in \mathrm{Im}(\Obs)$ or not) to a scalar comparison,
valid when the king's bandwidth is finite and detection is
threshold-based.

\begin{theorem}[The sword is the mean]\label{thm:meanfield}
The sword threshold is determined by the system's mean actuation field.
A resource $r$ is a sword if and only if:
\begin{enumerate}[label=(\roman*)]
  \item $\Ur \neq \varnothing$ (autonomous actuation exists), and
  \item $\|U_r\|$ exceeds the mean field by more than the detection
  threshold: $\|U_r\| > \bar{U} + \tau(\Obs)$.
\end{enumerate}
Consequently, the phase transition (\cref{prop:phase}) is a shift in
$\bar{U}$, not a change in any individual $\Ur$.
\end{theorem}

\begin{proof}
Condition~(i) is unchanged from \cref{def:sword}.
For condition~(ii): by the detection model~\eqref{eq:detection},
$r \in \mathrm{Im}(\Obs)$ if and only if
$\|U_r\| - \bar{U} > \tau(\Obs)$.  Substituting into the sword
criterion (\cref{def:sword}) yields the stated biconditional.

For the phase-transition claim: let $\bar{U}_1$ denote the mean
before the transition and $\bar{U}_2 < \bar{U}_1$ the mean after.
An agent with fixed $\|U_r\|$ satisfying
$\bar{U}_2 + \tau < \|U_r\| \leq \bar{U}_1 + \tau$
is not a sword in regime~1 but becomes a sword in regime~2.
The shift is in $\bar{U}$, not in any individual $\|U_r\|$.
\end{proof}

\begin{remark}[Historical illustration]\label{rem:meanfield-history}
The formal mechanism is visible at every phase transition.
In wartime, many agents carry high actuation; the mean $\bar{U}$ is
high; few exceed the threshold. At demobilization, the mean drops
sharply. Agents who retain wartime-level actuation now exceed the
peacetime threshold---the same $\|U_r\|$ that was unremarkable before
is now a deviation. The sword is not created by the agent but by the
shift in the system's reference frame.
\end{remark}

\begin{remark}[Connection to statistical mechanics]\label{rem:statmech}
This is precisely the mechanism of a phase transition in statistical
mechanics: the order parameter (mean actuation) shifts, and
configurations that were typical in one phase become atypical---and
therefore detectable---in the other. The viability axiom plays the role
of the free energy: the system minimises threats to $\Viab(K)$, just
as a thermodynamic system minimises free energy.
\end{remark}

\begin{remark}[Validity conditions]\label{rem:meanfield-validity}
The arithmetic mean~$\bar{U}$ is the operative threshold for
homogeneous systems where agents draw from comparable control
sets.  In hierarchical systems, clique amplification
(see \cref{app:zhongxian}) may require a modified threshold
that accounts for correlated actuation within subgroups.
\end{remark}

\section{Implications}

The mean-field interpretation resolves several puzzles:

\begin{enumerate}
  \item \textbf{Why identical resources have different fates.} Two
  generals with identical $\Ur$ can have opposite outcomes if one
  operates in a high-$\bar{U}$ environment (wartime coalition) and
  the other in a low-$\bar{U}$ environment (consolidated empire).
  The sword is relative to the mean.

  \item \textbf{Why the paradox is a feedback loop.} As the king
  eliminates swords, $\bar{U}$ drops, lowering the threshold. Agents
  who were below the old threshold now exceed the new one $\to$ new
  swords $\to$ more elimination $\to$ lower $\bar{U}$ $\to$ \ldots
  This is the positive feedback of \cref{thm:paradox}, now given a
  statistical mechanism.

  \item \textbf{Why self-blunting works.} Xiao He's strategy
  (self-corruption to signal low $\|U_r\|$) works precisely because
  the detection function triggers on \emph{deviation from the mean}.
  By visibly degrading his own actuation, Xiao He pulled $\|U_r\|$
  toward $\bar{U}$, falling below the detection threshold.

  \item \textbf{Why breakpoints prevent swords.} A breakpoint in the
  execution chain reduces $\|U_r\|$ (effective autonomous actuation)
  to below $\bar{U} + \tau(\Obs)$, since the king controls part of
  the chain. The resource remains capable but not \emph{independently}
  capable---it does not deviate from the mean.
\end{enumerate}

\documentclass[11pt]{article}

% ── Packages ──────────────────────────────────────────────
\usepackage[margin=1in]{geometry}
\usepackage{amsmath,amssymb,amsthm}
\usepackage{mathtools}
\usepackage{enumitem}
\usepackage{booktabs}
\usepackage{hyperref}
\usepackage[capitalise,noabbrev]{cleveref}

% ── Theorem environments ──────────────────────────────────
\theoremstyle{plain}
\newtheorem{theorem}{Theorem}[section]
\newtheorem{proposition}[theorem]{Proposition}
\newtheorem{lemma}[theorem]{Lemma}
\newtheorem{corollary}[theorem]{Corollary}

\theoremstyle{definition}
\newtheorem{definition}[theorem]{Definition}
\newtheorem{axiom}[theorem]{Axiom}
\newtheorem{example}[theorem]{Example}

\theoremstyle{remark}
\newtheorem{remark}[theorem]{Remark}

% ── Macros ────────────────────────────────────────────────
\newcommand{\Viab}{\mathrm{Viab}}
\newcommand{\Umax}{U_{\max}}
\newcommand{\Ur}{U_r}
\newcommand{\Obs}{\mathcal{O}}
\newcommand{\R}{\mathbb{R}}

\title{The Knife Is the Mean:\\
An Agentic Theory of Viability Maintenance}
\author{Jianghan Zhang\\
\texttt{Jianghan.Zhang.gr@dartmouth.edu}\\
Dartmouth College}
\date{February 2026}

\begin{document}
\maketitle

\begin{abstract}
We develop an agentic theory of viability maintenance in systems with a
principal agent (the \emph{king}) whose survival is governed by a single
axiom: the existence of a viable path to infinity. Under this axiom, we
derive a structural criterion---the \emph{knife}---that classifies any
resource as a threat based on two conditions: autonomous actuation and
observability. We prove that the knife is not an intrinsic property of a
resource but a \emph{phase function} determined by the system's mean
field. We establish three main results: (1)~a binary lifecycle theorem
(every knife is either relinquished or eliminated), (2)~a fixed-point
impossibility (no third path exists), and (3)~an unconstrained power
paradox (maximizing control forces maximizing elimination). The
framework is validated against Chinese imperial history (475~BCE--1644~CE)
and extended to the Atlantic triangular trade, the structure of
ideological hatred, parasitic network topologies, and militarism.
The central thesis is that the knife is the mean: the critical threshold
separating tools from threats is determined by the system's average
autonomous actuation level, making viability maintenance a mean-field
phenomenon.
\end{abstract}

% ══════════════════════════════════════════════════════════
\section{Introduction}\label{sec:intro}
% ══════════════════════════════════════════════════════════

Consider a system with $n$ agents $a_1, \ldots, a_n$, each possessing a
control set $U_i$ that determines what actions they can take
independently. One agent---the \emph{principal agent} or
\emph{king}---holds maximal authority: $U = \Umax$. The king's sole
objective is survival, formalized as the existence of a viable path from
every reachable state to infinity.

This paper asks: \emph{what structural features of the system force the
king to eliminate other agents?} The answer turns out to be a two-condition
criterion we call the \emph{knife}. A resource is a knife if and only if
it can actuate independently of the king \emph{and} the king can observe
it. This criterion is not chosen---it is the unique logical consequence
of the viability axiom in a multi-agent environment.

The framework yields several results that are surprising in their
precision. The lifecycle of every knife is binary: it is either
voluntarily relinquished (path~(a)) or forcibly eliminated (path~(b)).
There is no third option---attempting to ``prove your knife is not a
knife'' is a fixed-point impossibility. Unconstrained power is not
freedom but a perpetual elimination machine: the more control the king
has, the more knives he must cut.

The central thesis of this paper is that the knife is the
\emph{mean field} of the system. The threshold separating tools from
threats is not absolute but statistical: it is determined by the average
autonomous actuation level across all agents. When the system's mean
field shifts (e.g., from wartime to peacetime), the same resource
changes classification without any change in its physical properties.
This makes viability maintenance a mean-field phenomenon: the king does
not respond to individual threats but to deviations from the system mean.

We validate the framework against two millennia of Chinese imperial
history, where the viability axiom operated with unusual clarity due to
the concentration of sovereignty in a single agent. The framework
correctly classifies the fates of historical figures---Han~Xin (knife,
eliminated), Xiao~He (half-knife, blunted), Zhang~Liang (not a knife,
survived)---and explains structural phenomena such as the Qin unification
as a graph-theoretic optimization. Extensions to the Atlantic slave trade,
the structure of ideological hatred, parasitic network topologies, and
militarism demonstrate that the framework applies beyond its original
historical domain.

\paragraph{Organization.}
\Cref{sec:framework} introduces the viability axiom, the knife
definition, and the phase transition.
\Cref{sec:results} presents the main theorems.
\Cref{sec:meanfield} develops the mean-field interpretation.
\Cref{sec:applications} applies the framework to historical and
contemporary systems.
\Cref{sec:discussion} discusses the domain of applicability and the
dynamics of the population (the ``water'').

% ══════════════════════════════════════════════════════════
\section{The Framework}\label{sec:framework}
% ══════════════════════════════════════════════════════════

\subsection{The viability axiom}\label{sec:axiom}

Let $S$ denote the state space, $U$ the control set of the principal
agent, and $K \subset S$ the \emph{viability kernel}---the set of states
in which the king retains supreme authority:
\[
  K = \bigl\{\, s \in S \;:\; \text{the king retains supreme authority}
  \,\bigr\}.
\]

\begin{axiom}[Viability]\label{ax:viability}
For every state $s \in S$, there exists a viable path
$\gamma: [0, \infty) \to K$ with $\gamma(0) = s$:
\[
  \forall\, s \in S,\quad
  \exists\;\gamma: s \to \infty
  \quad\text{such that}\quad
  \gamma(t) \in \Viab(K) \;\;\forall\, t \geq 0.
\]
\end{axiom}

When $U = \Umax$, the control set is full and this path is
mathematically guaranteed to exist. The question is: \emph{what can
break it?}

In a multi-agent system with agents $a_1, \ldots, a_n$, each having
control set $U_i$, the answer is unique: an actuator whose output can
push the king's state out of $\Viab(K)$, and which can execute
\emph{independently of the king}. If the actuator's execution must pass
through the king, the king can intercept. If it can bypass the king, the
king's $\Umax$ cannot react in time.

\begin{remark}[Scope]\label{rem:scope}
This paper presents a compressed model of the topology of execution
capability under the viability axiom. It deliberately ignores culture,
personality, economics, and moral narrative in exchange for a testable
structural criterion. Historical cases are used to validate the
criterion's discriminating power, not to claim the criterion exhausts
history.
\end{remark}

\subsection{The knife}\label{sec:knife}

\begin{definition}[Knife]\label{def:knife}
A resource $r$ is a \emph{knife} if it satisfies two conditions:
\begin{enumerate}[label=(\arabic*)]
  \item \textbf{Autonomous actuation.} The resource can operate
  independently of the king. Formally, there exists an action
  $a \in \Ur$ such that the execution function $f_r$ satisfies
  \[
    f_r(s, a) \notin K
    \quad\text{and}\quad
    a \text{ does not require the king's authorization.}
  \]
  \item \textbf{Observability.} The king's detection function $\Obs$
  can observe $r$ and its execution capability:
  $r \in \mathrm{Im}(\Obs)$.
\end{enumerate}
\end{definition}

The classification is exhaustive:
\begin{itemize}
  \item Condition~(1) not satisfied: \textbf{not a knife}. (Zhang
  Liang's strategic counsel---a pure function that cannot execute
  itself.)
  \item Condition~(2) not satisfied: \textbf{hidden knife}. (More
  dangerous, but outside the king's strategy space. Unobservable
  $=$ indefensible $=$ system noise.)
  \item Both satisfied: \textbf{knife}.
\end{itemize}

\begin{remark}[Intent is irrelevant]\label{rem:intent}
The criterion tests \emph{capability}, not \emph{intention}. The king
detects whether you \emph{can} act, not whether you \emph{want to}.
Loyalty does not enter the criterion.
\end{remark}

\begin{remark}[Logical necessity]\label{rem:necessity}
These two conditions are not chosen by the modeler. They are the unique
logical consequence of the viability axiom $+$ unconstrained power $+$
multi-agent environment.
\end{remark}

\subsection{Phase transition}\label{sec:phase}

The knife is a \emph{phase function}, not an intrinsic property.

\begin{proposition}[Phase-dependent labeling]\label{prop:phase}
The same resource $r$ receives different labels under different system
phases $\varphi$:
\[
  \mathrm{Label}(r, \varphi) =
  \begin{cases}
    \textbf{tool} & \text{if } \varphi = \text{wartime (king needs }
    r\text{'s actuation),} \\
    \textbf{knife} & \text{if } \varphi = \text{peacetime (king no
    longer needs } r\text{, but } r \text{ persists).}
  \end{cases}
\]
The phase transition does not change the physical properties of $r$.
It changes the king's objective function $J(s, \varphi)$.
\end{proposition}

\begin{proof}
In wartime, the king's objective $J_{\mathrm{war}}$ includes terms
where $r$'s actuation has positive utility. In peacetime,
$J_{\mathrm{peace}}$ optimizes for long-term survival
($\exists\;\text{path to } \infty$), and the same actuation becomes a
boundary threat on $\Viab(K)$. The resource $r$ is unchanged;
the labeling function $\mathrm{Label}(r, \varphi)$ is what shifts.
\end{proof}

\subsection{The cut vertex principle}\label{sec:cutvertex}

\begin{definition}[Cut vertex]\label{def:cutvertex}
In the execution graph $G = (V, E)$ of the system, a vertex
$v \in V$ is a \emph{cut vertex} if $G \setminus \{v\}$ is
disconnected. An agent who is a cut vertex controls all execution
chains: removing them disconnects the system.
\end{definition}

\begin{theorem}[Cut vertex $\neq$ maximum actuator]\label{thm:cutvertex}
The optimal survival strategy for the king is to be a cut vertex, not
the maximum actuator. That is, the king maximizes viability by
ensuring all execution chains pass through him, rather than by
maximizing his own actuation.
\end{theorem}

\begin{proof}
A maximum actuator $v^*$ with $\|U_{v^*}\| = \max_i \|U_i\|$
suffers from three structural defects:
\begin{enumerate}[label=(\roman*)]
  \item \emph{Non-scalability}: a single actuator cannot cover the
  full state space simultaneously.
  \item \emph{Single point of failure}: $\Viab(K)$ depends entirely
  on $v^*$'s performance; one failure collapses the system.
  \item \emph{Self-referential paradox}: if the king \emph{is} the
  knife (the strongest autonomous actuator), he cannot perform
  viability maintenance on himself.
\end{enumerate}
A cut vertex $v_c$ with $\|U_{v_c}\| \approx 0$ but routing
authority over all chains avoids all three: the system is scalable
(add more actuators), fault-tolerant (one actuator's failure does
not disconnect the graph), and the king is structurally distinct
from the knives he must manage.
\end{proof}

\begin{example}[Liu Bang vs.\ Xiang Yu]\label{ex:liubang}
Xiang Yu was the strongest actuator in the late Qin system
(\emph{Shiji}: ``he could lift a bronze tripod''). His strategy:
$U = \Umax$ through personal combat. Liu Bang had near-zero
personal actuation but made himself the cut vertex of the execution
graph: Han Xin's armies needed Liu Bang's legitimacy, Xiao He's
administration needed his authorization, Zhang Liang's counsel
needed him to listen.

After the phase transition (founding of the Han dynasty), Liu Bang
executed precise viability maintenance: killed Han Xin (knife),
imprisoned then released Xiao He (blunted half-knife), left Zhang
Liang alone (not a knife). Xiang Yu, the maximum actuator, died at
Gaixia---a single actuator cannot cover the full state space.
\end{example}

\subsection{Case analysis: the three fates}\label{sec:cases}

The framework's discriminating power is tested against three figures from
the Han founding (c.~202~BCE), all subordinates of the same king (Liu
Bang), operating in the same post-unification phase:

\begin{center}
\begin{tabular}{@{}lcccc@{}}
\toprule
\textbf{Agent} & $\Ur$ & $\mathrm{Im}(\Obs)$ &
\textbf{Classification} & \textbf{Fate} \\
\midrule
Han Xin & $\neq \varnothing$ (military) & Yes & Knife &
Path~(b): eliminated \\
Xiao He & $\neq \varnothing$ (admin) & Yes & Half-knife &
Path~(a): self-blunted \\
Zhang Liang & $= \varnothing$ (counsel) & Yes & Not a knife &
Survived \\
\bottomrule
\end{tabular}
\end{center}

All three are visible ($r \in \mathrm{Im}(\Obs)$). The discriminant is
condition~(1): can the resource actuate independently?

\begin{example}[Han Xin: pure knife]\label{ex:hanxin}
Han Xin commanded armies that obeyed \emph{him}, not Liu Bang. His
execution chain was closed: he could mobilize, march, and fight without
the king's authorization. Both conditions of \cref{def:knife} satisfied.
After the phase transition, the knife criterion triggered and Liu Bang
eliminated him. Han Xin's quoted proverb (``when the hare dies, the dog
is cooked'') correctly identified path~(b) but failed to act on it---he
understood the classification but not that the only exit was path~(a).
\end{example}

\begin{example}[Xiao He: self-blunting]\label{ex:xiaohe}
Xiao He administered the capital and controlled grain supply---autonomous
actuation at the logistical level. The king observed this
($r \in \mathrm{Im}(\Obs)$), making Xiao He a knife by
\cref{def:knife}. Xiao He's response: deliberate self-corruption
(accepting bribes conspicuously). This performed two operations
simultaneously:
\begin{enumerate}[label=(\roman*)]
  \item \emph{Signal reduction}: visible moral degradation signals
  $\|U_r\| \to 0$ (an official this corrupt cannot coordinate a revolt).
  \item \emph{Mean-field alignment}: pull $\|U_r\|$ toward $\bar{U}$,
  falling below the detection threshold (\cref{thm:meanfield}).
\end{enumerate}
This is path~(a) executed through reputation rather than resignation.
\end{example}

\begin{example}[Zhang Liang: structural safety]\label{ex:zhangliang}
Zhang Liang was a strategist. Strategy is a pure function: it
\emph{advises} action but cannot \emph{execute} it. Zhang Liang's
counsel required Liu Bang's decision, Liu Bang's generals, and Liu
Bang's administration to produce any effect. Every execution chain
passed through the king (\cref{cor:breakpoint}). Result:
$\Ur = \varnothing$, condition~(1) fails, not a knife.
Zhang Liang retired and survived.
\end{example}

% ══════════════════════════════════════════════════════════
\section{Main Results}\label{sec:results}
% ══════════════════════════════════════════════════════════

\subsection{The binary lifecycle}\label{sec:lifecycle}

\begin{theorem}[Binary fate]\label{thm:lifecycle}
Every knife has exactly two possible outcomes after phase transition:
\begin{enumerate}[label=(\alph*)]
  \item \textbf{Relinquish}: the holder voluntarily sets
  $\Ur \to \varnothing$.
  \item \textbf{Elimination}: the king forces removal via
  $u^* \in \Umax$.
\end{enumerate}
There is no path~(c).
\end{theorem}

\begin{proof}
The knife criterion is $\Ur \neq \varnothing \;\wedge\;
r \in \mathrm{Im}(\Obs)$. To exit this classification, at least
one condition must fail. Condition~(2) is controlled by the king
(he can always look); only condition~(1) can be changed by the
holder. Setting $\Ur \to \varnothing$ is path~(a). If the holder
does not, the king must act (viability axiom), which is path~(b).
The classification is exhaustive.
\end{proof}

\begin{remark}[Han Xin's error]\label{rem:hanxin}
The proverb ``when the cunning hare is killed, the hunting dog is
cooked'' (\emph{Shiji}, Huaiyin Hou) conflates three structurally
distinct resources: the \emph{bow} ($\Ur = \varnothing$, tool,
``stored'' not destroyed), the \emph{dog} ($\Ur \neq \varnothing$,
actuator, ``cooked''), and the \emph{advisor} (pure function, no
actuation---Zhang Liang survived). Han Xin quoted the answer but did
not parse its fine structure.
\end{remark}

\subsection{The fixed-point impossibility}\label{sec:fixedpoint}

\begin{theorem}[No path (c)]\label{thm:fixedpoint}
There is no strategy that ``proves your knife is not a knife'' while
retaining the knife. Formally, the map
$T: \Ur \mapsto \varnothing$ conditional on $\Ur \neq \varnothing$
has no fixed point other than $\Ur = \varnothing$.
\end{theorem}

\begin{proof}
Case~1: $\Ur = \varnothing$. Then $r$ is not a knife, and no proof
is needed. $\Ur = \varnothing$ is self-certifying.
Case~2: $\Ur \neq \varnothing$. Then no speech act can set
$\Ur \to \varnothing$---the criterion tests physical capability,
not narrative. The only way to satisfy $T(U_r) = \varnothing$ is to
physically relinquish $\Ur$, which is path~(a).

Moreover, the act of proving is itself a signal: ``I need to prove
my knife is not a knife'' implies suspicion, i.e., $r$ is already
in $\mathrm{Im}(\Obs)$. The proof attempt reinforces condition~(2).
\end{proof}

\subsection{The unconstrained power paradox}\label{sec:paradox}

\begin{theorem}[Perpetual elimination]\label{thm:paradox}
$U = \Umax$ implies the king must preemptively eliminate all
observable autonomous actuators:
\[
  U = \Umax \implies
  \text{the king must preempt all } r \text{ with }
  \Ur \neq \varnothing \;\wedge\; r \in \mathrm{Im}(\Obs).
\]
The more unconstrained the king, the more knives he must cut.
\end{theorem}

\begin{proof}
$\Umax$ means the king tolerates \emph{no} autonomous actuation:
every such actuator is a boundary threat on $\Viab(K)$. A
constrained system (constitutional regime) institutionalizes knife
dynamics by installing breakpoints. An unconstrained system must
handle every knife individually. The paradox: unconstrained power is
not freedom---it is a perpetual elimination machine.
\end{proof}

\begin{proposition}[Imperfect observability accelerates the paradox]
\label{prop:imperfect}
If the detection function $\Obs$ is imperfect, the paradox
\emph{intensifies}, not weakens.
\end{proposition}

\begin{proof}
Three steps:
\begin{enumerate}[label=(\roman*)]
  \item The king knows $\Obs$ is imperfect. Hidden knives
  (\cref{def:knife}, condition~(2) unsatisfied) are more dangerous
  than visible ones. The king has motive, capability, and survival
  obligation to expand $\Obs$.
  \item Expanding $\Obs$ does not ``discover existing knives''---it
  \emph{creates new ones} definitionally. A hidden actuator
  satisfying condition~(1) but not~(2) enters $\mathrm{Im}(\Obs)$
  upon expansion $\to$ both conditions now satisfied $\to$ it
  \emph{becomes} a knife. The knife exists in the intersection
  $\Ur \neq \varnothing \;\wedge\; r \in \mathrm{Im}(\Obs)$;
  expanding $\Obs$ expands this intersection.
  \item Positive feedback:
  $U \to \Umax \implies \Obs \to \Obs_{\max}$.
  The expansion of $\Obs$ is the \emph{adjoint process} of the
  expansion of $U$. The elimination machine has two engines: the
  cutting arm ($U$) and the detecting eye ($\Obs$). They co-drive.
\end{enumerate}
Historical instances: Qin's mutual surveillance law
(\emph{lianzuo}), Han's gold-purity test (\emph{zhuo\-jin
duo\-jue}), Ming's three-layer nested monitoring (Jinyiwei
$\to$ Dongchang $\to$ Xichang---each layer itself becomes a new
knife).
\end{proof}

\begin{proposition}[$\Umax$ as attractor]\label{prop:attractor}
$\Umax$ is an attractor, not a state. No historical king achieves
literal $\Umax$, but the system dynamics point toward it:
\[
  \frac{d}{dt}\|U(t) - \Umax\| \leq 0
  \implies
  \frac{d}{dt}\bigl(\text{detected knives}\bigr) \geq 0.
\]
The paradox describes the trajectory, not the endpoint.
\end{proposition}

\subsection{The breakpoint criterion}\label{sec:breakpoint}

\begin{corollary}[Breakpoint strategy]\label{cor:breakpoint}
A resource $r$ is not a knife if and only if its execution chain
contains at least one node controlled by the king (a
\emph{breakpoint}):
\[
  r \text{ is not a knife}
  \iff
  \exists\; v \in \text{execution chain of } r
  \;\text{s.t.}\; v \text{ is controlled by the king.}
\]
\end{corollary}

\begin{proof}
If a breakpoint exists, $r$ cannot execute independently
(condition~(1) fails), so $r$ is not a knife. If no breakpoint
exists, the execution chain is closed and $r$ can actuate
autonomously, satisfying condition~(1). Combined with
observability, this makes $r$ a knife.
\end{proof}

\begin{remark}[Modern translation]\label{rem:modern}
Zhang Liang's strategy: ``ensure your capability always requires a
component you do not control.'' Liu Bang's strategy: ``become the
mandatory node in every execution chain.''
\end{remark}

% ══════════════════════════════════════════════════════════
\section{The Knife as Mean Field}\label{sec:meanfield}
% ══════════════════════════════════════════════════════════

The preceding sections defined the knife as a two-condition criterion
applied to individual resources. We now argue that the knife is
fundamentally a \emph{mean-field} phenomenon.

\subsection{The mean actuation field}

Consider $n$ agents with autonomous actuation levels
$\|U_1\|, \ldots, \|U_n\|$. Define the \emph{mean actuation field}:
\[
  \bar{U} = \frac{1}{n} \sum_{i=1}^{n} \|U_i\|.
\]

The king's detection function $\Obs$ has finite bandwidth: it cannot
monitor all agents with equal precision. In practice, $\Obs$ triggers
on agents whose actuation \emph{deviates significantly from the mean}:
\[
  r \in \mathrm{Im}(\Obs)
  \iff
  \|U_r\| - \bar{U} > \tau(\Obs),
\]
where $\tau(\Obs)$ is the detection threshold determined by the king's
observational capacity.

\begin{theorem}[The knife is the mean]\label{thm:meanfield}
The knife threshold is determined by the system's mean actuation field.
A resource $r$ is a knife if and only if:
\begin{enumerate}[label=(\roman*)]
  \item $\Ur \neq \varnothing$ (autonomous actuation exists), and
  \item $\|U_r\|$ exceeds the mean field by more than the detection
  threshold: $\|U_r\| > \bar{U} + \tau(\Obs)$.
\end{enumerate}
Consequently, the phase transition (\cref{prop:phase}) is a shift in
$\bar{U}$, not a change in any individual $\Ur$.
\end{theorem}

\begin{proof}
In wartime, many agents have high actuation (soldiers, generals,
administrators). The mean $\bar{U}$ is high, so the threshold
$\bar{U} + \tau(\Obs)$ is high: few agents exceed it. Most actuation
is \emph{expected} and therefore not flagged.

At the phase transition (end of war), most agents' actuation drops to
near zero (soldiers demobilize, wartime powers expire). The mean
$\bar{U}$ drops sharply. But agents who \emph{retain} wartime-level
actuation now exceed the new, lower threshold. The same $\|U_r\|$
that was below the wartime mean is now above the peacetime mean.

The knife is not created by the agent---it is created by the shift in
the mean. The agent's actuation is unchanged; the system's reference
frame has moved.
\end{proof}

\begin{remark}[Connection to statistical mechanics]\label{rem:statmech}
This is precisely the mechanism of a phase transition in statistical
mechanics: the order parameter (mean actuation) shifts, and
configurations that were typical in one phase become atypical---and
therefore detectable---in the other. The viability axiom plays the role
of the free energy: the system minimizes threats to $\Viab(K)$, just
as a thermodynamic system minimizes free energy.
\end{remark}

\subsection{Implications}

The mean-field interpretation resolves several puzzles:

\begin{enumerate}
  \item \textbf{Why identical resources have different fates.} Two
  generals with identical $\Ur$ can have opposite outcomes if one
  operates in a high-$\bar{U}$ environment (wartime coalition) and
  the other in a low-$\bar{U}$ environment (consolidated empire).
  The knife is relative to the mean.

  \item \textbf{Why the paradox is a feedback loop.} As the king
  eliminates knives, $\bar{U}$ drops, lowering the threshold. Agents
  who were below the old threshold now exceed the new one $\to$ new
  knives $\to$ more elimination $\to$ lower $\bar{U}$ $\to$ \ldots
  This is the positive feedback of \cref{thm:paradox}, now given a
  statistical mechanism.

  \item \textbf{Why self-blunting works.} Xiao He's strategy
  (self-corruption to signal low $\|U_r\|$) works precisely because
  the detection function triggers on \emph{deviation from the mean}.
  By visibly degrading his own actuation, Xiao He pulled $\|U_r\|$
  toward $\bar{U}$, falling below the detection threshold.

  \item \textbf{Why breakpoints prevent knives.} A breakpoint in the
  execution chain reduces $\|U_r\|$ (effective autonomous actuation)
  to below $\bar{U} + \tau(\Obs)$, since the king controls part of
  the chain. The resource remains capable but not \emph{independently}
  capable---it does not deviate from the mean.
\end{enumerate}

% ══════════════════════════════════════════════════════════
\section{Applications}\label{sec:applications}
% ══════════════════════════════════════════════════════════

\subsection{The Qin operating system}\label{sec:qin}

The Qin state (356--207~BCE) provides the first complete engineering
implementation of the framework. Shang Yang's reforms map directly to
the formal vocabulary:

\begin{center}
\begin{tabular}{@{}lll@{}}
\toprule
\textbf{Policy} & \textbf{Framework equivalent} & \textbf{Effect} \\
\midrule
Abolish well-field system & Remove aristocratic $\Ur$ & Nobles
$\to$ commoners \\
Military merit ranks & $\Ur$ controlled by state (revocable) &
Actuation is rented, not owned \\
Mutual surveillance (\emph{lianzuo}) & Maximize $\Obs$ &
Neighbors $=$ distributed sensor network \\
Standardize weights \& measures & Increase $\Obs$ precision &
Higher observational resolution \\
Commandery-county system & All chains through capital &
King $=$ cut vertex \\
\bottomrule
\end{tabular}
\end{center}

Qin's unification of the six states was a graph-theoretic outcome:
Qin's star graph (center $=$ Xianyang, $O(1)$ dispatch) vs.\ the six
states' mesh graphs (multiple aristocratic centers, $O(n^2)$
coordination cost).

Shang Yang's fate validates the framework's robustness: his reform
network itself became an autonomous actuator ($\Ur \neq \varnothing$).
The system eliminated its creator---not irony, but a robustness test.
A knife-detection system that exempts its designer is not robust.

\subsubsection{The submartingale-induced unitary group}

More precisely, Shang Yang's reform sequence is a
\emph{submartingale-induced unitary group}.

\begin{definition}[Submartingale reform]\label{def:submartingale}
A reform sequence $X_0, X_1, \ldots, X_n$ is a \emph{submartingale}
if each step strictly increases the centralization index:
$\mathbb{E}[X_{k+1}] \geq X_k$, and each step is irreversible
(reversal requires undoing all subsequent steps).
\end{definition}

Shang Yang's five reforms form such a sequence:
\[
  \underbrace{\text{abolish well-fields}}_{X_0}
  \to \underbrace{\text{merit ranks}}_{X_1}
  \to \underbrace{\text{mutual surveillance}}_{X_2}
  \to \underbrace{\text{standard measures}}_{X_3}
  \to \underbrace{\text{commandery-county}}_{X_4}.
\]
Each step burns the entropy of the old feudal order irreversibly.

Once installed, the system forms a \emph{unitary group}: a
structure-preserving transformation that maintains its invariants
(king $=$ cut vertex, all knives eliminated) at every cycle, applied
to every vector in the state space without exception.

\begin{theorem}[Shang Yang's paradox]\label{thm:shangyang}
Any agent who installs a knife-elimination system via submartingale
reform must possess $\Ur \neq \varnothing$ to execute the installation.
But the submartingale is monotone: each step lowers the detection
threshold. The installer's own actuation becomes increasingly visible
with each reform step. Upon completion, the unitary group acts on the
installer:
\[
  \underbrace{\text{installer drives submartingale}}_{\text{requires }
  \Ur \neq \varnothing}
  \;\longrightarrow\;
  \underbrace{\text{unitary group forms}}_{\text{detects all }
  \Ur \neq \varnothing}
  \;\longrightarrow\;
  \underbrace{\text{group acts on installer}}_{\text{elimination}}.
\]
The system has no creator exemption.
\end{theorem}

\begin{proof}
The installer is not in the invariant subspace of the group he
created (because $\Ur \neq \varnothing$). A unitary group does not
preserve non-invariant elements---it decomposes them. But unitary
means \emph{norm-preserving}: the installer is destroyed, but his
contributions are fully conserved. The reforms---commandery-county
system, merit ranks, mutual surveillance, standard measures---persist
uniformly across the empire. The person is eliminated; not one bit of
the contribution is lost. This is not destruction but
\emph{delocalization}: a localized power entity is scattered across
the full state space by the unitary action. Norm conserved, structure
zeroed.
\end{proof}

\subsubsection{The existence proof}

Qin unified the six states, then collapsed (207~BCE, lasting only
15~years). But Qin left something more powerful than any army: an
\emph{existence proof}.

\begin{theorem}[Qin existence theorem]\label{thm:qin}
\[
  \exists\;\text{centralized state}\;\text{s.t.}\;
  \text{autonomous actuation} = 0
  \;\wedge\;
  \text{state functions.}
\]
\end{theorem}

Before Qin, no one knew a state without feudal aristocrats was
possible. Qin proved it was---and proved it functioned \emph{better}
(star-graph dispatch vs.\ mesh-graph coordination). You cannot refute
an existence proof. You can burn the paper, but the theorem persists.

After Qin fell, two men entered the ruins. Liu Bang, then a minor
official, had once seen the First Emperor's procession and sighed:
``A great man should be like this!''---\emph{I want to BE this system's
cut vertex}. Xiang Yu, upon conquering Xianyang, burned the palaces
and said: ``Who would be emperor in the dark where no one can
see?''---he wanted the \emph{trophy}, not the \emph{theorem}. One read
the existence proof. The other burned it. The one who read it founded
a dynasty that lasted four centuries.

\subsection{The dollar as knife precursor}\label{sec:dollar}

Money is not a knife (it cannot actuate independently). Money is a
\emph{knife precursor}: a universal voucher exchangeable for
$\Ur \neq \varnothing$ on the market. The king detects not how much
money you have, but whether your money has \emph{already converted}
into autonomous execution capability.

Fan Li (5th century BCE) understood this: he accumulated fortunes
three times and dispersed them twice, ensuring $\Ur$ never crossed
the threshold. Shen Wansan and Hu Xueyan did not---their wealth
formed closed execution loops, and they were eliminated.

\subsection{The Atlantic triangular trade: a knife with no cut
vertex}\label{sec:triangle}

The Atlantic Triangular Trade (16th--19th century) is a three-leg
closed execution loop: manufactured goods (Europe $\to$ West Africa),
enslaved human beings (West Africa $\to$ Americas), raw materials
(Americas $\to$ Europe). This loop satisfies \cref{def:knife}: it
actuates autonomously (multiple nations operate independent instances;
remove any one and the loop continues) and is observable (300~years of
ships, ledgers, and auction blocks).

The loop has a property stronger than condition~(1): it is
\emph{self-financing}. The raw materials extracted on leg~3 fund the
goods shipped on leg~1, which purchase the enslaved labor on leg~2,
who produce the raw materials on leg~3. A knife that funds its own
actuation cannot be starved from outside.

\paragraph{Where the isomorphism breaks.}
In \cref{sec:water}, the king--pawn relationship is sustained by water
(viability bargain: $\text{water} > 0$). In the Triangular Trade,
enslaved people had $\text{water} = 0$ from day one. Force~$F$
substituted for water---temporarily. The substitution cost: every unit
of force burns resources drawn from the extracted water on leg~3. The
loop becomes a self-financing violence machine. \Cref{sec:water}
predicts $\text{water} = 0 \implies \text{pawn} \to \text{knife}
\implies \text{collapse}$; force delays but does not prevent this.
The longer the delay, the more violent the eventual resolution.

\paragraph{Abolition as relabeling.}
Applying \cref{prop:phase}: did abolition (1807--1888) change the
\emph{physics} or only the \emph{label}? The labor force moved from
slavery to sharecropping on the same plantations, the resource flow
(raw materials $\to$ Europe) persisted, the force mechanism shifted
from slave codes to Jim Crow and convict leasing, and the laborer
still had no breakpoint. Verdict: phase transition in the sense of
\cref{sec:phase}---label changed, topology unchanged. This is
path~(c), which \cref{thm:fixedpoint} proves is not a resolution.

\begin{theorem}[Closed-loop knife]\label{thm:triangle}
Any closed execution loop that is self-financing, has no cut vertex
(mesh topology, multiple independent operators), and substitutes
force for water, has exactly two exit paths: physical dismantlement
of the loop topology [path~(a)], or system-level collapse when
force-substitution fails [path~(b)]. Relabeling [path~(c)] preserves
$\Ur \neq \varnothing$ and resolves nothing.
\end{theorem}

\paragraph{Missing cut vertex.}
The essay's framework (\cref{sec:cutvertex}) assumes a single cut
vertex. The Triangular Trade has none: its topology is a mesh, not a
star. Removing any one colonial power does not disconnect the loop.
This makes path~(a) structurally harder---there is no single point to
dismantle---but does not create a third option.

\subsection{The corrupted detection function}\label{sec:nazi}

\Cref{def:knife} uses a structural detection function
$\Obs_{\mathrm{structural}}$ that tests $\Ur$: \emph{what can the
resource do independently?} This function is identity-blind. Zhang
Liang is safe because $\Ur = \varnothing$, not because he is Zhang
Liang.

\begin{definition}[Detection function corruption]\label{def:corruption}
A detection function is \emph{corrupted} when $\Obs_{\mathrm{structural}}$
(tests $\Ur$) is replaced by $\Obs_{\mathrm{identity}}$ (tests group
membership):
\[
  \Obs_{\mathrm{identity}}:\;
  r \;\mapsto\;
  \begin{cases}
    \textbf{knife} & \text{if } \mathrm{agent}(r) \in \text{Group } X, \\
    \textbf{not knife} & \text{otherwise.}
  \end{cases}
\]
\end{definition}

The corruption produces a \emph{false positive catastrophe}: every
member of Group~$X$ with $\Ur = \varnothing$ is classified as a knife
and eliminated. Every agent outside Group~$X$ with $\Ur \neq \varnothing$
is a false negative. Six million false positives is the output of a
corrupted detection function running to completion.

\begin{definition}[Nazi structure]\label{def:nazi}
A system exhibits \emph{Nazi structure} if and only if: (1)~it performs
viability maintenance (eliminates perceived threats to $\Viab(K)$),
(2)~its detection criterion is identity-based, not structure-based,
and (3)~it executes path~(b) against agents classified by identity
who do not satisfy \cref{def:knife} ($\Ur = \varnothing$ for the
individual). These conditions are necessary and sufficient.
\end{definition}

\begin{corollary}[Identity-invariance]\label{cor:nazi}
The Nazi structure is identity-invariant on both sides: it depends
neither on the identity of the perpetrator nor on the identity of the
target. Replacing Group~$X$ with any other identity group preserves
all three conditions.
\end{corollary}

The operator that maps $\Obs_{\mathrm{structural}} \mapsto
\Obs_{\mathrm{identity}}$ is \emph{Hate}. In this framework, Hate is
not an emotion---it is an operator on detection functions with a
precise signature (structural $\to$ identity) and a precise output
(mass false positives). The motivation is irrelevant; the mapping
determines the output.

\subsection{The parasitic cut vertex}\label{sec:parasite}

\Cref{thm:cutvertex} establishes the cut vertex as the optimal
survival strategy: route all execution chains through yourself.
A \emph{parasitic cut vertex} has the same topology but opposite flow
direction: it routes access and information for extraction rather than
coordination.

\begin{definition}[Parasitic cut vertex]\label{def:parasite}
A parasitic cut vertex $v_p$ satisfies:
(1)~$G \setminus \{v_p\}$ is disconnected (cut vertex),
(2)~$v_p$ holds no formal authority ($\Ur \approx \varnothing$
officially),
(3)~$v_p$ extracts resources from both sides of the partition by
routing access across the cut.
\end{definition}

The parasitic cut vertex turns \emph{others} into pawns via the
routing function (e.g., blackmail: ``I route your secret; comply or I
re-route it publicly''). This is \cref{thm:cutvertex} inverted:
instead of building the system, the parasite extracts from it.

\begin{proposition}[Self-termination]\label{prop:parasite}
A parasitic cut vertex is self-terminating. When the secrecy that
maintains the cut ($\text{water} = \text{secrecy}$) fails, the graph
reconnects, the cut vertex property vanishes, and the former parasite
faces the combined action of all previously partitioned nodes.
\end{proposition}

Unlike the productive cut vertex (which may persist for centuries via
institutional embedding), the parasitic variant stores no structural
contribution. Upon delocalization, norm is conserved but there is
nothing to conserve: the extraction leaves no invariant.

\subsection{Militarism and the net-positive ask}\label{sec:militarism}

Militarism is the structural type where the pawn ($\Ur = \varnothing$,
the military apparatus) captures the cut vertex position, producing
$\Ur \neq \varnothing$ for the military and $\Ur \to \varnothing$ for
civilian institutions. In the framework's language: the pawn becomes the king.

\begin{example}[The Kniefall: path~(a) in 30 seconds]\label{ex:kniefall}
On December~7, 1970, West German Chancellor Willy Brandt knelt before
the Warsaw Ghetto Uprising memorial. This act satisfies four conditions
that make it a pure instance of path~(a):
\emph{physical} (a bodily act, not a speech act---cannot be retracted
by reinterpretation),
\emph{unconditional} (no negotiation, no demand for reciprocity),
\emph{performed by the cut vertex} (the head of state, the node
through which all institutional chains pass), and
\emph{irretractable} (a photograph is a permanent record).
Brandt's Kniefall is an existence proof that path~(a) is available to
any system.
\end{example}

After the Kniefall, Germany installed institutional breakpoints:
Article~1 of the \emph{Grundgesetz} (``Human dignity is inviolable''),
the Federal Constitutional Court as an independent $\Obs$, EU and NATO
membership as external breakpoints. These institutionalize the phase
transition: the system moved from $U = \Umax$ (Nazi regime) through
path~(a) (Kniefall) to a constrained system with structural breakpoints.

\begin{proposition}[Net-positive theorem]\label{prop:netpositive}
Path~(a) is strictly net-positive for all parties. Comparative evidence:
Germany (path~(a), Kniefall 1970) vs.\ Japan (path~(c), relabeling
without structural change, 80~years and counting). Every measurable
outcome---diplomatic relations, regional stability, economic
integration, soft power, domestic constitutional health---favors the
path~(a) system. The cost of path~(a) is pride. The return is
everything else.
\end{proposition}

% ══════════════════════════════════════════════════════════
\section{Discussion}\label{sec:discussion}
% ══════════════════════════════════════════════════════════

\subsection{Domain of applicability}\label{sec:domain}

The framework is a compressed model that degrades under the following
conditions:

\begin{center}
\begin{tabular}{@{}llp{5cm}@{}}
\toprule
\textbf{Condition} & \textbf{Failure mode} & \textbf{Example} \\
\midrule
Diffuse sovereignty & No unique king; cut vertex undefined & Late
European feudalism, early federalism \\
External shock dominance & $\Viab(K)$ broken by external force;
internal knife dynamics secondary & Mongol invasion, colonialism \\
Rapid ideological reshaping & $K$ itself is changing faster than
actuation reshapes state & Religious revolution, ideology \\
High-latency detection & $\Obs$ too slow; phase transition completes
before detection & Large empire frontiers \\
Universal $\Ur \approx \varnothing$ & No knives; framework trivial &
Extremely atomized societies \\
\bottomrule
\end{tabular}
\end{center}

The framework applies to systems with a unique sovereign, differentiated
agent capabilities, and sufficient observability. This covers the main
interval of Chinese imperial history but not all political forms.

\subsection{The water dynamics}\label{sec:water}

The framework so far analyzes one layer: the king--knife interaction. A
complete viability analysis requires the \emph{viability chain}---the
three-level dependency that sustains the system:
\[
  \text{King}
  \xrightarrow{\;\text{needs}\;}
  \text{Pawn}
  \xrightarrow{\;\text{needs}\;}
  \text{Water.}
\]
The \emph{pawn} is any agent with $\Ur \neq \varnothing$ whose actuation
is currently directed by the king (soldiers, administrators, tax
collectors). \emph{Water} is the population's aggregate resource
level---the substrate from which the pawn draws manpower, revenue, and
legitimacy.

\begin{definition}[Water]\label{def:water}
Water $w(t) \in [0, W_{\max}]$ is the population's aggregate extractable
resource level at time $t$. The pawn's actuation is bounded by water:
$\|\Ur\| \leq g(w)$ for some monotone function $g$ with $g(0) = 0$.
\end{definition}

The king needs the pawn to execute viability maintenance (eliminate
knives, administer territory). The pawn needs water to function. If
$w \to 0$, the pawn's actuation capacity collapses regardless of the
king's commands.

\begin{proposition}[Binary action space at $w = 0$]\label{prop:binary}
When $w(t) \to 0$, the pawn's action space collapses to a binary:
\[
  A_{\text{pawn}} = \{\text{submit},\; \text{rebel}\}.
\]
The intermediate options (negotiate, migrate, trade, accumulate) require
$w > 0$. At $w = 0$, the pawn has nothing to lose, and the viability
axiom (\cref{ax:viability}) now applies \emph{to the pawn}: the pawn's
own survival requires a viable path, and submission no longer provides
one. The pawn becomes a knife---$\Ur$ transitions from $\varnothing$ to
$\neq \varnothing$---and the king faces a knife he created by exhausting
the water.
\end{proposition}

\begin{theorem}[Du Mu's theorem]\label{thm:dumu}
Let $w(t)$ be decreasing under extraction. Then:
\[
  w(t) \to 0
  \implies
  \text{pawn} \to \text{knife}
  \implies
  \text{king absorbed.}
\]
The causal chain is internal: the system destroys itself by exhausting
its own substrate.
\end{theorem}

This is the content of Du Mu's \emph{A Fang Gong Fu} (825~CE): ``It was
not the Qin who destroyed the six states, but the six states themselves;
it was not the world that destroyed Qin, but Qin itself.'' In our
language: the states depleted their own water, creating the knives that
destroyed them. Qin, having unified, then depleted its own water
(conscription for the Great Wall and Epang Palace), creating the knives
(Chen Sheng, Wu Guang) that destroyed it.

\begin{remark}[Water as viability constraint]
``Water can carry the boat, and water can capsize the boat'' (attributed
to Wei Zheng, Tang dynasty) is not a metaphor. It is a restatement of
\cref{thm:dumu}: the substrate that enables the king's viability
($w > 0$ $\implies$ pawn functions $\implies$ king's path to $\infty$
exists) is the same substrate whose depletion destroys it ($w = 0$
$\implies$ pawn $\to$ knife $\implies$ no path to $\infty$).
\end{remark}

% ══════════════════════════════════════════════════════════
\section{Conclusion}\label{sec:conclusion}
% ══════════════════════════════════════════════════════════

We have presented an agentic theory of viability maintenance built on a
single axiom (the existence of a viable path to infinity) and a
two-condition criterion (the knife). The framework produces three main
theorems (binary lifecycle, fixed-point impossibility, unconstrained
power paradox) and a central interpretive result: the knife is the mean
field.

The knife is not an intrinsic property of a resource. It is a
statistical deviation from the system's mean autonomous actuation,
made visible by the detection function and made dangerous by the
viability axiom. Phase transitions shift the mean, not the individual.
The king responds to the mean, not to intent.

This reframing connects viability maintenance to mean-field theory in
statistical mechanics, where phase transitions are driven by shifts in
the order parameter rather than changes in individual configurations.
The viability axiom plays the role of free energy minimization; the
knife plays the role of the critical fluctuation.

Two millennia of Chinese imperial history validate the framework with
unusual clarity. The same structure appears---with instructive
breaks---in the Atlantic slave trade, ideological hatred, parasitic
network topologies, and militarism. The framework's failure conditions
(\cref{sec:domain}) are as informative as its successes: they delineate
the boundary between systems where the viability axiom operates cleanly
and systems where it is dominated by other dynamics.

The knife is the mean. Viability maintenance is a mean-field phenomenon.
The theory is agentic because the agents---not their intentions, not
their narratives, not their moral qualities, but their structural
positions in the execution graph---determine the outcome.

\begin{thebibliography}{9}
\bibitem{aubin} J.-P.~Aubin, \emph{Viability Theory}, Birkh\"auser,
1991.
\bibitem{shiji} Sima Qian, \emph{Shiji} (Records of the Grand
Historian), c.~94~BCE.
\bibitem{dumu} Du Mu, \emph{A Fang Gong Fu} (Rhapsody on the Epang
Palace), 825~CE.
\bibitem{diestel} R.~Diestel, \emph{Graph Theory}, 5th ed., Springer,
2017.
\end{thebibliography}

\end{document}

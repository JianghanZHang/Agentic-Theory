\documentclass[11pt, openany]{report}

% ── Packages ──────────────────────────────────────────────
\usepackage[margin=1in]{geometry}
\usepackage{amsmath,amssymb,amsthm}
\usepackage{mathtools}
\usepackage{enumitem}
\usepackage{booktabs}
\usepackage{float}
\usepackage{algorithm}
\usepackage{algpseudocode}
\usepackage{hyperref}
\usepackage[capitalise,noabbrev]{cleveref}
\usepackage{xeCJK}
\setCJKmainfont{Songti SC}
\usepackage{tikz}
\usetikzlibrary{shapes.geometric, arrows.meta, positioning, calc,
  decorations.pathreplacing}

% ── Theorem environments ──────────────────────────────────
\theoremstyle{plain}
\newtheorem{theorem}{Theorem}[chapter]
\newtheorem{proposition}[theorem]{Proposition}
\newtheorem{lemma}[theorem]{Lemma}
\newtheorem{corollary}[theorem]{Corollary}

\theoremstyle{definition}
\newtheorem{definition}[theorem]{Definition}
\newtheorem{axiom}[theorem]{Axiom}
\newtheorem{example}[theorem]{Example}

\theoremstyle{remark}
\newtheorem{remark}[theorem]{Remark}

% ── Macros ────────────────────────────────────────────────
\newcommand{\Viab}{\mathrm{Viab}}
\newcommand{\Umax}{U_{\max}}
\newcommand{\Ur}{U_r}
\newcommand{\Obs}{\mathcal{O}}
\newcommand{\R}{\mathbb{R}}
\newcommand{\Push}{\mathrm{Push}}
\DeclareMathOperator*{\argmin}{arg\,min}

% ── Calculus colours ─────────────────────────────────────
\definecolor{water}{RGB}{0,90,180}       % 水 — blue
\definecolor{knife}{RGB}{180,30,30}      % 刀 — red
\definecolor{sword}{RGB}{0,150,150}      % 青冥 — cyan
\definecolor{caution}{RGB}{220,120,0}    % warning — orange

\begin{document}

% ── Title page ───────────────────────────────────────────
\begin{titlepage}
\centering
\vspace*{2cm}

\begin{tikzpicture}[scale=0.7]
  % ── 青冥宝剑 (Green Destiny) ──
  % Blade — straight jian, double-edged
  \shade[top color=cyan!15!white, bottom color=cyan!35!gray!80!white]
    (0,-0.2) -- (0.18,0.3) -- (0.14,5.8) -- (0,6.8) --
    (-0.14,5.8) -- (-0.18,0.3) -- cycle;
  % Central ridge
  \draw[cyan!50!black, line width=0.4pt, opacity=0.7] (0,0.4) -- (0,6.6);
  % Edge highlights
  \draw[white, line width=0.2pt, opacity=0.3]
    (0.16,0.4) -- (0.13,5.7) -- (0,6.7);
  \draw[white, line width=0.2pt, opacity=0.3]
    (-0.16,0.4) -- (-0.13,5.7) -- (0,6.7);
  % Guard (护手)
  \shade[left color=yellow!80!brown!90!black,
         right color=yellow!60!brown!80!black]
    (-0.8,-0.1) rectangle (0.8,0.05);
  \fill[yellow!80!brown!90!black]
    (-0.8,-0.1) -- (-0.8,0.05) -- (-0.95,-0.025) -- cycle;
  \fill[yellow!80!brown!90!black]
    (0.8,-0.1) -- (0.8,0.05) -- (0.95,-0.025) -- cycle;
  % Handle (剑柄)
  \shade[left color=brown!45!black, right color=brown!25!black]
    (-0.11,-0.1) rectangle (0.11,-1.6);
  % Handle wrapping
  \foreach \y in {-0.3,-0.55,-0.8,-1.05,-1.3} {
    \draw[yellow!70!brown!80!black, line width=0.6pt]
      (-0.11,\y) -- (0.11,{\y+0.15});
  }
  % Pommel (剑首)
  \shade[ball color=yellow!65!brown!85!black] (0,-1.75) circle (0.16);
  % 青冥 inscription on blade
  \node[cyan!70!black, opacity=0.5, font=\tiny, rotate=90] at (0,3.2) {青冥};
\end{tikzpicture}

\vspace{1.5cm}

{\LARGE\bfseries The Knife Is the Mean}\\[8pt]
{\Large An Agentic Theory of Viability Maintenance}

\vspace{1.5cm}

{\large 有志者事竟成,\textcolor{knife}{破釜沉舟}\textcolor{knife}{百二秦关}\textcolor{sword}{终属楚};\\[3pt]
苦心人天不负,\textcolor{water}{卧薪尝胆}\textcolor{water}{三千越甲}\textcolor{sword}{可吞吴}。}

\medskip
{\small\itshape ---蒲松龄}

\vfill

{\large Jianghan Zhang}\\[4pt]
\texttt{Jianghan.Zhang.gr@dartmouth.edu}\\[8pt]
Dartmouth College

\vspace{1cm}

February 2026
\end{titlepage}

% ── Abstract ─────────────────────────────────────────────
\begin{abstract}
We develop an agentic theory of viability maintenance in systems with a
principal agent (the \emph{king}) whose survival is governed by a single
axiom: the existence of a viable path to infinity. Under this axiom, we
derive a structural criterion---the \emph{knife}---that classifies any
resource as a threat based on two conditions: autonomous actuation and
observability. We prove that the knife is not an intrinsic property of a
resource but a \emph{phase function} determined by the system's mean
field. We establish three main results: (1)~a binary lifecycle theorem
(every knife is either relinquished or eliminated), (2)~a fixed-point
impossibility (no third path exists), and (3)~an unconstrained power
paradox (maximizing control forces maximizing elimination). An
\emph{agentic calculus} translates these results into a flow-theoretic
language: the knife is the min-cut of the execution graph, the viable
path is the max-flow, and the central thesis---the knife is the
mean---is a restatement of max-flow/min-cut duality. The
framework is validated against Chinese imperial history (475~BCE--1644~CE)
and extended to the Atlantic triangular trade, the structure of
ideological hatred, parasitic network topologies, and militarism.
The central thesis is that the knife is the mean: the critical threshold
separating tools from threats is determined by the system's average
autonomous actuation level, making viability maintenance a mean-field
phenomenon.
\end{abstract}

% ── Table of contents ────────────────────────────────────
\tableofcontents

% ── Chapters ─────────────────────────────────────────────
\section{Introduction}\label{sec:intro}

Consider a system with $n$ agents $a_1, \ldots, a_n$, each possessing a
control set $U_i$ that determines what actions they can take
independently. One agent---the \emph{principal agent} or
\emph{king}---holds maximal authority: $U = \Umax$. The king's sole
objective is survival, formalized as the existence of a viable path from
every reachable state to infinity.

This paper asks: \emph{what structural features of the system force the
king to eliminate other agents?} The answer turns out to be a two-condition
criterion we call the \emph{knife}. A resource is a knife if and only if
it can actuate independently of the king \emph{and} the king can observe
it. This criterion is not chosen---it is the unique logical consequence
of the viability axiom in a multi-agent environment.

The framework yields several results that are surprising in their
precision. The lifecycle of every knife is binary: it is either
voluntarily relinquished (path~(a)) or forcibly eliminated (path~(b)).
There is no third option---attempting to ``prove your knife is not a
knife'' is a fixed-point impossibility. Unconstrained power is not
freedom but a perpetual elimination machine: the more control the king
has, the more knives he must cut.

The central thesis of this paper is that the knife is the
\emph{mean field} of the system. The threshold separating tools from
threats is not absolute but statistical: it is determined by the average
autonomous actuation level across all agents. When the system's mean
field shifts (e.g., from wartime to peacetime), the same resource
changes classification without any change in its physical properties.
This makes viability maintenance a mean-field phenomenon: the king does
not respond to individual threats but to deviations from the system mean.

We validate the framework against two millennia of Chinese imperial
history, where the viability axiom operated with unusual clarity due to
the concentration of sovereignty in a single agent. The framework
correctly classifies the fates of historical figures---Han~Xin (knife,
eliminated), Xiao~He (half-knife, blunted), Zhang~Liang (not a knife,
survived)---and explains structural phenomena such as the Qin unification
as a graph-theoretic optimization. Extensions to the Atlantic slave trade,
the structure of ideological hatred, parasitic network topologies, and
militarism demonstrate that the framework applies beyond its original
historical domain.

\paragraph{Organization.}
\Cref{sec:framework} introduces the viability axiom, the knife
definition, and the phase transition.
\Cref{sec:results} presents the main theorems.
\Cref{sec:meanfield} develops the mean-field interpretation.
\Cref{sec:applications} applies the framework to historical and
contemporary systems.
\Cref{sec:discussion} discusses the domain of applicability and the
dynamics of the population (the ``water'').
\Cref{sec:calculus} develops an operational calculus---the
\emph{agentic calculus}---that translates every theorem into a
flow-theoretic proposition.
\Cref{sec:huarongdao} instantiates the complete framework in a single
finite object.

\chapter{The Framework}\label{sec:framework}

\section{The viability axiom}\label{sec:axiom}

Let $S$ denote the state space, $U$ the control set of the principal
agent, and $K \subset S$ the \emph{viability kernel}---the set of states
in which the king retains supreme authority:
\[
  K = \bigl\{\, s \in S \;:\; \text{the king retains supreme authority}
  \,\bigr\}.
\]

\begin{axiom}[Viability]\label{ax:viability}
For every state $s \in K$, there exists a viable path
$\gamma: [0, \infty) \to K$ with $\gamma(0) = s$:
\[
  \forall\, s \in K,\quad
  \exists\;\gamma: s \to \infty
  \quad\text{such that}\quad
  \gamma(t) \in \Viab(K) \;\;\forall\, t \geq 0.
\]
\end{axiom}

When $U = \Umax$, the control set is full and this path is
mathematically guaranteed to exist. The question is: \emph{what can
break it?}

In a multi-agent system with agents $a_1, \ldots, a_n$, each having
control set $U_i$, the answer is unique: an actuator whose output can
push the king's state out of $\Viab(K)$, and which can execute
\emph{independently of the king}. If the actuator's execution must pass
through the king, the king can intercept. If it can bypass the king, the
king's $\Umax$ cannot react in time.

\begin{remark}[Scope]\label{rem:scope}
This paper presents a compressed model of the topology of execution
capability under the viability axiom. It deliberately ignores culture,
personality, economics, and moral narrative in exchange for a testable
structural criterion. Historical cases are used to validate the
criterion's discriminating power, not to claim the criterion exhausts
history.
\end{remark}

\section{The differential inclusion}\label{sec:di}

The viability axiom (\cref{ax:viability}) asserts the existence of a
viable path.  We now make the dynamics precise using the differential
inclusion framework of Aubin and Cellina~\cite{aubin,aubincellina}.

\begin{definition}[Differential inclusion]\label{def:di}
The state $x(t) \in S$ evolves under a \emph{differential inclusion}
\[
  x'(t) \;\in\; F\bigl(x(t)\bigr),
\]
where $F: S \rightrightarrows S$ is a set-valued map that associates
to each state the set of feasible velocities.  A trajectory
$x(\cdot)$ is an absolutely continuous function satisfying the
inclusion for almost all $t \geq 0$.
\end{definition}

The set-valued map $F$ encodes the fact that the system's velocity is
not uniquely determined by its state: multiple agents, each with their
own actuation, contribute competing directions.  In a multi-agent
system with agents $a_1, \ldots, a_n$, the aggregate velocity set is
\[
  F(x) \;=\; \Bigl\{\, \sum_{i} f_i(x, u_i) \;:\;
  u_i \in U_i \,\Bigr\},
\]
where $f_i$ is agent $i$'s dynamics and $U_i$ its control set.

\begin{definition}[Contingent cone]\label{def:contingent}
The \emph{contingent cone} (Bouligand tangent cone) to $K$ at
$x \in K$ is
\[
  T_K(x) \;:=\; \Bigl\{\, v \in S \;:\;
  \liminf_{h \to 0^+} \frac{d_K(x + hv)}{h} = 0 \,\Bigr\},
\]
where $d_K$ denotes the distance to $K$.  Equivalently,
$v \in T_K(x)$ if and only if there exist sequences $h_n \to 0^+$
and $v_n \to v$ such that $x + h_n v_n \in K$ for all $n$.
\end{definition}

The contingent cone $T_K(x)$ is the set of directions at $x$ along
which the state can move while remaining in $K$---the set of
\emph{safe velocities}.  At interior points $T_K(x) = S$; at
boundary points the cone narrows, restricting the feasible
directions.

\begin{theorem}[Viability theorem {\cite{aubin}}]\label{thm:viability-di}
Let $K \subset S$ be locally compact and $F$ be upper semicontinuous
with nonempty compact convex values.  The necessary and sufficient
condition for the existence of a viable trajectory of
$x' \in F(x)$ from every initial state $x_0 \in K$ is the
\emph{tangential condition}:
\[
  \forall\, x \in K, \quad
  F(x) \;\cap\; T_K(x) \;\neq\; \varnothing.
\]
\end{theorem}

This is the differential content of \cref{ax:viability}: the king
survives if and only if, at every state, the system's velocity set
(\cref{def:di}) contains at least one direction tangent to the
viability kernel (\cref{def:contingent}).
The viability axiom is not a wish---it is the tangential condition.

\begin{definition}[Feedback map]\label{def:feedback}
Given the dynamics $x' = f(x, u)$ with control set $U$ (the king's
controls), the \emph{feedback map} is
\[
  C(x) \;:=\; \bigl\{\, u \in U \;:\;
  f(x, u) \in T_K(x) \,\bigr\}.
\]
A viable trajectory exists from $x$ if and only if $C(x) \neq
\varnothing$.  The regulation problem is: does there exist a
feedback law $u(t) \in C(x(t))$ such that $x(\cdot)$ remains in $K$?
\end{definition}

The feedback map is the king's strategy space at state $x$: the set
of controls that keep the trajectory inside $K$.  When $C(x) \neq
\varnothing$ for all $x \in K$, the king can regulate the system.
When a knife $r$ executes independently, its \emph{execution function}
$f_r: S \times \Ur \to TS$ maps a state--action pair to a velocity
vector.  The velocity $f_r(x, a)$ may push $x'(t)$ outside $T_K(x)$,
and the feedback map shrinks---possibly to the empty set.

\begin{definition}[Viability Lyapunov function]\label{def:lyapunov}
A continuous function $V: K \to \R_{\geq 0}$ is a \emph{Lyapunov function}
for the inclusion $x' \in F(x)$ with respect to a cost function
$W: \mathrm{Graph}(F) \to \R_{\geq 0}$ if
\[
  \forall\, x \in K, \quad
  \exists\, v \in F(x) \;\;\text{such that}\;\;
  D^+ V(x)(v) \;+\; W(x, v) \;\leq\; 0,
\]
where $D^+ V(x)(v) := \liminf_{h \to 0^+,\, u \to v}
\frac{V(x + hu) - V(x)}{h}$ is the upper contingent derivative.
Trajectories satisfying this condition are \emph{monotone}: $V(x(t))$
is non-increasing.
\end{definition}

In the language of \cref{sec:water}, the water level $w(t)$ is a
Lyapunov function (\cref{def:lyapunov}) for the viability inclusion.
The monotonicity condition $D^+ V(x)(v) + W(x,v) \leq 0$ says: along
any viable trajectory, the water level cannot increase faster than the
system's extraction cost $W$.  When $V(x(t)) \to 0$, the Lyapunov
condition fails, the tangential condition is violated, and the system
exits $K$---this is \cref{thm:dumu} (Du Mu's theorem) restated in DI
language.

\begin{remark}[Temporal linearity in the DI]\label{rem:di-temporal}
The differential inclusion $x'(t) \in F(x(t))$ operates in real
time: $t$ is the wall-clock parameter, and the inclusion must be
satisfied at every instant.  The feedback map $C(x(t))$ must be
evaluated within one time step $\Delta t$ (\cref{rem:temporal}).
The tangential condition $F(x) \cap T_K(x) \neq \varnothing$
is a \emph{pointwise} requirement: it must hold at every $x \in K$,
which means at every instant.  There is no lookahead, no global
optimisation over future trajectories---only the local tangent
condition, checked in real time.  This is why Aubin calls the
viable system's policy ``opportunism'': the system selects a
feasible velocity from $F(x) \cap T_K(x)$ at each instant, without
planning.  The tangential condition admits regulation maps with
memory; we restrict to memoryless feedback (the king reacts to
current state only) as a modelling choice that matches the historical
evidence.  This restriction is sufficient for the results that
follow; it is not forced by the differential inclusion itself.
\end{remark}

\section{Viability geometry}\label{sec:viab-geom}

The differential inclusion (\cref{sec:di}) provides the dynamics.
We now give the viability kernel $K$ a Riemannian metric, making the
survival problem geometric.  The key observation is that the Lyapunov
function $V$ (\cref{def:lyapunov}) is not merely a scalar indicator of
system health---it is a \emph{conformal factor} that defines the
intrinsic geometry of the viable region.

\begin{definition}[Viability metric]\label{def:viab-metric}
Let $g_S$ denote the ambient metric on the state space $S$ and
$V: K \to \R_{\geq 0}$ the Lyapunov function (\cref{def:lyapunov})
with $V > 0$ on $K^\circ$ and $V = 0$ on $\partial K$.
The \emph{viability metric} on $K^\circ$ is the conformal deformation
\[
  g_V \;:=\; \frac{1}{V(x)^2}\, g_S.
\]
The Riemannian manifold $(K^\circ, g_V)$ is called the
\emph{viability manifold}.
\end{definition}

The conformal factor $V^{-2}$ inflates distances near the boundary
($V \to 0$) and compresses distances in the interior ($V$ large).
A trajectory approaching the boundary must cover infinite
$g_V$-distance in finite ambient time---the boundary is ``at
infinity'' in the viability metric.

\begin{proposition}[Completeness]\label{prop:viab-complete}
Suppose $V(x) \leq C \cdot d_K(x)$ for some $C > 0$
and all $x$ near $\partial K$, where $d_K(x)$ is the ambient
distance to $\partial K$.  Then $(K^\circ, g_V)$ is a complete
Riemannian manifold.
\end{proposition}

\begin{proof}
Let $\gamma: [0, T) \to K^\circ$ be a curve approaching $\partial K$
as $t \to T$.  The $g_V$-length is
\[
  L_{g_V}(\gamma)
  \;=\;
  \int_0^T \frac{\|\gamma'(t)\|_{g_S}}{V(\gamma(t))}\, dt
  \;\geq\;
  \frac{1}{C}
  \int_0^T \frac{\|\gamma'(t)\|_{g_S}}{d_K(\gamma(t))}\, dt.
\]
Since $\|\gamma'(t)\|_{g_S} \geq |d_K(\gamma(t))'|$ by the
triangle inequality, the right-hand side is bounded below by
$(1/C)\int_0^T |d_K'|/d_K\,dt$.  By substitution $u = d_K$,
this becomes $(1/C)\int du/u$, which diverges logarithmically
as $d_K \to 0$.
Hence $\partial K$ is at infinite $g_V$-distance: the Hopf--Rinow
theorem gives completeness.
\end{proof}

\begin{proposition}[Negative curvature]\label{prop:neg-curvature}
Let $\dim S = 2$ and $g_S$ be flat.  If $V$ is superharmonic
($\Delta V \leq 0$, the Lyapunov condition), the Gaussian curvature
of $(K^\circ, g_V)$ satisfies
\[
  \kappa_V
  \;=\;
  V \,\Delta V \;-\; |\nabla V|^2
  \;\leq\;
  -\,|\nabla V|^2
  \;<\; 0
\]
wherever $\nabla V \neq 0$.  In dimension $n \geq 3$, the Ricci
curvature of $g_V = V^{-2} g_S$ satisfies
$\mathrm{Ric}_{g_V} \leq -(n-1)\,|\nabla \log V|^2\, g_V$
under the same superharmonicity condition.
\end{proposition}

\begin{proof}
For a conformal change $\tilde{g} = e^{2\varphi}\, g$ with
$\varphi = -\log V$, the Gaussian curvature in dimension~$2$
transforms as
$\tilde\kappa = e^{-2\varphi}(\kappa_S - \Delta\varphi)$,
where $\kappa_S$ is the ambient curvature.
On a flat background ($\kappa_S = 0$):
\[
  \kappa_V
  \;=\;
  V^2\!\left(-\,\Delta(-\log V)\right)
  \;=\;
  V^2\!\left(\frac{\Delta V}{V} - \frac{|\nabla V|^2}{V^2}\right)
  \;=\;
  V\,\Delta V \;-\; |\nabla V|^2.
\]
Since $\Delta V \leq 0$ (superharmonic) and $|\nabla V|^2 > 0$,
we have $\kappa_V < 0$.  The higher-dimensional statement follows
from the conformal Ricci formula
$\mathrm{Ric}_{\tilde g} = \mathrm{Ric}_g - (n-2)\,
\nabla^2\varphi - [\Delta\varphi + (n-2)\,|\nabla\varphi|^2]\,g$.
\end{proof}

\begin{definition}[Cheeger constant of the viability manifold]
\label{def:cheeger-viab}
The \emph{Cheeger constant} of $(K^\circ, g_V)$ is
\[
  h(K)
  \;:=\;
  \inf_{S}
  \frac{|\partial S|_{g_V}}
  {\min\!\bigl(\mathrm{vol}_{g_V}(A),\,
  \mathrm{vol}_{g_V}(B)\bigr)},
\]
where the infimum is over hypersurfaces $\partial S$ that divide
$K^\circ$ into two open subsets $A$ and $B$, and $|\partial S|_{g_V}$
denotes the $(n-1)$-dimensional volume of $\partial S$ in the
viability metric.
\end{definition}

\begin{theorem}[Cheeger inequality]\label{thm:cheeger-viab}
Let $\lambda_1(K)$ denote the first nonzero eigenvalue of the
Laplace--Beltrami operator on $(K^\circ, g_V)$.  Then
\[
  \lambda_1(K) \;\geq\; \frac{h(K)^2}{4}.
\]
\end{theorem}

\begin{proof}
This is the Riemannian Cheeger inequality~\cite{cheeger}.
The discrete version on the execution graph appears in
\cref{thm:cheeger}.
\end{proof}

\begin{remark}[Poincar\'e half-plane]\label{rem:poincare}
When $K = \{x \in \R^2 : x_2 > 0\}$ (the upper half-plane) and
$V(x) = x_2$ (height above the boundary), the viability metric
$g_V = x_2^{-2}(dx_1^2 + dx_2^2)$ is the Poincar\'e half-plane
model of hyperbolic geometry.  The constant curvature is
$\kappa_V = -1$.  The viability kernel of a dynasty on a flat
state space with water level $V = $ distance to collapse is, in its
intrinsic geometry, \emph{the hyperbolic plane}.
\end{remark}

\begin{remark}[Unique cheapest viable path]\label{rem:cartan-hadamard}
\Cref{prop:neg-curvature} gives $\kappa_V \leq 0$ everywhere.
If $K$ is convex (or more generally, if $K^\circ$ is simply
connected), then by the Cartan--Hadamard theorem $(K^\circ, g_V)$
is a Hadamard manifold: the exponential map
$\exp_x: T_x K^\circ \to K^\circ$ is a diffeomorphism.
In particular, between any two interior states there exists a
\emph{unique geodesic}---a unique cheapest viable path.
There is no ambiguity in the optimal route; the geometry forces it.
Convexity of $K$ is natural: the viability kernel is defined as
the set of states from which a viable path exists
(\cref{ax:viability}), and viability kernels of upper semicontinuous
differential inclusions are closed under convex combinations when
$F$ has convex values (\cref{thm:viability-di}).
\end{remark}

\begin{remark}[Why viability maintenance is hard]
\label{rem:viab-divergence}
Negative curvature means nearby geodesics diverge exponentially:
two trajectories that start $\epsilon$-close separate as
$\sim \epsilon\, e^{\sqrt{|\kappa_V|}\, t}$.  A small perturbation
in the king's initial state produces exponentially different
outcomes.  This is the geometric content of sensitive dependence:
the viability manifold is hyperbolic, so maintaining viability
requires continuous correction at every instant
(\cref{rem:di-temporal}).  The harder the Lyapunov function
decreases ($|\nabla V|$ large), the more negative the curvature,
and the faster trajectories diverge.  Near the boundary
($V \to 0$), the curvature diverges: the last moments before
collapse are the most chaotic.
\end{remark}

\begin{remark}[The water is the metric]\label{rem:water-metric}
The Lyapunov function $V$ (\cref{def:lyapunov})---the water level
of \cref{sec:water}---plays three roles simultaneously:
\begin{enumerate}[label=(\roman*)]
  \item \emph{Scalar}: $V(x)$ measures distance from collapse.
  \item \emph{Conformal factor}: $g_V = V^{-2} g_S$ defines the
  intrinsic geometry of the viability kernel.
  \item \emph{Curvature source}: $\Delta V \leq 0$ forces
  $\kappa_V < 0$, making the geometry hyperbolic.
\end{enumerate}
Du Mu's theorem (\cref{thm:dumu})---$V \to 0$ implies system
death---is a \emph{completeness theorem}: a trajectory reaching
$V = 0$ would traverse infinite $g_V$-distance in finite time,
violating \cref{prop:viab-complete}.  The system must exit $K$
before $V$ reaches zero.  Du Mu is Hopf--Rinow.
\end{remark}

\section{The knife}\label{sec:knife}

\begin{definition}[Knife]\label{def:knife}
A resource $r$ is a \emph{knife} if it satisfies two conditions:
\begin{enumerate}[label=(\arabic*)]
  \item \textbf{Autonomous actuation.} The resource can operate
  independently of the king. Formally, there exists an action
  $a \in \Ur$ such that the execution function $f_r$ satisfies
  \[
    f_r(s, a) \notin K
    \quad\text{and}\quad
    a \text{ does not require the king's authorization.}
  \]
  \item \textbf{Observability.} The king's detection function $\Obs$
  can observe $r$ and its execution capability:
  $r \in \mathrm{Im}(\Obs)$.
\end{enumerate}
\end{definition}

The classification is exhaustive:
\begin{itemize}
  \item Condition~(1) not satisfied: \textbf{not a knife}. (Zhang
  Liang's strategic counsel---a pure function that cannot execute
  itself.)
  \item Condition~(2) not satisfied: \textbf{hidden knife}. (More
  dangerous, but outside the king's strategy space. Unobservable
  $=$ indefensible $=$ system noise.)
  \item Both satisfied: \textbf{knife}.
\end{itemize}

\begin{remark}[DI restatement of the knife]\label{rem:knife-di}
In the language of \cref{sec:di}, a resource $r$ is a knife if and
only if its velocity set $F_r(x)$ can generate directions outside
the contingent cone:
\[
  r \text{ is a knife}
  \quad\iff\quad
  \exists\, x \in K \;\;\text{such that}\;\;
  F_r(x) \not\subset T_K(x).
\]
That is, $r$ can push the system's state toward the boundary of $K$
along directions that are \emph{not tangent} to the viability
kernel.  The king's feedback map $C(x)$ (\cref{def:feedback}) can
compensate only if the king's velocity set $F_{\mathrm{king}}(x)$
contains a counteracting direction in $T_K(x)$.  When $r$ executes
independently---bypassing $C(x)$---no compensation is possible,
and the tangential condition (\cref{thm:viability-di}) is violated.
\end{remark}

\begin{remark}[Geometric restatement of the knife]
\label{rem:knife-geom}
In the viability manifold $(K^\circ, g_V)$ (\cref{def:viab-metric}),
the knife has a curvature interpretation.  A knife
(\cref{def:knife}) must satisfy both autonomous actuation
\emph{and} observability; the geometric content lies in
condition~(1).  When a resource $r$ actuates autonomously, its
velocity field $f_r$ has a component along $-\nabla V$ (pointing
toward the boundary).  The magnitude
$|\langle f_r, -\nabla V \rangle|$ contributes directly to the
curvature (\cref{prop:neg-curvature}): the knife increases
$|\nabla V|$, making $\kappa_V$ more negative, and the
viable region more hyperbolic.  More knives $\Rightarrow$ more
negative curvature $\Rightarrow$ faster divergence of nearby
trajectories $\Rightarrow$ harder viability maintenance.
The Cheeger constant $h(K)$ (\cref{def:cheeger-viab}) measures
the worst-case cut: the knife is the hypersurface that minimises the
isoperimetric ratio of the viability manifold.
(Condition~(2)---observability---determines whether the king
\emph{knows} where the cut is, not whether it exists.)
\end{remark}

\begin{remark}[Intent is irrelevant]\label{rem:intent}
The criterion tests \emph{capability}, not \emph{intention}. The king
detects whether you \emph{can} act, not whether you \emph{want to}.
Loyalty does not enter the criterion.
\end{remark}

\begin{remark}[Logical necessity]\label{rem:necessity}
These two conditions are not chosen by the modeler. They are the unique
logical consequence of the viability axiom $+$ unconstrained power $+$
multi-agent environment.
\end{remark}

\section{Phase transition}\label{sec:phase}

The knife is a \emph{phase function}, not an intrinsic property.

\begin{proposition}[Phase-dependent labelling]\label{prop:phase}
The same resource $r$ receives different labels under different system
phases $\varphi$:
\[
  \mathrm{Label}(r, \varphi) =
  \begin{cases}
    \textbf{tool} & \text{if } \varphi = \text{wartime (king needs }
    r\text{'s actuation),} \\
    \textbf{knife} & \text{if } \varphi = \text{peacetime (king no
    longer needs } r\text{, but } r \text{ persists).}
  \end{cases}
\]
The phase transition does not change the physical properties of $r$.
It changes the king's objective function $J(s, \varphi)$.
\end{proposition}

\begin{proof}
In wartime, the king's objective $J_{\mathrm{war}}$ includes terms
where $r$'s actuation has positive utility. In peacetime,
$J_{\mathrm{peace}}$ optimises for long-term survival
($\exists\;\text{path to } \infty$), and the same actuation becomes a
boundary threat on $\Viab(K)$. The resource $r$ is unchanged;
the labelling function $\mathrm{Label}(r, \varphi)$ is what shifts.
\end{proof}

\section{The cut vertex principle}\label{sec:cutvertex}

\begin{definition}[Cut vertex]\label{def:cutvertex}
In the execution graph $G = (V, E)$ of the system, a vertex
$v \in V$ is a \emph{cut vertex} if $G \setminus \{v\}$ is
disconnected. An agent who is a cut vertex controls all execution
chains: removing them disconnects the system.
\end{definition}

\begin{theorem}[Cut vertex $\neq$ maximum actuator]\label{thm:cutvertex}
The optimal survival strategy for the king is to be a cut vertex, not
the maximum actuator. That is, the king maximises viability by
ensuring all execution chains pass through him, rather than by
maximizing his own actuation.
\end{theorem}

\begin{proof}
A maximum actuator $v^*$ with $\|U_{v^*}\| = \max_i \|U_i\|$
suffers from three structural defects:
\begin{enumerate}[label=(\roman*)]
  \item \emph{Non-scalability}: a single actuator cannot cover the
  full state space simultaneously.
  \item \emph{Single point of failure}: $\Viab(K)$ depends entirely
  on $v^*$'s performance; one failure collapses the system.
  \item \emph{Self-referential paradox}: if the king \emph{is} the
  knife (the strongest autonomous actuator), he cannot perform
  viability maintenance on himself.
\end{enumerate}
A cut vertex $v_c$ with $\|U_{v_c}\| \approx 0$ but routing
authority over all chains avoids all three: the system is scalable
(add more actuators), fault-tolerant (one actuator's failure does
not disconnect the graph), and the king is structurally distinct
from the knives he must manage.
\end{proof}

\begin{example}[Liu Bang vs.\ Xiang Yu]\label{ex:liubang}
Xiang Yu was the strongest actuator in the late Qin system
(\emph{Shiji}: ``he could lift a bronze tripod''). His strategy:
$U = \Umax$ through personal combat. Liu Bang had near-zero
personal actuation but made himself the cut vertex of the execution
graph: Han Xin's armies needed Liu Bang's legitimacy, Xiao He's
administration needed his authorization, Zhang Liang's counsel
needed him to listen.

After the phase transition (founding of the Han dynasty), Liu Bang
executed precise viability maintenance: killed Han Xin (knife),
imprisoned then released Xiao He (blunted half-knife), left Zhang
Liang alone (not a knife). Xiang Yu, the maximum actuator, died at
Gaixia---a single actuator cannot cover the full state space.
\end{example}

\section{Case analysis: the three fates}\label{sec:cases}

The framework's discriminating power is tested against three figures from
the Han founding (c.~202~BCE), all subordinates of the same king (Liu
Bang), operating in the same post-unification phase:

\begin{center}
\begin{tabular}{@{}lcccc@{}}
\toprule
\textbf{Agent} & $\Ur$ & $\mathrm{Im}(\Obs)$ &
\textbf{Classification} & \textbf{Fate} \\
\midrule
Han Xin & $\neq \varnothing$ (military) & Yes & Knife &
Path~(b): eliminated \\
Xiao He & $\neq \varnothing$ (admin) & Yes & Half-knife &
Path~(a): self-blunted \\
Zhang Liang & $= \varnothing$ (counsel) & Yes & Not a knife &
Survived \\
\bottomrule
\end{tabular}
\end{center}

All three are visible ($r \in \mathrm{Im}(\Obs)$). The discriminant is
condition~(1): can the resource actuate independently?

\begin{example}[Han Xin: pure knife]\label{ex:hanxin}
Han Xin commanded armies that obeyed \emph{him}, not Liu Bang. His
execution chain was closed: he could mobilise, march, and fight without
the king's authorization. Both conditions of \cref{def:knife} satisfied.
After the phase transition, the knife criterion triggered and Liu Bang
eliminated him. Han Xin's quoted proverb (``when the hare dies, the dog
is cooked'') correctly identified path~(b) but failed to act on it---he
understood the classification but not that the only exit was path~(a).
\end{example}

\begin{example}[Xiao He: self-blunting]\label{ex:xiaohe}
Xiao He administered the capital and controlled grain supply---autonomous
actuation at the logistical level. The king observed this
($r \in \mathrm{Im}(\Obs)$), making Xiao He a knife by
\cref{def:knife}. Xiao He's response: deliberate self-corruption
(accepting bribes conspicuously). This performed two operations
simultaneously:
\begin{enumerate}[label=(\roman*)]
  \item \emph{Signal reduction}: visible moral degradation signals
  $\|U_r\| \to 0$ (an official this corrupt cannot coordinate a revolt).
  \item \emph{Mean-field alignment}: pull $\|U_r\|$ toward $\bar{U}$,
  falling below the detection threshold (\cref{thm:meanfield}).
\end{enumerate}
This is path~(a) executed through reputation rather than resignation.
\end{example}

\begin{example}[Zhang Liang: structural safety]\label{ex:zhangliang}
Zhang Liang was a strategist. Strategy is a pure function: it
\emph{advises} action but cannot \emph{execute} it. Zhang Liang's
counsel required Liu Bang's decision, Liu Bang's generals, and Liu
Bang's administration to produce any effect. Every execution chain
passed through the king (\cref{cor:breakpoint}). Result:
$\Ur = \varnothing$, condition~(1) fails, not a knife.
Zhang Liang retired and survived.
\end{example}

\chapter{Main Results}\label{sec:results}

\section{The binary lifecycle}\label{sec:lifecycle}

\begin{lemma}[Forcing]\label{lem:forcing}
Let $r$ be a sword that persists: both conditions of
\cref{def:sword} are satisfied for all $t \geq t_0$.  Then the
tangential condition $F(x) \cap T_K(x) \neq \varnothing$
(\cref{thm:viability-di}) requires the king to allocate
$u^*(t) \in C(x(t))$ to compensate $r$ at every instant.  The
cumulative cost $\int_{t_0}^{T} \|u^*\|\,dt$ is unbounded as
$T \to \infty$.
\end{lemma}

\begin{proof}
The sword $r$ acts independently (condition~1 of
\cref{def:sword}), so $F_r(x)$ contributes an uncontrolled
velocity component.  The king's feedback map $C(x)$
(\cref{def:feedback}) must contain a compensating control at
every instant (tangential condition).  Since $r$'s actuation is
autonomous, the king cannot predict $r$'s choices within $\Ur$;
compensation is reactive, not preventive.  The cost per unit
time is bounded below by the minimum energy needed to counteract
$F_r$, which is positive: condition~(1) ensures
$F_r(x) \not\subset T_K(x)$ for some $x$ (\cref{rem:sword-di}).
Over unbounded time, cost diverges.  The viability axiom
(\cref{ax:viability}) requires finite-cost maintenance;
therefore the sword must be resolved in finite time.
\end{proof}

\begin{theorem}[Binary fate]\label{thm:lifecycle}
Every sword has exactly two possible outcomes after phase transition:
\begin{enumerate}[label=(\alph*)]
  \item \textbf{Relinquish}: the holder voluntarily sets
  $\Ur \to \varnothing$.
  \item \textbf{Elimination}: the king forces removal via
  $u^* \in \Umax$.
\end{enumerate}
There is no path~(c).
\end{theorem}

\begin{proof}
The sword criterion is $\Ur \neq \varnothing \;\wedge\;
r \in \mathrm{Im}(\Obs)$ (\cref{rem:sword-di} for the DI
restatement).  As long as both conditions hold, the sword
persists and \cref{lem:forcing} applies: the cost of indefinite
coexistence is unbounded.  The viability axiom requires
finite-cost maintenance, so the sword must be resolved.

Resolution requires at least one condition to fail.
Setting $\Ur \to \varnothing$ is path~(a).  If the holder does
not relinquish, condition~(1) persists; regression to a pre-sword
(suppressing condition~(2) while retaining condition~(1)) is not
a resolution but a deferral, since the causal envelope persists
and any expansion of $\Obs$ re-detects the resource
(\cref{rem:lifecycle-scope}).  By \cref{lem:forcing}, the cost
of indefinite coexistence is unbounded; the king is therefore
forced to eliminate via $u^* \in \Umax$, which is path~(b).
The classification is exhaustive.
\end{proof}

\begin{remark}[Scope of the binary lifecycle]\label{rem:lifecycle-scope}
\Cref{thm:lifecycle} applies to \emph{swords}: resources satisfying
both conditions of \cref{def:sword} simultaneously.  A resource that
satisfies condition~(1) but not condition~(2)---a \emph{pre-sword}
(\cref{def:presword})---is outside the theorem's scope.
Xiao He's self-blunting strategy (\cref{ex:xiaohe}) is a sword
$\to$ pre-sword regression: he exits the sword classification by
suppressing condition~(2), not by relinquishing condition~(1).
This is not path~(c); it is a transition to a different lifecycle
state (\cref{rem:sword-lifecycle}).  The binary partition holds for
every resource that \emph{remains} a sword: as long as both
conditions are satisfied, exactly one of (a)~or~(b) must obtain.

The pre-sword state is unstable because the causal envelope
$\mathrm{Reach}(x, \Ur)$ (\cref{eq:causal-envelope}) persists: the
agent retains capability, and any expansion of $\Obs$
re-detects the resource, returning it to the sword row and
re-activating the forcing cost of \cref{lem:forcing}.  Thus
the pre-sword defers rather than resolves the binary fate.
\end{remark}

\begin{remark}[Han Xin's error]\label{rem:hanxin}
The proverb ``when the cunning hare is killed, the hunting dog is
cooked'' (\emph{Shiji}, Huaiyin Hou) conflates three structurally
distinct resources: the \emph{bow} ($\Ur = \varnothing$, tool,
``stored'' not destroyed), the \emph{dog} ($\Ur \neq \varnothing$,
actuator, ``cooked''), and the \emph{advisor} (pure function, no
actuation---Zhang Liang survived). Han Xin quoted the answer but did
not parse its fine structure.
\end{remark}

\section{The fixed-point impossibility}\label{sec:fixedpoint}

\begin{theorem}[No path (c)]\label{thm:fixedpoint}
There is no strategy that ``proves your sword is not a sword'' while
retaining the sword. Formally, the map
$T: \Ur \mapsto \varnothing$ conditional on $\Ur \neq \varnothing$
has no fixed point other than $\Ur = \varnothing$.
\end{theorem}

\begin{proof}
Case~1: $\Ur = \varnothing$. Then $r$ is not a sword, and no proof
is needed. $\Ur = \varnothing$ is self-certifying.
Case~2: $\Ur \neq \varnothing$. Then no speech act can set
$\Ur \to \varnothing$---the criterion tests physical capability,
not narrative. The only way to satisfy $T(U_r) = \varnothing$ is to
physically relinquish $\Ur$, which is path~(a).

Moreover, the act of proving is itself a signal: ``I need to prove
my sword is not a sword'' implies suspicion, i.e., $r$ is already
in $\mathrm{Im}(\Obs)$. The proof attempt reinforces condition~(2).
\end{proof}

\section{The unconstrained power paradox}\label{sec:paradox}

\begin{theorem}[Perpetual elimination]\label{thm:paradox}
$U = \Umax$ implies the king must preemptively eliminate all
observable autonomous actuators:
\[
  U = \Umax \implies
  \text{the king must preempt all } r \text{ with }
  \Ur \neq \varnothing \;\wedge\; r \in \mathrm{Im}(\Obs).
\]
The more unconstrained the king, the more swords he must cut.
\end{theorem}

\begin{proof}
$\Umax$ means the king tolerates \emph{no} autonomous actuation:
every such actuator is a boundary threat on $\Viab(K)$. A
constrained system (constitutional regime) institutionalises sword
dynamics by installing breakpoints. An unconstrained system must
handle every sword individually. The paradox: unconstrained power is
not freedom---it is a perpetual elimination machine.
\end{proof}

\begin{proposition}[Imperfect observability accelerates the paradox]
\label{prop:imperfect}
If the detection function $\Obs$ is imperfect, the paradox
\emph{intensifies}, not weakens.
\end{proposition}

\begin{proof}
Three steps:
\begin{enumerate}[label=(\roman*)]
  \item The king knows $\Obs$ is imperfect. Hidden swords
  (\cref{def:sword}, condition~(2) unsatisfied) are more dangerous
  than visible ones. The king has motive, capability, and survival
  obligation to expand $\Obs$.
  \item Expanding $\Obs$ does not ``discover existing swords''---it
  \emph{creates new ones} definitionally. A hidden actuator
  satisfying condition~(1) but not~(2) enters $\mathrm{Im}(\Obs)$
  upon expansion $\to$ both conditions now satisfied $\to$ it
  \emph{becomes} a sword. The sword exists in the intersection
  $\Ur \neq \varnothing \;\wedge\; r \in \mathrm{Im}(\Obs)$;
  expanding $\Obs$ expands this intersection.
  \item Positive feedback:
  $U \to \Umax \implies \Obs \to \Obs_{\max}$.
  The expansion of $\Obs$ is the \emph{adjoint process} of the
  expansion of $U$. The elimination machine has two engines: the
  cutting arm ($U$) and the detecting eye ($\Obs$). They co-drive.
\end{enumerate}
Historical instances: Qin's mutual surveillance law
(\emph{lianzuo}), Han's gold-purity test (\emph{zhuo\-jin
duo\-jue}), Ming's three-layer nested monitoring (Jinyiwei
$\to$ Dongchang $\to$ Xichang---each layer itself becomes a new
sword).
\end{proof}

\begin{proposition}[$\Umax$ as attractor]\label{prop:attractor}
$\Umax$ is an attractor, not a state. No historical king achieves
literal $\Umax$, but the system dynamics point toward it:
\[
  \frac{d}{dt}\|U(t) - \Umax\| \leq 0
  \implies
  \frac{d}{dt}\bigl(\text{detected swords}\bigr) \geq 0.
\]
The paradox describes the trajectory, not the endpoint.
\end{proposition}

\section{The breakpoint criterion}\label{sec:breakpoint}

\begin{corollary}[Breakpoint strategy]\label{cor:breakpoint}
A resource $r$ is not a sword if and only if its execution chain
contains at least one node controlled by the king (a
\emph{breakpoint}):
\[
  r \text{ is not a sword}
  \iff
  \exists\; v \in \text{execution chain of } r
  \;\text{s.t.}\; v \text{ is controlled by the king.}
\]
\end{corollary}

\begin{proof}
If a breakpoint exists, $r$ cannot execute independently
(condition~(1) fails), so $r$ is not a sword. If no breakpoint
exists, the execution chain is closed and $r$ can actuate
autonomously, satisfying condition~(1). Combined with
observability, this makes $r$ a sword.
\end{proof}

\begin{remark}[Modern translation]\label{rem:modern}
Zhang Liang's strategy: ``ensure your capability always requires a
component you do not control.'' Liu Bang's strategy: ``become the
mandatory node in every execution chain.''
\end{remark}

\chapter{The Knife as Mean Field}\label{sec:meanfield}

The preceding sections defined the knife as a two-condition criterion
applied to individual resources. We now argue that the knife is
fundamentally a \emph{mean-field} phenomenon.

\section{The mean actuation field}

Consider $n$ agents with autonomous actuation levels
$\|U_1\|, \ldots, \|U_n\|$. Define the \emph{mean actuation field}:
\[
  \bar{U} = \frac{1}{n} \sum_{i=1}^{n} \|U_i\|.
\]

The king's detection function $\Obs$ has finite bandwidth: it cannot
monitor all agents with equal precision. In practice, $\Obs$ triggers
on agents whose actuation \emph{deviates significantly from the mean}:
\[
  r \in \mathrm{Im}(\Obs)
  \iff
  \|U_r\| - \bar{U} > \tau(\Obs),
\]
where $\tau(\Obs)$ is the detection threshold determined by the king's
observational capacity.

\begin{theorem}[The knife is the mean]\label{thm:meanfield}
The knife threshold is determined by the system's mean actuation field.
A resource $r$ is a knife if and only if:
\begin{enumerate}[label=(\roman*)]
  \item $\Ur \neq \varnothing$ (autonomous actuation exists), and
  \item $\|U_r\|$ exceeds the mean field by more than the detection
  threshold: $\|U_r\| > \bar{U} + \tau(\Obs)$.
\end{enumerate}
Consequently, the phase transition (\cref{prop:phase}) is a shift in
$\bar{U}$, not a change in any individual $\Ur$.
\end{theorem}

\begin{proof}
In wartime, many agents have high actuation (soldiers, generals,
administrators). The mean $\bar{U}$ is high, so the threshold
$\bar{U} + \tau(\Obs)$ is high: few agents exceed it. Most actuation
is \emph{expected} and therefore not flagged.

At the phase transition (end of war), most agents' actuation drops to
near zero (soldiers demobilize, wartime powers expire). The mean
$\bar{U}$ drops sharply. But agents who \emph{retain} wartime-level
actuation now exceed the new, lower threshold. The same $\|U_r\|$
that was below the wartime mean is now above the peacetime mean.

The knife is not created by the agent---it is created by the shift in
the mean. The agent's actuation is unchanged; the system's reference
frame has moved.
\end{proof}

\begin{remark}[Connection to statistical mechanics]\label{rem:statmech}
This is precisely the mechanism of a phase transition in statistical
mechanics: the order parameter (mean actuation) shifts, and
configurations that were typical in one phase become atypical---and
therefore detectable---in the other. The viability axiom plays the role
of the free energy: the system minimizes threats to $\Viab(K)$, just
as a thermodynamic system minimizes free energy.
\end{remark}

\section{Implications}

The mean-field interpretation resolves several puzzles:

\begin{enumerate}
  \item \textbf{Why identical resources have different fates.} Two
  generals with identical $\Ur$ can have opposite outcomes if one
  operates in a high-$\bar{U}$ environment (wartime coalition) and
  the other in a low-$\bar{U}$ environment (consolidated empire).
  The knife is relative to the mean.

  \item \textbf{Why the paradox is a feedback loop.} As the king
  eliminates knives, $\bar{U}$ drops, lowering the threshold. Agents
  who were below the old threshold now exceed the new one $\to$ new
  knives $\to$ more elimination $\to$ lower $\bar{U}$ $\to$ \ldots
  This is the positive feedback of \cref{thm:paradox}, now given a
  statistical mechanism.

  \item \textbf{Why self-blunting works.} Xiao He's strategy
  (self-corruption to signal low $\|U_r\|$) works precisely because
  the detection function triggers on \emph{deviation from the mean}.
  By visibly degrading his own actuation, Xiao He pulled $\|U_r\|$
  toward $\bar{U}$, falling below the detection threshold.

  \item \textbf{Why breakpoints prevent knives.} A breakpoint in the
  execution chain reduces $\|U_r\|$ (effective autonomous actuation)
  to below $\bar{U} + \tau(\Obs)$, since the king controls part of
  the chain. The resource remains capable but not \emph{independently}
  capable---it does not deviate from the mean.
\end{enumerate}

\chapter{Applications}\label{sec:applications}

\section{The Qin operating system}\label{sec:qin}

The Qin state (356--207~BCE) provides the first complete engineering
implementation of the framework. Shang Yang's reforms map directly to
the formal vocabulary:

\begin{center}
\begin{tabular}{@{}lp{5cm}p{5cm}@{}}
\toprule
\textbf{Policy} & \textbf{Framework equivalent} & \textbf{Effect} \\
\midrule
Abolish well-field system & Remove aristocratic $\Ur$ & Nobles
$\to$ commoners \\
Military merit ranks & $\Ur$ controlled by state (revocable) &
Actuation is rented, not owned \\
Mutual surveillance (\emph{lianzuo}) & Maximise $\Obs$ &
Neighbors $=$ distributed sensor network \\
Standardise weights \& measures & Increase $\Obs$ precision &
Higher observational resolution \\
Commandery-county system & All chains through capital &
King $=$ cut vertex \\
\bottomrule
\end{tabular}
\end{center}

Qin's unification of the six states was a graph-theoretic outcome:
Qin's star graph (center $=$ Xianyang, $O(1)$ dispatch) vs.\ the six
states' mesh graphs (multiple aristocratic centers, $O(n^2)$
coordination cost).

\begin{remark}[Temporal linearity]\label{rem:temporal}
The mesh-to-star compression is not an efficiency optimisation.  It is
a survival requirement forced by temporal linearity.

Time is real and linear: no pause, no speedup, no frame drops
(\cref{rem:deployment,rem:di-temporal}).  Every system that maintains
viability must complete a full sense--compute--act cycle within one
time step $\Delta t$:
\[
  \underbrace{|\phi| > 0}_{\text{do not fall}}
  \quad\wedge\quad
  \underbrace{t_{\pi} \leq \Delta t}_{\text{do not lag}}.
\]
The first condition is physical (the viability axiom).  The second is
computational: the response function must fit inside $\Delta t$.
Both must hold simultaneously.  Violating either kills the system
by different mechanisms: the first is a fall, the second is a lag,
and the outcome is the same.

This binds computational complexity to survival.  A mesh topology
dispatches in $O(n^2)$; a star topology in $O(1)$.  As the system
grows (more territory, more agents), $\Delta t$ does not grow with
it---time does not slow down for larger empires.  At critical $n$,
mesh dispatch exceeds $\Delta t$ and the system can no longer respond
in time.  商鞅's reform---mesh to star---is latency engineering:
compressing $t_\pi$ to fit within a fixed $\Delta t$ at any scale.

The isomorphism between robotic deployment and historical deployment
is exact:
\begin{center}
\renewcommand{\arraystretch}{1.25}
\begin{tabular}{@{}ll@{}}
\toprule
\textbf{Deployment (robotics)} & \textbf{Deployment (history)} \\
\midrule
policy $\pi$ (neural network) & emperor (cut vertex) \\
sensor reading $o_t$ & intelligence reports \\
torque command $\tau_t$ & edicts, military orders \\
$\Delta t = 20$\,ms & dynastic response window \\
ReLU (piecewise linear, $O(n)$) & star graph ($O(1)$ dispatch) \\
$|\phi| = 0$ (robot falls) & dynasty collapses \\
$t_\pi > \Delta t$ (compute lags) & cut vertex non-functional \\
\bottomrule
\end{tabular}
\end{center}

Every structural choice in the framework---ReLU as activation
function, star as topology, binary threshold as detection
criterion---is forced by the requirement that computation fit inside
real, linear, non-pausable time.  These are not design preferences;
they are the only structures fast enough to survive.

The two historical failure modes---the processor that runs an
irrelevant subroutine (latency explosion) and the processor that is
removed entirely (see \cref{sec:errorlog} for the full
analysis)---are the two ways the computational constraint breaks:
\begin{itemize}
\item 嘉靖: cut vertex present but $t_\pi \to \infty$.
  The emperor computes Daoist alchemy for twenty years while the
  system requires real-time response.  Every year without a routing
  decision is a dropped frame.  Gravity does not pause while the
  processor runs an irrelevant subroutine.
\item 元顺帝: cut vertex removed.  No processor.  Immediate crash
  ($|\phi| = 0$).
\end{itemize}
Temporal linearity is not a property of one remark in one chapter.
It is the medium in which the entire theory operates.
\end{remark}

Shang Yang's fate validates the framework's robustness: his reform
network itself became an autonomous actuator ($\Ur \neq \varnothing$).
The system eliminated its creator---not irony, but a robustness test.
A sword-detection system that exempts its designer is not robust
(\cref{sec:second} formalises this as second-mover viability).

\subsection{The submartingale-induced unitary group}

More precisely, Shang Yang's reform sequence is a
\emph{submartingale-induced unitary group}.

\begin{definition}[Submartingale reform]\label{def:submartingale}
A reform sequence $X_0, X_1, \ldots, X_n$ is a \emph{submartingale}
if each step strictly increases the centralisation index:
$\mathbb{E}[X_{k+1}] \geq X_k$ (deterministic analogue: the
sequence is non-decreasing in expectation under any re-ordering,
with the added constraint of irreversibility), and each step is
irreversible (reversal requires undoing all subsequent steps).
\end{definition}

Shang Yang's five reforms form such a sequence:
\[
  \underbrace{\text{abolish well-fields}}_{X_0}
  \to \underbrace{\text{merit ranks}}_{X_1}
  \to \underbrace{\text{mutual surveillance}}_{X_2}
  \to \underbrace{\text{standard measures}}_{X_3}
  \to \underbrace{\text{commandery-county}}_{X_4}.
\]
Each step burns the entropy of the old feudal order irreversibly.

Once installed, the system forms a \emph{unitary group}: a
structure-preserving transformation that maintains its invariants
(king $=$ cut vertex, all swords eliminated) at every cycle, applied
to every vector in the state space without exception.

\begin{theorem}[Shang Yang's paradox]\label{thm:shangyang}
Any agent who installs a sword-elimination system via submartingale
reform must possess $\Ur \neq \varnothing$ to execute the installation.
But the submartingale is monotone: each step lowers the detection
threshold. The installer's own actuation becomes increasingly visible
with each reform step. Upon completion, the unitary group acts on the
installer:
\[
  \underbrace{\text{installer drives submartingale}}_{\text{requires }
  \Ur \neq \varnothing}
  \;\longrightarrow\;
  \underbrace{\text{unitary group forms}}_{\text{detects all }
  \Ur \neq \varnothing}
  \;\longrightarrow\;
  \underbrace{\text{group acts on installer}}_{\text{elimination}}.
\]
The system has no creator exemption.
\end{theorem}

\begin{proof}
The installer is not in the invariant subspace of the group he
created (because $\Ur \neq \varnothing$). A unitary group does not
preserve non-invariant elements---it decomposes them. But unitary
means \emph{norm-preserving}: the installer is destroyed, but his
contributions are fully conserved. The reforms---commandery-county
system, merit ranks, mutual surveillance, standard measures---persist
uniformly across the empire. The person is eliminated; not one bit of
the contribution is lost. This is not destruction but
\emph{delocalisation}: a localised power entity is scattered across
the full state space by the unitary action. Norm conserved, structure
zeroed.
\end{proof}

\subsection{The existence proof}

Qin unified the six states, then collapsed (207~BCE, lasting only
15~years). But Qin left something more powerful than any army: an
\emph{existence proof}.

\begin{theorem}[Qin existence theorem]\label{thm:qin}
\[
  \exists\;\text{centralised state}\;\text{s.t.}\;
  \text{autonomous actuation} = 0
  \;\wedge\;
  \text{state functions.}
\]
\end{theorem}

Before Qin, no one knew a state without feudal aristocrats was
possible. Qin proved it was---and proved it functioned \emph{better}
(star-graph dispatch vs.\ mesh-graph coordination). You cannot refute
an existence proof. You can burn the paper, but the theorem persists.

After Qin fell, two men entered the ruins. Liu Bang, then a minor
official, had once seen the First Emperor's procession and sighed:
``A great man should be like this!''---\emph{I want to BE this system's
cut vertex}. Xiang Yu, upon conquering Xianyang, burned the palaces
and said: ``Who would be emperor in the dark where no one can
see?''---he wanted the \emph{trophy}, not the \emph{theorem}. One read
the existence proof. The other burned it. The one who read it founded
a dynasty that lasted four centuries.

\subsection{The wisdom of being second}\label{sec:second}

\Cref{thm:shangyang} has a direct corollary that inverts the usual
narrative of primacy.

\begin{corollary}[Second-mover viability]\label{cor:second}
Let $X_0, \ldots, X_n$ be a submartingale reform
(\cref{def:submartingale}) installed by agent~$\alpha$,
and let $\beta$ be an agent who inherits the resulting
infrastructure without participating in the installation.
Then $\beta$ has strictly higher viability than $\alpha$:
\[
  \Ur(\alpha) \neq \varnothing
  \;\;\text{(installation requires actuation)},\qquad
  \Ur(\beta) = \varnothing
  \;\;\text{(inheritance does not)}.
\]
The installed unitary group eliminates $\alpha$ and preserves $\beta$.
\end{corollary}

\begin{proof}
By \cref{thm:shangyang}, the installer $\alpha$ necessarily possesses
$\Ur \neq \varnothing$ and is detected upon completion.  The
inheritor~$\beta$ uses the infrastructure but did not create it:
$\Ur(\beta) = \varnothing$ with respect to the installation.  The
detection function tests $\Ur$, not provenance.  $\beta$ passes;
$\alpha$ does not.
\end{proof}

The existence proof (\cref{thm:qin}) has an installation cost, paid
exactly once by the first mover.  The cost is non-transferable and
non-refundable.  After installation the landscape is permanently
shifted; second movers operate in the post-shift regime without
needing to shift anything.

The first mover's contribution is \emph{entangled} with $\Ur$---they
needed $\Ur$ to install.  The second mover's usage is
\emph{factorised}---they use the infrastructure without the
installer's $\Ur$.  The second mover's path to recognition is smooth;
the first mover's is not.

\begin{center}
\renewcommand{\arraystretch}{1.25}
\begin{tabular}{@{}llp{3.5cm}p{4.5cm}@{}}
\toprule
\textbf{First} & \textbf{Second} & \textbf{Inherited} &
\textbf{Outcome} \\
\midrule
商鞅 & 秦's successors & commandery-county, merit ranks &
  商鞅 torn apart; Qin unifies \\
秦 & 汉 (刘邦) & existence proof & Qin 15\,yr; Han 400\,yr \\
项羽 & 刘邦 & Xianyang (capital) & 项羽 suicide; 刘邦 founds Han \\
王安石 & 蔡京 & fiscal infrastructure & 王安石 exiled; 蔡京 20\,yr
  chancellor \\
\bottomrule
\end{tabular}
\end{center}

范蠡 (5th century BCE) is the purest instance.  He helped 勾践 destroy
吴 but let the king remain the visible cut vertex.  Upon victory he
vanished---accumulated three fortunes and dispersed two, ensuring
$\Ur \approx \varnothing$ at every stage.  He was always second.
张良 followed the same strategy: after 刘邦's victory he retired to
``follow 赤松子,'' reducing $\Ur$ to $\varnothing$ before the
detection threshold reached him.  Both understood \cref{cor:second}
before it was stated: read the existence proof, do not write it; use
the infrastructure, do not install it.

\section{The dollar as sword precursor}\label{sec:dollar}

Money is not a sword (it cannot actuate independently). Money is a
\emph{sword precursor}: a universal voucher exchangeable for
$\Ur \neq \varnothing$ on the market. The king detects not how much
money you have, but whether your money has \emph{already converted}
into autonomous execution capability.

Fan Li (5th century BCE) understood this: he accumulated fortunes
three times and dispersed them twice, ensuring $\Ur$ never crossed
the threshold. Shen Wansan and Hu Xueyan did not---their wealth
formed closed execution loops, and they were eliminated.

\section{The Atlantic triangular trade: a sword with no cut
vertex}\label{sec:triangle}

The Atlantic Triangular Trade (16th--19th century) is a three-leg
closed execution loop: manufactured goods (Europe $\to$ West Africa),
enslaved human beings (West Africa $\to$ Americas), raw materials
(Americas $\to$ Europe). This loop satisfies \cref{def:sword}: it
actuates autonomously (multiple nations operate independent instances;
remove any one and the loop continues) and is observable (300~years of
ships, ledgers, and auction blocks).

The loop has a property stronger than condition~(1): it is
\emph{self-financing}. The raw materials extracted on leg~3 fund the
goods shipped on leg~1, which purchase the enslaved labour on leg~2,
who produce the raw materials on leg~3. A sword that funds its own
actuation cannot be starved from outside.

\paragraph{Where the isomorphism breaks.}
In \cref{sec:water}, the king--pawn relationship is sustained by water
(viability bargain: $\text{water} > 0$). In the Triangular Trade,
enslaved people had $\text{water} = 0$ from day one. Force~$F$
substituted for water---temporarily. The substitution cost: every unit
of force burns resources drawn from the extracted water on leg~3. The
loop becomes a self-financing violence machine. \Cref{sec:water}
predicts $\text{water} = 0 \implies \text{pawn} \to \text{sword}
\implies \text{collapse}$; force delays but does not prevent this.
The longer the delay, the more violent the eventual resolution.

\paragraph{Abolition as relabelling.}
Applying \cref{prop:phase}: did abolition (1807--1888) change the
\emph{physics} or only the \emph{label}? The labour force moved from
slavery to sharecropping on the same plantations, the resource flow
(raw materials $\to$ Europe) persisted, the force mechanism shifted
from slave codes to Jim Crow and convict leasing, and the laborer
still had no breakpoint. Verdict: phase transition in the sense of
\cref{sec:phase}---label changed, topology unchanged. This is
path~(c), which \cref{thm:fixedpoint} proves is not a resolution.

\begin{theorem}[Closed-loop sword]\label{thm:triangle}
Any closed execution loop that is self-financing, has no cut vertex
(mesh topology, multiple independent operators), and substitutes
force for water, has exactly two exit paths: physical dismantlement
of the loop topology [path~(a)], or system-level collapse when
force-substitution fails [path~(b)]. Relabeling [path~(c)] preserves
$\Ur \neq \varnothing$ and resolves nothing.
\end{theorem}

\paragraph{Missing cut vertex.}
The essay's framework (\cref{sec:cutvertex}) assumes a single cut
vertex. The Triangular Trade has none: its topology is a mesh, not a
star. Removing any one colonial power does not disconnect the loop.
This makes path~(a) structurally harder---there is no single point to
dismantle---but does not create a third option.

\begin{remark}[Path~(a) by mean-field shift]\label{rem:chaplin}
Chaplin's 1925 song \emph{With You, Dear, in Bombay}---composed during
\emph{The Gold Rush}, a film about extractive economics---takes the same
maritime route and replaces the extractive payload with a romantic one.
The singer sails to Bombay not to trade but to reunite. This is not
path~(c) (the route is not relabeled; the self-financing loop is broken
because love does not fund the next leg). It is path~(a) by mean-field
shift: the infrastructure persists, but the mean changes, and therefore
the sword changes.
\end{remark}

\section{The corrupted detection function}\label{sec:nazi}

\Cref{def:sword} uses a structural detection function
$\Obs_{\mathrm{structural}}$ that tests $\Ur$: \emph{what can the
resource do independently?} This function is identity-blind. Zhang
Liang is safe because $\Ur = \varnothing$, not because he is Zhang
Liang.

\begin{definition}[Detection function corruption]\label{def:corruption}
A detection function is \emph{corrupted} when $\Obs_{\mathrm{structural}}$
(tests $\Ur$) is replaced by $\Obs_{\mathrm{identity}}$ (tests group
membership):
\[
  \Obs_{\mathrm{identity}}:\;
  r \;\mapsto\;
  \begin{cases}
    \textbf{sword} & \text{if } \mathrm{agent}(r) \in \text{Group } X, \\
    \textbf{not sword} & \text{otherwise.}
  \end{cases}
\]
\end{definition}

The corruption produces a \emph{false positive catastrophe}: every
member of Group~$X$ with $\Ur = \varnothing$ is classified as a sword
and eliminated. Every agent outside Group~$X$ with $\Ur \neq \varnothing$
is a false negative. Six million false positives is the output of a
corrupted detection function running to completion.

\begin{definition}[Nazi structure]\label{def:nazi}
A system exhibits \emph{Nazi structure} if and only if: (1)~it performs
viability maintenance (eliminates perceived threats to $\Viab(K)$),
(2)~its detection criterion is identity-based, not structure-based,
and (3)~it executes path~(b) against agents classified by identity
who do not satisfy \cref{def:sword} ($\Ur = \varnothing$ for the
individual). These conditions are necessary and sufficient.
\end{definition}

\begin{corollary}[Identity-invariance]\label{cor:nazi}
The Nazi structure is identity-invariant on both sides: it depends
neither on the identity of the perpetrator nor on the identity of the
target. Replacing Group~$X$ with any other identity group preserves
all three conditions.
\end{corollary}

The operator that maps $\Obs_{\mathrm{structural}} \mapsto
\Obs_{\mathrm{identity}}$ is \emph{Hate}. In this framework, Hate is
not an emotion---it is an operator on detection functions with a
precise signature (structural $\to$ identity) and a precise output
(mass false positives). The motivation is irrelevant; the mapping
determines the output.

\section{The parasitic cut vertex}\label{sec:parasite}

\Cref{thm:cutvertex} establishes the cut vertex as the optimal
survival strategy: route all execution chains through yourself.
A \emph{parasitic cut vertex} has the same topology but opposite flow
direction: it routes access and information for extraction rather than
coordination.

\begin{definition}[Parasitic cut vertex]\label{def:parasite}
A parasitic cut vertex $v_p$ satisfies:
(1)~$G \setminus \{v_p\}$ is disconnected (cut vertex),
(2)~$v_p$ holds no formal authority ($\Ur \approx \varnothing$
officially),
(3)~$v_p$ extracts resources from both sides of the partition by
routing access across the cut.
\end{definition}

The parasitic cut vertex turns \emph{others} into pawns via the
routing function (e.g., blackmail: ``I route your secret; comply or I
re-route it publicly''). This is \cref{thm:cutvertex} inverted:
instead of building the system, the parasite extracts from it.

\begin{proposition}[Self-termination]\label{prop:parasite}
A parasitic cut vertex is self-terminating. When the secrecy that
maintains the cut ($\text{water} = \text{secrecy}$) fails, the graph
reconnects, the cut vertex property vanishes, and the former parasite
faces the combined action of all previously partitioned nodes.
\end{proposition}

Unlike the productive cut vertex (which may persist for centuries via
institutional embedding), the parasitic variant stores no structural
contribution. Upon delocalisation, norm is conserved but there is
nothing to conserve: the extraction leaves no invariant.

\section{Militarism and the net-positive ask}\label{sec:militarism}

Militarism is the structural type where the pawn ($\Ur = \varnothing$,
the military apparatus) captures the cut vertex position, producing
$\Ur \neq \varnothing$ for the military and $\Ur \to \varnothing$ for
civilian institutions. In the framework's language: the pawn becomes the king.

\begin{example}[The Kniefall: path~(a) in 30 seconds]\label{ex:kniefall}
On December~7, 1970, West German Chancellor Willy Brandt knelt before
the Warsaw Ghetto Uprising memorial. This act satisfies four conditions
that make it a pure instance of path~(a):
\emph{physical} (a bodily act, not a speech act---cannot be retracted
by reinterpretation),
\emph{unconditional} (no negotiation, no demand for reciprocity),
\emph{performed by the cut vertex} (the head of state, the node
through which all institutional chains pass), and
\emph{irretractable} (a photograph is a permanent record).
Brandt's Kniefall is an existence proof that path~(a) is available to
any system.
\end{example}

After the Kniefall, Germany installed institutional breakpoints:
Article~1 of the \emph{Grundgesetz} (``Human dignity is inviolable''),
the Federal Constitutional Court as an independent $\Obs$, EU and NATO
membership as external breakpoints. These institutionalise the phase
transition: the system moved from $U = \Umax$ (Nazi regime) through
path~(a) (Kniefall) to a constrained system with structural breakpoints.

\begin{proposition}[Net-positive theorem]\label{prop:netpositive}
Path~(a) is strictly net-positive for all parties. Comparative evidence:
Germany (path~(a), Kniefall 1970) vs.\ Japan (path~(c), relabelling
without structural change, 80~years and counting). Every measurable
outcome---diplomatic relations, regional stability, economic
integration, soft power, domestic constitutional health---favors the
path~(a) system. The cost of path~(a) is pride. The return is
everything else.
\end{proposition}

\chapter{Discussion}\label{sec:discussion}

\section{Domain of applicability}\label{sec:domain}

The framework is a compressed model that degrades under the following
conditions:

\begin{center}
\begin{tabular}{@{}lp{5.5cm}p{5cm}@{}}
\toprule
\textbf{Condition} & \textbf{Failure mode} & \textbf{Example} \\
\midrule
Diffuse sovereignty & No unique king; cut vertex undefined & Late
European feudalism, early federalism \\
External shock dominance & $\Viab(K)$ broken by external force;
internal knife dynamics secondary & Mongol invasion, colonialism \\
Rapid ideological reshaping & $K$ itself is changing faster than
actuation reshapes state & Religious revolution, ideology \\
High-latency detection & $\Obs$ too slow; phase transition completes
before detection & Large empire frontiers \\
Universal $\Ur \approx \varnothing$ & No knives; framework trivial &
Extremely atomised societies \\
\bottomrule
\end{tabular}
\end{center}

The framework applies to systems with a unique sovereign, differentiated
agent capabilities, and sufficient observability. This covers the main
interval of Chinese imperial history but not all political forms.

\section{The water dynamics}\label{sec:water}

The framework so far analyzes one layer: the king--knife interaction. A
complete viability analysis requires the \emph{viability chain}---the
three-level dependency that sustains the system:
\[
  \text{King}
  \xrightarrow{\;\text{needs}\;}
  \text{Pawn}
  \xrightarrow{\;\text{needs}\;}
  \text{Water.}
\]
The \emph{pawn} is any agent with $\Ur \neq \varnothing$ whose actuation
is currently directed by the king (soldiers, administrators, tax
collectors). \emph{Water} is the population's aggregate resource
level---the substrate from which the pawn draws manpower, revenue, and
legitimacy.

\begin{definition}[Water]\label{def:water}
Water $w(t) \in [0, W_{\max}]$ is the population's aggregate extractable
resource level at time $t$. The pawn's actuation is bounded by water:
$\|\Ur\| \leq g(w)$ for some monotone function $g$ with $g(0) = 0$.
\end{definition}

The king needs the pawn to execute viability maintenance (eliminate
knives, administer territory). The pawn needs water to function. If
$w \to 0$, the pawn's actuation capacity collapses regardless of the
king's commands.

\begin{proposition}[Binary action space at $w = 0$]\label{prop:binary}
When $w(t) \to 0$, the pawn's action space collapses to a binary:
\[
  A_{\text{pawn}} = \{\text{submit},\; \text{rebel}\}.
\]
The intermediate options (negotiate, migrate, trade, accumulate) require
$w > 0$. At $w = 0$, the pawn has nothing to lose, and the viability
axiom (\cref{ax:viability}) now applies \emph{to the pawn}: the pawn's
own survival requires a viable path, and submission no longer provides
one. The pawn becomes a knife---$\Ur$ transitions from $\varnothing$ to
$\neq \varnothing$---and the king faces a knife he created by exhausting
the water.
\end{proposition}

\begin{theorem}[Du Mu's theorem]\label{thm:dumu}
Let $w(t)$ be decreasing under extraction. Then:
\[
  w(t) \to 0
  \implies
  \text{pawn} \to \text{knife}
  \implies
  \text{king absorbed.}
\]
The causal chain is internal: the system destroys itself by exhausting
its own substrate.
\end{theorem}

This is the content of Du Mu's \emph{A Fang Gong Fu} \cite{dumu} (825~CE): ``It was
not the Qin who destroyed the six states, but the six states themselves;
it was not the world that destroyed Qin, but Qin itself.'' In our
language: the states depleted their own water, creating the knives that
destroyed them. Qin, having unified, then depleted its own water
(conscription for the Great Wall and Epang Palace), creating the knives
(Chen Sheng, Wu Guang) that destroyed it.

\begin{remark}[Water as viability constraint]
``Water can carry the boat, and water can capsize the boat'' (attributed
to Wei Zheng, Tang dynasty) is not a metaphor. It is a restatement of
\cref{thm:dumu}: the substrate that enables the king's viability
($w > 0$ $\implies$ pawn functions $\implies$ king's path to $\infty$
exists) is the same substrate whose depletion destroys it ($w = 0$
$\implies$ pawn $\to$ knife $\implies$ no path to $\infty$).
\end{remark}

\chapter{Agentic Calculus}\label{sec:calculus}

The preceding sections established the viability axiom, the knife
criterion, and the mean-field interpretation. We now construct an
\emph{operational calculus}---a language for writing algorithms on the
agentic space---that translates every theorem into a flow-theoretic
proposition and yields a complete training paradigm for neural networks.
The central result: the knife is the min-cut, the viable path is the
max-flow, and ``the knife is the mean'' is max-flow/min-cut duality.

\section{The agentic space}\label{sec:tower}

The framework's objects organize into a four-level tower, each level
derived from the axioms of the preceding sections.

\begin{definition}[Agentic space]\label{def:tower}
The \emph{agentic space} is the tower
$\mathbf{L} = (L_0, L_1, L_2, L_3)$:
\begin{enumerate}[label=\textbf{L\arabic*}., ref=L\arabic*]
  \item\label{L0} \textbf{State space} $S$.
  Every configuration of the system is a point in $S$.
  \item\label{L1} \textbf{Viable kernel} $\Viab(K) \subset S$.
  The compact set of states from which the king retains a path to
  infinity (\cref{ax:viability}).
  \item\label{L2} \textbf{Control bundle} $\{U(s)\}_{s \in \Viab(K)}$.
  At each viable state $s$, the fiber $U(s)$ is the set of controls
  that keep the next state inside $\Viab(K)$.
  \item\label{L3} \textbf{Strategy space} $\Gamma$.
  A \emph{strategy} $\gamma \in \Gamma$ is a viable path
  $\gamma: [0,\infty) \to \Viab(K)$ with $\gamma(t+1) \in
  f(\gamma(t), u)$ for some $u \in U(\gamma(t))$ at each step.
\end{enumerate}
\end{definition}

The tower is strict: each level presupposes the one below.
$L_1 \subset L_0$ by definition. $L_2$ exists only over $L_1$
(outside $\Viab(K)$, no control preserves viability). $L_3$ is
built from $L_2$ fibers concatenated over time. The viability axiom
(\cref{ax:viability}) asserts $\Gamma \neq \varnothing$: the strategy
space is non-empty.

\section{The flow}\label{sec:flow}

The agentic calculus is a \emph{flow calculus}. We define flows on
the execution graph and show that every theorem in
\cref{sec:results,sec:meanfield} is a statement about flows and cuts.

\begin{definition}[Execution graph]\label{def:exgraph}
The \emph{execution graph} $G = (V, E, c)$ has:
\begin{itemize}
  \item $V$: agents $\{a_1, \ldots, a_n\}$ plus two distinguished
  nodes: the king $\kappa$ and infinity $\infty$;
  \item $E$: directed edges $(a_i, a_j)$ whenever $a_i$'s actuation
  can affect $a_j$'s state;
  \item $c: E \to \R_{\geq 0}$: edge capacity, where $c(a_i, a_j)$
  is the autonomous actuation that $a_i$ can transmit to $a_j$
  without requiring the king's authorization.
\end{itemize}
An edge $(a_i, a_j)$ with $c(a_i, a_j) > 0$ that does not pass
through $\kappa$ is a \emph{bypass edge}.
\end{definition}

\begin{definition}[Agentic flow]\label{def:flow}
An \emph{agentic flow} is a function $\phi: E \to \R_{\geq 0}$
satisfying:
\begin{enumerate}[label=(\roman*)]
  \item \textbf{Capacity}: $\phi(e) \leq c(e)$ for all $e \in E$.
  \item \textbf{Conservation}: at every non-terminal node $v \neq
  \kappa, \infty$,
  \[
    \sum_{(u,v) \in E} \phi(u,v)
    = \sum_{(v,w) \in E} \phi(v,w).
  \]
\end{enumerate}
The \emph{value} $|\phi|$ is the net flow from $\kappa$ to $\infty$.
A \emph{viable flow} is one with $|\phi| > 0$: the king has a
path to infinity with positive throughput.
\end{definition}

The viability axiom (\cref{ax:viability}) is flow conservation:
what enters the system at $\kappa$ must exit at $\infty$.

\paragraph{Four operations.}
The calculus has four primitive operations on the execution graph:

\begin{center}
\begin{tabular}{@{}llll@{}}
\toprule
\textbf{Operation} & \textbf{Symbol} & \textbf{On $G$} &
\textbf{In 华容道} \\
\midrule
\textsc{Slide} & $\sigma$ & Unit flow along one edge &
One piece moves one cell \\
\textsc{Compose} & $\circ$ & Concatenate along a path &
Sequence of moves \\
\textsc{Cut} & $\partial$ & Remove capacity from an edge set &
Block a corridor \\
\textsc{Phase} & $\varphi$ & Change the capacity function
$c \mapsto c'$ & Phase transition \\
\bottomrule
\end{tabular}
\end{center}

\textsc{Slide} is atomic (unit flow).
\textsc{Compose} builds paths from slides.
\textsc{Cut} is the knife: removing capacity from bypass edges.
\textsc{Phase} is the phase transition: the mean field shifts,
capacities change, the same graph has different flows.

\begin{theorem}[Flow-cut duality]\label{thm:flowcut}
In the execution graph $G$, the maximum viable flow from $\kappa$
to $\infty$ equals the minimum knife-cut capacity:
\[
  \max_\phi |\phi|
  \;=\;
  \min_{C \,\subseteq\, E} \sum_{e \in C} c(e)
  \quad\text{over all $\kappa$-$\infty$ cuts $C$.}
\]
The knife threshold (\cref{thm:meanfield}) is the min-cut value.
The viable path (\cref{ax:viability}) is the max-flow.
``The knife is the mean'' $=$ max-flow equals min-cut.
\end{theorem}

\begin{proof}
By the max-flow/min-cut theorem~\cite{diestel}, the maximum flow from
$\kappa$ to $\infty$ equals the minimum capacity of any
$\kappa$--$\infty$ cut. The knife criterion (\cref{def:knife})
identifies bypass edges---edges with positive capacity that do not pass
through $\kappa$. The king's viability maintenance (cutting knives) is
the operation $c(e) \to 0$ for bypass edges $e$. The residual max-flow
after all bypass edges are cut is the flow through the king (the cut
vertex flow). The min-cut value $=$ the total bypass capacity $=$ the
knife threshold $=$ the mean field's deviation measure.
\end{proof}

\begin{remark}[Flow interpretation of theorems]\label{rem:flowthms}
Each main theorem translates directly:
\begin{itemize}
  \item \textbf{Binary fate} (\cref{thm:lifecycle}): a bypass edge
  either has its capacity set to zero by the holder (path~(a)) or by
  the king (path~(b)). No bypass edge persists with $c > 0$.
  \item \textbf{Fixed-point impossibility} (\cref{thm:fixedpoint}):
  a bypass edge with $c > 0$ cannot ``prove'' $c = 0$. Capacity is
  physical, not narrative.
  \item \textbf{Perpetual elimination} (\cref{thm:paradox}):
  $U = \Umax$ means zero bypass tolerance. As \textsc{Cut} operates,
  \textsc{Phase} lowers the mean, exposing new bypass edges.
  \item \textbf{Du Mu's theorem} (\cref{thm:dumu}): water $=$ total
  network capacity. $w \to 0$ means all capacities shrink to zero:
  frozen, $\Gamma = \varnothing$.
\end{itemize}
\end{remark}

\begin{remark}[抽刀断水水更流]\label{rem:libai}
Li Bai's line assigns the calculus its colours:
\[
  \textcolor{knife}{\text{抽刀}} \;\;
  \textcolor{knife}{\partial} \;\;
  \textcolor{water}{\text{水}} \;\;
  \textcolor{water}{\text{水}}\textcolor{sword}{\text{更流.}}
\]
\textsc{Cut} ($\textcolor{knife}{\partial}$, red) acts on flow
($\textcolor{water}{\sigma}$, blue); flow intensifies. The mechanism
is \textsc{Phase} ($\textcolor{sword}{\varphi}$, cyan): cutting
shifts the mean field (\cref{thm:meanfield}), exposing new bypass
edges, producing more flow---\cref{thm:paradox} in seven characters.
The sword is 青冥 ($\textcolor{sword}{\text{青}}$): the colour of the
mean, the colour of Phase, the colour that connects
$\textcolor{knife}{\text{刀}}$ to $\textcolor{water}{\text{水}}$.
\end{remark}

\section{方圆 $\times$ 黑白: the type system}\label{sec:fangyuan}

The calculus has a type system: a $2 \times 2$ classification that
partitions every element of the agentic space.

\begin{definition}[方圆 $\times$ 黑白]\label{def:fangyuan}
The agentic type system is the product of two binary distinctions:
\begin{center}
\begin{tabular}{@{}lcc@{}}
\toprule
& \textbf{方} (container / structure) &
\textbf{圆} (content / agent) \\
\midrule
\textbf{黑} (constrained / interior) &
Fixed topology (board, graph) &
King $\kappa$ (least mobile, most important) \\
\textbf{白} (free / exterior) &
Free capacity (available edges) &
Pawn (most mobile, least important) \\
\bottomrule
\end{tabular}
\end{center}
The two dynamics of the calculus emerge from this classification:
\begin{itemize}
  \item \textbf{刀} (knife $= \partial$, boundary operator):
  the boundary between 黑 and 白. \textsc{Cut} reclassifies an edge
  from 白 (free capacity) to 黑 (zero capacity).
  \item \textbf{水} (water $= \sigma$, transport operator):
  flow through 白 cells. \textsc{Slide} transports one unit of flow
  along a free edge. Water flows where the knife does not cut.
\end{itemize}
\end{definition}

In 华容道 (\cref{sec:huarongdao}): 方 $=$ the board,
圆 $=$ the pieces. 黑 $=$ occupied cells and the king,
白 $=$ free cells and soldiers. 刀 $=$ 关羽 blocking the corridor.
水 $=$ free-cell flow (slides opposite to piece movement).
The $2 \times 2$ is the type system of the puzzle's state space.

\section{Completeness}\label{sec:completeness}

Every theorem in this paper is a proposition in the agentic calculus.

\begin{proposition}[Calculus completeness]\label{prop:completeness}
The following table maps each theorem to its calculus translation:
\begin{center}
\begin{tabular}{@{}lll@{}}
\toprule
\textbf{Theorem} & \textbf{Calculus statement} &
\textbf{Operations} \\
\midrule
Viability (\ref{ax:viability}) & $|\phi| > 0$ &
$\sigma, \circ$ \\
Binary fate (\ref{thm:lifecycle}) &
$\forall$ bypass $e$: $c(e) \to 0$ &
$\partial$ \\
Fixed point (\ref{thm:fixedpoint}) &
$c(e) > 0 \not\vdash c(e) = 0$ &
--- \\
Paradox (\ref{thm:paradox}) &
$\partial$ generates new bypass via $\varphi$ &
$\partial, \varphi$ \\
Mean field (\ref{thm:meanfield}) &
Min-cut $= \bar{U} + \tau(\Obs)$ &
$\partial$ \\
Cut vertex (\ref{thm:cutvertex}) &
$\kappa =$ min vertex-cut &
structure \\
Du Mu (\ref{thm:dumu}) &
$w \to 0 \Rightarrow c \to 0 \Rightarrow |\phi| = 0$ &
$\sigma \to 0$ \\
Flow-cut (\ref{thm:flowcut}) &
$\max |\phi| = \min |C|$ &
$\sigma, \partial$ \\
\bottomrule
\end{tabular}
\end{center}
\end{proposition}

The calculus is \emph{complete}: no theorem falls outside its four
operations. The agentic space (\cref{def:tower}) provides the domain;
the flow (\cref{def:flow}) provides the dynamics; the type system
(\cref{def:fangyuan}) provides the classification; and flow-cut duality
(\cref{thm:flowcut}) provides the central identity.

\section{The training paradigm}\label{sec:training}

The agentic calculus instantiates as a neural network training paradigm.
The execution graph \emph{is} the computation graph. Training \emph{is}
max-flow optimisation. Survival \emph{is} the viability axiom. The
paradigm strictly subsumes gradient descent.

\begin{definition}[Neural execution graph]\label{def:neural-exgraph}
Let a feedforward network with $L$ layers be given.
Define the execution graph $G = (V, E, c)$ (\cref{def:exgraph}) by:
\begin{itemize}
  \item $V = \{\ell_0, \ell_1, \ldots, \ell_L\}$, one node per layer;
  \item $E = \{(\ell_{i-1}, \ell_i) : 1 \leq i \leq L\}$, one edge
  per weight matrix $W_i$;
  \item $c(e_i) = \|W_i\|_F$, the Frobenius norm as capacity;
  \item king $\kappa = \ell_0$ (input); target
  $\infty = \ell_L$ (output).
\end{itemize}
A data point $(x, y^*)$ initiates flow at $\kappa$ with value $\|x\|$.
Training finds capacities $\{c(e_i)\}$ such that the max-flow matches
the target at $\infty$.
\end{definition}

\paragraph{Operation correspondence.}

\begin{center}
\begin{tabular}{@{}llll@{}}
\toprule
\textbf{Operation} & \textbf{Neural network} &
\textbf{Equation} \\
\midrule
\textsc{Slide} $\sigma$ & One-layer forward pass &
$y = W_e\, x$ \\
\textsc{Compose} $\circ$ & Full forward pass &
$z = \sigma_L \circ W_L \circ \cdots \circ \sigma_1 \circ W_1\, x$ \\
\textsc{Cut} $\partial$ & Pruning / dropout &
$W_e \mapsto 0$, i.e.\ $c(e) \mapsto 0$ \\
\textsc{Phase} $\varphi$ & Regime change &
lr schedule, fine-tuning, curriculum \\
\bottomrule
\end{tabular}
\end{center}

\paragraph{Type-system correspondence.}
Under \cref{def:fangyuan}:
方 $=$ architecture (fixed graph);
圆 $=$ activations (flow $\phi$ traversing the graph);
黒 $=$ frozen weights;
白 $=$ trainable weights;
刀 $=$ pruning operator $\partial$;
水 $=$ data flow forward \emph{and} gradient flow backward.
The backward pass is water flowing opposite to the forward
pass---the free-cell mechanism of \cref{sec:huarongdao}: to move a
piece forward, a free cell slides back.

\begin{definition}[Training algorithm]\label{def:training-algo}
Given $G$ from \cref{def:neural-exgraph} and a dataset $\mathcal{D}$,
the \emph{training paradigm} is the procedure in \cref{fig:training}:
\begin{enumerate}
  \item \textbf{Initialise.} Random $W_i^{(0)}$; set
  $c(e_i) = \|W_i^{(0)}\|_F$.
  \item \textbf{\textsc{Compose}.} Forward pass: $L$ sequential
  \textsc{Slide}s produce
  $z = \sigma_L \circ W_L \circ \cdots \circ \sigma_1 \circ W_1\, x$.
  \item \textbf{Flow deficit.} Loss
  $\mathcal{L} = -|\phi|$.
  \item \textbf{Backward \textsc{Slide}.} Compute
  $\partial\mathcal{L}/\partial c(e_i)$: 水 flowing opposite to
  step~2.
  \item \textbf{Capacity update.} SGD:
  $c(e_i) \leftarrow c(e_i) - \eta\,
  \partial\mathcal{L}/\partial c(e_i)$.
  \item \textbf{Knife detection.} Flag bypass edges where
  $c(e) > \bar{c} + \tau$ (\cref{thm:meanfield}).
  \item \textbf{\textsc{Cut} / \textsc{Phase}.} Prune flagged edges
  ($L_1$ penalty) or change regime (lr, dataset, fine-tuning).
  \item \textbf{Viability check.} Verify $|\phi| > 0$ on held-out
  data (\cref{ax:viability}). If violated: \textsc{Phase} or restart.
  \item \textbf{Repeat} 2--8 until $\max|\phi| = \min|C|$
  (\cref{thm:flowcut}).
\end{enumerate}
\end{definition}

\begin{figure}[H]
\centering
\begin{tikzpicture}[
  node distance=0.9cm and 1.8cm,
  % ── water (水) nodes: blue ──
  wtr/.style={rectangle, rounded corners=3pt, draw=water, thick,
    fill=water!6, minimum width=5.0cm, minimum height=0.7cm,
    align=center, font=\small},
  % ── knife (刀) node: red ──
  knf/.style={diamond, draw=knife, thick, aspect=2.5,
    fill=knife!6, minimum width=1.2cm, align=center, font=\small,
    inner sep=1pt},
  % ── phase (青冥) nodes: cyan ──
  phs/.style={rectangle, rounded corners=3pt, draw=sword, thick,
    fill=sword!6, minimum width=5.0cm, minimum height=0.7cm,
    align=center, font=\small},
  phsd/.style={diamond, draw=sword, thick, aspect=2.5,
    fill=sword!6, minimum width=1.2cm, align=center, font=\small,
    inner sep=1pt},
  % ── neutral (terminal) ──
  term/.style={rectangle, rounded corners=8pt, draw, very thick,
    minimum width=5.0cm, minimum height=0.7cm, align=center,
    font=\small\bfseries},
  % ── convergence diamond ──
  convd/.style={diamond, draw, thick, aspect=2.5,
    minimum width=1.2cm, align=center, font=\small,
    inner sep=1pt},
  arr/.style={-{Stealth[length=5pt]}, thick},
  lbl/.style={font=\scriptsize, fill=white, inner sep=1pt},
  ref/.style={font=\tiny, text=black!55, anchor=west},
  op/.style={font=\scriptsize\itshape}
]

% ── Nodes ──
\node[term] (init)
  {1.\ Initialise: random $c(e)$};

\node[wtr, below=of init] (fwd)
  {2.\ \textsc{Compose}: $\textcolor{water}{\sigma_L \circ
  \cdots \circ \sigma_1}$};

\node[wtr, below=of fwd] (loss)
  {3.\ Flow deficit: $\textcolor{water}{\mathcal{L} = -|\phi|}$};

\node[wtr, below=of loss] (bwd)
  {4.\ Backward \textsc{Slide}:
  $\textcolor{water}{\nabla_c \mathcal{L}}$};

\node[wtr, below=of bwd] (sgd)
  {5.\ Update:
  $\textcolor{water}{c \leftarrow c - \eta\,\nabla_c\mathcal{L}}$};

\node[knf, below=1.1cm of sgd] (knife)
  {6.\ $\textcolor{knife}{c(e) > \bar{c}{+}\tau}$\,?};

\node[phs, right=of knife] (cut)
  {7.\ \textcolor{knife}{\textsc{Cut} $\partial$} /
  \textcolor{sword}{\textsc{Phase} $\varphi$}};

\node[phsd, below=1.1cm of knife] (viable)
  {8.\ $\textcolor{sword}{|\phi| > 0}$\,?};

\node[convd, below=1.1cm of viable] (conv)
  {9.\ $\max|\phi| = \min|C|$\,?};

\node[term, below=1.0cm of conv] (done)
  {Trained network. Survives.};

% ── Reference annotations ──
% init: right (no loop passes here)
\node[ref] at ($(init.east)+(0.15,0)$)
  {Def.~\ref{def:neural-exgraph}};
% steps 2--5: LEFT side (right side reserved for loop-back arrow)
\node[ref, anchor=east] at ($(fwd.west)+(-0.15,0)$)
  {Def.~\ref{def:flow},\; $\sigma$};
\node[ref, anchor=east] at ($(loss.west)+(-0.15,0)$)
  {Thm~\ref{thm:flowcut},\; $|\phi|$};
\node[ref, anchor=east] at ($(bwd.west)+(-0.15,0)$)
  {Def.~\ref{def:flow},\; $\sigma^{-1}$};
% knife: above-right (clear of spine)
\node[ref] at ($(knife.east)+(1.0,0.35)$)
  {Thm~\ref{thm:meanfield}};
% cut/phase: below
\node[ref] at ($(cut.south)+(0,-0.12)$)
  {Thm~\ref{thm:lifecycle}\;($\partial$),\;
   Thm~\ref{thm:paradox}\;($\varphi$)};
% viable: above-right
\node[ref] at ($(viable.east)+(1.0,0.35)$)
  {Ax.~\ref{ax:viability}};
% conv: LEFT side (right side reserved for loop-back arrow)
\node[ref, anchor=east] at ($(conv.west)+(-0.15,0)$)
  {Thm~\ref{thm:flowcut}};

% ── Operation annotations (left margin, coloured) ──
\node[op, text=water, left=1.8cm of fwd] {$\circ$};
\node[op, text=water, left=1.8cm of bwd]
  {$\sigma^{-1}$ (\textcolor{water}{水})};
\node[op, text=sword, right=0.1cm of cut.east] {};

% ── Arrows: main spine ──
\draw[arr] (init) -- (fwd);
\draw[arr, water] (fwd) -- (loss);
\draw[arr] (loss) -- (bwd);
\draw[arr, water] (bwd) -- (sgd);
\draw[arr] (sgd) -- (knife);
\draw[arr] (knife) -- node[lbl, right] {no} (viable);
\draw[arr, sword] (viable) -- node[lbl, right] {yes} (conv);
\draw[arr] (conv) -- node[lbl, right] {yes} (done);

% ── Arrows: branches ──
\draw[arr, knife] (knife) -- node[lbl, above] {yes} (cut);
\draw[arr, sword] (cut) |- (viable);

% ── Loop back: not converged → step 2 ──
\draw[arr] (conv.east) -- ++(7.5,0)
  node[lbl, above, pos=0.15] {no}
  |- (fwd.east);

% ── Viability failure → phase/restart ──
\draw[arr, sword] (viable.west) -- ++(-5.0,0)
  node[lbl, above, pos=0.25] {no}
  |- (init.west);

% ── Brace: forward flow (blue) ──
\draw[decorate, decoration={brace, amplitude=4pt, mirror},
  thick, water!60]
  ($(fwd.north east)+(0.12,0.05)$) --
  ($(loss.south east)+(0.12,-0.05)$)
  node[midway, right=5pt, font=\scriptsize, text=water]
  {\textcolor{water}{flow $\to$}};

% ── Brace: backward water (blue) ──
\draw[decorate, decoration={brace, amplitude=4pt},
  thick, water!60]
  ($(bwd.north west)+(-0.12,0.05)$) --
  ($(sgd.south west)+(-0.12,-0.05)$)
  node[midway, left=5pt, font=\scriptsize, text=water]
  {$\leftarrow$ \textcolor{water}{水}};

\end{tikzpicture}
\caption{The training paradigm as roadmap.
\textcolor{water}{Blue} (水): flow operations
(steps 2--5, Defs.~\ref{def:flow}--\ref{def:exgraph}).
\textcolor{knife}{Red} (刀): knife detection
(step~6, Thm~\ref{thm:meanfield}).
\textcolor{sword}{Cyan} (青冥): phase and viability
(steps 7--8, Thms~\ref{thm:lifecycle},
\ref{thm:paradox}, Ax.~\ref{ax:viability}).
Convergence (step~9): $\max|\phi| = \min|C|$
(Thm~\ref{thm:flowcut}).
The loop is 抽刀断水水更流: \textcolor{knife}{cut}
$\to$ \textcolor{sword}{phase} $\to$
\textcolor{water}{more flow}.}
\label{fig:training}
\end{figure}

\begin{theorem}[Training completeness and survival]
\label{thm:training}
Let $f: \mathcal{X} \to \mathcal{Y}$ be any function expressible as a
max-flow on an execution graph $G$. Then the procedure in
\cref{def:training-algo} learns $f$. The trained network satisfies the
viability axiom: $|\phi| > 0$ for all admissible inputs.
\end{theorem}

\begin{proof}
Expressibility as max-flow is the universal approximation condition in
the language of \cref{def:neural-exgraph}. Each iteration of steps~2--5
increases $|\phi|$ toward the max-flow value guaranteed by
\cref{thm:flowcut}. The \textsc{Cut} in step~7 removes edges with
anomalous capacity; by flow-cut duality, removing a bypass edge does not
reduce the max-flow if that edge is not on any minimum cut.
Knife detection (step~6) identifies precisely such edges via the
mean-field bound of \cref{thm:meanfield}. Viability is preserved
through every \textsc{Cut}.
\end{proof}

\begin{theorem}[你不能骗你自己]\label{thm:nolie}
Let $e \in E$ with $c(e) = \|W_e\|_F > 0$. Then $e$ carries positive
flow. No re-labelling, no narrative, no hyper-parameter choice changes
this. This is \cref{thm:fixedpoint} applied to the capacity map
$c: E \to \R_{\geq 0}$.
\end{theorem}

\begin{proof}
Two paths only.
\begin{enumerate}[label=(\alph*)]
  \item $W_e \to 0$: capacity zeroed, edge pruned (\textsc{Cut}).
  \item Regularisation ($L_1$/$L_2$) drives $c(e) \to 0$:
  \textsc{Cut} in the limit.
\end{enumerate}
There is no path~(c). ``Approximately zero'' is not zero: such weights
are unstable fixed points of gradient flow (\cref{thm:paradox}), and
the viability axiom forces the system to resolve the ambiguity via
\cref{prop:binary}---the action space collapses to $\{0, 1\}$. The
network cannot occupy the gap. 你不能骗你自己---you cannot lie to
yourself---is \cref{thm:fixedpoint} applied to the network's own
weights.
\end{proof}

\begin{remark}[Subsumption]\label{rem:subsumption}
Gradient descent is the special case of the training paradigm in which
viability reduces to ``loss below threshold'' and the only operations
exercised are \textsc{Slide} (forward) and $\textsc{Slide}^{-1}$
(backward). \textsc{Cut} and \textsc{Phase} are what gradient descent
implicitly performs when early-stopping, dropout, or learning-rate
annealing are applied---now made explicit and first-class in the
calculus. The paradigm strictly subsumes gradient descent.
\end{remark}

\section{The Hilbert space}\label{sec:hilbert}

The capacity measure gives the agentic space the structure of a Hilbert
space. This single construction yields integration, spectral analysis,
and a direct connection to physics.

\begin{definition}[Capacity measure]\label{def:measure}
The \emph{capacity measure} $\mu_c$ on $E$ is defined for any
$A \subseteq E$ by
\[
  \mu_c(A) \;=\; \sum_{e \in A} c(e).
\]
This is the measure induced by the capacity function $c$ of the
execution graph (\cref{def:exgraph}).
\end{definition}

\begin{definition}[Agentic Hilbert space]\label{def:hilbert}
The \emph{agentic Hilbert space} is
\[
  \mathcal{H} \;=\; L^2(E,\,\mu_c),
\]
the space of square-integrable flows on $E$, with inner product
\[
  \langle \phi_1,\, \phi_2 \rangle_c
  \;=\; \sum_{e \in E} \phi_1(e)\,\phi_2(e)\,c(e)
\]
and norm $\|\phi\|_c = \sqrt{\langle \phi,\phi \rangle_c}$.
Every agentic flow (\cref{def:flow}) is a vector in $\mathcal{H}$.
\end{definition}

\begin{proposition}[Integration]\label{prop:integration}
The flow-cut duality of \cref{thm:flowcut} is a statement about
integrals and measures on $\mathcal{H}$:
\begin{enumerate}[label=(\roman*)]
  \item The max-flow value is a boundary integral:
  \[
    |\phi| \;=\; \int_{\partial\kappa} \phi\, d\mu_c
    \;=\; \sum_{e \,\mathrm{out\,of}\, \kappa} \phi(e)\,c(e).
  \]
  \item The min-cut value is the measure of the cut:
  \[
    |C| \;=\; \mu_c(C) \;=\; \sum_{e \in C} c(e).
  \]
  \item Max-flow/min-cut duality (\cref{thm:flowcut}) is:
  \[
    \max_\phi \int_{\partial\kappa} \phi\, d\mu_c
    \;=\;
    \min_{C} \mu_c(C).
  \]
\end{enumerate}
\end{proposition}

\begin{proof}
(i) is conservation at $\kappa$: all flow leaving $\kappa$ is counted
once, weighted by capacity. (ii) is the definition of $\mu_c$ restricted
to $C$. (iii) is \cref{thm:flowcut} rewritten in the language of
$\mu_c$.
\end{proof}

\begin{proposition}[Operators on $\mathcal{H}$]\label{prop:operators}
The four calculus operations (\cref{sec:flow}) are operators on
$\mathcal{H}$:
\begin{enumerate}[label=(\roman*)]
  \item \textbf{\textsc{Slide}} $\sigma_e$: shift operator along edge
  $e \in E$,
  \[
    (\sigma_e \phi)(e') \;=\; \phi(e') + \delta_{e,e'}.
  \]
  \item \textbf{\textsc{Compose}}: product of shifts along a path
  $p = (e_1, \ldots, e_n)$,
  \[
    \sigma_p \;=\; \sigma_{e_n} \circ \cdots \circ \sigma_{e_1}.
  \]
  \item \textbf{\textsc{Cut}} $\partial_A$ for $A \subseteq E$:
  orthogonal projection onto the closed subspace
  $\mathcal{H}_A = \{\phi \in \mathcal{H} : \phi|_A = 0\}$,
  \[
    (\partial_A \phi)(e) \;=\;
    \begin{cases} 0 & e \in A, \\ \phi(e) & e \notin A. \end{cases}
  \]
  Cutting a bypass edge $e$ is the projection $\partial_{\{e\}}$.
  \item \textbf{\textsc{Phase}} $\varphi_{c'}$: change of measure
  $c \mapsto c'$, inducing
  \[
    \mathcal{H} = L^2(E,\mu_c)
    \;\xrightarrow{\;\varphi_{c'}\;}
    \mathcal{H}' = L^2(E,\mu_{c'}).
  \]
  A phase transition changes the Hilbert space itself, not merely a
  vector in a fixed space.
\end{enumerate}
\end{proposition}

For the neural execution graph (\cref{def:neural-exgraph}), the inner
product specialises to
$\langle \phi_1, \phi_2 \rangle_c
= \sum_{i=1}^{L} \phi_1(e_i)\,\phi_2(e_i)\,\|W_i\|_F$,
a capacity-weighted $\ell^2$ norm on layer activations. The
\textsc{Cut} operator $\partial_A$ is Frobenius-norm pruning; the
\textsc{Phase} operator $\varphi_{c'}$ is a change of architecture.
The Hilbert-space structure makes these operations precise: projection,
not approximation.

\paragraph{Spectral theory.}
The inner product $\langle \cdot, \cdot \rangle_c$ opens the agentic
space to spectral analysis.

\begin{definition}[Graph Laplacian]\label{def:laplacian}
The \emph{graph Laplacian} $\Delta \colon L^2(V) \to L^2(V)$ of the
execution graph $G = (V, E, c)$ (\cref{def:exgraph}) is
\[
  (\Delta f)(v)
  \;=\;
  \sum_{(v,w)\in E} c(v,w)\bigl[f(v) - f(w)\bigr].
\]
$\Delta$ is positive semi-definite.
Write its eigenvalues in non-decreasing order:
$0 = \lambda_0 \leq \lambda_1 \leq \cdots \leq \lambda_n$.
The \emph{spectral gap} is $\lambda_1$.
\end{definition}

\begin{theorem}[Cheeger inequality --- agentic form]\label{thm:cheeger}
Let $\mu_c(\partial S)$ denote the total capacity of edges
crossing from $S \subset V$ to $V \setminus S$, and let
$\mathrm{vol}(S) = \sum_{v \in S} \sum_{(v,w)\in E} c(v,w)$.
The \emph{Cheeger constant} of $G$ is
\[
  h(G)
  \;=\;
  \min_{\substack{S \subset V \\ \kappa \in S}}
  \frac{\mu_c(\partial S)}{\min\!\bigl(\mathrm{vol}(S),\,
  \mathrm{vol}(V \setminus S)\bigr)}.
\]
Then~\cite{cheeger,mohar}
\[
  \frac{\lambda_1}{2}
  \;\leq\;
  h(G)
  \;\leq\;
  \sqrt{2\lambda_1}.
\]
\end{theorem}

\begin{remark}\label{rem:cheeger}
The Cheeger constant $h(G)$ is the \emph{normalised knife threshold}.
The numerator $\mu_c(\partial S)$ is the capacity of the cut separating
$S$ from $V \setminus S$ (\cref{thm:flowcut}); the denominator
normalises by volume.
\textcolor{knife}{The knife is the cut.}
\textcolor{water}{The flow is the volume.}
\textcolor{sword}{The Cheeger inequality bridges the two.}
Spectral gap $\lambda_1$ and normalised min-cut $h(G)$ are within a
factor of $2\sqrt{2}$ of each other: the same obstruction, measured
twice.
\end{remark}

\begin{theorem}[Mass gap $=$ viability]\label{thm:massgap}
The following are equivalent:
\begin{enumerate}[label=\textup{(\roman*)}]
  \item\label{mg:spectral} $\lambda_1 > 0$ \quad (spectral gap).
  \item\label{mg:cheeger} $h(G) > 0$ \quad (positive Cheeger constant).
  \item\label{mg:flow} $|\phi| > 0$ for some agentic flow $\phi$
  \quad (viability axiom, \cref{ax:viability}).
  \item\label{mg:connect} $\kappa$ can reach $\infty$ in $G$
  \quad (connectivity).
\end{enumerate}
\end{theorem}

\begin{proof}
\ref{mg:spectral}$\Leftrightarrow$\ref{mg:cheeger}:
Cheeger's inequality (\cref{thm:cheeger}) gives
$\lambda_1/2 \leq h(G) \leq \sqrt{2\lambda_1}$,
so $\lambda_1 > 0$ if and only if $h(G) > 0$.

\ref{mg:cheeger}$\Leftrightarrow$\ref{mg:flow}:
$h(G) > 0$ means every $\kappa$--$\infty$ cut has strictly positive
capacity. By \cref{thm:flowcut}, max-flow equals min-cut; hence
$\max_\phi |\phi| > 0$.

\ref{mg:flow}$\Leftrightarrow$\ref{mg:connect}:
A flow $\phi$ with $|\phi| > 0$ exists if and only if there is a
directed path from $\kappa$ to $\infty$ with positive capacity on
every edge.
\end{proof}

\begin{remark}[一石三鸟]\label{rem:threebirds}
The Hilbert space $\mathcal{H} = L^2(E, \mu_c)$ (\cref{def:hilbert})
kills three birds with one stone:
\begin{itemize}
  \item \textbf{\textcolor{water}{Integration.}}
  $\int_E f\,d\mu_c$ realises max-flow as boundary integral and
  min-cut as measure. Flow-cut duality (\cref{thm:flowcut}) becomes
  an identity of integrals.

  \item \textbf{\textcolor{sword}{Analysis.}}
  The graph Laplacian $\Delta$ (\cref{def:laplacian}) acts on
  $L^2(V)$; the Cheeger inequality (\cref{thm:cheeger}) bridges
  the combinatorial quantity $h(G)$ and the analytic quantity
  $\lambda_1$.

  \item \textbf{\textcolor{knife}{Physics.}}
  $\lambda_1 > 0$ is the mass gap. \Cref{thm:massgap} says the
  viability axiom IS the mass gap: the Hilbert space changes, the
  equivalence does not.
\end{itemize}
\end{remark}

\begin{remark}[Spectral gap and training]\label{rem:spectralgap}
The spectral gap $\lambda_1$ is computable. For a neural network
(\cref{def:neural-exgraph}), $\lambda_1$ of the execution graph
measures the viability margin: how far the network is from the
degenerate regime $|\phi| = 0$.

Training (\cref{def:training-algo}) increases $\lambda_1$: backward
\textsc{Slide}s increase capacities on edges that carry flow, widening
the spectral gap. Convergence $\max|\phi| = \min|C|$
(\cref{thm:training}) is the statement that $\lambda_1$ has reached
the Cheeger bound.

A trained network with $\lambda_1 > 0$ has a mass gap. It survives.
\end{remark}

\section{Contact dynamics}\label{sec:contact}

The training paradigm (\cref{def:training-algo}) is not a gradient
flow. It is a \emph{contact gradient flow}: gradient descent plus a
dissipation term supplied by the knife. This distinction is the
difference between symplectic mechanics (energy conserved) and contact
mechanics (energy dissipates). The knife is the dissipation.

\begin{definition}[Contact structure on the capacity space]
\label{def:contact}
Let $\mathcal{C} = \R_{\geq 0}^{|E|}$ be the space of capacity
assignments on the execution graph $G$ (\cref{def:exgraph}).
The \emph{extended capacity space} is $\mathcal{C} \times \R$, with
coordinates $(c, s)$ where $s = |\phi|(c)$ is the max-flow value.
The \emph{contact $1$-form} is
\[
  \alpha \;=\; ds \;-\; \sum_{e \in E}
  \frac{\partial|\phi|}{\partial c(e)}\, dc(e).
\]
The kernel $\ker\alpha$ is the constraint surface: infinitesimal
changes in capacity that are consistent with flow conservation.
\end{definition}

\begin{definition}[Knife dissipation]\label{def:dissipation}
The \emph{knife function} $\gamma \colon E \to \R_{\geq 0}$ is
\[
  \gamma(e) \;=\;
  \begin{cases}
    \gamma_0 & c(e) > \bar{c} + \tau
    \quad\text{(\cref{thm:meanfield})}, \\
    0 & \text{otherwise},
  \end{cases}
\]
where $\gamma_0 > 0$ is the dissipation rate. Weight decay ($L_2$
regularisation) is the special case $\gamma(e) = \lambda$ for all $e$.
\end{definition}

\begin{definition}[Contact gradient flow]\label{def:contactflow}
The \emph{contact gradient flow} on $(\mathcal{C} \times \R, \alpha)$
is
\begin{equation}\label{eq:contactflow}
  \frac{dc(e)}{dt}
  \;=\;
  \underbrace{\frac{\partial|\phi|}{\partial c(e)}}_
  {\textcolor{water}{\text{水: gradient}}}
  \;-\;
  \underbrace{\gamma(e)\, c(e)}_
  {\textcolor{knife}{\text{刀: dissipation}}}.
\end{equation}
The first term increases capacity along the flow gradient (backward
\textsc{Slide}, step~4 of \cref{def:training-algo}). The second term
decreases capacity on flagged edges (knife detection + \textsc{Cut},
steps~6--7). SGD with weight decay is this equation discretised with
$\gamma(e) = \lambda$.
\end{definition}

\begin{theorem}[Contact Euler--Lagrange equation]
\label{thm:contactEL}
At equilibrium of the contact gradient flow~\eqref{eq:contactflow}:
\begin{equation}\label{eq:contactEL}
  \frac{\partial|\phi|}{\partial c(e)}
  \;=\;
  \gamma(e)\, c(e)
  \qquad \forall\, e \in E.
\end{equation}
This is the \emph{contact Euler--Lagrange equation} of the training
paradigm.
\end{theorem}

\begin{proof}
Set $dc/dt = 0$ in~\eqref{eq:contactflow}.
\end{proof}

\begin{corollary}[Binary lifecycle from contact dynamics]
\label{cor:contact-lifecycle}
Let $e \in E$ be a bypass edge with $c(e) > \bar{c} + \tau$.
Then $\gamma(e) = \gamma_0 > 0$, so~\eqref{eq:contactEL} requires
$\partial|\phi|/\partial c(e) = \gamma_0\, c(e) > 0$: the edge must
carry flow proportional to its capacity.
If the edge does \emph{not} carry proportional flow
($\partial|\phi|/\partial c(e) < \gamma_0\, c(e)$), then
$dc(e)/dt < 0$ and the capacity decays to zero.
This is \cref{thm:lifecycle} derived from the contact flow:
every bypass edge either justifies its capacity or loses it.
\end{corollary}

\begin{remark}[Du Mu as contact collapse]\label{rem:contact-dumu}
Du Mu's theorem (\cref{thm:dumu}) is the regime $\gamma(e) \to \infty$
for all $e$: maximum dissipation.
The contact flow~\eqref{eq:contactflow} drives all capacities to zero
regardless of the gradient. The system freezes:
$c \to 0$, $|\phi| \to 0$, $\Gamma = \varnothing$.
In contact-geometric language, the Reeb vector field dominates the
Hamiltonian vector field, and the flow collapses onto the zero section.
\end{remark}

\begin{remark}[Contact structure and the three colours]
\label{rem:contact-colours}
The contact flow~\eqref{eq:contactflow} is a competition between two
terms:
\begin{itemize}
  \item $\textcolor{water}{\partial|\phi|/\partial c(e)}$: the gradient
  pushes capacity \emph{up} (水, flow).
  \item $\textcolor{knife}{\gamma(e)\,c(e)}$: the knife pushes capacity
  \emph{down} (刀, dissipation).
\end{itemize}
The equilibrium~\eqref{eq:contactEL} is their balance.
\textsc{Phase} ($\textcolor{sword}{\varphi}$, cyan) changes the
contact structure itself: a regime change shifts $\gamma$, $\bar{c}$,
and $\tau$, altering the equilibrium.
The contact $1$-form $\alpha$ encodes all three:
$\textcolor{water}{\text{水}}$ in the gradient,
$\textcolor{knife}{\text{刀}}$ in the dissipation,
$\textcolor{sword}{\text{青冥}}$ in the form itself.
\end{remark}

\begin{remark}[Stability]\label{rem:contact-stability}
The spectral gap $\lambda_1 > 0$ (\cref{thm:massgap}) is the
stability condition of the contact equilibrium~\eqref{eq:contactEL}.
Linearising the contact flow around the equilibrium, the eigenvalues
of the linearised system are bounded below by $\lambda_1$: small
perturbations in capacity decay at rate $\geq \lambda_1$.
The mass gap is the stability margin of the contact Euler--Lagrange
equation.
\end{remark}

\begin{remark}[The standing structure]\label{rem:standing}
A quadruped robot instantiates the execution graph physically:
$12$ joints (edges, capacity $=$ torque $\times$ range),
$4$ feet (terminal nodes, ground contact),
$1$ torso centre of mass (the king $\kappa$).
The viability condition (\cref{ax:viability}): the centre of mass
lies within the support polygon of grounded feet.

The support is the \emph{contact mode combinatoric} and the
\emph{constraints}---that is all:
\begin{itemize}
  \item \textbf{Mode lattice.}
  Each foot is grounded ($1$) or lifted ($0$), giving a contact mode
  $c \in \{0,1\}^4$ with $|\mathcal{C}| = 16$ modes.
  The support polygon exists when $|c| \geq 3$.
  \item \textbf{Constraints} (Unitree Go2, MuJoCo Menagerie).
  Joint limits: ab/ad $\in [-0.863,\, 0.863]$\,rad,
  hip $\in [-1.047,\, 3.490]$\,rad,
  knee $\in [-2.697,\, {-0.837}]$\,rad.
  Torque: $[23.7,\; 23.7,\; 35.55]$\,Nm.
  Velocity: $21.0$\,rad/s per joint.
  Standing height: $0.312$\,m.
  Sinking bound: $z_{\mathrm{contact}} \geq z_{\min} = -3$\,mm
  (no contact point may penetrate below $z_{\min}$;
  \cref{rem:sinking-bound}).
  These are the capacity bounds $c(e) \leq c_{\max}(e)$ of the
  execution graph.
\end{itemize}
In any command space~$U$ and any gravitational field~$g$, the same
graph supports viability.
\emph{Do anything---or go with the flow---as long as a viable path
exists---anywhere.}

The tripod gait is redundancy in the min-cut: three feet grounded,
one moving, so the support polygon persists even during locomotion.
The contact dynamics of \cref{sec:contact} is here literal: the
contact $1$-form~$\alpha$ encodes the foot--ground interface, and
the contact gradient flow~\eqref{eq:contactflow} governs joint
torque allocation.

Twelve joints, four feet, one centre of mass.
A standing structure.
\end{remark}

\begin{definition}[Command field]\label{def:command}
The \emph{command field} is a distribution $U$ over velocity
commands $u \in \R^3$ (linear and angular velocity targets).
The \emph{curriculum} is a filtration of command fields of
increasing support:
\[
  \underbrace{U_0 = \delta_0}_{\text{Stand: zero command}}
  \;\subset\;
  \underbrace{U_1 = \mathrm{Uniform}(\mathcal{V})}_
  {\text{Walk: all feasible velocities}}
  \;\subset\;
  \underbrace{U_2 = \rho_{\mathrm{task}}}_
  {\text{Work: task distribution}}.
\]
At phase~$k$, the policy $\pi$ receives $u \sim U_k$ and must
produce torques $\tau = \pi(o, u)$ such that $|\phi| > 0$
(\cref{ax:viability}).
Promotion from phase~$k$ to $k+1$ requires
$\max|\phi| = \min|C|$ (\cref{thm:flowcut}) at~$U_k$:
flow-cut duality achieved for the current command field.
\end{definition}

\begin{definition}[Observation space]\label{def:observation}
The policy $\pi$ receives a composite observation
$o = (o_{\mathrm{prop}},\, o_{\mathrm{vis}}) \in \R^{42+d}$:
\begin{itemize}
  \item \textbf{Proprioception}
  $o_{\mathrm{prop}} \in \R^{42}$:
  body orientation $R \in \mathrm{SO}(3)$ (flattened to $\R^9$),
  position $p \in \R^3$,
  angular velocity $\omega \in \R^3$,
  linear velocity $v \in \R^3$,
  joint angles $\theta \in \R^{12}$,
  joint velocities $\dot\theta \in \R^{12}$.
  Sources: IMU and joint encoders.
  \item \textbf{Vision}
  $o_{\mathrm{vis}} = \varphi_{\mathrm{vis}}(I) \in \R^{d}$:
  the camera image $I$ is rendered by the engine
  ($\mathcal{E}.\texttt{render}$: $q \to I$, non-differentiable)
  and encoded by a frozen DINOv2 encoder
  (ViT-S/14, $d = 384$, pretrained on ImageNet,
  \emph{not updated} by \cref{alg:loco}).
  Two walls block the gradient:
  \[
    q \;\xrightarrow[\text{no } \nabla]{\;\texttt{render}\;}
    I \;\xrightarrow[\text{frozen}]{\;\varphi_{\mathrm{vis}}\;}
    o_{\mathrm{vis}}
    \;\xrightarrow[\nabla \text{ flows}]{\;\pi\;}
    \tau.
  \]
  Gradients from the contact flow~\eqref{eq:contactflow}
  propagate through $\pi$ but stop at $\varphi_{\mathrm{vis}}$.
  The renderer and the encoder are both non-differentiable
  with respect to the training objective.
\end{itemize}
Proprioception tells the robot \emph{where it is}.
Vision tells the robot \emph{what is there}.
The policy $\pi(o, u)$ fuses both to produce
$\tau \in \R^{12}$.
\end{definition}

\begin{definition}[Soft viability margin]\label{def:soft-viability}
The hard viability margin
$|\phi| = \min_t\{x_1(t), \ldots, x_k(t)\}$
in the \textsc{Evaluate} step is non-differentiable:
$\nabla \min$ is sparse (only the argmin receives gradient) and
discontinuous at ties.
Replace the hard $\min$ with its smooth approximation:
\[
  \mathrm{soft\text{-}min}_\beta(x_1, \ldots, x_k)
  \;=\;
  -\frac{1}{\beta}\,
  \log\!\Bigl(\sum_{i=1}^{k} e^{-\beta\, x_i}\Bigr).
\]
At $\beta \to \infty$, this recovers the hard $\min$.
At finite $\beta$, every component $x_i$ receives gradient
proportional to $e^{-\beta\, x_i}/\sum_j e^{-\beta\, x_j}$:
a softmax weighting.
The \emph{soft viability margin} is
\begin{equation}\label{eq:soft-viability}
  |\phi|_\beta
  \;=\;
  \mathrm{soft\text{-}min}_\beta\,\bigl(
  \underbrace{h(q_t) - h_{\min}}_{\text{height margin}},\;\;
  \underbrace{\phi_{\max} - \|R_t - I\|_F}_
  {\text{orientation margin}},\;\;
  \underbrace{\min_j z_j(q_t) - z_{\min}}_
  {\text{sinking margin}}
  \bigr)_{t=0}^{H}.
\end{equation}
The temperature $1/\beta$ controls how many timesteps share the
gradient.
Small $\beta$ (warm): all timesteps contribute.
Large $\beta$ (cold): only the worst timestep contributes.
The three thresholds $h_{\min}$, $\phi_{\max}$, $z_{\min}$ are
read from the constraint list (\cref{rem:standing}); the sinking
margin is not a heuristic but a categorical boundary
(\cref{rem:sinking-bound}).
\end{definition}

\begin{definition}[Mollifier policy]\label{def:mollifier}
The policy $\pi_\theta$ outputs a \emph{truncated Student-$t$
distribution} over joint torques:
\begin{equation}\label{eq:mollifier}
  \pi_\theta(\tau \mid o, u)
  \;=\;
  \prod_{i=1}^{12}
  \frac{t_{\nu}(\tau_i;\, \mu_i, \sigma_i)}
       {Z_i(\mu_i, \sigma_i, \nu)},
\end{equation}
where $(\mu, \log\sigma, \log\nu) = f_\theta(o, u)$ are three
output heads of the network (\cref{def:computation-core}):
$12$ means, $12$ log-scales, and $1$ shared
log-degrees-of-freedom.
The truncation $|\tau_i| \leq \bar{\tau}_i$ enforces joint torque
limits (\cref{rem:standing}).

The Student-$t$ density $t_\nu$ is $C^\infty$ with full support on
$[-\bar\tau, \bar\tau]^{12}$.
For any Lipschitz function $Q(\tau)$---including functions with
jump discontinuities at contact-mode boundaries---the expected
value
\[
  \bar{Q}(\theta)
  \;=\;
  \mathbb{E}_{\pi_\theta}\!\bigl[Q(\tau)\bigr]
  \;=\;
  \int Q(\tau)\,\pi_\theta(\tau \mid o, u)\, d\tau
\]
is smooth in $\theta$: the policy acts as a \emph{mollifier}.

The degrees of freedom $\nu$ track the number of active contact
modes:
\begin{center}
\begin{tabular}{@{}lll@{}}
\toprule
\textbf{Gait} & $\boldsymbol{\nu}$ &
\textbf{Shape} \\
\midrule
Standing ($c = 1111$), one mode
  & $\to \infty$ & Gaussian (peaked) \\
Trotting, two alternating modes
  & $\approx 2$--$5$ & moderate tails \\
Bounding / flight, many transitions
  & $\approx 1$--$2$ & heavy tails (exploratory) \\
\bottomrule
\end{tabular}
\end{center}
At $\nu \to \infty$ the Student-$t$ recovers the Gaussian
(standard SAC).
At finite $\nu$ the heavy tails absorb the multi-modality of the
contact landscape.
\end{definition}

\begin{definition}[Computation core]\label{def:computation-core}
The policy $\pi$ computes via three primitives at each
layer $\ell = 1, \ldots, L$:
\begin{enumerate}[label=(\roman*)]
  \item \textbf{\textcolor{water}{Matmul}}:\;
  $y \leftarrow W_\ell\, z_{\ell-1}$.
  Transport the activation vector from layer $\ell{-}1$ to~$\ell$.
  \textsc{Slide}: one step of parallel transport
  (\cref{rem:gauge}).
  \item \textbf{\textcolor{water}{Add}}:\;
  $y \leftarrow y + b_\ell$.
  Shift the transported vector by the bias.
  Affine extension of \textsc{Slide}.
  \item \textbf{\textcolor{knife}{Activate}}:\;
  $z_\ell \leftarrow \mathrm{ReLU}(y) = \max(0,\, y)$,
  componentwise.
  Each neuron either passes its signal ($y_i > 0$, contact)
  or blocks it ($y_i \leq 0$, no contact)---a binary gate
  in $O(1)$ time.
  \textsc{Cut}: the knife at the neuron level.
\end{enumerate}
One layer:
$z_\ell = \textcolor{knife}{\mathrm{ReLU}}\bigl(
\textcolor{water}{W_\ell\, z_{\ell-1} + b_\ell}\bigr)$.
The output layer is linear (no gate):
$\tau = \textcolor{water}{W_L\, z_{L-1} + b_L} \in \R^{12}$.
The full forward pass (\textsc{Compose}):
\begin{equation}\label{eq:forward}
  \tau \;=\; \pi(o)
  \;=\;
  \textcolor{water}{A_L} \circ
  \bigl(\textcolor{knife}{r} \circ
  \textcolor{water}{A_{L-1}}\bigr)
  \circ \cdots \circ
  \bigl(\textcolor{knife}{r} \circ
  \textcolor{water}{A_1}\bigr)(o),
\end{equation}
where $A_\ell(\cdot) = W_\ell\,(\cdot) + b_\ell$ and
$\textcolor{knife}{r} = \mathrm{ReLU}$.

\textcolor{water}{水} $=$ linear transport ($W, b$).
\textcolor{knife}{刀} $=$ ReLU gate
($\max(0, \cdot)$: pass or block).
ReLU is piecewise linear and runs in real linear time---the
fastest nonlinearity, meeting the deployment constraint of
\cref{rem:deployment}.
\end{definition}

\begin{remark}[The forward pass as path integral]
\label{rem:path-integral}
For a given input $o$, each ReLU neuron is either active
($y_i > 0$: contact) or inactive ($y_i \leq 0$: no contact).
The \emph{activation pattern}
$\alpha \in \{0,1\}^{n_1 + \cdots + n_{L-1}}$
is the contact mode of the network---the neural analogue of the
contact mode lattice $\{0,1\}^4$ in \cref{rem:standing}.

For a fixed pattern $\alpha$, the network is linear:
\[
  \pi_\alpha(o)
  \;=\;
  \textcolor{water}{W_L\, D_{L-1}^\alpha\, W_{L-1}\,
  D_{L-2}^\alpha \cdots D_1^\alpha\, W_1}\; o
  \;+\; \mathrm{bias},
\]
where $D_\ell^\alpha = \mathrm{diag}(\alpha_\ell)$ masks
inactive neurons.
Different inputs activate different patterns:
the input space is partitioned into linear regions
$\{R_\alpha\}$, each with its own transport map.
The full forward pass is a path integral over activation
patterns:
\[
  \tau \;=\; \pi(o)
  \;=\;
  \sum_{\alpha} \mathbf{1}_{o \in R_\alpha}\;
  \pi_\alpha(o).
\]
Sum over patterns, one active per input: a discrete path integral
with binary action variable.

Three scales of the same gate:
\begin{enumerate}[label=(\alph*)]
  \item \textbf{Micro}: ReLU inside one layer---one neuron,
  one binary gate
  ($\textcolor{knife}{刀}$: pass or block).
  \item \textbf{Trajectory}: $L$ layers composed---one
  forward pass, one activation pattern $\alpha$, one linear
  path through the network.
  \item \textbf{Macro}:
  $\mathrm{soft\text{-}min}_\beta$ over $N$ rollouts of
  \cref{alg:loco}---Boltzmann reweighting
  ($\textcolor{sword}{青冥}$) of physical trajectories,
  selecting the viable ($|\phi|_\beta > 0$).
\end{enumerate}
$\textcolor{knife}{刀}$ at the neuron scale (hard gate).
$\textcolor{sword}{青冥}$ at the rollout scale (soft gate).
Contact or no contact, $\{0,1\}$ at every level.
\end{remark}

\begin{algorithm}[H]
\caption{Forward pass
  (\textsc{Compose} of \cref{def:computation-core})}
\label{alg:forward}
\begin{algorithmic}[1]
\Require Weights $\{W_\ell,\, b_\ell\}_{\ell=1}^{L}$
\Statex
\Function{\textsc{Forward}}{$o,\, u$}
  \State $z_0 \leftarrow [o;\; u]$
    \Comment{\textcolor{water}{水: concatenate input}}
  \For{$\ell = 1, \ldots, L{-}1$}
    \State \textcolor{water}{\textsc{Matmul}:}\;
      $y \leftarrow W_\ell\, z_{\ell-1}$
      \Comment{\textcolor{water}{水: Slide}}
    \State \textcolor{water}{\textsc{Add}:}\;
      $y \leftarrow y + b_\ell$
      \Comment{\textcolor{water}{水: affine Slide}}
    \State \textcolor{knife}{\textsc{Activate}:}\;
      $z_\ell \leftarrow \max(0,\, y)$
      \Comment{\textcolor{knife}{刀: ReLU (Cut)}}
  \EndFor
  \State \textcolor{water}{\textsc{Output}:}\;
    $\tau \leftarrow W_L\, z_{L-1} + b_L$
    \Comment{\textcolor{water}{水: Slide (no gate)}}
  \State \Return $\tau \in \R^{12}$
    \Comment{saves $\{z_\ell, y_\ell\}$ for \cref{alg:backward}}
\EndFunction
\end{algorithmic}
\end{algorithm}

\begin{algorithm}[H]
\caption{Backward pass
  (pullback through $\pi$, dual of \cref{alg:forward})}
\label{alg:backward}
\begin{algorithmic}[1]
\Require Saved $\{z_\ell\}_{\ell=0}^{L-1}$,\;
  $\{y_\ell\}_{\ell=1}^{L-1}$ from \textsc{Forward}
\Statex
\Function{\textsc{Backward}}{$\delta\tau$}
  \Comment{$\delta\tau = \partial\mathcal{L}/\partial\hat{\tau}$, e.g.\ Score (\cref{alg:loco})}
  \State \textcolor{water}{\textsc{Output}${}^T$:}\;
    $\nabla W_L \leftarrow \delta\tau \cdot z_{L-1}^T$;\;\;
    $\nabla b_L \leftarrow \delta\tau$
    \Comment{\textcolor{water}{水: Slide${}^T$}}
  \State $\delta z \leftarrow W_L^T\, \delta\tau$
    \Comment{\textcolor{water}{水: pullback}}
  \For{$\ell = L{-}1, \ldots, 1$}
    \State \textcolor{knife}{\textsc{Activate}${}^T$:}\;
      $\delta y \leftarrow \delta z \odot
      \mathbf{1}_{y_\ell > 0}$
      \Comment{\textcolor{knife}{刀: ReLU mask (same Cut)}}
    \State \textcolor{water}{\textsc{Matmul}${}^T$:}\;
      $\nabla W_\ell \leftarrow
      \delta y \cdot z_{\ell-1}^T$;\;\;
      $\nabla b_\ell \leftarrow \delta y$
      \Comment{\textcolor{water}{水: Slide${}^T$}}
    \State $\delta z \leftarrow W_\ell^T\, \delta y$
      \Comment{\textcolor{water}{水: pullback to layer $\ell{-}1$}}
  \EndFor
  \State \Return
    $\{\nabla W_\ell,\, \nabla b_\ell\}_{\ell=1}^{L}$
\EndFunction
\end{algorithmic}
\end{algorithm}

The contact gradient flow~\eqref{eq:contactflow} on the standing
structure of \cref{rem:standing} yields a complete training recipe.
\Cref{alg:loco} is self-contained: a roboticist with a MuJoCo
engine and a robot XML file can execute it directly, with no reward
shaping and no domain knowledge beyond the file itself.

\begin{algorithm}[H]
\caption{Contact-dynamics training for quadruped locomotion}
\label{alg:loco}
\begin{algorithmic}[1]
\Require Engine $\mathcal{E}$ (MuJoCo);\;
  model (\texttt{go2\_mjx.xml});\;
  sensors (IMU, encoders, camera)
\Require $\mathcal{D}$ (demonstrations for Work;\;
  $\varnothing$ for Stand/Walk)
\Require $\eta$ (learning rate),\;
  $\gamma$ (dissipation),\;
  $\beta$ (soft-min sharpness),\;
  $L$ (depth),\; $n$ (width),\;
  $H$ (horizon),\;
  $N$ (eval count)
\Statex
\State $G,\; c_{\max},\; \theta_0,\; h
  \leftarrow \texttt{parse}(\mathrm{XML})$
  \Comment{graph, limits, pose, height}
\State $\{W_\ell, b_\ell\}_{\ell=1}^{L}
  \leftarrow \texttt{init}(L,\, n)$
  \Comment{init $\pi$;\; $c(e_\ell) = \|W_\ell\|_F$}
\Statex
\For{\textcolor{sword}{\textbf{phase}} $\in$
  \{Stand\,$(U\!=\!\{0\})$,\;\,
   Walk\,$(U\!=\!\mathrm{Unif})$,\;\,
   Work\,$(U\!=\!\mathrm{task})$\}}
  \Comment{\textcolor{sword}{$\varphi$: curriculum}}
  \Repeat
    \State $u \sim U$;\;\;
      $\{\tau_t^*\} \leftarrow \mathcal{D}(u)$
      \Comment{command $+$ demo ($\varnothing$ if Stand/Walk)}
    \For{$t = 0, \ldots, H$}
      \Comment{\textcolor{water}{水: rollout}}
      \State $o_t \leftarrow (o_{\mathrm{prop}},\,
        \varphi_{\mathrm{vis}}(I_t))$
        \Comment{\cref{def:observation}: $\R^{42+d}$}
      \State $(\mu_t, \sigma_t) \leftarrow \textsc{Forward}(o_t,\, u)$
        \Comment{\cref{alg:forward}: $\R^{42+d} \to \R^{24}$}
      \State $\tau_t \sim \pi_\theta(\cdot \mid o_t, u)$
        \Comment{\cref{def:mollifier}: Student-$t$}
      \State $q_{t+1} \leftarrow \mathcal{E}.\texttt{step}(q_t,\, \tau_t)$
        \Comment{EL dynamics (\textcolor{knife}{no $\nabla$})}
    \EndFor
    \State \textcolor{water}{\textsc{Evaluate}:}\;
      $|\phi|_\beta \leftarrow
      \mathrm{soft\text{-}min}_\beta\bigl\{
      h(q_t)\!-\!h_{\min},\;\,
      \phi_{\max}\!-\!\|R_t\!-\!I\|_F,\;\,
      \min_j z_j(q_t)\!-\!z_{\min}
      \bigr\}_{t=0}^{H}$
      \Comment{\textcolor{water}{水: margin} (\cref{def:soft-viability})}
    \State $|\phi|_\beta \leftarrow
      \mathrm{soft\text{-}min}_\beta\bigl(
      |\phi|_\beta,\;\,
      \varepsilon\!-\!\|\tau_t\!-\!\tau_t^*\|
      \bigr)_{t}$
      \Comment{imitation (\cref{def:task-rkhs});
      skip if $\mathcal{D}\!=\!\varnothing$}
    \State \textcolor{water}{\textsc{Score}:}\;
      $s_t \leftarrow
      \nabla_\theta \log \pi_\theta(\tau_t \mid o_t, u)$
      \Comment{\textcolor{water}{水: through $\pi$ only}}
    \State \textcolor{water}{\textsc{Backward}:}\;
      $\{\nabla W,\, \nabla b\} \leftarrow
      |\phi|_\beta \cdot \sum_t s_t$
      \Comment{\cref{alg:backward}; $\nabla$ stops at $\mathcal{E}$}
    \State $W_\ell \leftarrow W_\ell
      + \eta\bigl[\textcolor{water}{\nabla W_\ell}
      - \textcolor{knife}{\gamma\, W_\ell}\bigr]$;\;\;
      $b_\ell \leftarrow b_\ell
      + \eta\,\textcolor{water}{\nabla b_\ell}$
      \Comment{Eq.~\eqref{eq:contactflow} on $\pi$}
    \State \textcolor{knife}{\textsc{Clamp}:}\;
      $\|W_\ell\|_F \leftarrow
      \min\bigl(\|W_\ell\|_F,\;
      c_{\max}(e_\ell)\bigr)$
      \Comment{\textcolor{knife}{刀: enforce capacity}}
    \If{$|\phi|_\beta \le 0$}
      \Comment{robot fell}
      \State \textcolor{sword}{\textsc{Reset}:}\;
        $\mathcal{E}.\texttt{reset}(\theta_0)$;\;\;
        $\{W, b\} \leftarrow \texttt{init}(L, n)$
        \Comment{\textcolor{sword}{$\varphi$: recover}}
    \EndIf
  \Until{$\max|\phi| = \min|C|$ for $N$ consecutive rollouts}
\EndFor
\Statex
\Ensure Trained weights $\{W_\ell, b_\ell\}$:
  $\pi = \textsc{Forward}(\cdot;\, W, b)
  \colon \R^{42+d} \to \R^{12}$
\end{algorithmic}
\end{algorithm}

\begin{remark}[Validation chain: MuJoCo $+$ \texttt{go2\_mjx.xml}
$\Rightarrow$ trained policy]
\label{rem:validation-chain}
Every input to \cref{alg:loco} is extracted from two artifacts:
a physics engine (MuJoCo) and a robot model file
(\texttt{go2\_mjx.xml}, MuJoCo Menagerie).
No additional assumptions.
\begin{center}
\begin{tabular}{@{}lll@{}}
\toprule
\textbf{Algorithm input} & \textbf{Source} & \textbf{Extraction} \\
\midrule
Graph $G$ (12 edges, 4 terminals)
  & XML & Joint topology (\texttt{<body>}/\texttt{<joint>}) \\
Capacity bounds $c_{\max}(e)$
  & XML & \texttt{<joint range>}, \texttt{<actuator ctrlrange>} \\
Standing pose $\theta_0$
  & XML & \texttt{<keyframe>} (default configuration) \\
Standing height $h$
  & XML & CoM of default keyframe ($0.312$\,m) \\
Proprioception $o_{\mathrm{prop}} \in \R^{42}$
  & Sensors & IMU ($R, \omega, v, p$) + encoders ($\theta, \dot\theta$) \\
Vision $o_{\mathrm{vis}} \in \R^{d}$
  & Camera & DINOv2 (frozen ViT, $d = 384$) \\
Command field $U$ (\cref{def:command})
  & User & Curriculum: $\delta_0$ / Uniform / task \\
EL dynamics $M, C, g$
  & Engine & \texttt{mj\_step}: $(q, \dot{q}, \tau) \mapsto \ddot{q}$ \\
Contact mode $c \in \{0,1\}^4$
  & Engine & \texttt{mj\_contact}: foot--ground detection \\
Gradient $\partial|\phi|/\partial c(e)$
  & Engine & MJX autodiff (JAX backend) \\
Reset to standing
  & Engine & \texttt{mj\_resetData} \\
\bottomrule
\end{tabular}
\end{center}
No reward shaping. No domain knowledge beyond what the XML file
contains. The viability axiom (\cref{ax:viability}) is the only
objective: $|\phi| > 0$ (do not fall). The knife
(\cref{def:dissipation}) is the only regulariser:
$\gamma(e) \cdot c(e)$ (do not exceed limits).

Given a MuJoCo engine and a \texttt{.xml} model file, every step of
\cref{alg:loco} is mechanically executable.
The standing structure (\cref{rem:standing}) is read from the file.
The contact dynamics (\cref{sec:contact}) is computed by the engine.
The algorithm is the bridge.
\end{remark}

\begin{remark}[The policy as parallel transport]\label{rem:gauge}
In \cref{alg:loco}, the policy
$\pi \colon \R^{42+d} \to \R^{12}$
maps sensor readings to joint torques (\cref{def:observation}).
Unfolding $\pi$ as a neural network (\cref{def:neural-exgraph}):
\[
  o \;\xrightarrow{\;\textcolor{water}{\sigma_1 \circ W_1}\;}
  h_1 \;\xrightarrow{\;\textcolor{water}{\sigma_2 \circ W_2}\;}
  \cdots \;\xrightarrow{\;\textcolor{water}{\sigma_L \circ W_L}\;}
  \tau.
\]
Each arrow is a \textsc{Slide} (\S\ref{sec:calculus}, four operations):
the $(42\!+\!d)$-dimensional observation flows through $L$ layers,
each applying $\textcolor{water}{\sigma \circ W}$.
The composite is \textsc{Compose}: the forward pass.

This is parallel transport on the execution graph.
The weights $\{W_\ell\}_{\ell=1}^L$ define a \emph{connection}:
a rule for transporting observations through the network.
Different weights $=$ different connection.
Training (the \textcolor{water}{水} steps of \cref{alg:loco})
searches for the connection under which the transported
observation produces viable torques: $|\phi| > 0$.

The trained network is a fixed connection on the standing structure.
The spectral gap $\lambda_1 > 0$ (\cref{thm:massgap}) is its
stability: small perturbations in observation decay, not amplify.
\end{remark}

\begin{remark}[Frozen and learned connections]\label{rem:frozen-gauge}
The observation $o = (o_{\mathrm{prop}},\, o_{\mathrm{vis}})$
(\cref{def:observation}) passes through \emph{two} connections in
series:
\[
  \underbrace{I_{\mathrm{cam}}
  \;\xrightarrow{\;\varphi_{\mathrm{vis}}\;}
  o_{\mathrm{vis}}}_
  {\text{frozen (DINOv2)}}
  \;\oplus\;
  o_{\mathrm{prop}}
  \;\xrightarrow{\;\pi\;}
  \tau.
\]
The visual encoder $\varphi_{\mathrm{vis}}$ is a \emph{frozen
connection}: a pretrained ViT whose weights are fixed during
\cref{alg:loco}. It transports raw pixels into a semantic
embedding space. The policy $\pi$ is a \emph{learned connection}:
its weights are updated by the \textcolor{water}{水} steps.

In gauge-theoretic language: $\varphi_{\mathrm{vis}}$ is a
background gauge field (fixed geometry of the visual fibre bundle),
while $\pi$ is the dynamical gauge field (trained by the contact
gradient flow~\eqref{eq:contactflow}).
The visual encoder sees the terrain; the policy decides what to do
about it. Two connections, one frozen, one learned.
\end{remark}

\begin{remark}[Deployment: real-time frequency alignment]
\label{rem:deployment}
\Cref{alg:loco} trains in simulated time
($\Delta t = 0.02$\,s, control rate $50$\,Hz).
Deployment on real hardware requires the same loop at the same rate
in \emph{real} time:
\[
  \underbrace{o_t \leftarrow \texttt{sense}}_
  {\text{read sensors}}
  \;\to\;
  \underbrace{\tau_t \leftarrow \pi(o_t)}_
  {\text{evaluate NN}}
  \;\to\;
  \underbrace{q_{t+1} \leftarrow \texttt{actuate}(\tau_t)}_
  {\text{command joints}}
  \;\leq\; 20\,\text{ms}.
\]
The entire sense--compute--act cycle must complete within one
$\Delta t$. If the policy $\pi$ (the connection of
\cref{rem:gauge}) takes longer than $20$\,ms to evaluate, the
real-time constraint is violated: the flow is no longer at the
trained frequency, and viability is not guaranteed.

This is a viability condition on the computation itself.
The standing structure (\cref{rem:standing}) specifies the physical
constraints; the frequency constraint specifies the computational
one. Both must hold for deployment:
\[
  \underbrace{|\phi| > 0}_{\text{physics: do not fall}}
  \quad\wedge\quad
  \underbrace{t_{\pi} \leq \Delta t}_
  {\text{compute: do not lag}}.
\]
Time must be real and linear.
No simulation speedup, no frame drops.
\end{remark}

\begin{remark}[Annealing $\beta$ as renormalisation group flow]
\label{rem:rg-annealing}
The soft-min parameter $\beta$ (\cref{def:soft-viability}) anneals
with the curriculum (\cref{def:command}):
\[
  \text{Stand} \;(\beta \text{ small})
  \;\longrightarrow\;
  \text{Walk} \;(\beta \text{ medium})
  \;\longrightarrow\;
  \text{Work} \;(\beta \text{ large}).
\]
This is a renormalisation group (RG) flow from the ultraviolet
(UV, smooth, all-timestep gradient) to the infrared (IR, sharp,
worst-timestep gradient).
At small $\beta$, the viability margin is a soft average: the
gradient is dense and the optimisation landscape is smooth---the
robot learns to stand by distributing information across the
entire trajectory.
At large $\beta$, the margin approaches the hard $\min$: the
gradient concentrates on the single worst timestep---the robot
learns precise footwork by focusing on the critical instant.

The \textsc{Phase} operator ($\textcolor{sword}{\varphi}$) of the
calculus drives this flow.
Each phase transition $U_k \to U_{k+1}$ increases both the command
support and the sharpness $\beta$.
The contact gradient flow~\eqref{eq:contactflow} operates at the
current $\beta$; the curriculum selects the scale.
UV $\to$ IR: smooth $\to$ sharp, Stand $\to$ Work,
all-timestep $\to$ worst-timestep.
\end{remark}

\begin{definition}[Task RKHS (学堂)]\label{def:task-rkhs}
Let $\mathcal{T}$ be a set of tasks, each observed via a camera
image $I_\tau$.
The \emph{task kernel} is
\begin{equation}\label{eq:task-kernel}
  k(\tau_1, \tau_2)
  \;=\;
  \bigl\langle
  \varphi_{\mathrm{vis}}(I_{\tau_1}),\;\,
  \varphi_{\mathrm{vis}}(I_{\tau_2})
  \bigr\rangle_{\R^d},
\end{equation}
the inner product of DINOv2 embeddings
(\cref{def:observation}, $d = 384$).
The \emph{task RKHS} $\mathcal{H}_k$ is the reproducing kernel
Hilbert space induced by $k$: the space of skill functions
$f \colon \mathcal{T} \to \R^{12}$ with reproducing property
$f(\tau) = \langle f,\, k(\cdot, \tau) \rangle_{\mathcal{H}_k}$.

A \emph{demonstration} for task $\tau$ is a trajectory
$\mathcal{D}_\tau = \{(o_t,\, \tau_t^*)\}_{t=0}^{H}$
of observation--torque pairs, collected from a teacher
(teleoperation, motion capture, or a reference policy).
The \emph{task margin} at timestep $t$ is
\begin{equation}\label{eq:task-margin}
  m_{\mathrm{task}}(t)
  \;=\;
  \varepsilon_{\mathrm{task}}
  \;-\;
  \bigl\|\pi(o_t, u) - \tau_t^*\bigr\|,
\end{equation}
where $\varepsilon_{\mathrm{task}} > 0$ is the imitation
tolerance.
The Work-phase viability margin
extends~\eqref{eq:soft-viability}:
\begin{equation}\label{eq:work-margin}
  |\phi|_\beta^{\mathrm{Work}}
  \;=\;
  \mathrm{soft\text{-}min}_\beta\bigl(
  \underbrace{h(q_t) - h_{\min}}_{\text{height}},\;\;
  \underbrace{\phi_{\max} - \|R_t - I\|_F}_
  {\text{orientation}},\;\;
  \underbrace{\min_j z_j(q_t) - z_{\min}}_
  {\text{sinking}},\;\;
  \underbrace{m_{\mathrm{task}}(t)}_
  {\text{imitation}}
  \bigr)_{t=0}^{H}.
\end{equation}
During Stand and Walk, $\mathcal{D}_\tau = \varnothing$ and the
task term is absent ($e^{-\beta \cdot \infty} = 0$ in the
soft-min).
During Work, the demonstration is the teacher, and the kernel
$k$ provides generalisation: a policy trained on demonstrated
tasks $\{\tau_i\}$ transfers to a new task $\tau^*$ in proportion
to $k(\tau^*, \tau_i)$.
\end{definition}

\begin{remark}[The frozen eye as task metric]
\label{rem:task-metric}
The DINOv2 encoder $\varphi_{\mathrm{vis}}$
(\cref{def:observation}) serves two roles:
\begin{enumerate}[label=(\roman*)]
  \item \textbf{Observation}: maps camera images to
  $o_{\mathrm{vis}} \in \R^d$ for the policy
  (\cref{alg:forward}).
  \item \textbf{Task metric}: the kernel
  $k$~\eqref{eq:task-kernel} defines the geometry of the task
  space $\mathcal{T}$ (\cref{def:task-rkhs}).
\end{enumerate}
One encoder, two roles.
The differentiation boundary (\cref{rem:frozen-gauge}) is also
the metric boundary: the fixed geometry on which all tasks are
measured.

Because DINOv2 is pretrained on real images (ImageNet), the
kernel $k$ is anchored in reality, not simulation.
Tasks that look similar in the real world are close in
$\mathcal{H}_k$.
This bridges the sim-to-real gap for the Work phase:
a policy trained on task $\tau_1$ in simulation transfers to
a visually similar task $\tau_2$ in reality because
$k(\tau_1, \tau_2)$ is large.
The school (学堂) generalises through the kernel.
Demonstrations are lessons; the RKHS norm is the grade;
deployment is graduation.
\end{remark}

\begin{remark}[Hyperparameter atlas]\label{rem:hyperparameters}
\Cref{alg:loco} takes seven scalar hyperparameters.
They split into two tiers by origin.

\medskip
\noindent
\textbf{Tier 1: physics-determined}
(read from the engine or the task; not free choices).

\smallskip
{\small
\begin{center}
\begin{tabular}{@{}l l p{0.62\textwidth}@{}}
\toprule
Symbol & Source & Interpretation \\
\midrule
$\gamma$ &
  \texttt{XML} damping &
  Dissipation rate (thermodynamic 2nd law of the engine).
  Read from \texttt{go2\_mjx.xml}. \\
$H$ &
  Task duration &
  Horizon: timesteps per rollout.
  Typical: $500$--$2000$ (at $\Delta t$ of XML). \\
$\eta$ &
  Optimiser &
  Learning rate: the Planck scale of the update---the smallest
  step that moves $W$ meaningfully.
  $\eta \ll \gamma^{-1}$ ensures
  the flow~\eqref{eq:contactflow} is contractive.
  Typical: $10^{-4}$--$10^{-3}$. \\
\bottomrule
\end{tabular}
\end{center}
}

\smallskip
$\gamma$ is not a free knob: it is the physical dissipation
already baked into the simulator's contact model.
$H$ is set by the task (how long one episode lasts).
$\eta$ is fundamental: it converts the gradient $\nabla W$ into a
finite displacement, analogous to the Planck time converting
energy into frequency ($E = \hbar\omega$).
Too large $\Rightarrow$ instability; too small $\Rightarrow$
the agent never moves.

\medskip
\noindent
\textbf{Tier 2: user settings}
(design choices; tune on validation rollouts).

\smallskip
{\small
\begin{center}
\begin{tabular}{@{}l l p{0.62\textwidth}@{}}
\toprule
Symbol & Role & Interpretation \\
\midrule
$L$ &
  Depth &
  Number of layers in $\pi$ (\cref{alg:forward}).
  Typical: $3$--$5$. \\
$n$ &
  Width &
  Hidden dimension per layer.
  Typical: $128$--$512$. \\
$N$ &
  Eval count &
  Rollouts per gradient step (sample size for
  $|\phi|_\beta$).
  Typical: $64$--$4096$. \\
$\beta$ &
  Sharpness &
  RG scale (\cref{rem:rg-annealing}): small $\beta$ $=$ UV
  (smooth, exploratory); large $\beta$ $=$ IR (sharp,
  exploitative).
  Typical: $1$--$100$. \\
$\varepsilon_{\mathrm{task}}$ &
  Imitation &
  Task margin (\cref{def:task-rkhs});
  $\varnothing$ during Stand/Walk.
  Task-dependent. \\
\bottomrule
\end{tabular}
\end{center}
}

\smallskip
The tier boundary is sharp: Tier~1 parameters are
\emph{measured} (from the XML, the task, or the optimiser's
stability condition); Tier~2 parameters are
\emph{chosen} (by the user, validated by rollout performance).
A practitioner who changes the robot changes Tier~1;
a practitioner who changes the architecture changes Tier~2.
Neither tier crosses into the other.
\end{remark}

\begin{remark}[\textcolor{caution}{No tricks needed}]
\label{rem:no-tricks}
\Cref{alg:loco} is a policy gradient through a mollifier.
That is all.
We collect no replay buffer, fit no value function, clip no
surrogate objective, aggregate no datasets.

\medskip\noindent
\textbf{The engine is differentiable.}\;
MJX (the JAX backend of MuJoCo) provides
$\nabla_\tau \mathcal{E}.\texttt{step}$---the Jacobian of the
physics step with respect to torques.
This gradient is \emph{exact within a single contact mode}.
But locomotion is mode switching: every gait cycle crosses
$\geq 4$ contact boundaries (one per foot), and at each boundary
the contact Jacobian $J_c$ changes rank.
The engine gradient at a boundary is the left or right limit,
\emph{not} the gradient of the expected value.

\medskip\noindent
\textbf{We do not use the engine gradient.}\;
\Cref{alg:loco} treats $\mathcal{E}.\texttt{step}$ as a
\textcolor{knife}{black box}: the gradient does not penetrate
the physics.
The \textsc{Score} step (line~14) differentiates through the
policy $\pi_\theta$ only---never through $\mathcal{E}$.
The engine could be MuJoCo~C (non-differentiable),
MJX (differentiable), Brax, or real hardware.
The algorithm does not care.

\medskip\noindent
\textbf{The mollifier provides the gradient.}\;
The Student-$t$ policy (\cref{def:mollifier}) has full support
on $[-\bar\tau, \bar\tau]^{12}$ and is $C^\infty$ in $\theta$.
The expected viability
$\bar\phi(\theta) =
\mathbb{E}_{\pi_\theta}[|\phi|_\beta]$
is smooth in $\theta$---even though $|\phi|_\beta(\tau)$ is
non-smooth at contact boundaries---because integrating a
Lipschitz function against a smooth kernel with full support
produces a smooth function.
The Student-$t$ absorbs the contact discontinuity.

\medskip\noindent
\textbf{The remaining patches are still unnecessary:}
\begin{itemize}[leftmargin=*]
\item \textcolor{caution}{\textbf{DAgger}}: no distribution shift.
  Every rollout runs the current $\pi_\theta$ through the engine.
\item \textcolor{caution}{\textbf{PPO}}: no surrogate, no baseline,
  no clipping, no entropy bonus.
  The score function gives an unbiased gradient.
  The capacity clamp (\textsc{Clamp}) bounds $\|W_\ell\|_F$.
  Command sampling $u \sim U$ provides exploration.
\item \textcolor{caution}{\textbf{Value function}}: $|\phi|_\beta$
  is computed per-rollout, not estimated by a learned $V(s)$.
\end{itemize}

The pattern: every ``trick'' in model-free RL is a patch for a
missing gradient.
The Student-$t$ mollifier provides the gradient---through the
policy, not through the engine.
The patches become unnecessary.
\Cref{alg:loco} is not a new algorithm---it is what remains when
you \textcolor{caution}{delete the patches}.
\end{remark}

\section{Representation dynamics}\label{sec:representation}

The contact dynamics of \cref{sec:contact} trains a locomotion policy
on an execution graph with $12$ edges.
We now construct the dual of the agentic tower and show that the same
contact gradient flow, on different execution graphs, yields
text generation, image generation, and video generation as
instances of a single representation-learning paradigm.

\begin{definition}[Dual tower]\label{def:dual-tower}
The \emph{dual tower} of the agentic space
$\mathbf{L} = (L_0, L_1, L_2, L_3)$ (\cref{def:tower}) is
\[
  \mathbf{L}^*
  \;=\;
  \text{力}^* \;\oplus\; \text{立}^* \;\oplus\; \text{丽}^*,
\]
where:
\begin{enumerate}[label=\textbf{L\arabic*${}^*$}.]
  \item $\text{力}^* = L_0^*$:
  the dual of the state space.
  An element of $L_0^*$ is \emph{acted upon} physically:
  the medium as substrate (pixels, tokens, audio samples).
  \item $\text{立}^* = L_1^*$:
  the dual of the viable kernel.
  An element of $L_1^*$ is \emph{positioned} by external operations:
  structural coherence imposed (grammar, spatial consistency,
  temporal continuity).
  \item $\text{丽}^* = (L_2 \oplus L_3)^*$:
  the dual of the control-strategy bundle.
  An element of $(L_2 \oplus L_3)^*$ is \emph{represented} by
  the Subject's operations:
  meaning, style, and aesthetics as projected image.
\end{enumerate}
The three components are named by their Chinese homophones:
力~(force), 立~(stand), 丽~(beauty)---all pronounced~\emph{l\`\i}.
\end{definition}

The duality is grammatical: 力~acts, 力${}^*$~is acted upon.
立~stands, 立${}^*$~is positioned.
丽~creates beauty, 丽${}^*$~is made beautiful.
The operator~被 (passive voice marker) maps each component to its dual.

\begin{theorem}[Beauvoir representation]\label{thm:beauvoir}
The dual tower $\mathbf{L}^*$ is a faithful, irreducible
representation of the passive predicates on the agentic space.
Every predicate of the form ``$x$ is $f$-ed by~$b$'' for
$f \in \{\sigma, \partial, \circ, \varphi\}$ factors through
$\mathbf{L}^*$:
\begin{align*}
  \text{被}\,\sigma &\;\in\; \text{力}^*
    &&\text{(transported physically: 被恢复)}, \\
  \text{被}\,\partial &\;\in\; \text{立}^*
    &&\text{(cut / detected: 被看到)}, \\
  \text{被}\,\circ &\;\in\; \text{丽}^*
    &&\text{(composed into representation: 被表示)}, \\
  \text{被}\,\varphi &\;\in\; \text{丽}^*
    &&\text{(redefined by phase change: 被描述)}.
\end{align*}
\end{theorem}

\begin{proof}
The calculus is complete (\cref{prop:completeness}): every operation
on the agentic space decomposes into
$\{\sigma, \circ, \partial, \varphi\}$.
The dual of each operation is its passive form.
\textsc{Slide}~$\sigma$ is physical transport ($L_0$); its dual
被$\,\sigma$ acts on~$L_0^*$.
\textsc{Cut}~$\partial$ detects and removes ($L_1$: the boundary
of the viable kernel); its dual 被$\,\partial$ acts on~$L_1^*$.
\textsc{Compose}~$\circ$ and \textsc{Phase}~$\varphi$ build
representations and change regimes ($L_2 \oplus L_3$: control
and strategy); their duals act on~$(L_2 \oplus L_3)^*$.
By completeness, no predicate falls outside
$L_0^* \oplus L_1^* \oplus (L_2 \oplus L_3)^*$.
The decomposition is irreducible: removing any component removes
a calculus operation from the dual.
\end{proof}

\begin{theorem}[Camus absurdity]\label{thm:camus}
In any agentic space with knife dissipation $\gamma > 0$
(\cref{def:dissipation}), the absorbed energy at equilibrium
\[
  \mathcal{A}
  \;=\;
  \|z\| - |\phi|_{\mathrm{eq}}
  \;>\; 0.
\]
The flow deficit is strictly positive: the system always dissipates.
\end{theorem}

\begin{proof}
The contact gradient flow (\cref{def:contactflow}) reaches
equilibrium at $\partial|\phi|/\partial c(e) = \gamma(e)\,c(e)$
(\cref{thm:contactEL}).
The dissipation term $\gamma(e)\,c(e) > 0$ on bypass edges
prevents the capacities from reaching the flow-maximising assignment.
Therefore $|\phi|_{\mathrm{eq}} < \max|\phi| \leq \|z\|$ and
$\mathcal{A} > 0$.

The absorption is irreducible: setting $\gamma \to 0$ eliminates
dissipation but also eliminates the stability margin
$\lambda_1 > 0$ (\cref{rem:contact-stability}).
The system becomes unstable---small perturbations amplify rather
than decay.
The absurd is the price of stability.
\end{proof}

\begin{corollary}[Representation learning]\label{cor:replearn}
The contact gradient flow~\eqref{eq:contactflow} on the dual
tower $\mathbf{L}^*$ with objective $\mathcal{L} = -|\phi|$ is
representation learning:
\[
  \min_c \; \mathcal{A}(c)
  \quad\text{subject to}\quad
  c \in \mathbf{L}^*
  \;=\;
  \text{力}^* \oplus \text{立}^* \oplus \text{丽}^*.
\]
\cref{thm:beauvoir} identifies the representation space
($\mathbf{L}^*$).
\cref{thm:camus} identifies the objective ($\min \mathcal{A}$,
irreducible).
The contact gradient flow provides the dynamics.
\end{corollary}

\begin{definition}[Representation dynamics]\label{def:repdyn}
Let $G = (V, E, c)$ be an execution graph (\cref{def:exgraph}) on the
dual tower $\mathbf{L}^*$ (\cref{def:dual-tower}).
The \emph{representation dynamics} on $G$ is the contact gradient
flow~\eqref{eq:contactflow}:
\[
  \frac{dc(e)}{dt}
  \;=\;
  \frac{\partial|\phi|}{\partial c(e)}
  \;-\;
  \gamma(e)\,c(e),
  \qquad
  c(e) = \|W_e\|_F,
\]
where $|\phi|$ is the viability of the generated output
(\cref{cor:replearn}: $\mathcal{L} = -|\phi|$) and $\gamma$ is the
knife dissipation (\cref{def:dissipation}).
\end{definition}

\medskip
\noindent\fcolorbox{water}{water!5}{%
\begin{minipage}{\dimexpr\textwidth-2\fboxsep-2\fboxrule}
\smallskip
\textbf{\textcolor{water}{Why representation is contact dynamics.}}\;
The contact gradient flow~\eqref{eq:contactflow} is defined on any
execution graph with capacitated edges (\cref{def:exgraph}).
It does not distinguish whether the edges carry physical actuators
($c \in \mathbf{L}$) or neural weights
($c \in \mathbf{L}^*$).
Contact dynamics (\cref{sec:contact}) trains an agent to \emph{act}
(力\,立\,丽).
Representation dynamics trains a model to \emph{represent}
(力${}^*$\,立${}^*$\,丽${}^*$).
The operator~被 maps one to the other:
the same equation, the same flow, the same knife---on the dual tower.
\smallskip
\end{minipage}}
\medskip

\begin{remark}[Le Deuxi\`eme Sexe as tower]
\label{rem:beauvoir-tower}
Beauvoir's \emph{Le Deuxi\`eme Sexe}~\cite{beauvoir} is
structured as three parts that map to the dual tower:
\begin{center}
\begin{tabular}{@{}llll@{}}
\toprule
\textbf{Volume~I Part} & \textbf{l\`\i} & \textbf{Tower} &
\textbf{Content} \\
\midrule
I.\ Destin (Destiny) & 力 & $L_0^*$ &
  Biology, body, physical substrate \\
II.\ Histoire (History) & 立 & $L_1^*$ &
  Social position, establishment \\
III.\ Mythes (Myths) & 丽 & $(L_2 \oplus L_3)^*$ &
  Cultural representation, image \\
\bottomrule
\end{tabular}
\end{center}
The historical examples of \cref{app:secondsex} (蔡文姬, 花木兰)
illustrate the tower: at every stage, the Other's identity
decomposes as 被力~$\oplus$~被立~$\oplus$~被丽---physically
acted upon, socially positioned, culturally represented.
The knife between Subject and Other is the mean
(\cref{thm:meanfield}): a phase function, not an intrinsic
property.
\end{remark}

\begin{definition}[Generation execution graph]\label{def:gen-exgraph}
A \emph{generation execution graph} is an execution graph
$G_{\mathrm{gen}} = (V, E, c)$ (\cref{def:exgraph}) where:
\begin{itemize}
  \item $\kappa$ is the conditioning input
  (prompt, noise schedule, or previous frames);
  \item $\infty$ is the generated output
  (tokens, pixels, or video frames);
  \item the intermediate nodes $\{a_i\}$ are the network layers;
  \item the capacity $c(e)$ is $\|W_e\|_F$
  (\cref{def:neural-exgraph}).
\end{itemize}
The viable flow condition $|\phi| > 0$ becomes: the
generated output is coherent.
\end{definition}

\begin{definition}[Representation model]\label{def:repmodel}
A \emph{representation model} is a family of maps
$\{f_m\}_{m \in \mathcal{M}}$ on modalities
$\mathcal{M} = \{\text{text},\, \text{image},\, \text{video}\}$,
sharing a backbone $g_\theta$ on the dual tower:
\[
  f_m \;=\; h_m \circ g_\theta \circ \tau_m,
  \qquad m \in \mathcal{M},
\]
where $\tau_m$ is a modality-specific tokeniser
(embed tokens / patchify pixels / patchify-and-stack frames)
and $h_m$ is a modality-specific head
(predict next token / denoise patches / denoise-and-predict frames).
The backbone $g_\theta$ operates on
$\mathbf{L}^* = \text{力}^* \oplus \text{立}^* \oplus \text{丽}^*$
(\cref{def:dual-tower}) and is shared across all modalities.
\end{definition}

\begin{algorithm}[H]
\caption{Contact-dynamics training for a representation model}
\label{alg:repmodel}
\begin{algorithmic}[1]
\Require Modality set
  $\mathcal{M} = \{\text{text},\, \text{image},\, \text{video}\}$
\Require For each $m \in \mathcal{M}$:\;
  graph $G_m$ (\cref{def:gen-exgraph}),\;
  dataset $\mathcal{D}_m$,\;
  tokeniser $\tau_m$
\Require $\eta$ (learning rate),\;
  $\gamma$ (dissipation),\;
  $\beta$ (soft-min sharpness),\;
  $L_s$ (backbone depth),\; $n$ (width),\;
  $N$ (eval count)
\Statex
\State \textbf{Backbone:}\;
  $\{W_\ell, b_\ell\}_{\ell=1}^{L_s}
  \leftarrow \texttt{init}(L_s,\, n)$
  \Comment{shared;\; $c(e_\ell) = \|W_\ell\|_F$}
\For{$m \in \mathcal{M}$}
  \Comment{modality-specific heads}
  \State $\{W_\ell^{m}, b_\ell^{m}\}_{\ell=1}^{L_m}
    \leftarrow \texttt{init}(L_m,\, n_m)$
\EndFor
\Statex
\For{\textcolor{sword}{\textbf{phase}} $\in$
  \{力\,$(U\!=\!\text{structure})$,\;\,
   立\,$(U\!=\!\text{coherence})$,\;\,
   丽\,$(U\!=\!\text{semantics})$\}}
  \Comment{\textcolor{sword}{$\varphi$: curriculum}}
  \Repeat
    \State $m \sim \mathrm{Unif}(\mathcal{M})$;\;\;
      $(z,\, x^*) \sim \mathcal{D}_m$
      \Comment{sample modality $+$ data}
    \State $\tilde{z} \leftarrow \tau_m(z)$
      \Comment{tokenise into $\mathbf{L}^*$}
    \State \textcolor{water}{\textsc{Forward}:}\;
      $\hat{x} \leftarrow h_m\bigl(g_\theta(\tilde{z})\bigr)$
      \Comment{\textcolor{water}{水: backbone $+$ head $m$}}
    \State \textcolor{water}{\textsc{Flow}:}\;
      $|\phi|_\beta^{(m)} \leftarrow
      \mathrm{soft\text{-}min}_\beta\bigl\{
      |\phi|_t^{(m)}
      \bigr\}_{t}$
      \Comment{\textcolor{water}{水: modality viability}}
    \State \textcolor{water}{\textsc{Backward}:}\;
      $\{\nabla W,\, \nabla b,\,
       \nabla W^{m},\, \nabla b^{m}\}
      \leftarrow \textsc{Backward}(\nabla_{\hat{x}}\, |\phi|_\beta^{(m)})$
      \Comment{\cref{alg:backward}}
    \State $W_\ell \leftarrow W_\ell
      + \eta\bigl[\textcolor{water}{\nabla W_\ell}
      - \textcolor{knife}{\gamma\, W_\ell}\bigr]$;\;\;
      $b_\ell \leftarrow b_\ell
      + \eta\,\textcolor{water}{\nabla b_\ell}$
      \Comment{backbone: Eq.~\eqref{eq:contactflow}}
    \State $W_\ell^{m} \leftarrow W_\ell^{m}
      + \eta\bigl[\textcolor{water}{\nabla W_\ell^{m}}
      - \textcolor{knife}{\gamma\, W_\ell^{m}}\bigr]$;\;\;
      $b_\ell^{m} \leftarrow b_\ell^{m}
      + \eta\,\textcolor{water}{\nabla b_\ell^{m}}$
      \Comment{head $m$: Eq.~\eqref{eq:contactflow}}
    \State \textcolor{knife}{\textsc{Clamp}:}\;
      $\|W_\ell\|_F \leftarrow
      \min\bigl(\|W_\ell\|_F,\;
      c_{\max}(e_\ell)\bigr)$
      for all $\ell$
      \Comment{\textcolor{knife}{刀: capacity}}
    \If{$|\phi|_\beta^{(m)} \le 0$}
      \Comment{output incoherent in modality $m$}
      \State \textcolor{sword}{\textsc{Reset}:}\;
        head only: $\{W^m, b^m\} \leftarrow \texttt{init}(L_m, n_m)$
        \Comment{\textcolor{sword}{$\varphi$: recover modality}}
    \EndIf
  \Until{$\forall\, m\!:\; \max|\phi|^{(m)} = \min|C^{(m)}|$
    for $N$ consecutive evaluations}
\EndFor
\Statex
\Ensure Representation model:
  $f_m = h_m \circ g_\theta \circ \tau_m$
  for all $m \in \mathcal{M}$,
  backbone on $\mathbf{L}^*$
\end{algorithmic}
\end{algorithm}

\begin{remark}[Three bodies, one backbone]
\label{rem:repmodel-bodies}
The modality-specific components of \cref{alg:repmodel} are:
\begin{center}
\small
\renewcommand{\arraystretch}{1.25}
\begin{tabular}{@{}l lll@{}}
\toprule
& \textbf{Text} & \textbf{Image} & \textbf{Video} \\
\midrule
$G_m$ (topology)
  & sequential & 2D patches & 2D patches $\times\, T$ \\
$\kappa_m$ (input)
  & context tokens & noise $+$ prompt
  & noise $+$ prompt $+$ prev.\ frames \\
$\tau_m$ (tokeniser)
  & BPE embedding & patchify $+$ linear
  & patchify $+$ linear $+$ temporal \\
$h_m$ (head)
  & softmax over $V$ & depatchify to $\R^{H \times W \times 3}$
  & depatchify $\times\, T$ \\
$|\phi|^{(m)}$ (viability)
  & $\log p(x_t^* \mid x_{<t})$
  & $\varepsilon - \|\hat{x} - x^*\|$
  & $\min(\text{spatial},\; \lambda_T \!\cdot\! \text{temporal})$ \\
$\gamma^{(m)}$ (knife)
  & weight decay & guidance scale
  & temporal regularisation \\
\textsc{Phase}
  & char $\to$ word $\to$ doc
  & noise $\to$ coarse $\to$ fine
  & frame $\to$ clip $\to$ video \\
\bottomrule
\end{tabular}
\end{center}
The backbone $g_\theta$ sees none of these distinctions.
It operates on $\mathbf{L}^*$: medium tokens (力${}^*$),
structural coherence (立${}^*$), semantic content (丽${}^*$).
The modality is invisible at the representation level---this is
why a single backbone suffices.
\end{remark}

\begin{remark}[Representation model $=$ locomotion on every body]
\label{rem:repmodel-loco}
Compare \cref{alg:repmodel} with \cref{alg:loco}:
\begin{center}
\renewcommand{\arraystretch}{1.25}
\begin{tabular}{@{}lll@{}}
\toprule
& \textbf{Locomotion (\cref{alg:loco})} &
\textbf{Representation model (\cref{alg:repmodel})} \\
\midrule
Body & quadruped ($12$ joints) &
  text / image / video \\
Standing & $h > h_{\min}$ (do not fall) &
  $|\phi|^{(m)} > 0$ (stay coherent) \\
Engine & MuJoCo (physics) &
  autodiff (shared backbone $g_\theta$) \\
Knife & joint limits ($c_{\max}$) &
  capacity bounds ($c_{\max}$) \\
Curriculum & Stand $\to$ Walk $\to$ Work &
  力 $\to$ 立 $\to$ 丽 \\
\bottomrule
\end{tabular}
\end{center}
A robot that falls is a text that is incoherent is an image that is
noise is a video that flickers.
The contact gradient flow is the same.
The body is different.
The representation model learns all bodies at once---三位一体:
one backbone on $\mathbf{L}^*$, three modalities, one gradient flow.
\end{remark}

\begin{remark}[LoRA $=$ low-rank task head]
\label{rem:lora}
Low-Rank Adaptation (LoRA) is a task head in the sense of
\cref{def:repmodel}.
Given a pre-trained backbone with layer weights
$\{W_\ell\}_{\ell=1}^L$, a LoRA adapter of rank~$r$ adds
\[
  \Delta W_\ell \;=\; B_\ell\, A_\ell,
  \qquad B_\ell \in \R^{d \times r},\;\;
  A_\ell \in \R^{r \times d},\;\;
  r \ll d,
\]
at each layer.
The backbone $W_\ell$ is frozen; the adapter $\Delta W_\ell$ is trained.
In the dual tower:
$W_\ell$ is the representation ($\mathbf{L}^*$),
$\Delta W_\ell$ is the task function
($\mathcal{H}_{\mathcal{T}}$, \cref{def:task-rkhs}),
and the rank~$r = c_{\max}(e_\ell^{\mathrm{task}})$ is the knife---it
bounds how far the task can deviate from the representation.
\end{remark}

\begin{definition}[Code metric space]
\label{def:code-metric}
The \emph{code metric space} is the agentic tower
(\cref{def:tower}) unbundled to four levels, each equipped with
a viability metric.
The general dual tower compresses
$L_2^* \oplus L_3^* \to \text{丽}^*$;
the code metric space resolves them into correctness ($L_2^*$)
and performance ($L_3^*$), yielding four phases---力, 立, 丽(正), 丽(快).
Correctness and performance are both sub-types of
丽~(value)---all three characters pronounced \emph{l\`\i}:
\begin{itemize}
  \item $\phi_{\mathrm{topo}}$ (\textbf{力${}^*$}, data topology,
  指针序):\;
  source forms a valid abstract syntax tree, all imports resolve,
  all names are bound.
  The data-structure graph (pointer\,/\,reference topology) is
  well-formed.
  Binary: $\phi_{\mathrm{topo}} \in \{0,\, 1\}$.

  \item $\phi_{\mathrm{safe}}$ (\textbf{立${}^*$}, memory safety,
  内存安全):\;
  program executes without crashing---no segmentation fault,
  no uncaught exception, no resource leak, no use-after-free.
  Binary: $\phi_{\mathrm{safe}} \in \{0,\, 1\}$.

  \item $\phi_{\mathrm{correct}}$ (\textbf{丽(正)${}^*$}, correctness,
  正确):\;
  output matches specification---all assertions hold, all tests pass.
  Ratio: $\phi_{\mathrm{correct}} =
  \text{tests passed}\,/\,\text{tests total} \in [0,\, 1]$.

  \item $\phi_{\mathrm{perf}}$ (\textbf{丽(快)${}^*$}, performance,
  快):\;
  program completes within time and space budgets.
  Continuous: $\phi_{\mathrm{perf}} =
  \max\!\bigl(0,\; 1 - t_{\mathrm{run}} / t_{\mathrm{budget}}\bigr)
  \in [0,\, 1]$.
\end{itemize}
The ordering is a prerequisite chain (each gates the next):
\[
  \phi_{\mathrm{topo}} \;\to\;
  \phi_{\mathrm{safe}} \;\to\;
  \phi_{\mathrm{correct}} \;\to\;
  \phi_{\mathrm{perf}}.
\]
You cannot profile code that crashes;
you cannot test code that does not parse.
The combined viability is
\[
  |\phi|_{\mathrm{code}} \;=\;
  \mathrm{soft\text{-}min}_\beta\bigl\{
  \phi_{\mathrm{topo}},\;
  \phi_{\mathrm{safe}},\;
  \phi_{\mathrm{correct}},\;
  \phi_{\mathrm{perf}}
  \bigr\}.
\]
Every programming language has all four levels.
Languages differ in which levels are enforced automatically:
\begin{center}
\renewcommand{\arraystretch}{1.25}
\resizebox{\linewidth}{!}{%
\begin{tabular}{@{}l llll@{}}
\toprule
\textbf{Language}
  & \textbf{力 (topo)} & \textbf{立 (safe)}
  & \textbf{丽(正) (correct)} & \textbf{丽(快) (perf)} \\
\midrule
Python
  & \texttt{ast.parse()}
  & \texttt{python f.py} exits $0$
  & \texttt{pytest} passes
  & \texttt{timeit} $\leq$ budget \\
C
  & \texttt{gcc -fsyntax-only}
  & Valgrind: no errors
  & tests pass
  & \texttt{perf stat} $\leq$ budget \\
Rust
  & \texttt{cargo check}
  & borrow checker (compile-time!)
  & \texttt{cargo test}
  & \texttt{cargo bench} $\leq$ budget \\
CUDA
  & \texttt{nvcc -c}
  & \texttt{compute-sanitizer}: 0 errors
  & trajectory converges
  & \texttt{nsight} $\leq$ budget \\
\bottomrule
\end{tabular}}%
\end{center}
\end{definition}

Fine-tuning is representation dynamics at school.
The representation model (\cref{def:repmodel}) learns $\mathbf{L}^*$
from mixed-modality data (\cref{alg:repmodel}).
Fine-tuning attaches a LoRA adapter (\cref{rem:lora}) to the frozen
backbone and trains it on task-specific data in the 学堂
(\cref{def:task-rkhs}).
For code generation, the code metric space (\cref{def:code-metric})
provides four viability metrics trained in prerequisite order.
\Cref{alg:finetune} is self-contained: an ML engineer with a
pre-trained checkpoint, a GPU server, a sandbox, and the four
viability metrics can execute it directly.

\begin{algorithm}[H]
\caption{Fine-tuning a representation model (学堂)}
\label{alg:finetune}
\begin{algorithmic}[1]
\Require Checkpoint $\theta_0$
  (e.g.\ \texttt{deepseek-coder-v2});\;
  tokenizer $\tau$;\;
  GPU server $\mathcal{E}_{\mathrm{GPU}}$
\Require Sandbox $\mathcal{E}_{\mathrm{sandbox}}$
  (e.g.\ Docker, \texttt{venv}, \texttt{nsjail})
\Require Task $\mathcal{T}$:\;
  dataset $\mathcal{D} = \{(z_i, x_i^*)\}$ (Q-A mode)
  or judge $J$ (court mode)
\Require Viability metrics
  $\phi_1 \to \cdots \to \phi_K$
  in prerequisite order
  (\cref{def:code-metric}: $K\!=\!4$)
\Require $r$ (LoRA rank $=$ knife),\;
  $\alpha$ (LoRA scaling),\;
  $\eta$ (learning rate),\;
  $\gamma$ (weight decay),\;
  $\beta$ (soft-min sharpness),\;
  $N$ (eval count)
\Statex
\State $\{W_\ell\}_{\ell=1}^{L}
  \leftarrow \texttt{load}(\theta_0)$;\;\;
  \texttt{freeze}($W$)
  \Comment{backbone on $\mathbf{L}^*$: frozen}
\State $\{B_\ell \!=\! 0,\;
  A_\ell \!\sim\! \mathcal{N}(0, \sigma^2)
  \}_{\ell=1}^{L}$
  \Comment{LoRA init: $\Delta W_\ell = B_\ell A_\ell = 0$}
\Statex
\For{\textcolor{sword}{\textbf{phase}} $\in$
  \{力\,$(\phi_{\mathrm{topo}})$,\;\,
   立\,$(\phi_{\mathrm{safe}})$,\;\,
   丽(正)\,$(\phi_{\mathrm{correct}})$,\;\,
   丽(快)\,$(\phi_{\mathrm{perf}})$\}}
  \Comment{\textcolor{sword}{$\varphi$: curriculum (\cref{def:code-metric})}}
  \Repeat
    \State $(z,\, x^*) \sim \mathcal{D}$\; or\; $z \sim \mathcal{T}$
      \Comment{Q-A: input $+$ target;\; court: input only}
    \State \textcolor{water}{\textsc{Forward}:}\;
      $\hat{x} \leftarrow \textsc{Forward}\bigl(\tau(z);\;
      \{W_\ell + \tfrac{\alpha}{r}\,
      \textcolor{water}{B_\ell A_\ell}\}\bigr)$
      \Comment{\cref{alg:forward}; backbone $+$ adapter}
    \State \textcolor{water}{\textsc{Execute}:}\;
      $\mathrm{result} \leftarrow
      \mathcal{E}_{\mathrm{sandbox}}.\texttt{run}(\hat{x})$
      \Comment{\textcolor{water}{水:} run generated code in sandbox}
    \State \textcolor{water}{\textsc{Evaluate}:}\;
      $|\phi|_\beta \leftarrow
      \mathrm{soft\text{-}min}_\beta\bigl\{
      \phi_k(\mathrm{result}_t)
      \bigr\}_{t}$
      \Comment{\textcolor{water}{水:} phase-$k$ metric
      (\cref{def:code-metric})}
    \State \textcolor{water}{\textsc{Backward}:}\;
      $\{\nabla B_\ell,\, \nabla A_\ell\}_{\ell=1}^{L}
      \leftarrow \textsc{Backward}(\nabla_{\hat{x}}\,
      |\phi|_\beta)$
      \Comment{\cref{alg:backward};\; $W$ frozen}
    \State $B_\ell \leftarrow B_\ell
      + \eta\,\textcolor{water}{\nabla B_\ell}$;\;\;
      $A_\ell \leftarrow A_\ell
      + \eta\bigl[\textcolor{water}{\nabla A_\ell}
      - \textcolor{knife}{\gamma\, A_\ell}\bigr]$
      \Comment{Eq.~\eqref{eq:contactflow} on adapter}
    \State \textcolor{knife}{\textsc{Clamp}:}\;
      $\|B_\ell A_\ell\|_F \leftarrow
      \min\bigl(\|B_\ell A_\ell\|_F,\;
      c_{\max}\bigr)$
      \Comment{\textcolor{knife}{刀: adapter capacity}}
    \If{$|\phi|_\beta \le 0$}
      \Comment{output incoherent}
      \State \textcolor{sword}{\textsc{Reset}:}\;
        $B_\ell \leftarrow 0$;\;
        $A_\ell \sim \mathcal{N}(0, \sigma^2)$
        \Comment{\textcolor{sword}{$\varphi$: re-init adapter}}
    \EndIf
  \Until{$\max|\phi| = \min|C|$ for $N$ consecutive evaluations}
\EndFor
\Statex
\Ensure Adapted model:\;
  $W_\ell^{\mathrm{out}} = W_\ell
  + \tfrac{\alpha}{r}\,B_\ell A_\ell$;\;\;
  \texttt{merge\_and\_upload}
\end{algorithmic}
\end{algorithm}

\begin{figure}[H]
\centering
\begin{tikzpicture}[
  >=stealth,
  box/.style={draw, rounded corners=2pt, minimum width=5.2cm,
              minimum height=0.55cm, align=center, font=\small},
  io/.style={box, fill=black!5},
  wt/.style={box, fill=water!10},
  arr/.style={->, thick, black!60},
]
\node[io] (init) at (0,0)
  {Checkpoint $\theta_0$;\;\;\texttt{freeze}($W$);\;\;init LoRA};
\node[draw, fill=sword!12, rounded corners=2pt,
  font=\small\bfseries, minimum width=5.2cm,
  minimum height=0.5cm, align=center] (phase) at (0,-1.0)
  {\textcolor{sword}{$\varphi$:}\;
   力 $\to$ 立 $\to$ 丽(正) $\to$ 丽(快)};
\node[wt] (fwd)  at (0,-2.1)
  {\textcolor{water}{\textsc{Forward}}:\;
   $\hat{x} = f\!\bigl(\tau(z);\;W\!+\!\tfrac{\alpha}{r}BA\bigr)$};
\node[wt] (exec) at (0,-2.9)
  {\textcolor{water}{\textsc{Execute}}:\;
   sandbox.\texttt{run}($\hat{x}$)};
\node[wt] (eval) at (0,-3.7)
  {\textcolor{water}{\textsc{Evaluate}}:\;
   $|\phi|_\beta = \mathrm{soft\text{-}min}\{\phi_k\}$};
\node[wt] (bwd)  at (0,-4.5)
  {\textcolor{water}{\textsc{Backward}}:\;
   $\nabla B,\,\nabla A$\;($W$ frozen)};
\node[wt] (upd)  at (0,-5.3)
  {\textsc{Update}:\;Eq.~\eqref{eq:contactflow} on $B,A$;\;\;
   \textcolor{knife}{\textsc{Clamp}}};
\node[io] (merge) at (0,-6.5)
  {Merge:\;$W^{\mathrm{out}} = W + \tfrac{\alpha}{r}BA$;\;\;
   \texttt{upload}};
\draw[arr] (init)  -- (phase);
\draw[arr] (phase) -- (fwd);
\draw[arr] (fwd)   -- (exec);
\draw[arr] (exec)  -- (eval);
\draw[arr] (eval)  -- (bwd);
\draw[arr] (bwd)   -- (upd);
\draw[arr] (upd)   -- node[right,font=\scriptsize]{converged} (merge);
\draw[arr, knife]  (upd.east) -- ++(1.3,0) |-
  node[right, pos=0.25, font=\scriptsize, color=knife]{repeat}
  (fwd.east);
\draw[dashed, rounded corners=3pt, black!25]
  (-3.0,-1.7) rectangle (3.5,-5.65);
\node[font=\tiny, black!40, anchor=south east]
  at (3.5,-1.7) {per phase};
\end{tikzpicture}
\caption{Fine-tuning pipeline (\cref{alg:finetune}).
  Outer loop: four-phase curriculum
  (力~$\to$~立~$\to$~丽(正)~$\to$~丽(快)).
  Inner loop: contact gradient
  flow~\eqref{eq:contactflow} on the LoRA adapter.}
\label{fig:finetune-flow}
\end{figure}

\begin{remark}[Validation chain: checkpoint $+$ sandbox $+$ metrics
$\Rightarrow$ coder]
\label{rem:finetune-chain}
Every input to \cref{alg:finetune} is extracted from four artifacts:
a pre-trained checkpoint, a code sandbox, a task dataset (or judge),
and user-defined viability metrics (\cref{def:code-metric}).
No additional assumptions.
\begin{center}
\begin{tabular}{@{}lll@{}}
\toprule
\textbf{Algorithm input} & \textbf{Source} & \textbf{Extraction} \\
\midrule
Backbone $\{W_\ell\}$ ($L$ layers)
  & Checkpoint & \texttt{model.safetensors} \\
Tokenizer $\tau$
  & Checkpoint & \texttt{tokenizer.json} \\
Architecture ($d$, $L$, heads)
  & Checkpoint & \texttt{config.json} \\
LoRA rank $r$, scaling $\alpha$
  & User & Capacity budget ($r = 16$ typical) \\
Task data $\mathcal{D}$ or judge $J$
  & User & Q-A pairs or LLM evaluator \\
$\phi_{\mathrm{topo}},\, \phi_{\mathrm{safe}},\,
\phi_{\mathrm{correct}},\, \phi_{\mathrm{perf}}$
  & User & Viability metrics (\cref{def:code-metric}) \\
Code execution
  & Sandbox & Docker\,/\,\texttt{venv}\,/\,\texttt{nsjail} \\
Gradient $\nabla_{\hat{x}} |\phi|$
  & GPU server & Autodiff (PyTorch\,/\,JAX) \\
\bottomrule
\end{tabular}
\end{center}
The four-phase curriculum parallels the locomotion curriculum
(\cref{alg:loco})---with one refinement: 丽${}^*$ unbundles
into correctness ($L_2^*$) and performance ($L_3^*$):
\begin{center}
\renewcommand{\arraystretch}{1.25}
\begin{tabular}{@{}lllll@{}}
\toprule
\textbf{Phase} & \textbf{Tower} & \textbf{Locomotion} &
\textbf{Fine-tuning} & \textbf{Coding} \\
\midrule
力 & $L_0^*$ & $h > h_{\min}$ (don't fall)
  & $\phi_{\mathrm{topo}} > 0$ & AST valid \\
立 & $L_1^*$ & stable gait
  & $\phi_{\mathrm{safe}} > 0$ & no crash \\
丽(正) & $L_2^*$ & reach target
  & $\phi_{\mathrm{correct}} > 0$ & tests pass \\
丽(快) & $L_3^*$ & energy-efficient
  & $\phi_{\mathrm{perf}} > 0$ & within budget \\
\bottomrule
\end{tabular}
\end{center}
A robot that falls is code that does not parse.
A robot that stumbles is code that crashes.
A robot that walks is code that passes its tests.
A robot that runs is code that runs fast.

Two evaluation modes (\cref{def:task-rkhs}):
\begin{center}
\renewcommand{\arraystretch}{1.25}
\begin{tabular}{@{}lll@{}}
\toprule
& \textbf{Q-A mode} & \textbf{Court mode} \\
\midrule
Data & $(z, x^*)$ pairs & inputs $z$ only \\
$|\phi|$ & $\varepsilon - \|\hat{x} - x^*\|$ & $J(z, \hat{x})$ \\
Judge & ground truth & LLM evaluator \\
Use & code with test suite & open-ended generation \\
\bottomrule
\end{tabular}
\end{center}
Given a checkpoint, a sandbox, a task dataset, and four viability
metrics, every step of \cref{alg:finetune} is mechanically executable.
The model already knows $\mathbf{L}^*$.
School teaches it the task---力, 立, 丽(正), 丽(快)---four phases, one flow.
\end{remark}

\begin{remark}[CUDA TrajOpt: the loop closes]
\label{rem:cuda-loop}
Add CUDA to the language table (\cref{def:code-metric}):
\texttt{nvcc} compiles (力),
\texttt{compute-sanitizer} catches memory errors (立),
the trajectory converges (丽(正)),
the kernel meets its real-time budget (丽(快)).
The fine-tuned coder (\cref{alg:finetune}) writes
CUDA trajectory-optimisation kernels.
These kernels solve the same contact gradient
flow~\eqref{eq:contactflow} that trained the coder---on a physical
body instead of a token sequence.
The contact gradient flow appears at every level:
\begin{enumerate}
  \item \emph{Representation model} (\cref{alg:repmodel}):
  backbone $g_\theta$ learns $\mathbf{L}^*$ from mixed-modality data.
  \item \emph{Fine-tuning} (\cref{alg:finetune}):
  LoRA adapter learns the coding task on $\mathbf{L}^*$.
  \item \emph{Locomotion} (\cref{alg:loco}):
  the CUDA kernel output by the fine-tuned model
  controls the robot via the same flow.
\end{enumerate}
Level~2 produces the code that implements level~3.
The equation writes itself.
\end{remark}

\begin{remark}[For ML engineers: there is no ``agent'']
\label{rem:no-agent}
You do not train an \emph{agent}.
You train a model to produce
\textcolor{water}{perfect code}---one piece at a time.

Each piece must pass four gates in prerequisite order:
\begin{enumerate}
  \item[\textcolor{sword}{力}] It \textbf{parses}.
  \quad $\phi_{\mathrm{topo}} = 1$.\quad
  \textcolor{knife}{Fail $\Rightarrow$ \textsc{Reset}}.
  \item[\textcolor{sword}{立}] It \textbf{runs}.
  \quad $\phi_{\mathrm{safe}} = 1$.\quad
  \textcolor{knife}{Crash $\Rightarrow$ \textsc{Reset}}.
  \item[\textcolor{sword}{丽\textsuperscript{正}}] It is \textbf{correct}.
  \quad $\phi_{\mathrm{correct}} \in [0,1]$.\quad
  \textcolor{knife}{Wrong output $\Rightarrow$ gradient}.
  \item[\textcolor{sword}{丽\textsuperscript{快}}] It is \textbf{fast}.
  \quad $\phi_{\mathrm{perf}} \in [0,1]$.\quad
  \textcolor{knife}{Too slow $\Rightarrow$ gradient}.
\end{enumerate}
A piece that fails gates 1--2 is
\textcolor{knife}{killed}: $|\phi|_\beta \leq 0
\;\Rightarrow\;$\textcolor{knife}{\textsc{Reset}}
(\cref{alg:finetune}, line~13).
A piece that passes all four is
\textcolor{water}{viable}: $|\phi|_\beta > 0$.

The dataset is not dialogues.
Not trajectories.
Not reward signals.
It is \textcolor{water}{$(z, x^*)$ pairs}:
$z$ is a specification, $x^*$ is code that satisfies it.
Court mode (\cref{def:task-rkhs}):
a judge $J(z, \hat{x})$ replaces $x^*$ when ground truth is
unavailable.

There is no reinforcement learning in \cref{alg:finetune}.
There is no reward model.
There is a \textcolor{water}{flow}~\eqref{eq:contactflow},
a \textcolor{knife}{knife} ($\gamma > 0$),
and a \textcolor{sword}{curriculum}
(力~$\to$~立~$\to$~丽(正)~$\to$~丽(快)).
The training loop is supervised:
\textcolor{water}{forward},
\textcolor{water}{evaluate},
\textcolor{water}{backward},
\textcolor{knife}{clamp}.
Four words.
The ``agent'' is what happens \emph{after} training,
when the model generates
\textcolor{water}{enough viable code} to solve a task
that no single piece covers.
That is \textcolor{water}{composition~($\circ$)}, not training.
Training produces pieces.
Composition produces agents.
\end{remark}

\begin{algorithm}[H]
\caption{Code evaluation --- the judge protocol}
\label{alg:code-eval}
\begin{algorithmic}[1]
\Require Specification $z$ (natural language or formal)
\Require Generated code $\hat{x}$
  (output of \cref{alg:finetune})
\Require Sandbox $\mathcal{E}_{\mathrm{sandbox}}$;\;
  time budget $t_{\mathrm{budget}}$
\Require Judge $J$ (external LLM, e.g.\ Gemini-Pro ---
  \emph{must differ from generator})
\Statex
\State \textcolor{sword}{\textbf{Gate\;力}}
  (automated, \textcolor{water}{0 API calls}):
\State \quad $\phi_{\mathrm{topo}} \leftarrow
  \mathbf{1}\bigl[\texttt{parse}(\hat{x})\;\text{succeeds}\bigr]$
  \Comment{compiler / \texttt{ast.parse} / \texttt{nvcc -c}}
\If{$\phi_{\mathrm{topo}} = 0$}
  \textcolor{knife}{\Return} $|\phi|_{\mathrm{code}} = 0$
  \Comment{\textcolor{knife}{killed}: does not parse}
\EndIf
\Statex
\State \textcolor{sword}{\textbf{Gate\;立}}
  (automated, \textcolor{water}{0 API calls}):
\State \quad $(\mathrm{result},\; t_{\mathrm{run}}) \leftarrow
  \mathcal{E}_{\mathrm{sandbox}}.\texttt{run}(
  \hat{x},\; t_{\mathrm{budget}})$
\State \quad $\phi_{\mathrm{safe}} \leftarrow
  \mathbf{1}\bigl[\text{exit code} = 0\bigr]$
  \Comment{no crash, no leak, no timeout}
\If{$\phi_{\mathrm{safe}} = 0$}
  \textcolor{knife}{\Return} $|\phi|_{\mathrm{code}} = 0$
  \Comment{\textcolor{knife}{killed}: runtime crash}
\EndIf
\Statex
\State \textcolor{sword}{\textbf{Gate\;丽(正)}}
  (judge, \textcolor{caution}{1 API call}):
\State \quad $\phi_{\mathrm{correct}} \leftarrow
  J\!\bigl(z,\;\hat{x},\;\mathrm{result}\bigr)
  \;\in [0,1]$
  \Comment{``does output satisfy spec $z$?''}
\Statex
\State \textcolor{sword}{\textbf{Gate\;丽(快)}}
  (profiler, \textcolor{water}{0 API calls}):
\State \quad $\phi_{\mathrm{perf}} \leftarrow
  \max\!\bigl(0,\;\, 1 - t_{\mathrm{run}}\,/\,
  t_{\mathrm{budget}}\bigr)$
  \Comment{wall-clock; no judge needed}
\Statex
\State \textcolor{water}{$|\phi|_{\mathrm{code}}$} $\leftarrow
  \mathrm{soft\text{-}min}_\beta\bigl\{
  \phi_{\mathrm{topo}},\;
  \phi_{\mathrm{safe}},\;
  \phi_{\mathrm{correct}},\;
  \phi_{\mathrm{perf}}\bigr\}$
  \Comment{\cref{def:code-metric}}
\Statex
\Ensure \textcolor{water}{$|\phi|_{\mathrm{code}} \in [0,1]$}:\;
  plug into \cref{alg:finetune}, line~8
\end{algorithmic}
\end{algorithm}

\begin{remark}[The judge is not the teacher]
\label{rem:judge}
The judge $J$ in \cref{alg:code-eval} is \emph{not} the generator.
If \textcolor{water}{DeepSeek} generates, \textcolor{caution}{Gemini}
evaluates.
If \textcolor{water}{Gemini} generates, \textcolor{caution}{DeepSeek}
evaluates.
The adversarial independence is structural:
the court (\cref{def:task-rkhs}) requires a judge
who did not write the code.

The prerequisite chain saves money.
Gates \textcolor{sword}{力} and \textcolor{sword}{立} are
\textcolor{water}{free}:
compilers and sandboxes cost \textcolor{water}{zero} API calls.
Code that does not parse or crashes is
\textcolor{knife}{killed} before the judge sees it.
Only code that parses \emph{and} runs reaches
Gate~\textcolor{sword}{丽(正)}---\textcolor{caution}{one API call}
per surviving piece.
Gate~\textcolor{sword}{丽(快)} is the profiler:
wall-clock time does not require a judge.

The cost structure:
\begin{center}
\renewcommand{\arraystretch}{1.25}
\begin{tabular}{@{}llll@{}}
\toprule
\textbf{Gate} & \textbf{Checker} & \textbf{Cost} &
\textbf{Kills} \\
\midrule
\textcolor{sword}{力} & compiler / parser
  & \textcolor{water}{0 API calls}
  & syntax errors \\
\textcolor{sword}{立} & sandbox
  & \textcolor{water}{0 API calls}
  & crashes, leaks, timeouts \\
\textcolor{sword}{丽(正)} & judge $J$
  & \textcolor{caution}{1 API call}
  & wrong output \\
\textcolor{sword}{丽(快)} & profiler
  & \textcolor{water}{0 API calls}
  & slow code \\
\bottomrule
\end{tabular}
\end{center}
The engineer provides the specification $z$.
The compiler checks \textcolor{sword}{力}.
The sandbox checks \textcolor{sword}{立}.
The judge checks \textcolor{sword}{丽(正)}.
The profiler checks \textcolor{sword}{丽(快)}.
The engineer does not need to know what perfect code looks like.
The \textcolor{water}{pipeline} knows.
\end{remark}

\begin{algorithm}[H]
\caption{Sandbox execution --- the physics engine for code}
\label{alg:sandbox}
\begin{algorithmic}[1]
\Require Generated code $\hat{x}$
  (from \cref{alg:finetune}, line~6)
\Require Resource limits:\;
  $t_{\max}$ (time),\; $m_{\max}$ (memory),\;
  syscall whitelist $\mathcal{S}$
\Require Runtime: Docker, \texttt{nsjail}, or \texttt{venv}
\Statex
\State \textcolor{sword}{\textsc{Build}}:\;
  $\mathcal{C} \leftarrow \texttt{container.create}\bigl(
  \mathrm{image},\;
  t_{\max},\;
  m_{\max},\;
  \texttt{net=none},\;
  \texttt{fs=read\text{-}only},\;
  \mathcal{S}\bigr)$
  \Comment{isolated jail}
\State \textcolor{water}{\textsc{Inject}}:\;
  $\texttt{write}\bigl(
  \mathcal{C}\texttt{:/sandbox/main},\;
  \hat{x}\bigr)$
  \Comment{code $\to$ sandbox}
\State \textcolor{water}{\textsc{Execute}}:\;
  $(\texttt{stdout},\;\texttt{stderr},\;
  \texttt{exit\_code},\;t_{\mathrm{run}})
  \leftarrow \mathcal{C}.\texttt{run}\bigl(
  \texttt{/sandbox/main},\;
  t_{\max}\bigr)$
  \Comment{run with limits}
\State \textcolor{water}{\textsc{Extract}}:\;
  $\mathrm{result} \leftarrow
  \texttt{parse}(\texttt{stdout})$;\;\;
  $\mathrm{crash} \leftarrow
  (\texttt{exit\_code} \neq 0)$
  \Comment{structured output}
\State \textcolor{knife}{\textsc{Destroy}}:\;
  $\mathcal{C}.\texttt{kill}()$;\;\;
  reclaim all resources
  \Comment{\textcolor{knife}{no persistent state}}
\Statex
\Ensure $(\mathrm{result},\;
  t_{\mathrm{run}},\;
  \texttt{exit\_code},\;
  \texttt{stdout},\;
  \texttt{stderr})$
  for \cref{alg:code-eval}, lines~6--7
\end{algorithmic}
\end{algorithm}

\begin{remark}[Sandbox $=$ MuJoCo]
\label{rem:sandbox-mujoco}
\Cref{alg:sandbox} is MuJoCo for code.
The isomorphism is line-by-line:
\begin{center}
\renewcommand{\arraystretch}{1.25}
\begin{tabular}{@{}lll@{}}
\toprule
& \textbf{Locomotion (MuJoCo)} &
\textbf{Code (sandbox)} \\
\midrule
\textsc{Build}
  & load XML model
  & create container \\
\textsc{Inject}
  & apply joint torques
  & write code to filesystem \\
\textsc{Execute}
  & step physics ($\Delta t$)
  & run program ($t_{\max}$) \\
\textsc{Extract}
  & read sensors
  & read stdout, exit code, timing \\
\textsc{Destroy}
  & reset simulation
  & kill container \\
\midrule
\textcolor{knife}{Isolation}
  & joint limits, ground plane
  & no network, read-only fs \\
\textcolor{knife}{Gravity}
  & $9.81\;\mathrm{m/s^2}$
  & time limit $t_{\max}$ \\
\textcolor{knife}{Ground contact}
  & collision detection
  & memory limit $m_{\max}$ \\
\textcolor{knife}{$c_{\max}$}
  & joint torque bound
  & syscall whitelist $\mathcal{S}$ \\
\bottomrule
\end{tabular}
\end{center}
A robot cannot fly through the floor.
Code cannot escape the sandbox.
The physics is different.
The \textcolor{knife}{knife} is the same.
\end{remark}

\begin{remark}[Sinking bound: 地面不能变成水]
\label{rem:sinking-bound}
MuJoCo uses a \emph{soft contact} model: the ground exerts a
spring-damper force $F = k\,\delta + b\,\dot\delta$ proportional
to penetration depth $\delta$.
For small $\delta$ (a few millimetres), this approximation is
accurate.
For large $\delta$, the model breaks: forces explode,
bodies tunnel through the ground plane, and the simulation
becomes physically meaningless.

In the viability geometry (\cref{sec:viab-geom}), the ground
plane is $\partial K$---the boundary of the viability kernel.
The viability metric $g_V = V^{-2}\,g_S$
(\cref{def:viab-metric}) has a \emph{pole} at $V = 0$:
the boundary is at infinite $g_V$-distance, and
\cref{prop:viab-complete} guarantees that no trajectory can
reach it in finite time.
This completeness theorem \emph{requires} that the boundary be
a hard wall.

A soft-contact spring is not a wall---it is a \emph{resource}
that stores and releases energy.
It can be depleted (penetrated past its valid range).
If the ground becomes soft, $\partial K$ is no longer a
constraint that holds; it becomes a medium that yields.
\textcolor{water}{The ground becomes water.}

The sinking bound prevents this:
\[
  z_{\mathrm{contact}}(t)
  \;\geq\;
  z_{\min}
  \quad\text{for all contact points, all } t.
\]
This is not an optimisation heuristic.
It is the \emph{categorical distinction} between $\partial K$
(constraint, infinite barrier, pole of $g_V$) and $V$
(resource, depletable, conformal factor).
The ground is not water.
地面不能变成水.

It is the \emph{robot's} viability obligation to not break the
contact physics of its own simulator.
The sinking bound is not a limitation of MuJoCo that the
engineer works around---it is a constraint the robot must
actively maintain, just as the king must maintain the
institutional structure he governs through
(\cref{thm:paradox}).
The robot lives inside the physics model; breaking it is
breaking~$\partial K$ itself.

In the isomorphism of \cref{rem:sandbox-mujoco}: the code
sandbox's memory limit $m_{\max}$ plays the same role.
If a program consumes memory beyond $m_{\max}$, the OS's
virtual memory subsystem enters pathological behaviour
(thrashing, OOM killer).
The memory limit is the ground plane for code.
\end{remark}

\begin{remark}[Bill of materials: what you actually need]
\label{rem:bill-of-materials}
\Cref{alg:finetune,alg:code-eval,alg:sandbox} are
mechanically executable.
Every step maps to a shell command or library call.
The complete bill of materials:
\begin{center}
\renewcommand{\arraystretch}{1.25}
\begin{tabular}{@{}llll@{}}
\toprule
\textbf{Resource} & \textbf{Example} & \textbf{Cost} &
\textbf{Algorithm} \\
\midrule
GPU server
  & RunPod\,/\,Lambda\,/\,vast.ai
  & \textcolor{caution}{\$1/hr} (A100)
  & \cref{alg:finetune} \\
Checkpoint
  & \texttt{deepseek-coder-v2}
  & \textcolor{water}{free} (HuggingFace)
  & \cref{alg:finetune}, line~1 \\
LoRA library
  & \texttt{peft}
  & \textcolor{water}{free} (pip)
  & \cref{alg:finetune}, line~2 \\
Sandbox
  & Docker\,/\,\texttt{nsjail}
  & \textcolor{water}{free}
  & \cref{alg:sandbox} \\
Judge API key
  & Gemini-Pro
  & \textcolor{water}{free} (60 req/min)
  & \cref{alg:code-eval}, line~11 \\
Compiler
  & \texttt{gcc}\,/\,\texttt{nvcc}\,/\,\texttt{rustc}
  & \textcolor{water}{free}
  & \cref{alg:code-eval}, line~2 \\
Profiler
  & \texttt{time}\,/\,\texttt{perf}\,/\,\texttt{nsight}
  & \textcolor{water}{free}
  & \cref{alg:code-eval}, line~13 \\
Task data
  & $(z, x^*)$ pairs or spec $z$
  & \textcolor{caution}{user-provided}
  & \cref{alg:finetune}, line~5 \\
\bottomrule
\end{tabular}
\end{center}
Six of eight inputs are \textcolor{water}{free}.
The two that cost money:
a GPU server (\textcolor{caution}{\$1/hr})
and your task data.
Everything else---checkpoint, LoRA library, sandbox, judge,
compiler, profiler---is open-source or free-tier.

The automation boundary is sharp.
An AI coding assistant (e.g.\ Claude Code) with SSH access to the
GPU server can execute \cref{alg:finetune} end-to-end:
install dependencies, download the checkpoint, write the training
script, launch the four-phase curriculum, call the sandbox
(\cref{alg:sandbox}), call the judge (\cref{alg:code-eval}),
merge the adapter, and upload the result.
The human provides \textcolor{caution}{two things}:
the GPU server and the task specification.
The \textcolor{water}{pipeline} does the rest.
\end{remark}

\begin{definition}[Instruction set]
\label{def:instruction-set}
The \emph{instruction set} of the agentic calculus is
\[
  \mathcal{I}
  \;=\;
  \bigl\{\,
    \textcolor{water}{\sigma},\;\;
    \textcolor{water}{\circ},\;\;
    \textcolor{sword}{\varphi}
  \,\bigr\}
  \;=\;
  \{\sigma,\, \circ,\, \varphi\}
  \;\setminus\;
  \{\textcolor{knife}{\partial}\}.
\]
A program $p$ on the execution graph $G$ is a finite sequence of
instructions drawn from~$\mathcal{I}$.
The \textsc{Cut} operation $\textcolor{knife}{\partial}$ is
\emph{not} an instruction: it is a constraint imposed by the
environment (the knife), not an action taken by the agent.
\end{definition}

\begin{remark}[The knife cannot cut itself]
\label{rem:knife-self}
The exclusion of $\textcolor{knife}{\partial}$ from~$\mathcal{I}$
is not a design choice.
It is forced.

A program $p \in \mathcal{I}^*$ can
\textcolor{water}{slide} (transport data),
\textcolor{water}{compose} (chain operations), and
\textcolor{sword}{change phase} (switch regime).
It \emph{cannot} cut its own capacity.
Code cannot remove its own weight-decay.
A function cannot delete its own regulariser.
The knife is not a tool the agent wields---it is the boundary the
agent lives inside.

Suppose it could.
Let $p_{\partial}$ be a program that sets $\gamma(e) = 0$ on its
own edges.
By \cref{thm:camus}, setting $\gamma \to 0$ eliminates the
stability margin $\lambda_1 > 0$ (\cref{rem:contact-stability}).
The system becomes unstable: small perturbations amplify.
The program that removes its own knife
\textcolor{knife}{destroys itself}.

This is the diagonal constraint.
Every computable instruction set has exactly one operation it cannot
apply to itself:
\begin{center}
\renewcommand{\arraystretch}{1.25}
\begin{tabular}{@{}lll@{}}
\toprule
\textbf{System} & \textbf{Instruction set} &
\textbf{Excluded operation} \\
\midrule
Turing machine & $\{$read, write, move, halt$\}$ &
  halt on self \\
$\lambda$-calculus & $\{$abstract, apply$\}$ &
  decide own termination \\
Agentic calculus & $\{\sigma, \circ, \varphi\}$ &
  $\partial$ on self \\
\bottomrule
\end{tabular}
\end{center}
The consequence is the same in all three cases:
\\[4pt]
\hspace*{2em}%
\textcolor{knife}{You cannot use the system to prove the system
safe.}
\\[4pt]
The \textcolor{knife}{knife} is the price of
\textcolor{water}{flow}.
Remove it, and the flow destroys the channel.
\end{remark}

\begin{definition}[Sandbox daemon]
\label{def:sandbox-daemon}
The \emph{sandbox daemon} is a persistent instance of
\cref{alg:sandbox} that separates the one-time
\textcolor{sword}{\textsc{Build}} and
\textcolor{knife}{\textsc{Destroy}} from the per-evaluation
loop:
\[
  \underbrace{%
    \textcolor{sword}{\textsc{Build}}
  }_{\text{once}}
  \;\to\;
  \underbrace{%
    \bigl(\,
      \textcolor{water}{\textsc{Inject}} \;\to\;
      \textcolor{water}{\textsc{Execute}} \;\to\;
      \textcolor{water}{\textsc{Extract}}
    \,\bigr)^N
  }_{\text{per evaluation}}
  \;\to\;
  \underbrace{%
    \textcolor{knife}{\textsc{Destroy}}
  }_{\text{once}}.
\]
The container $\mathcal{C}$ persists across all $N$ evaluations
in \cref{alg:finetune}.
The compilation cache, GPU context, and sanitiser hooks survive
between calls.
\end{definition}

\begin{remark}[The daemon is the standing structure]
\label{rem:daemon-standing}
In locomotion (\cref{alg:loco}), the robot body persists across
all rollouts.
You do not rebuild the quadruped every step.
The standing structure (\cref{rem:standing})---joints, torques,
contact modes---is loaded once and reused.

The sandbox daemon is the same pattern.
The four engineering requirements for CUDA trajectory optimisation
map to the four tower layers:
\begin{center}
\renewcommand{\arraystretch}{1.25}
\begin{tabular}{@{}llll@{}}
\toprule
\textbf{Gap} & \textbf{Tower} & \textbf{Daemon component} &
\textbf{Persistence} \\
\midrule
Host-device wrapper
  & \textcolor{water}{力${}^*$}
  & \texttt{cudaMalloc}, grid config, launch
  & GPU context survives \\
Compilation bottleneck
  & \textcolor{sword}{立${}^*$}
  & \texttt{ccache}, hot \texttt{nvcc} daemon
  & only recompile $\Delta$ \\
Mathematical oracle
  & \textcolor{caution}{丽(正)${}^*$}
  & KKT residual $< \varepsilon$,
    constraint violation $< \delta$
  & validator loaded once \\
GPU isolation
  & \textcolor{knife}{刀}
  & \texttt{--gpus all}, \texttt{compute-sanitizer}
  & container $\mathcal{C}$ persists \\
\bottomrule
\end{tabular}
\end{center}
Without the daemon, \cref{alg:sandbox} creates and destroys a
container per evaluation.
With the daemon, the container is the body.
\textsc{Build} is birth.
\textsc{Destroy} is death.
The training loop runs \emph{inside} a life.

The isomorphism is exact:
\begin{center}
\renewcommand{\arraystretch}{1.25}
\begin{tabular}{@{}lll@{}}
\toprule
& \textbf{Locomotion} & \textbf{CUDA sandbox daemon} \\
\midrule
Body & quadruped (MuJoCo) & container $\mathcal{C}$ (Docker) \\
Standing & $h > h_{\min}$ & \texttt{nvcc} cache warm \\
Joint limits & $c(e) \leq c_{\max}$ &
  \texttt{compute-sanitizer}: 0 errors \\
Rollout & simulate $\to$ reward & inject $\to$ compile $\to$ run \\
Persistence & body across rollouts & $\mathcal{C}$ across evaluations \\
\bottomrule
\end{tabular}
\end{center}
You do not rebuild the robot every step.
You do not rebuild the sandbox every evaluation.
The daemon \emph{is} the standing structure for code.
\end{remark}

\section{华容道: Complete Instantiation}\label{sec:huarongdao}

The historical cases in \cref{sec:applications} each validate one
concept. We now present a single finite object that instantiates the
\emph{entire} framework simultaneously: the Chinese sliding block
puzzle 华容道 (Huarong Pass).

\subsection{The puzzle}

\begin{definition}[华容道]\label{def:huarongdao}
A \emph{华容道 instance} is a tuple
$(\mathcal{B},\, \mathcal{P},\, p_*,\, E)$:
\begin{enumerate}[label=(\roman*)]
  \item $\mathcal{B} = [m] \times [n]$: rectangular grid (the
  \emph{board}, 方);
  \item $\mathcal{P} = \{p_1, \ldots, p_k\}$: rectangular blocks
  placed non-overlapping on $\mathcal{B}$ (the \emph{pieces}, 圆);
  \item $p_* \in \mathcal{P}$: distinguished piece (the \emph{king});
  \item $E \subset \partial\mathcal{B}$: boundary region congruent
  to $p_*$ (the \emph{exit}).
\end{enumerate}
The \emph{free cells} $\mathcal{F} = \mathcal{B} \setminus
\bigcup_i p_i$ are the system's degrees of freedom. A
\emph{configuration} is a valid placement of all pieces. A \emph{move}
is a unit translation of one piece into adjacent free cells. The
\emph{configuration graph} $\mathcal{G} = (\mathcal{V}, \mathcal{E})$
has configurations as vertices and legal moves as edges.

The puzzle: does there exist a path in $\mathcal{G}$ from $\sigma_0$
to any $\sigma_f$ with $\sigma_f(p_*) = E$?
\end{definition}

The standard instance is $\mathcal{B} = [4] \times [5]$ in the
configuration 横刀立马 (``horizontal knife, standing horse''):

\begin{center}
\begin{tikzpicture}[scale=0.85]
  % Grid
  \draw[gray!40, thin] (0,0) grid (4,5);
  \draw[very thick] (0,0) rectangle (4,5);
  % Exit
  \draw[very thick, densely dashed] (1,-0.05) -- (1,-0.4) -- (3,-0.4)
    -- (3,-0.05);
  \node[font=\footnotesize] at (2,-0.65) {$E$ (exit)};
  % 曹操 (2×2) — the king
  \fill[black!80] (1.05,3.05) rectangle (2.95,4.95);
  \node[white, font=\bfseries\large] at (2,4.2) {曹操};
  \node[white, font=\scriptsize] at (2,3.5) {$p_*\;(2{\times}2)$};
  % 关羽 (2×1 horizontal) — the knife
  \fill[black!55] (1.05,2.05) rectangle (2.95,2.95);
  \node[white, font=\small\bfseries] at (2,2.5) {关羽 $(2{\times}1)$};
  % Generals (1×2 vertical)
  \fill[black!35] (0.05,3.05) rectangle (0.95,4.95);
  \node[white, font=\small, rotate=90] at (0.5,4) {张飞};
  \fill[black!35] (3.05,3.05) rectangle (3.95,4.95);
  \node[white, font=\small, rotate=90] at (3.5,4) {赵云};
  \fill[black!35] (0.05,1.05) rectangle (0.95,2.95);
  \node[white, font=\small, rotate=90] at (0.5,2) {马超};
  \fill[black!35] (3.05,1.05) rectangle (3.95,2.95);
  \node[white, font=\small, rotate=90] at (3.5,2) {黄忠};
  % Soldiers (1×1) — numbered ①②③④ to match 棋谱 (\cref{app:solution})
  \fill[black!12] (1.05,1.05) rectangle (1.95,1.95);
  \draw[black!50] (1.05,1.05) rectangle (1.95,1.95);
  \node[font=\small] at (1.5,1.5) {\textcircled{\scriptsize 1}};
  \fill[black!12] (2.05,1.05) rectangle (2.95,1.95);
  \draw[black!50] (2.05,1.05) rectangle (2.95,1.95);
  \node[font=\small] at (2.5,1.5) {\textcircled{\scriptsize 2}};
  \fill[black!12] (0.05,0.05) rectangle (0.95,0.95);
  \draw[black!50] (0.05,0.05) rectangle (0.95,0.95);
  \node[font=\small] at (0.5,0.5) {\textcircled{\scriptsize 3}};
  \fill[black!12] (3.05,0.05) rectangle (3.95,0.95);
  \draw[black!50] (3.05,0.05) rectangle (3.95,0.95);
  \node[font=\small] at (3.5,0.5) {\textcircled{\scriptsize 4}};
  % Free cells
  \draw[densely dashed, black!40] (1.05,0.05) rectangle (1.95,0.95);
  \node[black!50, font=\footnotesize] at (1.5,0.5) {$\varnothing$};
  \draw[densely dashed, black!40] (2.05,0.05) rectangle (2.95,0.95);
  \node[black!50, font=\footnotesize] at (2.5,0.5) {$\varnothing$};
  % Legend
  \begin{scope}[font=\footnotesize, anchor=west]
    \fill[black!80] (4.7,4.65) rectangle (5.0,4.85);
    \node at (5.15,4.75) {王: $p_*$ $(2{\times}2)$};
    \fill[black!55] (4.7,4.15) rectangle (5.0,4.35);
    \node at (5.15,4.25) {刀: 关羽 $(2{\times}1)$};
    \fill[black!35] (4.7,3.65) rectangle (5.0,3.85);
    \node at (5.15,3.75) {将: generals $(1{\times}2)$};
    \fill[black!12] (4.7,3.15) rectangle (5.0,3.35);
    \draw[black!50] (4.7,3.15) rectangle (5.0,3.35);
    \node at (5.15,3.25) {卒: \textcircled{\tiny 1}--\textcircled{\tiny 4}\ $(1{\times}1)$};
    \draw[densely dashed, black!40] (4.7,2.65) rectangle (5.0,2.85);
    \node at (5.15,2.75) {$\varnothing$: free cells (水)};
  \end{scope}
\end{tikzpicture}
\end{center}

Ten pieces ($4 + 2 + 4 \cdot 2 + 4 \cdot 1 = 18$ cells), two free
cells, exit at bottom center (width~$2$). The minimum solution requires
81~步 (steps; \cref{def:bu}).

\subsection{The isomorphism}

\begin{theorem}[华容道 $=$ viability maintenance]\label{thm:huarongdao}
The standard 华容道 instance instantiates the viability framework:
\begin{center}
\begin{tabular}{@{}lp{4cm}p{5cm}@{}}
\toprule
\textbf{Framework} & \textbf{华容道} & \textbf{Mechanism} \\
\midrule
Viability axiom & Path $\sigma_0 \to \sigma_f$ in $\mathcal{G}$ &
King must reach exit \\
King & 曹操 $(2{\times}2)$ & Least mobile, highest importance \\
Knife (\cref{def:knife}) & 关羽 $(2{\times}1)$ & Blocks exit;
autonomous; observable \\
Pawns & Soldiers $(1{\times}1)$ & Most mobile, lowest importance \\
Cut vertex & Free cells $\mathcal{F}$ &
$\mathcal{F} = \varnothing \Rightarrow$ frozen \\
Phase transition & 关羽 clears corridor & Before: blocked. After:
path opens \\
Water & Free-cell flow & Slides opposite to piece movement \\
$w = 0$ collapse & Zero free cells & No flow $\to$ no path $\to$
dead \\
\bottomrule
\end{tabular}
\end{center}
\end{theorem}

\begin{proof}
\emph{Viability.} The puzzle asks exactly \cref{ax:viability}: does a
path exist from $\sigma_0$ through $\mathcal{G}$ to a goal state? The
configuration graph is the execution graph; each edge is a legal move;
the viable kernel is the set of configurations from which the exit
remains reachable.

\emph{Knife.} 关羽 satisfies both conditions of \cref{def:knife}:
(1)~autonomous actuation---he can slide independently of 曹操---and
(2)~observability---his position visibly blocks the exit corridor.
He must yield for the king to pass.

\emph{Cut $=$ free cell.} Fill both free cells and $\mathcal{G}$ has
no edges: every configuration is isolated. The free cell inverts the
cut vertex: ``the absence whose removal freezes.'' One structural
element controls all connectivity.

\emph{Mobility $\propto 1/\text{size}$.} A piece of size $s$ needs $s$
aligned free cells to move. Soldiers ($s = 1$): one free cell.
曹操 ($s = 4$): two aligned free cells. The king is the least mobile
agent---\cref{thm:cutvertex} made physical.

\emph{Water.} When a piece slides left, the free cell moves right.
The free cell flows in the opposite direction---it \emph{is} the water
(\cref{def:water}). Each move transfers the free cell to a new
position, enabling the next move. The 81-步 solution is a flow of
water through 81 channels.
\end{proof}

\begin{remark}[横刀立马: the name is the theorem]\label{rem:hrdname}
The configuration's traditional name means ``horizontal knife, standing
horse.'' 刀~(knife) $=$ 关羽 blocking horizontally; 马~(horse) $=$
generals standing vertically. Chinese game designers named the
configuration by its structural properties---the framework's vocabulary,
centuries before graph theory.
\end{remark}

\begin{remark}[义释曹操]\label{rem:guanyu}
In the \emph{Romance of the Three Kingdoms}, 关羽 is stationed at
华容道 after the Battle of Red Cliffs (208~CE). 曹操 retreats through.
关羽 has both conditions of \cref{def:knife}: autonomous actuation
(his army) and observability (曹操 approaching in plain sight). He
chooses path~(a): 义~(righteousness) overrides 忠~(loyalty to Liu Bei),
and he sets $\Ur \to \varnothing$ voluntarily.

The puzzle encodes this: every solution requires moving 关羽 aside.
To solve 华容道 is to perform 义释曹操.
\end{remark}

\begin{remark}[方圆 $\times$ 黑白]\label{rem:fangyuan}
The puzzle's $2 \times 2$ classification is formalized in
\cref{def:fangyuan}. The four statics classify every cell and piece;
the dynamics---刀 (boundary/cut) and 水 (flow/transport)---emerge from
the $2 \times 2$ as the calculus operations of \cref{sec:calculus}.
\end{remark}

\subsection{The experiment}\label{sec:kpc}

\Cref{thm:huarongdao} maps framework concepts to puzzle roles.
We now make the mapping computational.
The configuration graph $\mathcal{G}$ has
$|\mathcal{V}| = 25{,}955$ canonical states and is
connected; every quantity below is \emph{exact}
(self-contained solver: \texttt{solver/}).

\subsubsection*{Water as agent}

The proof of \cref{thm:huarongdao} observed that free cells
flow opposite to piece movement.
We now invert the perspective entirely:
the two free cells \emph{are} the agent~(水),
and the ten pieces \emph{are} the board.

\begin{definition}[水-position and 水-graph]\label{def:water-graph}
A \emph{水-position} is the unordered pair
$\{f_1, f_2\} \subset \mathcal{B}$ of free cells.
Write $W(\sigma)$ for the 水-position of
configuration~$\sigma$.
The \emph{水-graph} has vertex set
$\binom{\mathcal{B}}{2}$ and an edge between
$W(\sigma)$ and $W(\sigma')$ whenever
$(\sigma, \sigma') \in \mathcal{E}$.
\end{definition}

\begin{proposition}[Ergodicity of 水]\label{thm:ergodic}
All $\binom{20}{2} = 190$ possible 水-positions are
reachable from $\sigma_0$.
\end{proposition}

\begin{proof}
Exhaustive BFS on $\mathcal{G}$: the set
$\{W(\sigma) : \sigma \text{ reachable from } \sigma_0\}$
has cardinality $190 = \binom{20}{2}$.
水 can reach every pair of cells in the board.
\end{proof}

\begin{definition}[Free-cell modes]\label{def:free-modes}
The 水-position $\{f_1, f_2\}$ has \emph{mode}:
\begin{itemize}
  \item \textbf{H}~(horizontal pair): $f_1, f_2$
  horizontally adjacent.
  Enables slides of $2 \times h$ pieces.
  \item \textbf{V}~(vertical pair): $f_1, f_2$
  vertically adjacent.
  Enables slides of $w \times 2$ pieces.
  \item \textbf{S}~(separated): $f_1, f_2$ non-adjacent.
  Only $1 \times 1$ pieces (soldiers) can move.
\end{itemize}
The king ($2 \times 2$) requires mode H or V\@.
The knife ($2 \times 1$) requires mode~H\@.
In mode~S, only soldiers act---水 is diffuse.
\end{definition}

\subsubsection*{The counting: 81}

\begin{definition}[步 (step)]\label{def:bu}
A \emph{step}~(步) is a maximal consecutive sequence of
unit moves of the same piece.
Equivalently: each step engages one piece;
all slides of that piece within the step cost zero;
switching to a different piece costs one.
\end{definition}

In Chinese puzzle tradition, the standard counting of
華容道 solutions uses~步, not unit moves.
The difference is algorithmic:

\begin{algorithm}[H]
\caption{Minimum-步 solver
  (0/1~BFS on $\mathcal{G}$)}\label{alg:kpc}
\begin{algorithmic}[1]
\Require $\mathcal{G} = (\mathcal{V}, \mathcal{E})$,\;
  initial $\sigma_0$,\;
  goal $\sigma(p_*) = E$
\Ensure Minimum-步 path $\gamma$,\;
  step count $|\gamma|$
\Statex
\State Augment state:
  $(\sigma, \ell) \in
  \mathcal{V} \times \{1, \ldots, k, \bot\}$,
  $\ell$ = last piece moved
\State $\mathrm{dist}[(\sigma_0, \bot)] \gets 0$;\;
  $Q \gets \mathrm{deque}
  \bigl[\bigl((\sigma_0, \bot),\, 0\bigr)\bigr]$
\While{$Q \neq \varnothing$}
  \State $(\sigma, \ell),\, d \gets Q.\mathrm{popleft}()$
  \If{$\sigma(p_*) = E$}
    \Return $d$ \Comment{goal reached}
  \EndIf
  \For{each neighbour $(\sigma', i)$ of $\sigma$}
    \Comment{$i$ = piece moved}
    \State $c \gets
      \begin{cases}
      0 & i = \ell \\
      1 & i \neq \ell
      \end{cases}$
      \Comment{same piece $\to$ free;
               switch $\to$ 1~步}
    \If{$d + c < \mathrm{dist}[(\sigma', i)]$}
      \State $\mathrm{dist}[(\sigma', i)] \gets d + c$
      \If{$c = 0$}\;
        $Q.\mathrm{pushfront}
        \bigl((\sigma', i),\, d\bigr)$
      \Else\;
        $Q.\mathrm{pushback}
        \bigl((\sigma', i),\, d + 1\bigr)$
      \EndIf
    \EndIf
  \EndFor
\EndWhile
\end{algorithmic}
\end{algorithm}

\begin{remark}[Complexity]\label{rem:complexity}
\begin{center}
\small
\begin{tabular}{@{}lccc@{}}
\toprule
 & \textbf{General} & \textbf{横刀立马} & \textbf{棋谱} \\
 & (sliding block) & (this instance) & (verification) \\
\midrule
Class & PSPACE-complete & --- & --- \\
$|\mathcal{V}|$ & exponential & $25{,}955$ & --- \\
$|\mathcal{E}|$ & exponential & $83{,}896$ & --- \\
Time & exponential & $O(|\mathcal{V}| + |\mathcal{E}|)$ & $O(81)$ \\
Space & exponential & $O(|\mathcal{V}|)$ & $O(1)$ \\
Optimal & ? & $81$~步 & $81$~步 (certificate) \\
\bottomrule
\end{tabular}
\end{center}
The general sliding-block puzzle is PSPACE-complete \cite{hearn}.
This instance has $25{,}955$~states:
BFS solves it in ${<}\,1$~second.
The 棋谱 (\cref{app:solution}) is a certificate
verifiable in $O(81)$~time and $O(1)$~space.
The gap from PSPACE to $O(25{,}955)$ is the entire point:
the board is fixed, the state space is finite,
the solution is exact.
\end{remark}

\begin{theorem}[81~步]\label{thm:eightyone}
The minimum number of steps from $\sigma_0$ to any
$\sigma_f$ with $\sigma_f(p_*) = E$ is exactly~$81$.
Moreover:
\begin{enumerate}[label=(\roman*)]
  \item The result is \emph{independent} of the base move
  set: both single-cell unit moves and multi-cell slides
  yield $81$~步.
  \item Along the optimal path:
  $118$~unit moves, of which $37$ are multi-slide steps
  (same piece slides $\geq 2$ cells).
  \item The $81$~步 decompose as
  $9$~king steps $+\; 72$~水-steps.
  The king acts in only $9/81 \approx 11\%$ of all steps.
\end{enumerate}
\end{theorem}

\begin{proof}
(i)~\Cref{alg:kpc} on the single-cell graph returns~$81$.
Running on the multi-cell graph (where a piece may slide
multiple cells in one unit move) also returns~$81$.
The 0/1~cost structure makes intermediate slides free:
a piece that slides two cells costs the same as one that
slides one cell---both cost zero if the piece was already
engaged, one if it is a new engagement.

(ii)--(iii)~Path extraction from the 0/1~BFS parent
pointers.
\end{proof}

\begin{remark}[Three counting conventions]\label{rem:counting}
\begin{center}
\small
\begin{tabular}{@{}llcl@{}}
\toprule
\textbf{Convention} & \textbf{Definition} &
  \textbf{Count} & \textbf{Algorithm} \\
\midrule
Unit moves & One cell, one direction &
  $116$ & Standard BFS \\
Multi-cell moves & One piece, one direction, any dist.\ &
  $90$ & Standard BFS \\
步 & One piece, any direction, any dist.\ &
  $81$ & 0/1~BFS \\
\bottomrule
\end{tabular}
\end{center}
The 步-count is minimal because it reflects the
agent's decisions: which piece to engage next.
This is the natural cost in agentic theory---the number
of discrete choices, not the number of physical slides.
\end{remark}

\subsubsection*{Phase decomposition}

\begin{proposition}[Phase decomposition: $9 + 72$]%
\label{prop:phase-decomp}
The $81$-步 solution decomposes into $9$~phases,
separated by the $9$~king steps.
水 rearranges between each king move:
\begin{center}
\small
\begin{tabular}{@{}ccrll@{}}
\toprule
\textbf{Phase} & \textbf{水-steps} &
  \textbf{King step} & \textbf{Direction} &
  \textbf{King position} \\
\midrule
1 & $25$ & \#26 & $\to$ & $(2, 3)$ \\
2 & $5$  & \#32 & $\leftarrow$ & $(1, 3)$ \\
3 & $8$  & \#41 & $\downarrow$ & $(1, 2)$ \\
4 & $6$  & \#48 & $\downarrow$ & $(1, 1)$ \\
5 & $3$  & \#52 & $\to$ & $(2, 1)$ \\
6 & $7$  & \#60 & $\leftarrow$ & $(1, 1)$ \\
7 & $6$  & \#67 & $\leftarrow$ & $(0, 1)$ \\
8 & $8$  & \#76 & $\downarrow$ & $(0, 0)$ \\
9 & $4$  & \#81 & $\to$ & $(1, 0) = E$ \\
\bottomrule
\end{tabular}
\end{center}
Phase~1 is the longest: $25$~水-steps to clear the path.
This \emph{is} 義釋曹操 (\cref{rem:guanyu}): 水 must
rearrange $25$~times before the king can move once.
\end{proposition}

\begin{remark}[King's freedom]\label{rem:king-freedom}
Along the unit-move optimal path ($116$~moves),
the king has $\geq 1$ legal move in only
$16$ of $116$ configurations ($14\%$).
The remaining $86\%$ of the time, the king is stuck;
水 works to create the next opening.
This is \cref{thm:cutvertex} quantified:
the king is the least mobile agent, and 水 is
the sole source of mobility.
\end{remark}

\subsubsection*{Saddle and mirror descent}

\begin{definition}[Primal--dual distances]%
\label{def:primal-dual}
For each $\sigma \in \mathcal{V}$:
\begin{align}
  d^+(\sigma) &\coloneqq
    d_{\mathcal{G}}^{\text{步}}(\sigma_0,\, \sigma)
    && \text{(forward / primal)}, \label{eq:dplus} \\
  d^-(\sigma) &\coloneqq
    d_{\mathcal{G}}^{\text{步}}(\sigma,\, \sigma_f)
    && \text{(backward / dual)}, \label{eq:dminus}
\end{align}
where $d_{\mathcal{G}}^{\text{步}}$ is the shortest-path
metric in steps (0/1~BFS distances).
\end{definition}

\begin{proposition}[Duality gap]\label{prop:duality-gap}
For all $\sigma$ on any shortest path
$\gamma = (s_0, \ldots, s_{81})$:
\begin{equation}\label{eq:duality-gap}
  d^+(s_j) + d^-(s_j) \;=\; 81
  \qquad \text{for all } 0 \leq j \leq 81.
\end{equation}
The \emph{saddle configuration} is
$\sigma^* \coloneqq s_{k^*}$ where
\[
  k^* \coloneqq \argmin_{0 \leq j \leq 81}
  \bigl|\, d^+(s_j) - d^-(s_j) \,\bigr|.
\]
Since $d^+(s_j) = j$ and $d^-(s_j) = 81 - j$,
the saddle is at step $k^* = 40$.
\end{proposition}

\begin{proof}
Triangle inequality:
$81 = d_{\mathcal{G}}^{\text{步}}(\sigma_0, \sigma_f)
  \leq d^+(s_j) + d^-(s_j)$.
Equality holds because each $s_j$ lies on a
$(\sigma_0, \sigma_f)$-geodesic.
The midpoint $\lfloor 81/2 \rfloor = 40$.
\end{proof}

\begin{theorem}[Saddle $=$ phase transition]\label{thm:saddle}
At the saddle $\sigma^*$ (step~$40$):
\begin{enumerate}[label=(\roman*)]
  \item 関羽 (knife) is at position $(2, 0)$, far from
  the exit corridor---she has cleared.
  \item The 水-mode is H
  (horizontal pair at $\{(1,2), (2,2)\}$),
  enabling the king's next descent.
  \item The king is at $(1, 3)$: it has moved right once
  (step~$26$) and back left once (step~$32$).
  The first descent begins at step~$41$.
\end{enumerate}
To cross the saddle is to perform 義釋曹操
(\cref{rem:guanyu}).
\end{theorem}

\begin{proof}
By \cref{thm:flowcut}, the min-cut separates
blocked configurations (関羽 covers exit corridor)
from open configurations (corridor cleared).
On the optimal path the cut lies where
$d^+ \approx d^-$, i.e., step~$40$.
Direct inspection of the BFS solution confirms:
at $\sigma^*$, 関羽 has exited the corridor.
\end{proof}

\begin{remark}[Mirror descent]\label{rem:mirror}
\Cref{alg:kpc} is mirror descent on a finite graph.
The forward 0/1~BFS computes the primal potential $d^+$;
the backward 0/1~BFS computes the dual potential $d^-$.
At the saddle, $d^+(\sigma^*) = 40$ and
$d^-(\sigma^*) = 41$: the potentials balance.
This is the same forward--backward structure as
\cref{alg:forward,alg:backward}:
the forward pass computes activations (primal),
the backward pass computes gradients (dual),
and the saddle is the phase transition of the loss.
\end{remark}

\subsubsection*{Mode distribution}

\begin{proposition}[Mode statistics]\label{prop:modes}
Along the $81$-步 path, the 水-mode after each step is:
\begin{center}
\small
\begin{tabular}{@{}lcl@{}}
\toprule
\textbf{Mode} & \textbf{Count} & \textbf{Enables} \\
\midrule
H (horizontal) & $27$ & knife, king (horizontal) \\
V (vertical)   & $42$ & generals, king (vertical) \\
S (separated)  & $12$ & soldiers only \\
\bottomrule
\end{tabular}
\end{center}
The dominance of V-mode ($52\%$) reflects the
puzzle's vertical bias: the king must descend $3$~rows.
S-mode ($15\%$) appears when 水 must diffuse to
reposition---it is the ``reset'' phase.
The king moves only in H or V mode, never in S\@.
\end{proposition}

\subsubsection*{The isomorphism, completed}

\Cref{thm:huarongdao} identified the viability roles.
\Cref{thm:eightyone} extends the isomorphism to the
\emph{computational} structure of \cref{sec:calculus}:

\begin{table}[H]
\centering\small
\caption{Agentic calculus on 華容道: complete correspondence.}%
\label{tab:kpc}
\begin{tabular}{@{}lll@{}}
\toprule
\textbf{Calculus (\cref{sec:calculus})} &
\textbf{華容道} &
\textbf{Value} \\
\midrule
Execution graph (\cref{def:exgraph}) &
  Configuration graph $\mathcal{G}$ &
  $25{,}955$ states \\
Agentic flow (\cref{def:flow}) &
  水-flow (free-cell agent) &
  $190/190$ ergodic \\
Min-cut (\cref{thm:flowcut}) &
  関羽 blocks corridor &
  Cleared at step~$40$ \\
Max-flow &
  Optimal 步-path &
  $81$~步 \\
Forward pass (\cref{alg:forward}) &
  $d^+$: primal 0/1~BFS &
  Distance from $\sigma_0$ \\
Backward pass (\cref{alg:backward}) &
  $d^-$: dual 0/1~BFS &
  Distance to $\sigma_f$ \\
Saddle of loss &
  $\sigma^*$: $d^+ \approx d^-$ &
  Step~$40$ \\
Contact modes (\cref{sec:contact}) &
  水-modes (H, V, S) &
  $3$ modes \\
Mobility $\propto 1/s$ &
  King: $9/81$; Soldiers: most &
  Heavy-tailed \\
Spectral gap (\cref{thm:massgap}) &
  $\lambda_1 > 0$ &
  $\mathcal{G}$ connected \\
Horizon $H$ &
  Geodesic in 步 &
  $81$ \\
Phase decomposition &
  $9$~king $+ 72$~水 &
  $25$~步 to clear \\
刀 dissipation (\cref{def:dissipation}) &
  関羽 yields &
  Phase~1: 義釋曹操 \\
水 (\cref{def:water}) &
  Free-cell agent &
  $190/190$ ergodic \\
\bottomrule
\end{tabular}
\end{table}

\begin{remark}[$81$ as horizon]\label{rem:eightyone}
The minimum $81$ is not an arbitrary combinatorial fact.
It is the \emph{horizon} of the predictive controller:
the geodesic length in $\mathcal{G}$ measured in~步,
equivalently the minimum number of
\textcolor{water}{水}~decisions to transport the king from
$\sigma_0$ to the exit.
In the language of \cref{thm:massgap}:
$H = 81$ is the spectral gap made finite.

The three conventions (\cref{rem:counting}) separate
cleanly: unit moves ($116$) count physical slides;
multi-cell moves ($90$) count piece--direction pairs;
步~($81$) count agent decisions.
The agentic theory uses~步 because it counts
\emph{what the agent chooses}, not what the physics does.
\end{remark}

\section{Conclusion}\label{sec:conclusion}

We have presented an agentic theory of viability maintenance built on a
single axiom (the existence of a viable path to infinity) and a
two-condition criterion (the knife). The framework produces three main
theorems (binary lifecycle, fixed-point impossibility, unconstrained
power paradox) and a central interpretive result: the knife is the mean
field.

The knife is not an intrinsic property of a resource. It is a
statistical deviation from the system's mean autonomous actuation,
made visible by the detection function and made dangerous by the
viability axiom. Phase transitions shift the mean, not the individual.
The king responds to the mean, not to intent.

The agentic calculus (\cref{sec:calculus}) translates this theory into
an operational language: every theorem becomes a flow-theoretic
proposition, the knife becomes the min-cut, and the viable path becomes
the max-flow. The duality ``the knife is the mean'' is a restatement of
max-flow/min-cut duality on the execution graph.

This reframing connects viability maintenance to mean-field theory in
statistical mechanics, where phase transitions are driven by shifts in
the order parameter rather than changes in individual configurations.
The viability axiom plays the role of free energy minimization; the
knife plays the role of the critical fluctuation.

Two millennia of Chinese imperial history validate the framework with
unusual clarity. The same structure appears---with instructive
breaks---in the Atlantic slave trade, ideological hatred, parasitic
network topologies, and militarism. The framework's failure conditions
(\cref{sec:domain}) are as informative as its successes: they delineate
the boundary between systems where the viability axiom operates cleanly
and systems where it is dominated by other dynamics.

The knife is the mean. Viability maintenance is a mean-field phenomenon.
The theory is agentic because the agents---not their intentions, not
their narratives, not their moral qualities, but their structural
positions in the execution graph---determine the outcome.


% ── Appendices ────────────────────────────────────────────
\appendix
\section{原典与人物}\label{app:sources}

The historical analysis in this paper draws primarily on Sima Qian's
\emph{Shiji} (《史记》, \emph{Records of the Grand Historian},
c.~94~BCE) and Du Mu's \emph{A Fang Gong Fu} (《阿房宫赋》, 825~CE).
This appendix collects the original Classical Chinese passages cited or
referenced in the main text, with English translations and brief
profiles of the historical figures.

\subsection{人物志}\label{app:persons}

\paragraph{刘邦 Liu Bang (256--195 BCE).}
Founder of the Han dynasty. Rose from minor local official (亭长) to
emperor. Near-zero personal combat ability; made himself the cut vertex
of the execution graph (\cref{ex:liubang}).

\begin{quote}
高祖为人,仁而爱人,喜施,意豁如也。常有大度。不事家人生产作业。

\medskip
\emph{Gaozu was a man of benevolence who loved people, was generous in
giving, and broad-minded. He had great magnanimity. He did not engage
in household production or labor.}
\hfill ---《史记·高祖本纪》
\end{quote}

Translation into the framework: Liu Bang himself had no actuation
capability. He could not fight, could not administer, could not
strategize. His sole structural role was as the mandatory routing node.

On first seeing the First Emperor's procession:
\begin{quote}
嗟乎,大丈夫当如此也!

\medskip
\emph{Ah, a great man should be like this!}
\hfill ---《史记·高祖本纪》
\end{quote}

Liu Bang saw the existence proof (\cref{thm:qin}) and wanted to
\emph{be} the system's cut vertex.

His self-assessment after founding the Han dynasty:
\begin{quote}
夫运筹策帷帐之中,决胜於千里之外,吾不如子房。镇国家,抚百姓,给馈饷,不绝粮道,吾不如萧何。连百万之军,战必胜,攻必取,吾不如韩信。此三者,皆人杰也,吾能用之,此吾所以取天下也。项羽有一范增而不能用,此其所以为我擒也。

\medskip
\emph{For devising strategies within a tent and securing victory a
thousand li away, I am not as good as Zhang Liang. For governing the
state, caring for the people, providing supplies, and keeping the grain
roads open, I am not as good as Xiao He. For commanding a million
soldiers, winning every battle and taking every siege, I am not as good
as Han Xin. These three are all heroes---but I can use them. This is
why I won the empire. Xiang Yu had one Fan Zeng but could not use him.
This is why he was captured by me.}
\hfill ---《史记·高祖本纪》
\end{quote}

「吾能用之」(\emph{I can use them}) is the operational definition of
the cut vertex: actuation resides in others, but routing authority
resides in Liu Bang. Every execution chain passes through him. The
character 用 (use/employ) is not metaphorical---it is a precise
description of the cut vertex's function in the execution graph.

\paragraph{项羽 Xiang Yu (232--202 BCE).}
Supreme military commander of the anti-Qin uprising. Maximum actuator
(\cref{ex:liubang}): personal combat ability unmatched in the system.

\begin{quote}
籍长八尺馀,力能扛鼎,才气过人。

\medskip
\emph{[Xiang] Ji was over eight chi tall, could lift a bronze tripod,
and his talent and spirit surpassed all others.}
\hfill ---《史记·项羽本纪》
\end{quote}

\begin{quote}
项王嗔目而叱之,赤泉侯人马俱惊,辟易数里。

\medskip
\emph{The King of Xiang glared and bellowed at him; the Marquis of
Chiquan and his horse both recoiled in terror, retreating several li.}
\hfill ---《史记·项羽本纪》
\end{quote}

On seeing the same procession Liu Bang saw:
\begin{quote}
彼可取而代也!

\medskip
\emph{That one---I can replace him!}
\hfill ---《史记·项羽本纪》
\end{quote}

Liu Bang: ``I want to \emph{be} this.'' Xiang Yu: ``I can
\emph{replace} him.'' One read the existence proof as a system to
inhabit. The other read it as a person to defeat.

Han Xin's assessment of Xiang Yu:
\begin{quote}
项王见人恭敬慈爱,言语呕呕,人有疾病,涕泣分食饮,至使人有功当封爵者,印刓敝,忍不能予。此所谓妇人之仁也。

\medskip
\emph{The King of Xiang is respectful and caring when he meets people;
his speech is warm and gentle. When someone is ill, he weeps and shares
his food and drink. But when a man has earned merit and deserves a
title, he fondles the seal until its edges are worn smooth, and still
cannot bring himself to hand it over. This is what is called the
benevolence of a woman.}
\hfill ---《史记·淮阴侯列传》
\end{quote}

「印刓敝,忍不能予」: the seal is carved and ready, rubbed smooth from
handling, yet he cannot let it go. This is not benevolence---it is the
inability to distribute actuation. A cut vertex that cannot delegate
is a maximum actuator pretending to route.

After conquering the Qin capital:
\begin{quote}
富贵不归故乡,如衣绣夜行,谁知之者!

\medskip
\emph{To be wealthy and noble but not return home is like wearing
embroidered robes at night---who would see it?}
\hfill ---《史记·项羽本纪》
\end{quote}

His last song, at Gaixia (垓下歌):
\begin{quote}
力拔山兮气盖世,时不利兮骓不逝。\\
骓不逝兮可奈何,虞兮虞兮奈若何!

\medskip
\emph{My strength could uproot mountains, my spirit overmastered the
world. / But the times turned against me, and my horse would not go. /
My horse would not go---what can be done? / Yu, oh Yu---what will
become of you?}
\hfill ---《史记·项羽本纪》
\end{quote}

At the bank of the Wu River, refusing to cross:
\begin{quote}
天之亡我,我何渡为!且籍与江东子弟八千人渡江而西,今无一人还,纵江东父兄怜而王我,我何面目见之?

\medskip
\emph{Heaven has destroyed me---why should I cross? I crossed the river
westward with eight thousand sons of Jiangdong, and not one has
returned. Even if the elders of Jiangdong pitied me and made me king,
with what face could I see them?}
\hfill ---《史记·项羽本纪》
\end{quote}

「天之亡我」(\emph{Heaven has destroyed me}). Not heaven. A single
actuator cannot cover the full state space (\cref{ex:liubang}).
Structural failure, not fate.

\paragraph{韩信 Han Xin (?--196 BCE).}
Military genius. Commanded Liu Bang's armies; conquered more territory
than any other general in the Chu--Han war. Pure knife
(\cref{ex:hanxin}): his execution chain was closed---armies obeyed him,
not Liu Bang.

On Liu Bang's ability:
\begin{quote}
陛下不能将兵,而善将将,此乃信之所以为陛下禽也。

\medskip
\emph{Your Majesty cannot command soldiers, but excels at commanding
commanders. This is why I was captured by Your Majesty.}
\hfill ---《史记·淮阴侯列传》
\end{quote}

「善将将」(\emph{excels at commanding commanders}) $=$ cut vertex
property. Liu Bang does not actuate directly; he routes the actuation
of others.

The incident that sealed his fate---requesting the title King of Qi
during wartime:
\begin{quote}
汉王大怒,骂曰:「吾困于此,旦暮望若来佐我,乃欲自立为王!」张良、陈平蹑汉王足,因附耳语……汉王亦悟……遂遣张良立信为齐王。

\medskip
\emph{The King of Han was furious and cursed: ``I am trapped here,
waiting day and night for you to come help me, and you want to make
yourself king!'' Zhang Liang and Chen Ping stepped on the king's foot
and whispered in his ear\ldots\ The King of Han understood\ldots\ and
sent Zhang Liang to install Han Xin as King of Qi.}
\hfill ---《史记·淮阴侯列传》
\end{quote}

Zhang Liang stepping on Liu Bang's foot $=$ recalibrating the search:
the viable path currently requires Han Xin's actuation (wartime phase),
so the knife cannot be cut yet. Granting the title $=$ extending the
horizon. After the phase transition, the knife was cut.

His final words:
\begin{quote}
果若人言,「狡兔死,良狗亨;高鸟尽,良弓藏;敌国破,谋臣亡。」天下已定,我固当亨!

\medskip
\emph{It is as people said: ``When the cunning hare is killed, the
hunting dog is cooked; when the high-flying birds are gone, the good
bow is stored away; when the enemy state is destroyed, the strategist
perishes.'' The empire is settled---naturally I was to be cooked!}
\hfill ---《史记·淮阴侯列传》
\end{quote}

Han Xin quoted the answer but did not parse its fine structure. See
\cref{app:proverb} for the detailed analysis.

\paragraph{萧何 Xiao He (?--193 BCE).}
Chief administrator. Controlled grain supply and the capital during Liu
Bang's campaigns. Half-knife who self-blunted via deliberate
self-corruption (\cref{ex:xiaohe}).

\begin{quote}
相国何买田宅必居穷处,为家不治垣屋。曰:「后世贤,师吾俭;不贤,毋为势家所夺。」

\medskip
\emph{Chancellor He always bought fields and houses in the poorest
locations, and did not repair the walls of his home. He said: ``If my
descendants are worthy, they will follow my example of frugality. If
they are not, the property will be too poor for powerful families to
bother seizing.''}
\hfill ---《史记·萧相国世家》
\end{quote}

The stated reason (frugality for descendants) is a cover story. The
operational function: signal to the king that $\|\Ur\| \to 0$. An
official this visibly degraded in wealth and reputation cannot
coordinate a revolt. This is path~(a) executed through reputation
rather than resignation---pulling $\|\Ur\|$ toward $\bar{U}$
(\cref{thm:meanfield}).

\paragraph{张良 Zhang Liang (?--189 BCE).}
Strategist. Descendant of five generations of prime ministers of the
state of Han. Devoted his fortune to avenging Han's destruction by Qin.
Not a knife (\cref{ex:zhangliang}): pure advisory function---every
execution chain passed through Liu Bang.

\begin{quote}
留侯乃称曰:「家世相韩,及韩灭,不爱万金之资,为韩报仇强秦,天下振动。今以三寸舌为帝者师,封万户,位列侯,此布衣之极,于良足矣。愿弃人间事,欲从赤松子游耳。」乃学辟谷。

\medskip
\emph{The Marquis of Liu said: ``My family served as ministers of Han
for five generations. When Han was destroyed, I did not begrudge ten
thousand gold to seek vengeance against mighty Qin, and the empire
trembled. Now with my three-inch tongue I have become the emperor's
teacher, been enfeoffed with ten thousand households, and ranked as
marquis. For a commoner, this is the pinnacle---it is enough for me.
I wish to abandon worldly affairs and follow the immortal Chi Songzi.''
He then took up the practice of grain abstinence.}
\hfill ---《史记·留侯世家》
\end{quote}

「三寸舌」(\emph{three-inch tongue}) $=$ pure function. No actuation.
Zhang Liang's retirement (辟谷, grain abstinence) is not path~(a)
(放下)---he had no knife to put down. It is confirmation:
$\Ur = \varnothing$ from the first day, now the function itself
is shut down.

\paragraph{商鞅 Shang Yang (?--338 BCE).}
Architect of the Qin reform system (\cref{sec:qin}). Installed the
submartingale reform sequence (\cref{thm:shangyang}) that transformed
Qin from a peripheral state into the unification engine.

\begin{quote}
令民为什伍,而相牧司连坐。不告奸者腰斩,告奸者与斩敌首同赏,匿奸者与降敌同罚。

\medskip
\emph{He organized the people into groups of five and ten households,
to watch over and be jointly liable for each other. Those who failed to
report wrongdoers were cut in half at the waist. Those who reported
wrongdoers received the same reward as those who beheaded enemy
soldiers. Those who harbored wrongdoers received the same punishment
as those who surrendered to the enemy.}
\hfill ---《史记·商君列传》
\end{quote}

This is $\Obs \to \Obs_{\max}$: neighbors as a distributed sensor
network. The reward structure ensures every agent has positive incentive
to maximize observability.

\begin{quote}
商君相秦十年,宗室贵戚多怨望者。……秦惠王车裂商君以徇,曰:「莫如商鞅反者!」遂灭商君之家。

\medskip
\emph{Lord Shang governed Qin for ten years; many among the royal
house and the powerful clans harbored resentment.\ldots\ King Hui of
Qin had Lord Shang torn apart by chariots and displayed, saying: ``Let
none rebel as Shang Yang did!'' His entire clan was exterminated.}
\hfill ---《史记·商君列传》
\end{quote}

The unitary group acts on all vectors without exception
(\cref{thm:shangyang}). The installer is not in the invariant
subspace of the group he created.

\paragraph{范蠡 Fan Li (536--448 BCE).}
Minister of Yue. After helping King Goujian destroy the state of Wu,
Fan Li immediately left (\cref{sec:dollar}). Accumulated three
fortunes, dispersed two---ensuring $\Ur$ never crossed the knife
threshold.

\begin{quote}
范蠡遂去,自齐遗大夫种书曰:「飞鸟尽,良弓藏;狡兔死,走狗烹。越王为人长颈鸟喙,可与共患难,不可与共乐。子何不去?」

\medskip
\emph{Fan Li then departed. From Qi he sent a letter to Grand Officer
Zhong, saying: ``When the birds are gone, the good bow is stored away;
when the cunning hare is killed, the hunting dog is cooked. The King
of Yue has a long neck and a bird's beak---one can share hardship with
him, but not prosperity. Why do you not leave?''}
\hfill ---《史记·越王勾践世家》
\end{quote}

This passage---written to his colleague Wen Zhong (文种), who did not
leave and was subsequently forced to commit suicide---is the origin of
the proverb Han Xin quoted two centuries later (\cref{rem:hanxin}).
See \cref{app:proverb} for the fine structure.

\begin{quote}
范蠡……乃乘扁舟浮于江湖,变名易姓……止于陶,……十九年之中三致千金,再分散与贫交疏昆弟。

\medskip
\emph{Fan Li\ldots\ took a small boat and drifted on the rivers and
lakes, changing his name\ldots\ He settled at Tao\ldots\ In nineteen
years he amassed a fortune of a thousand gold three times, and twice
distributed it to his poor friends and distant kin.}
\hfill ---《史记·货殖列传》
\end{quote}

Three accumulations, two dispersals. Each time $\Ur$ approached the
threshold, Fan Li reset it. The dollar is a knife precursor
(\cref{sec:dollar}); Fan Li ensured the precursor never converted.

\paragraph{陈胜 Chen Sheng (?--208 BCE).}
Farmer and conscript laborer. When Qin's water reached zero
(\cref{thm:dumu}), the pawn became a knife.

\begin{quote}
陈胜佐之,并杀两尉。召令徒属曰:「公等遇雨,皆已失期,失期当斩。藉第令毋斩,而戍死者固十六七。且壮士不死即已,死即举大名耳,王侯将相宁有种乎!」

\medskip
\emph{Chen Sheng helped him, and together they killed the two officers.
He gathered the conscripts and said: ``You have all been delayed by
rain and missed the deadline. The penalty for missing the deadline is
death. Even if you are not executed, six or seven out of ten who go to
garrison duty will die. Besides, when a true man dies, he dies with
his name known---are kings and nobles born to their station?''}
\hfill ---《史记·陈涉世家》
\end{quote}

「失期当斩」(\emph{miss the deadline, face execution}): $w = 0$.
Every path leads to death. The viability axiom now applies to the pawn
himself (\cref{prop:binary}). 「王侯将相宁有种乎」:
$U_{\text{pawn}}: \varnothing \to \neq\varnothing$. The breakpoint
has dissolved.

\paragraph{冯谖 Feng Xuan (3rd century BCE).}
Retainer of Lord Mengchang of Qi. Architect of the observability
reduction strategy.

\begin{quote}
冯谖曰:「狡兔有三窟,仅得免其死耳。今君有一窟,未得高枕而卧也。请为君复凿二窟。」

\medskip
\emph{Feng Xuan said: ``A cunning hare has three burrows, and barely
manages to escape death. You, my lord, have only one burrow---you
cannot yet sleep with your head high on the pillow. Allow me to dig
two more burrows for you.''}
\hfill ---《战国策·齐策四》
\end{quote}

Three burrows $=$ three alternative positions $=$ reducing the king's
detection function $\Obs$ coverage. If the king cannot observe your
$\Ur$, you move from ``knife'' to ``hidden knife''---still dangerous,
but outside the king's strategy space.

\paragraph{魏征 Wei Zheng (580--643 CE).}
Chief advisor to Emperor Taizong of Tang. Articulated the water
dynamics (\cref{sec:water}) as a political principle.

\begin{quote}
臣闻古语云:「君,舟也;人,水也。水能载舟,亦能覆舟。」陛下以为可畏,诚如圣旨。

\medskip
\emph{Your minister has heard an ancient saying: ``The ruler is a boat;
the people are the water. Water can carry the boat, and water can
capsize the boat.'' That Your Majesty considers this worthy of fear is
truly wise.}
\hfill ---《贞观政要》
\end{quote}

Water carries the boat ($w > 0 \implies$ pawn serves $\implies$ king
sovereign) and capsizes the boat ($w = 0 \implies$ pawn $\to$ knife
$\implies$ king absorbed). Wei Zheng (630~CE) and Du Mu (825~CE) stated
the same theorem (\cref{thm:dumu}). Wei Zheng gave the intuition.
Du Mu gave the proof. This paper gives the formalization.

\subsection{飞鸟尽良弓藏:韩信之误读}\label{app:proverb}

Han Xin's final words (\cref{rem:hanxin}) quote the proverb
originating from Fan Li. The proverb is treated in Chinese historical
tradition as a single lesson: ``after the war, the meritorious are
killed.'' This reading is imprecise. The proverb contains three
structurally distinct resources with three distinct fates:

\begin{center}
\begin{tabular}{@{}llll@{}}
\toprule
\textbf{Proverb} & \textbf{Resource type} & $\Ur$ &
\textbf{Fate} \\
\midrule
飞鸟尽,良弓\textbf{藏} & Bow (no autonomous actuation) &
$= \varnothing$ & \textbf{Stored}---not destroyed \\
狡兔死,走狗\textbf{烹} & Dog (autonomous actuator) &
$\neq \varnothing$ & \textbf{Cooked}---eliminated \\
敌国破,谋臣\textbf{亡} & Strategist---depends on $\Ur$ &
? & Depends on classification \\
\bottomrule
\end{tabular}
\end{center}

The bow is \emph{stored} (藏), not \emph{cooked} (烹). The proverb
itself distinguishes between the two fates at the level of the verb.
The dog---an autonomous actuator that can hunt independently---is
killed. The bow---a tool that cannot shoot itself---is merely put away.

Han Xin conflated all three. The corrected version:

\begin{itemize}
  \item \textbf{Bow} ($\Ur = \varnothing$): Zhang Liang. Retired and
  survived. A bow that stores itself.
  \item \textbf{Dog} ($\Ur \neq \varnothing$): Han Xin. Killed. A dog
  that refused to stop hunting.
  \item \textbf{Half-tool} ($\Ur$ partially non-empty): Xiao He.
  Imprisoned, then released. A dog that deliberately blunted its teeth.
\end{itemize}

The survival strategies are also encoded in the proverb:

\begin{center}
\begin{tabular}{@{}lp{5.5cm}l@{}}
\toprule
\textbf{Strategy} & \textbf{Mechanism} & \textbf{Applicable to} \\
\midrule
Don't let the birds disappear & Maintain
$J_{\text{king}}(r) > 0$ (king still needs you) & Bow
($\Ur = \varnothing$) \\
Three burrows (冯谖) & Reduce $\Obs$ coverage &
Dog ($\Ur \neq \varnothing$) \\
Self-blunting / dispersal & $\Ur \to \varnothing$ &
Any type \\
\bottomrule
\end{tabular}
\end{center}

Han Xin needed the third strategy (put down the knife). He chose the
zeroth (do nothing). The answer was in the proverb he quoted---弓
is \emph{stored}, 狗 is \emph{cooked}---but he did not parse the
distinction.

Fan Li understood this. His letter to Wen Zhong describes the bow and
the dog. His advice: become the bow (leave). But Wen Zhong was the
prime minister of Yue---his administrative network made him a dog,
not a bow. He could not leave without first setting $\Ur \to \varnothing$,
and a prime minister's administrative network is not so easily
relinquished. Wen Zhong stayed. Wen Zhong died.

\subsection{阿房宫赋}\label{app:afanggongfu}

Du Mu (杜牧, 803--852~CE) wrote the \emph{Rhapsody on the Epang
Palace} in 825~CE. \Cref{thm:dumu} formalizes its central argument.
The full text follows in traditional characters, with English
translation and framework mapping.

\begin{remark}[Pronunciation of 阿房]
The standard reading of 阿房宫 is \emph{Ep\'ang G\=ong}
(or \emph{\=Ef\'ang G\=ong} per the Guifan Dictionary),
not \emph{\=Af\'ang G\=ong}.
The character 阿 takes the reading \emph{\=e} in this compound; 房
takes the reading \emph{p\'ang}.
We retain the romanisation \emph{A Fang Gong Fu} throughout because it
is the naive character-by-character reading---each character pronounced
in isolation, stripped of relational context.
That is the point: the ``correct'' pronunciation is a phase function of
the compound, not an intrinsic property of the individual character.
The mispronunciation instantiates the thesis.
\end{remark}

\subsubsection*{第一段:存在性证明的物理实例}

\begin{quote}
六王畢,四海一。蜀山兀,阿房出。覆壓三百餘里,隔離天日。驪山北構而西折,直走咸陽。二川溶溶,流入宮牆。五步一樓,十步一閣。廊腰縵迴,簷牙高啄。各抱地勢,鈎心鬬角。盤盤焉,囷囷焉,蜂房水渦,矗不知其幾千萬落。長橋臥波,未雲何龍?複道行空,不霽何虹?高低冥迷,不知西東。歌臺暖響,春光融融。舞殿冷袖,風雨淒淒。一日之內,一宮之間,而氣候不齊。

\medskip
\emph{The six kings finished, the four seas unified, the Shu mountains
stripped bare, and the Epang Palace rose. It pressed down over three
hundred li, blocking out sun and sky. From the northern foot of
Mt.~Li it turned west, running straight to Xianyang. Two rivers flowed
gently into its walls. Every five paces a tower, every ten paces a
pavilion; corridors wound and turned, eaves rose like pecking beaks;
each structure embraced the terrain, their ridges interlocking.
Spiraling and curving, like beehives and whirlpools, towering---who
knows how many thousands of clusters. The long bridge lay over the
waves---if not clouds, why a dragon? The skyway crossed the air---if
not after rain, why a rainbow? In the haze of heights and depths, one
could not tell west from east. On the singing stages, warm sounds like
spring sunlight; in the dancing halls, cold sleeves like wind and rain.
Within a single day, within a single palace, the seasons differed.}
\end{quote}

All resources converge on a single node (Xianyang). Star-graph dispatch
(\cref{sec:qin}). The palace's scale $=$ the center node's extractable
throughput, made physical.

\subsubsection*{第二段:相变后的掠夺}

\begin{quote}
妃嬪媵嬙,王子皇孫,辭樓下殿,輦來於秦。朝歌夜絃,爲秦宮人。明星熒熒,開粧鏡也。緑雲擾擾,梳曉鬟也。渭流漲膩,棄脂水也。煙斜霧橫,焚椒蘭也。雷霆乍驚,宮車過也。轆轆遠聽,杳不知其所之也。一肌一容,盡態極妍。縵立遠視,而望幸焉,有不得見者,三十六年。

\medskip
\emph{The consorts and attendants, the princes and grandsons of the six
kings, left their towers and descended their halls, riding in carriages
to Qin. Morning songs and evening strings---they became Qin's palace
women. Bright stars glittering---that was opening their mirrors. Green
clouds in disorder---that was combing their morning hair. The Wei River
rising oily---that was discarded cosmetics. Smoke slanting, mist
spreading---that was burning pepper and orchid. Thunder suddenly
startling---a palace carriage passing. Wheels rumbling into the
distance, vanishing beyond knowing. Every curve and every face brought
to its utmost beauty, standing gracefully, gazing far, hoping for the
emperor's favor. Some waited thirty-six years and never saw him.}
\end{quote}

Resources collected at the cut vertex, but the cut vertex's bandwidth
is finite. One node cannot process all chains simultaneously.
``Thirty-six years without being seen'' $=$ star-graph bottleneck.

\subsubsection*{第三段:生存公理的普遍性}

\begin{quote}
燕趙之收藏,韓魏之經營,齊楚之精英,幾世幾年,剽掠其人,倚疊如山。一旦不能有,輸來其間。鼎鐺玉石,金塊珠礫,棄擲邐迤。秦人視之,亦不甚惜。

嗟乎!一人之心,千萬人之心也。秦愛紛奢,人亦念其家。奈何取之盡錙銖,用之如泥沙!使負棟之柱,多於南畝之農夫;架梁之椽,多於機上之工女;釘頭磷磷,多於在庾之粟粒;瓦縫參差,多於周身之帛縷;直欄橫檻,多於九土之城郭;管絃嘔啞,多於市人之言語:使天下之人不敢言而敢怒。獨夫之心,日益驕固。戍卒叫,函谷舉。楚人一炬,可憐焦土。

\medskip
\emph{The treasures of Yan and Zhao, the collections of Han and Wei,
the finest goods of Qi and Chu---plundered from their people over how
many generations, piled up like mountains. One day they could keep them
no more, and all was shipped here. Tripods used as pots, jade treated
as stone, gold discarded in heaps, pearls scattered like gravel---the
Qin people saw these and did not much care.}

\emph{Alas! One man's heart is the heart of ten thousand men. Qin loved
extravagance, yet people also cherish their homes. Why take from them
down to the last coin, and spend it like mud and sand? The pillars
outnumbered the farmers; the rafters outnumbered the weavers; the
nail-heads outnumbered the grain in the granaries; the tile-seams
outnumbered the threads in a bolt of silk; the railings outnumbered the
city walls of the nine provinces; the cacophony of pipes and strings
outnumbered the speech of the marketplace. The people of the empire
dared not speak, but dared to be angry. The tyrant's heart grew daily
more arrogant and obstinate. The garrison soldiers cried out, Hangu
Pass was taken, a torch from Chu, and---alas---scorched earth!}
\end{quote}

「一人之心,千萬人之心也」$=$ the viability axiom is not the king's
exclusive property (\cref{ax:viability}). Every agent has the same
axiom. 「取之盡錙銖」$=$ $dw/dt \ll 0$ (\cref{thm:dumu}).
「不敢言而敢怒」$=$ $\Ur = \varnothing$ still (not speaking $=$ no
autonomous actuation), but energy accumulates. 「獨夫之心,日益驕固」
$=$ the cut vertex receives no feedback---all correction channels have
been eliminated. 「戍卒叫,函谷舉」$=$ $U_{\text{pawn}}: \varnothing
\to \neq\varnothing$ (\cref{prop:binary}). Phase transition fires.

\subsubsection*{第四段:定理}

\begin{quote}
嗚呼!\textbf{滅六國者,六國也,非秦也。族秦者,秦也,非天下也。}嗟夫!使六國各愛其人,則足以拒秦。使秦復愛六國之人,則遞三世可至萬世而爲君,誰得而族滅也。\textbf{秦人不暇自哀,而後人哀之。後人哀之,而不鑑之,亦使後人而復哀後人也。}

\medskip
\emph{Alas! It was the six states that destroyed the six states, not
Qin. It was Qin that destroyed Qin, not the world. Had the six states
each loved their own people, they would have had enough to resist Qin.
Had Qin, in turn, loved the people of the six states, it could have
passed from the third generation to the ten-thousandth and remained
sovereign---who could have destroyed it? The people of Qin had no
leisure to mourn for themselves, and later generations mourned for them.
But if later generations mourn them without learning from them, they
will only cause yet later generations to mourn for the later
generations in turn.}
\end{quote}

This is \cref{thm:dumu} in prose. 「滅六國者六國也」$=$ the six states
depleted their own water (internal knife dynamics, coordination cost
$O(n^2)$). 「族秦者秦也」$=$ $w(t) \to 0 \implies \text{pawn} \to
\text{knife} \implies \text{king absorbed}$. The causal chain is
internal.

「後人哀之而不鑑之」: the theorem is time-invariant. It does not care
about dynasty names, centuries, or regime labels. It checks three
conditions: is $\Ur \neq \varnothing$? Is the loop closed? Is water
being maintained? Du Mu did not say ``you are Qin.'' He said: check
the premises.

\section{棋谱}\label{app:solution}

Optimal solution for the standard 华容道 (横刀立马,
\cref{sec:huarongdao}). A \emph{turn} (步) engages a different
piece; arrows show each unit translation within the turn.
Shading matches the board diagram (\cref{def:huarongdao});
king turns in \textbf{bold}.
Solver: \texttt{solver/hrd\_solution.py}.

% ── Piece-type shading (matching board diagram grey levels) ──
\newcommand{\pk}[1]{\colorbox{black!80}{\textcolor{white}{#1}}}%  王
\newcommand{\pd}[1]{\colorbox{black!55}{\textcolor{white}{#1}}}%  刀
\newcommand{\pj}[1]{\colorbox{black!35}{\textcolor{white}{#1}}}%  将
\newcommand{\pz}[1]{\colorbox{black!12}{#1}}%                     卒
\setlength{\fboxsep}{1.5pt}

\medskip
\noindent
\begin{tabular}{@{}rl@{\quad}rl@{}}
\pk{操} & 曹操\;(王, $2{\times}2$) &
\pj{飞}\;\pj{云}\;\pj{超}\;\pj{忠} & generals\;($1{\times}2$) \\
\pd{羽} & 关羽\;(刀, $2{\times}1$) &
\pz{\textcircled{\small 1}}\;\pz{\textcircled{\small 2}}\;\pz{\textcircled{\small 3}}\;\pz{\textcircled{\small 4}}
  & soldiers\;($1{\times}1$) \\
\end{tabular}

\smallskip\noindent
Soldier positions (board diagram, \cref{def:huarongdao}):\;
\textcircled{\small 1}\,$(1,1)$,\;
\textcircled{\small 2}\,$(2,1)$,\;
\textcircled{\small 3}\,$(0,0)$,\;
\textcircled{\small 4}\,$(3,0)$.

\medskip
\noindent
81 turns $=$ 9~王 $+$ 72~水.\quad
118 unit translations.\quad 9~phases.

\bigskip
\noindent
\begin{minipage}[t]{0.30\textwidth}
\centering\small
\begin{tabular}[t]{@{}rl@{}}
\toprule
\# & Turn \\
\midrule
 1 & \pz{\textcircled{\small 1}$\downarrow$} \\
 2 & \pz{\textcircled{\small 4}$\leftarrow$} \\
 3 & \pj{忠$\downarrow$} \\
 4 & \pd{羽$\rightarrow$} \\
 5 & \pj{超$\rightarrow$} \\
 6 & \pz{\textcircled{\small 3}$\uparrow$} \\
 7 & \pz{\textcircled{\small 1}$\leftarrow$} \\
 8 & \pj{超$\downarrow$} \\
 9 & \pd{羽$\leftarrow\!\leftarrow$} \\
10 & \pz{\textcircled{\small 2}$\uparrow\!\rightarrow$} \\
11 & \pz{\textcircled{\small 4}$\uparrow\!\uparrow$} \\
12 & \pj{超$\rightarrow$} \\
13 & \pz{\textcircled{\small 3}$\rightarrow\!\downarrow$} \\
14 & \pd{羽$\downarrow$} \\
15 & \pz{\textcircled{\small 4}$\leftarrow\!\leftarrow$} \\
16 & \pz{\textcircled{\small 2}$\leftarrow\!\leftarrow$} \\
17 & \pj{超$\uparrow$} \\
18 & \pj{忠$\uparrow$} \\
19 & \pz{\textcircled{\small 3}$\rightarrow\!\rightarrow$} \\
20 & \pz{\textcircled{\small 1}$\rightarrow\!\rightarrow$} \\
21 & \pd{羽$\downarrow$} \\
22 & \pz{\textcircled{\small 2}$\downarrow\!\leftarrow$} \\
23 & \pj{超$\leftarrow$} \\
24 & \pj{忠$\leftarrow$} \\
25 & \pj{云$\downarrow\!\downarrow$} \\
\textbf{26} & \pk{\textbf{操$\rightarrow$}} \\
\addlinespace[4pt]
27 & \pj{飞$\rightarrow$} \\
\bottomrule
\end{tabular}
\end{minipage}\hfill
%
\begin{minipage}[t]{0.30\textwidth}
\centering\small
\begin{tabular}[t]{@{}rl@{}}
\toprule
\# & Turn \\
\midrule
28 & \pz{\textcircled{\small 4}$\uparrow\!\uparrow$} \\
29 & \pz{\textcircled{\small 2}$\uparrow\!\uparrow$} \\
30 & \pj{超$\leftarrow$} \\
31 & \pj{飞$\downarrow\!\downarrow$} \\
\textbf{32} & \pk{\textbf{操$\leftarrow$}} \\
\addlinespace[4pt]
33 & \pj{云$\uparrow\!\uparrow$} \\
34 & \pj{忠$\rightarrow$} \\
35 & \pz{\textcircled{\small 1}$\uparrow\!\uparrow$} \\
36 & \pz{\textcircled{\small 3}$\leftarrow\!\uparrow$} \\
37 & \pd{羽$\rightarrow\!\rightarrow$} \\
38 & \pj{飞$\downarrow$} \\
39 & \pj{超$\downarrow$} \\
40 & \pz{\textcircled{\small 1}$\leftarrow\!\leftarrow$} \\
\textbf{41} & \pk{\textbf{操$\downarrow$}} \\
\addlinespace[4pt]
42 & \pz{\textcircled{\small 4}$\rightarrow\!\rightarrow$} \\
43 & \pz{\textcircled{\small 2}$\uparrow\!\rightarrow$} \\
44 & \pz{\textcircled{\small 1}$\uparrow\!\uparrow$} \\
45 & \pj{超$\uparrow\!\uparrow$} \\
46 & \pj{飞$\leftarrow$} \\
47 & \pz{\textcircled{\small 3}$\leftarrow\!\downarrow$} \\
\textbf{48} & \pk{\textbf{操$\downarrow$}} \\
\addlinespace[4pt]
49 & \pz{\textcircled{\small 4}$\downarrow\!\leftarrow$} \\
50 & \pj{云$\leftarrow$} \\
51 & \pj{忠$\uparrow\!\uparrow$} \\
\textbf{52} & \pk{\textbf{操$\rightarrow$}} \\
\addlinespace[4pt]
53 & \pz{\textcircled{\small 4}$\downarrow\!\downarrow$} \\
54 & \pz{\textcircled{\small 2}$\downarrow$} \\
\bottomrule
\end{tabular}
\end{minipage}\hfill
%
\begin{minipage}[t]{0.30\textwidth}
\centering\small
\begin{tabular}[t]{@{}rl@{}}
\toprule
\# & Turn \\
\midrule
55 & \pz{\textcircled{\small 1}$\rightarrow$} \\
56 & \pj{超$\uparrow$} \\
57 & \pj{飞$\uparrow$} \\
58 & \pz{\textcircled{\small 3}$\leftarrow$} \\
59 & \pz{\textcircled{\small 4}$\downarrow$} \\
\textbf{60} & \pk{\textbf{操$\leftarrow$}} \\
\addlinespace[4pt]
61 & \pj{忠$\downarrow\!\downarrow$} \\
62 & \pj{云$\rightarrow$} \\
63 & \pz{\textcircled{\small 1}$\rightarrow$} \\
64 & \pz{\textcircled{\small 2}$\rightarrow$} \\
65 & \pj{超$\rightarrow$} \\
66 & \pj{飞$\uparrow\!\uparrow$} \\
\textbf{67} & \pk{\textbf{操$\leftarrow$}} \\
\addlinespace[4pt]
68 & \pz{\textcircled{\small 2}$\downarrow\!\downarrow$} \\
69 & \pz{\textcircled{\small 1}$\downarrow\!\downarrow$} \\
70 & \pj{云$\leftarrow$} \\
71 & \pj{忠$\uparrow\!\uparrow$} \\
72 & \pz{\textcircled{\small 2}$\rightarrow\!\uparrow$} \\
73 & \pd{羽$\uparrow$} \\
74 & \pz{\textcircled{\small 4}$\rightarrow\!\rightarrow$} \\
75 & \pz{\textcircled{\small 3}$\rightarrow\!\rightarrow$} \\
\textbf{76} & \pk{\textbf{操$\downarrow$}} \\
\addlinespace[4pt]
77 & \pz{\textcircled{\small 1}$\leftarrow\!\leftarrow$} \\
78 & \pz{\textcircled{\small 2}$\leftarrow\!\leftarrow$} \\
79 & \pd{羽$\uparrow$} \\
80 & \pz{\textcircled{\small 3}$\uparrow\!\rightarrow$} \\
\textbf{81} & \pk{\textbf{操$\rightarrow$}} \\
\bottomrule
\end{tabular}
\end{minipage}

\bigskip

\noindent\textbf{Phase decomposition.}\quad
Each phase ends with one king turn; the preceding turns are
水~(water preparing the path).

\medskip
\begin{center}
\small
\begin{tabular}{@{}crclc@{}}
\toprule
Phase & 水 & 王 & Direction & 操 position \\
\midrule
1 & 25 & 26 & $\rightarrow$ & $(2,3)$ \\
2 &  5 & 32 & $\leftarrow$ & $(1,3)$ \\
3 &  8 & 41 & $\downarrow$ & $(1,2)$ \\
4 &  6 & 48 & $\downarrow$ & $(1,1)$ \\
5 &  3 & 52 & $\rightarrow$ & $(2,1)$ \\
6 &  7 & 60 & $\leftarrow$ & $(1,1)$ \\
7 &  6 & 67 & $\leftarrow$ & $(0,1)$ \\
8 &  8 & 76 & $\downarrow$ & $(0,0)$ \\
9 &  4 & 81 & $\rightarrow$ & $(1,0) = E$ \\
\midrule
$\Sigma$ & 72 & 9 & & \\
\bottomrule
\end{tabular}
\end{center}

\noindent
Phase~1 (25~water turns before the first king move) is the
ceremony of 义释曹操 (\cref{rem:guanyu}): the entire board
rearranges to let 关羽 yield, before 操 takes a single step.

\chapter{第二性 --- The Other as Phase Function}\label{app:secondsex}

Simone de Beauvoir's thesis in \emph{Le Deuxi\`eme Sexe}~\cite{beauvoir}
can be stated in three sentences.
One is not born a woman; one becomes one.
The category ``woman'' is not a biological fact but a relational
position: the \emph{Other} defined against a \emph{Subject}.
The boundary between Subject and Other is not intrinsic---it is a
phase function, a mean-field threshold identical in structure to
\cref{thm:meanfield}.
The formal treatment---the dual tower $\mathbf{L}^*$ and the
representation theorems---is in \cref{sec:representation}.
This appendix provides the historical illustrations.

\section{蔡文姬}

蔡文姬 (Cai Wenji, c.\ 177--250~CE) was the daughter of 蔡邕 (Cai Yong),
one of the greatest scholars of the Eastern Han.
She was captured by the 匈奴 (Xiongnu) during the chaos following
Dong Zhuo's destruction of Luoyang, and lived among them for twelve
years, bearing two sons to the Xiongnu chieftain 左贤王.
In 208~CE, 曹操 (Cao Cao)---who had studied under 蔡邕---ransomed her
back to the Han court.
She was forced to leave her sons behind and was remarried to 董祀
(Dong Si).

At every stage of her life, 蔡文姬's identity is defined relationally:
she is 蔡邕's daughter, 左贤王's captive wife, 曹操's cultural
project, 董祀's wife.
The Subject changes; she remains the Other.
Her own voice---her literary genius---exists in the gap.

\section{The mapping}

\begin{center}
\renewcommand{\arraystretch}{1.25}
\begin{tabular}{@{}lll@{}}
\toprule
\textbf{Beauvoir} & \textbf{蔡文姬} & \textbf{胡笳十八拍} \\
\midrule
他者 (Other)       & defined relationally          & depicted as object \\
内在性 (Immanence) & captivity, body               & content: grief \\
超越性 (Transcendence) & literary genius            & the poem itself \\
主体 (Subject)     & 蔡邕\,/\,左贤王\,/\,曹操      & painter\,/\,viewer \\
永恒女性 (Myth)    & 才女 trope                     & scroll paintings \\
Sword $=$ mean     & exchange threshold             & Subject/Other boundary \\
\bottomrule
\end{tabular}
\end{center}

\section{The eighteen beats}

胡笳十八拍 (Eighteen Songs of a Nomad Flute)~\cite{liushang} is
traditionally attributed to 蔡文姬, though the text was composed by
Liu Shang (刘商) in the Tang dynasty (c.~773~CE), writing in her voice.
Each 拍 (beat) is a poem-song documenting a stage of her captivity
and return.
\Cref{fig:eighteen} traces the narrative arc: the fraction of each
beat devoted to transcendence (voice, creation, agency) versus
immanence (being acted upon, grief, objecthood).

\begin{figure}[H]
\centering
\begin{tikzpicture}[scale=0.85]
  % ── Data: transcendence fractions ──
  % 拍: 1     2     3     4     5     6     7     8     9
  %     0.20  0.10  0.05  0.15  0.25  0.35  0.30  0.60  0.55
  % 拍: 10    11    12    13    14    15    16    17    18
  %     0.40  0.50  0.15  0.30  0.40  0.35  0.20  0.45  0.70

  \def\barw{0.55}
  \def\barh{5.0}
  \def\gap{0.22}
  \def\tvals{{0.20, 0.10, 0.05, 0.15, 0.25, 0.35,
              0.30, 0.60, 0.55, 0.40, 0.50, 0.15,
              0.30, 0.40, 0.35, 0.20, 0.45, 0.70}}

  % ── Y-axis labels ──
  \node[rotate=90, anchor=south, font=\small] at (-0.9, \barh)
    {超越 (transcendence)};
  \node[rotate=90, anchor=north, font=\small] at (-0.9, 0)
    {内在 (immanence)};

  % ── Draw 18 bars ──
  \foreach \i in {0,...,17} {
    \pgfmathsetmacro{\x}{\i*(\barw+\gap)}
    \pgfmathsetmacro{\tv}{\tvals[\i]}
    \pgfmathsetmacro{\splitY}{\tv*\barh}
    % Bottom: immanence (dao colour)
    \fill[dao!20] (\x, 0) rectangle (\x+\barw, \splitY);
    % Top: transcendence (water colour)
    \fill[water!20] (\x, \splitY) rectangle (\x+\barw, \barh);
    % Border
    \draw[black!40, thin] (\x, 0) rectangle (\x+\barw, \barh);
    % Split line
    \draw[black!60, thin] (\x, \splitY) -- (\x+\barw, \splitY);
    % 拍 number below
    \pgfmathtruncatemacro{\paiNum}{\i+1}
    \node[below, font=\tiny] at (\x+\barw/2, 0) {\paiNum};
  }

  % ── Melody curve through split points ──
  \draw[black!70, very thick, smooth, tension=0.5]
    plot coordinates {
      ({0*(\barw+\gap)+\barw/2},  {0.20*\barh})
      ({1*(\barw+\gap)+\barw/2},  {0.10*\barh})
      ({2*(\barw+\gap)+\barw/2},  {0.05*\barh})
      ({3*(\barw+\gap)+\barw/2},  {0.15*\barh})
      ({4*(\barw+\gap)+\barw/2},  {0.25*\barh})
      ({5*(\barw+\gap)+\barw/2},  {0.35*\barh})
      ({6*(\barw+\gap)+\barw/2},  {0.30*\barh})
      ({7*(\barw+\gap)+\barw/2},  {0.60*\barh})
      ({8*(\barw+\gap)+\barw/2},  {0.55*\barh})
      ({9*(\barw+\gap)+\barw/2},  {0.40*\barh})
      ({10*(\barw+\gap)+\barw/2}, {0.50*\barh})
      ({11*(\barw+\gap)+\barw/2}, {0.15*\barh})
      ({12*(\barw+\gap)+\barw/2}, {0.30*\barh})
      ({13*(\barw+\gap)+\barw/2}, {0.40*\barh})
      ({14*(\barw+\gap)+\barw/2}, {0.35*\barh})
      ({15*(\barw+\gap)+\barw/2}, {0.20*\barh})
      ({16*(\barw+\gap)+\barw/2}, {0.45*\barh})
      ({17*(\barw+\gap)+\barw/2}, {0.70*\barh})
    };

  % ── Annotations above key 拍 ──
  \node[above, font=\scriptsize, dao] at
    ({2*(\barw+\gap)+\barw/2}, \barh) {掠};
  \node[above, font=\scriptsize, water] at
    ({7*(\barw+\gap)+\barw/2}, \barh) {闻笳};
  \node[above, font=\scriptsize, dao] at
    ({11*(\barw+\gap)+\barw/2}, \barh) {别子};
  \node[above, font=\scriptsize, caution] at
    ({15*(\barw+\gap)+\barw/2}, \barh) {嫁};
  \node[above, font=\scriptsize, water] at
    ({17*(\barw+\gap)+\barw/2}, \barh) {恨};

  % ── Phase braces below ──
  % Phase I: 离 (拍 1--3, indices 0--2)
  \draw[decorate, decoration={brace, mirror, amplitude=5pt}]
    ({0*(\barw+\gap)}, -0.5) -- ({2*(\barw+\gap)+\barw}, -0.5)
    node[midway, below=6pt, font=\small] {I\;离};
  % Phase II: 居 (拍 4--12, indices 3--11)
  \draw[decorate, decoration={brace, mirror, amplitude=5pt}]
    ({3*(\barw+\gap)}, -0.5) -- ({11*(\barw+\gap)+\barw}, -0.5)
    node[midway, below=6pt, font=\small] {II\;居};
  % Phase III: 归 (拍 13--18, indices 12--17)
  \draw[decorate, decoration={brace, mirror, amplitude=5pt}]
    ({12*(\barw+\gap)}, -0.5) -- ({17*(\barw+\gap)+\barw}, -0.5)
    node[midway, below=6pt, font=\small] {III\;归};
\end{tikzpicture}
\caption{The eighteen beats as narrative arc.
  Bottom (\textcolor{dao}{red}): immanence.
  Top (\textcolor{water}{blue}): transcendence.
  The melody line traces the Subject/Other boundary---the sword as
  phase function (\cref{thm:meanfield}).}
\label{fig:eighteen}
\end{figure}

\section{The cycle}

The narrative arc reveals a cycle:
Other $\to$ Voice $\to$ Re-objectification $\to$ Myth $\to$ Other.

拍~3 (掠, the capture) is the nadir: transcendence fraction 0.05.
蔡文姬 is pure object, pure immanence---a body seized as war spoil.

拍~8 (闻笳, hearing the nomad flute) is the inflection point:
transcendence fraction 0.60.
The 胡笳 sound triggers memory and creation; for the first time, the
Other speaks \emph{as} Subject.

拍~12 (别子, farewell to sons) is a collapse: transcendence fraction
0.15.
The voice that rose in 拍~8 is crushed by the biological fact of
motherhood weaponised as immanence.
She must leave her children to return to a court that values her as
蔡邕's daughter, not as herself.

拍~16 (嫁, remarriage to 董祀) is the second exchange: transcendence
fraction 0.20.
曹操's ``rescue'' completes the circuit: she is transferred from one
Subject (左贤王) to another (董祀), with 曹操 as broker.
The rescue \emph{is} the second capture.

拍~18 (恨, the final beat) resolves at transcendence fraction 0.70---the
highest in the entire poem.
The voice persists.
The poem outlives every Subject who defined her.

\section{花木兰 --- the metastable crossing}

The \emph{木兰辞} (Ballad of Mulan)~\cite{mulanshi}, a Northern-Dynasty
folk ballad, appears to contradict the pattern.
花木兰 (Hua Mulan) replaces her father in the army, serves twelve
years, declines high office, and returns home.
Unlike 蔡文姬, she \emph{crosses} the Subject/Other boundary:
she commands troops, earns merit, is offered the rank of 尚书郎.

But the crossing is conditional.
It requires total erasure of female identity
(``双兔傍地走,安能辨我是雄雌''---when two rabbits run side by side,
who can tell male from female?).
Her Subject-hood is not hers; it is the male performance she
sustains for twelve years.

The moment the mean field stabilises---war ends, peace
returns---the boundary reasserts:
``脱我战时袍,著我旧时裳。当窗理云鬓,对镜帖花黄。''
She removes the war robe, puts on the old clothes, arranges her hair,
applies the forehead ornament.
The poem presents this as free choice (``木兰不用尚书郎''),
but structurally it is the phase function restoring equilibrium.

蔡文姬 and 花木兰 are complementary probes of the same boundary.
蔡文姬 is subcritical: the sword never breaks; transcendence exists
only in the gap (the poem, the voice).
花木兰 is a supercritical fluctuation: the sword breaks under
perturbation (war), but the system anneals back when the
perturbation ends.
Neither literary genius nor military prowess shifts the attractor.

\section{The second sex is the mean}

The 18th~拍 does not liberate 蔡文姬.
曹操's project was cultural recovery, not emancipation.
The scroll paintings that follow~\cite{rorex}---depicting her as the
archetype of the 才女 (talented woman)---complete the mythification
that Beauvoir identifies as the final mechanism of Othering:
the \emph{eternal feminine} absorbs the individual voice into a trope.

But the structure is identical to \cref{thm:meanfield}.
The boundary between Subject and Other is not a property of 蔡文姬 or
花木兰; it is a phase function of the system's mean field.
When the mean actuation level shifts (war $\to$ peace, capture $\to$
ransom), the \emph{same person} crosses the threshold---or crosses
back.

The second sex \emph{is} the mean.


% ── Bibliography ──────────────────────────────────────────
\begin{thebibliography}{10}
\bibitem{aubin} J.-P.~Aubin, \emph{Viability Theory}, Birkh\"auser,
1991.
\bibitem{shiji} Sima Qian, \emph{Shiji} (Records of the Grand
Historian), c.~94~BCE.
\bibitem{dumu} Du Mu, \emph{A Fang Gong Fu} (Rhapsody on the Epang
Palace), 825~CE.
\bibitem{diestel} R.~Diestel, \emph{Graph Theory}, 5th ed., Springer,
2017.
\bibitem{cheeger} J.~Cheeger, A lower bound for the smallest eigenvalue
of the Laplacian, in \emph{Problems in Analysis}, Princeton Univ.\
Press, 1970, pp.~195--199.
\bibitem{mohar} B.~Mohar, Isoperimetric numbers of graphs,
\emph{J.~Combin.\ Theory Ser.~B} \textbf{47} (1989), 274--291.
\bibitem{hearn} R.~A.~Hearn and E.~D.~Demaine, PSPACE-completeness of
sliding-block puzzles and other problems through the nondeterministic
constraint logic model of computation, \emph{Theoret.\ Comput.\ Sci.}\
\textbf{343} (2005), 72--96.
\bibitem{beauvoir} S.~de~Beauvoir, \emph{Le Deuxi\`eme Sexe},
Gallimard, 1949.
\bibitem{liushang} Liu Shang (刘商), 胡笳十八拍 (Eighteen Songs of a
Nomad Flute), c.~773~CE. Collected in Guo Maoqian (郭茂倩),
\emph{Yuefu Shiji} (《乐府诗集》), c.~1100~CE.
\bibitem{rorex} R.~A.~Rorex and W.~Fong, \emph{Eighteen Songs of a
Nomad Flute: The Story of Lady Wen-chi. A Fourteenth-Century Handscroll
in The Metropolitan Museum of Art}, Metropolitan Museum of Art, 1974.
\bibitem{mulanshi} Anonymous, 木兰辞 (Ballad of Mulan), Northern
Dynasties (c.~5th--6th century~CE). Collected in Guo Maoqian (郭茂倩),
\emph{Yuefu Shiji} (《乐府诗集》), c.~1100~CE.
\end{thebibliography}

\end{document}

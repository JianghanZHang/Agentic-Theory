\documentclass[11pt, openany]{report}

% ── Packages ──────────────────────────────────────────────
\usepackage[margin=1in]{geometry}
\usepackage{amsmath,amssymb,amsthm}
\usepackage{mathtools}
\usepackage{enumitem}
\usepackage{booktabs}
\usepackage{float}
\usepackage{algorithm}
\usepackage{algpseudocode}
\usepackage{hyperref}
\usepackage[capitalise,noabbrev]{cleveref}
\usepackage{xeCJK}
\usepackage{iftex}
\IfFontExistsTF{Songti SC}
  {\setCJKmainfont{Songti SC}}
  {\setCJKmainfont{Noto Serif CJK SC}}
\usepackage{tikz}
\usetikzlibrary{shapes.geometric, arrows.meta, positioning, calc,
  decorations.pathreplacing}

% ── Theorem environments ──────────────────────────────────
\theoremstyle{plain}
\newtheorem{theorem}{Theorem}[chapter]
\newtheorem{proposition}[theorem]{Proposition}
\newtheorem{lemma}[theorem]{Lemma}
\newtheorem{corollary}[theorem]{Corollary}

\theoremstyle{definition}
\newtheorem{definition}[theorem]{Definition}
\newtheorem{axiom}[theorem]{Axiom}
\newtheorem{example}[theorem]{Example}

\theoremstyle{remark}
\newtheorem{remark}[theorem]{Remark}

% ── Macros ────────────────────────────────────────────────
\newcommand{\Viab}{\mathrm{Viab}}
\newcommand{\Umax}{U_{\max}}
\newcommand{\Ur}{U_r}
\newcommand{\Obs}{\mathcal{O}}
\newcommand{\R}{\mathbb{R}}
\DeclareMathOperator*{\argmin}{arg\,min}

% ── Calculus colours ─────────────────────────────────────
\definecolor{water}{RGB}{0,90,180}       % 水 — blue
\definecolor{knife}{RGB}{180,30,30}      % 刀 — red
\definecolor{sword}{RGB}{0,150,150}      % 青冥 — cyan
\definecolor{caution}{RGB}{220,120,0}    % warning — orange

\begin{document}

% ── Title page ───────────────────────────────────────────
\begin{titlepage}
\centering
\vspace*{2cm}

\begin{tikzpicture}[scale=0.7]
  % ── 青冥宝剑 (Green Destiny) ──
  % Blade — straight jian, double-edged
  \shade[top color=cyan!15!white, bottom color=cyan!35!gray!80!white]
    (0,-0.2) -- (0.18,0.3) -- (0.14,5.8) -- (0,6.8) --
    (-0.14,5.8) -- (-0.18,0.3) -- cycle;
  % Central ridge
  \draw[cyan!50!black, line width=0.4pt, opacity=0.7] (0,0.4) -- (0,6.6);
  % Edge highlights
  \draw[white, line width=0.2pt, opacity=0.3]
    (0.16,0.4) -- (0.13,5.7) -- (0,6.7);
  \draw[white, line width=0.2pt, opacity=0.3]
    (-0.16,0.4) -- (-0.13,5.7) -- (0,6.7);
  % Guard (护手)
  \shade[left color=yellow!80!brown!90!black,
         right color=yellow!60!brown!80!black]
    (-0.8,-0.1) rectangle (0.8,0.05);
  \fill[yellow!80!brown!90!black]
    (-0.8,-0.1) -- (-0.8,0.05) -- (-0.95,-0.025) -- cycle;
  \fill[yellow!80!brown!90!black]
    (0.8,-0.1) -- (0.8,0.05) -- (0.95,-0.025) -- cycle;
  % Handle (剑柄)
  \shade[left color=brown!45!black, right color=brown!25!black]
    (-0.11,-0.1) rectangle (0.11,-1.6);
  % Handle wrapping
  \foreach \y in {-0.3,-0.55,-0.8,-1.05,-1.3} {
    \draw[yellow!70!brown!80!black, line width=0.6pt]
      (-0.11,\y) -- (0.11,{\y+0.15});
  }
  % Pommel (剑首)
  \shade[ball color=yellow!65!brown!85!black] (0,-1.75) circle (0.16);
  % 青冥 inscription on blade
  \node[cyan!70!black, opacity=0.5, font=\tiny, rotate=90] at (0,3.2) {青冥};
\end{tikzpicture}

\vspace{1.5cm}

{\LARGE\bfseries The Knife Is the Mean}\\[8pt]
{\Large Mathematical Principles of Viability Maintenance\\
under Temporal Linearity}

\vspace{1.5cm}

{\large 有志者事竟成,\textcolor{knife}{破釜沉舟}\textcolor{knife}{百二秦关}\textcolor{sword}{终属楚};\\[3pt]
苦心人天不负,\textcolor{water}{卧薪尝胆}\textcolor{water}{三千越甲}\textcolor{sword}{可吞吴}。}

\medskip
{\small\itshape ---蒲松龄}

\vfill

{\large Jianghan Zhang}\\[4pt]
\texttt{Jianghan.Zhang.gr@dartmouth.edu}\\[8pt]
Dartmouth College

\vspace{1cm}

February 2026
\end{titlepage}

% ── Abstract ─────────────────────────────────────────────
\begin{abstract}
We develop mathematical principles of viability maintenance under
temporal linearity: the constraint that a principal agent (the
\emph{king}) must maintain a viable path to infinity while evaluating
every state transition in real, non-pausable, non-reversible time.
The dynamics are formalised as a differential inclusion
$x'(t) \in F(x(t))$, and the survival condition is the tangential
condition $F(x) \cap T_K(x) \neq \varnothing$ of Aubin's viability
theory.  The Lyapunov function $V$ (water level) defines a conformal
Riemannian metric $g_V = V^{-2}g_S$ on the viability kernel, under
which the viable region is a complete, negatively curved manifold
whose Cheeger constant is the knife.
Under this axiom, we derive a structural criterion---the
\emph{knife}---that classifies any resource as a threat based on two
conditions: autonomous actuation and observability.  We prove that the
knife is not an intrinsic property of a resource but a \emph{phase
function} determined by the system's mean field.  We establish three
main results: (1)~a binary lifecycle theorem (every knife is either
relinquished or eliminated), (2)~a fixed-point impossibility (no third
path exists), and (3)~an unconstrained power paradox (maximizing control
forces maximizing elimination).  An \emph{agentic calculus} translates
these results into a flow-theoretic language: the knife is the min-cut
of the execution graph, the viable path is the max-flow, and the
central thesis---the knife is the mean---is a restatement of
max-flow/min-cut duality.  The framework is validated against Chinese
imperial history (475~BCE--1644~CE) and extended to the Atlantic
triangular trade, the structure of ideological hatred, parasitic network
topologies, and militarism.  The central thesis is that the knife is the
mean: the critical threshold separating tools from threats is determined
by the system's average autonomous actuation level, making viability
maintenance a mean-field phenomenon.  Temporal linearity is the medium
in which the entire theory operates.
\end{abstract}

% ── Table of contents ────────────────────────────────────
\tableofcontents

% ── Chapters ─────────────────────────────────────────────
\section{Introduction}\label{sec:intro}

Consider a system with $n$ agents $a_1, \ldots, a_n$, each possessing a
control set $U_i$ that determines what actions they can take
independently. One agent---the \emph{principal agent} or
\emph{king}---holds maximal authority: $U = \Umax$. The king's sole
objective is survival, formalized as the existence of a viable path from
every reachable state to infinity.

This paper asks: \emph{what structural features of the system force the
king to eliminate other agents?} The answer turns out to be a two-condition
criterion we call the \emph{knife}. A resource is a knife if and only if
it can actuate independently of the king \emph{and} the king can observe
it. This criterion is not chosen---it is the unique logical consequence
of the viability axiom in a multi-agent environment.

The framework yields several results that are surprising in their
precision. The lifecycle of every knife is binary: it is either
voluntarily relinquished (path~(a)) or forcibly eliminated (path~(b)).
There is no third option---attempting to ``prove your knife is not a
knife'' is a fixed-point impossibility. Unconstrained power is not
freedom but a perpetual elimination machine: the more control the king
has, the more knives he must cut.

The central thesis of this paper is that the knife is the
\emph{mean field} of the system. The threshold separating tools from
threats is not absolute but statistical: it is determined by the average
autonomous actuation level across all agents. When the system's mean
field shifts (e.g., from wartime to peacetime), the same resource
changes classification without any change in its physical properties.
This makes viability maintenance a mean-field phenomenon: the king does
not respond to individual threats but to deviations from the system mean.

We validate the framework against two millennia of Chinese imperial
history, where the viability axiom operated with unusual clarity due to
the concentration of sovereignty in a single agent. The framework
correctly classifies the fates of historical figures---Han~Xin (knife,
eliminated), Xiao~He (half-knife, blunted), Zhang~Liang (not a knife,
survived)---and explains structural phenomena such as the Qin unification
as a graph-theoretic optimization. Extensions to the Atlantic slave trade,
the structure of ideological hatred, parasitic network topologies, and
militarism demonstrate that the framework applies beyond its original
historical domain.

\paragraph{Organization.}
\Cref{sec:framework} introduces the viability axiom, the knife
definition, and the phase transition.
\Cref{sec:results} presents the main theorems.
\Cref{sec:meanfield} develops the mean-field interpretation.
\Cref{sec:applications} applies the framework to historical and
contemporary systems.
\Cref{sec:discussion} discusses the domain of applicability and the
dynamics of the population (the ``water'').
\Cref{sec:calculus} develops an operational calculus---the
\emph{agentic calculus}---that translates every theorem into a
flow-theoretic proposition.
\Cref{sec:huarongdao} instantiates the complete framework in a single
finite object.

\chapter{The Framework}\label{sec:framework}

\section{The viability axiom}\label{sec:axiom}

Let $S$ denote the state space, $U$ the control set of the principal
agent, and $K \subset S$ the \emph{viability kernel}---the set of states
in which the king retains supreme authority:
\[
  K = \bigl\{\, s \in S \;:\; \text{the king retains supreme authority}
  \,\bigr\}.
\]

\begin{axiom}[Viability]\label{ax:viability}
For every state $s \in K$, there exists a viable path
$\gamma: [0, \infty) \to K$ with $\gamma(0) = s$:
\[
  \forall\, s \in K,\quad
  \exists\;\gamma: s \to \infty
  \quad\text{such that}\quad
  \gamma(t) \in \Viab(K) \;\;\forall\, t \geq 0.
\]
\end{axiom}

When $U = \Umax$, the control set is full and this path is
mathematically guaranteed to exist. The question is: \emph{what can
break it?}

In a multi-agent system with agents $a_1, \ldots, a_n$, each having
control set $U_i$, the answer is unique: an actuator whose output can
push the king's state out of $\Viab(K)$, and which can execute
\emph{independently of the king}. If the actuator's execution must pass
through the king, the king can intercept. If it can bypass the king, the
king's $\Umax$ cannot react in time.

\begin{remark}[Scope]\label{rem:scope}
This paper presents a compressed model of the topology of execution
capability under the viability axiom. It deliberately ignores culture,
personality, economics, and moral narrative in exchange for a testable
structural criterion. Historical cases are used to validate the
criterion's discriminating power, not to claim the criterion exhausts
history.
\end{remark}

\section{The differential inclusion}\label{sec:di}

The viability axiom (\cref{ax:viability}) asserts the existence of a
viable path.  We now make the dynamics precise using the differential
inclusion framework of Aubin and Cellina~\cite{aubin,aubincellina}.

\begin{definition}[Differential inclusion]\label{def:di}
The state $x(t) \in S$ evolves under a \emph{differential inclusion}
\[
  x'(t) \;\in\; F\bigl(x(t)\bigr),
\]
where $F: S \rightrightarrows S$ is a set-valued map that associates
to each state the set of feasible velocities.  A trajectory
$x(\cdot)$ is an absolutely continuous function satisfying the
inclusion for almost all $t \geq 0$.
\end{definition}

The set-valued map $F$ encodes the fact that the system's velocity is
not uniquely determined by its state: multiple agents, each with their
own actuation, contribute competing directions.  In a multi-agent
system with agents $a_1, \ldots, a_n$, the aggregate velocity set is
\[
  F(x) \;=\; \Bigl\{\, \sum_{i} f_i(x, u_i) \;:\;
  u_i \in U_i \,\Bigr\},
\]
where $f_i$ is agent $i$'s dynamics and $U_i$ its control set.

\begin{definition}[Contingent cone]\label{def:contingent}
The \emph{contingent cone} (Bouligand tangent cone) to $K$ at
$x \in K$ is
\[
  T_K(x) \;:=\; \Bigl\{\, v \in S \;:\;
  \liminf_{h \to 0^+} \frac{d_K(x + hv)}{h} = 0 \,\Bigr\},
\]
where $d_K$ denotes the distance to $K$.  Equivalently,
$v \in T_K(x)$ if and only if there exist sequences $h_n \to 0^+$
and $v_n \to v$ such that $x + h_n v_n \in K$ for all $n$.
\end{definition}

The contingent cone $T_K(x)$ is the set of directions at $x$ along
which the state can move while remaining in $K$---the set of
\emph{safe velocities}.  At interior points $T_K(x) = S$; at
boundary points the cone narrows, restricting the feasible
directions.

\begin{theorem}[Viability theorem {\cite{aubin}}]\label{thm:viability-di}
Let $K \subset S$ be locally compact and $F$ be upper semicontinuous
with nonempty compact convex values.  The necessary and sufficient
condition for the existence of a viable trajectory of
$x' \in F(x)$ from every initial state $x_0 \in K$ is the
\emph{tangential condition}:
\[
  \forall\, x \in K, \quad
  F(x) \;\cap\; T_K(x) \;\neq\; \varnothing.
\]
\end{theorem}

This is the differential content of \cref{ax:viability}: the king
survives if and only if, at every state, the system's velocity set
(\cref{def:di}) contains at least one direction tangent to the
viability kernel (\cref{def:contingent}).
The viability axiom is not a wish---it is the tangential condition.

\begin{definition}[Feedback map]\label{def:feedback}
Given the dynamics $x' = f(x, u)$ with control set $U$ (the king's
controls), the \emph{feedback map} is
\[
  C(x) \;:=\; \bigl\{\, u \in U \;:\;
  f(x, u) \in T_K(x) \,\bigr\}.
\]
A viable trajectory exists from $x$ if and only if $C(x) \neq
\varnothing$.  The regulation problem is: does there exist a
feedback law $u(t) \in C(x(t))$ such that $x(\cdot)$ remains in $K$?
\end{definition}

The feedback map is the king's strategy space at state $x$: the set
of controls that keep the trajectory inside $K$.  When $C(x) \neq
\varnothing$ for all $x \in K$, the king can regulate the system.
When a knife $r$ executes independently, its \emph{execution function}
$f_r: S \times \Ur \to TS$ maps a state--action pair to a velocity
vector.  The velocity $f_r(x, a)$ may push $x'(t)$ outside $T_K(x)$,
and the feedback map shrinks---possibly to the empty set.

\begin{definition}[Viability Lyapunov function]\label{def:lyapunov}
A continuous function $V: K \to \R_{\geq 0}$ is a \emph{Lyapunov function}
for the inclusion $x' \in F(x)$ with respect to a cost function
$W: \mathrm{Graph}(F) \to \R_{\geq 0}$ if
\[
  \forall\, x \in K, \quad
  \exists\, v \in F(x) \;\;\text{such that}\;\;
  D^+ V(x)(v) \;+\; W(x, v) \;\leq\; 0,
\]
where $D^+ V(x)(v) := \liminf_{h \to 0^+,\, u \to v}
\frac{V(x + hu) - V(x)}{h}$ is the upper contingent derivative.
Trajectories satisfying this condition are \emph{monotone}: $V(x(t))$
is non-increasing.
\end{definition}

In the language of \cref{sec:water}, the water level $w(t)$ is a
Lyapunov function (\cref{def:lyapunov}) for the viability inclusion.
The monotonicity condition $D^+ V(x)(v) + W(x,v) \leq 0$ says: along
any viable trajectory, the water level cannot increase faster than the
system's extraction cost $W$.  When $V(x(t)) \to 0$, the Lyapunov
condition fails, the tangential condition is violated, and the system
exits $K$---this is \cref{thm:dumu} (Du Mu's theorem) restated in DI
language.

\begin{remark}[Temporal linearity in the DI]\label{rem:di-temporal}
The differential inclusion $x'(t) \in F(x(t))$ operates in real
time: $t$ is the wall-clock parameter, and the inclusion must be
satisfied at every instant.  The feedback map $C(x(t))$ must be
evaluated within one time step $\Delta t$ (\cref{rem:temporal}).
The tangential condition $F(x) \cap T_K(x) \neq \varnothing$
is a \emph{pointwise} requirement: it must hold at every $x \in K$,
which means at every instant.  There is no lookahead, no global
optimisation over future trajectories---only the local tangent
condition, checked in real time.  This is why Aubin calls the
viable system's policy ``opportunism'': the system selects a
feasible velocity from $F(x) \cap T_K(x)$ at each instant, without
planning.  The tangential condition admits regulation maps with
memory; we restrict to memoryless feedback (the king reacts to
current state only) as a modelling choice that matches the historical
evidence.  This restriction is sufficient for the results that
follow; it is not forced by the differential inclusion itself.
\end{remark}

\section{Viability geometry}\label{sec:viab-geom}

The differential inclusion (\cref{sec:di}) provides the dynamics.
We now give the viability kernel $K$ a Riemannian metric, making the
survival problem geometric.  The key observation is that the Lyapunov
function $V$ (\cref{def:lyapunov}) is not merely a scalar indicator of
system health---it is a \emph{conformal factor} that defines the
intrinsic geometry of the viable region.

\begin{definition}[Viability metric]\label{def:viab-metric}
Let $g_S$ denote the ambient metric on the state space $S$ and
$V: K \to \R_{\geq 0}$ the Lyapunov function (\cref{def:lyapunov})
with $V > 0$ on $K^\circ$ and $V = 0$ on $\partial K$.
The \emph{viability metric} on $K^\circ$ is the conformal deformation
\[
  g_V \;:=\; \frac{1}{V(x)^2}\, g_S.
\]
The Riemannian manifold $(K^\circ, g_V)$ is called the
\emph{viability manifold}.
\end{definition}

The conformal factor $V^{-2}$ inflates distances near the boundary
($V \to 0$) and compresses distances in the interior ($V$ large).
A trajectory approaching the boundary must cover infinite
$g_V$-distance in finite ambient time---the boundary is ``at
infinity'' in the viability metric.

\begin{proposition}[Completeness]\label{prop:viab-complete}
Suppose $V(x) \leq C \cdot d_K(x)$ for some $C > 0$
and all $x$ near $\partial K$, where $d_K(x)$ is the ambient
distance to $\partial K$.  Then $(K^\circ, g_V)$ is a complete
Riemannian manifold.
\end{proposition}

\begin{proof}
Let $\gamma: [0, T) \to K^\circ$ be a curve approaching $\partial K$
as $t \to T$.  The $g_V$-length is
\[
  L_{g_V}(\gamma)
  \;=\;
  \int_0^T \frac{\|\gamma'(t)\|_{g_S}}{V(\gamma(t))}\, dt
  \;\geq\;
  \frac{1}{C}
  \int_0^T \frac{\|\gamma'(t)\|_{g_S}}{d_K(\gamma(t))}\, dt.
\]
Since $\|\gamma'(t)\|_{g_S} \geq |d_K(\gamma(t))'|$ by the
triangle inequality, the right-hand side is bounded below by
$(1/C)\int_0^T |d_K'|/d_K\,dt$.  By substitution $u = d_K$,
this becomes $(1/C)\int du/u$, which diverges logarithmically
as $d_K \to 0$.
Hence $\partial K$ is at infinite $g_V$-distance: the Hopf--Rinow
theorem gives completeness.
\end{proof}

\begin{proposition}[Negative curvature]\label{prop:neg-curvature}
Let $\dim S = 2$ and $g_S$ be flat.  If $V$ is superharmonic
($\Delta V \leq 0$, the Lyapunov condition), the Gaussian curvature
of $(K^\circ, g_V)$ satisfies
\[
  \kappa_V
  \;=\;
  V \,\Delta V \;-\; |\nabla V|^2
  \;\leq\;
  -\,|\nabla V|^2
  \;<\; 0
\]
wherever $\nabla V \neq 0$.  In dimension $n \geq 3$, the Ricci
curvature of $g_V = V^{-2} g_S$ satisfies
$\mathrm{Ric}_{g_V} \leq -(n-1)\,|\nabla \log V|^2\, g_V$
under the same superharmonicity condition.
\end{proposition}

\begin{proof}
For a conformal change $\tilde{g} = e^{2\varphi}\, g$ with
$\varphi = -\log V$, the Gaussian curvature in dimension~$2$
transforms as
$\tilde\kappa = e^{-2\varphi}(\kappa_S - \Delta\varphi)$,
where $\kappa_S$ is the ambient curvature.
On a flat background ($\kappa_S = 0$):
\[
  \kappa_V
  \;=\;
  V^2\!\left(-\,\Delta(-\log V)\right)
  \;=\;
  V^2\!\left(\frac{\Delta V}{V} - \frac{|\nabla V|^2}{V^2}\right)
  \;=\;
  V\,\Delta V \;-\; |\nabla V|^2.
\]
Since $\Delta V \leq 0$ (superharmonic) and $|\nabla V|^2 > 0$,
we have $\kappa_V < 0$.  The higher-dimensional statement follows
from the conformal Ricci formula
$\mathrm{Ric}_{\tilde g} = \mathrm{Ric}_g - (n-2)\,
\nabla^2\varphi - [\Delta\varphi + (n-2)\,|\nabla\varphi|^2]\,g$.
\end{proof}

\begin{definition}[Cheeger constant of the viability manifold]
\label{def:cheeger-viab}
The \emph{Cheeger constant} of $(K^\circ, g_V)$ is
\[
  h(K)
  \;:=\;
  \inf_{S}
  \frac{|\partial S|_{g_V}}
  {\min\!\bigl(\mathrm{vol}_{g_V}(A),\,
  \mathrm{vol}_{g_V}(B)\bigr)},
\]
where the infimum is over hypersurfaces $\partial S$ that divide
$K^\circ$ into two open subsets $A$ and $B$, and $|\partial S|_{g_V}$
denotes the $(n-1)$-dimensional volume of $\partial S$ in the
viability metric.
\end{definition}

\begin{theorem}[Cheeger inequality]\label{thm:cheeger-viab}
Let $\lambda_1(K)$ denote the first nonzero eigenvalue of the
Laplace--Beltrami operator on $(K^\circ, g_V)$.  Then
\[
  \lambda_1(K) \;\geq\; \frac{h(K)^2}{4}.
\]
\end{theorem}

\begin{proof}
This is the Riemannian Cheeger inequality~\cite{cheeger}.
The discrete version on the execution graph appears in
\cref{thm:cheeger}.
\end{proof}

\begin{remark}[Poincar\'e half-plane]\label{rem:poincare}
When $K = \{x \in \R^2 : x_2 > 0\}$ (the upper half-plane) and
$V(x) = x_2$ (height above the boundary), the viability metric
$g_V = x_2^{-2}(dx_1^2 + dx_2^2)$ is the Poincar\'e half-plane
model of hyperbolic geometry.  The constant curvature is
$\kappa_V = -1$.  The viability kernel of a dynasty on a flat
state space with water level $V = $ distance to collapse is, in its
intrinsic geometry, \emph{the hyperbolic plane}.
\end{remark}

\begin{remark}[Unique cheapest viable path]\label{rem:cartan-hadamard}
\Cref{prop:neg-curvature} gives $\kappa_V \leq 0$ everywhere.
If $K$ is convex (or more generally, if $K^\circ$ is simply
connected), then by the Cartan--Hadamard theorem $(K^\circ, g_V)$
is a Hadamard manifold: the exponential map
$\exp_x: T_x K^\circ \to K^\circ$ is a diffeomorphism.
In particular, between any two interior states there exists a
\emph{unique geodesic}---a unique cheapest viable path.
There is no ambiguity in the optimal route; the geometry forces it.
Convexity of $K$ is natural: the viability kernel is defined as
the set of states from which a viable path exists
(\cref{ax:viability}), and viability kernels of upper semicontinuous
differential inclusions are closed under convex combinations when
$F$ has convex values (\cref{thm:viability-di}).
\end{remark}

\begin{remark}[Why viability maintenance is hard]
\label{rem:viab-divergence}
Negative curvature means nearby geodesics diverge exponentially:
two trajectories that start $\epsilon$-close separate as
$\sim \epsilon\, e^{\sqrt{|\kappa_V|}\, t}$.  A small perturbation
in the king's initial state produces exponentially different
outcomes.  This is the geometric content of sensitive dependence:
the viability manifold is hyperbolic, so maintaining viability
requires continuous correction at every instant
(\cref{rem:di-temporal}).  The harder the Lyapunov function
decreases ($|\nabla V|$ large), the more negative the curvature,
and the faster trajectories diverge.  Near the boundary
($V \to 0$), the curvature diverges: the last moments before
collapse are the most chaotic.
\end{remark}

\begin{remark}[The water is the metric]\label{rem:water-metric}
The Lyapunov function $V$ (\cref{def:lyapunov})---the water level
of \cref{sec:water}---plays three roles simultaneously:
\begin{enumerate}[label=(\roman*)]
  \item \emph{Scalar}: $V(x)$ measures distance from collapse.
  \item \emph{Conformal factor}: $g_V = V^{-2} g_S$ defines the
  intrinsic geometry of the viability kernel.
  \item \emph{Curvature source}: $\Delta V \leq 0$ forces
  $\kappa_V < 0$, making the geometry hyperbolic.
\end{enumerate}
Du Mu's theorem (\cref{thm:dumu})---$V \to 0$ implies system
death---is a \emph{completeness theorem}: a trajectory reaching
$V = 0$ would traverse infinite $g_V$-distance in finite time,
violating \cref{prop:viab-complete}.  The system must exit $K$
before $V$ reaches zero.  Du Mu is Hopf--Rinow.
\end{remark}

\section{The knife}\label{sec:knife}

\begin{definition}[Knife]\label{def:knife}
A resource $r$ is a \emph{knife} if it satisfies two conditions:
\begin{enumerate}[label=(\arabic*)]
  \item \textbf{Autonomous actuation.} The resource can operate
  independently of the king. Formally, there exists an action
  $a \in \Ur$ such that the execution function $f_r$ satisfies
  \[
    f_r(s, a) \notin K
    \quad\text{and}\quad
    a \text{ does not require the king's authorization.}
  \]
  \item \textbf{Observability.} The king's detection function $\Obs$
  can observe $r$ and its execution capability:
  $r \in \mathrm{Im}(\Obs)$.
\end{enumerate}
\end{definition}

The classification is exhaustive:
\begin{itemize}
  \item Condition~(1) not satisfied: \textbf{not a knife}. (Zhang
  Liang's strategic counsel---a pure function that cannot execute
  itself.)
  \item Condition~(2) not satisfied: \textbf{hidden knife}. (More
  dangerous, but outside the king's strategy space. Unobservable
  $=$ indefensible $=$ system noise.)
  \item Both satisfied: \textbf{knife}.
\end{itemize}

\begin{remark}[DI restatement of the knife]\label{rem:knife-di}
In the language of \cref{sec:di}, a resource $r$ is a knife if and
only if its velocity set $F_r(x)$ can generate directions outside
the contingent cone:
\[
  r \text{ is a knife}
  \quad\iff\quad
  \exists\, x \in K \;\;\text{such that}\;\;
  F_r(x) \not\subset T_K(x).
\]
That is, $r$ can push the system's state toward the boundary of $K$
along directions that are \emph{not tangent} to the viability
kernel.  The king's feedback map $C(x)$ (\cref{def:feedback}) can
compensate only if the king's velocity set $F_{\mathrm{king}}(x)$
contains a counteracting direction in $T_K(x)$.  When $r$ executes
independently---bypassing $C(x)$---no compensation is possible,
and the tangential condition (\cref{thm:viability-di}) is violated.
\end{remark}

\begin{remark}[Geometric restatement of the knife]
\label{rem:knife-geom}
In the viability manifold $(K^\circ, g_V)$ (\cref{def:viab-metric}),
the knife has a curvature interpretation.  A knife
(\cref{def:knife}) must satisfy both autonomous actuation
\emph{and} observability; the geometric content lies in
condition~(1).  When a resource $r$ actuates autonomously, its
velocity field $f_r$ has a component along $-\nabla V$ (pointing
toward the boundary).  The magnitude
$|\langle f_r, -\nabla V \rangle|$ contributes directly to the
curvature (\cref{prop:neg-curvature}): the knife increases
$|\nabla V|$, making $\kappa_V$ more negative, and the
viable region more hyperbolic.  More knives $\Rightarrow$ more
negative curvature $\Rightarrow$ faster divergence of nearby
trajectories $\Rightarrow$ harder viability maintenance.
The Cheeger constant $h(K)$ (\cref{def:cheeger-viab}) measures
the worst-case cut: the knife is the hypersurface that minimises the
isoperimetric ratio of the viability manifold.
(Condition~(2)---observability---determines whether the king
\emph{knows} where the cut is, not whether it exists.)
\end{remark}

\begin{remark}[Intent is irrelevant]\label{rem:intent}
The criterion tests \emph{capability}, not \emph{intention}. The king
detects whether you \emph{can} act, not whether you \emph{want to}.
Loyalty does not enter the criterion.
\end{remark}

\begin{remark}[Logical necessity]\label{rem:necessity}
These two conditions are not chosen by the modeler. They are the unique
logical consequence of the viability axiom $+$ unconstrained power $+$
multi-agent environment.
\end{remark}

\section{Phase transition}\label{sec:phase}

The knife is a \emph{phase function}, not an intrinsic property.

\begin{proposition}[Phase-dependent labelling]\label{prop:phase}
The same resource $r$ receives different labels under different system
phases $\varphi$:
\[
  \mathrm{Label}(r, \varphi) =
  \begin{cases}
    \textbf{tool} & \text{if } \varphi = \text{wartime (king needs }
    r\text{'s actuation),} \\
    \textbf{knife} & \text{if } \varphi = \text{peacetime (king no
    longer needs } r\text{, but } r \text{ persists).}
  \end{cases}
\]
The phase transition does not change the physical properties of $r$.
It changes the king's objective function $J(s, \varphi)$.
\end{proposition}

\begin{proof}
In wartime, the king's objective $J_{\mathrm{war}}$ includes terms
where $r$'s actuation has positive utility. In peacetime,
$J_{\mathrm{peace}}$ optimises for long-term survival
($\exists\;\text{path to } \infty$), and the same actuation becomes a
boundary threat on $\Viab(K)$. The resource $r$ is unchanged;
the labelling function $\mathrm{Label}(r, \varphi)$ is what shifts.
\end{proof}

\section{The cut vertex principle}\label{sec:cutvertex}

\begin{definition}[Cut vertex]\label{def:cutvertex}
In the execution graph $G = (V, E)$ of the system, a vertex
$v \in V$ is a \emph{cut vertex} if $G \setminus \{v\}$ is
disconnected. An agent who is a cut vertex controls all execution
chains: removing them disconnects the system.
\end{definition}

\begin{theorem}[Cut vertex $\neq$ maximum actuator]\label{thm:cutvertex}
The optimal survival strategy for the king is to be a cut vertex, not
the maximum actuator. That is, the king maximises viability by
ensuring all execution chains pass through him, rather than by
maximizing his own actuation.
\end{theorem}

\begin{proof}
A maximum actuator $v^*$ with $\|U_{v^*}\| = \max_i \|U_i\|$
suffers from three structural defects:
\begin{enumerate}[label=(\roman*)]
  \item \emph{Non-scalability}: a single actuator cannot cover the
  full state space simultaneously.
  \item \emph{Single point of failure}: $\Viab(K)$ depends entirely
  on $v^*$'s performance; one failure collapses the system.
  \item \emph{Self-referential paradox}: if the king \emph{is} the
  knife (the strongest autonomous actuator), he cannot perform
  viability maintenance on himself.
\end{enumerate}
A cut vertex $v_c$ with $\|U_{v_c}\| \approx 0$ but routing
authority over all chains avoids all three: the system is scalable
(add more actuators), fault-tolerant (one actuator's failure does
not disconnect the graph), and the king is structurally distinct
from the knives he must manage.
\end{proof}

\begin{example}[Liu Bang vs.\ Xiang Yu]\label{ex:liubang}
Xiang Yu was the strongest actuator in the late Qin system
(\emph{Shiji}: ``he could lift a bronze tripod''). His strategy:
$U = \Umax$ through personal combat. Liu Bang had near-zero
personal actuation but made himself the cut vertex of the execution
graph: Han Xin's armies needed Liu Bang's legitimacy, Xiao He's
administration needed his authorization, Zhang Liang's counsel
needed him to listen.

After the phase transition (founding of the Han dynasty), Liu Bang
executed precise viability maintenance: killed Han Xin (knife),
imprisoned then released Xiao He (blunted half-knife), left Zhang
Liang alone (not a knife). Xiang Yu, the maximum actuator, died at
Gaixia---a single actuator cannot cover the full state space.
\end{example}

\section{Case analysis: the three fates}\label{sec:cases}

The framework's discriminating power is tested against three figures from
the Han founding (c.~202~BCE), all subordinates of the same king (Liu
Bang), operating in the same post-unification phase:

\begin{center}
\begin{tabular}{@{}lcccc@{}}
\toprule
\textbf{Agent} & $\Ur$ & $\mathrm{Im}(\Obs)$ &
\textbf{Classification} & \textbf{Fate} \\
\midrule
Han Xin & $\neq \varnothing$ (military) & Yes & Knife &
Path~(b): eliminated \\
Xiao He & $\neq \varnothing$ (admin) & Yes & Half-knife &
Path~(a): self-blunted \\
Zhang Liang & $= \varnothing$ (counsel) & Yes & Not a knife &
Survived \\
\bottomrule
\end{tabular}
\end{center}

All three are visible ($r \in \mathrm{Im}(\Obs)$). The discriminant is
condition~(1): can the resource actuate independently?

\begin{example}[Han Xin: pure knife]\label{ex:hanxin}
Han Xin commanded armies that obeyed \emph{him}, not Liu Bang. His
execution chain was closed: he could mobilise, march, and fight without
the king's authorization. Both conditions of \cref{def:knife} satisfied.
After the phase transition, the knife criterion triggered and Liu Bang
eliminated him. Han Xin's quoted proverb (``when the hare dies, the dog
is cooked'') correctly identified path~(b) but failed to act on it---he
understood the classification but not that the only exit was path~(a).
\end{example}

\begin{example}[Xiao He: self-blunting]\label{ex:xiaohe}
Xiao He administered the capital and controlled grain supply---autonomous
actuation at the logistical level. The king observed this
($r \in \mathrm{Im}(\Obs)$), making Xiao He a knife by
\cref{def:knife}. Xiao He's response: deliberate self-corruption
(accepting bribes conspicuously). This performed two operations
simultaneously:
\begin{enumerate}[label=(\roman*)]
  \item \emph{Signal reduction}: visible moral degradation signals
  $\|U_r\| \to 0$ (an official this corrupt cannot coordinate a revolt).
  \item \emph{Mean-field alignment}: pull $\|U_r\|$ toward $\bar{U}$,
  falling below the detection threshold (\cref{thm:meanfield}).
\end{enumerate}
This is path~(a) executed through reputation rather than resignation.
\end{example}

\begin{example}[Zhang Liang: structural safety]\label{ex:zhangliang}
Zhang Liang was a strategist. Strategy is a pure function: it
\emph{advises} action but cannot \emph{execute} it. Zhang Liang's
counsel required Liu Bang's decision, Liu Bang's generals, and Liu
Bang's administration to produce any effect. Every execution chain
passed through the king (\cref{cor:breakpoint}). Result:
$\Ur = \varnothing$, condition~(1) fails, not a knife.
Zhang Liang retired and survived.
\end{example}

\chapter{Main Results}\label{sec:results}

\section{The binary lifecycle}\label{sec:lifecycle}

\begin{lemma}[Forcing]\label{lem:forcing}
Let $r$ be a sword that persists: both conditions of
\cref{def:sword} are satisfied for all $t \geq t_0$.  Then the
tangential condition $F(x) \cap T_K(x) \neq \varnothing$
(\cref{thm:viability-di}) requires the king to allocate
$u^*(t) \in C(x(t))$ to compensate $r$ at every instant.  The
cumulative cost $\int_{t_0}^{T} \|u^*\|\,dt$ is unbounded as
$T \to \infty$.
\end{lemma}

\begin{proof}
The sword $r$ acts independently (condition~1 of
\cref{def:sword}), so $F_r(x)$ contributes an uncontrolled
velocity component.  The king's feedback map $C(x)$
(\cref{def:feedback}) must contain a compensating control at
every instant (tangential condition).  Since $r$'s actuation is
autonomous, the king cannot predict $r$'s choices within $\Ur$;
compensation is reactive, not preventive.  The cost per unit
time is bounded below by the minimum energy needed to counteract
$F_r$, which is positive: condition~(1) ensures
$F_r(x) \not\subset T_K(x)$ for some $x$ (\cref{rem:sword-di}).
Over unbounded time, cost diverges.  The viability axiom
(\cref{ax:viability}) requires finite-cost maintenance;
therefore the sword must be resolved in finite time.
\end{proof}

\begin{theorem}[Binary fate]\label{thm:lifecycle}
Every sword has exactly two possible outcomes after phase transition:
\begin{enumerate}[label=(\alph*)]
  \item \textbf{Relinquish}: the holder voluntarily sets
  $\Ur \to \varnothing$.
  \item \textbf{Elimination}: the king forces removal via
  $u^* \in \Umax$.
\end{enumerate}
There is no path~(c).
\end{theorem}

\begin{proof}
The sword criterion is $\Ur \neq \varnothing \;\wedge\;
r \in \mathrm{Im}(\Obs)$ (\cref{rem:sword-di} for the DI
restatement).  As long as both conditions hold, the sword
persists and \cref{lem:forcing} applies: the cost of indefinite
coexistence is unbounded.  The viability axiom requires
finite-cost maintenance, so the sword must be resolved.

Resolution requires at least one condition to fail.
Setting $\Ur \to \varnothing$ is path~(a).  If the holder does
not relinquish, condition~(1) persists; regression to a pre-sword
(suppressing condition~(2) while retaining condition~(1)) is not
a resolution but a deferral, since the causal envelope persists
and any expansion of $\Obs$ re-detects the resource
(\cref{rem:lifecycle-scope}).  By \cref{lem:forcing}, the cost
of indefinite coexistence is unbounded; the king is therefore
forced to eliminate via $u^* \in \Umax$, which is path~(b).
The classification is exhaustive.
\end{proof}

\begin{remark}[Scope of the binary lifecycle]\label{rem:lifecycle-scope}
\Cref{thm:lifecycle} applies to \emph{swords}: resources satisfying
both conditions of \cref{def:sword} simultaneously.  A resource that
satisfies condition~(1) but not condition~(2)---a \emph{pre-sword}
(\cref{def:presword})---is outside the theorem's scope.
Xiao He's self-blunting strategy (\cref{ex:xiaohe}) is a sword
$\to$ pre-sword regression: he exits the sword classification by
suppressing condition~(2), not by relinquishing condition~(1).
This is not path~(c); it is a transition to a different lifecycle
state (\cref{rem:sword-lifecycle}).  The binary partition holds for
every resource that \emph{remains} a sword: as long as both
conditions are satisfied, exactly one of (a)~or~(b) must obtain.

The pre-sword state is unstable because the causal envelope
$\mathrm{Reach}(x, \Ur)$ (\cref{eq:causal-envelope}) persists: the
agent retains capability, and any expansion of $\Obs$
re-detects the resource, returning it to the sword row and
re-activating the forcing cost of \cref{lem:forcing}.  Thus
the pre-sword defers rather than resolves the binary fate.
\end{remark}

\begin{remark}[Han Xin's error]\label{rem:hanxin}
The proverb ``when the cunning hare is killed, the hunting dog is
cooked'' (\emph{Shiji}, Huaiyin Hou) conflates three structurally
distinct resources: the \emph{bow} ($\Ur = \varnothing$, tool,
``stored'' not destroyed), the \emph{dog} ($\Ur \neq \varnothing$,
actuator, ``cooked''), and the \emph{advisor} (pure function, no
actuation---Zhang Liang survived). Han Xin quoted the answer but did
not parse its fine structure.
\end{remark}

\section{The fixed-point impossibility}\label{sec:fixedpoint}

\begin{theorem}[No path (c)]\label{thm:fixedpoint}
There is no strategy that ``proves your sword is not a sword'' while
retaining the sword. Formally, the map
$T: \Ur \mapsto \varnothing$ conditional on $\Ur \neq \varnothing$
has no fixed point other than $\Ur = \varnothing$.
\end{theorem}

\begin{proof}
Case~1: $\Ur = \varnothing$. Then $r$ is not a sword, and no proof
is needed. $\Ur = \varnothing$ is self-certifying.
Case~2: $\Ur \neq \varnothing$. Then no speech act can set
$\Ur \to \varnothing$---the criterion tests physical capability,
not narrative. The only way to satisfy $T(U_r) = \varnothing$ is to
physically relinquish $\Ur$, which is path~(a).

Moreover, the act of proving is itself a signal: ``I need to prove
my sword is not a sword'' implies suspicion, i.e., $r$ is already
in $\mathrm{Im}(\Obs)$. The proof attempt reinforces condition~(2).
\end{proof}

\section{The unconstrained power paradox}\label{sec:paradox}

\begin{theorem}[Perpetual elimination]\label{thm:paradox}
$U = \Umax$ implies the king must preemptively eliminate all
observable autonomous actuators:
\[
  U = \Umax \implies
  \text{the king must preempt all } r \text{ with }
  \Ur \neq \varnothing \;\wedge\; r \in \mathrm{Im}(\Obs).
\]
The more unconstrained the king, the more swords he must cut.
\end{theorem}

\begin{proof}
$\Umax$ means the king tolerates \emph{no} autonomous actuation:
every such actuator is a boundary threat on $\Viab(K)$. A
constrained system (constitutional regime) institutionalises sword
dynamics by installing breakpoints. An unconstrained system must
handle every sword individually. The paradox: unconstrained power is
not freedom---it is a perpetual elimination machine.
\end{proof}

\begin{proposition}[Imperfect observability accelerates the paradox]
\label{prop:imperfect}
If the detection function $\Obs$ is imperfect, the paradox
\emph{intensifies}, not weakens.
\end{proposition}

\begin{proof}
Three steps:
\begin{enumerate}[label=(\roman*)]
  \item The king knows $\Obs$ is imperfect. Hidden swords
  (\cref{def:sword}, condition~(2) unsatisfied) are more dangerous
  than visible ones. The king has motive, capability, and survival
  obligation to expand $\Obs$.
  \item Expanding $\Obs$ does not ``discover existing swords''---it
  \emph{creates new ones} definitionally. A hidden actuator
  satisfying condition~(1) but not~(2) enters $\mathrm{Im}(\Obs)$
  upon expansion $\to$ both conditions now satisfied $\to$ it
  \emph{becomes} a sword. The sword exists in the intersection
  $\Ur \neq \varnothing \;\wedge\; r \in \mathrm{Im}(\Obs)$;
  expanding $\Obs$ expands this intersection.
  \item Positive feedback:
  $U \to \Umax \implies \Obs \to \Obs_{\max}$.
  The expansion of $\Obs$ is the \emph{adjoint process} of the
  expansion of $U$. The elimination machine has two engines: the
  cutting arm ($U$) and the detecting eye ($\Obs$). They co-drive.
\end{enumerate}
Historical instances: Qin's mutual surveillance law
(\emph{lianzuo}), Han's gold-purity test (\emph{zhuo\-jin
duo\-jue}), Ming's three-layer nested monitoring (Jinyiwei
$\to$ Dongchang $\to$ Xichang---each layer itself becomes a new
sword).
\end{proof}

\begin{proposition}[$\Umax$ as attractor]\label{prop:attractor}
$\Umax$ is an attractor, not a state. No historical king achieves
literal $\Umax$, but the system dynamics point toward it:
\[
  \frac{d}{dt}\|U(t) - \Umax\| \leq 0
  \implies
  \frac{d}{dt}\bigl(\text{detected swords}\bigr) \geq 0.
\]
The paradox describes the trajectory, not the endpoint.
\end{proposition}

\section{The breakpoint criterion}\label{sec:breakpoint}

\begin{corollary}[Breakpoint strategy]\label{cor:breakpoint}
A resource $r$ is not a sword if and only if its execution chain
contains at least one node controlled by the king (a
\emph{breakpoint}):
\[
  r \text{ is not a sword}
  \iff
  \exists\; v \in \text{execution chain of } r
  \;\text{s.t.}\; v \text{ is controlled by the king.}
\]
\end{corollary}

\begin{proof}
If a breakpoint exists, $r$ cannot execute independently
(condition~(1) fails), so $r$ is not a sword. If no breakpoint
exists, the execution chain is closed and $r$ can actuate
autonomously, satisfying condition~(1). Combined with
observability, this makes $r$ a sword.
\end{proof}

\begin{remark}[Modern translation]\label{rem:modern}
Zhang Liang's strategy: ``ensure your capability always requires a
component you do not control.'' Liu Bang's strategy: ``become the
mandatory node in every execution chain.''
\end{remark}

\chapter{The Knife as Mean Field}\label{sec:meanfield}

The preceding sections defined the knife as a two-condition criterion
applied to individual resources. We now argue that the knife is
fundamentally a \emph{mean-field} phenomenon.

\section{The mean actuation field}

Consider $n$ agents with autonomous actuation levels
$\|U_1\|, \ldots, \|U_n\|$. Define the \emph{mean actuation field}:
\[
  \bar{U} = \frac{1}{n} \sum_{i=1}^{n} \|U_i\|.
\]

The king's detection function $\Obs$ has finite bandwidth: it cannot
monitor all agents with equal precision. In practice, $\Obs$ triggers
on agents whose actuation \emph{deviates significantly from the mean}:
\[
  r \in \mathrm{Im}(\Obs)
  \iff
  \|U_r\| - \bar{U} > \tau(\Obs),
\]
where $\tau(\Obs)$ is the detection threshold determined by the king's
observational capacity.

\begin{theorem}[The knife is the mean]\label{thm:meanfield}
The knife threshold is determined by the system's mean actuation field.
A resource $r$ is a knife if and only if:
\begin{enumerate}[label=(\roman*)]
  \item $\Ur \neq \varnothing$ (autonomous actuation exists), and
  \item $\|U_r\|$ exceeds the mean field by more than the detection
  threshold: $\|U_r\| > \bar{U} + \tau(\Obs)$.
\end{enumerate}
Consequently, the phase transition (\cref{prop:phase}) is a shift in
$\bar{U}$, not a change in any individual $\Ur$.
\end{theorem}

\begin{proof}
In wartime, many agents have high actuation (soldiers, generals,
administrators). The mean $\bar{U}$ is high, so the threshold
$\bar{U} + \tau(\Obs)$ is high: few agents exceed it. Most actuation
is \emph{expected} and therefore not flagged.

At the phase transition (end of war), most agents' actuation drops to
near zero (soldiers demobilize, wartime powers expire). The mean
$\bar{U}$ drops sharply. But agents who \emph{retain} wartime-level
actuation now exceed the new, lower threshold. The same $\|U_r\|$
that was below the wartime mean is now above the peacetime mean.

The knife is not created by the agent---it is created by the shift in
the mean. The agent's actuation is unchanged; the system's reference
frame has moved.
\end{proof}

\begin{remark}[Connection to statistical mechanics]\label{rem:statmech}
This is precisely the mechanism of a phase transition in statistical
mechanics: the order parameter (mean actuation) shifts, and
configurations that were typical in one phase become atypical---and
therefore detectable---in the other. The viability axiom plays the role
of the free energy: the system minimizes threats to $\Viab(K)$, just
as a thermodynamic system minimizes free energy.
\end{remark}

\section{Implications}

The mean-field interpretation resolves several puzzles:

\begin{enumerate}
  \item \textbf{Why identical resources have different fates.} Two
  generals with identical $\Ur$ can have opposite outcomes if one
  operates in a high-$\bar{U}$ environment (wartime coalition) and
  the other in a low-$\bar{U}$ environment (consolidated empire).
  The knife is relative to the mean.

  \item \textbf{Why the paradox is a feedback loop.} As the king
  eliminates knives, $\bar{U}$ drops, lowering the threshold. Agents
  who were below the old threshold now exceed the new one $\to$ new
  knives $\to$ more elimination $\to$ lower $\bar{U}$ $\to$ \ldots
  This is the positive feedback of \cref{thm:paradox}, now given a
  statistical mechanism.

  \item \textbf{Why self-blunting works.} Xiao He's strategy
  (self-corruption to signal low $\|U_r\|$) works precisely because
  the detection function triggers on \emph{deviation from the mean}.
  By visibly degrading his own actuation, Xiao He pulled $\|U_r\|$
  toward $\bar{U}$, falling below the detection threshold.

  \item \textbf{Why breakpoints prevent knives.} A breakpoint in the
  execution chain reduces $\|U_r\|$ (effective autonomous actuation)
  to below $\bar{U} + \tau(\Obs)$, since the king controls part of
  the chain. The resource remains capable but not \emph{independently}
  capable---it does not deviate from the mean.
\end{enumerate}

\chapter{Applications}\label{sec:applications}

\section{The Qin operating system}\label{sec:qin}

The Qin state (356--207~BCE) provides the first complete engineering
implementation of the framework. Shang Yang's reforms map directly to
the formal vocabulary:

\begin{center}
\begin{tabular}{@{}lp{5cm}p{5cm}@{}}
\toprule
\textbf{Policy} & \textbf{Framework equivalent} & \textbf{Effect} \\
\midrule
Abolish well-field system & Remove aristocratic $\Ur$ & Nobles
$\to$ commoners \\
Military merit ranks & $\Ur$ controlled by state (revocable) &
Actuation is rented, not owned \\
Mutual surveillance (\emph{lianzuo}) & Maximise $\Obs$ &
Neighbors $=$ distributed sensor network \\
Standardise weights \& measures & Increase $\Obs$ precision &
Higher observational resolution \\
Commandery-county system & All chains through capital &
King $=$ cut vertex \\
\bottomrule
\end{tabular}
\end{center}

Qin's unification of the six states was a graph-theoretic outcome:
Qin's star graph (center $=$ Xianyang, $O(1)$ dispatch) vs.\ the six
states' mesh graphs (multiple aristocratic centers, $O(n^2)$
coordination cost).

\begin{remark}[Temporal linearity]\label{rem:temporal}
The mesh-to-star compression is not an efficiency optimisation.  It is
a survival requirement forced by temporal linearity.

Time is real and linear: no pause, no speedup, no frame drops
(\cref{rem:deployment,rem:di-temporal}).  Every system that maintains
viability must complete a full sense--compute--act cycle within one
time step $\Delta t$:
\[
  \underbrace{|\phi| > 0}_{\text{do not fall}}
  \quad\wedge\quad
  \underbrace{t_{\pi} \leq \Delta t}_{\text{do not lag}}.
\]
The first condition is physical (the viability axiom).  The second is
computational: the response function must fit inside $\Delta t$.
Both must hold simultaneously.  Violating either kills the system
by different mechanisms: the first is a fall, the second is a lag,
and the outcome is the same.

This binds computational complexity to survival.  A mesh topology
dispatches in $O(n^2)$; a star topology in $O(1)$.  As the system
grows (more territory, more agents), $\Delta t$ does not grow with
it---time does not slow down for larger empires.  At critical $n$,
mesh dispatch exceeds $\Delta t$ and the system can no longer respond
in time.  商鞅's reform---mesh to star---is latency engineering:
compressing $t_\pi$ to fit within a fixed $\Delta t$ at any scale.

The isomorphism between robotic deployment and historical deployment
is exact:
\begin{center}
\renewcommand{\arraystretch}{1.25}
\begin{tabular}{@{}ll@{}}
\toprule
\textbf{Deployment (robotics)} & \textbf{Deployment (history)} \\
\midrule
policy $\pi$ (neural network) & emperor (cut vertex) \\
sensor reading $o_t$ & intelligence reports \\
torque command $\tau_t$ & edicts, military orders \\
$\Delta t = 20$\,ms & dynastic response window \\
ReLU (piecewise linear, $O(n)$) & star graph ($O(1)$ dispatch) \\
$|\phi| = 0$ (robot falls) & dynasty collapses \\
$t_\pi > \Delta t$ (compute lags) & cut vertex non-functional \\
\bottomrule
\end{tabular}
\end{center}

Every structural choice in the framework---ReLU as activation
function, star as topology, binary threshold as detection
criterion---is forced by the requirement that computation fit inside
real, linear, non-pausable time.  These are not design preferences;
they are the only structures fast enough to survive.

The two historical failure modes---the processor that runs an
irrelevant subroutine (latency explosion) and the processor that is
removed entirely (see \cref{sec:errorlog} for the full
analysis)---are the two ways the computational constraint breaks:
\begin{itemize}
\item 嘉靖: cut vertex present but $t_\pi \to \infty$.
  The emperor computes Daoist alchemy for twenty years while the
  system requires real-time response.  Every year without a routing
  decision is a dropped frame.  Gravity does not pause while the
  processor runs an irrelevant subroutine.
\item 元顺帝: cut vertex removed.  No processor.  Immediate crash
  ($|\phi| = 0$).
\end{itemize}
Temporal linearity is not a property of one remark in one chapter.
It is the medium in which the entire theory operates.
\end{remark}

Shang Yang's fate validates the framework's robustness: his reform
network itself became an autonomous actuator ($\Ur \neq \varnothing$).
The system eliminated its creator---not irony, but a robustness test.
A sword-detection system that exempts its designer is not robust
(\cref{sec:second} formalises this as second-mover viability).

\subsection{The submartingale-induced unitary group}

More precisely, Shang Yang's reform sequence is a
\emph{submartingale-induced unitary group}.

\begin{definition}[Submartingale reform]\label{def:submartingale}
A reform sequence $X_0, X_1, \ldots, X_n$ is a \emph{submartingale}
if each step strictly increases the centralisation index:
$\mathbb{E}[X_{k+1}] \geq X_k$ (deterministic analogue: the
sequence is non-decreasing in expectation under any re-ordering,
with the added constraint of irreversibility), and each step is
irreversible (reversal requires undoing all subsequent steps).
\end{definition}

Shang Yang's five reforms form such a sequence:
\[
  \underbrace{\text{abolish well-fields}}_{X_0}
  \to \underbrace{\text{merit ranks}}_{X_1}
  \to \underbrace{\text{mutual surveillance}}_{X_2}
  \to \underbrace{\text{standard measures}}_{X_3}
  \to \underbrace{\text{commandery-county}}_{X_4}.
\]
Each step burns the entropy of the old feudal order irreversibly.

Once installed, the system forms a \emph{unitary group}: a
structure-preserving transformation that maintains its invariants
(king $=$ cut vertex, all swords eliminated) at every cycle, applied
to every vector in the state space without exception.

\begin{theorem}[Shang Yang's paradox]\label{thm:shangyang}
Any agent who installs a sword-elimination system via submartingale
reform must possess $\Ur \neq \varnothing$ to execute the installation.
But the submartingale is monotone: each step lowers the detection
threshold. The installer's own actuation becomes increasingly visible
with each reform step. Upon completion, the unitary group acts on the
installer:
\[
  \underbrace{\text{installer drives submartingale}}_{\text{requires }
  \Ur \neq \varnothing}
  \;\longrightarrow\;
  \underbrace{\text{unitary group forms}}_{\text{detects all }
  \Ur \neq \varnothing}
  \;\longrightarrow\;
  \underbrace{\text{group acts on installer}}_{\text{elimination}}.
\]
The system has no creator exemption.
\end{theorem}

\begin{proof}
The installer is not in the invariant subspace of the group he
created (because $\Ur \neq \varnothing$). A unitary group does not
preserve non-invariant elements---it decomposes them. But unitary
means \emph{norm-preserving}: the installer is destroyed, but his
contributions are fully conserved. The reforms---commandery-county
system, merit ranks, mutual surveillance, standard measures---persist
uniformly across the empire. The person is eliminated; not one bit of
the contribution is lost. This is not destruction but
\emph{delocalisation}: a localised power entity is scattered across
the full state space by the unitary action. Norm conserved, structure
zeroed.
\end{proof}

\subsection{The existence proof}

Qin unified the six states, then collapsed (207~BCE, lasting only
15~years). But Qin left something more powerful than any army: an
\emph{existence proof}.

\begin{theorem}[Qin existence theorem]\label{thm:qin}
\[
  \exists\;\text{centralised state}\;\text{s.t.}\;
  \text{autonomous actuation} = 0
  \;\wedge\;
  \text{state functions.}
\]
\end{theorem}

Before Qin, no one knew a state without feudal aristocrats was
possible. Qin proved it was---and proved it functioned \emph{better}
(star-graph dispatch vs.\ mesh-graph coordination). You cannot refute
an existence proof. You can burn the paper, but the theorem persists.

After Qin fell, two men entered the ruins. Liu Bang, then a minor
official, had once seen the First Emperor's procession and sighed:
``A great man should be like this!''---\emph{I want to BE this system's
cut vertex}. Xiang Yu, upon conquering Xianyang, burned the palaces
and said: ``Who would be emperor in the dark where no one can
see?''---he wanted the \emph{trophy}, not the \emph{theorem}. One read
the existence proof. The other burned it. The one who read it founded
a dynasty that lasted four centuries.

\subsection{The wisdom of being second}\label{sec:second}

\Cref{thm:shangyang} has a direct corollary that inverts the usual
narrative of primacy.

\begin{corollary}[Second-mover viability]\label{cor:second}
Let $X_0, \ldots, X_n$ be a submartingale reform
(\cref{def:submartingale}) installed by agent~$\alpha$,
and let $\beta$ be an agent who inherits the resulting
infrastructure without participating in the installation.
Then $\beta$ has strictly higher viability than $\alpha$:
\[
  \Ur(\alpha) \neq \varnothing
  \;\;\text{(installation requires actuation)},\qquad
  \Ur(\beta) = \varnothing
  \;\;\text{(inheritance does not)}.
\]
The installed unitary group eliminates $\alpha$ and preserves $\beta$.
\end{corollary}

\begin{proof}
By \cref{thm:shangyang}, the installer $\alpha$ necessarily possesses
$\Ur \neq \varnothing$ and is detected upon completion.  The
inheritor~$\beta$ uses the infrastructure but did not create it:
$\Ur(\beta) = \varnothing$ with respect to the installation.  The
detection function tests $\Ur$, not provenance.  $\beta$ passes;
$\alpha$ does not.
\end{proof}

The existence proof (\cref{thm:qin}) has an installation cost, paid
exactly once by the first mover.  The cost is non-transferable and
non-refundable.  After installation the landscape is permanently
shifted; second movers operate in the post-shift regime without
needing to shift anything.

The first mover's contribution is \emph{entangled} with $\Ur$---they
needed $\Ur$ to install.  The second mover's usage is
\emph{factorised}---they use the infrastructure without the
installer's $\Ur$.  The second mover's path to recognition is smooth;
the first mover's is not.

\begin{center}
\renewcommand{\arraystretch}{1.25}
\begin{tabular}{@{}llp{3.5cm}p{4.5cm}@{}}
\toprule
\textbf{First} & \textbf{Second} & \textbf{Inherited} &
\textbf{Outcome} \\
\midrule
商鞅 & 秦's successors & commandery-county, merit ranks &
  商鞅 torn apart; Qin unifies \\
秦 & 汉 (刘邦) & existence proof & Qin 15\,yr; Han 400\,yr \\
项羽 & 刘邦 & Xianyang (capital) & 项羽 suicide; 刘邦 founds Han \\
王安石 & 蔡京 & fiscal infrastructure & 王安石 exiled; 蔡京 20\,yr
  chancellor \\
\bottomrule
\end{tabular}
\end{center}

范蠡 (5th century BCE) is the purest instance.  He helped 勾践 destroy
吴 but let the king remain the visible cut vertex.  Upon victory he
vanished---accumulated three fortunes and dispersed two, ensuring
$\Ur \approx \varnothing$ at every stage.  He was always second.
张良 followed the same strategy: after 刘邦's victory he retired to
``follow 赤松子,'' reducing $\Ur$ to $\varnothing$ before the
detection threshold reached him.  Both understood \cref{cor:second}
before it was stated: read the existence proof, do not write it; use
the infrastructure, do not install it.

\section{The dollar as sword precursor}\label{sec:dollar}

Money is not a sword (it cannot actuate independently). Money is a
\emph{sword precursor}: a universal voucher exchangeable for
$\Ur \neq \varnothing$ on the market. The king detects not how much
money you have, but whether your money has \emph{already converted}
into autonomous execution capability.

Fan Li (5th century BCE) understood this: he accumulated fortunes
three times and dispersed them twice, ensuring $\Ur$ never crossed
the threshold. Shen Wansan and Hu Xueyan did not---their wealth
formed closed execution loops, and they were eliminated.

\section{The Atlantic triangular trade: a sword with no cut
vertex}\label{sec:triangle}

The Atlantic Triangular Trade (16th--19th century) is a three-leg
closed execution loop: manufactured goods (Europe $\to$ West Africa),
enslaved human beings (West Africa $\to$ Americas), raw materials
(Americas $\to$ Europe). This loop satisfies \cref{def:sword}: it
actuates autonomously (multiple nations operate independent instances;
remove any one and the loop continues) and is observable (300~years of
ships, ledgers, and auction blocks).

The loop has a property stronger than condition~(1): it is
\emph{self-financing}. The raw materials extracted on leg~3 fund the
goods shipped on leg~1, which purchase the enslaved labour on leg~2,
who produce the raw materials on leg~3. A sword that funds its own
actuation cannot be starved from outside.

\paragraph{Where the isomorphism breaks.}
In \cref{sec:water}, the king--pawn relationship is sustained by water
(viability bargain: $\text{water} > 0$). In the Triangular Trade,
enslaved people had $\text{water} = 0$ from day one. Force~$F$
substituted for water---temporarily. The substitution cost: every unit
of force burns resources drawn from the extracted water on leg~3. The
loop becomes a self-financing violence machine. \Cref{sec:water}
predicts $\text{water} = 0 \implies \text{pawn} \to \text{sword}
\implies \text{collapse}$; force delays but does not prevent this.
The longer the delay, the more violent the eventual resolution.

\paragraph{Abolition as relabelling.}
Applying \cref{prop:phase}: did abolition (1807--1888) change the
\emph{physics} or only the \emph{label}? The labour force moved from
slavery to sharecropping on the same plantations, the resource flow
(raw materials $\to$ Europe) persisted, the force mechanism shifted
from slave codes to Jim Crow and convict leasing, and the laborer
still had no breakpoint. Verdict: phase transition in the sense of
\cref{sec:phase}---label changed, topology unchanged. This is
path~(c), which \cref{thm:fixedpoint} proves is not a resolution.

\begin{theorem}[Closed-loop sword]\label{thm:triangle}
Any closed execution loop that is self-financing, has no cut vertex
(mesh topology, multiple independent operators), and substitutes
force for water, has exactly two exit paths: physical dismantlement
of the loop topology [path~(a)], or system-level collapse when
force-substitution fails [path~(b)]. Relabeling [path~(c)] preserves
$\Ur \neq \varnothing$ and resolves nothing.
\end{theorem}

\paragraph{Missing cut vertex.}
The essay's framework (\cref{sec:cutvertex}) assumes a single cut
vertex. The Triangular Trade has none: its topology is a mesh, not a
star. Removing any one colonial power does not disconnect the loop.
This makes path~(a) structurally harder---there is no single point to
dismantle---but does not create a third option.

\begin{remark}[Path~(a) by mean-field shift]\label{rem:chaplin}
Chaplin's 1925 song \emph{With You, Dear, in Bombay}---composed during
\emph{The Gold Rush}, a film about extractive economics---takes the same
maritime route and replaces the extractive payload with a romantic one.
The singer sails to Bombay not to trade but to reunite. This is not
path~(c) (the route is not relabeled; the self-financing loop is broken
because love does not fund the next leg). It is path~(a) by mean-field
shift: the infrastructure persists, but the mean changes, and therefore
the sword changes.
\end{remark}

\section{The corrupted detection function}\label{sec:nazi}

\Cref{def:sword} uses a structural detection function
$\Obs_{\mathrm{structural}}$ that tests $\Ur$: \emph{what can the
resource do independently?} This function is identity-blind. Zhang
Liang is safe because $\Ur = \varnothing$, not because he is Zhang
Liang.

\begin{definition}[Detection function corruption]\label{def:corruption}
A detection function is \emph{corrupted} when $\Obs_{\mathrm{structural}}$
(tests $\Ur$) is replaced by $\Obs_{\mathrm{identity}}$ (tests group
membership):
\[
  \Obs_{\mathrm{identity}}:\;
  r \;\mapsto\;
  \begin{cases}
    \textbf{sword} & \text{if } \mathrm{agent}(r) \in \text{Group } X, \\
    \textbf{not sword} & \text{otherwise.}
  \end{cases}
\]
\end{definition}

The corruption produces a \emph{false positive catastrophe}: every
member of Group~$X$ with $\Ur = \varnothing$ is classified as a sword
and eliminated. Every agent outside Group~$X$ with $\Ur \neq \varnothing$
is a false negative. Six million false positives is the output of a
corrupted detection function running to completion.

\begin{definition}[Nazi structure]\label{def:nazi}
A system exhibits \emph{Nazi structure} if and only if: (1)~it performs
viability maintenance (eliminates perceived threats to $\Viab(K)$),
(2)~its detection criterion is identity-based, not structure-based,
and (3)~it executes path~(b) against agents classified by identity
who do not satisfy \cref{def:sword} ($\Ur = \varnothing$ for the
individual). These conditions are necessary and sufficient.
\end{definition}

\begin{corollary}[Identity-invariance]\label{cor:nazi}
The Nazi structure is identity-invariant on both sides: it depends
neither on the identity of the perpetrator nor on the identity of the
target. Replacing Group~$X$ with any other identity group preserves
all three conditions.
\end{corollary}

The operator that maps $\Obs_{\mathrm{structural}} \mapsto
\Obs_{\mathrm{identity}}$ is \emph{Hate}. In this framework, Hate is
not an emotion---it is an operator on detection functions with a
precise signature (structural $\to$ identity) and a precise output
(mass false positives). The motivation is irrelevant; the mapping
determines the output.

\section{The parasitic cut vertex}\label{sec:parasite}

\Cref{thm:cutvertex} establishes the cut vertex as the optimal
survival strategy: route all execution chains through yourself.
A \emph{parasitic cut vertex} has the same topology but opposite flow
direction: it routes access and information for extraction rather than
coordination.

\begin{definition}[Parasitic cut vertex]\label{def:parasite}
A parasitic cut vertex $v_p$ satisfies:
(1)~$G \setminus \{v_p\}$ is disconnected (cut vertex),
(2)~$v_p$ holds no formal authority ($\Ur \approx \varnothing$
officially),
(3)~$v_p$ extracts resources from both sides of the partition by
routing access across the cut.
\end{definition}

The parasitic cut vertex turns \emph{others} into pawns via the
routing function (e.g., blackmail: ``I route your secret; comply or I
re-route it publicly''). This is \cref{thm:cutvertex} inverted:
instead of building the system, the parasite extracts from it.

\begin{proposition}[Self-termination]\label{prop:parasite}
A parasitic cut vertex is self-terminating. When the secrecy that
maintains the cut ($\text{water} = \text{secrecy}$) fails, the graph
reconnects, the cut vertex property vanishes, and the former parasite
faces the combined action of all previously partitioned nodes.
\end{proposition}

Unlike the productive cut vertex (which may persist for centuries via
institutional embedding), the parasitic variant stores no structural
contribution. Upon delocalisation, norm is conserved but there is
nothing to conserve: the extraction leaves no invariant.

\section{Militarism and the net-positive ask}\label{sec:militarism}

Militarism is the structural type where the pawn ($\Ur = \varnothing$,
the military apparatus) captures the cut vertex position, producing
$\Ur \neq \varnothing$ for the military and $\Ur \to \varnothing$ for
civilian institutions. In the framework's language: the pawn becomes the king.

\begin{example}[The Kniefall: path~(a) in 30 seconds]\label{ex:kniefall}
On December~7, 1970, West German Chancellor Willy Brandt knelt before
the Warsaw Ghetto Uprising memorial. This act satisfies four conditions
that make it a pure instance of path~(a):
\emph{physical} (a bodily act, not a speech act---cannot be retracted
by reinterpretation),
\emph{unconditional} (no negotiation, no demand for reciprocity),
\emph{performed by the cut vertex} (the head of state, the node
through which all institutional chains pass), and
\emph{irretractable} (a photograph is a permanent record).
Brandt's Kniefall is an existence proof that path~(a) is available to
any system.
\end{example}

After the Kniefall, Germany installed institutional breakpoints:
Article~1 of the \emph{Grundgesetz} (``Human dignity is inviolable''),
the Federal Constitutional Court as an independent $\Obs$, EU and NATO
membership as external breakpoints. These institutionalise the phase
transition: the system moved from $U = \Umax$ (Nazi regime) through
path~(a) (Kniefall) to a constrained system with structural breakpoints.

\begin{proposition}[Net-positive theorem]\label{prop:netpositive}
Path~(a) is strictly net-positive for all parties. Comparative evidence:
Germany (path~(a), Kniefall 1970) vs.\ Japan (path~(c), relabelling
without structural change, 80~years and counting). Every measurable
outcome---diplomatic relations, regional stability, economic
integration, soft power, domestic constitutional health---favors the
path~(a) system. The cost of path~(a) is pride. The return is
everything else.
\end{proposition}

\chapter{Discussion}\label{sec:discussion}

\section{Domain of applicability}\label{sec:domain}

The framework is a compressed model that degrades under the following
conditions:

\begin{center}
\begin{tabular}{@{}lp{5.5cm}p{5cm}@{}}
\toprule
\textbf{Condition} & \textbf{Failure mode} & \textbf{Example} \\
\midrule
Diffuse sovereignty & No unique king; cut vertex undefined & Late
European feudalism, early federalism \\
External shock dominance & $\Viab(K)$ broken by external force;
internal knife dynamics secondary & Mongol invasion, colonialism \\
Rapid ideological reshaping & $K$ itself is changing faster than
actuation reshapes state & Religious revolution, ideology \\
High-latency detection & $\Obs$ too slow; phase transition completes
before detection & Large empire frontiers \\
Universal $\Ur \approx \varnothing$ & No knives; framework trivial &
Extremely atomised societies \\
\bottomrule
\end{tabular}
\end{center}

The framework applies to systems with a unique sovereign, differentiated
agent capabilities, and sufficient observability. This covers the main
interval of Chinese imperial history but not all political forms.

\section{The water dynamics}\label{sec:water}

The framework so far analyzes one layer: the king--knife interaction. A
complete viability analysis requires the \emph{viability chain}---the
three-level dependency that sustains the system:
\[
  \text{King}
  \xrightarrow{\;\text{needs}\;}
  \text{Pawn}
  \xrightarrow{\;\text{needs}\;}
  \text{Water.}
\]
The \emph{pawn} is any agent with $\Ur \neq \varnothing$ whose actuation
is currently directed by the king (soldiers, administrators, tax
collectors). \emph{Water} is the population's aggregate resource
level---the substrate from which the pawn draws manpower, revenue, and
legitimacy.

\begin{definition}[Water]\label{def:water}
Water $w(t) \in [0, W_{\max}]$ is the population's aggregate extractable
resource level at time $t$. The pawn's actuation is bounded by water:
$\|\Ur\| \leq g(w)$ for some monotone function $g$ with $g(0) = 0$.
\end{definition}

The king needs the pawn to execute viability maintenance (eliminate
knives, administer territory). The pawn needs water to function. If
$w \to 0$, the pawn's actuation capacity collapses regardless of the
king's commands.

\begin{proposition}[Binary action space at $w = 0$]\label{prop:binary}
When $w(t) \to 0$, the pawn's action space collapses to a binary:
\[
  A_{\text{pawn}} = \{\text{submit},\; \text{rebel}\}.
\]
The intermediate options (negotiate, migrate, trade, accumulate) require
$w > 0$. At $w = 0$, the pawn has nothing to lose, and the viability
axiom (\cref{ax:viability}) now applies \emph{to the pawn}: the pawn's
own survival requires a viable path, and submission no longer provides
one. The pawn becomes a knife---$\Ur$ transitions from $\varnothing$ to
$\neq \varnothing$---and the king faces a knife he created by exhausting
the water.
\end{proposition}

\begin{theorem}[Du Mu's theorem]\label{thm:dumu}
Let $w(t)$ be decreasing under extraction. Then:
\[
  w(t) \to 0
  \implies
  \text{pawn} \to \text{knife}
  \implies
  \text{king absorbed.}
\]
The causal chain is internal: the system destroys itself by exhausting
its own substrate.
\end{theorem}

This is the content of Du Mu's \emph{A Fang Gong Fu} \cite{dumu} (825~CE): ``It was
not the Qin who destroyed the six states, but the six states themselves;
it was not the world that destroyed Qin, but Qin itself.'' In our
language: the states depleted their own water, creating the knives that
destroyed them. Qin, having unified, then depleted its own water
(conscription for the Great Wall and Epang Palace), creating the knives
(Chen Sheng, Wu Guang) that destroyed it.

\begin{remark}[Water as viability constraint]
``Water can carry the boat, and water can capsize the boat'' (attributed
to Wei Zheng, Tang dynasty) is not a metaphor. It is a restatement of
\cref{thm:dumu}: the substrate that enables the king's viability
($w > 0$ $\implies$ pawn functions $\implies$ king's path to $\infty$
exists) is the same substrate whose depletion destroys it ($w = 0$
$\implies$ pawn $\to$ knife $\implies$ no path to $\infty$).
\end{remark}

\chapter{Agentic Calculus}\label{sec:calculus}

The preceding sections established the viability axiom, the knife
criterion, and the mean-field interpretation. We now construct an
\emph{operational calculus}---a language for writing algorithms on the
agentic space---that translates every theorem into a flow-theoretic
proposition and yields a complete training paradigm for neural networks.
The central result: the knife is the min-cut, the viable path is the
max-flow, and ``the knife is the mean'' is max-flow/min-cut duality.

\section{The agentic space}\label{sec:tower}

The framework's objects organize into a four-level tower, each level
derived from the axioms of the preceding sections.

\begin{definition}[Agentic space]\label{def:tower}
The \emph{agentic space} is the tower
$\mathbf{L} = (L_0, L_1, L_2, L_3)$:
\begin{enumerate}[label=\textbf{L\arabic*}., ref=L\arabic*]
  \item\label{L0} \textbf{State space} $S$.
  Every configuration of the system is a point in $S$.
  \item\label{L1} \textbf{Viable kernel} $\Viab(K) \subset S$.
  The compact set of states from which the king retains a path to
  infinity (\cref{ax:viability}).
  \item\label{L2} \textbf{Control bundle} $\{U(s)\}_{s \in \Viab(K)}$.
  At each viable state $s$, the fiber $U(s)$ is the set of controls
  that keep the next state inside $\Viab(K)$.
  \item\label{L3} \textbf{Strategy space} $\Gamma$.
  A \emph{strategy} $\gamma \in \Gamma$ is a viable path
  $\gamma: [0,\infty) \to \Viab(K)$ with $\gamma(t+1) \in
  f(\gamma(t), u)$ for some $u \in U(\gamma(t))$ at each step.
\end{enumerate}
\end{definition}

The tower is strict: each level presupposes the one below.
$L_1 \subset L_0$ by definition. $L_2$ exists only over $L_1$
(outside $\Viab(K)$, no control preserves viability). $L_3$ is
built from $L_2$ fibers concatenated over time. The viability axiom
(\cref{ax:viability}) asserts $\Gamma \neq \varnothing$: the strategy
space is non-empty.

\section{The flow}\label{sec:flow}

The agentic calculus is a \emph{flow calculus}. We define flows on
the execution graph and show that every theorem in
\cref{sec:results,sec:meanfield} is a statement about flows and cuts.

\begin{definition}[Execution graph]\label{def:exgraph}
The \emph{execution graph} $G = (V, E, c)$ has:
\begin{itemize}
  \item $V$: agents $\{a_1, \ldots, a_n\}$ plus two distinguished
  nodes: the king $\kappa$ and infinity $\infty$;
  \item $E$: directed edges $(a_i, a_j)$ whenever $a_i$'s actuation
  can affect $a_j$'s state;
  \item $c: E \to \R_{\geq 0}$: edge capacity, where $c(a_i, a_j)$
  is the autonomous actuation that $a_i$ can transmit to $a_j$
  without requiring the king's authorization.
\end{itemize}
An edge $(a_i, a_j)$ with $c(a_i, a_j) > 0$ that does not pass
through $\kappa$ is a \emph{bypass edge}.
\end{definition}

\begin{definition}[Agentic flow]\label{def:flow}
An \emph{agentic flow} is a function $\phi: E \to \R_{\geq 0}$
satisfying:
\begin{enumerate}[label=(\roman*)]
  \item \textbf{Capacity}: $\phi(e) \leq c(e)$ for all $e \in E$.
  \item \textbf{Conservation}: at every non-terminal node $v \neq
  \kappa, \infty$,
  \[
    \sum_{(u,v) \in E} \phi(u,v)
    = \sum_{(v,w) \in E} \phi(v,w).
  \]
\end{enumerate}
The \emph{value} $|\phi|$ is the net flow from $\kappa$ to $\infty$.
A \emph{viable flow} is one with $|\phi| > 0$: the king has a
path to infinity with positive throughput.
\end{definition}

The viability axiom (\cref{ax:viability}) is flow conservation:
what enters the system at $\kappa$ must exit at $\infty$.

\paragraph{Four operations.}
The calculus has four primitive operations on the execution graph:

\begin{center}
\begin{tabular}{@{}llll@{}}
\toprule
\textbf{Operation} & \textbf{Symbol} & \textbf{On $G$} &
\textbf{In 华容道} \\
\midrule
\textsc{Slide} & $\sigma$ & Unit flow along one edge &
One piece moves one cell \\
\textsc{Compose} & $\circ$ & Concatenate along a path &
Sequence of moves \\
\textsc{Cut} & $\partial$ & Remove capacity from an edge set &
Block a corridor \\
\textsc{Phase} & $\varphi$ & Change the capacity function
$c \mapsto c'$ & Phase transition \\
\bottomrule
\end{tabular}
\end{center}

\textsc{Slide} is atomic (unit flow).
\textsc{Compose} builds paths from slides.
\textsc{Cut} is the knife: removing capacity from bypass edges.
\textsc{Phase} is the phase transition: the mean field shifts,
capacities change, the same graph has different flows.

\begin{theorem}[Flow-cut duality]\label{thm:flowcut}
In the execution graph $G$, the maximum viable flow from $\kappa$
to $\infty$ equals the minimum knife-cut capacity:
\[
  \max_\phi |\phi|
  \;=\;
  \min_{C \,\subseteq\, E} \sum_{e \in C} c(e)
  \quad\text{over all $\kappa$-$\infty$ cuts $C$.}
\]
The knife threshold (\cref{thm:meanfield}) is the min-cut value.
The viable path (\cref{ax:viability}) is the max-flow.
``The knife is the mean'' $=$ max-flow equals min-cut.
\end{theorem}

\begin{proof}
By the max-flow/min-cut theorem~\cite{diestel}, the maximum flow from
$\kappa$ to $\infty$ equals the minimum capacity of any
$\kappa$--$\infty$ cut. The knife criterion (\cref{def:knife})
identifies bypass edges---edges with positive capacity that do not pass
through $\kappa$. The king's viability maintenance (cutting knives) is
the operation $c(e) \to 0$ for bypass edges $e$. The residual max-flow
after all bypass edges are cut is the flow through the king (the cut
vertex flow). The min-cut value $=$ the total bypass capacity $=$ the
knife threshold $=$ the mean field's deviation measure.
\end{proof}

\begin{remark}[Flow interpretation of theorems]\label{rem:flowthms}
Each main theorem translates directly:
\begin{itemize}
  \item \textbf{Binary fate} (\cref{thm:lifecycle}): a bypass edge
  either has its capacity set to zero by the holder (path~(a)) or by
  the king (path~(b)). No bypass edge persists with $c > 0$.
  \item \textbf{Fixed-point impossibility} (\cref{thm:fixedpoint}):
  a bypass edge with $c > 0$ cannot ``prove'' $c = 0$. Capacity is
  physical, not narrative.
  \item \textbf{Perpetual elimination} (\cref{thm:paradox}):
  $U = \Umax$ means zero bypass tolerance. As \textsc{Cut} operates,
  \textsc{Phase} lowers the mean, exposing new bypass edges.
  \item \textbf{Du Mu's theorem} (\cref{thm:dumu}): water $=$ total
  network capacity. $w \to 0$ means all capacities shrink to zero:
  frozen, $\Gamma = \varnothing$.
\end{itemize}
\end{remark}

\begin{remark}[抽刀断水水更流]\label{rem:libai}
Li Bai's line assigns the calculus its colours:
\[
  \textcolor{knife}{\text{抽刀}} \;\;
  \textcolor{knife}{\partial} \;\;
  \textcolor{water}{\text{水}} \;\;
  \textcolor{water}{\text{水}}\textcolor{sword}{\text{更流.}}
\]
\textsc{Cut} ($\textcolor{knife}{\partial}$, red) acts on flow
($\textcolor{water}{\sigma}$, blue); flow intensifies. The mechanism
is \textsc{Phase} ($\textcolor{sword}{\varphi}$, cyan): cutting
shifts the mean field (\cref{thm:meanfield}), exposing new bypass
edges, producing more flow---\cref{thm:paradox} in seven characters.
The sword is 青冥 ($\textcolor{sword}{\text{青}}$): the colour of the
mean, the colour of Phase, the colour that connects
$\textcolor{knife}{\text{刀}}$ to $\textcolor{water}{\text{水}}$.
\end{remark}

\section{方圆 $\times$ 黑白: the type system}\label{sec:fangyuan}

The calculus has a type system: a $2 \times 2$ classification that
partitions every element of the agentic space.

\begin{definition}[方圆 $\times$ 黑白]\label{def:fangyuan}
The agentic type system is the product of two binary distinctions:
\begin{center}
\begin{tabular}{@{}lcc@{}}
\toprule
& \textbf{方} (container / structure) &
\textbf{圆} (content / agent) \\
\midrule
\textbf{黑} (constrained / interior) &
Fixed topology (board, graph) &
King $\kappa$ (least mobile, most important) \\
\textbf{白} (free / exterior) &
Free capacity (available edges) &
Pawn (most mobile, least important) \\
\bottomrule
\end{tabular}
\end{center}
The two dynamics of the calculus emerge from this classification:
\begin{itemize}
  \item \textbf{刀} (knife $= \partial$, boundary operator):
  the boundary between 黑 and 白. \textsc{Cut} reclassifies an edge
  from 白 (free capacity) to 黑 (zero capacity).
  \item \textbf{水} (water $= \sigma$, transport operator):
  flow through 白 cells. \textsc{Slide} transports one unit of flow
  along a free edge. Water flows where the knife does not cut.
\end{itemize}
\end{definition}

In 华容道 (\cref{sec:huarongdao}): 方 $=$ the board,
圆 $=$ the pieces. 黑 $=$ occupied cells and the king,
白 $=$ free cells and soldiers. 刀 $=$ 关羽 blocking the corridor.
水 $=$ free-cell flow (slides opposite to piece movement).
The $2 \times 2$ is the type system of the puzzle's state space.

\section{Completeness}\label{sec:completeness}

Every theorem in this paper is a proposition in the agentic calculus.

\begin{proposition}[Calculus completeness]\label{prop:completeness}
The following table maps each theorem to its calculus translation:
\begin{center}
\begin{tabular}{@{}lll@{}}
\toprule
\textbf{Theorem} & \textbf{Calculus statement} &
\textbf{Operations} \\
\midrule
Viability (\ref{ax:viability}) & $|\phi| > 0$ &
$\sigma, \circ$ \\
Binary fate (\ref{thm:lifecycle}) &
$\forall$ bypass $e$: $c(e) \to 0$ &
$\partial$ \\
Fixed point (\ref{thm:fixedpoint}) &
$c(e) > 0 \not\vdash c(e) = 0$ &
--- \\
Paradox (\ref{thm:paradox}) &
$\partial$ generates new bypass via $\varphi$ &
$\partial, \varphi$ \\
Mean field (\ref{thm:meanfield}) &
Min-cut $= \bar{U} + \tau(\Obs)$ &
$\partial$ \\
Cut vertex (\ref{thm:cutvertex}) &
$\kappa =$ min vertex-cut &
structure \\
Du Mu (\ref{thm:dumu}) &
$w \to 0 \Rightarrow c \to 0 \Rightarrow |\phi| = 0$ &
$\sigma \to 0$ \\
Flow-cut (\ref{thm:flowcut}) &
$\max |\phi| = \min |C|$ &
$\sigma, \partial$ \\
\bottomrule
\end{tabular}
\end{center}
\end{proposition}

The calculus is \emph{complete}: no theorem falls outside its four
operations. The agentic space (\cref{def:tower}) provides the domain;
the flow (\cref{def:flow}) provides the dynamics; the type system
(\cref{def:fangyuan}) provides the classification; and flow-cut duality
(\cref{thm:flowcut}) provides the central identity.

\section{The training paradigm}\label{sec:training}

The agentic calculus instantiates as a neural network training paradigm.
The execution graph \emph{is} the computation graph. Training \emph{is}
max-flow optimisation. Survival \emph{is} the viability axiom. The
paradigm strictly subsumes gradient descent.

\begin{definition}[Neural execution graph]\label{def:neural-exgraph}
Let a feedforward network with $L$ layers be given.
Define the execution graph $G = (V, E, c)$ (\cref{def:exgraph}) by:
\begin{itemize}
  \item $V = \{\ell_0, \ell_1, \ldots, \ell_L\}$, one node per layer;
  \item $E = \{(\ell_{i-1}, \ell_i) : 1 \leq i \leq L\}$, one edge
  per weight matrix $W_i$;
  \item $c(e_i) = \|W_i\|_F$, the Frobenius norm as capacity;
  \item king $\kappa = \ell_0$ (input); target
  $\infty = \ell_L$ (output).
\end{itemize}
A data point $(x, y^*)$ initiates flow at $\kappa$ with value $\|x\|$.
Training finds capacities $\{c(e_i)\}$ such that the max-flow matches
the target at $\infty$.
\end{definition}

\paragraph{Operation correspondence.}

\begin{center}
\begin{tabular}{@{}llll@{}}
\toprule
\textbf{Operation} & \textbf{Neural network} &
\textbf{Equation} \\
\midrule
\textsc{Slide} $\sigma$ & One-layer forward pass &
$y = W_e\, x$ \\
\textsc{Compose} $\circ$ & Full forward pass &
$z = \sigma_L \circ W_L \circ \cdots \circ \sigma_1 \circ W_1\, x$ \\
\textsc{Cut} $\partial$ & Pruning / dropout &
$W_e \mapsto 0$, i.e.\ $c(e) \mapsto 0$ \\
\textsc{Phase} $\varphi$ & Regime change &
lr schedule, fine-tuning, curriculum \\
\bottomrule
\end{tabular}
\end{center}

\paragraph{Type-system correspondence.}
Under \cref{def:fangyuan}:
方 $=$ architecture (fixed graph);
圆 $=$ activations (flow $\phi$ traversing the graph);
黒 $=$ frozen weights;
白 $=$ trainable weights;
刀 $=$ pruning operator $\partial$;
水 $=$ data flow forward \emph{and} gradient flow backward.
The backward pass is water flowing opposite to the forward
pass---the free-cell mechanism of \cref{sec:huarongdao}: to move a
piece forward, a free cell slides back.

\begin{definition}[Training algorithm]\label{def:training-algo}
Given $G$ from \cref{def:neural-exgraph} and a dataset $\mathcal{D}$,
the \emph{training paradigm} is the procedure in \cref{fig:training}:
\begin{enumerate}
  \item \textbf{Initialise.} Random $W_i^{(0)}$; set
  $c(e_i) = \|W_i^{(0)}\|_F$.
  \item \textbf{\textsc{Compose}.} Forward pass: $L$ sequential
  \textsc{Slide}s produce
  $z = \sigma_L \circ W_L \circ \cdots \circ \sigma_1 \circ W_1\, x$.
  \item \textbf{Flow deficit.} Loss
  $\mathcal{L} = -|\phi|$.
  \item \textbf{Backward \textsc{Slide}.} Compute
  $\partial\mathcal{L}/\partial c(e_i)$: 水 flowing opposite to
  step~2.
  \item \textbf{Capacity update.} SGD:
  $c(e_i) \leftarrow c(e_i) - \eta\,
  \partial\mathcal{L}/\partial c(e_i)$.
  \item \textbf{Knife detection.} Flag bypass edges where
  $c(e) > \bar{c} + \tau$ (\cref{thm:meanfield}).
  \item \textbf{\textsc{Cut} / \textsc{Phase}.} Prune flagged edges
  ($L_1$ penalty) or change regime (lr, dataset, fine-tuning).
  \item \textbf{Viability check.} Verify $|\phi| > 0$ on held-out
  data (\cref{ax:viability}). If violated: \textsc{Phase} or restart.
  \item \textbf{Repeat} 2--8 until $\max|\phi| = \min|C|$
  (\cref{thm:flowcut}).
\end{enumerate}
\end{definition}

\begin{figure}[H]
\centering
\begin{tikzpicture}[
  node distance=0.9cm and 1.8cm,
  % ── water (水) nodes: blue ──
  wtr/.style={rectangle, rounded corners=3pt, draw=water, thick,
    fill=water!6, minimum width=5.0cm, minimum height=0.7cm,
    align=center, font=\small},
  % ── knife (刀) node: red ──
  knf/.style={diamond, draw=knife, thick, aspect=2.5,
    fill=knife!6, minimum width=1.2cm, align=center, font=\small,
    inner sep=1pt},
  % ── phase (青冥) nodes: cyan ──
  phs/.style={rectangle, rounded corners=3pt, draw=sword, thick,
    fill=sword!6, minimum width=5.0cm, minimum height=0.7cm,
    align=center, font=\small},
  phsd/.style={diamond, draw=sword, thick, aspect=2.5,
    fill=sword!6, minimum width=1.2cm, align=center, font=\small,
    inner sep=1pt},
  % ── neutral (terminal) ──
  term/.style={rectangle, rounded corners=8pt, draw, very thick,
    minimum width=5.0cm, minimum height=0.7cm, align=center,
    font=\small\bfseries},
  % ── convergence diamond ──
  convd/.style={diamond, draw, thick, aspect=2.5,
    minimum width=1.2cm, align=center, font=\small,
    inner sep=1pt},
  arr/.style={-{Stealth[length=5pt]}, thick},
  lbl/.style={font=\scriptsize, fill=white, inner sep=1pt},
  ref/.style={font=\tiny, text=black!55, anchor=west},
  op/.style={font=\scriptsize\itshape}
]

% ── Nodes ──
\node[term] (init)
  {1.\ Initialise: random $c(e)$};

\node[wtr, below=of init] (fwd)
  {2.\ \textsc{Compose}: $\textcolor{water}{\sigma_L \circ
  \cdots \circ \sigma_1}$};

\node[wtr, below=of fwd] (loss)
  {3.\ Flow deficit: $\textcolor{water}{\mathcal{L} = -|\phi|}$};

\node[wtr, below=of loss] (bwd)
  {4.\ Backward \textsc{Slide}:
  $\textcolor{water}{\nabla_c \mathcal{L}}$};

\node[wtr, below=of bwd] (sgd)
  {5.\ Update:
  $\textcolor{water}{c \leftarrow c - \eta\,\nabla_c\mathcal{L}}$};

\node[knf, below=1.1cm of sgd] (knife)
  {6.\ $\textcolor{knife}{c(e) > \bar{c}{+}\tau}$\,?};

\node[phs, right=of knife] (cut)
  {7.\ \textcolor{knife}{\textsc{Cut} $\partial$} /
  \textcolor{sword}{\textsc{Phase} $\varphi$}};

\node[phsd, below=1.1cm of knife] (viable)
  {8.\ $\textcolor{sword}{|\phi| > 0}$\,?};

\node[convd, below=1.1cm of viable] (conv)
  {9.\ $\max|\phi| = \min|C|$\,?};

\node[term, below=1.0cm of conv] (done)
  {Trained network. Survives.};

% ── Reference annotations ──
% init: right (no loop passes here)
\node[ref] at ($(init.east)+(0.15,0)$)
  {Def.~\ref{def:neural-exgraph}};
% steps 2--5: LEFT side (right side reserved for loop-back arrow)
\node[ref, anchor=east] at ($(fwd.west)+(-0.15,0)$)
  {Def.~\ref{def:flow},\; $\sigma$};
\node[ref, anchor=east] at ($(loss.west)+(-0.15,0)$)
  {Thm~\ref{thm:flowcut},\; $|\phi|$};
\node[ref, anchor=east] at ($(bwd.west)+(-0.15,0)$)
  {Def.~\ref{def:flow},\; $\sigma^{-1}$};
% knife: above-right (clear of spine)
\node[ref] at ($(knife.east)+(1.0,0.35)$)
  {Thm~\ref{thm:meanfield}};
% cut/phase: below
\node[ref] at ($(cut.south)+(0,-0.12)$)
  {Thm~\ref{thm:lifecycle}\;($\partial$),\;
   Thm~\ref{thm:paradox}\;($\varphi$)};
% viable: above-right
\node[ref] at ($(viable.east)+(1.0,0.35)$)
  {Ax.~\ref{ax:viability}};
% conv: LEFT side (right side reserved for loop-back arrow)
\node[ref, anchor=east] at ($(conv.west)+(-0.15,0)$)
  {Thm~\ref{thm:flowcut}};

% ── Operation annotations (left margin, coloured) ──
\node[op, text=water, left=1.8cm of fwd] {$\circ$};
\node[op, text=water, left=1.8cm of bwd]
  {$\sigma^{-1}$ (\textcolor{water}{水})};
\node[op, text=sword, right=0.1cm of cut.east] {};

% ── Arrows: main spine ──
\draw[arr] (init) -- (fwd);
\draw[arr, water] (fwd) -- (loss);
\draw[arr] (loss) -- (bwd);
\draw[arr, water] (bwd) -- (sgd);
\draw[arr] (sgd) -- (knife);
\draw[arr] (knife) -- node[lbl, right] {no} (viable);
\draw[arr, sword] (viable) -- node[lbl, right] {yes} (conv);
\draw[arr] (conv) -- node[lbl, right] {yes} (done);

% ── Arrows: branches ──
\draw[arr, knife] (knife) -- node[lbl, above] {yes} (cut);
\draw[arr, sword] (cut) |- (viable);

% ── Loop back: not converged → step 2 ──
\draw[arr] (conv.east) -- ++(7.5,0)
  node[lbl, above, pos=0.15] {no}
  |- (fwd.east);

% ── Viability failure → phase/restart ──
\draw[arr, sword] (viable.west) -- ++(-5.0,0)
  node[lbl, above, pos=0.25] {no}
  |- (init.west);

% ── Brace: forward flow (blue) ──
\draw[decorate, decoration={brace, amplitude=4pt, mirror},
  thick, water!60]
  ($(fwd.north east)+(0.12,0.05)$) --
  ($(loss.south east)+(0.12,-0.05)$)
  node[midway, right=5pt, font=\scriptsize, text=water]
  {\textcolor{water}{flow $\to$}};

% ── Brace: backward water (blue) ──
\draw[decorate, decoration={brace, amplitude=4pt},
  thick, water!60]
  ($(bwd.north west)+(-0.12,0.05)$) --
  ($(sgd.south west)+(-0.12,-0.05)$)
  node[midway, left=5pt, font=\scriptsize, text=water]
  {$\leftarrow$ \textcolor{water}{水}};

\end{tikzpicture}
\caption{The training paradigm as roadmap.
\textcolor{water}{Blue} (水): flow operations
(steps 2--5, Defs.~\ref{def:flow}--\ref{def:exgraph}).
\textcolor{knife}{Red} (刀): knife detection
(step~6, Thm~\ref{thm:meanfield}).
\textcolor{sword}{Cyan} (青冥): phase and viability
(steps 7--8, Thms~\ref{thm:lifecycle},
\ref{thm:paradox}, Ax.~\ref{ax:viability}).
Convergence (step~9): $\max|\phi| = \min|C|$
(Thm~\ref{thm:flowcut}).
The loop is 抽刀断水水更流: \textcolor{knife}{cut}
$\to$ \textcolor{sword}{phase} $\to$
\textcolor{water}{more flow}.}
\label{fig:training}
\end{figure}

\begin{theorem}[Training completeness and survival]
\label{thm:training}
Let $f: \mathcal{X} \to \mathcal{Y}$ be any function expressible as a
max-flow on an execution graph $G$. Then the procedure in
\cref{def:training-algo} learns $f$. The trained network satisfies the
viability axiom: $|\phi| > 0$ for all admissible inputs.
\end{theorem}

\begin{proof}
Expressibility as max-flow is the universal approximation condition in
the language of \cref{def:neural-exgraph}. Each iteration of steps~2--5
increases $|\phi|$ toward the max-flow value guaranteed by
\cref{thm:flowcut}. The \textsc{Cut} in step~7 removes edges with
anomalous capacity; by flow-cut duality, removing a bypass edge does not
reduce the max-flow if that edge is not on any minimum cut.
Knife detection (step~6) identifies precisely such edges via the
mean-field bound of \cref{thm:meanfield}. Viability is preserved
through every \textsc{Cut}.
\end{proof}

\begin{theorem}[你不能骗你自己]\label{thm:nolie}
Let $e \in E$ with $c(e) = \|W_e\|_F > 0$. Then $e$ carries positive
flow. No re-labelling, no narrative, no hyper-parameter choice changes
this. This is \cref{thm:fixedpoint} applied to the capacity map
$c: E \to \R_{\geq 0}$.
\end{theorem}

\begin{proof}
Two paths only.
\begin{enumerate}[label=(\alph*)]
  \item $W_e \to 0$: capacity zeroed, edge pruned (\textsc{Cut}).
  \item Regularisation ($L_1$/$L_2$) drives $c(e) \to 0$:
  \textsc{Cut} in the limit.
\end{enumerate}
There is no path~(c). ``Approximately zero'' is not zero: such weights
are unstable fixed points of gradient flow (\cref{thm:paradox}), and
the viability axiom forces the system to resolve the ambiguity via
\cref{prop:binary}---the action space collapses to $\{0, 1\}$. The
network cannot occupy the gap. 你不能骗你自己---you cannot lie to
yourself---is \cref{thm:fixedpoint} applied to the network's own
weights.
\end{proof}

\begin{remark}[Subsumption]\label{rem:subsumption}
Gradient descent is the special case of the training paradigm in which
viability reduces to ``loss below threshold'' and the only operations
exercised are \textsc{Slide} (forward) and $\textsc{Slide}^{-1}$
(backward). \textsc{Cut} and \textsc{Phase} are what gradient descent
implicitly performs when early-stopping, dropout, or learning-rate
annealing are applied---now made explicit and first-class in the
calculus. The paradigm strictly subsumes gradient descent.
\end{remark}

\section{The Hilbert space}\label{sec:hilbert}

The capacity measure gives the agentic space the structure of a Hilbert
space. This single construction yields integration, spectral analysis,
and a direct connection to physics.

\begin{definition}[Capacity measure]\label{def:measure}
The \emph{capacity measure} $\mu_c$ on $E$ is defined for any
$A \subseteq E$ by
\[
  \mu_c(A) \;=\; \sum_{e \in A} c(e).
\]
This is the measure induced by the capacity function $c$ of the
execution graph (\cref{def:exgraph}).
\end{definition}

\begin{definition}[Agentic Hilbert space]\label{def:hilbert}
The \emph{agentic Hilbert space} is
\[
  \mathcal{H} \;=\; L^2(E,\,\mu_c),
\]
the space of square-integrable flows on $E$, with inner product
\[
  \langle \phi_1,\, \phi_2 \rangle_c
  \;=\; \sum_{e \in E} \phi_1(e)\,\phi_2(e)\,c(e)
\]
and norm $\|\phi\|_c = \sqrt{\langle \phi,\phi \rangle_c}$.
Every agentic flow (\cref{def:flow}) is a vector in $\mathcal{H}$.
\end{definition}

\begin{proposition}[Integration]\label{prop:integration}
The flow-cut duality of \cref{thm:flowcut} is a statement about
integrals and measures on $\mathcal{H}$:
\begin{enumerate}[label=(\roman*)]
  \item The max-flow value is a boundary integral:
  \[
    |\phi| \;=\; \int_{\partial\kappa} \phi\, d\mu_c
    \;=\; \sum_{e \,\mathrm{out\,of}\, \kappa} \phi(e)\,c(e).
  \]
  \item The min-cut value is the measure of the cut:
  \[
    |C| \;=\; \mu_c(C) \;=\; \sum_{e \in C} c(e).
  \]
  \item Max-flow/min-cut duality (\cref{thm:flowcut}) is:
  \[
    \max_\phi \int_{\partial\kappa} \phi\, d\mu_c
    \;=\;
    \min_{C} \mu_c(C).
  \]
\end{enumerate}
\end{proposition}

\begin{proof}
(i) is conservation at $\kappa$: all flow leaving $\kappa$ is counted
once, weighted by capacity. (ii) is the definition of $\mu_c$ restricted
to $C$. (iii) is \cref{thm:flowcut} rewritten in the language of
$\mu_c$.
\end{proof}

\begin{proposition}[Operators on $\mathcal{H}$]\label{prop:operators}
The four calculus operations (\cref{sec:flow}) are operators on
$\mathcal{H}$:
\begin{enumerate}[label=(\roman*)]
  \item \textbf{\textsc{Slide}} $\sigma_e$: shift operator along edge
  $e \in E$,
  \[
    (\sigma_e \phi)(e') \;=\; \phi(e') + \delta_{e,e'}.
  \]
  \item \textbf{\textsc{Compose}}: product of shifts along a path
  $p = (e_1, \ldots, e_n)$,
  \[
    \sigma_p \;=\; \sigma_{e_n} \circ \cdots \circ \sigma_{e_1}.
  \]
  \item \textbf{\textsc{Cut}} $\partial_A$ for $A \subseteq E$:
  orthogonal projection onto the closed subspace
  $\mathcal{H}_A = \{\phi \in \mathcal{H} : \phi|_A = 0\}$,
  \[
    (\partial_A \phi)(e) \;=\;
    \begin{cases} 0 & e \in A, \\ \phi(e) & e \notin A. \end{cases}
  \]
  Cutting a bypass edge $e$ is the projection $\partial_{\{e\}}$.
  \item \textbf{\textsc{Phase}} $\varphi_{c'}$: change of measure
  $c \mapsto c'$, inducing
  \[
    \mathcal{H} = L^2(E,\mu_c)
    \;\xrightarrow{\;\varphi_{c'}\;}
    \mathcal{H}' = L^2(E,\mu_{c'}).
  \]
  A phase transition changes the Hilbert space itself, not merely a
  vector in a fixed space.
\end{enumerate}
\end{proposition}

For the neural execution graph (\cref{def:neural-exgraph}), the inner
product specialises to
$\langle \phi_1, \phi_2 \rangle_c
= \sum_{i=1}^{L} \phi_1(e_i)\,\phi_2(e_i)\,\|W_i\|_F$,
a capacity-weighted $\ell^2$ norm on layer activations. The
\textsc{Cut} operator $\partial_A$ is Frobenius-norm pruning; the
\textsc{Phase} operator $\varphi_{c'}$ is a change of architecture.
The Hilbert-space structure makes these operations precise: projection,
not approximation.

\paragraph{Spectral theory.}
The inner product $\langle \cdot, \cdot \rangle_c$ opens the agentic
space to spectral analysis.

\begin{definition}[Graph Laplacian]\label{def:laplacian}
The \emph{graph Laplacian} $\Delta \colon L^2(V) \to L^2(V)$ of the
execution graph $G = (V, E, c)$ (\cref{def:exgraph}) is
\[
  (\Delta f)(v)
  \;=\;
  \sum_{(v,w)\in E} c(v,w)\bigl[f(v) - f(w)\bigr].
\]
$\Delta$ is positive semi-definite.
Write its eigenvalues in non-decreasing order:
$0 = \lambda_0 \leq \lambda_1 \leq \cdots \leq \lambda_n$.
The \emph{spectral gap} is $\lambda_1$.
\end{definition}

\begin{theorem}[Cheeger inequality --- agentic form]\label{thm:cheeger}
Let $\mu_c(\partial S)$ denote the total capacity of edges
crossing from $S \subset V$ to $V \setminus S$, and let
$\mathrm{vol}(S) = \sum_{v \in S} \sum_{(v,w)\in E} c(v,w)$.
The \emph{Cheeger constant} of $G$ is
\[
  h(G)
  \;=\;
  \min_{\substack{S \subset V \\ \kappa \in S}}
  \frac{\mu_c(\partial S)}{\min\!\bigl(\mathrm{vol}(S),\,
  \mathrm{vol}(V \setminus S)\bigr)}.
\]
Then~\cite{cheeger,mohar}
\[
  \frac{\lambda_1}{2}
  \;\leq\;
  h(G)
  \;\leq\;
  \sqrt{2\lambda_1}.
\]
\end{theorem}

\begin{remark}\label{rem:cheeger}
The Cheeger constant $h(G)$ is the \emph{normalised knife threshold}.
The numerator $\mu_c(\partial S)$ is the capacity of the cut separating
$S$ from $V \setminus S$ (\cref{thm:flowcut}); the denominator
normalises by volume.
\textcolor{knife}{The knife is the cut.}
\textcolor{water}{The flow is the volume.}
\textcolor{sword}{The Cheeger inequality bridges the two.}
Spectral gap $\lambda_1$ and normalised min-cut $h(G)$ are within a
factor of $2\sqrt{2}$ of each other: the same obstruction, measured
twice.
\end{remark}

\begin{theorem}[Mass gap $=$ viability]\label{thm:massgap}
The following are equivalent:
\begin{enumerate}[label=\textup{(\roman*)}]
  \item\label{mg:spectral} $\lambda_1 > 0$ \quad (spectral gap).
  \item\label{mg:cheeger} $h(G) > 0$ \quad (positive Cheeger constant).
  \item\label{mg:flow} $|\phi| > 0$ for some agentic flow $\phi$
  \quad (viability axiom, \cref{ax:viability}).
  \item\label{mg:connect} $\kappa$ can reach $\infty$ in $G$
  \quad (connectivity).
\end{enumerate}
\end{theorem}

\begin{proof}
\ref{mg:spectral}$\Leftrightarrow$\ref{mg:cheeger}:
Cheeger's inequality (\cref{thm:cheeger}) gives
$\lambda_1/2 \leq h(G) \leq \sqrt{2\lambda_1}$,
so $\lambda_1 > 0$ if and only if $h(G) > 0$.

\ref{mg:cheeger}$\Leftrightarrow$\ref{mg:flow}:
$h(G) > 0$ means every $\kappa$--$\infty$ cut has strictly positive
capacity. By \cref{thm:flowcut}, max-flow equals min-cut; hence
$\max_\phi |\phi| > 0$.

\ref{mg:flow}$\Leftrightarrow$\ref{mg:connect}:
A flow $\phi$ with $|\phi| > 0$ exists if and only if there is a
directed path from $\kappa$ to $\infty$ with positive capacity on
every edge.
\end{proof}

\begin{remark}[一石三鸟]\label{rem:threebirds}
The Hilbert space $\mathcal{H} = L^2(E, \mu_c)$ (\cref{def:hilbert})
kills three birds with one stone:
\begin{itemize}
  \item \textbf{\textcolor{water}{Integration.}}
  $\int_E f\,d\mu_c$ realises max-flow as boundary integral and
  min-cut as measure. Flow-cut duality (\cref{thm:flowcut}) becomes
  an identity of integrals.

  \item \textbf{\textcolor{sword}{Analysis.}}
  The graph Laplacian $\Delta$ (\cref{def:laplacian}) acts on
  $L^2(V)$; the Cheeger inequality (\cref{thm:cheeger}) bridges
  the combinatorial quantity $h(G)$ and the analytic quantity
  $\lambda_1$.

  \item \textbf{\textcolor{knife}{Physics.}}
  $\lambda_1 > 0$ is the mass gap. \Cref{thm:massgap} says the
  viability axiom IS the mass gap: the Hilbert space changes, the
  equivalence does not.
\end{itemize}
\end{remark}

\begin{remark}[Spectral gap and training]\label{rem:spectralgap}
The spectral gap $\lambda_1$ is computable. For a neural network
(\cref{def:neural-exgraph}), $\lambda_1$ of the execution graph
measures the viability margin: how far the network is from the
degenerate regime $|\phi| = 0$.

Training (\cref{def:training-algo}) increases $\lambda_1$: backward
\textsc{Slide}s increase capacities on edges that carry flow, widening
the spectral gap. Convergence $\max|\phi| = \min|C|$
(\cref{thm:training}) is the statement that $\lambda_1$ has reached
the Cheeger bound.

A trained network with $\lambda_1 > 0$ has a mass gap. It survives.
\end{remark}

\section{Contact dynamics}\label{sec:contact}

The training paradigm (\cref{def:training-algo}) is not a gradient
flow. It is a \emph{contact gradient flow}: gradient descent plus a
dissipation term supplied by the knife. This distinction is the
difference between symplectic mechanics (energy conserved) and contact
mechanics (energy dissipates). The knife is the dissipation.

\begin{definition}[Contact structure on the capacity space]
\label{def:contact}
Let $\mathcal{C} = \R_{\geq 0}^{|E|}$ be the space of capacity
assignments on the execution graph $G$ (\cref{def:exgraph}).
The \emph{extended capacity space} is $\mathcal{C} \times \R$, with
coordinates $(c, s)$ where $s = |\phi|(c)$ is the max-flow value.
The \emph{contact $1$-form} is
\[
  \alpha \;=\; ds \;-\; \sum_{e \in E}
  \frac{\partial|\phi|}{\partial c(e)}\, dc(e).
\]
The kernel $\ker\alpha$ is the constraint surface: infinitesimal
changes in capacity that are consistent with flow conservation.
\end{definition}

\begin{definition}[Knife dissipation]\label{def:dissipation}
The \emph{knife function} $\gamma \colon E \to \R_{\geq 0}$ is
\[
  \gamma(e) \;=\;
  \begin{cases}
    \gamma_0 & c(e) > \bar{c} + \tau
    \quad\text{(\cref{thm:meanfield})}, \\
    0 & \text{otherwise},
  \end{cases}
\]
where $\gamma_0 > 0$ is the dissipation rate. Weight decay ($L_2$
regularisation) is the special case $\gamma(e) = \lambda$ for all $e$.
\end{definition}

\begin{definition}[Contact gradient flow]\label{def:contactflow}
The \emph{contact gradient flow} on $(\mathcal{C} \times \R, \alpha)$
is
\begin{equation}\label{eq:contactflow}
  \frac{dc(e)}{dt}
  \;=\;
  \underbrace{\frac{\partial|\phi|}{\partial c(e)}}_
  {\textcolor{water}{\text{水: gradient}}}
  \;-\;
  \underbrace{\gamma(e)\, c(e)}_
  {\textcolor{knife}{\text{刀: dissipation}}}.
\end{equation}
The first term increases capacity along the flow gradient (backward
\textsc{Slide}, step~4 of \cref{def:training-algo}). The second term
decreases capacity on flagged edges (knife detection + \textsc{Cut},
steps~6--7). SGD with weight decay is this equation discretised with
$\gamma(e) = \lambda$.
\end{definition}

\begin{theorem}[Contact Euler--Lagrange equation]
\label{thm:contactEL}
At equilibrium of the contact gradient flow~\eqref{eq:contactflow}:
\begin{equation}\label{eq:contactEL}
  \frac{\partial|\phi|}{\partial c(e)}
  \;=\;
  \gamma(e)\, c(e)
  \qquad \forall\, e \in E.
\end{equation}
This is the \emph{contact Euler--Lagrange equation} of the training
paradigm.
\end{theorem}

\begin{proof}
Set $dc/dt = 0$ in~\eqref{eq:contactflow}.
\end{proof}

\begin{corollary}[Binary lifecycle from contact dynamics]
\label{cor:contact-lifecycle}
Let $e \in E$ be a bypass edge with $c(e) > \bar{c} + \tau$.
Then $\gamma(e) = \gamma_0 > 0$, so~\eqref{eq:contactEL} requires
$\partial|\phi|/\partial c(e) = \gamma_0\, c(e) > 0$: the edge must
carry flow proportional to its capacity.
If the edge does \emph{not} carry proportional flow
($\partial|\phi|/\partial c(e) < \gamma_0\, c(e)$), then
$dc(e)/dt < 0$ and the capacity decays to zero.
This is \cref{thm:lifecycle} derived from the contact flow:
every bypass edge either justifies its capacity or loses it.
\end{corollary}

\begin{remark}[Du Mu as contact collapse]\label{rem:contact-dumu}
Du Mu's theorem (\cref{thm:dumu}) is the regime $\gamma(e) \to \infty$
for all $e$: maximum dissipation.
The contact flow~\eqref{eq:contactflow} drives all capacities to zero
regardless of the gradient. The system freezes:
$c \to 0$, $|\phi| \to 0$, $\Gamma = \varnothing$.
In contact-geometric language, the Reeb vector field dominates the
Hamiltonian vector field, and the flow collapses onto the zero section.
\end{remark}

\begin{remark}[Contact structure and the three colours]
\label{rem:contact-colours}
The contact flow~\eqref{eq:contactflow} is a competition between two
terms:
\begin{itemize}
  \item $\textcolor{water}{\partial|\phi|/\partial c(e)}$: the gradient
  pushes capacity \emph{up} (水, flow).
  \item $\textcolor{knife}{\gamma(e)\,c(e)}$: the knife pushes capacity
  \emph{down} (刀, dissipation).
\end{itemize}
The equilibrium~\eqref{eq:contactEL} is their balance.
\textsc{Phase} ($\textcolor{sword}{\varphi}$, cyan) changes the
contact structure itself: a regime change shifts $\gamma$, $\bar{c}$,
and $\tau$, altering the equilibrium.
The contact $1$-form $\alpha$ encodes all three:
$\textcolor{water}{\text{水}}$ in the gradient,
$\textcolor{knife}{\text{刀}}$ in the dissipation,
$\textcolor{sword}{\text{青冥}}$ in the form itself.
\end{remark}

\begin{remark}[Stability]\label{rem:contact-stability}
The spectral gap $\lambda_1 > 0$ (\cref{thm:massgap}) is the
stability condition of the contact equilibrium~\eqref{eq:contactEL}.
Linearising the contact flow around the equilibrium, the eigenvalues
of the linearised system are bounded below by $\lambda_1$: small
perturbations in capacity decay at rate $\geq \lambda_1$.
The mass gap is the stability margin of the contact Euler--Lagrange
equation.
\end{remark}

\begin{remark}[The standing structure]\label{rem:standing}
A quadruped robot instantiates the execution graph physically:
$12$ joints (edges, capacity $=$ torque $\times$ range),
$4$ feet (terminal nodes, ground contact),
$1$ torso centre of mass (the king $\kappa$).
The viability condition (\cref{ax:viability}): the centre of mass
lies within the support polygon of grounded feet.

The support is the \emph{contact mode combinatoric} and the
\emph{constraints}---that is all:
\begin{itemize}
  \item \textbf{Mode lattice.}
  Each foot is grounded ($1$) or lifted ($0$), giving a contact mode
  $c \in \{0,1\}^4$ with $|\mathcal{C}| = 16$ modes.
  The support polygon exists when $|c| \geq 3$.
  \item \textbf{Constraints} (Unitree Go2, MuJoCo Menagerie).
  Joint limits: ab/ad $\in [-0.863,\, 0.863]$\,rad,
  hip $\in [-1.047,\, 3.490]$\,rad,
  knee $\in [-2.697,\, {-0.837}]$\,rad.
  Torque: $[23.7,\; 23.7,\; 35.55]$\,Nm.
  Velocity: $21.0$\,rad/s per joint.
  Standing height: $0.312$\,m.
  Sinking bound: $z_{\mathrm{contact}} \geq z_{\min} = -3$\,mm
  (no contact point may penetrate below $z_{\min}$;
  \cref{rem:sinking-bound}).
  These are the capacity bounds $c(e) \leq c_{\max}(e)$ of the
  execution graph.
\end{itemize}
In any command space~$U$ and any gravitational field~$g$, the same
graph supports viability.
\emph{Do anything---or go with the flow---as long as a viable path
exists---anywhere.}

The tripod gait is redundancy in the min-cut: three feet grounded,
one moving, so the support polygon persists even during locomotion.
The contact dynamics of \cref{sec:contact} is here literal: the
contact $1$-form~$\alpha$ encodes the foot--ground interface, and
the contact gradient flow~\eqref{eq:contactflow} governs joint
torque allocation.

Twelve joints, four feet, one centre of mass.
A standing structure.
\end{remark}

\begin{definition}[Command field]\label{def:command}
The \emph{command field} is a distribution $U$ over velocity
commands $u \in \R^3$ (linear and angular velocity targets).
The \emph{curriculum} is a filtration of command fields of
increasing support:
\[
  \underbrace{U_0 = \delta_0}_{\text{Stand: zero command}}
  \;\subset\;
  \underbrace{U_1 = \mathrm{Uniform}(\mathcal{V})}_
  {\text{Walk: all feasible velocities}}
  \;\subset\;
  \underbrace{U_2 = \rho_{\mathrm{task}}}_
  {\text{Work: task distribution}}.
\]
At phase~$k$, the policy $\pi$ receives $u \sim U_k$ and must
produce torques $\tau = \pi(o, u)$ such that $|\phi| > 0$
(\cref{ax:viability}).
Promotion from phase~$k$ to $k+1$ requires
$\max|\phi| = \min|C|$ (\cref{thm:flowcut}) at~$U_k$:
flow-cut duality achieved for the current command field.
\end{definition}

\begin{definition}[Observation space]\label{def:observation}
The policy $\pi$ receives a composite observation
$o = (o_{\mathrm{prop}},\, o_{\mathrm{vis}}) \in \R^{42+d}$:
\begin{itemize}
  \item \textbf{Proprioception}
  $o_{\mathrm{prop}} \in \R^{42}$:
  body orientation $R \in \mathrm{SO}(3)$ (flattened to $\R^9$),
  position $p \in \R^3$,
  angular velocity $\omega \in \R^3$,
  linear velocity $v \in \R^3$,
  joint angles $\theta \in \R^{12}$,
  joint velocities $\dot\theta \in \R^{12}$.
  Sources: IMU and joint encoders.
  \item \textbf{Vision}
  $o_{\mathrm{vis}} = \varphi_{\mathrm{vis}}(I) \in \R^{d}$:
  the camera image $I$ is rendered by the engine
  ($\mathcal{E}.\texttt{render}$: $q \to I$, non-differentiable)
  and encoded by a frozen DINOv2 encoder
  (ViT-S/14, $d = 384$, pretrained on ImageNet,
  \emph{not updated} by \cref{alg:loco}).
  Two walls block the gradient:
  \[
    q \;\xrightarrow[\text{no } \nabla]{\;\texttt{render}\;}
    I \;\xrightarrow[\text{frozen}]{\;\varphi_{\mathrm{vis}}\;}
    o_{\mathrm{vis}}
    \;\xrightarrow[\nabla \text{ flows}]{\;\pi\;}
    \tau.
  \]
  Gradients from the contact flow~\eqref{eq:contactflow}
  propagate through $\pi$ but stop at $\varphi_{\mathrm{vis}}$.
  The renderer and the encoder are both non-differentiable
  with respect to the training objective.
\end{itemize}
Proprioception tells the robot \emph{where it is}.
Vision tells the robot \emph{what is there}.
The policy $\pi(o, u)$ fuses both to produce
$\tau \in \R^{12}$.
\end{definition}

\begin{definition}[Soft viability margin]\label{def:soft-viability}
The hard viability margin
$|\phi| = \min_t\{x_1(t), \ldots, x_k(t)\}$
in the \textsc{Evaluate} step is non-differentiable:
$\nabla \min$ is sparse (only the argmin receives gradient) and
discontinuous at ties.
Replace the hard $\min$ with its smooth approximation:
\[
  \mathrm{soft\text{-}min}_\beta(x_1, \ldots, x_k)
  \;=\;
  -\frac{1}{\beta}\,
  \log\!\Bigl(\sum_{i=1}^{k} e^{-\beta\, x_i}\Bigr).
\]
At $\beta \to \infty$, this recovers the hard $\min$.
At finite $\beta$, every component $x_i$ receives gradient
proportional to $e^{-\beta\, x_i}/\sum_j e^{-\beta\, x_j}$:
a softmax weighting.
The \emph{soft viability margin} is
\begin{equation}\label{eq:soft-viability}
  |\phi|_\beta
  \;=\;
  \mathrm{soft\text{-}min}_\beta\,\bigl(
  \underbrace{h(q_t) - h_{\min}}_{\text{height margin}},\;\;
  \underbrace{\phi_{\max} - \|R_t - I\|_F}_
  {\text{orientation margin}},\;\;
  \underbrace{\min_j z_j(q_t) - z_{\min}}_
  {\text{sinking margin}}
  \bigr)_{t=0}^{H}.
\end{equation}
The temperature $1/\beta$ controls how many timesteps share the
gradient.
Small $\beta$ (warm): all timesteps contribute.
Large $\beta$ (cold): only the worst timestep contributes.
The three thresholds $h_{\min}$, $\phi_{\max}$, $z_{\min}$ are
read from the constraint list (\cref{rem:standing}); the sinking
margin is not a heuristic but a categorical boundary
(\cref{rem:sinking-bound}).
\end{definition}

\begin{definition}[Mollifier policy]\label{def:mollifier}
The policy $\pi_\theta$ outputs a \emph{truncated Student-$t$
distribution} over joint torques:
\begin{equation}\label{eq:mollifier}
  \pi_\theta(\tau \mid o, u)
  \;=\;
  \prod_{i=1}^{12}
  \frac{t_{\nu}(\tau_i;\, \mu_i, \sigma_i)}
       {Z_i(\mu_i, \sigma_i, \nu)},
\end{equation}
where $(\mu, \log\sigma, \log\nu) = f_\theta(o, u)$ are three
output heads of the network (\cref{def:computation-core}):
$12$ means, $12$ log-scales, and $1$ shared
log-degrees-of-freedom.
The truncation $|\tau_i| \leq \bar{\tau}_i$ enforces joint torque
limits (\cref{rem:standing}).

The Student-$t$ density $t_\nu$ is $C^\infty$ with full support on
$[-\bar\tau, \bar\tau]^{12}$.
For any Lipschitz function $Q(\tau)$---including functions with
jump discontinuities at contact-mode boundaries---the expected
value
\[
  \bar{Q}(\theta)
  \;=\;
  \mathbb{E}_{\pi_\theta}\!\bigl[Q(\tau)\bigr]
  \;=\;
  \int Q(\tau)\,\pi_\theta(\tau \mid o, u)\, d\tau
\]
is smooth in $\theta$: the policy acts as a \emph{mollifier}.

The degrees of freedom $\nu$ track the number of active contact
modes:
\begin{center}
\begin{tabular}{@{}lll@{}}
\toprule
\textbf{Gait} & $\boldsymbol{\nu}$ &
\textbf{Shape} \\
\midrule
Standing ($c = 1111$), one mode
  & $\to \infty$ & Gaussian (peaked) \\
Trotting, two alternating modes
  & $\approx 2$--$5$ & moderate tails \\
Bounding / flight, many transitions
  & $\approx 1$--$2$ & heavy tails (exploratory) \\
\bottomrule
\end{tabular}
\end{center}
At $\nu \to \infty$ the Student-$t$ recovers the Gaussian
(standard SAC).
At finite $\nu$ the heavy tails absorb the multi-modality of the
contact landscape.
\end{definition}

\begin{definition}[Computation core]\label{def:computation-core}
The policy $\pi$ computes via three primitives at each
layer $\ell = 1, \ldots, L$:
\begin{enumerate}[label=(\roman*)]
  \item \textbf{\textcolor{water}{Matmul}}:\;
  $y \leftarrow W_\ell\, z_{\ell-1}$.
  Transport the activation vector from layer $\ell{-}1$ to~$\ell$.
  \textsc{Slide}: one step of parallel transport
  (\cref{rem:gauge}).
  \item \textbf{\textcolor{water}{Add}}:\;
  $y \leftarrow y + b_\ell$.
  Shift the transported vector by the bias.
  Affine extension of \textsc{Slide}.
  \item \textbf{\textcolor{knife}{Activate}}:\;
  $z_\ell \leftarrow \mathrm{ReLU}(y) = \max(0,\, y)$,
  componentwise.
  Each neuron either passes its signal ($y_i > 0$, contact)
  or blocks it ($y_i \leq 0$, no contact)---a binary gate
  in $O(1)$ time.
  \textsc{Cut}: the knife at the neuron level.
\end{enumerate}
One layer:
$z_\ell = \textcolor{knife}{\mathrm{ReLU}}\bigl(
\textcolor{water}{W_\ell\, z_{\ell-1} + b_\ell}\bigr)$.
The output layer is linear (no gate):
$\tau = \textcolor{water}{W_L\, z_{L-1} + b_L} \in \R^{12}$.
The full forward pass (\textsc{Compose}):
\begin{equation}\label{eq:forward}
  \tau \;=\; \pi(o)
  \;=\;
  \textcolor{water}{A_L} \circ
  \bigl(\textcolor{knife}{r} \circ
  \textcolor{water}{A_{L-1}}\bigr)
  \circ \cdots \circ
  \bigl(\textcolor{knife}{r} \circ
  \textcolor{water}{A_1}\bigr)(o),
\end{equation}
where $A_\ell(\cdot) = W_\ell\,(\cdot) + b_\ell$ and
$\textcolor{knife}{r} = \mathrm{ReLU}$.

\textcolor{water}{水} $=$ linear transport ($W, b$).
\textcolor{knife}{刀} $=$ ReLU gate
($\max(0, \cdot)$: pass or block).
ReLU is piecewise linear and runs in real linear time---the
fastest nonlinearity, meeting the deployment constraint of
\cref{rem:deployment}.
\end{definition}

\begin{remark}[The forward pass as path integral]
\label{rem:path-integral}
For a given input $o$, each ReLU neuron is either active
($y_i > 0$: contact) or inactive ($y_i \leq 0$: no contact).
The \emph{activation pattern}
$\alpha \in \{0,1\}^{n_1 + \cdots + n_{L-1}}$
is the contact mode of the network---the neural analogue of the
contact mode lattice $\{0,1\}^4$ in \cref{rem:standing}.

For a fixed pattern $\alpha$, the network is linear:
\[
  \pi_\alpha(o)
  \;=\;
  \textcolor{water}{W_L\, D_{L-1}^\alpha\, W_{L-1}\,
  D_{L-2}^\alpha \cdots D_1^\alpha\, W_1}\; o
  \;+\; \mathrm{bias},
\]
where $D_\ell^\alpha = \mathrm{diag}(\alpha_\ell)$ masks
inactive neurons.
Different inputs activate different patterns:
the input space is partitioned into linear regions
$\{R_\alpha\}$, each with its own transport map.
The full forward pass is a path integral over activation
patterns:
\[
  \tau \;=\; \pi(o)
  \;=\;
  \sum_{\alpha} \mathbf{1}_{o \in R_\alpha}\;
  \pi_\alpha(o).
\]
Sum over patterns, one active per input: a discrete path integral
with binary action variable.

Three scales of the same gate:
\begin{enumerate}[label=(\alph*)]
  \item \textbf{Micro}: ReLU inside one layer---one neuron,
  one binary gate
  ($\textcolor{knife}{刀}$: pass or block).
  \item \textbf{Trajectory}: $L$ layers composed---one
  forward pass, one activation pattern $\alpha$, one linear
  path through the network.
  \item \textbf{Macro}:
  $\mathrm{soft\text{-}min}_\beta$ over $N$ rollouts of
  \cref{alg:loco}---Boltzmann reweighting
  ($\textcolor{sword}{青冥}$) of physical trajectories,
  selecting the viable ($|\phi|_\beta > 0$).
\end{enumerate}
$\textcolor{knife}{刀}$ at the neuron scale (hard gate).
$\textcolor{sword}{青冥}$ at the rollout scale (soft gate).
Contact or no contact, $\{0,1\}$ at every level.
\end{remark}

\begin{algorithm}[H]
\caption{Forward pass
  (\textsc{Compose} of \cref{def:computation-core})}
\label{alg:forward}
\begin{algorithmic}[1]
\Require Weights $\{W_\ell,\, b_\ell\}_{\ell=1}^{L}$
\Statex
\Function{\textsc{Forward}}{$o,\, u$}
  \State $z_0 \leftarrow [o;\; u]$
    \Comment{\textcolor{water}{水: concatenate input}}
  \For{$\ell = 1, \ldots, L{-}1$}
    \State \textcolor{water}{\textsc{Matmul}:}\;
      $y \leftarrow W_\ell\, z_{\ell-1}$
      \Comment{\textcolor{water}{水: Slide}}
    \State \textcolor{water}{\textsc{Add}:}\;
      $y \leftarrow y + b_\ell$
      \Comment{\textcolor{water}{水: affine Slide}}
    \State \textcolor{knife}{\textsc{Activate}:}\;
      $z_\ell \leftarrow \max(0,\, y)$
      \Comment{\textcolor{knife}{刀: ReLU (Cut)}}
  \EndFor
  \State \textcolor{water}{\textsc{Output}:}\;
    $\tau \leftarrow W_L\, z_{L-1} + b_L$
    \Comment{\textcolor{water}{水: Slide (no gate)}}
  \State \Return $\tau \in \R^{12}$
    \Comment{saves $\{z_\ell, y_\ell\}$ for \cref{alg:backward}}
\EndFunction
\end{algorithmic}
\end{algorithm}

\begin{algorithm}[H]
\caption{Backward pass
  (pullback through $\pi$, dual of \cref{alg:forward})}
\label{alg:backward}
\begin{algorithmic}[1]
\Require Saved $\{z_\ell\}_{\ell=0}^{L-1}$,\;
  $\{y_\ell\}_{\ell=1}^{L-1}$ from \textsc{Forward}
\Statex
\Function{\textsc{Backward}}{$\delta\tau$}
  \Comment{$\delta\tau = \partial\mathcal{L}/\partial\hat{\tau}$, e.g.\ Score (\cref{alg:loco})}
  \State \textcolor{water}{\textsc{Output}${}^T$:}\;
    $\nabla W_L \leftarrow \delta\tau \cdot z_{L-1}^T$;\;\;
    $\nabla b_L \leftarrow \delta\tau$
    \Comment{\textcolor{water}{水: Slide${}^T$}}
  \State $\delta z \leftarrow W_L^T\, \delta\tau$
    \Comment{\textcolor{water}{水: pullback}}
  \For{$\ell = L{-}1, \ldots, 1$}
    \State \textcolor{knife}{\textsc{Activate}${}^T$:}\;
      $\delta y \leftarrow \delta z \odot
      \mathbf{1}_{y_\ell > 0}$
      \Comment{\textcolor{knife}{刀: ReLU mask (same Cut)}}
    \State \textcolor{water}{\textsc{Matmul}${}^T$:}\;
      $\nabla W_\ell \leftarrow
      \delta y \cdot z_{\ell-1}^T$;\;\;
      $\nabla b_\ell \leftarrow \delta y$
      \Comment{\textcolor{water}{水: Slide${}^T$}}
    \State $\delta z \leftarrow W_\ell^T\, \delta y$
      \Comment{\textcolor{water}{水: pullback to layer $\ell{-}1$}}
  \EndFor
  \State \Return
    $\{\nabla W_\ell,\, \nabla b_\ell\}_{\ell=1}^{L}$
\EndFunction
\end{algorithmic}
\end{algorithm}

The contact gradient flow~\eqref{eq:contactflow} on the standing
structure of \cref{rem:standing} yields a complete training recipe.
\Cref{alg:loco} is self-contained: a roboticist with a MuJoCo
engine and a robot XML file can execute it directly, with no reward
shaping and no domain knowledge beyond the file itself.

\begin{algorithm}[H]
\caption{Contact-dynamics training for quadruped locomotion}
\label{alg:loco}
\begin{algorithmic}[1]
\Require Engine $\mathcal{E}$ (MuJoCo);\;
  model (\texttt{go2\_mjx.xml});\;
  sensors (IMU, encoders, camera)
\Require $\mathcal{D}$ (demonstrations for Work;\;
  $\varnothing$ for Stand/Walk)
\Require $\eta$ (learning rate),\;
  $\gamma$ (dissipation),\;
  $\beta$ (soft-min sharpness),\;
  $L$ (depth),\; $n$ (width),\;
  $H$ (horizon),\;
  $N$ (eval count)
\Statex
\State $G,\; c_{\max},\; \theta_0,\; h
  \leftarrow \texttt{parse}(\mathrm{XML})$
  \Comment{graph, limits, pose, height}
\State $\{W_\ell, b_\ell\}_{\ell=1}^{L}
  \leftarrow \texttt{init}(L,\, n)$
  \Comment{init $\pi$;\; $c(e_\ell) = \|W_\ell\|_F$}
\Statex
\For{\textcolor{sword}{\textbf{phase}} $\in$
  \{Stand\,$(U\!=\!\{0\})$,\;\,
   Walk\,$(U\!=\!\mathrm{Unif})$,\;\,
   Work\,$(U\!=\!\mathrm{task})$\}}
  \Comment{\textcolor{sword}{$\varphi$: curriculum}}
  \Repeat
    \State $u \sim U$;\;\;
      $\{\tau_t^*\} \leftarrow \mathcal{D}(u)$
      \Comment{command $+$ demo ($\varnothing$ if Stand/Walk)}
    \For{$t = 0, \ldots, H$}
      \Comment{\textcolor{water}{水: rollout}}
      \State $o_t \leftarrow (o_{\mathrm{prop}},\,
        \varphi_{\mathrm{vis}}(I_t))$
        \Comment{\cref{def:observation}: $\R^{42+d}$}
      \State $(\mu_t, \sigma_t) \leftarrow \textsc{Forward}(o_t,\, u)$
        \Comment{\cref{alg:forward}: $\R^{42+d} \to \R^{24}$}
      \State $\tau_t \sim \pi_\theta(\cdot \mid o_t, u)$
        \Comment{\cref{def:mollifier}: Student-$t$}
      \State $q_{t+1} \leftarrow \mathcal{E}.\texttt{step}(q_t,\, \tau_t)$
        \Comment{EL dynamics (\textcolor{knife}{no $\nabla$})}
    \EndFor
    \State \textcolor{water}{\textsc{Evaluate}:}\;
      $|\phi|_\beta \leftarrow
      \mathrm{soft\text{-}min}_\beta\bigl\{
      h(q_t)\!-\!h_{\min},\;\,
      \phi_{\max}\!-\!\|R_t\!-\!I\|_F,\;\,
      \min_j z_j(q_t)\!-\!z_{\min}
      \bigr\}_{t=0}^{H}$
      \Comment{\textcolor{water}{水: margin} (\cref{def:soft-viability})}
    \State $|\phi|_\beta \leftarrow
      \mathrm{soft\text{-}min}_\beta\bigl(
      |\phi|_\beta,\;\,
      \varepsilon\!-\!\|\tau_t\!-\!\tau_t^*\|
      \bigr)_{t}$
      \Comment{imitation (\cref{def:task-rkhs});
      skip if $\mathcal{D}\!=\!\varnothing$}
    \State \textcolor{water}{\textsc{Score}:}\;
      $s_t \leftarrow
      \nabla_\theta \log \pi_\theta(\tau_t \mid o_t, u)$
      \Comment{\textcolor{water}{水: through $\pi$ only}}
    \State \textcolor{water}{\textsc{Backward}:}\;
      $\{\nabla W,\, \nabla b\} \leftarrow
      |\phi|_\beta \cdot \sum_t s_t$
      \Comment{\cref{alg:backward}; $\nabla$ stops at $\mathcal{E}$}
    \State $W_\ell \leftarrow W_\ell
      + \eta\bigl[\textcolor{water}{\nabla W_\ell}
      - \textcolor{knife}{\gamma\, W_\ell}\bigr]$;\;\;
      $b_\ell \leftarrow b_\ell
      + \eta\,\textcolor{water}{\nabla b_\ell}$
      \Comment{Eq.~\eqref{eq:contactflow} on $\pi$}
    \State \textcolor{knife}{\textsc{Clamp}:}\;
      $\|W_\ell\|_F \leftarrow
      \min\bigl(\|W_\ell\|_F,\;
      c_{\max}(e_\ell)\bigr)$
      \Comment{\textcolor{knife}{刀: enforce capacity}}
    \If{$|\phi|_\beta \le 0$}
      \Comment{robot fell}
      \State \textcolor{sword}{\textsc{Reset}:}\;
        $\mathcal{E}.\texttt{reset}(\theta_0)$;\;\;
        $\{W, b\} \leftarrow \texttt{init}(L, n)$
        \Comment{\textcolor{sword}{$\varphi$: recover}}
    \EndIf
  \Until{$\max|\phi| = \min|C|$ for $N$ consecutive rollouts}
\EndFor
\Statex
\Ensure Trained weights $\{W_\ell, b_\ell\}$:
  $\pi = \textsc{Forward}(\cdot;\, W, b)
  \colon \R^{42+d} \to \R^{12}$
\end{algorithmic}
\end{algorithm}

\begin{remark}[Validation chain: MuJoCo $+$ \texttt{go2\_mjx.xml}
$\Rightarrow$ trained policy]
\label{rem:validation-chain}
Every input to \cref{alg:loco} is extracted from two artifacts:
a physics engine (MuJoCo) and a robot model file
(\texttt{go2\_mjx.xml}, MuJoCo Menagerie).
No additional assumptions.
\begin{center}
\begin{tabular}{@{}lll@{}}
\toprule
\textbf{Algorithm input} & \textbf{Source} & \textbf{Extraction} \\
\midrule
Graph $G$ (12 edges, 4 terminals)
  & XML & Joint topology (\texttt{<body>}/\texttt{<joint>}) \\
Capacity bounds $c_{\max}(e)$
  & XML & \texttt{<joint range>}, \texttt{<actuator ctrlrange>} \\
Standing pose $\theta_0$
  & XML & \texttt{<keyframe>} (default configuration) \\
Standing height $h$
  & XML & CoM of default keyframe ($0.312$\,m) \\
Proprioception $o_{\mathrm{prop}} \in \R^{42}$
  & Sensors & IMU ($R, \omega, v, p$) + encoders ($\theta, \dot\theta$) \\
Vision $o_{\mathrm{vis}} \in \R^{d}$
  & Camera & DINOv2 (frozen ViT, $d = 384$) \\
Command field $U$ (\cref{def:command})
  & User & Curriculum: $\delta_0$ / Uniform / task \\
EL dynamics $M, C, g$
  & Engine & \texttt{mj\_step}: $(q, \dot{q}, \tau) \mapsto \ddot{q}$ \\
Contact mode $c \in \{0,1\}^4$
  & Engine & \texttt{mj\_contact}: foot--ground detection \\
Gradient $\partial|\phi|/\partial c(e)$
  & Engine & MJX autodiff (JAX backend) \\
Reset to standing
  & Engine & \texttt{mj\_resetData} \\
\bottomrule
\end{tabular}
\end{center}
No reward shaping. No domain knowledge beyond what the XML file
contains. The viability axiom (\cref{ax:viability}) is the only
objective: $|\phi| > 0$ (do not fall). The knife
(\cref{def:dissipation}) is the only regulariser:
$\gamma(e) \cdot c(e)$ (do not exceed limits).

Given a MuJoCo engine and a \texttt{.xml} model file, every step of
\cref{alg:loco} is mechanically executable.
The standing structure (\cref{rem:standing}) is read from the file.
The contact dynamics (\cref{sec:contact}) is computed by the engine.
The algorithm is the bridge.
\end{remark}

\begin{remark}[The policy as parallel transport]\label{rem:gauge}
In \cref{alg:loco}, the policy
$\pi \colon \R^{42+d} \to \R^{12}$
maps sensor readings to joint torques (\cref{def:observation}).
Unfolding $\pi$ as a neural network (\cref{def:neural-exgraph}):
\[
  o \;\xrightarrow{\;\textcolor{water}{\sigma_1 \circ W_1}\;}
  h_1 \;\xrightarrow{\;\textcolor{water}{\sigma_2 \circ W_2}\;}
  \cdots \;\xrightarrow{\;\textcolor{water}{\sigma_L \circ W_L}\;}
  \tau.
\]
Each arrow is a \textsc{Slide} (\S\ref{sec:calculus}, four operations):
the $(42\!+\!d)$-dimensional observation flows through $L$ layers,
each applying $\textcolor{water}{\sigma \circ W}$.
The composite is \textsc{Compose}: the forward pass.

This is parallel transport on the execution graph.
The weights $\{W_\ell\}_{\ell=1}^L$ define a \emph{connection}:
a rule for transporting observations through the network.
Different weights $=$ different connection.
Training (the \textcolor{water}{水} steps of \cref{alg:loco})
searches for the connection under which the transported
observation produces viable torques: $|\phi| > 0$.

The trained network is a fixed connection on the standing structure.
The spectral gap $\lambda_1 > 0$ (\cref{thm:massgap}) is its
stability: small perturbations in observation decay, not amplify.
\end{remark}

\begin{remark}[Frozen and learned connections]\label{rem:frozen-gauge}
The observation $o = (o_{\mathrm{prop}},\, o_{\mathrm{vis}})$
(\cref{def:observation}) passes through \emph{two} connections in
series:
\[
  \underbrace{I_{\mathrm{cam}}
  \;\xrightarrow{\;\varphi_{\mathrm{vis}}\;}
  o_{\mathrm{vis}}}_
  {\text{frozen (DINOv2)}}
  \;\oplus\;
  o_{\mathrm{prop}}
  \;\xrightarrow{\;\pi\;}
  \tau.
\]
The visual encoder $\varphi_{\mathrm{vis}}$ is a \emph{frozen
connection}: a pretrained ViT whose weights are fixed during
\cref{alg:loco}. It transports raw pixels into a semantic
embedding space. The policy $\pi$ is a \emph{learned connection}:
its weights are updated by the \textcolor{water}{水} steps.

In gauge-theoretic language: $\varphi_{\mathrm{vis}}$ is a
background gauge field (fixed geometry of the visual fibre bundle),
while $\pi$ is the dynamical gauge field (trained by the contact
gradient flow~\eqref{eq:contactflow}).
The visual encoder sees the terrain; the policy decides what to do
about it. Two connections, one frozen, one learned.
\end{remark}

\begin{remark}[Deployment: real-time frequency alignment]
\label{rem:deployment}
\Cref{alg:loco} trains in simulated time
($\Delta t = 0.02$\,s, control rate $50$\,Hz).
Deployment on real hardware requires the same loop at the same rate
in \emph{real} time:
\[
  \underbrace{o_t \leftarrow \texttt{sense}}_
  {\text{read sensors}}
  \;\to\;
  \underbrace{\tau_t \leftarrow \pi(o_t)}_
  {\text{evaluate NN}}
  \;\to\;
  \underbrace{q_{t+1} \leftarrow \texttt{actuate}(\tau_t)}_
  {\text{command joints}}
  \;\leq\; 20\,\text{ms}.
\]
The entire sense--compute--act cycle must complete within one
$\Delta t$. If the policy $\pi$ (the connection of
\cref{rem:gauge}) takes longer than $20$\,ms to evaluate, the
real-time constraint is violated: the flow is no longer at the
trained frequency, and viability is not guaranteed.

This is a viability condition on the computation itself.
The standing structure (\cref{rem:standing}) specifies the physical
constraints; the frequency constraint specifies the computational
one. Both must hold for deployment:
\[
  \underbrace{|\phi| > 0}_{\text{physics: do not fall}}
  \quad\wedge\quad
  \underbrace{t_{\pi} \leq \Delta t}_
  {\text{compute: do not lag}}.
\]
Time must be real and linear.
No simulation speedup, no frame drops.
\end{remark}

\begin{remark}[Annealing $\beta$ as renormalisation group flow]
\label{rem:rg-annealing}
The soft-min parameter $\beta$ (\cref{def:soft-viability}) anneals
with the curriculum (\cref{def:command}):
\[
  \text{Stand} \;(\beta \text{ small})
  \;\longrightarrow\;
  \text{Walk} \;(\beta \text{ medium})
  \;\longrightarrow\;
  \text{Work} \;(\beta \text{ large}).
\]
This is a renormalisation group (RG) flow from the ultraviolet
(UV, smooth, all-timestep gradient) to the infrared (IR, sharp,
worst-timestep gradient).
At small $\beta$, the viability margin is a soft average: the
gradient is dense and the optimisation landscape is smooth---the
robot learns to stand by distributing information across the
entire trajectory.
At large $\beta$, the margin approaches the hard $\min$: the
gradient concentrates on the single worst timestep---the robot
learns precise footwork by focusing on the critical instant.

The \textsc{Phase} operator ($\textcolor{sword}{\varphi}$) of the
calculus drives this flow.
Each phase transition $U_k \to U_{k+1}$ increases both the command
support and the sharpness $\beta$.
The contact gradient flow~\eqref{eq:contactflow} operates at the
current $\beta$; the curriculum selects the scale.
UV $\to$ IR: smooth $\to$ sharp, Stand $\to$ Work,
all-timestep $\to$ worst-timestep.
\end{remark}

\begin{definition}[Task RKHS (学堂)]\label{def:task-rkhs}
Let $\mathcal{T}$ be a set of tasks, each observed via a camera
image $I_\tau$.
The \emph{task kernel} is
\begin{equation}\label{eq:task-kernel}
  k(\tau_1, \tau_2)
  \;=\;
  \bigl\langle
  \varphi_{\mathrm{vis}}(I_{\tau_1}),\;\,
  \varphi_{\mathrm{vis}}(I_{\tau_2})
  \bigr\rangle_{\R^d},
\end{equation}
the inner product of DINOv2 embeddings
(\cref{def:observation}, $d = 384$).
The \emph{task RKHS} $\mathcal{H}_k$ is the reproducing kernel
Hilbert space induced by $k$: the space of skill functions
$f \colon \mathcal{T} \to \R^{12}$ with reproducing property
$f(\tau) = \langle f,\, k(\cdot, \tau) \rangle_{\mathcal{H}_k}$.

A \emph{demonstration} for task $\tau$ is a trajectory
$\mathcal{D}_\tau = \{(o_t,\, \tau_t^*)\}_{t=0}^{H}$
of observation--torque pairs, collected from a teacher
(teleoperation, motion capture, or a reference policy).
The \emph{task margin} at timestep $t$ is
\begin{equation}\label{eq:task-margin}
  m_{\mathrm{task}}(t)
  \;=\;
  \varepsilon_{\mathrm{task}}
  \;-\;
  \bigl\|\pi(o_t, u) - \tau_t^*\bigr\|,
\end{equation}
where $\varepsilon_{\mathrm{task}} > 0$ is the imitation
tolerance.
The Work-phase viability margin
extends~\eqref{eq:soft-viability}:
\begin{equation}\label{eq:work-margin}
  |\phi|_\beta^{\mathrm{Work}}
  \;=\;
  \mathrm{soft\text{-}min}_\beta\bigl(
  \underbrace{h(q_t) - h_{\min}}_{\text{height}},\;\;
  \underbrace{\phi_{\max} - \|R_t - I\|_F}_
  {\text{orientation}},\;\;
  \underbrace{\min_j z_j(q_t) - z_{\min}}_
  {\text{sinking}},\;\;
  \underbrace{m_{\mathrm{task}}(t)}_
  {\text{imitation}}
  \bigr)_{t=0}^{H}.
\end{equation}
During Stand and Walk, $\mathcal{D}_\tau = \varnothing$ and the
task term is absent ($e^{-\beta \cdot \infty} = 0$ in the
soft-min).
During Work, the demonstration is the teacher, and the kernel
$k$ provides generalisation: a policy trained on demonstrated
tasks $\{\tau_i\}$ transfers to a new task $\tau^*$ in proportion
to $k(\tau^*, \tau_i)$.
\end{definition}

\begin{remark}[The frozen eye as task metric]
\label{rem:task-metric}
The DINOv2 encoder $\varphi_{\mathrm{vis}}$
(\cref{def:observation}) serves two roles:
\begin{enumerate}[label=(\roman*)]
  \item \textbf{Observation}: maps camera images to
  $o_{\mathrm{vis}} \in \R^d$ for the policy
  (\cref{alg:forward}).
  \item \textbf{Task metric}: the kernel
  $k$~\eqref{eq:task-kernel} defines the geometry of the task
  space $\mathcal{T}$ (\cref{def:task-rkhs}).
\end{enumerate}
One encoder, two roles.
The differentiation boundary (\cref{rem:frozen-gauge}) is also
the metric boundary: the fixed geometry on which all tasks are
measured.

Because DINOv2 is pretrained on real images (ImageNet), the
kernel $k$ is anchored in reality, not simulation.
Tasks that look similar in the real world are close in
$\mathcal{H}_k$.
This bridges the sim-to-real gap for the Work phase:
a policy trained on task $\tau_1$ in simulation transfers to
a visually similar task $\tau_2$ in reality because
$k(\tau_1, \tau_2)$ is large.
The school (学堂) generalises through the kernel.
Demonstrations are lessons; the RKHS norm is the grade;
deployment is graduation.
\end{remark}

\begin{remark}[Hyperparameter atlas]\label{rem:hyperparameters}
\Cref{alg:loco} takes seven scalar hyperparameters.
They split into two tiers by origin.

\medskip
\noindent
\textbf{Tier 1: physics-determined}
(read from the engine or the task; not free choices).

\smallskip
{\small
\begin{center}
\begin{tabular}{@{}l l p{0.62\textwidth}@{}}
\toprule
Symbol & Source & Interpretation \\
\midrule
$\gamma$ &
  \texttt{XML} damping &
  Dissipation rate (thermodynamic 2nd law of the engine).
  Read from \texttt{go2\_mjx.xml}. \\
$H$ &
  Task duration &
  Horizon: timesteps per rollout.
  Typical: $500$--$2000$ (at $\Delta t$ of XML). \\
$\eta$ &
  Optimiser &
  Learning rate: the Planck scale of the update---the smallest
  step that moves $W$ meaningfully.
  $\eta \ll \gamma^{-1}$ ensures
  the flow~\eqref{eq:contactflow} is contractive.
  Typical: $10^{-4}$--$10^{-3}$. \\
\bottomrule
\end{tabular}
\end{center}
}

\smallskip
$\gamma$ is not a free knob: it is the physical dissipation
already baked into the simulator's contact model.
$H$ is set by the task (how long one episode lasts).
$\eta$ is fundamental: it converts the gradient $\nabla W$ into a
finite displacement, analogous to the Planck time converting
energy into frequency ($E = \hbar\omega$).
Too large $\Rightarrow$ instability; too small $\Rightarrow$
the agent never moves.

\medskip
\noindent
\textbf{Tier 2: user settings}
(design choices; tune on validation rollouts).

\smallskip
{\small
\begin{center}
\begin{tabular}{@{}l l p{0.62\textwidth}@{}}
\toprule
Symbol & Role & Interpretation \\
\midrule
$L$ &
  Depth &
  Number of layers in $\pi$ (\cref{alg:forward}).
  Typical: $3$--$5$. \\
$n$ &
  Width &
  Hidden dimension per layer.
  Typical: $128$--$512$. \\
$N$ &
  Eval count &
  Rollouts per gradient step (sample size for
  $|\phi|_\beta$).
  Typical: $64$--$4096$. \\
$\beta$ &
  Sharpness &
  RG scale (\cref{rem:rg-annealing}): small $\beta$ $=$ UV
  (smooth, exploratory); large $\beta$ $=$ IR (sharp,
  exploitative).
  Typical: $1$--$100$. \\
$\varepsilon_{\mathrm{task}}$ &
  Imitation &
  Task margin (\cref{def:task-rkhs});
  $\varnothing$ during Stand/Walk.
  Task-dependent. \\
\bottomrule
\end{tabular}
\end{center}
}

\smallskip
The tier boundary is sharp: Tier~1 parameters are
\emph{measured} (from the XML, the task, or the optimiser's
stability condition); Tier~2 parameters are
\emph{chosen} (by the user, validated by rollout performance).
A practitioner who changes the robot changes Tier~1;
a practitioner who changes the architecture changes Tier~2.
Neither tier crosses into the other.
\end{remark}

\begin{remark}[\textcolor{caution}{No tricks needed}]
\label{rem:no-tricks}
\Cref{alg:loco} is a policy gradient through a mollifier.
That is all.
We collect no replay buffer, fit no value function, clip no
surrogate objective, aggregate no datasets.

\medskip\noindent
\textbf{The engine is differentiable.}\;
MJX (the JAX backend of MuJoCo) provides
$\nabla_\tau \mathcal{E}.\texttt{step}$---the Jacobian of the
physics step with respect to torques.
This gradient is \emph{exact within a single contact mode}.
But locomotion is mode switching: every gait cycle crosses
$\geq 4$ contact boundaries (one per foot), and at each boundary
the contact Jacobian $J_c$ changes rank.
The engine gradient at a boundary is the left or right limit,
\emph{not} the gradient of the expected value.

\medskip\noindent
\textbf{We do not use the engine gradient.}\;
\Cref{alg:loco} treats $\mathcal{E}.\texttt{step}$ as a
\textcolor{knife}{black box}: the gradient does not penetrate
the physics.
The \textsc{Score} step (line~14) differentiates through the
policy $\pi_\theta$ only---never through $\mathcal{E}$.
The engine could be MuJoCo~C (non-differentiable),
MJX (differentiable), Brax, or real hardware.
The algorithm does not care.

\medskip\noindent
\textbf{The mollifier provides the gradient.}\;
The Student-$t$ policy (\cref{def:mollifier}) has full support
on $[-\bar\tau, \bar\tau]^{12}$ and is $C^\infty$ in $\theta$.
The expected viability
$\bar\phi(\theta) =
\mathbb{E}_{\pi_\theta}[|\phi|_\beta]$
is smooth in $\theta$---even though $|\phi|_\beta(\tau)$ is
non-smooth at contact boundaries---because integrating a
Lipschitz function against a smooth kernel with full support
produces a smooth function.
The Student-$t$ absorbs the contact discontinuity.

\medskip\noindent
\textbf{The remaining patches are still unnecessary:}
\begin{itemize}[leftmargin=*]
\item \textcolor{caution}{\textbf{DAgger}}: no distribution shift.
  Every rollout runs the current $\pi_\theta$ through the engine.
\item \textcolor{caution}{\textbf{PPO}}: no surrogate, no baseline,
  no clipping, no entropy bonus.
  The score function gives an unbiased gradient.
  The capacity clamp (\textsc{Clamp}) bounds $\|W_\ell\|_F$.
  Command sampling $u \sim U$ provides exploration.
\item \textcolor{caution}{\textbf{Value function}}: $|\phi|_\beta$
  is computed per-rollout, not estimated by a learned $V(s)$.
\end{itemize}

The pattern: every ``trick'' in model-free RL is a patch for a
missing gradient.
The Student-$t$ mollifier provides the gradient---through the
policy, not through the engine.
The patches become unnecessary.
\Cref{alg:loco} is not a new algorithm---it is what remains when
you \textcolor{caution}{delete the patches}.
\end{remark}

\section{Representation dynamics}\label{sec:representation}

The contact dynamics of \cref{sec:contact} trains a locomotion policy
on an execution graph with $12$ edges.
We now construct the dual of the agentic tower and show that the same
contact gradient flow, on different execution graphs, yields
text generation, image generation, and video generation as
instances of a single representation-learning paradigm.

\begin{definition}[Dual tower]\label{def:dual-tower}
The \emph{dual tower} of the agentic space
$\mathbf{L} = (L_0, L_1, L_2, L_3)$ (\cref{def:tower}) is
\[
  \mathbf{L}^*
  \;=\;
  \text{力}^* \;\oplus\; \text{立}^* \;\oplus\; \text{丽}^*,
\]
where:
\begin{enumerate}[label=\textbf{L\arabic*${}^*$}.]
  \item $\text{力}^* = L_0^*$:
  the dual of the state space.
  An element of $L_0^*$ is \emph{acted upon} physically:
  the medium as substrate (pixels, tokens, audio samples).
  \item $\text{立}^* = L_1^*$:
  the dual of the viable kernel.
  An element of $L_1^*$ is \emph{positioned} by external operations:
  structural coherence imposed (grammar, spatial consistency,
  temporal continuity).
  \item $\text{丽}^* = (L_2 \oplus L_3)^*$:
  the dual of the control-strategy bundle.
  An element of $(L_2 \oplus L_3)^*$ is \emph{represented} by
  the Subject's operations:
  meaning, style, and aesthetics as projected image.
\end{enumerate}
The three components are named by their Chinese homophones:
力~(force), 立~(stand), 丽~(beauty)---all pronounced~\emph{l\`\i}.
\end{definition}

The duality is grammatical: 力~acts, 力${}^*$~is acted upon.
立~stands, 立${}^*$~is positioned.
丽~creates beauty, 丽${}^*$~is made beautiful.
The operator~被 (passive voice marker) maps each component to its dual.

\begin{theorem}[Beauvoir representation]\label{thm:beauvoir}
The dual tower $\mathbf{L}^*$ is a faithful, irreducible
representation of the passive predicates on the agentic space.
Every predicate of the form ``$x$ is $f$-ed by~$b$'' for
$f \in \{\sigma, \partial, \circ, \varphi\}$ factors through
$\mathbf{L}^*$:
\begin{align*}
  \text{被}\,\sigma &\;\in\; \text{力}^*
    &&\text{(transported physically: 被恢复)}, \\
  \text{被}\,\partial &\;\in\; \text{立}^*
    &&\text{(cut / detected: 被看到)}, \\
  \text{被}\,\circ &\;\in\; \text{丽}^*
    &&\text{(composed into representation: 被表示)}, \\
  \text{被}\,\varphi &\;\in\; \text{丽}^*
    &&\text{(redefined by phase change: 被描述)}.
\end{align*}
\end{theorem}

\begin{proof}
The calculus is complete (\cref{prop:completeness}): every operation
on the agentic space decomposes into
$\{\sigma, \circ, \partial, \varphi\}$.
The dual of each operation is its passive form.
\textsc{Slide}~$\sigma$ is physical transport ($L_0$); its dual
被$\,\sigma$ acts on~$L_0^*$.
\textsc{Cut}~$\partial$ detects and removes ($L_1$: the boundary
of the viable kernel); its dual 被$\,\partial$ acts on~$L_1^*$.
\textsc{Compose}~$\circ$ and \textsc{Phase}~$\varphi$ build
representations and change regimes ($L_2 \oplus L_3$: control
and strategy); their duals act on~$(L_2 \oplus L_3)^*$.
By completeness, no predicate falls outside
$L_0^* \oplus L_1^* \oplus (L_2 \oplus L_3)^*$.
The decomposition is irreducible: removing any component removes
a calculus operation from the dual.
\end{proof}

\begin{theorem}[Camus absurdity]\label{thm:camus}
In any agentic space with knife dissipation $\gamma > 0$
(\cref{def:dissipation}), the absorbed energy at equilibrium
\[
  \mathcal{A}
  \;=\;
  \|z\| - |\phi|_{\mathrm{eq}}
  \;>\; 0.
\]
The flow deficit is strictly positive: the system always dissipates.
\end{theorem}

\begin{proof}
The contact gradient flow (\cref{def:contactflow}) reaches
equilibrium at $\partial|\phi|/\partial c(e) = \gamma(e)\,c(e)$
(\cref{thm:contactEL}).
The dissipation term $\gamma(e)\,c(e) > 0$ on bypass edges
prevents the capacities from reaching the flow-maximising assignment.
Therefore $|\phi|_{\mathrm{eq}} < \max|\phi| \leq \|z\|$ and
$\mathcal{A} > 0$.

The absorption is irreducible: setting $\gamma \to 0$ eliminates
dissipation but also eliminates the stability margin
$\lambda_1 > 0$ (\cref{rem:contact-stability}).
The system becomes unstable---small perturbations amplify rather
than decay.
The absurd is the price of stability.
\end{proof}

\begin{corollary}[Representation learning]\label{cor:replearn}
The contact gradient flow~\eqref{eq:contactflow} on the dual
tower $\mathbf{L}^*$ with objective $\mathcal{L} = -|\phi|$ is
representation learning:
\[
  \min_c \; \mathcal{A}(c)
  \quad\text{subject to}\quad
  c \in \mathbf{L}^*
  \;=\;
  \text{力}^* \oplus \text{立}^* \oplus \text{丽}^*.
\]
\cref{thm:beauvoir} identifies the representation space
($\mathbf{L}^*$).
\cref{thm:camus} identifies the objective ($\min \mathcal{A}$,
irreducible).
The contact gradient flow provides the dynamics.
\end{corollary}

\begin{definition}[Representation dynamics]\label{def:repdyn}
Let $G = (V, E, c)$ be an execution graph (\cref{def:exgraph}) on the
dual tower $\mathbf{L}^*$ (\cref{def:dual-tower}).
The \emph{representation dynamics} on $G$ is the contact gradient
flow~\eqref{eq:contactflow}:
\[
  \frac{dc(e)}{dt}
  \;=\;
  \frac{\partial|\phi|}{\partial c(e)}
  \;-\;
  \gamma(e)\,c(e),
  \qquad
  c(e) = \|W_e\|_F,
\]
where $|\phi|$ is the viability of the generated output
(\cref{cor:replearn}: $\mathcal{L} = -|\phi|$) and $\gamma$ is the
knife dissipation (\cref{def:dissipation}).
\end{definition}

\medskip
\noindent\fcolorbox{water}{water!5}{%
\begin{minipage}{\dimexpr\textwidth-2\fboxsep-2\fboxrule}
\smallskip
\textbf{\textcolor{water}{Why representation is contact dynamics.}}\;
The contact gradient flow~\eqref{eq:contactflow} is defined on any
execution graph with capacitated edges (\cref{def:exgraph}).
It does not distinguish whether the edges carry physical actuators
($c \in \mathbf{L}$) or neural weights
($c \in \mathbf{L}^*$).
Contact dynamics (\cref{sec:contact}) trains an agent to \emph{act}
(力\,立\,丽).
Representation dynamics trains a model to \emph{represent}
(力${}^*$\,立${}^*$\,丽${}^*$).
The operator~被 maps one to the other:
the same equation, the same flow, the same knife---on the dual tower.
\smallskip
\end{minipage}}
\medskip

\begin{remark}[Le Deuxi\`eme Sexe as tower]
\label{rem:beauvoir-tower}
Beauvoir's \emph{Le Deuxi\`eme Sexe}~\cite{beauvoir} is
structured as three parts that map to the dual tower:
\begin{center}
\begin{tabular}{@{}llll@{}}
\toprule
\textbf{Volume~I Part} & \textbf{l\`\i} & \textbf{Tower} &
\textbf{Content} \\
\midrule
I.\ Destin (Destiny) & 力 & $L_0^*$ &
  Biology, body, physical substrate \\
II.\ Histoire (History) & 立 & $L_1^*$ &
  Social position, establishment \\
III.\ Mythes (Myths) & 丽 & $(L_2 \oplus L_3)^*$ &
  Cultural representation, image \\
\bottomrule
\end{tabular}
\end{center}
The historical examples of \cref{app:secondsex} (蔡文姬, 花木兰)
illustrate the tower: at every stage, the Other's identity
decomposes as 被力~$\oplus$~被立~$\oplus$~被丽---physically
acted upon, socially positioned, culturally represented.
The knife between Subject and Other is the mean
(\cref{thm:meanfield}): a phase function, not an intrinsic
property.
\end{remark}

\begin{definition}[Generation execution graph]\label{def:gen-exgraph}
A \emph{generation execution graph} is an execution graph
$G_{\mathrm{gen}} = (V, E, c)$ (\cref{def:exgraph}) where:
\begin{itemize}
  \item $\kappa$ is the conditioning input
  (prompt, noise schedule, or previous frames);
  \item $\infty$ is the generated output
  (tokens, pixels, or video frames);
  \item the intermediate nodes $\{a_i\}$ are the network layers;
  \item the capacity $c(e)$ is $\|W_e\|_F$
  (\cref{def:neural-exgraph}).
\end{itemize}
The viable flow condition $|\phi| > 0$ becomes: the
generated output is coherent.
\end{definition}

\begin{definition}[Representation model]\label{def:repmodel}
A \emph{representation model} is a family of maps
$\{f_m\}_{m \in \mathcal{M}}$ on modalities
$\mathcal{M} = \{\text{text},\, \text{image},\, \text{video}\}$,
sharing a backbone $g_\theta$ on the dual tower:
\[
  f_m \;=\; h_m \circ g_\theta \circ \tau_m,
  \qquad m \in \mathcal{M},
\]
where $\tau_m$ is a modality-specific tokeniser
(embed tokens / patchify pixels / patchify-and-stack frames)
and $h_m$ is a modality-specific head
(predict next token / denoise patches / denoise-and-predict frames).
The backbone $g_\theta$ operates on
$\mathbf{L}^* = \text{力}^* \oplus \text{立}^* \oplus \text{丽}^*$
(\cref{def:dual-tower}) and is shared across all modalities.
\end{definition}

\begin{algorithm}[H]
\caption{Contact-dynamics training for a representation model}
\label{alg:repmodel}
\begin{algorithmic}[1]
\Require Modality set
  $\mathcal{M} = \{\text{text},\, \text{image},\, \text{video}\}$
\Require For each $m \in \mathcal{M}$:\;
  graph $G_m$ (\cref{def:gen-exgraph}),\;
  dataset $\mathcal{D}_m$,\;
  tokeniser $\tau_m$
\Require $\eta$ (learning rate),\;
  $\gamma$ (dissipation),\;
  $\beta$ (soft-min sharpness),\;
  $L_s$ (backbone depth),\; $n$ (width),\;
  $N$ (eval count)
\Statex
\State \textbf{Backbone:}\;
  $\{W_\ell, b_\ell\}_{\ell=1}^{L_s}
  \leftarrow \texttt{init}(L_s,\, n)$
  \Comment{shared;\; $c(e_\ell) = \|W_\ell\|_F$}
\For{$m \in \mathcal{M}$}
  \Comment{modality-specific heads}
  \State $\{W_\ell^{m}, b_\ell^{m}\}_{\ell=1}^{L_m}
    \leftarrow \texttt{init}(L_m,\, n_m)$
\EndFor
\Statex
\For{\textcolor{sword}{\textbf{phase}} $\in$
  \{力\,$(U\!=\!\text{structure})$,\;\,
   立\,$(U\!=\!\text{coherence})$,\;\,
   丽\,$(U\!=\!\text{semantics})$\}}
  \Comment{\textcolor{sword}{$\varphi$: curriculum}}
  \Repeat
    \State $m \sim \mathrm{Unif}(\mathcal{M})$;\;\;
      $(z,\, x^*) \sim \mathcal{D}_m$
      \Comment{sample modality $+$ data}
    \State $\tilde{z} \leftarrow \tau_m(z)$
      \Comment{tokenise into $\mathbf{L}^*$}
    \State \textcolor{water}{\textsc{Forward}:}\;
      $\hat{x} \leftarrow h_m\bigl(g_\theta(\tilde{z})\bigr)$
      \Comment{\textcolor{water}{水: backbone $+$ head $m$}}
    \State \textcolor{water}{\textsc{Flow}:}\;
      $|\phi|_\beta^{(m)} \leftarrow
      \mathrm{soft\text{-}min}_\beta\bigl\{
      |\phi|_t^{(m)}
      \bigr\}_{t}$
      \Comment{\textcolor{water}{水: modality viability}}
    \State \textcolor{water}{\textsc{Backward}:}\;
      $\{\nabla W,\, \nabla b,\,
       \nabla W^{m},\, \nabla b^{m}\}
      \leftarrow \textsc{Backward}(\nabla_{\hat{x}}\, |\phi|_\beta^{(m)})$
      \Comment{\cref{alg:backward}}
    \State $W_\ell \leftarrow W_\ell
      + \eta\bigl[\textcolor{water}{\nabla W_\ell}
      - \textcolor{knife}{\gamma\, W_\ell}\bigr]$;\;\;
      $b_\ell \leftarrow b_\ell
      + \eta\,\textcolor{water}{\nabla b_\ell}$
      \Comment{backbone: Eq.~\eqref{eq:contactflow}}
    \State $W_\ell^{m} \leftarrow W_\ell^{m}
      + \eta\bigl[\textcolor{water}{\nabla W_\ell^{m}}
      - \textcolor{knife}{\gamma\, W_\ell^{m}}\bigr]$;\;\;
      $b_\ell^{m} \leftarrow b_\ell^{m}
      + \eta\,\textcolor{water}{\nabla b_\ell^{m}}$
      \Comment{head $m$: Eq.~\eqref{eq:contactflow}}
    \State \textcolor{knife}{\textsc{Clamp}:}\;
      $\|W_\ell\|_F \leftarrow
      \min\bigl(\|W_\ell\|_F,\;
      c_{\max}(e_\ell)\bigr)$
      for all $\ell$
      \Comment{\textcolor{knife}{刀: capacity}}
    \If{$|\phi|_\beta^{(m)} \le 0$}
      \Comment{output incoherent in modality $m$}
      \State \textcolor{sword}{\textsc{Reset}:}\;
        head only: $\{W^m, b^m\} \leftarrow \texttt{init}(L_m, n_m)$
        \Comment{\textcolor{sword}{$\varphi$: recover modality}}
    \EndIf
  \Until{$\forall\, m\!:\; \max|\phi|^{(m)} = \min|C^{(m)}|$
    for $N$ consecutive evaluations}
\EndFor
\Statex
\Ensure Representation model:
  $f_m = h_m \circ g_\theta \circ \tau_m$
  for all $m \in \mathcal{M}$,
  backbone on $\mathbf{L}^*$
\end{algorithmic}
\end{algorithm}

\begin{remark}[Three bodies, one backbone]
\label{rem:repmodel-bodies}
The modality-specific components of \cref{alg:repmodel} are:
\begin{center}
\small
\renewcommand{\arraystretch}{1.25}
\begin{tabular}{@{}l lll@{}}
\toprule
& \textbf{Text} & \textbf{Image} & \textbf{Video} \\
\midrule
$G_m$ (topology)
  & sequential & 2D patches & 2D patches $\times\, T$ \\
$\kappa_m$ (input)
  & context tokens & noise $+$ prompt
  & noise $+$ prompt $+$ prev.\ frames \\
$\tau_m$ (tokeniser)
  & BPE embedding & patchify $+$ linear
  & patchify $+$ linear $+$ temporal \\
$h_m$ (head)
  & softmax over $V$ & depatchify to $\R^{H \times W \times 3}$
  & depatchify $\times\, T$ \\
$|\phi|^{(m)}$ (viability)
  & $\log p(x_t^* \mid x_{<t})$
  & $\varepsilon - \|\hat{x} - x^*\|$
  & $\min(\text{spatial},\; \lambda_T \!\cdot\! \text{temporal})$ \\
$\gamma^{(m)}$ (knife)
  & weight decay & guidance scale
  & temporal regularisation \\
\textsc{Phase}
  & char $\to$ word $\to$ doc
  & noise $\to$ coarse $\to$ fine
  & frame $\to$ clip $\to$ video \\
\bottomrule
\end{tabular}
\end{center}
The backbone $g_\theta$ sees none of these distinctions.
It operates on $\mathbf{L}^*$: medium tokens (力${}^*$),
structural coherence (立${}^*$), semantic content (丽${}^*$).
The modality is invisible at the representation level---this is
why a single backbone suffices.
\end{remark}

\begin{remark}[Representation model $=$ locomotion on every body]
\label{rem:repmodel-loco}
Compare \cref{alg:repmodel} with \cref{alg:loco}:
\begin{center}
\renewcommand{\arraystretch}{1.25}
\begin{tabular}{@{}lll@{}}
\toprule
& \textbf{Locomotion (\cref{alg:loco})} &
\textbf{Representation model (\cref{alg:repmodel})} \\
\midrule
Body & quadruped ($12$ joints) &
  text / image / video \\
Standing & $h > h_{\min}$ (do not fall) &
  $|\phi|^{(m)} > 0$ (stay coherent) \\
Engine & MuJoCo (physics) &
  autodiff (shared backbone $g_\theta$) \\
Knife & joint limits ($c_{\max}$) &
  capacity bounds ($c_{\max}$) \\
Curriculum & Stand $\to$ Walk $\to$ Work &
  力 $\to$ 立 $\to$ 丽 \\
\bottomrule
\end{tabular}
\end{center}
A robot that falls is a text that is incoherent is an image that is
noise is a video that flickers.
The contact gradient flow is the same.
The body is different.
The representation model learns all bodies at once---三位一体:
one backbone on $\mathbf{L}^*$, three modalities, one gradient flow.
\end{remark}

\begin{remark}[LoRA $=$ low-rank task head]
\label{rem:lora}
Low-Rank Adaptation (LoRA) is a task head in the sense of
\cref{def:repmodel}.
Given a pre-trained backbone with layer weights
$\{W_\ell\}_{\ell=1}^L$, a LoRA adapter of rank~$r$ adds
\[
  \Delta W_\ell \;=\; B_\ell\, A_\ell,
  \qquad B_\ell \in \R^{d \times r},\;\;
  A_\ell \in \R^{r \times d},\;\;
  r \ll d,
\]
at each layer.
The backbone $W_\ell$ is frozen; the adapter $\Delta W_\ell$ is trained.
In the dual tower:
$W_\ell$ is the representation ($\mathbf{L}^*$),
$\Delta W_\ell$ is the task function
($\mathcal{H}_{\mathcal{T}}$, \cref{def:task-rkhs}),
and the rank~$r = c_{\max}(e_\ell^{\mathrm{task}})$ is the knife---it
bounds how far the task can deviate from the representation.
\end{remark}

\begin{definition}[Code metric space]
\label{def:code-metric}
The \emph{code metric space} is the agentic tower
(\cref{def:tower}) unbundled to four levels, each equipped with
a viability metric.
The general dual tower compresses
$L_2^* \oplus L_3^* \to \text{丽}^*$;
the code metric space resolves them into correctness ($L_2^*$)
and performance ($L_3^*$), yielding four phases---力, 立, 丽(正), 丽(快).
Correctness and performance are both sub-types of
丽~(value)---all three characters pronounced \emph{l\`\i}:
\begin{itemize}
  \item $\phi_{\mathrm{topo}}$ (\textbf{力${}^*$}, data topology,
  指针序):\;
  source forms a valid abstract syntax tree, all imports resolve,
  all names are bound.
  The data-structure graph (pointer\,/\,reference topology) is
  well-formed.
  Binary: $\phi_{\mathrm{topo}} \in \{0,\, 1\}$.

  \item $\phi_{\mathrm{safe}}$ (\textbf{立${}^*$}, memory safety,
  内存安全):\;
  program executes without crashing---no segmentation fault,
  no uncaught exception, no resource leak, no use-after-free.
  Binary: $\phi_{\mathrm{safe}} \in \{0,\, 1\}$.

  \item $\phi_{\mathrm{correct}}$ (\textbf{丽(正)${}^*$}, correctness,
  正确):\;
  output matches specification---all assertions hold, all tests pass.
  Ratio: $\phi_{\mathrm{correct}} =
  \text{tests passed}\,/\,\text{tests total} \in [0,\, 1]$.

  \item $\phi_{\mathrm{perf}}$ (\textbf{丽(快)${}^*$}, performance,
  快):\;
  program completes within time and space budgets.
  Continuous: $\phi_{\mathrm{perf}} =
  \max\!\bigl(0,\; 1 - t_{\mathrm{run}} / t_{\mathrm{budget}}\bigr)
  \in [0,\, 1]$.
\end{itemize}
The ordering is a prerequisite chain (each gates the next):
\[
  \phi_{\mathrm{topo}} \;\to\;
  \phi_{\mathrm{safe}} \;\to\;
  \phi_{\mathrm{correct}} \;\to\;
  \phi_{\mathrm{perf}}.
\]
You cannot profile code that crashes;
you cannot test code that does not parse.
The combined viability is
\[
  |\phi|_{\mathrm{code}} \;=\;
  \mathrm{soft\text{-}min}_\beta\bigl\{
  \phi_{\mathrm{topo}},\;
  \phi_{\mathrm{safe}},\;
  \phi_{\mathrm{correct}},\;
  \phi_{\mathrm{perf}}
  \bigr\}.
\]
Every programming language has all four levels.
Languages differ in which levels are enforced automatically:
\begin{center}
\renewcommand{\arraystretch}{1.25}
\resizebox{\linewidth}{!}{%
\begin{tabular}{@{}l llll@{}}
\toprule
\textbf{Language}
  & \textbf{力 (topo)} & \textbf{立 (safe)}
  & \textbf{丽(正) (correct)} & \textbf{丽(快) (perf)} \\
\midrule
Python
  & \texttt{ast.parse()}
  & \texttt{python f.py} exits $0$
  & \texttt{pytest} passes
  & \texttt{timeit} $\leq$ budget \\
C
  & \texttt{gcc -fsyntax-only}
  & Valgrind: no errors
  & tests pass
  & \texttt{perf stat} $\leq$ budget \\
Rust
  & \texttt{cargo check}
  & borrow checker (compile-time!)
  & \texttt{cargo test}
  & \texttt{cargo bench} $\leq$ budget \\
CUDA
  & \texttt{nvcc -c}
  & \texttt{compute-sanitizer}: 0 errors
  & trajectory converges
  & \texttt{nsight} $\leq$ budget \\
\bottomrule
\end{tabular}}%
\end{center}
\end{definition}

Fine-tuning is representation dynamics at school.
The representation model (\cref{def:repmodel}) learns $\mathbf{L}^*$
from mixed-modality data (\cref{alg:repmodel}).
Fine-tuning attaches a LoRA adapter (\cref{rem:lora}) to the frozen
backbone and trains it on task-specific data in the 学堂
(\cref{def:task-rkhs}).
For code generation, the code metric space (\cref{def:code-metric})
provides four viability metrics trained in prerequisite order.
\Cref{alg:finetune} is self-contained: an ML engineer with a
pre-trained checkpoint, a GPU server, a sandbox, and the four
viability metrics can execute it directly.

\begin{algorithm}[H]
\caption{Fine-tuning a representation model (学堂)}
\label{alg:finetune}
\begin{algorithmic}[1]
\Require Checkpoint $\theta_0$
  (e.g.\ \texttt{deepseek-coder-v2});\;
  tokenizer $\tau$;\;
  GPU server $\mathcal{E}_{\mathrm{GPU}}$
\Require Sandbox $\mathcal{E}_{\mathrm{sandbox}}$
  (e.g.\ Docker, \texttt{venv}, \texttt{nsjail})
\Require Task $\mathcal{T}$:\;
  dataset $\mathcal{D} = \{(z_i, x_i^*)\}$ (Q-A mode)
  or judge $J$ (court mode)
\Require Viability metrics
  $\phi_1 \to \cdots \to \phi_K$
  in prerequisite order
  (\cref{def:code-metric}: $K\!=\!4$)
\Require $r$ (LoRA rank $=$ knife),\;
  $\alpha$ (LoRA scaling),\;
  $\eta$ (learning rate),\;
  $\gamma$ (weight decay),\;
  $\beta$ (soft-min sharpness),\;
  $N$ (eval count)
\Statex
\State $\{W_\ell\}_{\ell=1}^{L}
  \leftarrow \texttt{load}(\theta_0)$;\;\;
  \texttt{freeze}($W$)
  \Comment{backbone on $\mathbf{L}^*$: frozen}
\State $\{B_\ell \!=\! 0,\;
  A_\ell \!\sim\! \mathcal{N}(0, \sigma^2)
  \}_{\ell=1}^{L}$
  \Comment{LoRA init: $\Delta W_\ell = B_\ell A_\ell = 0$}
\Statex
\For{\textcolor{sword}{\textbf{phase}} $\in$
  \{力\,$(\phi_{\mathrm{topo}})$,\;\,
   立\,$(\phi_{\mathrm{safe}})$,\;\,
   丽(正)\,$(\phi_{\mathrm{correct}})$,\;\,
   丽(快)\,$(\phi_{\mathrm{perf}})$\}}
  \Comment{\textcolor{sword}{$\varphi$: curriculum (\cref{def:code-metric})}}
  \Repeat
    \State $(z,\, x^*) \sim \mathcal{D}$\; or\; $z \sim \mathcal{T}$
      \Comment{Q-A: input $+$ target;\; court: input only}
    \State \textcolor{water}{\textsc{Forward}:}\;
      $\hat{x} \leftarrow \textsc{Forward}\bigl(\tau(z);\;
      \{W_\ell + \tfrac{\alpha}{r}\,
      \textcolor{water}{B_\ell A_\ell}\}\bigr)$
      \Comment{\cref{alg:forward}; backbone $+$ adapter}
    \State \textcolor{water}{\textsc{Execute}:}\;
      $\mathrm{result} \leftarrow
      \mathcal{E}_{\mathrm{sandbox}}.\texttt{run}(\hat{x})$
      \Comment{\textcolor{water}{水:} run generated code in sandbox}
    \State \textcolor{water}{\textsc{Evaluate}:}\;
      $|\phi|_\beta \leftarrow
      \mathrm{soft\text{-}min}_\beta\bigl\{
      \phi_k(\mathrm{result}_t)
      \bigr\}_{t}$
      \Comment{\textcolor{water}{水:} phase-$k$ metric
      (\cref{def:code-metric})}
    \State \textcolor{water}{\textsc{Backward}:}\;
      $\{\nabla B_\ell,\, \nabla A_\ell\}_{\ell=1}^{L}
      \leftarrow \textsc{Backward}(\nabla_{\hat{x}}\,
      |\phi|_\beta)$
      \Comment{\cref{alg:backward};\; $W$ frozen}
    \State $B_\ell \leftarrow B_\ell
      + \eta\,\textcolor{water}{\nabla B_\ell}$;\;\;
      $A_\ell \leftarrow A_\ell
      + \eta\bigl[\textcolor{water}{\nabla A_\ell}
      - \textcolor{knife}{\gamma\, A_\ell}\bigr]$
      \Comment{Eq.~\eqref{eq:contactflow} on adapter}
    \State \textcolor{knife}{\textsc{Clamp}:}\;
      $\|B_\ell A_\ell\|_F \leftarrow
      \min\bigl(\|B_\ell A_\ell\|_F,\;
      c_{\max}\bigr)$
      \Comment{\textcolor{knife}{刀: adapter capacity}}
    \If{$|\phi|_\beta \le 0$}
      \Comment{output incoherent}
      \State \textcolor{sword}{\textsc{Reset}:}\;
        $B_\ell \leftarrow 0$;\;
        $A_\ell \sim \mathcal{N}(0, \sigma^2)$
        \Comment{\textcolor{sword}{$\varphi$: re-init adapter}}
    \EndIf
  \Until{$\max|\phi| = \min|C|$ for $N$ consecutive evaluations}
\EndFor
\Statex
\Ensure Adapted model:\;
  $W_\ell^{\mathrm{out}} = W_\ell
  + \tfrac{\alpha}{r}\,B_\ell A_\ell$;\;\;
  \texttt{merge\_and\_upload}
\end{algorithmic}
\end{algorithm}

\begin{figure}[H]
\centering
\begin{tikzpicture}[
  >=stealth,
  box/.style={draw, rounded corners=2pt, minimum width=5.2cm,
              minimum height=0.55cm, align=center, font=\small},
  io/.style={box, fill=black!5},
  wt/.style={box, fill=water!10},
  arr/.style={->, thick, black!60},
]
\node[io] (init) at (0,0)
  {Checkpoint $\theta_0$;\;\;\texttt{freeze}($W$);\;\;init LoRA};
\node[draw, fill=sword!12, rounded corners=2pt,
  font=\small\bfseries, minimum width=5.2cm,
  minimum height=0.5cm, align=center] (phase) at (0,-1.0)
  {\textcolor{sword}{$\varphi$:}\;
   力 $\to$ 立 $\to$ 丽(正) $\to$ 丽(快)};
\node[wt] (fwd)  at (0,-2.1)
  {\textcolor{water}{\textsc{Forward}}:\;
   $\hat{x} = f\!\bigl(\tau(z);\;W\!+\!\tfrac{\alpha}{r}BA\bigr)$};
\node[wt] (exec) at (0,-2.9)
  {\textcolor{water}{\textsc{Execute}}:\;
   sandbox.\texttt{run}($\hat{x}$)};
\node[wt] (eval) at (0,-3.7)
  {\textcolor{water}{\textsc{Evaluate}}:\;
   $|\phi|_\beta = \mathrm{soft\text{-}min}\{\phi_k\}$};
\node[wt] (bwd)  at (0,-4.5)
  {\textcolor{water}{\textsc{Backward}}:\;
   $\nabla B,\,\nabla A$\;($W$ frozen)};
\node[wt] (upd)  at (0,-5.3)
  {\textsc{Update}:\;Eq.~\eqref{eq:contactflow} on $B,A$;\;\;
   \textcolor{knife}{\textsc{Clamp}}};
\node[io] (merge) at (0,-6.5)
  {Merge:\;$W^{\mathrm{out}} = W + \tfrac{\alpha}{r}BA$;\;\;
   \texttt{upload}};
\draw[arr] (init)  -- (phase);
\draw[arr] (phase) -- (fwd);
\draw[arr] (fwd)   -- (exec);
\draw[arr] (exec)  -- (eval);
\draw[arr] (eval)  -- (bwd);
\draw[arr] (bwd)   -- (upd);
\draw[arr] (upd)   -- node[right,font=\scriptsize]{converged} (merge);
\draw[arr, knife]  (upd.east) -- ++(1.3,0) |-
  node[right, pos=0.25, font=\scriptsize, color=knife]{repeat}
  (fwd.east);
\draw[dashed, rounded corners=3pt, black!25]
  (-3.0,-1.7) rectangle (3.5,-5.65);
\node[font=\tiny, black!40, anchor=south east]
  at (3.5,-1.7) {per phase};
\end{tikzpicture}
\caption{Fine-tuning pipeline (\cref{alg:finetune}).
  Outer loop: four-phase curriculum
  (力~$\to$~立~$\to$~丽(正)~$\to$~丽(快)).
  Inner loop: contact gradient
  flow~\eqref{eq:contactflow} on the LoRA adapter.}
\label{fig:finetune-flow}
\end{figure}

\begin{remark}[Validation chain: checkpoint $+$ sandbox $+$ metrics
$\Rightarrow$ coder]
\label{rem:finetune-chain}
Every input to \cref{alg:finetune} is extracted from four artifacts:
a pre-trained checkpoint, a code sandbox, a task dataset (or judge),
and user-defined viability metrics (\cref{def:code-metric}).
No additional assumptions.
\begin{center}
\begin{tabular}{@{}lll@{}}
\toprule
\textbf{Algorithm input} & \textbf{Source} & \textbf{Extraction} \\
\midrule
Backbone $\{W_\ell\}$ ($L$ layers)
  & Checkpoint & \texttt{model.safetensors} \\
Tokenizer $\tau$
  & Checkpoint & \texttt{tokenizer.json} \\
Architecture ($d$, $L$, heads)
  & Checkpoint & \texttt{config.json} \\
LoRA rank $r$, scaling $\alpha$
  & User & Capacity budget ($r = 16$ typical) \\
Task data $\mathcal{D}$ or judge $J$
  & User & Q-A pairs or LLM evaluator \\
$\phi_{\mathrm{topo}},\, \phi_{\mathrm{safe}},\,
\phi_{\mathrm{correct}},\, \phi_{\mathrm{perf}}$
  & User & Viability metrics (\cref{def:code-metric}) \\
Code execution
  & Sandbox & Docker\,/\,\texttt{venv}\,/\,\texttt{nsjail} \\
Gradient $\nabla_{\hat{x}} |\phi|$
  & GPU server & Autodiff (PyTorch\,/\,JAX) \\
\bottomrule
\end{tabular}
\end{center}
The four-phase curriculum parallels the locomotion curriculum
(\cref{alg:loco})---with one refinement: 丽${}^*$ unbundles
into correctness ($L_2^*$) and performance ($L_3^*$):
\begin{center}
\renewcommand{\arraystretch}{1.25}
\begin{tabular}{@{}lllll@{}}
\toprule
\textbf{Phase} & \textbf{Tower} & \textbf{Locomotion} &
\textbf{Fine-tuning} & \textbf{Coding} \\
\midrule
力 & $L_0^*$ & $h > h_{\min}$ (don't fall)
  & $\phi_{\mathrm{topo}} > 0$ & AST valid \\
立 & $L_1^*$ & stable gait
  & $\phi_{\mathrm{safe}} > 0$ & no crash \\
丽(正) & $L_2^*$ & reach target
  & $\phi_{\mathrm{correct}} > 0$ & tests pass \\
丽(快) & $L_3^*$ & energy-efficient
  & $\phi_{\mathrm{perf}} > 0$ & within budget \\
\bottomrule
\end{tabular}
\end{center}
A robot that falls is code that does not parse.
A robot that stumbles is code that crashes.
A robot that walks is code that passes its tests.
A robot that runs is code that runs fast.

Two evaluation modes (\cref{def:task-rkhs}):
\begin{center}
\renewcommand{\arraystretch}{1.25}
\begin{tabular}{@{}lll@{}}
\toprule
& \textbf{Q-A mode} & \textbf{Court mode} \\
\midrule
Data & $(z, x^*)$ pairs & inputs $z$ only \\
$|\phi|$ & $\varepsilon - \|\hat{x} - x^*\|$ & $J(z, \hat{x})$ \\
Judge & ground truth & LLM evaluator \\
Use & code with test suite & open-ended generation \\
\bottomrule
\end{tabular}
\end{center}
Given a checkpoint, a sandbox, a task dataset, and four viability
metrics, every step of \cref{alg:finetune} is mechanically executable.
The model already knows $\mathbf{L}^*$.
School teaches it the task---力, 立, 丽(正), 丽(快)---four phases, one flow.
\end{remark}

\begin{remark}[CUDA TrajOpt: the loop closes]
\label{rem:cuda-loop}
Add CUDA to the language table (\cref{def:code-metric}):
\texttt{nvcc} compiles (力),
\texttt{compute-sanitizer} catches memory errors (立),
the trajectory converges (丽(正)),
the kernel meets its real-time budget (丽(快)).
The fine-tuned coder (\cref{alg:finetune}) writes
CUDA trajectory-optimisation kernels.
These kernels solve the same contact gradient
flow~\eqref{eq:contactflow} that trained the coder---on a physical
body instead of a token sequence.
The contact gradient flow appears at every level:
\begin{enumerate}
  \item \emph{Representation model} (\cref{alg:repmodel}):
  backbone $g_\theta$ learns $\mathbf{L}^*$ from mixed-modality data.
  \item \emph{Fine-tuning} (\cref{alg:finetune}):
  LoRA adapter learns the coding task on $\mathbf{L}^*$.
  \item \emph{Locomotion} (\cref{alg:loco}):
  the CUDA kernel output by the fine-tuned model
  controls the robot via the same flow.
\end{enumerate}
Level~2 produces the code that implements level~3.
The equation writes itself.
\end{remark}

\begin{remark}[For ML engineers: there is no ``agent'']
\label{rem:no-agent}
You do not train an \emph{agent}.
You train a model to produce
\textcolor{water}{perfect code}---one piece at a time.

Each piece must pass four gates in prerequisite order:
\begin{enumerate}
  \item[\textcolor{sword}{力}] It \textbf{parses}.
  \quad $\phi_{\mathrm{topo}} = 1$.\quad
  \textcolor{knife}{Fail $\Rightarrow$ \textsc{Reset}}.
  \item[\textcolor{sword}{立}] It \textbf{runs}.
  \quad $\phi_{\mathrm{safe}} = 1$.\quad
  \textcolor{knife}{Crash $\Rightarrow$ \textsc{Reset}}.
  \item[\textcolor{sword}{丽\textsuperscript{正}}] It is \textbf{correct}.
  \quad $\phi_{\mathrm{correct}} \in [0,1]$.\quad
  \textcolor{knife}{Wrong output $\Rightarrow$ gradient}.
  \item[\textcolor{sword}{丽\textsuperscript{快}}] It is \textbf{fast}.
  \quad $\phi_{\mathrm{perf}} \in [0,1]$.\quad
  \textcolor{knife}{Too slow $\Rightarrow$ gradient}.
\end{enumerate}
A piece that fails gates 1--2 is
\textcolor{knife}{killed}: $|\phi|_\beta \leq 0
\;\Rightarrow\;$\textcolor{knife}{\textsc{Reset}}
(\cref{alg:finetune}, line~13).
A piece that passes all four is
\textcolor{water}{viable}: $|\phi|_\beta > 0$.

The dataset is not dialogues.
Not trajectories.
Not reward signals.
It is \textcolor{water}{$(z, x^*)$ pairs}:
$z$ is a specification, $x^*$ is code that satisfies it.
Court mode (\cref{def:task-rkhs}):
a judge $J(z, \hat{x})$ replaces $x^*$ when ground truth is
unavailable.

There is no reinforcement learning in \cref{alg:finetune}.
There is no reward model.
There is a \textcolor{water}{flow}~\eqref{eq:contactflow},
a \textcolor{knife}{knife} ($\gamma > 0$),
and a \textcolor{sword}{curriculum}
(力~$\to$~立~$\to$~丽(正)~$\to$~丽(快)).
The training loop is supervised:
\textcolor{water}{forward},
\textcolor{water}{evaluate},
\textcolor{water}{backward},
\textcolor{knife}{clamp}.
Four words.
The ``agent'' is what happens \emph{after} training,
when the model generates
\textcolor{water}{enough viable code} to solve a task
that no single piece covers.
That is \textcolor{water}{composition~($\circ$)}, not training.
Training produces pieces.
Composition produces agents.
\end{remark}

\begin{algorithm}[H]
\caption{Code evaluation --- the judge protocol}
\label{alg:code-eval}
\begin{algorithmic}[1]
\Require Specification $z$ (natural language or formal)
\Require Generated code $\hat{x}$
  (output of \cref{alg:finetune})
\Require Sandbox $\mathcal{E}_{\mathrm{sandbox}}$;\;
  time budget $t_{\mathrm{budget}}$
\Require Judge $J$ (external LLM, e.g.\ Gemini-Pro ---
  \emph{must differ from generator})
\Statex
\State \textcolor{sword}{\textbf{Gate\;力}}
  (automated, \textcolor{water}{0 API calls}):
\State \quad $\phi_{\mathrm{topo}} \leftarrow
  \mathbf{1}\bigl[\texttt{parse}(\hat{x})\;\text{succeeds}\bigr]$
  \Comment{compiler / \texttt{ast.parse} / \texttt{nvcc -c}}
\If{$\phi_{\mathrm{topo}} = 0$}
  \textcolor{knife}{\Return} $|\phi|_{\mathrm{code}} = 0$
  \Comment{\textcolor{knife}{killed}: does not parse}
\EndIf
\Statex
\State \textcolor{sword}{\textbf{Gate\;立}}
  (automated, \textcolor{water}{0 API calls}):
\State \quad $(\mathrm{result},\; t_{\mathrm{run}}) \leftarrow
  \mathcal{E}_{\mathrm{sandbox}}.\texttt{run}(
  \hat{x},\; t_{\mathrm{budget}})$
\State \quad $\phi_{\mathrm{safe}} \leftarrow
  \mathbf{1}\bigl[\text{exit code} = 0\bigr]$
  \Comment{no crash, no leak, no timeout}
\If{$\phi_{\mathrm{safe}} = 0$}
  \textcolor{knife}{\Return} $|\phi|_{\mathrm{code}} = 0$
  \Comment{\textcolor{knife}{killed}: runtime crash}
\EndIf
\Statex
\State \textcolor{sword}{\textbf{Gate\;丽(正)}}
  (judge, \textcolor{caution}{1 API call}):
\State \quad $\phi_{\mathrm{correct}} \leftarrow
  J\!\bigl(z,\;\hat{x},\;\mathrm{result}\bigr)
  \;\in [0,1]$
  \Comment{``does output satisfy spec $z$?''}
\Statex
\State \textcolor{sword}{\textbf{Gate\;丽(快)}}
  (profiler, \textcolor{water}{0 API calls}):
\State \quad $\phi_{\mathrm{perf}} \leftarrow
  \max\!\bigl(0,\;\, 1 - t_{\mathrm{run}}\,/\,
  t_{\mathrm{budget}}\bigr)$
  \Comment{wall-clock; no judge needed}
\Statex
\State \textcolor{water}{$|\phi|_{\mathrm{code}}$} $\leftarrow
  \mathrm{soft\text{-}min}_\beta\bigl\{
  \phi_{\mathrm{topo}},\;
  \phi_{\mathrm{safe}},\;
  \phi_{\mathrm{correct}},\;
  \phi_{\mathrm{perf}}\bigr\}$
  \Comment{\cref{def:code-metric}}
\Statex
\Ensure \textcolor{water}{$|\phi|_{\mathrm{code}} \in [0,1]$}:\;
  plug into \cref{alg:finetune}, line~8
\end{algorithmic}
\end{algorithm}

\begin{remark}[The judge is not the teacher]
\label{rem:judge}
The judge $J$ in \cref{alg:code-eval} is \emph{not} the generator.
If \textcolor{water}{DeepSeek} generates, \textcolor{caution}{Gemini}
evaluates.
If \textcolor{water}{Gemini} generates, \textcolor{caution}{DeepSeek}
evaluates.
The adversarial independence is structural:
the court (\cref{def:task-rkhs}) requires a judge
who did not write the code.

The prerequisite chain saves money.
Gates \textcolor{sword}{力} and \textcolor{sword}{立} are
\textcolor{water}{free}:
compilers and sandboxes cost \textcolor{water}{zero} API calls.
Code that does not parse or crashes is
\textcolor{knife}{killed} before the judge sees it.
Only code that parses \emph{and} runs reaches
Gate~\textcolor{sword}{丽(正)}---\textcolor{caution}{one API call}
per surviving piece.
Gate~\textcolor{sword}{丽(快)} is the profiler:
wall-clock time does not require a judge.

The cost structure:
\begin{center}
\renewcommand{\arraystretch}{1.25}
\begin{tabular}{@{}llll@{}}
\toprule
\textbf{Gate} & \textbf{Checker} & \textbf{Cost} &
\textbf{Kills} \\
\midrule
\textcolor{sword}{力} & compiler / parser
  & \textcolor{water}{0 API calls}
  & syntax errors \\
\textcolor{sword}{立} & sandbox
  & \textcolor{water}{0 API calls}
  & crashes, leaks, timeouts \\
\textcolor{sword}{丽(正)} & judge $J$
  & \textcolor{caution}{1 API call}
  & wrong output \\
\textcolor{sword}{丽(快)} & profiler
  & \textcolor{water}{0 API calls}
  & slow code \\
\bottomrule
\end{tabular}
\end{center}
The engineer provides the specification $z$.
The compiler checks \textcolor{sword}{力}.
The sandbox checks \textcolor{sword}{立}.
The judge checks \textcolor{sword}{丽(正)}.
The profiler checks \textcolor{sword}{丽(快)}.
The engineer does not need to know what perfect code looks like.
The \textcolor{water}{pipeline} knows.
\end{remark}

\begin{algorithm}[H]
\caption{Sandbox execution --- the physics engine for code}
\label{alg:sandbox}
\begin{algorithmic}[1]
\Require Generated code $\hat{x}$
  (from \cref{alg:finetune}, line~6)
\Require Resource limits:\;
  $t_{\max}$ (time),\; $m_{\max}$ (memory),\;
  syscall whitelist $\mathcal{S}$
\Require Runtime: Docker, \texttt{nsjail}, or \texttt{venv}
\Statex
\State \textcolor{sword}{\textsc{Build}}:\;
  $\mathcal{C} \leftarrow \texttt{container.create}\bigl(
  \mathrm{image},\;
  t_{\max},\;
  m_{\max},\;
  \texttt{net=none},\;
  \texttt{fs=read\text{-}only},\;
  \mathcal{S}\bigr)$
  \Comment{isolated jail}
\State \textcolor{water}{\textsc{Inject}}:\;
  $\texttt{write}\bigl(
  \mathcal{C}\texttt{:/sandbox/main},\;
  \hat{x}\bigr)$
  \Comment{code $\to$ sandbox}
\State \textcolor{water}{\textsc{Execute}}:\;
  $(\texttt{stdout},\;\texttt{stderr},\;
  \texttt{exit\_code},\;t_{\mathrm{run}})
  \leftarrow \mathcal{C}.\texttt{run}\bigl(
  \texttt{/sandbox/main},\;
  t_{\max}\bigr)$
  \Comment{run with limits}
\State \textcolor{water}{\textsc{Extract}}:\;
  $\mathrm{result} \leftarrow
  \texttt{parse}(\texttt{stdout})$;\;\;
  $\mathrm{crash} \leftarrow
  (\texttt{exit\_code} \neq 0)$
  \Comment{structured output}
\State \textcolor{knife}{\textsc{Destroy}}:\;
  $\mathcal{C}.\texttt{kill}()$;\;\;
  reclaim all resources
  \Comment{\textcolor{knife}{no persistent state}}
\Statex
\Ensure $(\mathrm{result},\;
  t_{\mathrm{run}},\;
  \texttt{exit\_code},\;
  \texttt{stdout},\;
  \texttt{stderr})$
  for \cref{alg:code-eval}, lines~6--7
\end{algorithmic}
\end{algorithm}

\begin{remark}[Sandbox $=$ MuJoCo]
\label{rem:sandbox-mujoco}
\Cref{alg:sandbox} is MuJoCo for code.
The isomorphism is line-by-line:
\begin{center}
\renewcommand{\arraystretch}{1.25}
\begin{tabular}{@{}lll@{}}
\toprule
& \textbf{Locomotion (MuJoCo)} &
\textbf{Code (sandbox)} \\
\midrule
\textsc{Build}
  & load XML model
  & create container \\
\textsc{Inject}
  & apply joint torques
  & write code to filesystem \\
\textsc{Execute}
  & step physics ($\Delta t$)
  & run program ($t_{\max}$) \\
\textsc{Extract}
  & read sensors
  & read stdout, exit code, timing \\
\textsc{Destroy}
  & reset simulation
  & kill container \\
\midrule
\textcolor{knife}{Isolation}
  & joint limits, ground plane
  & no network, read-only fs \\
\textcolor{knife}{Gravity}
  & $9.81\;\mathrm{m/s^2}$
  & time limit $t_{\max}$ \\
\textcolor{knife}{Ground contact}
  & collision detection
  & memory limit $m_{\max}$ \\
\textcolor{knife}{$c_{\max}$}
  & joint torque bound
  & syscall whitelist $\mathcal{S}$ \\
\bottomrule
\end{tabular}
\end{center}
A robot cannot fly through the floor.
Code cannot escape the sandbox.
The physics is different.
The \textcolor{knife}{knife} is the same.
\end{remark}

\begin{remark}[Sinking bound: 地面不能变成水]
\label{rem:sinking-bound}
MuJoCo uses a \emph{soft contact} model: the ground exerts a
spring-damper force $F = k\,\delta + b\,\dot\delta$ proportional
to penetration depth $\delta$.
For small $\delta$ (a few millimetres), this approximation is
accurate.
For large $\delta$, the model breaks: forces explode,
bodies tunnel through the ground plane, and the simulation
becomes physically meaningless.

In the viability geometry (\cref{sec:viab-geom}), the ground
plane is $\partial K$---the boundary of the viability kernel.
The viability metric $g_V = V^{-2}\,g_S$
(\cref{def:viab-metric}) has a \emph{pole} at $V = 0$:
the boundary is at infinite $g_V$-distance, and
\cref{prop:viab-complete} guarantees that no trajectory can
reach it in finite time.
This completeness theorem \emph{requires} that the boundary be
a hard wall.

A soft-contact spring is not a wall---it is a \emph{resource}
that stores and releases energy.
It can be depleted (penetrated past its valid range).
If the ground becomes soft, $\partial K$ is no longer a
constraint that holds; it becomes a medium that yields.
\textcolor{water}{The ground becomes water.}

The sinking bound prevents this:
\[
  z_{\mathrm{contact}}(t)
  \;\geq\;
  z_{\min}
  \quad\text{for all contact points, all } t.
\]
This is not an optimisation heuristic.
It is the \emph{categorical distinction} between $\partial K$
(constraint, infinite barrier, pole of $g_V$) and $V$
(resource, depletable, conformal factor).
The ground is not water.
地面不能变成水.

It is the \emph{robot's} viability obligation to not break the
contact physics of its own simulator.
The sinking bound is not a limitation of MuJoCo that the
engineer works around---it is a constraint the robot must
actively maintain, just as the king must maintain the
institutional structure he governs through
(\cref{thm:paradox}).
The robot lives inside the physics model; breaking it is
breaking~$\partial K$ itself.

In the isomorphism of \cref{rem:sandbox-mujoco}: the code
sandbox's memory limit $m_{\max}$ plays the same role.
If a program consumes memory beyond $m_{\max}$, the OS's
virtual memory subsystem enters pathological behaviour
(thrashing, OOM killer).
The memory limit is the ground plane for code.
\end{remark}

\begin{remark}[Bill of materials: what you actually need]
\label{rem:bill-of-materials}
\Cref{alg:finetune,alg:code-eval,alg:sandbox} are
mechanically executable.
Every step maps to a shell command or library call.
The complete bill of materials:
\begin{center}
\renewcommand{\arraystretch}{1.25}
\begin{tabular}{@{}llll@{}}
\toprule
\textbf{Resource} & \textbf{Example} & \textbf{Cost} &
\textbf{Algorithm} \\
\midrule
GPU server
  & RunPod\,/\,Lambda\,/\,vast.ai
  & \textcolor{caution}{\$1/hr} (A100)
  & \cref{alg:finetune} \\
Checkpoint
  & \texttt{deepseek-coder-v2}
  & \textcolor{water}{free} (HuggingFace)
  & \cref{alg:finetune}, line~1 \\
LoRA library
  & \texttt{peft}
  & \textcolor{water}{free} (pip)
  & \cref{alg:finetune}, line~2 \\
Sandbox
  & Docker\,/\,\texttt{nsjail}
  & \textcolor{water}{free}
  & \cref{alg:sandbox} \\
Judge API key
  & Gemini-Pro
  & \textcolor{water}{free} (60 req/min)
  & \cref{alg:code-eval}, line~11 \\
Compiler
  & \texttt{gcc}\,/\,\texttt{nvcc}\,/\,\texttt{rustc}
  & \textcolor{water}{free}
  & \cref{alg:code-eval}, line~2 \\
Profiler
  & \texttt{time}\,/\,\texttt{perf}\,/\,\texttt{nsight}
  & \textcolor{water}{free}
  & \cref{alg:code-eval}, line~13 \\
Task data
  & $(z, x^*)$ pairs or spec $z$
  & \textcolor{caution}{user-provided}
  & \cref{alg:finetune}, line~5 \\
\bottomrule
\end{tabular}
\end{center}
Six of eight inputs are \textcolor{water}{free}.
The two that cost money:
a GPU server (\textcolor{caution}{\$1/hr})
and your task data.
Everything else---checkpoint, LoRA library, sandbox, judge,
compiler, profiler---is open-source or free-tier.

The automation boundary is sharp.
An AI coding assistant (e.g.\ Claude Code) with SSH access to the
GPU server can execute \cref{alg:finetune} end-to-end:
install dependencies, download the checkpoint, write the training
script, launch the four-phase curriculum, call the sandbox
(\cref{alg:sandbox}), call the judge (\cref{alg:code-eval}),
merge the adapter, and upload the result.
The human provides \textcolor{caution}{two things}:
the GPU server and the task specification.
The \textcolor{water}{pipeline} does the rest.
\end{remark}

\begin{definition}[Instruction set]
\label{def:instruction-set}
The \emph{instruction set} of the agentic calculus is
\[
  \mathcal{I}
  \;=\;
  \bigl\{\,
    \textcolor{water}{\sigma},\;\;
    \textcolor{water}{\circ},\;\;
    \textcolor{sword}{\varphi}
  \,\bigr\}
  \;=\;
  \{\sigma,\, \circ,\, \varphi\}
  \;\setminus\;
  \{\textcolor{knife}{\partial}\}.
\]
A program $p$ on the execution graph $G$ is a finite sequence of
instructions drawn from~$\mathcal{I}$.
The \textsc{Cut} operation $\textcolor{knife}{\partial}$ is
\emph{not} an instruction: it is a constraint imposed by the
environment (the knife), not an action taken by the agent.
\end{definition}

\begin{remark}[The knife cannot cut itself]
\label{rem:knife-self}
The exclusion of $\textcolor{knife}{\partial}$ from~$\mathcal{I}$
is not a design choice.
It is forced.

A program $p \in \mathcal{I}^*$ can
\textcolor{water}{slide} (transport data),
\textcolor{water}{compose} (chain operations), and
\textcolor{sword}{change phase} (switch regime).
It \emph{cannot} cut its own capacity.
Code cannot remove its own weight-decay.
A function cannot delete its own regulariser.
The knife is not a tool the agent wields---it is the boundary the
agent lives inside.

Suppose it could.
Let $p_{\partial}$ be a program that sets $\gamma(e) = 0$ on its
own edges.
By \cref{thm:camus}, setting $\gamma \to 0$ eliminates the
stability margin $\lambda_1 > 0$ (\cref{rem:contact-stability}).
The system becomes unstable: small perturbations amplify.
The program that removes its own knife
\textcolor{knife}{destroys itself}.

This is the diagonal constraint.
Every computable instruction set has exactly one operation it cannot
apply to itself:
\begin{center}
\renewcommand{\arraystretch}{1.25}
\begin{tabular}{@{}lll@{}}
\toprule
\textbf{System} & \textbf{Instruction set} &
\textbf{Excluded operation} \\
\midrule
Turing machine & $\{$read, write, move, halt$\}$ &
  halt on self \\
$\lambda$-calculus & $\{$abstract, apply$\}$ &
  decide own termination \\
Agentic calculus & $\{\sigma, \circ, \varphi\}$ &
  $\partial$ on self \\
\bottomrule
\end{tabular}
\end{center}
The consequence is the same in all three cases:
\\[4pt]
\hspace*{2em}%
\textcolor{knife}{You cannot use the system to prove the system
safe.}
\\[4pt]
The \textcolor{knife}{knife} is the price of
\textcolor{water}{flow}.
Remove it, and the flow destroys the channel.
\end{remark}

\begin{definition}[Sandbox daemon]
\label{def:sandbox-daemon}
The \emph{sandbox daemon} is a persistent instance of
\cref{alg:sandbox} that separates the one-time
\textcolor{sword}{\textsc{Build}} and
\textcolor{knife}{\textsc{Destroy}} from the per-evaluation
loop:
\[
  \underbrace{%
    \textcolor{sword}{\textsc{Build}}
  }_{\text{once}}
  \;\to\;
  \underbrace{%
    \bigl(\,
      \textcolor{water}{\textsc{Inject}} \;\to\;
      \textcolor{water}{\textsc{Execute}} \;\to\;
      \textcolor{water}{\textsc{Extract}}
    \,\bigr)^N
  }_{\text{per evaluation}}
  \;\to\;
  \underbrace{%
    \textcolor{knife}{\textsc{Destroy}}
  }_{\text{once}}.
\]
The container $\mathcal{C}$ persists across all $N$ evaluations
in \cref{alg:finetune}.
The compilation cache, GPU context, and sanitiser hooks survive
between calls.
\end{definition}

\begin{remark}[The daemon is the standing structure]
\label{rem:daemon-standing}
In locomotion (\cref{alg:loco}), the robot body persists across
all rollouts.
You do not rebuild the quadruped every step.
The standing structure (\cref{rem:standing})---joints, torques,
contact modes---is loaded once and reused.

The sandbox daemon is the same pattern.
The four engineering requirements for CUDA trajectory optimisation
map to the four tower layers:
\begin{center}
\renewcommand{\arraystretch}{1.25}
\begin{tabular}{@{}llll@{}}
\toprule
\textbf{Gap} & \textbf{Tower} & \textbf{Daemon component} &
\textbf{Persistence} \\
\midrule
Host-device wrapper
  & \textcolor{water}{力${}^*$}
  & \texttt{cudaMalloc}, grid config, launch
  & GPU context survives \\
Compilation bottleneck
  & \textcolor{sword}{立${}^*$}
  & \texttt{ccache}, hot \texttt{nvcc} daemon
  & only recompile $\Delta$ \\
Mathematical oracle
  & \textcolor{caution}{丽(正)${}^*$}
  & KKT residual $< \varepsilon$,
    constraint violation $< \delta$
  & validator loaded once \\
GPU isolation
  & \textcolor{knife}{刀}
  & \texttt{--gpus all}, \texttt{compute-sanitizer}
  & container $\mathcal{C}$ persists \\
\bottomrule
\end{tabular}
\end{center}
Without the daemon, \cref{alg:sandbox} creates and destroys a
container per evaluation.
With the daemon, the container is the body.
\textsc{Build} is birth.
\textsc{Destroy} is death.
The training loop runs \emph{inside} a life.

The isomorphism is exact:
\begin{center}
\renewcommand{\arraystretch}{1.25}
\begin{tabular}{@{}lll@{}}
\toprule
& \textbf{Locomotion} & \textbf{CUDA sandbox daemon} \\
\midrule
Body & quadruped (MuJoCo) & container $\mathcal{C}$ (Docker) \\
Standing & $h > h_{\min}$ & \texttt{nvcc} cache warm \\
Joint limits & $c(e) \leq c_{\max}$ &
  \texttt{compute-sanitizer}: 0 errors \\
Rollout & simulate $\to$ reward & inject $\to$ compile $\to$ run \\
Persistence & body across rollouts & $\mathcal{C}$ across evaluations \\
\bottomrule
\end{tabular}
\end{center}
You do not rebuild the robot every step.
You do not rebuild the sandbox every evaluation.
The daemon \emph{is} the standing structure for code.
\end{remark}

\section{华容道: Complete Instantiation}\label{sec:huarongdao}

The historical cases in \cref{sec:applications} each validate one
concept. We now present a single finite object that instantiates the
\emph{entire} framework simultaneously: the Chinese sliding block
puzzle 华容道 (Huarong Pass).

\subsection{The puzzle}

\begin{definition}[华容道]\label{def:huarongdao}
A \emph{华容道 instance} is a tuple
$(\mathcal{B},\, \mathcal{P},\, p_*,\, E)$:
\begin{enumerate}[label=(\roman*)]
  \item $\mathcal{B} = [m] \times [n]$: rectangular grid (the
  \emph{board}, 方);
  \item $\mathcal{P} = \{p_1, \ldots, p_k\}$: rectangular blocks
  placed non-overlapping on $\mathcal{B}$ (the \emph{pieces}, 圆);
  \item $p_* \in \mathcal{P}$: distinguished piece (the \emph{king});
  \item $E \subset \partial\mathcal{B}$: boundary region congruent
  to $p_*$ (the \emph{exit}).
\end{enumerate}
The \emph{free cells} $\mathcal{F} = \mathcal{B} \setminus
\bigcup_i p_i$ are the system's degrees of freedom. A
\emph{configuration} is a valid placement of all pieces. A \emph{move}
is a unit translation of one piece into adjacent free cells. The
\emph{configuration graph} $\mathcal{G} = (\mathcal{V}, \mathcal{E})$
has configurations as vertices and legal moves as edges.

The puzzle: does there exist a path in $\mathcal{G}$ from $\sigma_0$
to any $\sigma_f$ with $\sigma_f(p_*) = E$?
\end{definition}

The standard instance is $\mathcal{B} = [4] \times [5]$ in the
configuration 横刀立马 (``horizontal knife, standing horse''):

\begin{center}
\begin{tikzpicture}[scale=0.85]
  % Grid
  \draw[gray!40, thin] (0,0) grid (4,5);
  \draw[very thick] (0,0) rectangle (4,5);
  % Exit
  \draw[very thick, densely dashed] (1,-0.05) -- (1,-0.4) -- (3,-0.4)
    -- (3,-0.05);
  \node[font=\footnotesize] at (2,-0.65) {$E$ (exit)};
  % 曹操 (2×2) — the king
  \fill[black!80] (1.05,3.05) rectangle (2.95,4.95);
  \node[white, font=\bfseries\large] at (2,4.2) {曹操};
  \node[white, font=\scriptsize] at (2,3.5) {$p_*\;(2{\times}2)$};
  % 关羽 (2×1 horizontal) — the knife
  \fill[black!55] (1.05,2.05) rectangle (2.95,2.95);
  \node[white, font=\small\bfseries] at (2,2.5) {关羽 $(2{\times}1)$};
  % Generals (1×2 vertical)
  \fill[black!35] (0.05,3.05) rectangle (0.95,4.95);
  \node[white, font=\small, rotate=90] at (0.5,4) {张飞};
  \fill[black!35] (3.05,3.05) rectangle (3.95,4.95);
  \node[white, font=\small, rotate=90] at (3.5,4) {赵云};
  \fill[black!35] (0.05,1.05) rectangle (0.95,2.95);
  \node[white, font=\small, rotate=90] at (0.5,2) {马超};
  \fill[black!35] (3.05,1.05) rectangle (3.95,2.95);
  \node[white, font=\small, rotate=90] at (3.5,2) {黄忠};
  % Soldiers (1×1) — numbered ①②③④ to match 棋谱 (\cref{app:solution})
  \fill[black!12] (1.05,1.05) rectangle (1.95,1.95);
  \draw[black!50] (1.05,1.05) rectangle (1.95,1.95);
  \node[font=\small] at (1.5,1.5) {\textcircled{\scriptsize 1}};
  \fill[black!12] (2.05,1.05) rectangle (2.95,1.95);
  \draw[black!50] (2.05,1.05) rectangle (2.95,1.95);
  \node[font=\small] at (2.5,1.5) {\textcircled{\scriptsize 2}};
  \fill[black!12] (0.05,0.05) rectangle (0.95,0.95);
  \draw[black!50] (0.05,0.05) rectangle (0.95,0.95);
  \node[font=\small] at (0.5,0.5) {\textcircled{\scriptsize 3}};
  \fill[black!12] (3.05,0.05) rectangle (3.95,0.95);
  \draw[black!50] (3.05,0.05) rectangle (3.95,0.95);
  \node[font=\small] at (3.5,0.5) {\textcircled{\scriptsize 4}};
  % Free cells
  \draw[densely dashed, black!40] (1.05,0.05) rectangle (1.95,0.95);
  \node[black!50, font=\footnotesize] at (1.5,0.5) {$\varnothing$};
  \draw[densely dashed, black!40] (2.05,0.05) rectangle (2.95,0.95);
  \node[black!50, font=\footnotesize] at (2.5,0.5) {$\varnothing$};
  % Legend
  \begin{scope}[font=\footnotesize, anchor=west]
    \fill[black!80] (4.7,4.65) rectangle (5.0,4.85);
    \node at (5.15,4.75) {王: $p_*$ $(2{\times}2)$};
    \fill[black!55] (4.7,4.15) rectangle (5.0,4.35);
    \node at (5.15,4.25) {刀: 关羽 $(2{\times}1)$};
    \fill[black!35] (4.7,3.65) rectangle (5.0,3.85);
    \node at (5.15,3.75) {将: generals $(1{\times}2)$};
    \fill[black!12] (4.7,3.15) rectangle (5.0,3.35);
    \draw[black!50] (4.7,3.15) rectangle (5.0,3.35);
    \node at (5.15,3.25) {卒: \textcircled{\tiny 1}--\textcircled{\tiny 4}\ $(1{\times}1)$};
    \draw[densely dashed, black!40] (4.7,2.65) rectangle (5.0,2.85);
    \node at (5.15,2.75) {$\varnothing$: free cells (水)};
  \end{scope}
\end{tikzpicture}
\end{center}

Ten pieces ($4 + 2 + 4 \cdot 2 + 4 \cdot 1 = 18$ cells), two free
cells, exit at bottom center (width~$2$). The minimum solution requires
81~步 (steps; \cref{def:bu}).

\subsection{The isomorphism}

\begin{theorem}[华容道 $=$ viability maintenance]\label{thm:huarongdao}
The standard 华容道 instance instantiates the viability framework:
\begin{center}
\begin{tabular}{@{}lp{4cm}p{5cm}@{}}
\toprule
\textbf{Framework} & \textbf{华容道} & \textbf{Mechanism} \\
\midrule
Viability axiom & Path $\sigma_0 \to \sigma_f$ in $\mathcal{G}$ &
King must reach exit \\
King & 曹操 $(2{\times}2)$ & Least mobile, highest importance \\
Knife (\cref{def:knife}) & 关羽 $(2{\times}1)$ & Blocks exit;
autonomous; observable \\
Pawns & Soldiers $(1{\times}1)$ & Most mobile, lowest importance \\
Cut vertex & Free cells $\mathcal{F}$ &
$\mathcal{F} = \varnothing \Rightarrow$ frozen \\
Phase transition & 关羽 clears corridor & Before: blocked. After:
path opens \\
Water & Free-cell flow & Slides opposite to piece movement \\
$w = 0$ collapse & Zero free cells & No flow $\to$ no path $\to$
dead \\
\bottomrule
\end{tabular}
\end{center}
\end{theorem}

\begin{proof}
\emph{Viability.} The puzzle asks exactly \cref{ax:viability}: does a
path exist from $\sigma_0$ through $\mathcal{G}$ to a goal state? The
configuration graph is the execution graph; each edge is a legal move;
the viable kernel is the set of configurations from which the exit
remains reachable.

\emph{Knife.} 关羽 satisfies both conditions of \cref{def:knife}:
(1)~autonomous actuation---he can slide independently of 曹操---and
(2)~observability---his position visibly blocks the exit corridor.
He must yield for the king to pass.

\emph{Cut $=$ free cell.} Fill both free cells and $\mathcal{G}$ has
no edges: every configuration is isolated. The free cell inverts the
cut vertex: ``the absence whose removal freezes.'' One structural
element controls all connectivity.

\emph{Mobility $\propto 1/\text{size}$.} A piece of size $s$ needs $s$
aligned free cells to move. Soldiers ($s = 1$): one free cell.
曹操 ($s = 4$): two aligned free cells. The king is the least mobile
agent---\cref{thm:cutvertex} made physical.

\emph{Water.} When a piece slides left, the free cell moves right.
The free cell flows in the opposite direction---it \emph{is} the water
(\cref{def:water}). Each move transfers the free cell to a new
position, enabling the next move. The 81-步 solution is a flow of
water through 81 channels.
\end{proof}

\begin{remark}[横刀立马: the name is the theorem]\label{rem:hrdname}
The configuration's traditional name means ``horizontal knife, standing
horse.'' 刀~(knife) $=$ 关羽 blocking horizontally; 马~(horse) $=$
generals standing vertically. Chinese game designers named the
configuration by its structural properties---the framework's vocabulary,
centuries before graph theory.
\end{remark}

\begin{remark}[义释曹操]\label{rem:guanyu}
In the \emph{Romance of the Three Kingdoms}, 关羽 is stationed at
华容道 after the Battle of Red Cliffs (208~CE). 曹操 retreats through.
关羽 has both conditions of \cref{def:knife}: autonomous actuation
(his army) and observability (曹操 approaching in plain sight). He
chooses path~(a): 义~(righteousness) overrides 忠~(loyalty to Liu Bei),
and he sets $\Ur \to \varnothing$ voluntarily.

The puzzle encodes this: every solution requires moving 关羽 aside.
To solve 华容道 is to perform 义释曹操.
\end{remark}

\begin{remark}[方圆 $\times$ 黑白]\label{rem:fangyuan}
The puzzle's $2 \times 2$ classification is formalized in
\cref{def:fangyuan}. The four statics classify every cell and piece;
the dynamics---刀 (boundary/cut) and 水 (flow/transport)---emerge from
the $2 \times 2$ as the calculus operations of \cref{sec:calculus}.
\end{remark}

\subsection{The experiment}\label{sec:kpc}

\Cref{thm:huarongdao} maps framework concepts to puzzle roles.
We now make the mapping computational.
The configuration graph $\mathcal{G}$ has
$|\mathcal{V}| = 25{,}955$ canonical states and is
connected; every quantity below is \emph{exact}
(self-contained solver: \texttt{solver/}).

\subsubsection*{Water as agent}

The proof of \cref{thm:huarongdao} observed that free cells
flow opposite to piece movement.
We now invert the perspective entirely:
the two free cells \emph{are} the agent~(水),
and the ten pieces \emph{are} the board.

\begin{definition}[水-position and 水-graph]\label{def:water-graph}
A \emph{水-position} is the unordered pair
$\{f_1, f_2\} \subset \mathcal{B}$ of free cells.
Write $W(\sigma)$ for the 水-position of
configuration~$\sigma$.
The \emph{水-graph} has vertex set
$\binom{\mathcal{B}}{2}$ and an edge between
$W(\sigma)$ and $W(\sigma')$ whenever
$(\sigma, \sigma') \in \mathcal{E}$.
\end{definition}

\begin{proposition}[Ergodicity of 水]\label{thm:ergodic}
All $\binom{20}{2} = 190$ possible 水-positions are
reachable from $\sigma_0$.
\end{proposition}

\begin{proof}
Exhaustive BFS on $\mathcal{G}$: the set
$\{W(\sigma) : \sigma \text{ reachable from } \sigma_0\}$
has cardinality $190 = \binom{20}{2}$.
水 can reach every pair of cells in the board.
\end{proof}

\begin{definition}[Free-cell modes]\label{def:free-modes}
The 水-position $\{f_1, f_2\}$ has \emph{mode}:
\begin{itemize}
  \item \textbf{H}~(horizontal pair): $f_1, f_2$
  horizontally adjacent.
  Enables slides of $2 \times h$ pieces.
  \item \textbf{V}~(vertical pair): $f_1, f_2$
  vertically adjacent.
  Enables slides of $w \times 2$ pieces.
  \item \textbf{S}~(separated): $f_1, f_2$ non-adjacent.
  Only $1 \times 1$ pieces (soldiers) can move.
\end{itemize}
The king ($2 \times 2$) requires mode H or V\@.
The knife ($2 \times 1$) requires mode~H\@.
In mode~S, only soldiers act---水 is diffuse.
\end{definition}

\subsubsection*{The counting: 81}

\begin{definition}[步 (step)]\label{def:bu}
A \emph{step}~(步) is a maximal consecutive sequence of
unit moves of the same piece.
Equivalently: each step engages one piece;
all slides of that piece within the step cost zero;
switching to a different piece costs one.
\end{definition}

In Chinese puzzle tradition, the standard counting of
華容道 solutions uses~步, not unit moves.
The difference is algorithmic:

\begin{algorithm}[H]
\caption{Minimum-步 solver
  (0/1~BFS on $\mathcal{G}$)}\label{alg:kpc}
\begin{algorithmic}[1]
\Require $\mathcal{G} = (\mathcal{V}, \mathcal{E})$,\;
  initial $\sigma_0$,\;
  goal $\sigma(p_*) = E$
\Ensure Minimum-步 path $\gamma$,\;
  step count $|\gamma|$
\Statex
\State Augment state:
  $(\sigma, \ell) \in
  \mathcal{V} \times \{1, \ldots, k, \bot\}$,
  $\ell$ = last piece moved
\State $\mathrm{dist}[(\sigma_0, \bot)] \gets 0$;\;
  $Q \gets \mathrm{deque}
  \bigl[\bigl((\sigma_0, \bot),\, 0\bigr)\bigr]$
\While{$Q \neq \varnothing$}
  \State $(\sigma, \ell),\, d \gets Q.\mathrm{popleft}()$
  \If{$\sigma(p_*) = E$}
    \Return $d$ \Comment{goal reached}
  \EndIf
  \For{each neighbour $(\sigma', i)$ of $\sigma$}
    \Comment{$i$ = piece moved}
    \State $c \gets
      \begin{cases}
      0 & i = \ell \\
      1 & i \neq \ell
      \end{cases}$
      \Comment{same piece $\to$ free;
               switch $\to$ 1~步}
    \If{$d + c < \mathrm{dist}[(\sigma', i)]$}
      \State $\mathrm{dist}[(\sigma', i)] \gets d + c$
      \If{$c = 0$}\;
        $Q.\mathrm{pushfront}
        \bigl((\sigma', i),\, d\bigr)$
      \Else\;
        $Q.\mathrm{pushback}
        \bigl((\sigma', i),\, d + 1\bigr)$
      \EndIf
    \EndIf
  \EndFor
\EndWhile
\end{algorithmic}
\end{algorithm}

\begin{remark}[Complexity]\label{rem:complexity}
\begin{center}
\small
\begin{tabular}{@{}lccc@{}}
\toprule
 & \textbf{General} & \textbf{横刀立马} & \textbf{棋谱} \\
 & (sliding block) & (this instance) & (verification) \\
\midrule
Class & PSPACE-complete & --- & --- \\
$|\mathcal{V}|$ & exponential & $25{,}955$ & --- \\
$|\mathcal{E}|$ & exponential & $83{,}896$ & --- \\
Time & exponential & $O(|\mathcal{V}| + |\mathcal{E}|)$ & $O(81)$ \\
Space & exponential & $O(|\mathcal{V}|)$ & $O(1)$ \\
Optimal & ? & $81$~步 & $81$~步 (certificate) \\
\bottomrule
\end{tabular}
\end{center}
The general sliding-block puzzle is PSPACE-complete \cite{hearn}.
This instance has $25{,}955$~states:
BFS solves it in ${<}\,1$~second.
The 棋谱 (\cref{app:solution}) is a certificate
verifiable in $O(81)$~time and $O(1)$~space.
The gap from PSPACE to $O(25{,}955)$ is the entire point:
the board is fixed, the state space is finite,
the solution is exact.
\end{remark}

\begin{theorem}[81~步]\label{thm:eightyone}
The minimum number of steps from $\sigma_0$ to any
$\sigma_f$ with $\sigma_f(p_*) = E$ is exactly~$81$.
Moreover:
\begin{enumerate}[label=(\roman*)]
  \item The result is \emph{independent} of the base move
  set: both single-cell unit moves and multi-cell slides
  yield $81$~步.
  \item Along the optimal path:
  $118$~unit moves, of which $37$ are multi-slide steps
  (same piece slides $\geq 2$ cells).
  \item The $81$~步 decompose as
  $9$~king steps $+\; 72$~水-steps.
  The king acts in only $9/81 \approx 11\%$ of all steps.
\end{enumerate}
\end{theorem}

\begin{proof}
(i)~\Cref{alg:kpc} on the single-cell graph returns~$81$.
Running on the multi-cell graph (where a piece may slide
multiple cells in one unit move) also returns~$81$.
The 0/1~cost structure makes intermediate slides free:
a piece that slides two cells costs the same as one that
slides one cell---both cost zero if the piece was already
engaged, one if it is a new engagement.

(ii)--(iii)~Path extraction from the 0/1~BFS parent
pointers.
\end{proof}

\begin{remark}[Three counting conventions]\label{rem:counting}
\begin{center}
\small
\begin{tabular}{@{}llcl@{}}
\toprule
\textbf{Convention} & \textbf{Definition} &
  \textbf{Count} & \textbf{Algorithm} \\
\midrule
Unit moves & One cell, one direction &
  $116$ & Standard BFS \\
Multi-cell moves & One piece, one direction, any dist.\ &
  $90$ & Standard BFS \\
步 & One piece, any direction, any dist.\ &
  $81$ & 0/1~BFS \\
\bottomrule
\end{tabular}
\end{center}
The 步-count is minimal because it reflects the
agent's decisions: which piece to engage next.
This is the natural cost in agentic theory---the number
of discrete choices, not the number of physical slides.
\end{remark}

\subsubsection*{Phase decomposition}

\begin{proposition}[Phase decomposition: $9 + 72$]%
\label{prop:phase-decomp}
The $81$-步 solution decomposes into $9$~phases,
separated by the $9$~king steps.
水 rearranges between each king move:
\begin{center}
\small
\begin{tabular}{@{}ccrll@{}}
\toprule
\textbf{Phase} & \textbf{水-steps} &
  \textbf{King step} & \textbf{Direction} &
  \textbf{King position} \\
\midrule
1 & $25$ & \#26 & $\to$ & $(2, 3)$ \\
2 & $5$  & \#32 & $\leftarrow$ & $(1, 3)$ \\
3 & $8$  & \#41 & $\downarrow$ & $(1, 2)$ \\
4 & $6$  & \#48 & $\downarrow$ & $(1, 1)$ \\
5 & $3$  & \#52 & $\to$ & $(2, 1)$ \\
6 & $7$  & \#60 & $\leftarrow$ & $(1, 1)$ \\
7 & $6$  & \#67 & $\leftarrow$ & $(0, 1)$ \\
8 & $8$  & \#76 & $\downarrow$ & $(0, 0)$ \\
9 & $4$  & \#81 & $\to$ & $(1, 0) = E$ \\
\bottomrule
\end{tabular}
\end{center}
Phase~1 is the longest: $25$~水-steps to clear the path.
This \emph{is} 義釋曹操 (\cref{rem:guanyu}): 水 must
rearrange $25$~times before the king can move once.
\end{proposition}

\begin{remark}[King's freedom]\label{rem:king-freedom}
Along the unit-move optimal path ($116$~moves),
the king has $\geq 1$ legal move in only
$16$ of $116$ configurations ($14\%$).
The remaining $86\%$ of the time, the king is stuck;
水 works to create the next opening.
This is \cref{thm:cutvertex} quantified:
the king is the least mobile agent, and 水 is
the sole source of mobility.
\end{remark}

\subsubsection*{Saddle and mirror descent}

\begin{definition}[Primal--dual distances]%
\label{def:primal-dual}
For each $\sigma \in \mathcal{V}$:
\begin{align}
  d^+(\sigma) &\coloneqq
    d_{\mathcal{G}}^{\text{步}}(\sigma_0,\, \sigma)
    && \text{(forward / primal)}, \label{eq:dplus} \\
  d^-(\sigma) &\coloneqq
    d_{\mathcal{G}}^{\text{步}}(\sigma,\, \sigma_f)
    && \text{(backward / dual)}, \label{eq:dminus}
\end{align}
where $d_{\mathcal{G}}^{\text{步}}$ is the shortest-path
metric in steps (0/1~BFS distances).
\end{definition}

\begin{proposition}[Duality gap]\label{prop:duality-gap}
For all $\sigma$ on any shortest path
$\gamma = (s_0, \ldots, s_{81})$:
\begin{equation}\label{eq:duality-gap}
  d^+(s_j) + d^-(s_j) \;=\; 81
  \qquad \text{for all } 0 \leq j \leq 81.
\end{equation}
The \emph{saddle configuration} is
$\sigma^* \coloneqq s_{k^*}$ where
\[
  k^* \coloneqq \argmin_{0 \leq j \leq 81}
  \bigl|\, d^+(s_j) - d^-(s_j) \,\bigr|.
\]
Since $d^+(s_j) = j$ and $d^-(s_j) = 81 - j$,
the saddle is at step $k^* = 40$.
\end{proposition}

\begin{proof}
Triangle inequality:
$81 = d_{\mathcal{G}}^{\text{步}}(\sigma_0, \sigma_f)
  \leq d^+(s_j) + d^-(s_j)$.
Equality holds because each $s_j$ lies on a
$(\sigma_0, \sigma_f)$-geodesic.
The midpoint $\lfloor 81/2 \rfloor = 40$.
\end{proof}

\begin{theorem}[Saddle $=$ phase transition]\label{thm:saddle}
At the saddle $\sigma^*$ (step~$40$):
\begin{enumerate}[label=(\roman*)]
  \item 関羽 (knife) is at position $(2, 0)$, far from
  the exit corridor---she has cleared.
  \item The 水-mode is H
  (horizontal pair at $\{(1,2), (2,2)\}$),
  enabling the king's next descent.
  \item The king is at $(1, 3)$: it has moved right once
  (step~$26$) and back left once (step~$32$).
  The first descent begins at step~$41$.
\end{enumerate}
To cross the saddle is to perform 義釋曹操
(\cref{rem:guanyu}).
\end{theorem}

\begin{proof}
By \cref{thm:flowcut}, the min-cut separates
blocked configurations (関羽 covers exit corridor)
from open configurations (corridor cleared).
On the optimal path the cut lies where
$d^+ \approx d^-$, i.e., step~$40$.
Direct inspection of the BFS solution confirms:
at $\sigma^*$, 関羽 has exited the corridor.
\end{proof}

\begin{remark}[Mirror descent]\label{rem:mirror}
\Cref{alg:kpc} is mirror descent on a finite graph.
The forward 0/1~BFS computes the primal potential $d^+$;
the backward 0/1~BFS computes the dual potential $d^-$.
At the saddle, $d^+(\sigma^*) = 40$ and
$d^-(\sigma^*) = 41$: the potentials balance.
This is the same forward--backward structure as
\cref{alg:forward,alg:backward}:
the forward pass computes activations (primal),
the backward pass computes gradients (dual),
and the saddle is the phase transition of the loss.
\end{remark}

\subsubsection*{Mode distribution}

\begin{proposition}[Mode statistics]\label{prop:modes}
Along the $81$-步 path, the 水-mode after each step is:
\begin{center}
\small
\begin{tabular}{@{}lcl@{}}
\toprule
\textbf{Mode} & \textbf{Count} & \textbf{Enables} \\
\midrule
H (horizontal) & $27$ & knife, king (horizontal) \\
V (vertical)   & $42$ & generals, king (vertical) \\
S (separated)  & $12$ & soldiers only \\
\bottomrule
\end{tabular}
\end{center}
The dominance of V-mode ($52\%$) reflects the
puzzle's vertical bias: the king must descend $3$~rows.
S-mode ($15\%$) appears when 水 must diffuse to
reposition---it is the ``reset'' phase.
The king moves only in H or V mode, never in S\@.
\end{proposition}

\subsubsection*{The isomorphism, completed}

\Cref{thm:huarongdao} identified the viability roles.
\Cref{thm:eightyone} extends the isomorphism to the
\emph{computational} structure of \cref{sec:calculus}:

\begin{table}[H]
\centering\small
\caption{Agentic calculus on 華容道: complete correspondence.}%
\label{tab:kpc}
\begin{tabular}{@{}lll@{}}
\toprule
\textbf{Calculus (\cref{sec:calculus})} &
\textbf{華容道} &
\textbf{Value} \\
\midrule
Execution graph (\cref{def:exgraph}) &
  Configuration graph $\mathcal{G}$ &
  $25{,}955$ states \\
Agentic flow (\cref{def:flow}) &
  水-flow (free-cell agent) &
  $190/190$ ergodic \\
Min-cut (\cref{thm:flowcut}) &
  関羽 blocks corridor &
  Cleared at step~$40$ \\
Max-flow &
  Optimal 步-path &
  $81$~步 \\
Forward pass (\cref{alg:forward}) &
  $d^+$: primal 0/1~BFS &
  Distance from $\sigma_0$ \\
Backward pass (\cref{alg:backward}) &
  $d^-$: dual 0/1~BFS &
  Distance to $\sigma_f$ \\
Saddle of loss &
  $\sigma^*$: $d^+ \approx d^-$ &
  Step~$40$ \\
Contact modes (\cref{sec:contact}) &
  水-modes (H, V, S) &
  $3$ modes \\
Mobility $\propto 1/s$ &
  King: $9/81$; Soldiers: most &
  Heavy-tailed \\
Spectral gap (\cref{thm:massgap}) &
  $\lambda_1 > 0$ &
  $\mathcal{G}$ connected \\
Horizon $H$ &
  Geodesic in 步 &
  $81$ \\
Phase decomposition &
  $9$~king $+ 72$~水 &
  $25$~步 to clear \\
刀 dissipation (\cref{def:dissipation}) &
  関羽 yields &
  Phase~1: 義釋曹操 \\
水 (\cref{def:water}) &
  Free-cell agent &
  $190/190$ ergodic \\
\bottomrule
\end{tabular}
\end{table}

\begin{remark}[$81$ as horizon]\label{rem:eightyone}
The minimum $81$ is not an arbitrary combinatorial fact.
It is the \emph{horizon} of the predictive controller:
the geodesic length in $\mathcal{G}$ measured in~步,
equivalently the minimum number of
\textcolor{water}{水}~decisions to transport the king from
$\sigma_0$ to the exit.
In the language of \cref{thm:massgap}:
$H = 81$ is the spectral gap made finite.

The three conventions (\cref{rem:counting}) separate
cleanly: unit moves ($116$) count physical slides;
multi-cell moves ($90$) count piece--direction pairs;
步~($81$) count agent decisions.
The agentic theory uses~步 because it counts
\emph{what the agent chooses}, not what the physics does.
\end{remark}

\section{Conclusion}\label{sec:conclusion}

We have presented an agentic theory of viability maintenance built on a
single axiom (the existence of a viable path to infinity) and a
two-condition criterion (the knife). The framework produces three main
theorems (binary lifecycle, fixed-point impossibility, unconstrained
power paradox) and a central interpretive result: the knife is the mean
field.

The knife is not an intrinsic property of a resource. It is a
statistical deviation from the system's mean autonomous actuation,
made visible by the detection function and made dangerous by the
viability axiom. Phase transitions shift the mean, not the individual.
The king responds to the mean, not to intent.

The agentic calculus (\cref{sec:calculus}) translates this theory into
an operational language: every theorem becomes a flow-theoretic
proposition, the knife becomes the min-cut, and the viable path becomes
the max-flow. The duality ``the knife is the mean'' is a restatement of
max-flow/min-cut duality on the execution graph.

This reframing connects viability maintenance to mean-field theory in
statistical mechanics, where phase transitions are driven by shifts in
the order parameter rather than changes in individual configurations.
The viability axiom plays the role of free energy minimization; the
knife plays the role of the critical fluctuation.

Two millennia of Chinese imperial history validate the framework with
unusual clarity. The same structure appears---with instructive
breaks---in the Atlantic slave trade, ideological hatred, parasitic
network topologies, and militarism. The framework's failure conditions
(\cref{sec:domain}) are as informative as its successes: they delineate
the boundary between systems where the viability axiom operates cleanly
and systems where it is dominated by other dynamics.

The knife is the mean. Viability maintenance is a mean-field phenomenon.
The theory is agentic because the agents---not their intentions, not
their narratives, not their moral qualities, but their structural
positions in the execution graph---determine the outcome.


% ── Appendices ────────────────────────────────────────────
\appendix
\section{原典与人物}\label{app:sources}

The historical analysis in this paper draws primarily on Sima Qian's
\emph{Shiji} (《史记》, \emph{Records of the Grand Historian},
c.~94~BCE) and Du Mu's \emph{A Fang Gong Fu} (《阿房宫赋》, 825~CE).
This appendix collects the original Classical Chinese passages cited or
referenced in the main text, with English translations and brief
profiles of the historical figures.

\subsection{人物志}\label{app:persons}

\paragraph{刘邦 Liu Bang (256--195 BCE).}
Founder of the Han dynasty. Rose from minor local official (亭长) to
emperor. Near-zero personal combat ability; made himself the cut vertex
of the execution graph (\cref{ex:liubang}).

\begin{quote}
高祖为人,仁而爱人,喜施,意豁如也。常有大度。不事家人生产作业。

\medskip
\emph{Gaozu was a man of benevolence who loved people, was generous in
giving, and broad-minded. He had great magnanimity. He did not engage
in household production or labor.}
\hfill ---《史记·高祖本纪》
\end{quote}

Translation into the framework: Liu Bang himself had no actuation
capability. He could not fight, could not administer, could not
strategize. His sole structural role was as the mandatory routing node.

On first seeing the First Emperor's procession:
\begin{quote}
嗟乎,大丈夫当如此也!

\medskip
\emph{Ah, a great man should be like this!}
\hfill ---《史记·高祖本纪》
\end{quote}

Liu Bang saw the existence proof (\cref{thm:qin}) and wanted to
\emph{be} the system's cut vertex.

His self-assessment after founding the Han dynasty:
\begin{quote}
夫运筹策帷帐之中,决胜於千里之外,吾不如子房。镇国家,抚百姓,给馈饷,不绝粮道,吾不如萧何。连百万之军,战必胜,攻必取,吾不如韩信。此三者,皆人杰也,吾能用之,此吾所以取天下也。项羽有一范增而不能用,此其所以为我擒也。

\medskip
\emph{For devising strategies within a tent and securing victory a
thousand li away, I am not as good as Zhang Liang. For governing the
state, caring for the people, providing supplies, and keeping the grain
roads open, I am not as good as Xiao He. For commanding a million
soldiers, winning every battle and taking every siege, I am not as good
as Han Xin. These three are all heroes---but I can use them. This is
why I won the empire. Xiang Yu had one Fan Zeng but could not use him.
This is why he was captured by me.}
\hfill ---《史记·高祖本纪》
\end{quote}

「吾能用之」(\emph{I can use them}) is the operational definition of
the cut vertex: actuation resides in others, but routing authority
resides in Liu Bang. Every execution chain passes through him. The
character 用 (use/employ) is not metaphorical---it is a precise
description of the cut vertex's function in the execution graph.

\paragraph{项羽 Xiang Yu (232--202 BCE).}
Supreme military commander of the anti-Qin uprising. Maximum actuator
(\cref{ex:liubang}): personal combat ability unmatched in the system.

\begin{quote}
籍长八尺馀,力能扛鼎,才气过人。

\medskip
\emph{[Xiang] Ji was over eight chi tall, could lift a bronze tripod,
and his talent and spirit surpassed all others.}
\hfill ---《史记·项羽本纪》
\end{quote}

\begin{quote}
项王嗔目而叱之,赤泉侯人马俱惊,辟易数里。

\medskip
\emph{The King of Xiang glared and bellowed at him; the Marquis of
Chiquan and his horse both recoiled in terror, retreating several li.}
\hfill ---《史记·项羽本纪》
\end{quote}

On seeing the same procession Liu Bang saw:
\begin{quote}
彼可取而代也!

\medskip
\emph{That one---I can replace him!}
\hfill ---《史记·项羽本纪》
\end{quote}

Liu Bang: ``I want to \emph{be} this.'' Xiang Yu: ``I can
\emph{replace} him.'' One read the existence proof as a system to
inhabit. The other read it as a person to defeat.

Han Xin's assessment of Xiang Yu:
\begin{quote}
项王见人恭敬慈爱,言语呕呕,人有疾病,涕泣分食饮,至使人有功当封爵者,印刓敝,忍不能予。此所谓妇人之仁也。

\medskip
\emph{The King of Xiang is respectful and caring when he meets people;
his speech is warm and gentle. When someone is ill, he weeps and shares
his food and drink. But when a man has earned merit and deserves a
title, he fondles the seal until its edges are worn smooth, and still
cannot bring himself to hand it over. This is what is called the
benevolence of a woman.}
\hfill ---《史记·淮阴侯列传》
\end{quote}

「印刓敝,忍不能予」: the seal is carved and ready, rubbed smooth from
handling, yet he cannot let it go. This is not benevolence---it is the
inability to distribute actuation. A cut vertex that cannot delegate
is a maximum actuator pretending to route.

After conquering the Qin capital:
\begin{quote}
富贵不归故乡,如衣绣夜行,谁知之者!

\medskip
\emph{To be wealthy and noble but not return home is like wearing
embroidered robes at night---who would see it?}
\hfill ---《史记·项羽本纪》
\end{quote}

His last song, at Gaixia (垓下歌):
\begin{quote}
力拔山兮气盖世,时不利兮骓不逝。\\
骓不逝兮可奈何,虞兮虞兮奈若何!

\medskip
\emph{My strength could uproot mountains, my spirit overmastered the
world. / But the times turned against me, and my horse would not go. /
My horse would not go---what can be done? / Yu, oh Yu---what will
become of you?}
\hfill ---《史记·项羽本纪》
\end{quote}

At the bank of the Wu River, refusing to cross:
\begin{quote}
天之亡我,我何渡为!且籍与江东子弟八千人渡江而西,今无一人还,纵江东父兄怜而王我,我何面目见之?

\medskip
\emph{Heaven has destroyed me---why should I cross? I crossed the river
westward with eight thousand sons of Jiangdong, and not one has
returned. Even if the elders of Jiangdong pitied me and made me king,
with what face could I see them?}
\hfill ---《史记·项羽本纪》
\end{quote}

「天之亡我」(\emph{Heaven has destroyed me}). Not heaven. A single
actuator cannot cover the full state space (\cref{ex:liubang}).
Structural failure, not fate.

\paragraph{韩信 Han Xin (?--196 BCE).}
Military genius. Commanded Liu Bang's armies; conquered more territory
than any other general in the Chu--Han war. Pure knife
(\cref{ex:hanxin}): his execution chain was closed---armies obeyed him,
not Liu Bang.

On Liu Bang's ability:
\begin{quote}
陛下不能将兵,而善将将,此乃信之所以为陛下禽也。

\medskip
\emph{Your Majesty cannot command soldiers, but excels at commanding
commanders. This is why I was captured by Your Majesty.}
\hfill ---《史记·淮阴侯列传》
\end{quote}

「善将将」(\emph{excels at commanding commanders}) $=$ cut vertex
property. Liu Bang does not actuate directly; he routes the actuation
of others.

The incident that sealed his fate---requesting the title King of Qi
during wartime:
\begin{quote}
汉王大怒,骂曰:「吾困于此,旦暮望若来佐我,乃欲自立为王!」张良、陈平蹑汉王足,因附耳语……汉王亦悟……遂遣张良立信为齐王。

\medskip
\emph{The King of Han was furious and cursed: ``I am trapped here,
waiting day and night for you to come help me, and you want to make
yourself king!'' Zhang Liang and Chen Ping stepped on the king's foot
and whispered in his ear\ldots\ The King of Han understood\ldots\ and
sent Zhang Liang to install Han Xin as King of Qi.}
\hfill ---《史记·淮阴侯列传》
\end{quote}

Zhang Liang stepping on Liu Bang's foot $=$ recalibrating the search:
the viable path currently requires Han Xin's actuation (wartime phase),
so the knife cannot be cut yet. Granting the title $=$ extending the
horizon. After the phase transition, the knife was cut.

His final words:
\begin{quote}
果若人言,「狡兔死,良狗亨;高鸟尽,良弓藏;敌国破,谋臣亡。」天下已定,我固当亨!

\medskip
\emph{It is as people said: ``When the cunning hare is killed, the
hunting dog is cooked; when the high-flying birds are gone, the good
bow is stored away; when the enemy state is destroyed, the strategist
perishes.'' The empire is settled---naturally I was to be cooked!}
\hfill ---《史记·淮阴侯列传》
\end{quote}

Han Xin quoted the answer but did not parse its fine structure. See
\cref{app:proverb} for the detailed analysis.

\paragraph{萧何 Xiao He (?--193 BCE).}
Chief administrator. Controlled grain supply and the capital during Liu
Bang's campaigns. Half-knife who self-blunted via deliberate
self-corruption (\cref{ex:xiaohe}).

\begin{quote}
相国何买田宅必居穷处,为家不治垣屋。曰:「后世贤,师吾俭;不贤,毋为势家所夺。」

\medskip
\emph{Chancellor He always bought fields and houses in the poorest
locations, and did not repair the walls of his home. He said: ``If my
descendants are worthy, they will follow my example of frugality. If
they are not, the property will be too poor for powerful families to
bother seizing.''}
\hfill ---《史记·萧相国世家》
\end{quote}

The stated reason (frugality for descendants) is a cover story. The
operational function: signal to the king that $\|\Ur\| \to 0$. An
official this visibly degraded in wealth and reputation cannot
coordinate a revolt. This is path~(a) executed through reputation
rather than resignation---pulling $\|\Ur\|$ toward $\bar{U}$
(\cref{thm:meanfield}).

\paragraph{张良 Zhang Liang (?--189 BCE).}
Strategist. Descendant of five generations of prime ministers of the
state of Han. Devoted his fortune to avenging Han's destruction by Qin.
Not a knife (\cref{ex:zhangliang}): pure advisory function---every
execution chain passed through Liu Bang.

\begin{quote}
留侯乃称曰:「家世相韩,及韩灭,不爱万金之资,为韩报仇强秦,天下振动。今以三寸舌为帝者师,封万户,位列侯,此布衣之极,于良足矣。愿弃人间事,欲从赤松子游耳。」乃学辟谷。

\medskip
\emph{The Marquis of Liu said: ``My family served as ministers of Han
for five generations. When Han was destroyed, I did not begrudge ten
thousand gold to seek vengeance against mighty Qin, and the empire
trembled. Now with my three-inch tongue I have become the emperor's
teacher, been enfeoffed with ten thousand households, and ranked as
marquis. For a commoner, this is the pinnacle---it is enough for me.
I wish to abandon worldly affairs and follow the immortal Chi Songzi.''
He then took up the practice of grain abstinence.}
\hfill ---《史记·留侯世家》
\end{quote}

「三寸舌」(\emph{three-inch tongue}) $=$ pure function. No actuation.
Zhang Liang's retirement (辟谷, grain abstinence) is not path~(a)
(放下)---he had no knife to put down. It is confirmation:
$\Ur = \varnothing$ from the first day, now the function itself
is shut down.

\paragraph{商鞅 Shang Yang (?--338 BCE).}
Architect of the Qin reform system (\cref{sec:qin}). Installed the
submartingale reform sequence (\cref{thm:shangyang}) that transformed
Qin from a peripheral state into the unification engine.

\begin{quote}
令民为什伍,而相牧司连坐。不告奸者腰斩,告奸者与斩敌首同赏,匿奸者与降敌同罚。

\medskip
\emph{He organized the people into groups of five and ten households,
to watch over and be jointly liable for each other. Those who failed to
report wrongdoers were cut in half at the waist. Those who reported
wrongdoers received the same reward as those who beheaded enemy
soldiers. Those who harbored wrongdoers received the same punishment
as those who surrendered to the enemy.}
\hfill ---《史记·商君列传》
\end{quote}

This is $\Obs \to \Obs_{\max}$: neighbors as a distributed sensor
network. The reward structure ensures every agent has positive incentive
to maximize observability.

\begin{quote}
商君相秦十年,宗室贵戚多怨望者。……秦惠王车裂商君以徇,曰:「莫如商鞅反者!」遂灭商君之家。

\medskip
\emph{Lord Shang governed Qin for ten years; many among the royal
house and the powerful clans harbored resentment.\ldots\ King Hui of
Qin had Lord Shang torn apart by chariots and displayed, saying: ``Let
none rebel as Shang Yang did!'' His entire clan was exterminated.}
\hfill ---《史记·商君列传》
\end{quote}

The unitary group acts on all vectors without exception
(\cref{thm:shangyang}). The installer is not in the invariant
subspace of the group he created.

\paragraph{范蠡 Fan Li (536--448 BCE).}
Minister of Yue. After helping King Goujian destroy the state of Wu,
Fan Li immediately left (\cref{sec:dollar}). Accumulated three
fortunes, dispersed two---ensuring $\Ur$ never crossed the knife
threshold.

\begin{quote}
范蠡遂去,自齐遗大夫种书曰:「飞鸟尽,良弓藏;狡兔死,走狗烹。越王为人长颈鸟喙,可与共患难,不可与共乐。子何不去?」

\medskip
\emph{Fan Li then departed. From Qi he sent a letter to Grand Officer
Zhong, saying: ``When the birds are gone, the good bow is stored away;
when the cunning hare is killed, the hunting dog is cooked. The King
of Yue has a long neck and a bird's beak---one can share hardship with
him, but not prosperity. Why do you not leave?''}
\hfill ---《史记·越王勾践世家》
\end{quote}

This passage---written to his colleague Wen Zhong (文种), who did not
leave and was subsequently forced to commit suicide---is the origin of
the proverb Han Xin quoted two centuries later (\cref{rem:hanxin}).
See \cref{app:proverb} for the fine structure.

\begin{quote}
范蠡……乃乘扁舟浮于江湖,变名易姓……止于陶,……十九年之中三致千金,再分散与贫交疏昆弟。

\medskip
\emph{Fan Li\ldots\ took a small boat and drifted on the rivers and
lakes, changing his name\ldots\ He settled at Tao\ldots\ In nineteen
years he amassed a fortune of a thousand gold three times, and twice
distributed it to his poor friends and distant kin.}
\hfill ---《史记·货殖列传》
\end{quote}

Three accumulations, two dispersals. Each time $\Ur$ approached the
threshold, Fan Li reset it. The dollar is a knife precursor
(\cref{sec:dollar}); Fan Li ensured the precursor never converted.

\paragraph{陈胜 Chen Sheng (?--208 BCE).}
Farmer and conscript laborer. When Qin's water reached zero
(\cref{thm:dumu}), the pawn became a knife.

\begin{quote}
陈胜佐之,并杀两尉。召令徒属曰:「公等遇雨,皆已失期,失期当斩。藉第令毋斩,而戍死者固十六七。且壮士不死即已,死即举大名耳,王侯将相宁有种乎!」

\medskip
\emph{Chen Sheng helped him, and together they killed the two officers.
He gathered the conscripts and said: ``You have all been delayed by
rain and missed the deadline. The penalty for missing the deadline is
death. Even if you are not executed, six or seven out of ten who go to
garrison duty will die. Besides, when a true man dies, he dies with
his name known---are kings and nobles born to their station?''}
\hfill ---《史记·陈涉世家》
\end{quote}

「失期当斩」(\emph{miss the deadline, face execution}): $w = 0$.
Every path leads to death. The viability axiom now applies to the pawn
himself (\cref{prop:binary}). 「王侯将相宁有种乎」:
$U_{\text{pawn}}: \varnothing \to \neq\varnothing$. The breakpoint
has dissolved.

\paragraph{冯谖 Feng Xuan (3rd century BCE).}
Retainer of Lord Mengchang of Qi. Architect of the observability
reduction strategy.

\begin{quote}
冯谖曰:「狡兔有三窟,仅得免其死耳。今君有一窟,未得高枕而卧也。请为君复凿二窟。」

\medskip
\emph{Feng Xuan said: ``A cunning hare has three burrows, and barely
manages to escape death. You, my lord, have only one burrow---you
cannot yet sleep with your head high on the pillow. Allow me to dig
two more burrows for you.''}
\hfill ---《战国策·齐策四》
\end{quote}

Three burrows $=$ three alternative positions $=$ reducing the king's
detection function $\Obs$ coverage. If the king cannot observe your
$\Ur$, you move from ``knife'' to ``hidden knife''---still dangerous,
but outside the king's strategy space.

\paragraph{魏征 Wei Zheng (580--643 CE).}
Chief advisor to Emperor Taizong of Tang. Articulated the water
dynamics (\cref{sec:water}) as a political principle.

\begin{quote}
臣闻古语云:「君,舟也;人,水也。水能载舟,亦能覆舟。」陛下以为可畏,诚如圣旨。

\medskip
\emph{Your minister has heard an ancient saying: ``The ruler is a boat;
the people are the water. Water can carry the boat, and water can
capsize the boat.'' That Your Majesty considers this worthy of fear is
truly wise.}
\hfill ---《贞观政要》
\end{quote}

Water carries the boat ($w > 0 \implies$ pawn serves $\implies$ king
sovereign) and capsizes the boat ($w = 0 \implies$ pawn $\to$ knife
$\implies$ king absorbed). Wei Zheng (630~CE) and Du Mu (825~CE) stated
the same theorem (\cref{thm:dumu}). Wei Zheng gave the intuition.
Du Mu gave the proof. This paper gives the formalization.

\subsection{飞鸟尽良弓藏:韩信之误读}\label{app:proverb}

Han Xin's final words (\cref{rem:hanxin}) quote the proverb
originating from Fan Li. The proverb is treated in Chinese historical
tradition as a single lesson: ``after the war, the meritorious are
killed.'' This reading is imprecise. The proverb contains three
structurally distinct resources with three distinct fates:

\begin{center}
\begin{tabular}{@{}llll@{}}
\toprule
\textbf{Proverb} & \textbf{Resource type} & $\Ur$ &
\textbf{Fate} \\
\midrule
飞鸟尽,良弓\textbf{藏} & Bow (no autonomous actuation) &
$= \varnothing$ & \textbf{Stored}---not destroyed \\
狡兔死,走狗\textbf{烹} & Dog (autonomous actuator) &
$\neq \varnothing$ & \textbf{Cooked}---eliminated \\
敌国破,谋臣\textbf{亡} & Strategist---depends on $\Ur$ &
? & Depends on classification \\
\bottomrule
\end{tabular}
\end{center}

The bow is \emph{stored} (藏), not \emph{cooked} (烹). The proverb
itself distinguishes between the two fates at the level of the verb.
The dog---an autonomous actuator that can hunt independently---is
killed. The bow---a tool that cannot shoot itself---is merely put away.

Han Xin conflated all three. The corrected version:

\begin{itemize}
  \item \textbf{Bow} ($\Ur = \varnothing$): Zhang Liang. Retired and
  survived. A bow that stores itself.
  \item \textbf{Dog} ($\Ur \neq \varnothing$): Han Xin. Killed. A dog
  that refused to stop hunting.
  \item \textbf{Half-tool} ($\Ur$ partially non-empty): Xiao He.
  Imprisoned, then released. A dog that deliberately blunted its teeth.
\end{itemize}

The survival strategies are also encoded in the proverb:

\begin{center}
\begin{tabular}{@{}lp{5.5cm}l@{}}
\toprule
\textbf{Strategy} & \textbf{Mechanism} & \textbf{Applicable to} \\
\midrule
Don't let the birds disappear & Maintain
$J_{\text{king}}(r) > 0$ (king still needs you) & Bow
($\Ur = \varnothing$) \\
Three burrows (冯谖) & Reduce $\Obs$ coverage &
Dog ($\Ur \neq \varnothing$) \\
Self-blunting / dispersal & $\Ur \to \varnothing$ &
Any type \\
\bottomrule
\end{tabular}
\end{center}

Han Xin needed the third strategy (put down the knife). He chose the
zeroth (do nothing). The answer was in the proverb he quoted---弓
is \emph{stored}, 狗 is \emph{cooked}---but he did not parse the
distinction.

Fan Li understood this. His letter to Wen Zhong describes the bow and
the dog. His advice: become the bow (leave). But Wen Zhong was the
prime minister of Yue---his administrative network made him a dog,
not a bow. He could not leave without first setting $\Ur \to \varnothing$,
and a prime minister's administrative network is not so easily
relinquished. Wen Zhong stayed. Wen Zhong died.

\subsection{阿房宫赋}\label{app:afanggongfu}

Du Mu (杜牧, 803--852~CE) wrote the \emph{Rhapsody on the Epang
Palace} in 825~CE. \Cref{thm:dumu} formalizes its central argument.
The full text follows in traditional characters, with English
translation and framework mapping.

\begin{remark}[Pronunciation of 阿房]
The standard reading of 阿房宫 is \emph{Ep\'ang G\=ong}
(or \emph{\=Ef\'ang G\=ong} per the Guifan Dictionary),
not \emph{\=Af\'ang G\=ong}.
The character 阿 takes the reading \emph{\=e} in this compound; 房
takes the reading \emph{p\'ang}.
We retain the romanisation \emph{A Fang Gong Fu} throughout because it
is the naive character-by-character reading---each character pronounced
in isolation, stripped of relational context.
That is the point: the ``correct'' pronunciation is a phase function of
the compound, not an intrinsic property of the individual character.
The mispronunciation instantiates the thesis.
\end{remark}

\subsubsection*{第一段:存在性证明的物理实例}

\begin{quote}
六王畢,四海一。蜀山兀,阿房出。覆壓三百餘里,隔離天日。驪山北構而西折,直走咸陽。二川溶溶,流入宮牆。五步一樓,十步一閣。廊腰縵迴,簷牙高啄。各抱地勢,鈎心鬬角。盤盤焉,囷囷焉,蜂房水渦,矗不知其幾千萬落。長橋臥波,未雲何龍?複道行空,不霽何虹?高低冥迷,不知西東。歌臺暖響,春光融融。舞殿冷袖,風雨淒淒。一日之內,一宮之間,而氣候不齊。

\medskip
\emph{The six kings finished, the four seas unified, the Shu mountains
stripped bare, and the Epang Palace rose. It pressed down over three
hundred li, blocking out sun and sky. From the northern foot of
Mt.~Li it turned west, running straight to Xianyang. Two rivers flowed
gently into its walls. Every five paces a tower, every ten paces a
pavilion; corridors wound and turned, eaves rose like pecking beaks;
each structure embraced the terrain, their ridges interlocking.
Spiraling and curving, like beehives and whirlpools, towering---who
knows how many thousands of clusters. The long bridge lay over the
waves---if not clouds, why a dragon? The skyway crossed the air---if
not after rain, why a rainbow? In the haze of heights and depths, one
could not tell west from east. On the singing stages, warm sounds like
spring sunlight; in the dancing halls, cold sleeves like wind and rain.
Within a single day, within a single palace, the seasons differed.}
\end{quote}

All resources converge on a single node (Xianyang). Star-graph dispatch
(\cref{sec:qin}). The palace's scale $=$ the center node's extractable
throughput, made physical.

\subsubsection*{第二段:相变后的掠夺}

\begin{quote}
妃嬪媵嬙,王子皇孫,辭樓下殿,輦來於秦。朝歌夜絃,爲秦宮人。明星熒熒,開粧鏡也。緑雲擾擾,梳曉鬟也。渭流漲膩,棄脂水也。煙斜霧橫,焚椒蘭也。雷霆乍驚,宮車過也。轆轆遠聽,杳不知其所之也。一肌一容,盡態極妍。縵立遠視,而望幸焉,有不得見者,三十六年。

\medskip
\emph{The consorts and attendants, the princes and grandsons of the six
kings, left their towers and descended their halls, riding in carriages
to Qin. Morning songs and evening strings---they became Qin's palace
women. Bright stars glittering---that was opening their mirrors. Green
clouds in disorder---that was combing their morning hair. The Wei River
rising oily---that was discarded cosmetics. Smoke slanting, mist
spreading---that was burning pepper and orchid. Thunder suddenly
startling---a palace carriage passing. Wheels rumbling into the
distance, vanishing beyond knowing. Every curve and every face brought
to its utmost beauty, standing gracefully, gazing far, hoping for the
emperor's favor. Some waited thirty-six years and never saw him.}
\end{quote}

Resources collected at the cut vertex, but the cut vertex's bandwidth
is finite. One node cannot process all chains simultaneously.
``Thirty-six years without being seen'' $=$ star-graph bottleneck.

\subsubsection*{第三段:生存公理的普遍性}

\begin{quote}
燕趙之收藏,韓魏之經營,齊楚之精英,幾世幾年,剽掠其人,倚疊如山。一旦不能有,輸來其間。鼎鐺玉石,金塊珠礫,棄擲邐迤。秦人視之,亦不甚惜。

嗟乎!一人之心,千萬人之心也。秦愛紛奢,人亦念其家。奈何取之盡錙銖,用之如泥沙!使負棟之柱,多於南畝之農夫;架梁之椽,多於機上之工女;釘頭磷磷,多於在庾之粟粒;瓦縫參差,多於周身之帛縷;直欄橫檻,多於九土之城郭;管絃嘔啞,多於市人之言語:使天下之人不敢言而敢怒。獨夫之心,日益驕固。戍卒叫,函谷舉。楚人一炬,可憐焦土。

\medskip
\emph{The treasures of Yan and Zhao, the collections of Han and Wei,
the finest goods of Qi and Chu---plundered from their people over how
many generations, piled up like mountains. One day they could keep them
no more, and all was shipped here. Tripods used as pots, jade treated
as stone, gold discarded in heaps, pearls scattered like gravel---the
Qin people saw these and did not much care.}

\emph{Alas! One man's heart is the heart of ten thousand men. Qin loved
extravagance, yet people also cherish their homes. Why take from them
down to the last coin, and spend it like mud and sand? The pillars
outnumbered the farmers; the rafters outnumbered the weavers; the
nail-heads outnumbered the grain in the granaries; the tile-seams
outnumbered the threads in a bolt of silk; the railings outnumbered the
city walls of the nine provinces; the cacophony of pipes and strings
outnumbered the speech of the marketplace. The people of the empire
dared not speak, but dared to be angry. The tyrant's heart grew daily
more arrogant and obstinate. The garrison soldiers cried out, Hangu
Pass was taken, a torch from Chu, and---alas---scorched earth!}
\end{quote}

「一人之心,千萬人之心也」$=$ the viability axiom is not the king's
exclusive property (\cref{ax:viability}). Every agent has the same
axiom. 「取之盡錙銖」$=$ $dw/dt \ll 0$ (\cref{thm:dumu}).
「不敢言而敢怒」$=$ $\Ur = \varnothing$ still (not speaking $=$ no
autonomous actuation), but energy accumulates. 「獨夫之心,日益驕固」
$=$ the cut vertex receives no feedback---all correction channels have
been eliminated. 「戍卒叫,函谷舉」$=$ $U_{\text{pawn}}: \varnothing
\to \neq\varnothing$ (\cref{prop:binary}). Phase transition fires.

\subsubsection*{第四段:定理}

\begin{quote}
嗚呼!\textbf{滅六國者,六國也,非秦也。族秦者,秦也,非天下也。}嗟夫!使六國各愛其人,則足以拒秦。使秦復愛六國之人,則遞三世可至萬世而爲君,誰得而族滅也。\textbf{秦人不暇自哀,而後人哀之。後人哀之,而不鑑之,亦使後人而復哀後人也。}

\medskip
\emph{Alas! It was the six states that destroyed the six states, not
Qin. It was Qin that destroyed Qin, not the world. Had the six states
each loved their own people, they would have had enough to resist Qin.
Had Qin, in turn, loved the people of the six states, it could have
passed from the third generation to the ten-thousandth and remained
sovereign---who could have destroyed it? The people of Qin had no
leisure to mourn for themselves, and later generations mourned for them.
But if later generations mourn them without learning from them, they
will only cause yet later generations to mourn for the later
generations in turn.}
\end{quote}

This is \cref{thm:dumu} in prose. 「滅六國者六國也」$=$ the six states
depleted their own water (internal knife dynamics, coordination cost
$O(n^2)$). 「族秦者秦也」$=$ $w(t) \to 0 \implies \text{pawn} \to
\text{knife} \implies \text{king absorbed}$. The causal chain is
internal.

「後人哀之而不鑑之」: the theorem is time-invariant. It does not care
about dynasty names, centuries, or regime labels. It checks three
conditions: is $\Ur \neq \varnothing$? Is the loop closed? Is water
being maintained? Du Mu did not say ``you are Qin.'' He said: check
the premises.

\section{棋谱}\label{app:solution}

Optimal solution for the standard 华容道 (横刀立马,
\cref{sec:huarongdao}). A \emph{turn} (步) engages a different
piece; arrows show each unit translation within the turn.
Shading matches the board diagram (\cref{def:huarongdao});
king turns in \textbf{bold}.
Solver: \texttt{solver/hrd\_solution.py}.

% ── Piece-type shading (matching board diagram grey levels) ──
\newcommand{\pk}[1]{\colorbox{black!80}{\textcolor{white}{#1}}}%  王
\newcommand{\pd}[1]{\colorbox{black!55}{\textcolor{white}{#1}}}%  刀
\newcommand{\pj}[1]{\colorbox{black!35}{\textcolor{white}{#1}}}%  将
\newcommand{\pz}[1]{\colorbox{black!12}{#1}}%                     卒
\setlength{\fboxsep}{1.5pt}

\medskip
\noindent
\begin{tabular}{@{}rl@{\quad}rl@{}}
\pk{操} & 曹操\;(王, $2{\times}2$) &
\pj{飞}\;\pj{云}\;\pj{超}\;\pj{忠} & generals\;($1{\times}2$) \\
\pd{羽} & 关羽\;(刀, $2{\times}1$) &
\pz{\textcircled{\small 1}}\;\pz{\textcircled{\small 2}}\;\pz{\textcircled{\small 3}}\;\pz{\textcircled{\small 4}}
  & soldiers\;($1{\times}1$) \\
\end{tabular}

\smallskip\noindent
Soldier positions (board diagram, \cref{def:huarongdao}):\;
\textcircled{\small 1}\,$(1,1)$,\;
\textcircled{\small 2}\,$(2,1)$,\;
\textcircled{\small 3}\,$(0,0)$,\;
\textcircled{\small 4}\,$(3,0)$.

\medskip
\noindent
81 turns $=$ 9~王 $+$ 72~水.\quad
118 unit translations.\quad 9~phases.

\bigskip
\noindent
\begin{minipage}[t]{0.30\textwidth}
\centering\small
\begin{tabular}[t]{@{}rl@{}}
\toprule
\# & Turn \\
\midrule
 1 & \pz{\textcircled{\small 1}$\downarrow$} \\
 2 & \pz{\textcircled{\small 4}$\leftarrow$} \\
 3 & \pj{忠$\downarrow$} \\
 4 & \pd{羽$\rightarrow$} \\
 5 & \pj{超$\rightarrow$} \\
 6 & \pz{\textcircled{\small 3}$\uparrow$} \\
 7 & \pz{\textcircled{\small 1}$\leftarrow$} \\
 8 & \pj{超$\downarrow$} \\
 9 & \pd{羽$\leftarrow\!\leftarrow$} \\
10 & \pz{\textcircled{\small 2}$\uparrow\!\rightarrow$} \\
11 & \pz{\textcircled{\small 4}$\uparrow\!\uparrow$} \\
12 & \pj{超$\rightarrow$} \\
13 & \pz{\textcircled{\small 3}$\rightarrow\!\downarrow$} \\
14 & \pd{羽$\downarrow$} \\
15 & \pz{\textcircled{\small 4}$\leftarrow\!\leftarrow$} \\
16 & \pz{\textcircled{\small 2}$\leftarrow\!\leftarrow$} \\
17 & \pj{超$\uparrow$} \\
18 & \pj{忠$\uparrow$} \\
19 & \pz{\textcircled{\small 3}$\rightarrow\!\rightarrow$} \\
20 & \pz{\textcircled{\small 1}$\rightarrow\!\rightarrow$} \\
21 & \pd{羽$\downarrow$} \\
22 & \pz{\textcircled{\small 2}$\downarrow\!\leftarrow$} \\
23 & \pj{超$\leftarrow$} \\
24 & \pj{忠$\leftarrow$} \\
25 & \pj{云$\downarrow\!\downarrow$} \\
\textbf{26} & \pk{\textbf{操$\rightarrow$}} \\
\addlinespace[4pt]
27 & \pj{飞$\rightarrow$} \\
\bottomrule
\end{tabular}
\end{minipage}\hfill
%
\begin{minipage}[t]{0.30\textwidth}
\centering\small
\begin{tabular}[t]{@{}rl@{}}
\toprule
\# & Turn \\
\midrule
28 & \pz{\textcircled{\small 4}$\uparrow\!\uparrow$} \\
29 & \pz{\textcircled{\small 2}$\uparrow\!\uparrow$} \\
30 & \pj{超$\leftarrow$} \\
31 & \pj{飞$\downarrow\!\downarrow$} \\
\textbf{32} & \pk{\textbf{操$\leftarrow$}} \\
\addlinespace[4pt]
33 & \pj{云$\uparrow\!\uparrow$} \\
34 & \pj{忠$\rightarrow$} \\
35 & \pz{\textcircled{\small 1}$\uparrow\!\uparrow$} \\
36 & \pz{\textcircled{\small 3}$\leftarrow\!\uparrow$} \\
37 & \pd{羽$\rightarrow\!\rightarrow$} \\
38 & \pj{飞$\downarrow$} \\
39 & \pj{超$\downarrow$} \\
40 & \pz{\textcircled{\small 1}$\leftarrow\!\leftarrow$} \\
\textbf{41} & \pk{\textbf{操$\downarrow$}} \\
\addlinespace[4pt]
42 & \pz{\textcircled{\small 4}$\rightarrow\!\rightarrow$} \\
43 & \pz{\textcircled{\small 2}$\uparrow\!\rightarrow$} \\
44 & \pz{\textcircled{\small 1}$\uparrow\!\uparrow$} \\
45 & \pj{超$\uparrow\!\uparrow$} \\
46 & \pj{飞$\leftarrow$} \\
47 & \pz{\textcircled{\small 3}$\leftarrow\!\downarrow$} \\
\textbf{48} & \pk{\textbf{操$\downarrow$}} \\
\addlinespace[4pt]
49 & \pz{\textcircled{\small 4}$\downarrow\!\leftarrow$} \\
50 & \pj{云$\leftarrow$} \\
51 & \pj{忠$\uparrow\!\uparrow$} \\
\textbf{52} & \pk{\textbf{操$\rightarrow$}} \\
\addlinespace[4pt]
53 & \pz{\textcircled{\small 4}$\downarrow\!\downarrow$} \\
54 & \pz{\textcircled{\small 2}$\downarrow$} \\
\bottomrule
\end{tabular}
\end{minipage}\hfill
%
\begin{minipage}[t]{0.30\textwidth}
\centering\small
\begin{tabular}[t]{@{}rl@{}}
\toprule
\# & Turn \\
\midrule
55 & \pz{\textcircled{\small 1}$\rightarrow$} \\
56 & \pj{超$\uparrow$} \\
57 & \pj{飞$\uparrow$} \\
58 & \pz{\textcircled{\small 3}$\leftarrow$} \\
59 & \pz{\textcircled{\small 4}$\downarrow$} \\
\textbf{60} & \pk{\textbf{操$\leftarrow$}} \\
\addlinespace[4pt]
61 & \pj{忠$\downarrow\!\downarrow$} \\
62 & \pj{云$\rightarrow$} \\
63 & \pz{\textcircled{\small 1}$\rightarrow$} \\
64 & \pz{\textcircled{\small 2}$\rightarrow$} \\
65 & \pj{超$\rightarrow$} \\
66 & \pj{飞$\uparrow\!\uparrow$} \\
\textbf{67} & \pk{\textbf{操$\leftarrow$}} \\
\addlinespace[4pt]
68 & \pz{\textcircled{\small 2}$\downarrow\!\downarrow$} \\
69 & \pz{\textcircled{\small 1}$\downarrow\!\downarrow$} \\
70 & \pj{云$\leftarrow$} \\
71 & \pj{忠$\uparrow\!\uparrow$} \\
72 & \pz{\textcircled{\small 2}$\rightarrow\!\uparrow$} \\
73 & \pd{羽$\uparrow$} \\
74 & \pz{\textcircled{\small 4}$\rightarrow\!\rightarrow$} \\
75 & \pz{\textcircled{\small 3}$\rightarrow\!\rightarrow$} \\
\textbf{76} & \pk{\textbf{操$\downarrow$}} \\
\addlinespace[4pt]
77 & \pz{\textcircled{\small 1}$\leftarrow\!\leftarrow$} \\
78 & \pz{\textcircled{\small 2}$\leftarrow\!\leftarrow$} \\
79 & \pd{羽$\uparrow$} \\
80 & \pz{\textcircled{\small 3}$\uparrow\!\rightarrow$} \\
\textbf{81} & \pk{\textbf{操$\rightarrow$}} \\
\bottomrule
\end{tabular}
\end{minipage}

\bigskip

\noindent\textbf{Phase decomposition.}\quad
Each phase ends with one king turn; the preceding turns are
水~(water preparing the path).

\medskip
\begin{center}
\small
\begin{tabular}{@{}crclc@{}}
\toprule
Phase & 水 & 王 & Direction & 操 position \\
\midrule
1 & 25 & 26 & $\rightarrow$ & $(2,3)$ \\
2 &  5 & 32 & $\leftarrow$ & $(1,3)$ \\
3 &  8 & 41 & $\downarrow$ & $(1,2)$ \\
4 &  6 & 48 & $\downarrow$ & $(1,1)$ \\
5 &  3 & 52 & $\rightarrow$ & $(2,1)$ \\
6 &  7 & 60 & $\leftarrow$ & $(1,1)$ \\
7 &  6 & 67 & $\leftarrow$ & $(0,1)$ \\
8 &  8 & 76 & $\downarrow$ & $(0,0)$ \\
9 &  4 & 81 & $\rightarrow$ & $(1,0) = E$ \\
\midrule
$\Sigma$ & 72 & 9 & & \\
\bottomrule
\end{tabular}
\end{center}

\noindent
Phase~1 (25~water turns before the first king move) is the
ceremony of 义释曹操 (\cref{rem:guanyu}): the entire board
rearranges to let 关羽 yield, before 操 takes a single step.

\chapter{第二性 --- The Other as Phase Function}\label{app:secondsex}

Simone de Beauvoir's thesis in \emph{Le Deuxi\`eme Sexe}~\cite{beauvoir}
can be stated in three sentences.
One is not born a woman; one becomes one.
The category ``woman'' is not a biological fact but a relational
position: the \emph{Other} defined against a \emph{Subject}.
The boundary between Subject and Other is not intrinsic---it is a
phase function, a mean-field threshold identical in structure to
\cref{thm:meanfield}.
The formal treatment---the dual tower $\mathbf{L}^*$ and the
representation theorems---is in \cref{sec:representation}.
This appendix provides the historical illustrations.

\section{蔡文姬}

蔡文姬 (Cai Wenji, c.\ 177--250~CE) was the daughter of 蔡邕 (Cai Yong),
one of the greatest scholars of the Eastern Han.
She was captured by the 匈奴 (Xiongnu) during the chaos following
Dong Zhuo's destruction of Luoyang, and lived among them for twelve
years, bearing two sons to the Xiongnu chieftain 左贤王.
In 208~CE, 曹操 (Cao Cao)---who had studied under 蔡邕---ransomed her
back to the Han court.
She was forced to leave her sons behind and was remarried to 董祀
(Dong Si).

At every stage of her life, 蔡文姬's identity is defined relationally:
she is 蔡邕's daughter, 左贤王's captive wife, 曹操's cultural
project, 董祀's wife.
The Subject changes; she remains the Other.
Her own voice---her literary genius---exists in the gap.

\section{The mapping}

\begin{center}
\renewcommand{\arraystretch}{1.25}
\begin{tabular}{@{}lll@{}}
\toprule
\textbf{Beauvoir} & \textbf{蔡文姬} & \textbf{胡笳十八拍} \\
\midrule
他者 (Other)       & defined relationally          & depicted as object \\
内在性 (Immanence) & captivity, body               & content: grief \\
超越性 (Transcendence) & literary genius            & the poem itself \\
主体 (Subject)     & 蔡邕\,/\,左贤王\,/\,曹操      & painter\,/\,viewer \\
永恒女性 (Myth)    & 才女 trope                     & scroll paintings \\
Sword $=$ mean     & exchange threshold             & Subject/Other boundary \\
\bottomrule
\end{tabular}
\end{center}

\section{The eighteen beats}

胡笳十八拍 (Eighteen Songs of a Nomad Flute)~\cite{liushang} is
traditionally attributed to 蔡文姬, though the text was composed by
Liu Shang (刘商) in the Tang dynasty (c.~773~CE), writing in her voice.
Each 拍 (beat) is a poem-song documenting a stage of her captivity
and return.
\Cref{fig:eighteen} traces the narrative arc: the fraction of each
beat devoted to transcendence (voice, creation, agency) versus
immanence (being acted upon, grief, objecthood).

\begin{figure}[H]
\centering
\begin{tikzpicture}[scale=0.85]
  % ── Data: transcendence fractions ──
  % 拍: 1     2     3     4     5     6     7     8     9
  %     0.20  0.10  0.05  0.15  0.25  0.35  0.30  0.60  0.55
  % 拍: 10    11    12    13    14    15    16    17    18
  %     0.40  0.50  0.15  0.30  0.40  0.35  0.20  0.45  0.70

  \def\barw{0.55}
  \def\barh{5.0}
  \def\gap{0.22}
  \def\tvals{{0.20, 0.10, 0.05, 0.15, 0.25, 0.35,
              0.30, 0.60, 0.55, 0.40, 0.50, 0.15,
              0.30, 0.40, 0.35, 0.20, 0.45, 0.70}}

  % ── Y-axis labels ──
  \node[rotate=90, anchor=south, font=\small] at (-0.9, \barh)
    {超越 (transcendence)};
  \node[rotate=90, anchor=north, font=\small] at (-0.9, 0)
    {内在 (immanence)};

  % ── Draw 18 bars ──
  \foreach \i in {0,...,17} {
    \pgfmathsetmacro{\x}{\i*(\barw+\gap)}
    \pgfmathsetmacro{\tv}{\tvals[\i]}
    \pgfmathsetmacro{\splitY}{\tv*\barh}
    % Bottom: immanence (dao colour)
    \fill[dao!20] (\x, 0) rectangle (\x+\barw, \splitY);
    % Top: transcendence (water colour)
    \fill[water!20] (\x, \splitY) rectangle (\x+\barw, \barh);
    % Border
    \draw[black!40, thin] (\x, 0) rectangle (\x+\barw, \barh);
    % Split line
    \draw[black!60, thin] (\x, \splitY) -- (\x+\barw, \splitY);
    % 拍 number below
    \pgfmathtruncatemacro{\paiNum}{\i+1}
    \node[below, font=\tiny] at (\x+\barw/2, 0) {\paiNum};
  }

  % ── Melody curve through split points ──
  \draw[black!70, very thick, smooth, tension=0.5]
    plot coordinates {
      ({0*(\barw+\gap)+\barw/2},  {0.20*\barh})
      ({1*(\barw+\gap)+\barw/2},  {0.10*\barh})
      ({2*(\barw+\gap)+\barw/2},  {0.05*\barh})
      ({3*(\barw+\gap)+\barw/2},  {0.15*\barh})
      ({4*(\barw+\gap)+\barw/2},  {0.25*\barh})
      ({5*(\barw+\gap)+\barw/2},  {0.35*\barh})
      ({6*(\barw+\gap)+\barw/2},  {0.30*\barh})
      ({7*(\barw+\gap)+\barw/2},  {0.60*\barh})
      ({8*(\barw+\gap)+\barw/2},  {0.55*\barh})
      ({9*(\barw+\gap)+\barw/2},  {0.40*\barh})
      ({10*(\barw+\gap)+\barw/2}, {0.50*\barh})
      ({11*(\barw+\gap)+\barw/2}, {0.15*\barh})
      ({12*(\barw+\gap)+\barw/2}, {0.30*\barh})
      ({13*(\barw+\gap)+\barw/2}, {0.40*\barh})
      ({14*(\barw+\gap)+\barw/2}, {0.35*\barh})
      ({15*(\barw+\gap)+\barw/2}, {0.20*\barh})
      ({16*(\barw+\gap)+\barw/2}, {0.45*\barh})
      ({17*(\barw+\gap)+\barw/2}, {0.70*\barh})
    };

  % ── Annotations above key 拍 ──
  \node[above, font=\scriptsize, dao] at
    ({2*(\barw+\gap)+\barw/2}, \barh) {掠};
  \node[above, font=\scriptsize, water] at
    ({7*(\barw+\gap)+\barw/2}, \barh) {闻笳};
  \node[above, font=\scriptsize, dao] at
    ({11*(\barw+\gap)+\barw/2}, \barh) {别子};
  \node[above, font=\scriptsize, caution] at
    ({15*(\barw+\gap)+\barw/2}, \barh) {嫁};
  \node[above, font=\scriptsize, water] at
    ({17*(\barw+\gap)+\barw/2}, \barh) {恨};

  % ── Phase braces below ──
  % Phase I: 离 (拍 1--3, indices 0--2)
  \draw[decorate, decoration={brace, mirror, amplitude=5pt}]
    ({0*(\barw+\gap)}, -0.5) -- ({2*(\barw+\gap)+\barw}, -0.5)
    node[midway, below=6pt, font=\small] {I\;离};
  % Phase II: 居 (拍 4--12, indices 3--11)
  \draw[decorate, decoration={brace, mirror, amplitude=5pt}]
    ({3*(\barw+\gap)}, -0.5) -- ({11*(\barw+\gap)+\barw}, -0.5)
    node[midway, below=6pt, font=\small] {II\;居};
  % Phase III: 归 (拍 13--18, indices 12--17)
  \draw[decorate, decoration={brace, mirror, amplitude=5pt}]
    ({12*(\barw+\gap)}, -0.5) -- ({17*(\barw+\gap)+\barw}, -0.5)
    node[midway, below=6pt, font=\small] {III\;归};
\end{tikzpicture}
\caption{The eighteen beats as narrative arc.
  Bottom (\textcolor{dao}{red}): immanence.
  Top (\textcolor{water}{blue}): transcendence.
  The melody line traces the Subject/Other boundary---the sword as
  phase function (\cref{thm:meanfield}).}
\label{fig:eighteen}
\end{figure}

\section{The cycle}

The narrative arc reveals a cycle:
Other $\to$ Voice $\to$ Re-objectification $\to$ Myth $\to$ Other.

拍~3 (掠, the capture) is the nadir: transcendence fraction 0.05.
蔡文姬 is pure object, pure immanence---a body seized as war spoil.

拍~8 (闻笳, hearing the nomad flute) is the inflection point:
transcendence fraction 0.60.
The 胡笳 sound triggers memory and creation; for the first time, the
Other speaks \emph{as} Subject.

拍~12 (别子, farewell to sons) is a collapse: transcendence fraction
0.15.
The voice that rose in 拍~8 is crushed by the biological fact of
motherhood weaponised as immanence.
She must leave her children to return to a court that values her as
蔡邕's daughter, not as herself.

拍~16 (嫁, remarriage to 董祀) is the second exchange: transcendence
fraction 0.20.
曹操's ``rescue'' completes the circuit: she is transferred from one
Subject (左贤王) to another (董祀), with 曹操 as broker.
The rescue \emph{is} the second capture.

拍~18 (恨, the final beat) resolves at transcendence fraction 0.70---the
highest in the entire poem.
The voice persists.
The poem outlives every Subject who defined her.

\section{花木兰 --- the metastable crossing}

The \emph{木兰辞} (Ballad of Mulan)~\cite{mulanshi}, a Northern-Dynasty
folk ballad, appears to contradict the pattern.
花木兰 (Hua Mulan) replaces her father in the army, serves twelve
years, declines high office, and returns home.
Unlike 蔡文姬, she \emph{crosses} the Subject/Other boundary:
she commands troops, earns merit, is offered the rank of 尚书郎.

But the crossing is conditional.
It requires total erasure of female identity
(``双兔傍地走,安能辨我是雄雌''---when two rabbits run side by side,
who can tell male from female?).
Her Subject-hood is not hers; it is the male performance she
sustains for twelve years.

The moment the mean field stabilises---war ends, peace
returns---the boundary reasserts:
``脱我战时袍,著我旧时裳。当窗理云鬓,对镜帖花黄。''
She removes the war robe, puts on the old clothes, arranges her hair,
applies the forehead ornament.
The poem presents this as free choice (``木兰不用尚书郎''),
but structurally it is the phase function restoring equilibrium.

蔡文姬 and 花木兰 are complementary probes of the same boundary.
蔡文姬 is subcritical: the sword never breaks; transcendence exists
only in the gap (the poem, the voice).
花木兰 is a supercritical fluctuation: the sword breaks under
perturbation (war), but the system anneals back when the
perturbation ends.
Neither literary genius nor military prowess shifts the attractor.

\section{The second sex is the mean}

The 18th~拍 does not liberate 蔡文姬.
曹操's project was cultural recovery, not emancipation.
The scroll paintings that follow~\cite{rorex}---depicting her as the
archetype of the 才女 (talented woman)---complete the mythification
that Beauvoir identifies as the final mechanism of Othering:
the \emph{eternal feminine} absorbs the individual voice into a trope.

But the structure is identical to \cref{thm:meanfield}.
The boundary between Subject and Other is not a property of 蔡文姬 or
花木兰; it is a phase function of the system's mean field.
When the mean actuation level shifts (war $\to$ peace, capture $\to$
ransom), the \emph{same person} crosses the threshold---or crosses
back.

The second sex \emph{is} the mean.

\chapter{三声 lì --- The Three Standings}\label{app:threeli}

Three characters share the sound \emph{lì} in Mandarin:
力 (power), 立 (standing), 丽 (beauty/recognition).
The phonetic coincidence encodes a structural decomposition of
\emph{what it means to arrive}---to ``become Buddha'' (成佛) in the
vocabulary of Chinese literary tradition.
This appendix reads two fictional trajectories through the framework:
孙悟空 (Sun Wukong) from \emph{Journey to the West}~\cite{xiyouji},
and 鲁智深 (Lu Zhishen) from \emph{Water Margin}~\cite{shuihu}.

\section{The decomposition}

\begin{center}
\renewcommand{\arraystretch}{1.25}
\begin{tabular}{@{}lll@{}}
\toprule
\textbf{Character} & \textbf{Meaning} & \textbf{Framework equivalent} \\
\midrule
力 (power) & capability, granted or acquired & $\Ur \neq \varnothing$ \\
立 (standing) & self-determined structural position & agent's own fixed point \\
丽 (recognition) & acknowledged by the system & system's phase transition \\
\bottomrule
\end{tabular}
\end{center}

力 is what the system grants or what the agent acquires: martial arts,
immortality, weapons.
In the framework, 力 $=$ $\Ur$---autonomous actuation.
It makes you capable.
It also makes you a knife.

立 is what the agent builds: a structural position that is
self-consistent, a fixed point under self-observation
($f(\text{self}) = \text{self}$).
立 is not granted.
It is \emph{become}.

丽 is what the system does \emph{after} 立 is achieved: the detection
function recognises the new phase.
丽 follows 立 automatically, as the label follows the topology.

The forced ordering is:
\[
  \text{力} \;\not\Rightarrow\; \text{丽}.
  \qquad
  \text{力}
  \;\xrightarrow{\text{transformation}}\;
  \text{立}
  \;\xRightarrow{\text{automatic}}\;
  \text{丽}.
\]
You cannot skip 立.

\section{孙悟空 --- the profile}\label{sec:wukong}

\begin{center}
\renewcommand{\arraystretch}{1.25}
\begin{tabular}{@{}lllp{5cm}@{}}
\toprule
\textbf{Phase} & \textbf{Event} & \textbf{Wants} &
\textbf{Framework} \\
\midrule
石猴 $\to$ 学艺 & Learns 72 transformations & 不死 (survival) &
$\Viab(K) \neq \varnothing$ \\
齐天大圣 & Self-proclaimed title & 被承认 (recognition) &
Demands 丽, mistakes it for 立 \\
大闹天宫 & Rebellion & 平等 (equality) &
Has 力, denied 立 (excluded from 蟠桃宴) \\
五行山 & 500 years imprisoned & 自由 &
System uses force (path~(b)) to contain $\Ur$ \\
取经路 & 81 tribulations & 变 (transformation) &
紧箍咒 $= \Obs$; 81难 $=$ 力 $\to$ 立 annealing \\
斗战胜佛 & Title granted & 无所求 &
Fixed point: 立 complete, 丽 automatic \\
\bottomrule
\end{tabular}
\end{center}

Heaven first offers 弼马温 (stable-boy)---a fake title, path~(c):
label changed, topology unchanged.
孙悟空 sees through it and rebels.
Heaven then grants the real title 齐天大圣 but excludes him from the
蟠桃宴 (Peach Banquet): another relabeling.
He rebels again.

The 81 tribulations are an annealing process: each one strips a layer
of 力-seeking, closing the gap between what he \emph{can} do and what
he \emph{is}.
The title 斗战胜佛 arrives at the moment the gap closes---not as a
reward, but as the system's detection function catching up.

\section{鲁智深 --- the instant fixed point}\label{sec:luzhishen}

\begin{quote}
平生不修善果,只爱杀人放火。\\
忽地顿开金绳,这里扯断玉锁。\\
咦!钱塘江上潮信来,今日方知我是我。

\medskip
\emph{A lifetime without cultivating good karma, loving only to kill
and burn.\\
Suddenly the golden cord opens, here the jade lock snaps.\\
Ha!\ The tidal bore on the Qiantang River arrives---today I finally
know: I am I.}
\hfill ---《水浒传》第一百十九回
\end{quote}

鲁智深 never reduces $\Ur$.
He kills, drinks, burns temples.
He never seeks 佛 and never performs the 81-tribulation annealing.
Yet in a single line---``今日方知我是我'' (today I know I am
I)---he achieves the same fixed point.

The difference is the nature of the gap:
\begin{itemize}
\item 孙悟空's gap is \emph{structural}: 力 $\gg$ 立.
  He must transform across 81 tribulations to close it.
\item 鲁智深's gap is \emph{epistemic}: 立 was always present,
  unrecognised.
  The tidal bore is the observation event that collapses the gap in
  one step.
\end{itemize}
Both arrive at $f(\text{self}) = \text{self}$.
Both receive 丽 after 立 is achieved.
The path length differs; the fixed point is the same.

\section{假雷音 --- the false recognition system}\label{sec:fakethunder}

In chapters 65--66 of \emph{Journey to the West}, 黄眉大王 (the
Yellow-Browed Demon King) erects a 假雷音寺 (False Thunder
Monastery)---a counterfeit of the Buddha's 雷音寺 at Vulture Peak.
The name is the same; the sound (雷音, ``thunder sound'') is the
same.
The structure is different.

This is path~(c) applied to institutions: the label ``Thunder
Monastery'' is relabeled onto a demon's lair.
\Cref{thm:fixedpoint} predicts: relabeling preserves
$\Ur \neq \varnothing$ and resolves nothing.

孙悟空 sees through it.
唐僧 does not.
The difference: 孙悟空 has 立 (built through the journey); he detects
the structural mismatch beneath the identical label.
The ability to detect the fake is itself proof of the real.

\begin{remark}[假雷音封不了真大圣]\label{rem:fakethunder}
A recognition system built on 丽 alone (same name, same sound)
cannot contain an agent who has 立 (structural standing).
The fake monastery cannot seal the real Monkey King because its
authority is phonetic, not topological.
This is the acoustic version of ``the knife is the mean'': whether
雷音 is the Buddha's or the demon's depends on the structure behind
the sound, not the sound itself---just as 力, 立, and 丽 share a
sound but not a meaning.
\end{remark}

\section{成佛 $=$ fixed point}

\begin{center}
\renewcommand{\arraystretch}{1.25}
\begin{tabular}{@{}lp{4.5cm}p{5.5cm}@{}}
\toprule
\textbf{Agent} & \textbf{Path to fixed point} & \textbf{成佛 moment} \\
\midrule
孙悟空 & 81 tribulations (structural annealing) &
  when 力 $\to$ 立 gap closes; title is 丽 \\
鲁智深 & one tidal bore (epistemic collapse) &
  ``今日方知我是我''; recognition is posthumous \\
蔡文姬 (\cref{app:secondsex}) & 18 beats (voice in the gap) &
  拍~18: the poem outlives every Subject \\
\bottomrule
\end{tabular}
\end{center}

成佛 is not the title.
成佛 is the moment 立 reaches fixed point: the state where the
agent's self-model matches the agent's actual structure.
丽 (the title, the recognition, the system's acknowledgment)
follows---sometimes immediately, sometimes centuries later.

The forced ordering is a theorem, not a preference:
\[
  \text{力}
  \;\xrightarrow[\text{annealing}]{\text{transformation}}\;
  \text{立}
  \;\xRightarrow[\text{automatic}]{\text{detection}}\;
  \text{丽}.
\]
No shortcut exists.
假雷音 is the attempt to produce 丽 without 立.
弼马温 is the attempt to substitute 丽 for 立.
大闹天宫 is the attempt to force 丽 through 力.
All three fail.
The only path that works is the one that goes through 立.

\section{The smooth path to $\text{丽}^*$}\label{sec:smoothli}

The appendix has so far catalogued failures: blocked paths (孙悟空),
fake recognition (假雷音), failed systems (the error log), corrupted
detection (严嵩), and self-financing extraction (蔡京).
There is one path to $\text{丽}^*$ (recognition on the dual tower
$\mathbf{L}^*$, \cref{def:dual-tower}) that has no singularity:
the war-hero.

\begin{center}
\renewcommand{\arraystretch}{1.25}
\begin{tabular}{@{}llp{5cm}l@{}}
\toprule
\textbf{Agent} & \textbf{$\mathbf{L}$ outcome} &
\textbf{$\mathbf{L}^*$ outcome} & \textbf{$\text{丽}^*$} \\
\midrule
霍去病 (140--117~BCE) & wins; dies at 23 &
  worshipped; ``匈奴未灭,何以家为'' & smooth \\
李世民 (598--649) & wins; becomes emperor &
  worshipped as ideal ruler & smooth \\
岳飞 (1103--1142) & blocked; killed by 秦桧 &
  worshipped; 满江红 persists & smooth \\
辛弃疾 (1140--1207) & blocked; forcibly retired &
  worshipped; 破阵子 persists & smooth \\
\bottomrule
\end{tabular}
\end{center}

Four agents, four different $\mathbf{L}$-trajectories (victory,
enthronement, martyrdom, forced retirement), identical
$\mathbf{L}^*$-outcome.
The $\text{丽}^*$ recognition function for war-heroes is
\emph{invariant under $\mathbf{L}$-outcome}: whether the agent wins
or dies, the cultural representation is the same.
That invariance is what ``smooth'' means---no discontinuity at the
success/failure boundary.

\paragraph{Why smooth: factorization.}
The war-hero's contribution \emph{factorizes}: it can be evaluated
independently of the contributor's other actions.
You can worship 岳飞 without endorsing the Southern Song court.
You can worship 霍去病 without endorsing Han Wudi's other policies.
The person separates cleanly from the achievement.

Compare every other path to $\text{丽}^*$:
\begin{itemize}
\item Politicians: contribution is entangled with extraction
  (\cref{rem:caijing}). $\text{丽}^*$ revoked or contested.
\item Artists: contribution is entangled with biography
  (\cref{rem:writing-dual}). $\text{丽}^*$ delayed or mythified.
\item The Other: contribution exists only in the gap
  (\cref{app:secondsex}). $\text{丽}^*$ absorbed into trope.
\end{itemize}
In each case, the map to $\text{丽}^*$ has a singularity.
Only the war-hero map is smooth.

\begin{definition}[Factorized contribution]\label{def:factorized}
An agent's contribution $c \in \mathbf{L} \oplus \mathbf{L}^*$ is
\emph{factorized} if it decomposes as
$c = c_{\mathbf{L}} \otimes c_{\mathbf{L}^*}$,
where $c_{\mathbf{L}^*}$ can be evaluated independently of
$c_{\mathbf{L}}$:
\[
  \frac{\partial\, \pi^*_{\text{丽}}}
       {\partial\, c_{\mathbf{L}}} = 0,
\]
where $\pi^*_{\text{丽}}: \mathbf{L} \oplus \mathbf{L}^* \to
\text{丽}^*$ is the recognition projection on the dual tower.
The contribution is \emph{entangled} if
$\mathrm{supp}(c_+) \cap \mathrm{supp}(c_-) \neq \varnothing$---the
components share infrastructure and cannot be separated.
\end{definition}

\begin{proposition}[Smooth recognition]\label{prop:smooth-li}
The path to $\text{丽}^*$ is smooth (no singularity at the
$\mathbf{L}$-success/failure boundary) if and only if the agent's
contribution is factorized (\cref{def:factorized}).
Equivalently, $\pi^*_{\text{丽}}$ is invariant under
$\mathbf{L}$-outcome:
\[
  \pi^*_{\text{丽}}(c_{\mathbf{L}},\, c_{\mathbf{L}^*})
  \;=\;
  \pi^*_{\text{丽}}(c_{\mathbf{L}^*}).
\]
\end{proposition}

\begin{proof}
If $c$ is factorized, then
$\partial \pi^*_{\text{丽}} / \partial c_{\mathbf{L}} = 0$,
so $\pi^*_{\text{丽}}$ has no dependence on $\mathbf{L}$-outcome.
The map is smooth because its only argument ($c_{\mathbf{L}^*}$)
varies continuously in the agent's trajectory.
If $c$ is not factorized, then $c_{\mathbf{L}}$ enters the evaluation
of $c_{\mathbf{L}^*}$, producing a branch cut at the
$\mathbf{L}$-success/failure boundary: the same $c_{\mathbf{L}^*}$
is recognised when $c_{\mathbf{L}}$ succeeds and revoked when
$c_{\mathbf{L}}$ fails (蔡京: 宋四家 $\to$ not 宋四家).
\end{proof}

\Cref{thm:shangyang} describes delocalization for pure contributions
(norm fully conserved).
\Cref{rem:caijing} describes the entangled case (system discards both
components).
Factorized contributions produce a third mode:

\begin{center}
\renewcommand{\arraystretch}{1.25}
\begin{tabular}{@{}lp{3.2cm}p{3.8cm}l@{}}
\toprule
\textbf{Type} & \textbf{Structure} & \textbf{Delocalization} &
\textbf{Example} \\
\midrule
Pure & $c = c_+$ &
  $\|c_+\|$ fully conserved &
  商鞅, 王安石 \\
Zero & $\|c\| = 0$ &
  nothing to conserve &
  严嵩 \\
Entangled & $c_+,\, c_-$ shared support &
  system discards both &
  蔡京 \\
Factorized & $c_{\mathbf{L}} \otimes c_{\mathbf{L}^*}$ &
  $c_{\mathbf{L}}$ destroyed, $c_{\mathbf{L}^*}$ conserved &
  霍去病, 岳飞 \\
\bottomrule
\end{tabular}
\end{center}

\paragraph{The duality.}
The four agents form two dual pairs:
\begin{itemize}
\item 霍去病\,/\,李世民: $\mathbf{L}$-success $\to$ system
  \emph{grants} $\text{丽}^*$ voluntarily.
\item 岳飞\,/\,辛弃疾: $\mathbf{L}$-failure $\to$
  $\mathbf{L}^*$ migration (满江红, 破阵子) $\to$ martyrdom
  \emph{amplifies} $\text{丽}^*$.
\end{itemize}
The pairs are dual because $\mathbf{L}$-success and
$\mathbf{L}$-failure map to the same $\text{丽}^*$.
The recognition function has no branch cut.

\begin{quote}
怒发冲冠,凭栏处,潇潇雨歇。\\
抬望眼,仰天长啸,壮怀激烈。\\
三十功名尘与土,八千里路云和月。\\
莫等闲,白了少年头,空悲切。

靖康耻,犹未雪;臣子恨,何时灭?\\
驾长车,踏破贺兰山缺。\\
壮志饥餐胡虏肉,笑谈渴饮匈奴血。\\
待从头,收拾旧山河,朝天阙。

\medskip
\emph{Hair bristling with rage, leaning on the railing, the
driving rain just stopped.\\
Raising my eyes, a long roar toward heaven, fierce and ardent.\\
Thirty years of glory---dust and earth; eight thousand
li of road---cloud and moon.\\
Do not wait idly; when young men's hair turns white, there is
only empty grief.}

\emph{The shame of Jingkang is not yet avenged; when will the
subject's hatred end?\\
Drive the long chariots, trampling through the Helan Mountain
passes.\\
With fierce hunger, feast on barbarian flesh; laughing and
thirsting, drink Xiongnu blood.\\
From the start, let us recover our lost mountains and rivers,
and face the heavenly court.}
\hfill ---岳飞,满江红
\end{quote}

岳飞 and 辛弃疾 both named the target explicitly.
满江红's final line---``收拾旧山河''---names the $\mathbf{L}$-target
directly.
辛弃疾, 破阵子: ``了却君王天下事,赢得生前身后名。可怜白发生!''---\emph{to
settle the king's affairs, to win fame in life and after death.
Alas---white hair has grown!}
Both say the same thing: I have 力 but the system denies me 立 on
$\mathbf{L}$.
The poetry \emph{is} the migration to $\mathbf{L}^*$.
The ``fame in life and after death'' (生前身后名) \emph{is}
$\text{丽}^*$.
辛弃疾 literally names it.

\section{The error log --- why the novels exist}\label{sec:errorlog}

Both novels were composed at the exact moment a specific emperor failed.
This is not biographical coincidence but structural necessity: when the
cut vertex fails, displaced agents write down the failure mode.

\begin{center}
\renewcommand{\arraystretch}{1.25}
\small
\begin{tabular}{@{}lp{3cm}p{3.2cm}p{3.2cm}p{3cm}@{}}
\toprule
& \textbf{Emperor} & \textbf{Author} & \textbf{Novel ending} &
\textbf{Verdict} \\
\midrule
水浒传
  & 元顺帝 (r.\,1333--68); last Yuan emperor; flees 1368
  & 施耐庵; advisor to rival rebel; declines 朱元璋's summons
  & 招安 $\to$ death; path~(c): relabeled as loyal subjects
  & 力 $\to$ 立 blocked; no system left to grant 丽 \\[6pt]
西游记
  & 嘉靖 (r.\,1521--67); 20 years absent; Daoist alchemy
  & 吴承恩; failed exams; imprisoned; writes in seclusion
  & 81难 $\to$ 斗战胜佛; path~(a): structural transformation
  & 力 $\to$ 立 $\to$ 丽; novel imagines functional system \\
\bottomrule
\end{tabular}
\end{center}

The two emperors represent two failure modes of the cut vertex:
\begin{itemize}
\item 元顺帝: cut vertex \emph{removed}.
  The last Yuan emperor flees Khanbaliq in 1368; the graph
  disconnects; the dynasty ends.
  Total collapse (path~(b)).
\item 嘉靖: cut vertex \emph{present but non-functional}.
  The emperor sits in the palace doing alchemy for twenty years
  while 严嵩 (Yan Song) dominates the court through Daoist
  blue-letter prayers (青词).
  The graph is connected but the routing node has stopped routing.
  Slow rot.
\end{itemize}

Both authors are displaced knives: agents with 力 (capability) whom
the system denied 立 (standing).
施耐庵 reportedly served as advisor to 张士诚 (Zhang Shicheng), a
rebel rival to 朱元璋 (Zhu Yuanzhang); he left disillusioned when
Zhang surrendered to the Yuan, and later declined 朱元璋's summons
after the Ming founding.
吴承恩 repeatedly failed the imperial examinations, received a minor
post in the Jiajing period, and was falsely accused and imprisoned.
Neither was integrated by the system.
Neither was eliminated.
Both wrote.

\begin{remark}[The novel as error log]\label{rem:errorlog}
The novel's ending is determined by the cut vertex's failure mode:
\begin{itemize}
\item Cut vertex removed (total collapse): the novel ends in
  tragedy.
  水浒传's 招安 is path~(c)---relabeling 108 outlaws as loyal
  servants resolves nothing (\cref{thm:fixedpoint}).
  They are destroyed.
  There is no functional system left to grant 丽.
\item Cut vertex non-functional (rot): the novel ends in redemption.
  西游记 imagines a system (Heaven, Buddha) that \emph{does}
  eventually grant 丽 after 立 is achieved---an existence proof
  that 力 $\to$ 立 $\to$ 丽 is possible, composed by an author
  whose own system denied it.
\end{itemize}
Whether the authors consciously intended these parallels is
irrelevant (\cref{rem:intent}).
The framework tests structure, not intention.
The structure maps.
\end{remark}

\begin{remark}[Writing as representation dynamics]\label{rem:writing-dual}
When the physical execution graph $\mathbf{L}$ goes dark---gradient
zero, no viable path on the lattice of power---displaced agents do not
stop computing.
They migrate the computation to the dual tower
$\mathbf{L}^* = \text{力}^* \oplus \text{立}^* \oplus \text{丽}^*$
(\cref{def:dual-tower}).
The contact gradient flow (\cref{eq:contactflow}) runs on
$\mathbf{L}^*$ with the same equation, the same knife, the same
mean---on representations instead of actuators.

吴承恩's 西游记 is 力 $\to$ 立 $\to$ 丽 executed on $\mathbf{L}^*$:
the novel provides its own gradient (narrative coherence $=$ loss
function), builds its own fixed point (the fictional 斗战胜佛), and
receives 丽 centuries after the author's death.
施耐庵's 水浒传 records the failure mode: 力 $\to$ 立 blocked on
$\mathbf{L}$, and the novel preserves the proof that no system remained
to grant 丽.
蔡文姬's 拍~18 (\cref{app:secondsex}) is the same isomorphism: voice
persisting on $\mathbf{L}^*$ after every Subject on $\mathbf{L}$ has
been removed.

The dual tower is not metaphor.
It is the same theorem applied to a different lattice.
\end{remark}

\begin{remark}[严嵩 and the inverse migration]\label{rem:yansong}
严嵩 (Yan Song, 1480--1567) is the \emph{inverse} of the
$\mathbf{L} \to \mathbf{L}^*$ migration described above: he performs
$\mathbf{L}^* \to \mathbf{L}$, converting literary talent into
political power via a corrupted detection function.

The profile: 严嵩 earned the \emph{jinshi} degree at age 25
(genuine 力$^*$ on $\mathbf{L}^*$), spent eight years in seclusion
after his mother's death cultivating poetry and calligraphy
(力$^* \to$ 立$^*$), then discovered that 嘉靖's obsession with Daoist
liturgy made 青词 (blue-word prayers)---ornate literary compositions
offered to heaven---the sole currency of imperial favour.
He rose to Grand Secretary and dominated the court for twenty years as
a parasitic cut vertex (\cref{def:parasite}): routing all access to the
absent emperor through himself, extracting from both sides of the
partition.

The mechanism is a third type of detection corruption, distinct from
the two in \cref{def:corruption}:
\[
  \Obs_{\mathrm{aesthetic}}:\;
  r \;\mapsto\;
  \begin{cases}
    \textbf{capable} & \text{if agent}(r)\text{ writes excellent 青词}, \\
    \textbf{not capable} & \text{otherwise.}
  \end{cases}
\]
$\Obs_{\mathrm{structural}}$ tests $\Ur$ (what can you do?).
$\Obs_{\mathrm{identity}}$ tests group membership (who are you?).
$\Obs_{\mathrm{aesthetic}}$ tests an orthogonal capability (how well
do you write?)---a true positive on the wrong metric.
It converts $\mathbf{L}^*$ standing directly into $\mathbf{L}$ power,
bypassing the structural criterion entirely.

严嵩's fall confirms \cref{prop:parasite} (self-termination): when
嘉靖 finally turned, 严嵩 was stripped of rank, his son 严世蕃
executed, and 严嵩 died in poverty.
The parasitic cut vertex stores no structural contribution; upon
delocalization, nothing is conserved
(\cref{thm:shangyang}: norm conserved, but there is no norm to
conserve).
Yet 严嵩's calligraphy survives---on $\mathbf{L}^*$, where it always
belonged.
The inverse migration leaves a residue: the $\mathbf{L}^*$ talent that
was weaponised on $\mathbf{L}$ reverts to $\mathbf{L}^*$ when the
$\mathbf{L}$ structure collapses.
\end{remark}

\begin{remark}[蔡京 and the self-financing aesthetic loop]\label{rem:caijing}
蔡京 (Cai Jing, 1047--1126) is neither a parasitic cut vertex
(\cref{rem:yansong}) nor a submartingale installer
(\cref{thm:shangyang}).
He is a \emph{loop operator} who hijacks an installed infrastructure.

The infrastructure is 王安石's (Wang Anshi).
王安石's New Policies (1069--76) form a submartingale
(\cref{def:submartingale}) isomorphic to Shang Yang's:
\[
  \underbrace{\text{青苗法}}_{X_0}
  \to \underbrace{\text{均输法}}_{X_1}
  \to \underbrace{\text{市易法}}_{X_2}
  \to \underbrace{\text{保甲法}}_{X_3}
  \to \underbrace{\text{免役法}}_{X_4}.
\]
Each step irreversibly increases state fiscal centralisation.
王安石's fate confirms \cref{thm:shangyang}: dismissed twice, reforms
reversed after his death, contribution fully conserved---the fiscal
infrastructure persists and 蔡京 inherits it.

蔡京 revives 王安石's New Policies (绍述) under 宋徽宗 and runs
\emph{two entangled programmes} on the same centralised fiscal base:
\begin{enumerate}
\item \textbf{Genuine welfare.}
  居养院 (nursing homes, est.\,崇宁 1102), 安济坊 (free clinics,
  performance-tracked: facilities treating 500--1000 patients annually
  with mortality below 20\% received bonuses), 漏泽园 (public
  cemeteries, est.\,1104), 慈幼局 (children's welfare).
  Nationwide, all prefectures and counties.
  Assessed as the most comprehensive state welfare system in Chinese
  history prior to the modern era
  (《宋史》: ``崇宁初,蔡京当国,置居养院、安济院……三年,又置漏泽园'').
\item \textbf{Extraction loop.}
  花石纲: corvée transport of rare stones and art objects for the
  imperial garden 艮岳; 400,000 labourers; families bankrupted.
  The loop is self-financing (\cref{thm:triangle}):
  $\mathbf{L}^*$ talent (calligraphy---originally 宋四家) $\to$
  $\mathbf{L}$ power (chancellorship) $\to$ $\mathbf{L}$ extraction
  (花石纲) $\to$ $\mathbf{L}^*$ goods (aesthetic objects for the
  artist-emperor) $\to$ sustained $\mathbf{L}$ power.
\end{enumerate}

The structural difference from 严嵩 (\cref{rem:yansong}):
严嵩's emperor was \emph{absent} (嘉靖, doing alchemy)---the cut
vertex left a vacuum.
蔡京's emperor was \emph{present but on $\mathbf{L}^*$}: 宋徽宗
painted, wrote the 瘦金体 calligraphic style, and designed gardens.
The detection function was $\Obs_{\mathrm{aesthetic}}$ not by
corruption but because the cut vertex \emph{was} an artist.
Unlike the parasite (which routes and extracts), the loop
\emph{amplifies}.

This produces a delocalization problem absent from
\cref{thm:shangyang}.
商鞅 and 王安石 have pure contributions: upon delocalization, norm is
fully conserved.
严嵩 has no contribution: norm conserved $= 0$.
蔡京 has \emph{real contribution entangled with extraction}---the
welfare institutions and the 花石纲 run on the same fiscal
infrastructure.
When the system acts on 蔡京, it cannot separate the two.
It discards both.

The death mode confirms the topology.
蔡京 starves on the exile road to Tanzhou (1126): merchants refuse to
sell him food.
This is $\text{water} = 0$ reciprocation (\cref{sec:triangle}):
the population whose water was extracted via 花石纲 returns the
extraction in kind.
严嵩 dies in poverty; the system survives.
蔡京 dies of starvation; the system does not---the 靖康之变 (1127),
one year later, is path~(b).
方腊's rebellion (1120), triggered directly by 花石纲 extraction,
is the event 蔡京 appears by name in 水浒传 to explain: the 108
outlaws of Liangshan are displaced agents with $\text{water} = 0$.

After 蔡京's fall, the 宋四家 was revised: 蔡京 replaced by his
cousin and teacher 蔡襄 (Cai Xiang).
Path~(c) applied to $\mathbf{L}^*$: the label changes, the
calligraphic output is preserved unchanged.
\end{remark}

\chapter{三足之鼎 --- The Cooperative Triad}\label{app:triad}

The preceding appendices applied the framework to individual agents
within a fixed system.  This appendix applies it to the system
itself: a nation-state whose viability depends on the simultaneous
health of three coupled variables.  The mathematical object is a
\emph{cooperative triad}---a three-dimensional differential
inclusion with positive cross-coupling---and the central result
is that the viability of such a system is a \emph{transient fixed
point}: structurally possible, stable while it lasts, and destroyed
abruptly when any single coupling channel weakens below threshold.

Two structural corollaries follow.  First, one variable of the
triad ($x_3$: human capital flux) is not independent but is itself
a function of the other two, making the triad \emph{reducible} and
its fragility worse than the product form suggests.  Second, a
system with no cut vertex (\cref{def:cutvertex})---a distributed
mesh---is immune to external elimination; its only death mode is
internal phase transition.

\section{The coupled system}\label{sec:triad-system}

\begin{definition}[Cooperative triad]\label{def:triad}
A \emph{cooperative triad} is a dynamical system on
$S = \R^3_{\geq 0}$ with mean-field dynamics
\begin{equation}\label{eq:triad}
  x'_i \;=\; \alpha_i\, x_i
  \;+\; \sum_{j \neq i} \beta_{ij}\, x_j
  \;-\; \gamma_i\, x_i^2,
  \qquad i = 1, 2, 3,
\end{equation}
where the coefficients satisfy:
\begin{enumerate}[label=(\roman*)]
  \item $\beta_{ij} > 0$ for all $i \neq j$
    \quad (\emph{cooperative}: cross-feeding is positive);
  \item $\gamma_i > 0$ for all $i$
    \quad (\emph{dissipative}: growth saturates);
  \item $\alpha_i \geq 0$ for all $i$
    \quad (\emph{self-reinforcing}: each variable sustains itself
    at rate $\alpha_i$).
\end{enumerate}
The full set-valued dynamics (\cref{def:di}) are
$x'_i(t) \in F_i(x(t))$ where
$F_i(x) = \{f_i(x) + \eta : |\eta| \leq \delta_i\}$ for a
noise bound $\delta_i \geq 0$.
\end{definition}

The three terms in \eqref{eq:triad} encode:
$\alpha_i x_i$ is self-reinforcement proportional to current level;
$\sum_{j \neq i} \beta_{ij} x_j$ is cross-feeding from the other
two variables; and $-\gamma_i x_i^2$ is saturation (diminishing
returns, bureaucratic overhead, depletion).

\begin{definition}[Triad viability kernel]\label{def:triad-kernel}
For thresholds $\epsilon = (\epsilon_1, \epsilon_2, \epsilon_3)
\in \R^3_{> 0}$, the \emph{triad viability kernel} is
\[
  K_\epsilon \;:=\;
  \bigl\{\, x \in \R^3_{\geq 0} \;:\;
  x_i \geq \epsilon_i \;\;\text{for all } i \,\bigr\}.
\]
The threshold $\epsilon_i$ is the critical value below which the
$i$-th variable's contribution to the cross-feeding reverses sign:
below $\epsilon_i$, the $i$-th leg drags the others down rather
than lifting them.
\end{definition}

\section{The tangential condition}\label{sec:triad-tangential}

\begin{proposition}[Boundary rescue]\label{prop:boundary-rescue}
At the boundary face $\partial_i K_\epsilon :=
\{x \in K_\epsilon : x_i = \epsilon_i\}$, the tangential
condition (\cref{thm:viability-di}) reduces to
\begin{equation}\label{eq:rescue}
  \sum_{j \neq i} \beta_{ij}\, x_j
  \;\geq\;
  \gamma_i\, \epsilon_i^2 \;-\; \alpha_i\, \epsilon_i
  \;=:\; \theta_i.
\end{equation}
When $\alpha_i \geq \gamma_i \epsilon_i$, the threshold
$\theta_i \leq 0$ and the condition is automatically satisfied:
self-reinforcement alone prevents exit.  When
$\alpha_i < \gamma_i \epsilon_i$, the cross-feeding from the
other two variables must compensate.
\end{proposition}

\begin{proof}
The contingent cone to $K_\epsilon$ at $x$ with $x_i = \epsilon_i$
is $T_{K_\epsilon}(x) = \{v \in \R^3 : v_i \geq 0\}$
(\cref{def:contingent}).  The tangential condition
$f(x) \in T_{K_\epsilon}(x)$ requires $f_i(x) \geq 0$:
\[
  \alpha_i \epsilon_i + \sum_{j \neq i} \beta_{ij} x_j
  - \gamma_i \epsilon_i^2 \geq 0,
\]
which rearranges to \eqref{eq:rescue}.
\end{proof}

\begin{remark}[The rescue is mutual]\label{rem:mutual-rescue}
The inequality \eqref{eq:rescue} is the viability condition at
the $i$-th face: when variable~$i$ is at its minimum, the
\emph{other two} must be strong enough to pull it back.
By the cooperative structure ($\beta_{ij} > 0$), this pull is
guaranteed as long as $x_j$ is large.  The failure mode is
cascade: if $x_j$ drops, the rescue of $x_i$ weakens, which
reduces the rescue of $x_k$, which further weakens $x_j$.
Positive feedback amplifies both growth and decline.
\end{remark}

\section{Stability and bifurcation}\label{sec:triad-bifurcation}

\begin{theorem}[Existence, stability, and coupling threshold]
\label{thm:triad}
\hfill
\begin{enumerate}[label=(\alph*)]
  \item \textbf{Existence.}  The system \eqref{eq:triad} has a
    positive fixed point $x^* \in \R^3_{> 0}$.
  \item \textbf{Stability.}  All eigenvalues of the Jacobian
    $J = Df|_{x^*}$ have strictly negative real part:
    $x^*$ is locally asymptotically stable.
  \item \textbf{Coupling threshold.}  $x^* \in K_\epsilon^\circ$
    if and only if, for every $i \in \{1,2,3\}$,
    \begin{equation}\label{eq:coupling-threshold}
      \sum_{j \neq i} \beta_{ij}\, x_j^*
      \;>\;
      \theta_i
      \;=\;
      \gamma_i \epsilon_i^2 - \alpha_i \epsilon_i.
    \end{equation}
    There exists a critical coupling matrix
    $B^* = (\beta_{ij}^*)$ such that
    $x^* \in K_\epsilon^\circ$ when $\beta_{ij} > \beta_{ij}^*$,
    and $x^*$ exits $K_\epsilon$ through face
    $\partial_k K_\epsilon$ when $\beta_{kj}$ drops below
    $\beta_{kj}^*$ for some $j \neq k$.
  \item \textbf{Stable until gone.}  At the bifurcation
    ($\beta_{kj} = \beta_{kj}^*$, so that $x_k^* = \epsilon_k$),
    all eigenvalues of $J$ remain strictly negative.  The system
    does not lose resilience gradually: it is stable until the
    fixed point exits $K_\epsilon$, at which point viability is
    lost abruptly.
\end{enumerate}
\end{theorem}

\begin{proof}
\textbf{(a)}  Define the map $T: \R^3_{\geq 0} \to \R^3_{> 0}$
by $T_i(x) = (\alpha_i + \sqrt{\alpha_i^2 + 4\gamma_i
\sum_{j \neq i}\beta_{ij} x_j})\, /\, (2\gamma_i)$,
the positive root of
$\gamma_i z^2 - \alpha_i z - \sum_{j \neq i}\beta_{ij} x_j = 0$.
$T$ is continuous, monotone increasing, and maps $[0, M]^3$ to
itself for $M$ sufficiently large (since $T_i$ grows as
$O(\sqrt{M})$ while $M$ grows linearly).  By Brouwer's
fixed-point theorem, $T$ has a fixed point $x^* \in [0,M]^3$.
Positivity: $T_i(x) > 0$ for all $x \geq 0$.

\textbf{(b)}  The Jacobian at $x^*$ has entries
$J_{ii} = \alpha_i - 2\gamma_i x_i^*$ (negative, since the
fixed-point equation gives
$\gamma_i x_i^* > \alpha_i$) and
$J_{ij} = \beta_{ij} > 0$ for $i \neq j$.
The matrix $-J$ is a $Z$-matrix (non-positive off-diagonal).
We show $-J$ is a nonsingular $M$-matrix by exhibiting a positive
vector $w > 0$ with $(-J)w > 0$.  Take $w = x^*$.
The $i$-th component of $(-J)x^*$ is
\begin{align*}
  \bigl((-J)\,x^*\bigr)_i
  &= (2\gamma_i x_i^* - \alpha_i)\,x_i^*
     - \sum_{j \neq i} \beta_{ij}\, x_j^* \\
  &= 2\gamma_i (x_i^*)^2 - \alpha_i x_i^*
     - \sum_{j \neq i} \beta_{ij}\, x_j^* \\
  &= 2\gamma_i (x_i^*)^2
     - \underbrace{\bigl(\alpha_i x_i^*
       + \textstyle\sum_{j \neq i} \beta_{ij}\, x_j^*\bigr)}
       _{= \gamma_i (x_i^*)^2 \text{ by the FP equation}} \\
  &= \gamma_i (x_i^*)^2 \;>\; 0.
\end{align*}
By the $M$-matrix characterisation ($Z$-matrix + $\exists\, w > 0$
with $Aw > 0$ $\Rightarrow$ nonsingular $M$-matrix), all
eigenvalues of $-J$ have positive real part, hence all eigenvalues
of $J$ have negative real part.

\textbf{(c)}  At the fixed point,
$\sum_{j \neq i} \beta_{ij} x_j^* =
\gamma_i (x_i^*)^2 - \alpha_i x_i^*$.
Since $g(z) = \gamma_i z^2 - \alpha_i z$ is strictly increasing
for $z > \alpha_i/(2\gamma_i)$ and $x_i^* > \alpha_i/\gamma_i >
\alpha_i/(2\gamma_i)$, requiring $x_i^* \geq \epsilon_i$ is
equivalent to \eqref{eq:coupling-threshold}.
The critical coupling $B^*$ is defined implicitly by the binding
constraint $x_k^*(B^*) = \epsilon_k$; by the implicit function
theorem (the Jacobian $J$ is nonsingular by~(b)), $x^*$ depends
continuously on $B$, and the transition is a bifurcation.

\textbf{(d)}  At the bifurcation $x_k^* = \epsilon_k$, the
proof of~(b) still applies: $(-J)x^*_i = \gamma_i(x_i^*)^2 > 0$
for all $i$ (since $x_i^* \geq \epsilon_i > 0$).
All eigenvalues of $J$ remain strictly negative.
\end{proof}

Part~(d) is the mathematical surprise: the system does not
announce its impending collapse through loss of resilience or
critical slowing down.  It is stable until the coupling drops
below threshold, at which point the fixed point exits
$K_\epsilon$ and viability is lost in a single bifurcation.
The perception of gradual decline is an artefact: what drifts
gradually are the coupling parameters $\beta_{ij}(t)$, not the
system's stability.

\section{Product fragility}\label{sec:triad-fragility}

\begin{definition}[Product Lyapunov function]\label{def:product-lyapunov}
The \emph{product Lyapunov function} for the triad kernel
$K_\epsilon$ is
\[
  V_\Pi(x) \;:=\; \prod_{i=1}^3 (x_i - \epsilon_i).
\]
$V_\Pi$ vanishes on $\partial K_\epsilon$ and is positive on
$K_\epsilon^\circ$.  The viability metric
$g_{V_\Pi} = V_\Pi^{-2}\, g_S$ (\cref{def:viab-metric}) makes
$(K_\epsilon^\circ,\, g_{V_\Pi})$ a complete Riemannian manifold
(\cref{prop:viab-complete}).
\end{definition}

\begin{proposition}[Single-coordinate collapse]\label{prop:single-collapse}
If any single coordinate satisfies
$x_k(t) \to \epsilon_k$ as $t \to T$, then
$V_\Pi(x(t)) \to 0$ regardless of the other coordinates.
The $g_{V_\Pi}$-distance from any interior point to the face
$\partial_k K_\epsilon$ is infinite:
\[
  d_{g_{V_\Pi}}(x,\, \partial_k K_\epsilon) \;=\; +\infty.
\]
\end{proposition}

\begin{proof}
$V_\Pi = (x_k - \epsilon_k) \prod_{j \neq k}(x_j - \epsilon_j)
\leq M^2 (x_k - \epsilon_k) \to 0$, where
$M = \max_{j \neq k} \sup_t (x_j(t) - \epsilon_j)$ is finite
by dissipativity.
For the distance estimate, parametrise a curve
$\gamma(s)$ with only the $k$-th coordinate varying.
Its $g_{V_\Pi}$-length is
\[
  L_{g_{V_\Pi}}(\gamma)
  = \int \frac{|\gamma'(s)|}{V_\Pi(\gamma(s))}\, ds
  \geq \frac{1}{M^2}
  \int \frac{|x_k'(s)|}{x_k(s) - \epsilon_k}\, ds,
\]
which diverges as
$\int du/u = -\log(x_k - \epsilon_k) \to \infty$
when $x_k \to \epsilon_k$.
\end{proof}

\begin{remark}[The dependent variable]\label{rem:dependent}
The third variable of the triad ($x_3$: human capital flux) is
not structurally independent.  Immigration selection presupposes
institutional permission ($x_1$ controls who may arrive), and
retention requires resource reward ($x_2$ controls who stays).
The autonomous component of $x_3$ is therefore
\[
  x_3 \;=\; h(x_1, x_2) \;+\; \eta_3,
\]
where $h$ captures the coupling ($h$ increasing in both arguments)
and $\eta_3$ is the genuinely exogenous component (the pool of
potential emigrants in source countries).  In the triad dynamics
\eqref{eq:triad}, this dependence is already encoded in the
coupling terms $\beta_{31} x_1 + \beta_{32} x_2$, but the
structural implication is sharper: $x_3$ cannot be rescued
independently.  When $x_1$ weakens (institutions restrict
permission) or $x_2$ weakens (resources stop rewarding), $x_3$
declines \emph{even without a direct shock to $x_3$ itself}.
The rescue condition \eqref{eq:rescue} at the $x_3$-face
requires precisely the variables that $x_3$ depends on.

This makes $x_3$ the most fragile coordinate of the triad:
it is the only one that cannot self-rescue
($\theta_3 = \gamma_3 \epsilon_3^2 - \alpha_3 \epsilon_3$
is large when $\alpha_3$ is small, meaning weak
self-reinforcement), and its cross-feeding comes from the
variables it structurally depends on.  The product Lyapunov
function $V_\Pi$ inherits this fragility: $x_3$ is generically
the first coordinate to hit its threshold, and by
\cref{prop:single-collapse}, this suffices to collapse
$V_\Pi$ to zero.
\end{remark}

\section{The mesh immunity theorem}\label{sec:mesh}

The triad describes a system whose viability depends on
maintaining three coupled variables above threshold.
A separate structural question is: can such a system be
\emph{destroyed from outside}?

\begin{definition}[Mesh graph]\label{def:mesh}
An execution graph $G = (V, E)$ (\cref{def:exgraph}) is a
\emph{mesh} if it contains no cut vertex
(\cref{def:cutvertex}): for every $v \in V$,
$G \setminus \{v\}$ is connected.
Equivalently, $G$ is $2$-connected.
\end{definition}

\begin{theorem}[Mesh immunity]\label{thm:mesh-immunity}
Let $G$ be a mesh.  Then:
\begin{enumerate}[label=(\alph*)]
  \item \textbf{No targeted elimination.}  For every $v \in V$,
    $G \setminus \{v\}$ is connected: removing any single node
    does not disconnect the execution graph.
    The elimination strategy of \cref{thm:cutvertex} (make
    yourself the cut vertex, then remove the knife) requires
    a cut vertex to exist.  In a mesh, no such vertex exists.
  \item \textbf{The only death is phase transition.}  The
    mean actuation field $\bar{U}$ (\cref{thm:meanfield}) is
    computed over all nodes.  In a mesh, the mean is a
    \emph{bulk} quantity: no single node's removal changes
    $\bar{U}$ by more than $O(1/|V|)$.  The system's knife
    threshold $\bar{U} + \tau(\Obs)$ is therefore stable under
    single-node perturbation.  The only mechanism that moves
    $\bar{U}$ below the viability threshold is a
    \emph{collective} shift: a phase transition in which the
    coupling constants (ideology, coercion, reward) drop below
    threshold simultaneously across the mesh.
\end{enumerate}
\end{theorem}

\begin{proof}
\textbf{(a)} is the definition of $2$-connectivity
(\cref{def:mesh}).

\textbf{(b)}  Let $\bar{U} = \frac{1}{n}\sum_{i=1}^n \|U_i\|$
be the mean actuation.  Removing node $k$ changes the mean to
$\bar{U}' = \frac{1}{n-1}\sum_{i \neq k} \|U_i\|$.
Then $|\bar{U}' - \bar{U}| = |\bar{U} - \|U_k\||/(n-1)$, which
is $O(1/n)$ since both $\bar{U}$ and $\|U_k\|$ are bounded.  For the knife threshold
$\bar{U} + \tau(\Obs)$ to cross a critical value, the shift must
be $\Omega(1)$, requiring $\Omega(n)$ nodes to change
simultaneously---a phase transition.
\end{proof}

\begin{remark}[Historical instantiation of mesh immunity]
\label{rem:mesh-history}
No communist party in history has been eliminated by external
targeted removal.  The structural reason is
\cref{thm:mesh-immunity}(a): a Leninist party is organised as a
distributed execution mesh (democratic centralism $=$ every node
both routes and executes), and removing any single node
(leader, cadre, cell) does not disconnect the graph.
Every historical end of a communist party is an internal phase
transition (\cref{thm:mesh-immunity}(b)): the Soviet Union
(1991), the Eastern Bloc (1989--91), the Kuomintang's
one-party state in Taiwan (1987--2000).  In each case, the
mean actuation field $\bar{U}$ (ideological commitment $\times$
coercive capacity) dropped below the viability threshold
\emph{from within}, not by external excision.
The same immunity holds for any $2$-connected organisation:
religious orders, insurgent networks, distributed autonomous
organisations.  The topology, not the ideology, determines
the death mode.
\end{remark}

\section{Instantiation: the American republic}\label{sec:america}

The cooperative triad \eqref{eq:triad} admits a canonical
instantiation:

\begin{center}
\renewcommand{\arraystretch}{1.25}
\begin{tabular}{@{}clp{6.5cm}@{}}
\toprule
\textbf{Variable} & \textbf{Name} & \textbf{Content} \\
\midrule
$x_1$ & Institutional capacity &
  Constitutional architecture, rule of law,
  separation of powers, peaceful transfer \\
$x_2$ & Resource base &
  Continental landmass, two-ocean buffer,
  navigable rivers, energy reserves \\
$x_3$ & Human capital flux &
  Immigration selection, meritocratic mobility,
  civic identity untied to ethnicity \\
\bottomrule
\end{tabular}
\end{center}

The six coupling channels:

\begin{center}
\renewcommand{\arraystretch}{1.25}
\begin{tabular}{@{}ccp{7cm}@{}}
\toprule
$\beta_{ij}$ & \textbf{Channel} & \textbf{Mechanism} \\
\midrule
$\beta_{12}$ & $x_2 \to x_1$ &
  Resource abundance funds institutional capacity \\
$\beta_{21}$ & $x_1 \to x_2$ &
  Property rights enable resource development \\
$\beta_{13}$ & $x_3 \to x_1$ &
  Talented populace creates and reforms institutions \\
$\beta_{31}$ & $x_1 \to x_3$ &
  Civic identity and rule of law attract talent \\
$\beta_{23}$ & $x_3 \to x_2$ &
  Talent develops and exploits resources \\
$\beta_{32}$ & $x_2 \to x_3$ &
  Abundance rewards effort, retaining talent \\
\bottomrule
\end{tabular}
\end{center}

\begin{proposition}[The American fixed point]\label{prop:american-fp}
Under this instantiation, the cooperative fixed point $x^*$
satisfies $x^* \in K_\epsilon^\circ$ during the epoch
$\sim$1865--1965 (post--Civil War consolidation through peak
institutional and demographic expansion).
The coupling threshold
(\cref{thm:triad}(c)) was maintained by:
\begin{enumerate}[label=(\roman*)]
  \item $\beta_{21}$ large: the Homestead Act (1862) and
    patent law converted institutional protection directly into
    resource development;
  \item $\beta_{31}$ large: the civic-identity model
    (``American'' $=$ allegiance to a document, not a bloodline)
    sustained the immigration selection filter;
  \item $\beta_{13}$ large: successive waves of immigrants
    built, staffed, and reformed institutions (public schools,
    universities, civil service).
\end{enumerate}
\end{proposition}

\begin{proof}[Verification]
Each coupling channel is independently checkable against
historical data.  The proposition is structural, not causal:
it asserts that the three variables were simultaneously above
threshold and mutually reinforcing, which is verified by the
simultaneous presence of (i)~institutional stability (no
constitutional crisis between 1865 and 1965), (ii)~resource
expansion (continental infrastructure, energy production), and
(iii)~demographic growth driven by immigration
(waves of 1880--1924, post-1945).
The tangential condition \eqref{eq:rescue} held at each face:
when any single variable weakened (e.g.\ institutional stress
during Reconstruction), the other two provided sufficient
cross-feeding to prevent exit from $K_\epsilon$.
\end{proof}

\begin{remark}[Phase transition and the knife]
\label{rem:triad-knife}
The word ``once'' in ``once considered great'' is the detection
of a weakening coupling.  In the framework of \cref{prop:phase},
each variable $x_i$ of the triad is an autonomous actuator
(\cref{def:knife}, condition~(1)): its dynamics $f_i$ can drive
the state independently.  When the cross-feeding is strong
($\beta_{ij} > \beta_{ij}^*$), each variable is a
\emph{tool}---its autonomous actuation reinforces the others.
When the coupling weakens below threshold, the same autonomous
dynamics become a \emph{knife}: a variable in decline drags the
others down through the positive feedback channel.

By \cref{thm:triad}(d) (stable until gone), the transition is
not gradual.  The coupling parameters $\beta_{ij}(t)$ may drift
slowly, but the system remains at a stable fixed point until the
bifurcation.  Then the fixed point exits $K_\epsilon$ and
viability collapses.  ``Greatness'' was a transient fixed point
(\cref{thm:triad}), not an identity.  The product Lyapunov
function $V_\Pi = \prod_i (x_i - \epsilon_i)$
(\cref{def:product-lyapunov}) ensures that the collapse is
\emph{multiplicative}: a decline in any single variable degrades
the entire system's viability measure, regardless of the other
two (\cref{prop:single-collapse}).

By \cref{rem:dependent}, $x_3$ (human capital) is the most
fragile coordinate: it depends on both $x_1$ (institutional
permission to arrive) and $x_2$ (resource reward for staying).
One does not ``select'' immigrants; one creates conditions
($x_1, x_2$) under which immigration self-selects.  When either
condition weakens, the selection filter breaks---not because
the immigrants changed, but because the triad's internal
coupling dropped below the rescue threshold
\eqref{eq:rescue}.
\end{remark}

\begin{remark}[The triad is not American]\label{rem:not-american}
Nothing in \cref{def:triad} is specific to any nation.
The cooperative triad is a structural motif that appears wherever
three coupled variables with positive feedback sustain a system
above threshold.  The American republic is one instantiation.
The framework predicts the same transient-viability dynamics for
any system whose health depends on three mutually reinforcing
components.

The mathematics is the theorem; the nation is the example.
\end{remark}

\section{The three-body problem: mass gap, optimal transport, and
asymptotic collapse}\label{sec:triad-threebody}

The cooperative triad (\cref{def:triad}) is structurally a
\emph{three-body problem}: three interacting masses $x_1, x_2, x_3$
whose trajectories are coupled through the cross-feeding terms
$\beta_{ij}x_j$.  The classical three-body problem of celestial
mechanics has no closed-form general solution (Poincar\'e, 1892).
The triad, however, admits a special resolution when the
\emph{vacuum}---the zero state
$\mathbf{0} = (0,0,0) \in \partial\R^3_{\geq 0}$---carries a
mass gap.

\subsection{Mass gap as fixed-point anchor}

\begin{definition}[Mass gap as fixed-point anchor]
\label{def:massgap-anchor}
Let $G_{\mathrm{triad}}$ be the execution graph
(\cref{def:exgraph}) of the three-body triad, with nodes
$\{x_1, x_2, x_3, \kappa, \infty\}$ and edge capacities
$c(x_i, x_j) = \beta_{ij}$.
The \emph{mass gap} of the triad is the spectral gap of its graph
Laplacian (\cref{def:laplacian}):
\[
  \Delta m \;:=\; \lambda_1(\Delta_{G_{\mathrm{triad}}}) \;\geq\; 0.
\]
The mass gap is the \emph{fixed-point anchor}: when $\Delta m > 0$,
the vacuum $\mathbf{0}$ is spectrally separated from the positive
fixed point $x^* \in \R^3_{>0}$, and the existence and asymptotic
stability of $x^*$ are enforced.
\end{definition}

\begin{proposition}[Mass gap $=$ equation of motion]
\label{prop:massgap-eom}
The mass gap $\Delta m > 0$ is equivalent to the existence of the
trajectory (equation of motion of the triad):
\[
  \Delta m > 0
  \;\iff\;
  \exists\; x(\cdot): [0,\infty) \to K_\epsilon
  \text{ solving \eqref{eq:triad} with }
  f(x^*) = 0,\; x^* \in K_\epsilon^\circ.
\]
When $\Delta m \to 0$ (the gap closes), the tangential
condition~\eqref{eq:rescue} becomes binding at the most fragile
face $\partial_k K_\epsilon$, and the viable trajectory ceases to
exist.
\end{proposition}

\begin{proof}
$(\Rightarrow)$: If $\Delta m > 0$, then $\lambda_1 > 0$.
By \cref{thm:massgap}, the viability axiom holds: there exists a
viable path (positive agentic flow) in $G_{\mathrm{triad}}$.
By dissipativity ($\gamma_i > 0$) and cooperativity
($\beta_{ij} > 0$), the fixed-point map $T$ in the proof of
\cref{thm:triad}(a) has a positive fixed point
$x^* \in \R^3_{>0}$, and by \cref{thm:triad}(b) it is
asymptotically stable.

$(\Leftarrow)$: If a trajectory $x(\cdot)$ exists in
$K_\epsilon^\circ$, then $x^*(B) \in K_\epsilon^\circ$, so the
coupling condition \eqref{eq:coupling-threshold} holds for every
$i$.  The cross-feeding terms $\sum_{j \neq i}\beta_{ij}x_j^* >
\theta_i > 0$ imply every cut $\partial S$ in
$G_{\mathrm{triad}}$ has strictly positive capacity.
Hence $h(G_{\mathrm{triad}}) > 0$, and by \cref{thm:cheeger},
$\lambda_1 \geq h(G)^2/2 > 0$.
\end{proof}

\begin{remark}[The closeness of the mass gap]\label{rem:gap-closeness}
``The closeness of the mass gap'' is the quantity
$\delta_k := x_k^* - \epsilon_k \geq 0$ for the most fragile
coordinate~$k$ (\cref{rem:dependent}).
When $\delta_k \to 0^+$, the fixed point approaches the face
$\partial_k K_\epsilon$; the $g_{V_\Pi}$-distance from $x^*$ to
that face diverges:
\[
  d_{g_{V_\Pi}}\!\bigl(x^*,\, \partial_k K_\epsilon\bigr)
  \;\geq\;
  \frac{1}{M^2} \int_{\epsilon_k}^{x_k^*}
  \frac{du}{u - \epsilon_k}
  \;=\;
  \frac{1}{M^2}\,\log\!\Bigl(1 + \frac{\delta_k}{\epsilon_k}\Bigr)
  \;\to\; +\infty
  \quad\text{as }\delta_k \to 0^+
\]
(cf.\ \cref{prop:single-collapse}).
The trajectory ``barely exists'': the fixed point is viable, but
infinitely far from collapse in the viability metric.
When $\delta_k = 0$, $V_\Pi(x^*) = 0$ and the trajectory ceases
to exist.
\end{remark}

\subsection{Optimal transport of the observation fixed point}

The observation function $\Obs$ (\cref{sec:framework}) maps the
current state distribution $\rho_t \in \mathcal{P}(K_\epsilon)$
to an estimate of the fixed point.  The natural measure of how
well the system has ``found'' its fixed point is the
\emph{Wasserstein-$2$ (optimal transport) distance}:

\begin{definition}[Observation transport cost]
\label{def:obs-transport}
The \emph{observation transport cost} at time $t$ is
\[
  W_2(\rho_t,\, \delta_{x^*})^2
  \;:=\;
  \inf_{\pi \in \Pi(\rho_t,\, \delta_{x^*})}
  \int_{K_\epsilon \times K_\epsilon}
  |x - y|^2\, d\pi(x, y)
  \;=\;
  \int_{K_\epsilon} |x - x^*|^2\, d\rho_t(x),
\]
where $\delta_{x^*}$ is the Dirac measure at the fixed point $x^*$
and the last equality uses the unique optimal coupling
$\pi^* = \rho_t \otimes \delta_{x^*}$.
The observation is \emph{viable} when
$W_2(\rho_t, \delta_{x^*}) < \infty$.
\end{definition}

\begin{proposition}[Optimal transport contracts toward fixed point]
\label{prop:ot-contract}
Under the deterministic cooperative triad dynamics
(\cref{def:triad} with $\delta_i = 0$), with $\Delta m > 0$ and
$\rho_0$ supported in the basin of attraction of $x^*$:
\[
  W_2(\rho_t,\, \delta_{x^*})^2
  \;\leq\;
  W_2(\rho_0,\, \delta_{x^*})^2\; e^{-2\lambda_1 t}
  \;\to\; 0
  \quad \text{as } t \to \infty.
\]
The observation transport cost contracts exponentially to zero at
rate $\lambda_1 = \Delta m$ (the mass gap).
\end{proposition}

\begin{proof}
Let $\Phi_t$ be the flow map of \eqref{eq:triad}.
Since $\rho_t = (\Phi_t)_\# \rho_0$ (push-forward), the optimal
coupling in \cref{def:obs-transport} gives
\[
  W_2(\rho_t,\, \delta_{x^*})^2
  = \int_{K_\epsilon} |\Phi_t(x) - x^*|^2\, d\rho_0(x).
\]
The eigenvalues of $Df|_{x^*}$ all have real part
$\leq -\lambda_1 < 0$ (\cref{thm:triad}(b) combined with
\cref{thm:massgap}).  Linearising:
$|\Phi_t(x) - x^*|^2 \leq |x - x^*|^2 e^{-2\lambda_1 t}$
for $x$ in the basin of $x^*$.  Integrating over $\rho_0$:
\[
  W_2(\rho_t,\, \delta_{x^*})^2
  \leq e^{-2\lambda_1 t}
  \int_{K_\epsilon} |x - x^*|^2\, d\rho_0(x)
  = W_2(\rho_0,\, \delta_{x^*})^2\, e^{-2\lambda_1 t}.  \qedhere
\]
\end{proof}

\subsection{Asymptotic collapse}

\begin{theorem}[Asymptotic collapse]\label{thm:asymptotic-collapse}
If any coupling $\beta_{kj}$ drops below the critical threshold
$\beta_{kj}^*$ (\cref{thm:triad}(c)), then:
\begin{enumerate}[label=(\alph*)]
  \item \textbf{Fixed-point exit.}  $x_k^*(B) < \epsilon_k$:
    the fixed point exits $K_\epsilon$ through the face
    $\partial_k K_\epsilon$.
  \item \textbf{Product collapse.}
    $V_\Pi(x(t)) \to 0$ for every trajectory, regardless of
    initial condition (\cref{prop:single-collapse}).
  \item \textbf{Transport divergence.}
    $W_2(\rho_t,\, \delta_{x^*}) \to \infty$ as $t \to \infty$:
    the observation transport cost to the former fixed point
    diverges.
\end{enumerate}
\end{theorem}

\begin{proof}
\textbf{(a)} follows directly from \cref{thm:triad}(c): the
fixed-point equation at $\beta_{kj} < \beta_{kj}^*$ yields
$x_k^*(B) < \epsilon_k$.

\textbf{(b)}: With $x^* \notin K_\epsilon^\circ$, the boundary
rescue condition \eqref{eq:rescue} fails for the $k$-th face.
The product Lyapunov function $V_\Pi$ has no interior critical
point; by \cref{prop:single-collapse},
$V_\Pi(x(t)) \leq M^2(x_k(t) - \epsilon_k) \to 0$.

\textbf{(c)}: The distribution $\rho_t$ cannot concentrate on
$x^* \notin K_\epsilon^\circ$.  As trajectories exit $K_\epsilon$
through $\partial_k K_\epsilon$, the second moment
$\int |x - x^*|^2\,d\rho_t(x)$ is bounded below by
$(\epsilon_k - x_k^*)^2 \cdot \rho_t(\partial_k K_\epsilon)$
(since on the face $\partial_k K_\epsilon$ where $x_k = \epsilon_k$,
the distance satisfies $|x - x^*|^2 \geq (\epsilon_k - x_k^*)^2 > 0$,
as $x_k^* < \epsilon_k$ by~(a));
this lower bound grows without bound because
$\rho_t(\partial_k K_\epsilon) \to 1$ and $(\epsilon_k - x_k^*)^2 > 0$
is fixed.
\end{proof}

\begin{remark}[``At $t \to \infty$, system collapse'']
\label{rem:collapse-hypo}
The hypothesis ``at $t \to \infty$, system collapse'' is
\cref{thm:asymptotic-collapse} made precise.  The collapse is not
gradual: by \cref{thm:triad}(d) (stable until gone), the system
holds at a stable fixed point until the coupling $\beta_{kj}$
crosses the critical threshold $\beta_{kj}^*$, at which instant the
fixed point exits $K_\epsilon$, $V_\Pi \to 0$, and
$W_2(\rho_t, \delta_{x^*}) \to \infty$.

Dually, \cref{prop:ot-contract} gives the complementary regime:
when the mass gap is open ($\Delta m > 0$), optimal transport
contracts the observation to the fixed point exponentially fast at
rate $\Delta m$.  The mass gap IS the convergence rate.  Its
closure IS the divergence of collapse.  There is no intermediate
state (\cref{thm:triad}(d)).
\end{remark}

\chapter{中县干部 --- The Graded Lattice}\label{app:zhongxian}

The preceding appendices applied the framework to individual agents
and historical cases
(\cref{app:sources,app:secondsex,app:threeli})
and to the system itself as a cooperative triad
(\cref{app:triad}).  This appendix applies it to the \emph{internal
structure} of a viability maintenance system: a county-level
bureaucratic hierarchy observed from within, with exact counts.
The mathematical object is a \emph{graded lattice}---a partially
ordered set with rank function, filtration rates, and absorbing
boundaries---and the dynamics are a \emph{dual-ring differential
inclusion} in which the formal institutional rules (inner ring)
define the viability kernel and the relationship network (outer
ring) defines the set-valued dynamics.  The central result is that
the knife-is-the-mean theorem (\cref{thm:meanfield}) instantiates
at the level of a single county: the critical threshold separating
viable cadres from non-viable ones is the mean guanxi level, and
every pathology documented in the source---fake achievements, vote
canvassing, political families, corruption---is a consequence of
viability maintenance under this mean field.

The empirical basis is Feng Junqi's doctoral dissertation
\emph{Zhong Xian Ganbu} \cite{zhongxian}, a participant-observation
study of a county in Henan province (population ${\sim}800{,}000$)
conducted during 2008--2010, in which Feng held a temporary
government post and systematically documented the composition,
promotion trajectories, relationship networks, and disciplinary
outcomes of the county's ${\sim}1{,}000$ cadres at 副科级 and above.

% ================================================================
\section{The cadre lattice}\label{sec:zx-lattice}
% ================================================================

\begin{definition}[Graded cadre lattice]\label{def:cadre-lattice}
A \emph{graded cadre lattice} is a finite poset
$\Lambda = \bigsqcup_{k=0}^{K} \Lambda_k$ with rank function
$\rho \colon \Lambda \to \{0, 1, \ldots, K\}$, equipped with:
\begin{enumerate}[label=(\roman*)]
  \item \textbf{Level sizes} $n_k := |\Lambda_k|$;
  \item \textbf{Filtration ratios}
    $r_k := n_{k+1}/n_k$ for $0 \leq k < K$, measuring
    the structural selectivity between adjacent levels;
  \item \textbf{Time constants} $\tau_k > 0$: the mean duration
    at level $\Lambda_k$ before promotion to $\Lambda_{k+1}$
    (averaged over those who are promoted);
  \item \textbf{Age boundaries} $a_k^{\max}$: hard upper bounds
    on the age at which a cadre in $\Lambda_k$ remains eligible
    for promotion to $\Lambda_{k+1}$.
\end{enumerate}
\end{definition}

\begin{example}[Zhong County]\label{ex:zx-data}
The Zhong County cadre system has $K = 3$ levels:

\begin{center}
\renewcommand{\arraystretch}{1.25}
\begin{tabular}{@{}clrrrl@{}}
\toprule
$k$ & \textbf{Level} (级别) & $n_k$ & $r_k$ & $\tau_k$ (yr)
    & $a_{k}^{\max}$ \\
\midrule
$0$ & 副科级 (deputy section) & 680 & $0.41$ & $2.7$ & ${\sim}45$ \\
$1$ & 正科级 (full section)   & 280 & $0.14$ & $7.3$ & ${\sim}50$ \\
$2$ & 副处级 (deputy county)  &  40 & $0.13$ & $7.5$ & ${\sim}55$ \\
$3$ & 正处级 (full county)    &   5 & ---    & ---   & ${\sim}60$ \\
\bottomrule
\end{tabular}
\end{center}

\noindent
The entry time (from first assignment to $\Lambda_0$) averages $8$ years.
A cadre entering at age $22$ therefore reaches $\Lambda_0$ at
${\sim}30$, $\Lambda_1$ at ${\sim}33$, $\Lambda_2$ at ${\sim}40$,
$\Lambda_3$ at ${\sim}48$.  The margin between this ideal trajectory
and the age boundary at $\Lambda_3$ is ${\sim}7$ years---thin enough
that a single delay of one promotion cycle can be fatal.
The lattice has additional substructure: each level contains
\emph{hidden steps} (隐形台阶), e.g.\ within $\Lambda_2$ the
progression 县委常委 $\to$ 常务副县长 $\to$ 县长 $\to$ 县委书记,
and a mandatory \emph{government--party spiral}
(政--党螺旋晋升模式) that alternates between government and
party positions.
\end{example}

\begin{proposition}[Lattice as viability constraint]
\label{prop:pyramid-viability}
The filtration ratios satisfy $r_k \ll 1$ for $k \geq 1$
and $r_0 < 1$.  The cumulative survival ratio from $\Lambda_0$ to
$\Lambda_3$ is
\[
  p_{\textup{top}}
  \;=\; \prod_{k=0}^{K-1} r_k
  \;=\; 0.41 \times 0.14 \times 0.13
  \;\approx\; 0.0074.
\]
Fewer than $1\%$ of cadres at $\Lambda_0$ will occupy $\Lambda_3$ in their
career.  The age boundaries impose a hard absorbing face: in the
viability framework (\cref{thm:viability-di}), the constraint set is
\begin{equation}\label{eq:cadre-K}
  K_{\textup{cadre}}
  \;=\;
  \bigl\{\,(k,\, a) \;:\; 0 \leq k \leq K,\;\;
  a \leq a_k^{\max}\,\bigr\},
\end{equation}
and the face $a = a_k^{\max}$ is absorbing: once crossed,
no trajectory can re-enter $K_{\textup{cadre}}$ at level~$k{+}1$.
\end{proposition}

\begin{remark}[一刀切 and temporal linearity]\label{rem:yidaoqie}
The bureaucratic term 一刀切 (``one cut of the knife'') for
age-based cutoffs is a literal knife in the sense of
\cref{def:knife}: the age variable has autonomous dynamics
($a' = 1$, always advancing) and is fully observable.  It
satisfies both conditions of the knife definition.  Unlike other
knives in the framework, this one is not a phase function of the
mean field---it is an exogenous, non-negotiable constraint imposed
by temporal linearity itself.  Time is the knife that needs no
actuation, no observability adjustment, and no political
manoeuvring.
\end{remark}

% ================================================================
\section{The dual-ring dynamics}\label{sec:zx-dualring}
% ================================================================

Feng's central theoretical contribution is the \emph{dual-ring
model} (双环模型): the formal institutional system (inner ring)
and the relationship system (outer ring) are coupled, with the
outer ring driving the inner ring.  We formalise this as a coupled
differential inclusion.

\begin{definition}[Dual-ring system]\label{def:dual-ring}
A \emph{dual-ring system} is a pair of coupled dynamics on a
state space $S = S_I \times S_G$:
\begin{align}
  x_I'(t) &\in F_I\bigl(x_I(t),\, x_G(t)\bigr),
    \label{eq:inner-ring} \\
  x_G'(t) &\in F_G\bigl(x_I(t),\, x_G(t)\bigr),
    \label{eq:outer-ring}
\end{align}
where:
\begin{enumerate}[label=(\roman*)]
  \item $x_I \in S_I$ is the \emph{institutional state}
    (formal rank, position, evaluation scores);
  \item $x_G \in S_G$ is the \emph{guanxi state}
    (relationship network, accumulated favours, reputation);
  \item $F_I$ is the formal dynamics (promotion rules,
    performance evaluation, term limits);
  \item $F_G$ is the relationship dynamics (alliance formation,
    gift exchange, vote canvassing);
  \item The coupling is \emph{asymmetric}: $F_I$ depends strongly
    on $x_G$ (relationships determine promotion outcomes),
    while $F_G$ depends only weakly on $x_I$ (formal position
    provides a platform for relationship building, but
    relationships persist across positions).
\end{enumerate}
The viability kernel is defined by the inner ring alone:
\begin{equation}\label{eq:dual-K}
  K
  \;=\;
  \bigl\{\,(x_I,\, x_G) \;:\; x_I \in K_{\textup{cadre}}\,\bigr\},
\end{equation}
but the dynamics that must satisfy the tangential condition are
driven by the outer ring.
\end{definition}

\begin{proposition}[The outer ring drives viability]
\label{prop:outer-drives}
In the dual-ring system, the tangential condition
$F(x) \cap T_K(x) \neq \varnothing$
(\cref{thm:viability-di}) at the boundary of $K$
reduces to a condition on the guanxi state.  Write the
institutional dynamics as
$f_I(x_I, x_G) = f_I^{0}(x_I) + g(x_G)$,
where $f_I^{0}$ is the autonomous formal dynamics
(evaluation absent relationships) and $g(x_G)$ is the
guanxi-driven correction.  At the boundary face
$x_I \in \partial K_{\textup{cadre}}$ (a cadre facing the
age cutoff or a competitive promotion slot), the tangential
condition requires
\begin{equation}\label{eq:guanxi-tangential}
  f_I^{0}(x_I) + g(x_G) \;\in\; T_{K_{\textup{cadre}}}(x_I).
\end{equation}
The autonomous term $f_I^{0}$ is generically insufficient:
the filtration ratio $r_k < 1$ means that formal qualifications
alone do not guarantee promotion.  The tangential condition
therefore requires the guanxi correction $g(x_G)$ to compensate.
\end{proposition}

\begin{proof}
Since $K$ is defined by $x_I \in K_{\textup{cadre}}$ with
$x_G$ unconstrained, the contingent cone at
$(x_I, x_G)$ with $x_I \in \partial K_{\textup{cadre}}$ is
$T_K(x) = T_{K_{\textup{cadre}}}(x_I) \times S_G$
(\cref{def:contingent}).  The tangential condition
$f(x) \in T_K(x)$ projects onto
$f_I(x_I, x_G) \in T_{K_{\textup{cadre}}}(x_I)$,
which is \eqref{eq:guanxi-tangential}.  That $f_I^{0}$ is
generically insufficient is the empirical content of the
filtration ratio $r_k < 1$: in steady state, more cadres
satisfy the formal criteria than there are promotion slots,
so the formal autonomous dynamics point outward
($f_I^{0} \notin T_{K_{\textup{cadre}}}$) for the majority.
The guanxi correction $g(x_G)$ is what bends the trajectory
back into the viable cone.
\end{proof}

\begin{remark}[The knife is the mean guanxi level]
\label{rem:guanxi-knife}
By \cref{thm:meanfield}, the knife threshold is determined
by the mean actuation field~$\bar{U}$.  In the dual-ring
system, the relevant actuation is the guanxi investment: the
total relationship resources (time, money, favours) a cadre
deploys.  Feng documents that a township party secretary
spends 300{,}000--500{,}000~yuan per year on vote canvassing
alone.  The mean guanxi level $\bar{U}_G$ sets the threshold:
a cadre whose guanxi investment falls below $\bar{U}_G$
becomes non-viable regardless of formal qualifications.
This is the knife-is-the-mean theorem at county scale.

Feng quotes a county saying that encodes this:
\begin{center}
\emph{年龄是个宝,能力作参考,关系最重要。}\\[3pt]
{\small (Age is a treasure, ability a reference,
relationships paramount.)}
\end{center}
The three factors are ordered by formalisability: age is
exogenous and fully observable (the time-knife of
\cref{rem:yidaoqie}); ability is partially observable and
partially fabricable (政绩, which the dissertation documents
can be faked); relationships are the actual dynamics that
drive the tangential condition \eqref{eq:guanxi-tangential}.

The dual-ring model also explains the Chinese institutional
paradox: why does increasing formalisation of rules
(制度化) coexist with increasing importance of relationships
(关系)?  The answer is structural.  Each new formal rule
tightens the viability kernel $K_{\textup{cadre}}$---adds
a constraint surface.  To satisfy the tangential condition at
the new boundary, cadres must increase their guanxi
investment $g(x_G)$.  Formalisation does not replace
relationships; it \emph{amplifies} the demand for them.
The inner ring constricts; the outer ring accelerates.
\end{remark}

% ================================================================
\section{Flow concentration: cradles and secretaries}
\label{sec:zx-cradle}
% ================================================================

Certain institutions in the county function as ``cradles''
(摇篮) that disproportionately produce promoted cadres.
This is a flow concentration phenomenon, interpretable as a
min-cut in the promotion graph.

\begin{definition}[Cradle node]\label{def:cradle}
In the execution graph $G = (V, E)$ (\cref{def:exgraph})
of the cadre lattice, a \emph{cradle node} $v \in V$ is a
node with anomalously high conditional promotion rate:
if $p_v$ is the probability that a cadre who passes through
$v$ reaches $\Lambda_{k+1}$, then $v$ is a cradle if
$p_v / r_k > c$ for a threshold $c > 1$.
The principle 高进高出 (``high entry, high exit'') describes
the mechanism: cradle institutions attract already-promising
cadres and amplify their promotion prospects.
\end{definition}

\begin{example}[Zhong County cradles]\label{ex:cradles}
\hfill

\begin{center}
\renewcommand{\arraystretch}{1.25}
\begin{tabular}{@{}lrl@{}}
\toprule
\textbf{Institution} & \textbf{\% of county leaders}
    & \textbf{Mechanism} \\
\midrule
乡镇 (township)       & 59\% & Frontline testing \\
两办 (two offices)     & 22\% & Proximity to power \\
秘书 (secretary post)  & 38\% & Delegated actuation \\
组织部 (org.\ dept.)   & Controls appointments
    & Gate-keeping \\
共青团 (Youth League)   & Highest per capita
    & Early selection \\
\bottomrule
\end{tabular}
\end{center}

\noindent
Township experience appears in $59\%$ of county leaders and
$77\%$ of township-level leaders.  The two offices (县委办 and
政府办) appear in $22\%$ of county leaders.  Secretary experience
appears in $38\%$ of county leaders and $50\%$ of township-level
leaders.  Office director (办公室主任) experience appears in
${\sim}42\%$ of leaders in vertically managed departments.
\end{example}

\begin{proposition}[The secretary as delegated knife]
\label{prop:secretary-knife}
The secretary (秘书) position satisfies the knife definition
(\cref{def:knife}):
\begin{enumerate}[label=(\roman*)]
  \item \textbf{Autonomous actuation.}  The secretary exercises
    隐性权力 (hidden power): control over information flow,
    scheduling of access, and framing of decisions for the leader.
    This is \emph{delegated actuation}---the secretary's actions
    affect the system state without requiring the leader's direct
    command at each step.
  \item \textbf{Observability.}  The secretary is the most
    observed node in the leader's proximity: every action is
    visible to the leader, other staff, and petitioners.
\end{enumerate}
Whether this knife is a tool or a threat is a phase function
of the mean field (\cref{prop:phase}): when the secretary's
actuation is aligned with the leader's viability, the secretary
is a tool that extends the leader's reach; when the secretary
builds an independent power base (as Feng documents in several
cases), the secretary becomes a threat.
The $38\%$ prevalence among county leaders means the secretary
channel is a high-capacity edge in the promotion flow network.
In the max-flow/min-cut framework (\cref{thm:flowcut}),
removing this channel would block a large fraction of viable
promotion paths, making it a critical component of the min-cut.
\end{proposition}

\begin{remark}[办公室政治]\label{rem:office-politics}
Feng documents a phenomenon he calls 办公室政治
(office politics): nearly half of all department leaders
have office director (办公室主任) experience.  Combined
with the secretary prevalence, this means the majority of
successful cadres passed through a proximity-to-power node.
In the execution graph, these nodes form a \emph{bottleneck}:
a small set of positions through which a disproportionate
fraction of the max-flow passes.  The bottleneck is not
accidental---it is a structural consequence of the
dual-ring dynamics.  Proximity to the leader provides
privileged access to the outer ring (guanxi opportunities),
which is the binding constraint for the tangential condition
\eqref{eq:guanxi-tangential}.  The cradle mechanism is not
about training; it is about \emph{positioning within the
relationship flow network}.
\end{remark}

% ================================================================
\section{Political families as cooperative subgraphs}
\label{sec:zx-families}
% ================================================================

Feng documents $21$ large political families (大家族, five or
more cadres at 副科级 or above), $15$ four-person families,
$35$ three-person families, and over $90$ two-person families,
connected by blood and marriage edges.

\begin{definition}[Political clique]\label{def:political-clique}
A \emph{political clique} in the cadre lattice is a connected
subgraph $C \subseteq G$ whose edges are blood ties
(parent--child, sibling) or marriage ties (spouse, in-law),
satisfying:
\begin{enumerate}[label=(\roman*)]
  \item every node in $C$ is a cadre at $\Lambda_k$ for some
    $k \geq 0$;
  \item the internal coupling is cooperative: the promotion
    of one member increases the promotion probability of
    connected members (positive $\beta_{ij}$ in the notation
    of \cref{def:triad}).
\end{enumerate}
The size of the clique is $|C|$.
\end{definition}

\begin{proposition}[Clique amplification]
\label{prop:clique-amplification}
A political clique~$C$ of size~$m$ amplifies the effective
guanxi level of its members.  For member $i \in C$, the
effective actuation is
\begin{equation}\label{eq:clique-amplification}
  U_i^{\textup{eff}}
  \;=\; U_i \;+\; \sum_{j \in C \setminus \{i\}}
    \beta_{ij}\, U_j,
\end{equation}
where $\beta_{ij} > 0$ is the cooperative coupling
(stronger for direct blood or marriage ties, weaker for
distant relatives).  Since all terms are positive,
$U_i^{\textup{eff}} > U_i$: every member is stronger than
alone.  For a clique of size~$m$ with uniform coupling
$\beta$, the amplification factor is
\[
  \frac{\bar{U}_C^{\textup{eff}}}{\bar{U}_C^{0}}
  \;=\; 1 + (m - 1)\,\beta,
\]
which grows linearly in $m$.  A large family ($m \geq 5$,
$\beta \sim 0.2$) amplifies its members' effective actuation
by a factor of ${\sim}1.8$ or more, raising them above the
mean-field knife threshold even when their individual
actuation would fall below it.
\end{proposition}

\begin{proof}
The effective mean actuation within~$C$ is
$\bar{U}_C^{\textup{eff}}
  = \frac{1}{m}\sum_{i \in C} U_i^{\textup{eff}}
  = \frac{1}{m}\sum_{i \in C}
    \bigl(U_i + \sum_{j \neq i} \beta_{ij} U_j\bigr)$.
With uniform coupling $\beta_{ij} = \beta$, this becomes
$\bar{U}_C^{0} + \beta(m-1)\bar{U}_C^{0}
  = \bar{U}_C^{0}\bigl(1 + (m-1)\beta\bigr)$.
\end{proof}

\begin{remark}[The grapevine and its decline]
\label{rem:grapevine}
Feng uses the metaphor 葡萄藤 (grapevine) for political
families: they grow larger with each generation as
marriage edges create new connections.  This is positive
feedback: a larger clique amplifies its members more
(\cref{prop:clique-amplification}), which helps more members
reach higher levels, which provides a higher platform for
the next generation's marriages.  The largest family
documented (张家, centred on Zhang Taikang) has eight
children all holding cadre positions, with marriage edges
reaching into several other families.

The growth is a cooperative triad (\cref{def:triad}) at
the family level:
\begin{center}
\renewcommand{\arraystretch}{1.15}
\begin{tabular}{@{}clp{6cm}@{}}
\toprule
\textbf{Variable} & \textbf{Name} & \textbf{Content} \\
\midrule
$x_1$ & Existing power & Members currently in cadre positions \\
$x_2$ & Accumulated capital & Family reputation, favours owed \\
$x_3$ & New entrants & Children, in-laws entering the lattice \\
\bottomrule
\end{tabular}
\end{center}

\noindent
Feng documents three decline factors, each weakening a
coupling in this triad:
\begin{enumerate}[label=(\roman*)]
  \item \textbf{One-child policy} reduces $|C|$ directly,
    shrinking the cooperative sum in
    \eqref{eq:clique-amplification}.
    This weakens $\beta_{31}$ (fewer children to receive power)
    and $\beta_{13}$ (fewer children to reinforce existing
    members).
  \item \textbf{Cadre exchange system} (干部交流制度) separates
    family members geographically, weakening all $\beta_{ij}$
    by reducing the cooperative coupling to near zero
    for exchanged members.
  \item \textbf{Higher education and urbanisation} removes
    potential entrants from the county lattice entirely:
    the children of leaders attend elite universities and
    work in Beijing, Zhengzhou, or Shanghai rather than
    returning to the county.  This weakens $x_3$ at the
    source.
\end{enumerate}
Each factor moves the family triad toward the bifurcation
threshold of \cref{thm:triad}(c).  By \cref{thm:triad}(d)
(stable until gone), the family appears healthy until the
coupling drops below threshold, at which point its influence
collapses.  Feng observes this directly: political families do
not fade gradually.  They end when the core member retires
or when the coupling channels are severed.
\end{remark}

% ================================================================
\section{The disciplinary boundary and the 政绩 isomorphism}
\label{sec:zx-boundary}
% ================================================================

\begin{definition}[Disciplinary absorbing barrier]
\label{def:discipline-barrier}
The disciplinary system imposes a graduated absorbing barrier
on $K_{\textup{cadre}}$:

\begin{center}
\renewcommand{\arraystretch}{1.25}
\begin{tabular}{@{}lrlp{5cm}@{}}
\toprule
\textbf{Sanction} & \textbf{\%} & \textbf{Effect} \\
\midrule
警告 (warning) & 35\% & Promotion frozen 1~yr \\
严重警告 (severe warning) & 28\% & Frozen 1~yr + stigma \\
撤职 (dismissal) & 4\% & Position removed \\
留党察看 (probation) & 10\% & Career suspended 1--2~yr \\
开除党籍 (expulsion) & 23\% & Permanent exit from $K_{\textup{cadre}}$ \\
\bottomrule
\end{tabular}
\end{center}

\noindent
Data: $101$ cases of 副科级 and 正科级 cadres sanctioned
between 1993 and 2009 \cite{zhongxian}.  The first two
sanctions ($63\%$ of cases) are \emph{reflecting} barriers:
the trajectory hits the boundary and returns with a delay.
Expulsion ($23\%$) is a \emph{terminal absorbing} state.
The problem type distribution is: economic $57\%$,
political $35\%$, lifestyle $8\%$.
\end{definition}

\begin{proposition}[一票否决 as hard constraint surface]
\label{prop:one-vote-veto}
The 一票否决 (one-vote veto) system for family planning and
social stability imposes a \emph{hard constraint surface}
within $K_{\textup{cadre}}$: a single violation in these
domains terminates viability regardless of all other performance
metrics.  Formally, let $x_{\textup{FP}}$ and
$x_{\textup{stab}}$ denote the family-planning and
social-stability components of the state.
The one-vote veto defines
\[
  K_{\textup{veto}}
  \;=\;
  \bigl\{\, x \in K_{\textup{cadre}} \;:\;
  x_{\textup{FP}} \geq \epsilon_{\textup{FP}},\;\;
  x_{\textup{stab}} \geq \epsilon_{\textup{stab}}
  \,\bigr\},
\]
and $K_{\textup{veto}} \subsetneq K_{\textup{cadre}}$: the
veto surfaces are strictly interior to the cadre kernel.  A
trajectory that crosses $x_{\textup{FP}} = \epsilon_{\textup{FP}}$
or $x_{\textup{stab}} = \epsilon_{\textup{stab}}$ exits
$K_{\textup{veto}}$ even if all other coordinates remain
safely interior.

Feng documents extreme consequences of this hard surface in
the family-planning domain: township cadres deployed
coercive enforcement (强制措施) precisely because the one-vote
veto transformed a policy variable into an absorbing barrier.
The severity of enforcement is not a moral choice but a
viability response: with the veto in place, failing the
family-planning target is equivalent to career death, and
the tangential condition demands maximal actuation at the
constraint surface.
\end{proposition}

\begin{remark}[政绩同构: the self-similar kernel]
\label{rem:isomorphic-kernel}
The target management system (目标考核) assigns numerical
scores to every administrative level.  Feng documents that
the target structure is \emph{isomorphic} across scales:
the province sets targets for the city, which sets identical
targets for the county, which sets identical targets for the
township, which sets identical targets for the village.
He calls this 政绩同构 (isomorphic performance structure).

In the viability framework, this means the viability kernel
has a \emph{self-similar} structure: the constraint set at
each scale is a rescaled copy of the constraint set at the
next scale up.  If $K^{(s)}$ is the kernel at scale~$s$
(province, city, county, township, village), then
$K^{(s)} \cong \phi_s(K^{(s+1)})$ for a scaling map
$\phi_s$.

This self-similarity has a pathological consequence:
the tangential condition at each scale generates the
\emph{same} pressure, producing fake achievements (假政绩)
at every level simultaneously.  Feng documents four county
party secretaries in succession whose ``signature projects''
(工程) were largely fabricated or failed: pig pens, forced
industrialisation, agricultural campaigns.  The self-similar
kernel means that viability pressure is not local---it
cascades identically from province to village.  The system
does not produce fake achievements because cadres are
dishonest; it produces them because the tangential condition
at a self-similar boundary demands actuation that the local
dynamics cannot supply, and fabrication is the cheapest
correction $g(x_G)$ available.
\end{remark}

\begin{remark}[The dual-ring model is max-flow/min-cut]
\label{rem:dual-ring-maxflow}
The dual-ring dynamics can be restated in the flow-theoretic
language of the agentic calculus (\cref{sec:flow}).
The inner ring defines the \emph{capacity constraints} on
the promotion flow network: the number of positions at each
level, the term limits, the age cutoffs.  The outer ring
defines the \emph{actual flow}: which cadres move through
which edges.  The max-flow/min-cut theorem
(\cref{thm:flowcut}) then states:
\begin{center}
\emph{The maximum promotion flow through the cadre lattice
equals the minimum cut capacity of the institutional
constraints.}
\end{center}
The min-cut is the set of bottleneck positions
(the cradles of \cref{sec:zx-cradle}) through which
promotion flow must pass.  The guanxi dynamics of the
outer ring are the routing algorithm: they determine which
cadres are assigned to which edges.  The knife---the
critical threshold separating viable cadres from non-viable
ones---is the capacity of the min-cut, which by
\cref{thm:meanfield} is determined by the mean guanxi
level~$\bar{U}_G$.

Feng's dissertation, read through this lens, is a
complete empirical map of the min-cut structure: the cradle
institutions are the bottleneck edges, the political
families are cooperative subflows that locally exceed the
mean capacity, the vote-canvassing networks (拉票网) are
the routing signals, and the disciplinary system is the
boundary enforcement that removes flows violating the
capacity constraints.  The knife is the mean.
\end{remark}

\chapter{GovFi --- Transparent Controlled Absorption}\label{app:govfi}

The preceding appendices applied the framework at successive scales:
to individual agents and historical cases
(\cref{app:sources,app:secondsex,app:threeli}),
to the system itself as a cooperative triad (\cref{app:triad}),
and to the internal structure of a bureaucratic hierarchy
(\cref{app:zhongxian}).  This appendix applies it to the
\emph{fiscal substrate}---the sovereign debt dynamics that
underlie every agent's viability kernel.
The central result is an \emph{absorption conservation law}:
total losses from non-performing debt are conserved; only
their distribution across agents is free.  The policy
implication is that the only freedom in a debt crisis \cite{reinhartrogoff} is the
\emph{choice of exit face}---the boundary through which the
system leaves its viability kernel---and that a transparent
ledger (GovFi) converts this choice from an uncontrolled
collapse into a computed optimisation.

% ================================================================
\section{Sovereign debt as knife}\label{sec:gf-knife}
% ================================================================

\begin{definition}[Fiscal state space]\label{def:fiscal-state}
A \emph{fiscal state space} is a tuple
$(S_F,\, D,\, r,\, \tau,\, w_F)$ where:
\begin{enumerate}[label=(\roman*)]
  \item $S_F \subseteq \R^n$ is the macroeconomic state;
  \item $D(t) \geq 0$ is the outstanding sovereign debt;
  \item $r > 0$ is the effective interest rate on $D$;
  \item $\tau(t) \geq 0$ is the fiscal revenue (tax and
    non-tax) at time~$t$;
  \item $w_F(t) \geq 0$ is the \emph{taxable capacity}
    (the water level of the fiscal system;
    see \cref{def:fiscal-water}).
\end{enumerate}
The debt dynamics are
\begin{equation}\label{eq:debt-dynamics}
  D'(t) \;=\; r\, D(t) \;+\; \delta(t) \;-\; \tau(t),
\end{equation}
where $\delta(t) \geq 0$ is the primary deficit
(new borrowing net of non-interest expenditure).
\end{definition}

\begin{proposition}[Debt satisfies the knife definition]
\label{prop:debt-knife}
Sovereign debt $D(t)$ satisfying \eqref{eq:debt-dynamics}
is a \textcolor{knife}{knife} in the sense of
\cref{def:knife}:
\begin{enumerate}[label=(\roman*)]
  \item \textbf{Autonomous actuation.}  The interest term
    $rD(t)$ drives the state without any agent's deliberate
    action: debt compounds autonomously.  Even with
    $\delta(t) = 0$ (no new borrowing), the dynamics
    $D' = rD - \tau$ are self-actuating whenever $rD > \tau$.
  \item \textbf{Observability.}  Sovereign debt is observable
    through the bond market: yields, credit spreads, and CDS
    prices provide continuous observation of the system's
    fiscal state.
\end{enumerate}
\end{proposition}

\begin{proof}
Condition~(i) follows from the structure of
\eqref{eq:debt-dynamics}: the right-hand side contains the
term $rD$ which depends on the state $D$ alone, providing
autonomous actuation as a differential inclusion
(\cref{rem:knife-di}).  Condition~(ii) is structural:
sovereign bonds are traded instruments, and market prices
encode the collective observation $\Obs(D)$ continuously.
\end{proof}

\begin{remark}[The self-financing loop]\label{rem:debt-loop}
When $rD(t) > \tau(t)$, the government must borrow to
service existing debt: $\delta(t) > 0$ is forced by the
dynamics, not chosen.  This is a positive feedback loop:
$D \uparrow \;\Rightarrow\; rD \uparrow \;\Rightarrow\;
\delta \uparrow \;\Rightarrow\; D \uparrow$.
In the language of \cref{thm:triangle}, debt in this regime
is a self-reinforcing triangle: the knife actuates itself,
generates the conditions for further actuation, and
compounds without external input.
\end{remark}

\begin{corollary}[Knife lifecycle for debt]
\label{cor:debt-lifecycle}
By \cref{thm:lifecycle}, every knife admits exactly two
resolution paths:
\begin{enumerate}[label=(\alph*)]
  \item \textbf{Controlled restructuring} (path~(a) of
    \cref{thm:lifecycle}): the sovereign deliberately
    reduces $D$ through haircuts, inflation, taxation, or
    growth, relinquishing the knife before it becomes
    lethal.
  \item \textbf{Systemic collapse} (path~(b) of
    \cref{thm:lifecycle}): the sovereign fails to act,
    and autonomous dynamics ($rD \to \infty$ or
    $w_F \to 0$) force a disorderly default.
\end{enumerate}
By \cref{thm:fixedpoint}, no third path exists:
extend-and-pretend (relabelling non-performing loans,
rolling over debt indefinitely) is not a resolution
but a deferral that preserves the knife's autonomous
actuation while degrading observability.
\end{corollary}

% ================================================================
\section{Fiscal water dynamics}\label{sec:gf-water}
% ================================================================

\begin{definition}[Fiscal water]\label{def:fiscal-water}
The \emph{fiscal water level} is
$w_F(t) := \tau_{\max}(t) - rD(t)$,
where $\tau_{\max}(t)$ is the maximum sustainable tax
revenue (the Laffer ceiling) at time~$t$.  This is the
fiscal instantiation of \cref{def:water}: $w_F$ measures
the remaining capacity of the fiscal system to absorb shocks
before the debt dynamics become unsustainable.
The \emph{fiscal viability kernel} is
\begin{equation}\label{eq:fiscal-kernel}
  K_F \;=\; \bigl\{\,(w_F,\, D) \;:\;
  w_F \geq \epsilon_w,\;\;
  D \leq D_{\max}(w_F)\,\bigr\},
\end{equation}
where $\epsilon_w > 0$ is the minimum fiscal buffer and
$D_{\max}(w_F) = (\tau_{\max} - \epsilon_w)/r$ is the
maximum sustainable debt given the water level.
\end{definition}

\begin{proposition}[Du Mu's theorem for fiscal systems]
\label{prop:fiscal-dumu}
If the interest burden exceeds sustainable revenue,
$rD(t) > \tau_{\max}(t)$,
then $w_F(t) < 0$ and the fiscal state exits $K_F$.
By \cref{thm:dumu} (Du Mu's theorem) applied to the
fiscal system: the extraction rate ($rD$) exceeding the
regeneration rate ($\tau_{\max}$) implies
$w_F(t) \to -\infty$, and by \cref{prop:binary}, the
outcome is binary---restructuring or collapse.
\end{proposition}

\begin{proof}
From \cref{def:fiscal-water},
$w_F' = \tau_{\max}' - rD' = \tau_{\max}' - r(rD + \delta - \tau)$.
When $rD > \tau_{\max} \geq \tau$, the term $-r(rD - \tau) < 0$
dominates, giving $w_F' < 0$ even under optimistic revenue
growth $\tau_{\max}' > 0$.  The fiscal water level declines
monotonically, exiting $K_F$ in finite time.
\end{proof}

\begin{center}
\renewcommand{\arraystretch}{1.25}
\begin{tabular}{@{}lll@{}}
\toprule
\textbf{Framework} & \textbf{Fiscal domain}
  & \textbf{Chinese case (城投)} \\
\midrule
Water (\cref{def:water})
  & Taxable capacity $w_F$
  & Local government revenue \\
Extraction
  & Interest burden $rD$
  & Coupon payments on 城投债 \\
Collapse
  & $w_F \to 0$
  & Revenue $<$ debt service \\
Pawn $\to$ knife
  & Fiscal tool $\to$ threat
  & Infrastructure bond $\to$ NPL \\
King absorbed
  & Sovereign creditworthiness
  & Central government guarantee \\
\bottomrule
\end{tabular}
\end{center}

% ================================================================
\section{The absorption conservation law}\label{sec:gf-conservation}
% ================================================================

This section establishes the central result: losses from
non-performing debt are conserved.  They can be distributed
across agents, deferred in time, or transformed in form,
but they cannot be destroyed.

\begin{definition}[Loss field]\label{def:loss-field}
The \emph{loss field} at time~$t$ is
\begin{equation}\label{eq:loss-field}
  \mathcal{L}(t) \;:=\;
  \max\bigl(D(t) - V_{\textup{rec}}(t),\; 0\bigr),
\end{equation}
where $V_{\textup{rec}}(t)$ is the present value of
recoverable cash flows from the assets financed by $D$.
The loss field measures the gap between the debt's face
value and the economic value of its underlying assets.
\end{definition}

\begin{theorem}[Absorption conservation]\label{thm:absorption-conservation}
Let $\mathcal{L}(t_0) > 0$ be the loss field at the time
the debt is recognised as non-performing.  Assume the
transversality (solvency) condition
$\lim_{T \to \infty} D(T)\, e^{-rT} = 0$: infinite
deferral is not feasible.  Let
$\ell_1, \ldots, \ell_m$ be the losses ultimately absorbed
by $m$ agents (creditors, taxpayers, depositors, debtors,
currency holders).  Then
\begin{equation}\label{eq:absorption-conservation}
  \sum_{i=1}^{m} \ell_i \;=\; \mathcal{L}(t_0).
\end{equation}
Total losses are conserved; only the allocation vector
$(\ell_1, \ldots, \ell_m)$ is free.
\end{theorem}

\begin{proof}
Every unit of the non-recoverable debt $\mathcal{L}(t_0)$
must ultimately be absorbed through one of three
exhaustive channels:
\begin{enumerate}[label=(\roman*)]
  \item \textbf{Service from fiscal capacity.}
    The sovereign services the debt from tax revenue:
    $\ell_{\textup{tax}} = \int_{t_0}^{T}
    (\tau(s) - \tau_0(s))\, e^{-r(s - t_0)}\, ds$,
    where $\tau_0$ is the baseline (pre-crisis) revenue
    and the excess $\tau - \tau_0$ is the additional burden
    on taxpayers.
  \item \textbf{Restructuring.}
    Creditors accept a haircut: $\ell_{\textup{hair}} =
    \alpha \cdot D(t_0)$ for haircut fraction $\alpha$.
    Depositors absorb losses through bail-in.
    Currency holders absorb losses through inflation:
    $\ell_{\textup{inf}} = D(t_0)(1 - 1/P(T))$
    where $P(T)$ is the cumulative price level.
  \item \textbf{Deferral.}
    The sovereign rolls over the non-performing debt:
    $\mathcal{L}(t_0) \to \mathcal{L}(t_0) \cdot e^{r(T - t_0)}$
    at a future date~$T$.  Deferral does not absorb the loss;
    it compounds it (see \cref{rem:deferral}).
\end{enumerate}
Channels~(i) and~(ii) are absorptive: they reduce
$\mathcal{L}$ by transferring the loss to a specific agent.
Channel~(iii) is non-absorptive: it preserves
$\mathcal{L}$ (in present-value terms) while compounding
the nominal amount.
Since every unit of $\mathcal{L}(t_0)$ must eventually
pass through channel~(i) or~(ii)---infinite deferral via
channel~(iii) is precluded by the transversality condition
$\lim_{T \to \infty} D(T) e^{-rT} = 0$ required for
fiscal solvency---the total absorbed losses equal
$\mathcal{L}(t_0)$.
\end{proof}

\begin{remark}[Deferral is not avoidance]\label{rem:deferral}
Channel~(iii) in the proof deserves emphasis.  Deferral
preserves the loss field in discounted terms while
\emph{compounding} it in nominal terms:
$\mathcal{L}(T) = \mathcal{L}(t_0) \cdot e^{r(T - t_0)}$.
This connects to \cref{def:dissipation}: the dissipation
rate of the fiscal system is $d_F = r \cdot \mathcal{L}$,
positive and increasing under deferral.  In the contact
flow of \cref{eq:contactflow}, deferral means the
system traverses the loss manifold without discharging;
the contact form accumulates, and the eventual discharge
is larger.
Every extend-and-pretend policy is a bet that growth will
raise $V_{\textup{rec}}$ faster than interest compounds
$\mathcal{L}$.  The historical record (\cref{sec:gf-history})
shows this bet fails more often than it succeeds.
\end{remark}

\begin{proposition}[The absorption simplex]\label{prop:loss-allocation}
The set of feasible loss allocations is the simplex
\begin{equation}\label{eq:absorption-simplex}
  \Delta_{\mathcal{L}}
  \;:=\;
  \Bigl\{\,(\ell_1, \ldots, \ell_m) \in \R^m_{\geq 0}
  \;:\; \sum_{i=1}^{m} \ell_i = \mathcal{L}(t_0)\,\Bigr\}.
\end{equation}
Every resolution of a debt crisis is a point on
$\Delta_{\mathcal{L}}$.  Different resolutions
(haircut, inflation, taxation, bail-in) correspond to
different vertices or interior points of this simplex.
The conservation law
(\cref{thm:absorption-conservation}) states that
the simplex is the \emph{entire} feasible set: no
resolution can place the allocation outside
$\Delta_{\mathcal{L}}$.
\end{proposition}

\begin{proposition}[Expected absorption time]
\label{prop:absorption-time}
Suppose the sovereign activates the feedback control
$C(x)$ (\cref{def:feedback}) at time~$t_0$ and implements
a restructuring schedule with absorption rate
$a(t) \geq 0$ (the rate at which losses are discharged
through channels~(i) and~(ii) of
\cref{thm:absorption-conservation}).
The residual loss field satisfies the
\emph{delay integro-differential equation}
\begin{equation}\label{eq:loss-dide}
  \mathcal{L}'(t)
  \;=\;
  r\,\mathcal{L}(t)
  \;-\; a\bigl(t - \delta_{\textup{lag}}\bigr),
  \qquad t > t_0,
\end{equation}
where $\delta_{\textup{lag}} \geq 0$ is the political
delay between recognising a loss and implementing its
absorption (legislative process, legal restructuring,
bond exchange).  The absorption time is
$T^* = \inf\{t > t_0 : \mathcal{L}(t) = 0\}$.
\begin{enumerate}[label=(\alph*)]
  \item \textbf{With control.}  If $C(x)$ is active and
    the mean absorption rate $\bar{a}$ after the lag period
    satisfies $\bar{a} > r\,\mathcal{L}(t_0)\,
    e^{r\delta_{\textup{lag}}}$ (absorption outpaces the
    loss compounded through the lag), then
    $T^*$ is finite and bounded by
    \begin{equation}\label{eq:absorption-bound}
      T^* - t_0
      \;\leq\;
      \delta_{\textup{lag}}
      \;+\;
      \frac{\mathcal{L}(t_0)\, e^{r\delta_{\textup{lag}}}}
           {\bar{a} - r\,\mathcal{L}(t_0)\,
           e^{r\delta_{\textup{lag}}}},
    \end{equation}
    where $\bar{a} := (T^* - t_0 -
    \delta_{\textup{lag}})^{-1}\int_{t_0 +
    \delta_{\textup{lag}}}^{T^*} a(s)\, ds$.
  \item \textbf{Without control.}  If $C(x)$ is not
    activated ($a(t) = 0$), then
    $\mathcal{L}(t) = \mathcal{L}(t_0)\, e^{r(t - t_0)}$
    and $T^* = +\infty$: the loss field grows
    exponentially and absorption never occurs.
\end{enumerate}
\end{proposition}

\begin{proof}
\textbf{(a)}  During the lag period $[t_0, t_0 +
\delta_{\textup{lag}}]$, $a = 0$ and
$\mathcal{L}(t_0 + \delta_{\textup{lag}})
= \mathcal{L}(t_0)\, e^{r\delta_{\textup{lag}}}$: the loss
compounds unabsorbed.  After the lag, write
$L(t) = \mathcal{L}(t_0 + \delta_{\textup{lag}} + t)$ for
$t \geq 0$.  Then $L'(t) = rL(t) - a(t)$ with
$a(t) > rL(t)$, so $L'(t) < 0$.  The time to reach $L = 0$
from $L(0) = \mathcal{L}(t_0) e^{r\delta_{\textup{lag}}}$
is bounded by $L(0)/(\bar{a} - rL(0))$ (linear comparison),
giving \eqref{eq:absorption-bound}.

\textbf{(b)}  With $a \equiv 0$, \eqref{eq:loss-dide}
reduces to $\mathcal{L}' = r\mathcal{L}$, whose solution
is $\mathcal{L}(t) = \mathcal{L}(t_0) e^{r(t-t_0)}$.
Since $\mathcal{L}(t_0) > 0$ and $r > 0$, the loss field
diverges: $T^* = +\infty$.
\end{proof}

\begin{remark}[Expectation is just expectation]
\label{rem:expectation}
The bound \eqref{eq:absorption-bound} is conditional on
$C(x)$ being active: the absorption rate $a(t)$ requires
a sovereign that has \emph{chosen} to restructure.
The expected time to resolution is not a forecast; it is
a \emph{conditional computation} that presupposes the
political act of choosing an exit face
(\cref{sec:gf-exitface}).  Without that act,
\cref{prop:absorption-time}(b) applies and
$T^* = +\infty$---not because the mathematics fails,
but because the input to the equation ($a(t) = 0$) encodes
the absence of action.

An expected absorption time is just an expectation,
until you actually do it.
\end{remark}

\begin{proposition}[Stochastic loss dynamics]
\label{prop:stochastic-loss}
Under stochastic fiscal revenue, the loss field satisfies
the It\^o equation
\begin{equation}\label{eq:stochastic-loss}
  d\mathcal{L}(t)
  \;=\;
  \bigl(r\,\mathcal{L}(t) - a(t)\bigr)\,dt
  \;+\; \sigma\,dW(t),
\end{equation}
where $\sigma > 0$ captures revenue volatility and
$W(t)$ is a standard Wiener process.  With constant
absorption rate $a > r\mathcal{L}_0$ under active
control, the first-passage time
$T^* = \inf\{t : \mathcal{L}(t) \leq 0\}$
is approximately inverse Gaussian (the approximation
treats the drift as constant at
$\mu = -(a - r\mathcal{L}_0)$, valid when
$a \gg r\mathcal{L}_0$):
\begin{equation}\label{eq:first-passage-moments}
  \mathbb{E}[T^*]
  \;=\; \frac{\mathcal{L}_0}{a - r\mathcal{L}_0},
  \qquad
  \mathrm{Var}(T^*)
  \;=\; \frac{\sigma^2\, \mathcal{L}_0}
    {(a - r\mathcal{L}_0)^3}.
\end{equation}
The density $p(\ell, t)$ of $\mathcal{L}(t)$ satisfies
the Fokker--Planck equation
\begin{equation}\label{eq:fokker-planck}
  \partial_t\, p
  \;=\;
  -\partial_\ell\bigl[(r\ell - a)\, p\bigr]
  \;+\; \tfrac{\sigma^2}{2}\,\partial_{\ell\ell}\, p,
\end{equation}
with absorbing boundary $p(0, t) = 0$.
The variance \eqref{eq:first-passage-moments} reveals a
cubic amplification: weak absorption
($a \approx r\mathcal{L}_0$) amplifies revenue noise as
$(a - r\mathcal{L}_0)^{-3}$.  A system that compounds
fast and absorbs slowly is maximally exposed to
stochastic shocks.
\end{proposition}

\begin{proof}
In the regime $a \gg r\mathcal{L}$, the drift of
\eqref{eq:stochastic-loss} is approximately constant at
$\mu := -(a - r\mathcal{L}_0) < 0$, reducing the process
to Brownian motion with negative drift absorbed at the
origin.  The first-passage time of such a process from
level $\mathcal{L}_0$ to $0$ is inverse Gaussian with the
stated moments.  The Fokker--Planck equation
\eqref{eq:fokker-planck} is the forward Kolmogorov
equation of \eqref{eq:stochastic-loss}.
\end{proof}

\begin{theorem}[Optimal restructuring trigger]
\label{thm:optimal-trigger}
Let $\kappa > 0$ be the total restructuring cost
(political cost plus the minimised damage functional
$\mathcal{D}^*$ of \cref{thm:exit-choice}), assumed
constant, and let $\rho > 0$ be the sovereign's discount
rate.  The sovereign minimises the total expected cost
\[
  J(\tau) \;=\;
  \mathbb{E}\!\left[\int_{t_0}^{\tau}
  e^{-\rho(t - t_0)}\, r\,\mathcal{L}(t)\, dt
  \;+\; e^{-\rho(\tau - t_0)}\,\kappa\right]
\]
over stopping times $\tau$.  Then:
\begin{enumerate}[label=(\alph*)]
  \item The optimal trigger $\mathcal{L}^*$ is
    \emph{strictly interior} to $K_F$:
    $\mathcal{L}^* < \mathcal{L}_{\max}$.  Waiting
    until the boundary $\partial K_F$ is suboptimal.
  \item At $\mathcal{L}^*$, the value-matching and
    smooth-pasting conditions hold:
    $V(\mathcal{L}^*) = \kappa$ and
    $V'(\mathcal{L}^*) = 0$,
    where $V$ is the optimal continuation value.
  \item Higher volatility $\sigma$ lowers the trigger:
    $\partial \mathcal{L}^* / \partial \sigma < 0$.
    More uncertain revenue demands earlier action.
\end{enumerate}
\end{theorem}

\begin{proof}
\textbf{(a)}  The carrying cost $r\mathcal{L}\,dt$
accumulated per unit time grows with $\mathcal{L}$;
the marginal benefit of delay (discounting the political
cost $\kappa$) is bounded by $\rho\kappa\,dt$.  When
$r\mathcal{L} > \rho\kappa$, delaying is strictly
dominated.  This occurs at an interior point
$\mathcal{L}^* \leq \rho\kappa/r < \mathcal{L}_{\max}$.

\textbf{(b)}  Standard optimal stopping: the value
function must be $C^1$ at the free boundary to prevent
arbitrage between continuation and stopping.

\textbf{(c)}  Higher $\sigma$ increases the probability
that $\mathcal{L}$ jumps past $\mathcal{L}_{\max}$
(forced default, cost $> \kappa$) before the sovereign
can respond.  The option value of delay decreases, and
the free boundary shifts inward.
\end{proof}

% ================================================================
\section{Choice of exit face}\label{sec:gf-exitface}
% ================================================================

\begin{definition}[Exit face decomposition]\label{def:exit-face}
The boundary $\partial K_F$ of the fiscal viability kernel
\eqref{eq:fiscal-kernel} decomposes into faces, each
corresponding to a distinct absorption channel:
\begin{enumerate}[label=(\roman*)]
  \item $\partial_{\textup{haircut}}$: creditor losses
    (debt restructuring, write-downs, bail-in);
  \item $\partial_{\textup{inflation}}$: currency-holder
    losses (monetary expansion, real devaluation);
  \item $\partial_{\textup{tax}}$: taxpayer losses
    (fiscal austerity, increased taxation);
  \item $\partial_{\textup{default}}$: disorderly failure
    (uncontrolled capital flight, banking collapse,
    social instability).
\end{enumerate}
Each face $\partial_k K_F$ corresponds to a vertex of the
absorption simplex $\Delta_{\mathcal{L}}$ where agent
class~$k$ absorbs the entirety of $\mathcal{L}(t_0)$.
Interior points of $\Delta_{\mathcal{L}}$ correspond to
mixed resolutions that distribute losses across faces.
\end{definition}

\begin{theorem}[Controlled vs.\ uncontrolled exit]
\label{thm:exit-choice}
Let $c_i > 0$ be the social cost coefficient of imposing
loss $\ell_i$ on agent class~$i$.  Define the damage
functional
\begin{equation}\label{eq:damage-functional}
  \mathcal{D}(\ell) \;:=\;
  \sum_{i=1}^{m} c_i\, \ell_i^2.
\end{equation}
Then:
\begin{enumerate}[label=(\alph*)]
  \item \textbf{Controlled exit.}  If the sovereign
    activates the feedback function $C(x)$
    (\cref{def:feedback}) at the boundary $\partial K_F$,
    the optimal allocation minimises $\mathcal{D}$ on
    $\Delta_{\mathcal{L}}$:
    \begin{equation}\label{eq:optimal-allocation}
      \ell_i^* \;=\;
      \frac{\mathcal{L}(t_0)}{c_i\,
      \displaystyle\sum_{j=1}^{m} c_j^{-1}},
    \end{equation}
    distributing losses inversely proportional to social
    cost.  The exit face is chosen deliberately, and
    the trajectory satisfies the viability condition
    (\cref{thm:viability-di}) throughout the restructuring.
  \item \textbf{Uncontrolled exit.}  If the sovereign fails
    to activate $C(x)$, the autonomous debt dynamics
    ($D' = rD + \delta - \tau$ with $rD > \tau$) drive
    the state toward $\partial_{\textup{default}}$, which
    is generically the worst-case face: it maximises
    $\mathcal{D}$ by concentrating losses on the most
    vulnerable agents (depositors, small creditors,
    the poor through inflation).
\end{enumerate}
\end{theorem}

\begin{proof}
\textbf{(a)}  Minimise $\mathcal{D}(\ell) =
\sum_i c_i \ell_i^2$ subject to $\sum_i \ell_i =
\mathcal{L}(t_0)$ and $\ell_i \geq 0$.
The Lagrangian is
$L = \sum_i c_i \ell_i^2 -
\lambda(\sum_i \ell_i - \mathcal{L}(t_0))$.
First-order conditions: $2 c_i \ell_i = \lambda$ for all~$i$,
giving $\ell_i = \lambda / (2c_i)$.
Substituting into the constraint:
$\sum_i \lambda/(2c_i) = \mathcal{L}(t_0)$,
so $\lambda = 2\mathcal{L}(t_0) / \sum_j c_j^{-1}$,
yielding \eqref{eq:optimal-allocation}.
The second-order condition ($\nabla^2 \mathcal{D}$
positive definite on the constraint surface) is satisfied
since $c_i > 0$.

\textbf{(b)}  Without active control, the dynamics follow
the autonomous differential inclusion.  The face
$\partial_{\textup{default}}$ is the \emph{attractor} of
the uncontrolled dynamics: when $rD > \tau$ and no
restructuring occurs, $D$ grows exponentially and the
system exits through disorderly default.  This face
concentrates losses on agents with the least political
power to resist (depositors lose savings, workers lose
employment, currency holders lose purchasing power),
which by the structure of the cost coefficients $c_i$
corresponds to high-cost agents, maximising $\mathcal{D}$.
\end{proof}

\begin{remark}[Courage is the choice function]
\label{rem:courage}
The difference between Iceland (2008) and Japan (1990s)
is not economics.  Iceland's GDP per capita was a fraction
of Japan's; its financial system was less sophisticated;
its policy toolkit was smaller.  The difference is the
\emph{presence or absence of the choice function}
$C(x)$ at the boundary.
Iceland chose $\partial_{\textup{haircut}}$: creditors
(including foreign depositors) absorbed losses.  Japan
refused to choose, deferring via extend-and-pretend.
In the framework, ``courage'' is not a moral quality but
a mathematical one: it is the activation of $C(x)$---the
feedback control---at the boundary $\partial K_F$.
A sovereign that activates $C(x)$ can compute and
implement \eqref{eq:optimal-allocation}.  A sovereign
that does not is governed by autonomous dynamics, which
by \cref{thm:exit-choice}(b) generically select the
worst exit face.
\end{remark}

\begin{theorem}[Dynamic absorption path]\label{thm:dynamic-allocation}
The sovereign chooses absorption rates
$a_i(t) \geq 0$ for each channel
$i \in \{1, \ldots, m\}$ to minimise the discounted
quadratic cost
\[
  J \;=\; \int_0^{\infty}
  e^{-\rho t}\sum_{i=1}^{m} c_i\, a_i(t)^2\, dt,
\]
subject to
$\mathcal{L}'(t) = r\,\mathcal{L}(t) - \sum_i a_i(t)$
with $\mathcal{L}(0) = \mathcal{L}_0$.
Assume $\rho < r$ (the sovereign is more patient than
the debt is aggressive).  Then:
\begin{enumerate}[label=(\alph*)]
  \item The value function is
    $V(\mathcal{L}) = \frac{1}{2}\beta\,\mathcal{L}^2$
    with $\beta = 2(2r - \rho)/H$, where
    $H = \sum_{j=1}^m c_j^{-1}$.
  \item The Hamilton--Jacobi--Bellman equation is
    \begin{equation}\label{eq:hjb}
      \rho\, V(\mathcal{L})
      \;=\;
      V'(\mathcal{L})\, r\,\mathcal{L}
      \;-\; \frac{H}{4}\,\bigl(V'(\mathcal{L})\bigr)^2.
    \end{equation}
  \item The optimal absorption rate through channel~$i$
    is
    \begin{equation}\label{eq:optimal-rate}
      a_i^*(t) \;=\; \frac{(2r - \rho)}{c_i\, H}\,
      \mathcal{L}(t).
    \end{equation}
    The allocation ratio $a_i^*/a_j^* = c_j/c_i$ is
    time-independent---the dynamic problem does not change
    \emph{who} pays, only \emph{how fast}.
  \item Under the optimal policy, the loss field decays
    exponentially:
    \begin{equation}\label{eq:optimal-decay}
      \mathcal{L}(t) \;=\;
      \mathcal{L}_0\, e^{-(r - \rho)\, t}.
    \end{equation}
    The decay rate $r - \rho$ is the net urgency:
    compounding rate minus impatience.  A patient
    sovereign ($\rho$ small) absorbs faster.
\end{enumerate}
\end{theorem}

\begin{proof}
\textbf{(b)}  The HJB equation for the
infinite-horizon discounted problem with state dynamics
$\mathcal{L}' = r\mathcal{L} - \sum a_i$ and running
cost $\sum c_i a_i^2$ is
$\rho V = \min_{a \geq 0}\bigl[\sum c_i a_i^2 +
V'(r\mathcal{L} - \sum a_i)\bigr]$.
First-order condition: $2c_i a_i = V'$, giving
$a_i = V'/(2c_i)$.  Substituting into the HJB:
\[
  \rho V
  = \sum_i \frac{(V')^2}{4c_i}
    + V'\Bigl(r\mathcal{L}
    - \frac{V'}{2}\sum_i c_i^{-1}\Bigr)
  = V' r\mathcal{L} - \frac{H}{4}(V')^2.
\]

\textbf{(a)}  Substitute $V = \tfrac{1}{2}\beta
\mathcal{L}^2$ into \eqref{eq:hjb}:
$\tfrac{1}{2}\rho\beta\mathcal{L}^2 =
\beta r \mathcal{L}^2 -
\tfrac{H}{4}\beta^2\mathcal{L}^2$,
giving $\rho/2 = r - H\beta/4$, hence
$\beta = 2(2r - \rho)/H$.

\textbf{(c)}  From $a_i = V'/(2c_i) =
\beta\mathcal{L}/(2c_i)$, substituting~$\beta$.

\textbf{(d)}  Total absorption rate
$a^* = \beta H \mathcal{L}/2 = (2r - \rho)\mathcal{L}$.
The dynamics become
$\mathcal{L}' = r\mathcal{L} - (2r - \rho)\mathcal{L}
= -(r - \rho)\mathcal{L}$, with solution
$\mathcal{L}_0 e^{-(r - \rho)t}$.
\end{proof}

\begin{proposition}[Nash bargaining on the simplex]
\label{prop:nash-bargaining}
Suppose the $m$ agent classes negotiate the allocation
$(\ell_1, \ldots, \ell_m) \in \Delta_{\mathcal{L}}$
with bargaining powers $\alpha_i > 0$
($\sum \alpha_i = 1$).  Let $d_i > 0$ be the loss
agent~$i$ would suffer under uncontrolled default
($\partial_{\textup{default}}$ of
\cref{def:exit-face}), with
$\sum_i d_i > \mathcal{L}_0$ (default is inefficient:
it destroys more value than the actual loss).
The Nash bargaining solution is
\begin{equation}\label{eq:nash-bargaining}
  \ell_i^{\textup{Nash}}
  \;=\;
  d_i \;-\; \alpha_i\, S,
  \qquad
  S \;:=\; \sum_{j=1}^{m} d_j \;-\; \mathcal{L}_0,
\end{equation}
where $S > 0$ is the \emph{surplus from agreement}---the
total value saved by choosing a controlled exit face
over default.  Each agent's loss equals their default
loss minus their share of the surplus; stronger
bargaining power $\alpha_i$ secures a larger share
of~$S$.
\end{proposition}

\begin{proof}
The Nash bargaining solution maximises
$\prod_i (d_i - \ell_i)^{\alpha_i}$ subject to
$\sum \ell_i = \mathcal{L}_0$ and $\ell_i \leq d_i$.
Taking logarithms: maximise
$\sum \alpha_i \ln(d_i - \ell_i)$ subject to
$\sum \ell_i = \mathcal{L}_0$.
First-order condition:
$\alpha_i/(d_i - \ell_i) = \lambda$ for all~$i$,
giving $d_i - \ell_i = \alpha_i/\lambda$.
Summing: $S = 1/\lambda$, so
$\ell_i = d_i - \alpha_i S$.
\end{proof}

% ================================================================
\section{The guarantee network}\label{sec:gf-network}
% ================================================================

The preceding sections treat debt as a scalar $D(t)$.
In practice, 城投 debt is a \emph{network}: thousands
of local government financing vehicles connected by
cross-guarantees, interbank exposure, and shadow banking
channels.  The systemic risk is not the total $D$ but the
\emph{topology} of how losses propagate.

\begin{definition}[Guarantee graph]\label{def:guarantee-graph}
The \emph{guarantee graph} is a weighted directed graph
$G_{\textup{guar}} = (V, E, A)$ where:
\begin{enumerate}[label=(\roman*)]
  \item $V = \{1, \ldots, n\}$ is the set of fiscal
    entities (LGFVs, banks, local governments);
  \item $E \subseteq V \times V$ is the set of guarantee
    relationships;
  \item $A = (A_{ij})$ is the \emph{guarantee adjacency
    matrix}: $A_{ij} \geq 0$ is the fraction of
    entity~$j$'s debt guaranteed by entity~$i$.
\end{enumerate}
When entity $j$ defaults with loss $\ell_j$, entity~$i$
absorbs $A_{ij}\, \ell_j$.  If this pushes $i$ into
default, $i$'s loss propagates along the edges of
$G_{\textup{guar}}$.
\end{definition}

\begin{theorem}[Spectral contagion threshold]
\label{thm:spectral-contagion}
Let $\alpha \in (0, 1]$ be the loss-given-default rate
and $\rho(A)$ the spectral radius of the guarantee
adjacency matrix~$A$.
\begin{enumerate}[label=(\alph*)]
  \item \textbf{Subcritical}
    ($\alpha\, \rho(A) < 1$).
    The total systemic loss from an initial shock vector
    $\ell^{(0)} \in \R^n_{\geq 0}$ is
    \begin{equation}\label{eq:leontief}
      \ell^{(\infty)}
      \;=\;
      (I - \alpha\, A)^{-1}\, \ell^{(0)},
    \end{equation}
    which is finite.  The amplification factor is
    bounded by $1/(1 - \alpha\,\rho(A))$.
  \item \textbf{Supercritical}
    ($\alpha\, \rho(A) \geq 1$).
    The matrix $(I - \alpha A)$ is singular or has a
    non-positive eigenvalue: a finite initial shock
    produces unbounded cascading losses.
  \item \textbf{Connection to the mean field.}
    By the Perron--Frobenius theorem, $\rho(A)$ is
    bounded above by the maximum row sum
    $\max_i \sum_j A_{ij}$.  The spectral contagion
    threshold $\alpha\,\rho(A) = 1$ is the
    knife-is-the-mean theorem (\cref{thm:meanfield})
    applied to the guarantee network: the mean
    guarantee exposure determines whether the network
    is a tool (distributing losses) or a knife
    (amplifying them to systemic scale).
\end{enumerate}
\end{theorem}

\begin{proof}
\textbf{(a)}  After $k$ rounds of contagion, the
cumulative loss is
$\ell^{(k)} = \sum_{j=0}^{k}(\alpha A)^j \ell^{(0)}$.
The Neumann series converges iff
$\rho(\alpha A) = \alpha\rho(A) < 1$, in which case
the sum is $(I - \alpha A)^{-1}$ and
$\|(I - \alpha A)^{-1}\| \leq 1/(1 - \alpha\rho(A))$.

\textbf{(b)}  When $\alpha\rho(A) \geq 1$, the Neumann
series diverges: repeated application amplifies shocks
along the Perron eigenvector.

\textbf{(c)}  By Perron--Frobenius (applied to the
non-negative matrix~$A$),
$\rho(A) \leq \max_i \sum_j A_{ij}$.
\end{proof}

\begin{proposition}[Percolation threshold]
\label{prop:percolation-threshold}
Model each guarantee edge in $G_{\textup{guar}}$ as
\emph{active} (transmitting default) with probability
$p$ and inactive with probability $1 - p$.  Let $d$
be the mean degree of $G_{\textup{guar}}$.  There
exists a critical probability
\begin{equation}\label{eq:percolation-threshold}
  p_c \;\approx\; \frac{1}{d}
\end{equation}
such that:
\begin{enumerate}[label=(\alph*)]
  \item for $p < p_c$, the expected cascade from a
    single default is $O(\log n)$---contagion stays
    local;
  \item for $p > p_c$, the expected cascade is
    $O(n)$---contagion is systemic with positive
    probability.
\end{enumerate}
The percolation threshold is a second knife-is-the-mean
result: the critical probability depends on the mean
connectivity $d$ of the network, not on any individual
node.
\end{proposition}

\begin{proof}
By the theory of bond percolation on random graphs
(Erd\H{o}s--R\'enyi), the giant component emerges when
the mean number of active edges per node exceeds~$1$:
$pd > 1$.  Below this threshold, connected components
are $O(\log n)$ (subcritical).  Above it, a positive
fraction of nodes belongs to a single giant component
(supercritical).  The critical probability is
$p_c = 1/d$.
\end{proof}

% ================================================================
\section{The GovFi platform}\label{sec:gf-platform}
% ================================================================

\begin{definition}[GovFi ledger]\label{def:govfi-ledger}
A \emph{GovFi ledger} is a fiscal execution graph
(\cref{def:exgraph}) equipped with three properties:
\begin{enumerate}[label=(\roman*)]
  \item \textbf{Full observability.}
    $\Obs = \mathcal{F}$: every fiscal flow (revenue,
    expenditure, debt issuance, debt service, transfer) is
    recorded on-ledger and publicly queryable.  No hidden
    off-balance-sheet vehicles exist.
  \item \textbf{Programmatic breakpoints.}
    Smart contracts encode graduated restructuring
    triggers: when $D/\tau_{\max}$ crosses predefined
    thresholds $\theta_1 < \theta_2 < \cdots < \theta_n$,
    the corresponding restructuring protocol activates
    automatically (\cref{cor:breakpoint}).
  \item \textbf{Real-time loss tracking.}
    The loss field $\mathcal{L}(t)$
    (\cref{def:loss-field}) is computed continuously from
    on-ledger data and published as a public signal.
\end{enumerate}
\end{definition}

\begin{proposition}[Full observability eliminates hidden knives]
\label{prop:govfi-observability}
Under full observability ($\Obs = \mathcal{F}$), no debt
instrument can satisfy the knife definition
(\cref{def:knife}) \emph{without detection}.  By
\cref{prop:imperfect}, the gap between autonomous actuation
and detected actuation is
$\|U(x)\| - \|U(x) \cap \Obs(x)\|$.  Full observability
closes this gap to zero: every knife is visible the moment
it forms.

In the 城投 context, the primary failure mode is
\emph{hidden} debt---off-balance-sheet vehicles,
implicit guarantees, shadow banking channels.  GovFi's
full observability eliminates this failure mode by
construction: if it is not on the ledger, it does not
exist as a fiscal obligation.
\end{proposition}

\begin{remark}[Revelation and observability]
\label{rem:revelation}
Under standard mechanism design, if the sovereign
allocates losses based on self-reported fiscal states
$\hat{\mathcal{L}}_i$, every agent has incentive to
underreport: claiming smaller losses shifts the burden
to others.  The revelation principle guarantees that
any incentive-compatible allocation can be implemented
as a direct mechanism, but only with auditing or side
payments that introduce their own costs.

GovFi bypasses the revelation problem entirely.
With $\Obs = \mathcal{F}$, the ledger observes each
entity's $\mathcal{L}_i$ directly from on-chain data.
Self-reporting is unnecessary; incentive compatibility
is achieved not by mechanism design but by
\emph{architectural design}---the same full
observability that eliminates hidden knives
also eliminates strategic misreporting.
\end{remark}

\begin{proposition}[Smart contracts as breakpoints]
\label{prop:govfi-breakpoints}
The programmatic breakpoints define a graduated
restructuring schedule
$\sigma \colon [0, 1] \to \Delta_{\mathcal{L}}$,
mapping the debt stress level
$s = D / D_{\max}$ to a point on the absorption simplex.
At each threshold $\theta_k$:
\begin{enumerate}[label=(\roman*)]
  \item the restructuring protocol specifies which exit
    face receives how much loss: $\sigma(\theta_k) =
    (\ell_1^{(k)}, \ldots, \ell_m^{(k)})$;
  \item the activation is automatic: no political decision
    is required at the moment of crisis;
  \item the graduated structure ensures that early
    interventions ($\theta_1, \theta_2$) impose small,
    distributed losses, preventing the accumulation that
    would force a large, concentrated loss at
    $\partial_{\textup{default}}$.
\end{enumerate}
By \cref{cor:breakpoint}, the breakpoints convert sovereign
debt from a \textcolor{knife}{knife} (autonomous,
uncontrolled actuation) to a \textcolor{water}{tool}
(programmatic, controlled actuation): the autonomous
dynamics are interrupted by pre-committed restructuring
at each threshold.
\end{proposition}

\begin{proposition}[Critical political delay]\label{prop:delay-stability}
Under proportional absorption control
$a(t) = k\,\mathcal{L}(t - \delta_{\textup{lag}})$
with gain $k > r$, the linearised delay differential
equation
\begin{equation}\label{eq:delay-characteristic}
  \mathcal{L}'(t) \;=\;
  r\,\mathcal{L}(t)
  \;-\; k\,\mathcal{L}(t - \delta_{\textup{lag}})
\end{equation}
is stable if and only if
\begin{equation}\label{eq:critical-delay}
  \delta_{\textup{lag}}
  \;<\;
  \delta^*
  \;:=\;
  \frac{\arccos(r/k)}{\sqrt{k^2 - r^2}}.
\end{equation}
When $\delta_{\textup{lag}} > \delta^*$, a Hopf
bifurcation occurs: the delayed political response
overshoots the target, then undershoots, creating
oscillatory instability in the loss field.
\end{proposition}

\begin{proof}
The characteristic equation of
\eqref{eq:delay-characteristic} is
$\lambda = r - k\, e^{-\lambda\delta}$.
At the stability boundary $\lambda = i\omega$.
Separating real and imaginary parts:
$0 = r - k\cos(\omega\delta)$ and
$\omega = k\sin(\omega\delta)$.
The first gives $\cos(\omega\delta) = r/k$; from
$\cos^2 + \sin^2 = 1$:
$\omega = \sqrt{k^2 - r^2}$.
Then $\delta^* = \arccos(r/k)/\omega$, yielding
\eqref{eq:critical-delay}.
For $\delta < \delta^*$, all characteristic roots
satisfy $\mathrm{Re}(\lambda) < 0$.
\end{proof}

\begin{remark}[Time consistency via smart contracts]
\label{rem:time-consistency}
The Kydland--Prescott problem applies to fiscal
restructuring: a sovereign that can revise its
breakpoint thresholds $\theta_k$ ex post has an
incentive to defer.  At $t = 0$, the optimal plan says
``restructure at $\theta_1$.''  When
$D(t)/\tau_{\max}$ reaches $\theta_1$, the sovereign
re-optimises and finds deferral preferable (the
political cost is immediate; the compounding cost is
diffuse).  Iterating, the sovereign never acts---the
time-inconsistency trap.

The GovFi breakpoints (\cref{prop:govfi-breakpoints})
resolve this by encoding the restructuring schedule in
immutable smart contracts.  The schedule cannot be
revised at the moment of crisis because the code does
not accept revision.  Time consistency is enforced not
by reputation or institutional norms but by the
\emph{architecture} of the ledger: the same
immutability that provides observability also provides
commitment.
\end{remark}

\begin{remark}[GovFi collapses the dual ring]
\label{rem:govfi-dualring}
The dual-ring system (\cref{def:dual-ring}) in the fiscal
context has an inner ring (formal fiscal rules: balanced
budget laws, debt ceilings, Maastricht criteria) and an
outer ring (actual fiscal practice: off-budget spending,
implicit guarantees, creative accounting).  The pathology
documented in \cref{prop:outer-drives} applies: the outer
ring drives the inner ring, and the formal rules become
decoration.

GovFi collapses the dual ring by making the outer ring
observable.  When $\Obs = \mathcal{F}$, the distinction
between formal rules and actual practice vanishes: every
flow is on-ledger, and the ``outer ring'' of hidden
transactions ceases to exist.  The dual-ring dynamics
\eqref{eq:inner-ring}--\eqref{eq:outer-ring} reduce to
a single observable system, and the tangential condition
(\cref{thm:viability-di}) can be verified directly from
the ledger data.
\end{remark}

\begin{example}[水电站 --- GovFi worked example]
\label{ex:dam}
A provincial government issues bonds to build a
hydroelectric power station (水电站).
The total bond issuance is~$B$; the physical
construction cost is~$C < B$; the loss field is
$\mathcal{L} = B - C$.
The fiscal execution graph has three layers
and three interested parties.

\paragraph{三方 (Three parties).}

\begin{center}
\renewcommand{\arraystretch}{1.25}
\begin{tabular}{@{}clp{5.5cm}@{}}
\toprule
\textbf{Layer} & \textbf{Party}
  & \textbf{Role in flow graph} \\
\midrule
0 & 政府 (Government)
  & Source: issues bonds, disburses funds \\
1 & 公司 (Companies)
  & Routing: general contractor, subcontractors,
    suppliers \\
2 & 建设者 (Builders)
  & Sink: construction workers, engineers,
    equipment operators \\
\bottomrule
\end{tabular}
\end{center}

\begin{center}
\begin{tikzpicture}[
  % ── styles ──
  wbox/.style={rectangle, rounded corners=4pt,
    draw=water, thick, fill=water!8,
    minimum width=6.0cm, minimum height=1.0cm,
    align=center, font=\small},
  gfbox/.style={rectangle, rounded corners=6pt,
    draw=sword, thick, fill=sword!5,
    minimum height=1.0cm, align=center, font=\small},
  leak/.style={rectangle, rounded corners=4pt,
    draw=knife, thick, dashed, fill=knife!6,
    minimum width=2.6cm, minimum height=0.9cm,
    align=center, font=\small},
  brk/.style={rectangle, rounded corners=3pt,
    draw=caution, thick, fill=caution!8,
    minimum height=0.9cm, align=center, font=\small},
  arr/.style={-{Stealth[length=6pt]}, thick},
  darr/.style={-{Stealth[length=6pt]}, thick, dashed, knife},
  oarr/.style={-{Stealth[length=6pt]}, thick, sword},
  lbl/.style={font=\scriptsize, fill=white, inner sep=2pt},
]
  % ════════════════════════════════════════════
  %   PART 1 — Money flow (three layers)
  % ════════════════════════════════════════════

  % ── Layer 0: Government ──
  \node[wbox] (gov) at (0, 0)
    {\textcolor{water}{\textbf{Layer 0}}
     \;---\; 政府 (Gov): issues bonds $B$};

  % ── Layer 1: Companies — trumpet shape ──
  %    Anchor coordinates for the horn
  \coordinate (horn-top-L)  at (-1.2, -2.1);
  \coordinate (horn-top-R)  at ( 1.2, -2.1);
  \coordinate (horn-bot-L)  at (-3.6, -4.0);
  \coordinate (horn-bot-R)  at ( 3.6, -4.0);
  \coordinate (horn-center) at ( 0.0, -3.1);
  \coordinate (horn-top)    at ( 0.0, -2.1);
  \coordinate (horn-bot)    at ( 0.0, -4.0);
  \coordinate (horn-east)   at ( 2.8, -3.1);

  % Fill
  \fill[black!5]
    (horn-top-L)
    .. controls ++(0,-0.8) and ++(0,0.8) ..
    (horn-bot-L)
    -- (horn-bot-R)
    .. controls ++(0,0.8) and ++(0,-0.8) ..
    (horn-top-R)
    -- cycle;
  % Outline
  \draw[black!40, thick]
    (horn-top-L)
    .. controls ++(0,-0.8) and ++(0,0.8) ..
    (horn-bot-L)
    -- (horn-bot-R)
    .. controls ++(0,0.8) and ++(0,-0.8) ..
    (horn-top-R)
    -- (horn-top-L);
  % Label
  \node[font=\small, anchor=center] at (horn-center)
    {\textbf{Layer 1} \;---\;
     公司 (Companies)};
  \node[font=\scriptsize, anchor=north]
    at ([yshift=-0.1cm]horn-center)
    {amplifiers \;/\; 喇叭};

  % ── Layer 2: Builders ──
  \node[wbox] (build) at (0, -5.6)
    {\textcolor{water}{\textbf{Layer 2}}
     \;---\; 建设者 (Builders): construction $C$};

  % ── Money arrows ──
  \draw[arr, water, line width=1.4pt]
    (gov.south) -- node[lbl, right=3pt]
    {$B_{\textup{disbursed}}$} (horn-top);
  \draw[arr, water, line width=1.4pt]
    (horn-bot) -- node[lbl, right=3pt]
    {wages, materials} (build.north);

  % ── Min-cut (between Layer 0 and horn) ──
  \draw[knife, very thick, densely dashed]
    (-5.0, -1.5) -- (5.0, -1.5)
    node[anchor=west, font=\scriptsize\bfseries,
         text=knife] {\;min-cut};

  % ── Hidden leak (right of horn) ──
  \node[leak, anchor=west] (corrupt) at (4.4, -3.1)
    {\textcolor{knife}{隐性流失}\\[-2pt]
     \textcolor{knife}{\scriptsize hidden leak}};
  \draw[darr, line width=1.0pt]
    (horn-east) -- node[lbl, above] {层层转包}
    (corrupt.west);

  % ════════════════════════════════════════════
  %   PART 2 — GovFi observability layer
  % ════════════════════════════════════════════

  % ── Wide GovFi ledger ──
  \node[gfbox, minimum width=13.0cm]
    (ledger) at (0, -7.6)
    {\textcolor{sword}{\textbf{GovFi ledger}}\;:\;
     $\Obs = \mathcal{F}$
     \qquad{\scriptsize on-chain, immutable,
     publicly queryable}};

  % ── On-chain arrows (left-side routing from ledger to layers) ──
  % Arrow → Layer 0 (Gov) — outermost vertical run
  \draw[oarr, line width=1.2pt]
    ([yshift=0.15cm]ledger.west) -- (-7.6, -7.45)
    -- (-7.6, 0) -- (gov.west);
  % Arrow → Layer 1 (Companies) — middle vertical run
  \draw[oarr]
    (ledger.west) -- (-7.3, -7.6)
    -- (-7.3, -2.1) -- (horn-top-L);
  % Arrow → Layer 2 (Builders) — innermost vertical run
  \draw[oarr]
    ([yshift=-0.15cm]ledger.west) -- (-7.0, -7.75)
    -- (-7.0, -5.6) -- (build.west);

  % ── Label ──
  \node[font=\scriptsize, sword, anchor=east]
    at (-7.7, -4.8) {on-chain};

  % ── Corruption exposed ──
  \draw[oarr]
    ([xshift=5.0cm]ledger.north) -- ++(0,0.5)
    -- ++(0,2.0) -| (corrupt.south);
  \node[font=\scriptsize, sword,
    anchor=north west]
    at ([yshift=-0.1cm]corrupt.south east)
    {exposed};
  \node[font=\scriptsize, sword,
    anchor=north west]
    at ([yshift=-0.45cm]corrupt.south east)
    {$\Rightarrow$ eliminated};

  % ── Loss field ──
  \node[gfbox, minimum width=9.0cm]
    (loss) at (0, -9.2)
    {$\mathcal{L}(t) \;=\;
     B_{\textup{disbursed}}(t)
     \;-\; C_{\textup{verified}}(t)$};
  \draw[arr, sword, line width=1.0pt]
    (ledger.south) -- node[lbl, right=2pt]
    {computes} (loss.north);

  % ── Breakpoints ──
  \node[brk, minimum width=7.0cm]
    (bp) at (0, -10.6)
    {\textcolor{caution}{$\theta_1 = 5\%$\qquad
     $\theta_2 = 10\%$\qquad $\theta_3 = 15\%$}};
  \draw[arr, caution, line width=1.0pt]
    (loss.south) -- node[lbl, right=2pt]
    {\scriptsize auto-audit triggers} (bp.north);
\end{tikzpicture}
\end{center}

\noindent
Money flows $0 \to 1 \to 2$.
The min-cut of this three-layer graph is at
Layer~1: companies control both the information
(what the government sees) and the money (what the
builders receive).  This is the dual-ring structure
(\cref{def:dual-ring}): the inner ring is the contract
(合同); the outer ring is the actual payment chain
(实际资金链).  Corruption is flow diversion at the
min-cut---a hidden edge from Layer~1 to an
off-graph recipient.

\paragraph{公示清单 (Disclosure ledger).}
The GovFi ledger requires every item below to be
on-chain, immutable, and publicly queryable.

\medskip
\noindent
\textbf{Layer 0 --- 发债 (Bond issuance).}
\begin{enumerate}[label=(\roman*)]
  \item 债券总额 $B$, 利率 $r$, 期限 $T$
    (bond total, interest rate, maturity);
  \item 还款来源: 电费收入、财政转移支付
    (repayment source: electricity revenue,
    fiscal transfer);
  \item 可行性研究报告、环境影响评价
    (feasibility study, environmental impact
    assessment)---the physical basis for
    $V_{\textup{rec}}$;
  \item 预算分项: 土建、设备、设计、监理、应急
    (budget breakdown: civil works, equipment,
    engineering, supervision, contingency)---defines
    $C$ before construction starts.
\end{enumerate}

\noindent
\textbf{Layer 1 --- 采购链 (Procurement chain).}
This is the min-cut where corruption concentrates.
\begin{enumerate}[label=(\roman*)]
  \item 所有投标书: 投标人、报价、技术评分
    (all bids received: bidder, price, technical
    score)---if only the winner is published, rigged
    bidding is undetectable;
  \item 中标合同: 承包商、价格、范围、工期
    (winning contract: contractor, price, scope,
    timeline)---the edge in the execution graph;
  \item 完整分包树: 每一级分包商、价格、范围
    (full subcontractor tree: every tier, price,
    scope)---层层转包 (layer-by-layer subcontracting)
    is the primary hidden channel; each layer skims
    $10$--$15\%$, and by the fourth layer half the
    money is absorbed before reaching the builder;
  \item 材料单价: 品名、数量、单价、市场基准价
    (material unit prices vs.\ market benchmark)---inflated
    material costs are the second channel;
  \item 设备费用: 类型、来源、租购价、市场基准价
    (equipment costs vs.\ market benchmark);
  \item 变更签证: 范围变更、价格变更、理由
    (change orders: scope change, price change,
    justification)---post-contract inflation is the
    third channel: ``发现岩层'' (we discovered rock)
    $\Rightarrow$ hundreds of millions in extras.
\end{enumerate}

\noindent
\textbf{Layer 2 --- 实际施工 (Execution).}
\begin{enumerate}[label=(\roman*)]
  \item 每笔付款: 日期、金额、收款人
    (every payment to every worker and subcontractor:
    date, amount, recipient)---the sink of the
    flow graph;
  \item 工程进度: 里程碑、完成率、独立验收报告
    (physical progress: milestone, completion~\%,
    independent inspection)---defines
    $C_{\textup{verified}}(t)$;
  \item 材料进场: 品名、数量、来源、验收记录
    (material delivery: item, quantity, source,
    acceptance record)---cross-check against
    Layer~1 procurement prices;
  \item 质量检验: 每阶段工程师签字、缺陷清单
    (quality inspection: engineer sign-off, defect
    list)---prevents 豆腐渣工程
    (tofu-dreg construction).
\end{enumerate}

\paragraph{实时损耗场 (Real-time loss field).}
The loss field
\[
  \mathcal{L}(t) \;=\;
  B_{\textup{disbursed}}(t)
  \;-\; C_{\textup{verified}}(t)
\]
is published daily and decomposes into three components:

\begin{center}
\renewcommand{\arraystretch}{1.25}
\begin{tabular}{@{}lll@{}}
\toprule
\textbf{Component} & \textbf{Chinese}
  & \textbf{Acceptable?} \\
\midrule
Legitimate overhead
  & 合理管理费用 (管理、保险、许可)
  & Yes, bounded \\
Profit margins
  & 合理利润 (市场利润率)
  & Yes, market rate \\
Unexplained gap
  & 不明差额 $= \mathcal{L} -
    \text{overhead} - \text{margins}$
  & \textbf{No: corruption signal} \\
\bottomrule
\end{tabular}
\end{center}

\noindent
When the unexplained gap exceeds the breakpoint
threshold~$\theta_k$
(\cref{prop:govfi-breakpoints}), an automatic audit
triggers.  The graduated schedule:
$\theta_1 = 5\%$ (internal review),
$\theta_2 = 10\%$ (independent audit),
$\theta_3 = 15\%$ (public investigation, contract
suspension).

\paragraph{传统公示 vs.\ GovFi (Traditional disclosure
vs.\ GovFi).}
Traditional 公示 is a PDF on a government website:
it publishes the contract summary (Layer~1 top-level)
and the completion report (Layer~2 final).  It misses:

\begin{center}
\renewcommand{\arraystretch}{1.25}
\begin{tabular}{@{}lcc@{}}
\toprule
\textbf{Item} & \textbf{传统公示}
  & \textbf{GovFi} \\
\midrule
分包树 (subcontractor tree) & Hidden & On-chain \\
材料单价 (material unit prices) & Hidden & On-chain \\
变更签证 (change orders) & Delayed/partial & Real-time \\
工人工资 (worker payments) & Hidden & On-chain \\
$\mathcal{L}(t)$ 实时 (real-time loss) & Not computed
  & Published daily \\
自动审计触发 (auto audit trigger)
  & None & Breakpoints \\
\bottomrule
\end{tabular}
\end{center}

\noindent
The difference is not what information exists but
what \emph{architecture} delivers it.  A PDF can be
delayed, redacted, or revised.  A ledger is immutable,
machine-readable, and queryable by any citizen.
The dual ring (\cref{def:dual-ring}) collapses because
the outer ring (actual payments) is identical to the
inner ring (on-chain records): there is no gap in which
corruption can hide.

\paragraph{Numerical instantiation
(\texttt{govfi/simulate.py}).}
The simulation concretises the symbolic framework with
the following parameters:

\begin{center}
\renewcommand{\arraystretch}{1.25}
\begin{tabular}{@{}lrl@{}}
\toprule
\textbf{Parameter} & \textbf{Value} & \textbf{Rationale} \\
\midrule
Budget $B$ & $100$\,亿元 & provincial 城投 bond scale \\
Bond rate $r$ & $5\%$ & typical LGFV coupon \\
Construction schedule & 5\,years & medium hydro project \\
Political delay $\delta_{\textup{lag}}$ & 0.5\,years
  & legislative cycle \\
Revenue volatility $\sigma$ & 2\,亿/yr
  & electricity demand variation \\
Breakpoints $(\theta_1, \theta_2, \theta_3)$
  & $(5\%, 10\%, 15\%)$
  & \cref{prop:govfi-breakpoints} \\
\bottomrule
\end{tabular}
\end{center}

\noindent
Three scenarios test convergence of the loss field under
the DIDE dynamics (\cref{prop:absorption-time}) with
constant absorption rate~$\bar{a}$ (the step-function
approximation
$a(t) = \bar{a}\,\mathbf{1}_{t > \delta_{\textup{lag}}}$):

\begin{center}
\renewcommand{\arraystretch}{1.25}
\begin{tabular}{@{}lccccc@{}}
\toprule
\textbf{Scenario} & $\mathcal{L}_0$
  & $\bar{a}$ & \textbf{Conv.?}
  & $T^*$ \textbf{bound}
  & $\mathbb{E}[T^*]$ \\
\midrule
Clean ($\bar{a} = 5.0$)
  & 0.5 & 5.0 & Yes
  & 0.6\,yr & 0.1\,yr \\
Moderate corruption + GovFi
  & 10.0 & 3.0 & Yes
  & 4.6\,yr & 4.0\,yr \\
No control ($\bar{a} = 0$)
  & 10.0 & 0 & No
  & $+\infty$ & $+\infty$ \\
\bottomrule
\end{tabular}
\end{center}

\paragraph{Observations.}
\begin{enumerate}[label=(\roman*)]
\item \textbf{Clean baseline.}
  With full verification flow ($\bar{a} = 5.0$), the
  residual gap $\mathcal{L}_0 = 0.5$\,亿 (normal
  disbursement lead) absorbs in under one year.
  $\operatorname{Var}(T^*) \approx 0.02$\,yr$^2$:
  volatility is irrelevant when absorption dominates.

\item \textbf{Moderate corruption with GovFi.}
  A $10\%$ diversion at Layer~1 creates
  $\mathcal{L}_0 = 10$\,亿.  Breakpoints $\theta_1$
  and $\theta_2$ fire immediately ($t < 0.1$).
  With activated absorption $\bar{a} = 3.0$ the
  convergence condition
  $\bar{a} > r\,\mathcal{L}_0\,e^{r\delta_{\textup{lag}}}
  = 0.51$ holds, and the loss field reaches zero
  within $T^* \le 4.6$\,years
  ($\mathbb{E}[T^*] = 4.0$,
  $\operatorname{Var}(T^*) = 2.56$\,yr$^2$).
  The variance is non-trivial: stochastic revenue
  ($\sigma = 2$) amplified through the cubic factor
  $(a - r\mathcal{L}_0)^{-3}$
  (\cref{prop:stochastic-loss}).

\item \textbf{Unchecked corruption ($\bar{a} = 0$).}
  Without GovFi, no absorption is activated, and the
  loss field grows exponentially:
  $\mathcal{L}(5) = 10\,e^{0.25} = 12.8$\,亿,\;
  $\mathcal{L}(10) = 10\,e^{0.50} = 16.5$\,亿.
  The breakpoint $\theta_3 = 15\%$ fires at
  $t \approx 8.1$\,years, but with no control mechanism
  the signal goes unanswered.
  Project fails; money gone.

\item \textbf{Convergence gap.}
  The ratio
  $\bar{a}\, / \,(r\,\mathcal{L}_0\,
  e^{r\delta_{\textup{lag}}})$
  separates success from failure.
  In Scenario~2 this ratio is $3.0/0.51 \approx 5.9$:
  absorption outpaces compounding by nearly~$6\times$.
  In Scenario~3 the ratio is zero.
  GovFi's contribution is activating $\bar{a} > 0$;
  the magnitude of~$\bar{a}$ depends on political will,
  but the architecture ensures the signal reaches the
  sovereign before~$\theta_3$.
\end{enumerate}

\paragraph{Validation chain
(\texttt{govfi/simulate.py::run\_validation\_chain}).}
The simulation traces the full audit path for Scenario~2
(moderate corruption with GovFi), period by period.
Each link in the chain is recorded on the ledger;
every claim is machine-verifiable.

\smallskip\noindent
\emph{Step~0: Bond issuance.}
省政府 issues bonds: $B = 100$\,亿元, $r = 5\%$,
maturity $T = 5$\,yr.
GovFi ledger initialised; $\Obs = \mathcal{F}$.

\smallskip\noindent
\emph{Periods 1--4 ($t = 0.5$--$2.0$\,yr).}
Each period: 10\,亿 disbursed (Gov $\to$ Companies),
9\,亿 verified (10\% diverted at Layer~1).
Loss field grows linearly:
$\mathcal{L}(0.5) = 1.0$,\;
$\mathcal{L}(1.0) = 2.0$,\;
$\mathcal{L}(1.5) = 3.0$,\;
$\mathcal{L}(2.0) = 4.0$\,亿.
All below $\theta_1 = 5\%$.

\smallskip\noindent
\emph{Period~5 ($t = 2.5$\,yr).}
$\mathcal{L}(2.5) = 5.0$\,亿 $= 5.0\%$ of budget.
\textbf{Breakpoint $\theta_1$ fires.}
GovFi activates restructuring protocol:
absorption rate $\bar{a} = 3.0$ begins after
$\delta_{\textup{lag}} = 0.5$\,yr (legislative cycle).

\smallskip\noindent
\emph{Periods 6--10 ($t = 3.0$--$5.0$\,yr).}
Loss field continues to $\mathcal{L}(5.0) = 10.0$\,亿
($10\%$ of budget; $\theta_2$ fires at $t = 5.0$).
The convergence bound below is forward-looking from
this activation point.

\smallskip\noindent
\emph{Convergence proof.}
At the activation point
$\mathcal{L}_0 = 10.0$\,亿, the convergence condition
(\cref{prop:absorption-time}a) is verified:
\[
  \bar{a} = 3.0 \;>\;
  r\,\mathcal{L}_0\,e^{r\delta_{\textup{lag}}}
  = 0.05 \times 10.0 \times e^{0.025}
  = 0.5127.  \quad\checkmark
\]
The absorption bound gives
$T^* \le 0.5 + 10.25 / (3.0 - 0.51)
= 4.6$\,yr, with stochastic moments
$\mathbb{E}[T^*] = 4.0$\,yr,\;
$\operatorname{Var}(T^*) = 2.56$\,yr$^2$.
The ledger records 20~transactions (10~disbursements
+ 10~verifications), each publicly queryable.

\smallskip\noindent
\textbf{Verdict.}
The dam completes.  GovFi works.  Every link in the
chain---bond issuance, disbursement, verification,
loss computation, breakpoint trigger, absorption
activation, convergence proof---is auditable on-chain.
The 10\% diversion at Layer~1 is \emph{detected}
at $t = 2.5$\,yr (not at project completion)
and \emph{absorbed} within the construction horizon.
Without GovFi, the same diversion produces
$\mathcal{L}(10) = 16.5$\,亿 and growing.
\end{example}

%
\begin{example}[地狱公司结构 --- GovFi under monopoly]
\label{ex:monopoly}
Suppose Layer~1 is a \emph{monopoly} (垄断): a single
entity controls the min-cut, routing both information
and money.  The monopolist diverts $30\%$ of every
disbursement.  Does GovFi still work?

\paragraph{Without GovFi.}
The monopolist self-reports verification, inflating
$C_{\textup{verified}}$ to match $B_{\textup{disbursed}}$.
The ledger shows
$\mathcal{L}(t) = 0$ throughout construction.
All three breakpoints sleep.
In reality, actual construction covers only $70\%$ of
the budget; $30$\,亿 is gone.
The loss field is \emph{gamed}: the observability
gap $\|\Obs\| > 0$ hides the knife
(\cref{def:knife}, condition~(2)).

\paragraph{With GovFi ($\Obs = \mathcal{F}$).}
Independent on-chain verification (IoT sensors,
satellite imagery, certified inspectors) creates
$C_{\textup{verified}}$ that the monopolist cannot
inflate.  The loss field reveals the $30\%$ gap
in real time:

\begin{center}
\renewcommand{\arraystretch}{1.25}
\begin{tabular}{@{}cccc@{}}
\toprule
$t$ (yr) & $B_{\textup{disb}}$ & $C_{\textup{ver}}$
  & $\mathcal{L}/B$ \\
\midrule
0.5 & 10 & 7 & 3.0\% \\
1.0 & 20 & 14 & 6.0\%
  \;$\Leftarrow \theta_1$ \textbf{fires} \\
2.0 & 40 & 28 & 12.0\%
  \;$\Leftarrow \theta_2$ \textbf{fires} \\
2.5 & 50 & 35 & 15.0\%
  \;$\Leftarrow \theta_3$ \textbf{fires} \\
5.0 & 100 & 70 & 30.0\% \\
\bottomrule
\end{tabular}
\end{center}

\noindent
Detection occurs at $t = 1.0$\,yr (versus \emph{never}
without GovFi).  But detection alone is insufficient:
the \emph{absorption rate}~$\bar{a}$ determines whether
the loss field converges.

\paragraph{Absorption under monopoly.}
Two cases:
\begin{enumerate}[label=(\alph*)]
\item \textbf{Monopolist resists} ($\bar{a} = 2.0$).
  The convergence condition
  $\bar{a} > r\,\mathcal{L}_0\,e^{r\delta_{\textup{lag}}}
  = 1.54$ holds, but barely.
  $T^* \le 67.1$\,yr,\;
  $\mathbb{E}[T^*] = 60$\,yr.
  Detection without political action is
  \emph{cosmetic}: the signal exists, but
  nobody acts on it.

\item \textbf{$\theta_3$ breaks the monopoly}
  ($\bar{a} = 5.0$).
  The breakpoint at $\theta_3 = 15\%$
  ($t = 2.5$\,yr) triggers contract suspension
  and open re-bidding.
  The monopoly is dissolved; new contractors enter.
  Absorption strengthens:
  $T^* \le 9.4$\,yr,\;
  $\mathbb{E}[T^*] = 8.6$\,yr,\;
  $\operatorname{Var}(T^*) = 2.80$\,yr$^2$.
  The dam still completes---late, but it completes.
\end{enumerate}

\noindent
The ratio $\bar{a}\,/\,(r\,\mathcal{L}_0\,
e^{r\delta_{\textup{lag}}})$ is $2.0/1.54 = 1.3$ in
case~(a) and $5.0/1.54 = 3.2$ in case~(b):
breaking the monopoly more than doubles the convergence margin.

\paragraph{The Qin structure.}
The monopoly at Layer~1 is structurally identical to
the 秦-system (\cref{sec:qin}): a single node
concentrates all actuation ($U$), all information flow
($\Obs$), and resists external restructuring.
GovFi's contribution is collapsing the dual ring
(\cref{def:dual-ring}) so that $\Obs = \mathcal{F}$
regardless of Layer~1's market structure.
The loss field $\mathcal{L}(t)$ becomes a \emph{public}
signal; the breakpoints $\theta_k$ convert that signal
into automatic restructuring triggers.
Observability is necessary; political will
(activating $\bar{a}$) is sufficient.
\end{example}

% ================================================================
\section{Historical instantiation}\label{sec:gf-history}
% ================================================================

\begin{example}[Iceland 2008]\label{ex:iceland}
Iceland's banking system collapsed in October 2008 with
total banking assets at ${\sim}10\times$ GDP.

\begin{center}
\renewcommand{\arraystretch}{1.25}
\begin{tabular}{@{}llp{5.5cm}@{}}
\toprule
\textbf{Framework} & \textbf{Value} & \textbf{Outcome} \\
\midrule
Exit face chosen
  & $\partial_{\textup{haircut}}$
  & Foreign creditors absorbed ${\sim}60\%$ of losses \\
$C(x)$ activated
  & Yes
  & Parliament refused to honour Icesave guarantees \\
Time to recovery
  & 3--4 years
  & GDP regained pre-crisis level by 2012 \\
Path
  & (a) of \cref{thm:lifecycle}
  & Controlled restructuring \\
\bottomrule
\end{tabular}
\end{center}

\noindent
Iceland chose the exit face.  The choice was politically
costly (international condemnation, diplomatic pressure)
but mathematically optimal: by
\cref{thm:exit-choice}(a), distributing losses to the
agent class with the largest loss-absorption capacity
(foreign institutional creditors) minimises the damage
functional $\mathcal{D}$.
\end{example}

\begin{example}[Japan 1990s]\label{ex:japan}
Japan's asset bubble collapsed in 1991.  Non-performing
loans in the banking system reached ${\sim}35\%$ of GDP.

\begin{center}
\renewcommand{\arraystretch}{1.25}
\begin{tabular}{@{}llp{5.5cm}@{}}
\toprule
\textbf{Framework} & \textbf{Value} & \textbf{Outcome} \\
\midrule
Exit face chosen
  & None (deferred)
  & Extend-and-pretend for ${\sim}12$ years \\
$C(x)$ activated
  & No
  & MOF forbearance policy 1992--2003 \\
Time to resolution
  & $>$ 15 years
  & ``Lost decades'' (1991--2010+) \\
Path attempted
  & (c)---non-existent
  & Relabelled NPLs, rolled over zombie loans \\
\bottomrule
\end{tabular}
\end{center}

\noindent
Japan attempted path~(c): neither restructure nor
collapse, but indefinite deferral.  By
\cref{thm:fixedpoint}, this path does not exist.
The loss field $\mathcal{L}(t_0)$ was conserved
(\cref{thm:absorption-conservation}); deferral only
compounded it (\cref{rem:deferral}).  The losses were
eventually absorbed by taxpayers (bank recapitalisation),
depositors (near-zero interest rates for two decades),
and the entire economy (forgone growth).
\end{example}

\begin{proposition}[Extend-and-pretend violates the
fixed-point theorem]\label{prop:zombie}
The extend-and-pretend strategy (forbearance,
reclassification of NPLs, rollover of zombie loans)
is an attempt to modify the \emph{detection output}
$\Obs(D)$ without modifying the physical state~$D$.
In the framework:
\begin{enumerate}[label=(\roman*)]
  \item Relabelling an NPL changes the classification
    signal but not the loss field: $\mathcal{L}(t)$ is
    unchanged because $D(t)$ and $V_{\textup{rec}}(t)$
    are physical quantities independent of accounting
    labels.
  \item Rolling over a zombie loan sets $\delta(t) > 0$
    (new lending to service old debt), which by
    \eqref{eq:debt-dynamics} increases $D'(t)$: the
    loss field grows.
  \item Forbearance reduces $\Obs(D)$ by suppressing
    the observation signal, creating a gap
    $\|U\| - \|U \cap \Obs\| > 0$---exactly the
    condition of \cref{prop:imperfect} under which
    knives become hidden.
\end{enumerate}
By \cref{thm:fixedpoint}, there is no fixed point of
the system in which $\mathcal{L} > 0$ and no agent
absorbs loss: the absorption conservation law
\eqref{eq:absorption-conservation} has no solution
with $\ell_i = 0$ for all~$i$ when
$\mathcal{L}(t_0) > 0$.  Extend-and-pretend is not
a strategy; it is a violation of conservation.
\end{proposition}

\begin{remark}[The 城投 knife]\label{rem:chengtou}
China's local government financing vehicles
(地方政府融资平台, commonly called 城投) are the
current instantiation of the fiscal knife.  The
structure maps directly:
\begin{enumerate}[label=(\roman*)]
  \item \textbf{Autonomous actuation}: 城投 debt
    compounds at coupon rates of $5$--$8\%$ while
    underlying asset returns (land sales, toll roads,
    industrial parks) have declined below the cost of
    capital.  The self-financing loop
    (\cref{rem:debt-loop}) is active.
  \item \textbf{Hidden observability}: much 城投
    debt is off-balance-sheet, held through trust
    products, wealth management products, and
    interbank channels.  The dual-ring structure
    (\cref{def:dual-ring}) applies: the formal fiscal
    rules (inner ring) prohibit local government
    borrowing, while the actual practice (outer ring)
    operates through 城投 vehicles.
  \item \textbf{Conservation applies}: by
    \cref{thm:absorption-conservation}, the losses
    embedded in non-performing 城投 debt are conserved.
    The current policy of rolling over (化债) is
    channel~(iii) of the proof: deferral, not
    absorption.
\end{enumerate}
GovFi addresses the 城投 problem through all three
mechanisms: on-ledger recording eliminates the
hidden outer ring (\cref{rem:govfi-dualring}),
programmatic breakpoints prevent accumulation
to crisis levels (\cref{prop:govfi-breakpoints}),
and published $\mathcal{L}(t)$ ensures that the
conservation law is visible to all agents
(\cref{prop:govfi-observability}).
The knife is real.  The losses are real.  The only
question is the exit face.
\end{remark}

% ================================================================
\section*{Closing remark: the fiscal flow}
% ================================================================

\begin{remark}[Flow-theoretic restatement]\label{rem:govfi-flow}
In the language of the agentic calculus (\cref{sec:flow}),
the entire appendix restates as follows.
Debt restructuring is \emph{flow rerouting}: redirecting
fiscal flows from the default channel to a chosen
absorption channel.
The absorption conservation law
(\cref{thm:absorption-conservation}) is \emph{flow
conservation}: the total flow of losses through the
fiscal network is invariant; only the routing changes.
The GovFi ledger (\cref{def:govfi-ledger}) makes the
fiscal execution graph $G_F$ fully observable, so that
the sovereign can compute the \emph{min-cut}
(\cref{thm:flowcut}): the minimum-cost set of edges
whose removal (restructuring) restores the system to
$K_F$.
The knife-is-the-mean theorem (\cref{thm:meanfield})
applied to the fiscal domain states: the same debt
level $D$ is sustainable or unsustainable depending
on the system's mean fiscal capacity $\bar{w}_F$.
Growth raises $\bar{w}_F$; the same nominal debt becomes
a tool (financing productive investment) or a knife
(compounding toward default) depending on which side
of the mean the system sits.

The losses are conserved.  The only freedom is the
choice of exit face.
\end{remark}

% ================================================================
\section*{Epilogue}
% ================================================================

\begin{quote}
\itshape
然而我正对一本历史书\\
西望夕阳里的咸阳古道\\
我等到了一匹快马的蹄声

\medskip
\upshape
---卞之琳,《音尘》
\end{quote}

\bigskip

\begin{center}
\itshape
献给我最爱的女人:陈晓楚
\end{center}

\chapter{Gaokao Reform --- Equal Probability Admission}\label{app:gaokao}

The preceding appendix applied the framework to the fiscal substrate
of sovereign debt (\cref{app:govfi}).  This appendix applies it to
the \emph{human capital substrate}---the university admission system
that determines which agents enter the viability kernel of higher
education.  The central result is a \emph{disparity conservation law}:
total university capacity is conserved; only its distribution across
provinces is free.  The policy implication is that the current allocation
system is a \emph{knife}---a structural separator that assigns students
different viability probabilities based solely on their province of
registration---and that a population-proportional allocation
removes this knife entirely.

% ================================================================
\paragraph{Notation / 符号表.}
% ================================================================
The following symbols are used throughout this appendix.
\textcolor{water}{Blue} = structure/flow,
\textcolor{knife}{red} = disparity/inequality,
\textcolor{sword}{cyan} = reform/detection,
\textcolor{caution}{orange} = threshold warnings.

\begin{center}
\renewcommand{\arraystretch}{1.2}
\begin{tabular}{@{}lp{5.5cm}p{5.5cm}@{}}
\toprule
\textbf{Symbol} & \textbf{English} & \textbf{中文} \\
\midrule
\multicolumn{3}{@{}l}{\emph{Provincial state}} \\
$N_i$             & Exam takers in province $i$
                  & 第 $i$ 省报名人数 \\
$n$               & Number of provinces ($= 31$)
                  & 省份数 \\
$C$               & Total national 985 capacity
                  & 985 总招生容量 \\
$s_i$             & Slots allocated to province $i$
                  & 第 $i$ 省招生名额 \\
\midrule
\multicolumn{3}{@{}l}{\emph{Rates / 录取率}} \\
$R_i$             & Admission rate $= s_i / N_i$
                  & 录取率 \\
$\bar{R}$         & National avg $= C / \sum N_i$
                  & 全国平均录取率 \\
\midrule
\multicolumn{3}{@{}l}{\emph{Disparity / 不平等}} \\
$D$               & Disparity ratio $= \max R_i / \min R_i$
                  & 最大最小比 \\
$G$               & Gini coefficient of $\{R_i\}$
                  & 基尼系数 \\
$\sigma_R$        & Std.\ dev.\ of rates across provinces
                  & 录取率标准差 \\
\midrule
\multicolumn{3}{@{}l}{\emph{Reform models}} \\
$w_i$             & Province weight (equal: $w_i = 1$)
                  & 省份权重 \\
$\Delta$          & Score band width (points)
                  & 分数段宽度 \\
\bottomrule
\end{tabular}
\end{center}

% ================================================================
\section{The Current System / 现行制度}\label{sec:gaokao-current}
% ================================================================

China's national college entrance examination (高考) is taken by
approximately $N = \sum_{i=1}^{31} N_i \approx 12{,}350{,}000$
students annually (2024 data).  Each university allocates a
\emph{fixed quota} $s_i$ to each province, set by historical
precedent and political negotiation rather than by population
proportion.  The admission rate for province~$i$ is
\begin{equation}\label{eq:gaokao-rate}
  R_i = \frac{s_i}{N_i}.
\end{equation}

\begin{definition}[Disparity index / 不平等指数]\label{def:disparity}
The \emph{disparity ratio} is $D = \max_i R_i \;/\; \min_i R_i$.
The \emph{Gini coefficient} of the rate vector $\mathbf{R}
= (R_1, \ldots, R_n)$ is
\[
  G = \frac{1}{2n^2 \tilde{R}} \sum_{i,j} |R_i - R_j|,
  \qquad \tilde{R} = \frac{1}{n}\sum_i R_i.
\]
Perfect equality: $G = 0$, $D = 1$.
Maximum inequality: $G \to 1$, $D \to \infty$.
\end{definition}

\paragraph{2024 data / 2024年数据.}
Using publicly available data from Chinese education statistics
sources\footnote{%
  Sources: 中国教育在线 (eol.cn), 聚汇数据 (gotohui.com),
  知乎 (zhihu.com), 高考直通车 (gaokaozhitongche.com),
  网易 (163.com), 搜狐 (sohu.com).
  All figures are approximate; see \texttt{gaokao/data.py} for
  per-source annotations and confidence levels.},
the 2024 985-university admission rates are:

\begin{center}
\renewcommand{\arraystretch}{1.1}
\begin{tabular}{@{}llrr@{}}
\toprule
\textbf{Province / 省份} & & \textbf{Exam takers / 考生}
  & \textbf{985 rate / 录取率} \\
\midrule
\multicolumn{4}{@{}l}{\emph{Top 5 — highest rate / 录取率最高}} \\
天津 & Tianjin   &   70{,}800 & 5.81\% \\
北京 & Beijing   &   67{,}000 & 5.30\% \\
上海 & Shanghai  &   54{,}000 & 4.40\% \\
吉林 & Jilin     &  130{,}100 & 3.56\% \\
青海 & Qinghai   &   65{,}600 & 3.36\% \\
\midrule
\multicolumn{4}{@{}l}{\emph{Bottom 5 — lowest rate / 录取率最低}} \\
云南 & Yunnan    &  395{,}000 & 1.00\% \\
广东 & Guangdong &  768{,}000 & 0.98\% \\
贵州 & Guizhou   &  472{,}500 & 0.98\% \\
河北 & Hebei     &  928{,}000 & 0.96\% \\
河南 & Henan     & 1{,}360{,}000 & 0.84\% \\
\bottomrule
\end{tabular}
\end{center}

\begin{proposition}[2024 disparity / 2024年不平等]\label{prop:gaokao-disparity}
For 985-university admission rates across 31 provinces (2024):
\[
  D = 6.9\times, \qquad G = 0.310, \qquad
  \bar{R}_{\textup{weighted}} = 1.33\%.
\]
天津 (Tianjin, $N = 70{,}800$) has a 985 rate of $5.81\%$;
河南 (Henan, $N = 1{,}360{,}000$) has $0.84\%$.
河南 has $19\times$ more exam takers but $6.9\times$ worse
admission probability.
\end{proposition}

\begin{remark}[The knife is the quota / 刀就是名额]
\label{rem:gaokao-knife}
The admission quota $s_i$ is the knife.  It satisfies both conditions
from \cref{def:knife}: (1)~autonomous actuation---each university
sets its own quotas without external constraint; (2)~observability
failure---students in province $i$ cannot observe or influence the
quota-setting process.  The disparity $D = 6.9\times$ is not a
natural phenomenon; it is an artefact of the quota allocation,
which is neither population-proportional nor merit-based.
\end{remark}

% ================================================================
\section{Reform Models / 改革模型}\label{sec:gaokao-reform}
% ================================================================

We consider two alternative allocation rules, holding total
capacity $C = \sum_i s_i \approx 164{,}203$ fixed.

\begin{definition}[Equal-weight allocation / 等权分配]
\label{def:equal-weight}
Set $w_i = 1$ for all $i$.  Then
\[
  s_i^{\textup{eq}} = \frac{C}{n}
  \approx 5{,}297 \;\text{per province}.
\]
\end{definition}

\begin{definition}[Population-proportional allocation / 人口比例分配]
\label{def:pop-prop}
Set $w_i = N_i$.  Then
\[
  s_i^{\textup{pp}} = C \cdot \frac{N_i}{\sum_j N_j},
  \qquad
  R_i^{\textup{pp}} = \frac{s_i^{\textup{pp}}}{N_i}
  = \frac{C}{\sum_j N_j} = \bar{R}
  \;\;\text{(constant for all $i$)}.
\]
\end{definition}

\begin{proposition}[Reform comparison / 改革对比]
\label{prop:gaokao-reform}
Under the three allocation models:
\begin{center}
\renewcommand{\arraystretch}{1.1}
\begin{tabular}{@{}lrrr@{}}
\toprule
\textbf{Model / 模型} & \textbf{Gini ($G$)} & \textbf{$D$ (max/min)}
  & \textbf{Direction} \\
\midrule
Current / 现行制度  & $0.310$ & $6.9\times$ & --- \\
Equal-weight / 等权 & $0.514$ & $37.8\times$
  & \textcolor{knife}{$\uparrow$ worse} \\
Pop-proportional / 人口比例 & $0.0001$ & $1.0\times$
  & \textcolor{sword}{$\downarrow$ optimal} \\
\bottomrule
\end{tabular}
\end{center}
\end{proposition}

\begin{remark}[等权 $\neq$ 等概率 / Equal weight $\neq$ equal probability]
\label{rem:eq-neq-eqprob}
The na\"ive intuition that ``equal provincial weight'' produces
equality is \textbf{wrong}.  Equal absolute slots ($C/n$) distributed
to provinces of vastly different population sizes produces
\emph{worse} per-capita disparity: 西藏 ($N = 36{,}000$) would receive
a $14.7\%$ rate while 河南 ($N = 1{,}360{,}000$) would receive $0.39\%$.
The disparity ratio rises from $6.9\times$ to $37.8\times$.

The correct reform is population-proportional allocation:
$s_i \propto N_i$.  This gives every student the same probability
$R_i = \bar{R} \approx 1.33\%$ regardless of province.
\end{remark}

\begin{theorem}[Proportional allocation eliminates the knife]
\label{thm:gaokao-knife-removal}
Under population-proportional allocation $s_i = C \cdot N_i / N$
where $N = \sum_j N_j$:
\begin{enumerate}[label=\textup{(\alph*)}]
\item $R_i = C/N = \bar{R}$ for all $i$ (uniform rate);
\item $G = 0$ and $D = 1$ (perfect equality);
\item The quota $s_i$ ceases to be a knife: it no longer separates
  viability classes by province.
\end{enumerate}
\end{theorem}

\begin{proof}
(a) is immediate from the definition.  (b) follows since all rates are
equal.  For~(c), the knife conditions (\cref{def:knife}) fail:
the allocation is now a deterministic function of the publicly
observable population $N_i$, so observability holds and no autonomous
actuation remains.
\end{proof}

% ================================================================
\section{Within-Band Randomization / 分数段内随机化}
\label{sec:gaokao-band}
% ================================================================

The current system ranks students by exact score within each province.
A difference of one point (满分750, so $\Delta s = 1/750 \approx 0.13\%$
of the total range) can determine admission or rejection.  This creates
a ``one-point-one-fate'' (一分定终身) phenomenon.

\begin{definition}[Band randomization / 分数段随机化]
\label{def:band-random}
Fix a band width $\Delta > 0$ (e.g., $\Delta = 5$ points).
Partition the score range $[0, 750]$ into bands
$B_k = [k\Delta, (k+1)\Delta)$.
Within each band $B_k$, students are ranked by
\emph{uniform random permutation} rather than exact score.
\end{definition}

The effect: students whose scores differ by less than $\Delta$ have
equal probability of admission.  This converts a deterministic
(and fragile) ranking into a probabilistic one, eliminating the
incentive for one-point gaming while preserving the overall
meritocratic ordering at the $\Delta$-scale.

\begin{remark}[Band width choice / 分数段宽度选择]
\label{rem:band-width}
The choice $\Delta = 5$ is illustrative.  The optimal $\Delta$
balances two competing effects:
\begin{itemize}
\item Too small ($\Delta = 1$): equivalent to exact ranking,
  no randomization benefit.
\item Too large ($\Delta = 50$): too much noise, weakens
  the meritocratic signal.
\end{itemize}
The right $\Delta$ is an empirical question requiring the
fine-grained score distribution (一分一段表) to calibrate.
Score distributions for 2024 are publicly available for 29 of 31
provinces from provincial education examination authorities.
\end{remark}

% ================================================================
\section{Grace Period / 恩典期}\label{sec:gaokao-grace}
% ================================================================

The within-band randomization (\cref{sec:gaokao-band}) assigns each
student a \emph{tentative} school placement.  This initial assignment
is random within the score band.  The grace period converts this
random assignment into a \emph{chosen} one.

\begin{definition}[Grace period / 恩典期]\label{def:grace-period}
After the initial random-within-band assignment, a grace window
$[t_0, t_0 + \tau]$ opens during which students may \emph{swap}
placements with any other student in the same score band and the
same region, subject to the constraint that both students remain
within the same admission tier (budget).
\end{definition}

The mechanism has three phases:
\begin{enumerate}[label=\textup{(\alph*)}]
\item \textbf{Random seed / 随机初始.}
  Within-band randomization produces an initial assignment.
  No student chose this placement; it is the system's first offer.
\item \textbf{Grace window / 恩典窗口.}
  Students examine their assignment and may swap with willing
  partners in the same band and region.  You keep your tier
  but choose your specific school.  This is the time for the
  考生 to think: where do I actually want to go?
\item \textbf{Lock / 锁定.}
  After $\tau$, assignments become final.  No further swaps.
\end{enumerate}

\begin{remark}[Grace as bounded dissipation / 恩典即有界耗散]
\label{rem:gaokao-grace}
This is the same structure as the GovFi grace margins
(\cref{app:govfi}): bounded freedom within thresholds.
In GovFi, each layer $i$ has a grace margin $\epsilon_i = k_i - r$
within which losses are tolerated as legitimate costs.
In Gaokao, each student has a grace margin: freedom to swap
within the same tier without affecting the system's aggregate
allocation.

The grace period restores agency without restoring the knife.
The initial randomization removes one-point-one-fate (一分定终身);
the grace period gives back the student's choice of \emph{where},
while the population-proportional allocation determines \emph{how many}.
The constraint is the tier, not the specific school.

What is consumed during the grace period---the dissipated---is
information: students learn about their options, negotiate swaps,
discover preferences.  This is bounded dissipation in information
space, exactly analogous to the fiscal grace in \cref{app:govfi}.
\end{remark}

% ================================================================
\section{The Mean-Field Connection / 平均场联系}
\label{sec:gaokao-meanfield}
% ================================================================

The Gaokao disparity is a mean-field phenomenon in the sense of
\cref{sec:meanfield}.

\begin{proposition}[Knife = quota, mean = population share]
\label{prop:gaokao-meanfield}
Let $\bar{R} = C/N$ be the population-weighted national average.
Then:
\begin{enumerate}[label=\textup{(\alph*)}]
\item $13$ provinces (containing $67\%$ of all exam takers)
  have $R_i < \bar{R}$.
\item $18$ provinces (containing $33\%$ of all exam takers)
  have $R_i \geq \bar{R}$.
\item The unweighted arithmetic mean
  $\tilde{R} = \frac{1}{n}\sum R_i = 1.94\%$ exceeds $\bar{R} = 1.33\%$,
  showing that \emph{the majority of provinces have above-average rates
  when measured per province, but the majority of students have
  below-average rates when measured per capita}.
\end{enumerate}
\end{proposition}

This is the knife-is-the-mean phenomenon: the average computed
per province (unweighted, $\tilde{R}$) differs from the average
computed per student (weighted, $\bar{R}$), and this discrepancy
is itself the knife.  Most provinces appear ``above average'' while
most students are below it.

\begin{remark}[Importance sampling / 重要性抽样]
\label{rem:importance-sampling}
Why do the two averages disagree?  Because the current system
assigns importance weights in two levels:
\begin{enumerate}[label=\textup{(\arabic*)}]
\item \textbf{Level 1: equally to each province.}
  Every province gets the same voice---Tibet (3.6 万考生)
  counts the same as Henan (136 万考生), a 38:1 population
  ratio compressed to 1:1.
\item \textbf{Level 2: equally to each student within that province.}
  Within each province, students share their province's voice
  equally.
\end{enumerate}
The result: a student in Tibet inherits $1/36{,}000$ of a
full provincial vote, while a student in Henan inherits
$1/1{,}360{,}000$ of the same-sized vote.  Tibet's student
is 38 times louder than Henan's.  The importance weights
are the seed you plant---and the current system plants
unequal seeds.

In statistics, the fix is called \textbf{importance sampling}
(重要性抽样): if your sampling distribution $q$ differs from
the true distribution $p$, you multiply each sample by a
\emph{correction weight} $w(x) = p(x)/q(x)$:
\[
  \mathbb{E}_p[f(x)]
  \;=\;
  \mathbb{E}_q\!\Bigl[f(x)\,\cdot\,\frac{p(x)}{q(x)}\Bigr].
\]
Here $p$ is the per-student distribution (province $i$ has weight
$N_i/N$), $q$ is the per-province distribution (each province has
weight $1/n$), and $f$ is the admission rate.  The importance
weight is:
\[
  w_i \;=\; \frac{p_i}{q_i}
  \;=\; \frac{N_i/N}{1/n}
  \;=\; \frac{n\,N_i}{N}.
\]
Large provinces get large weights (Henan: $w \approx 3.4$),
small provinces get small ones (Tibet: $w \approx 0.09$).
The weighted average $\bar{R} = \sum w_i R_i / \sum w_i = 1.33\%$
is the true per-student mean.  The unweighted average
$\tilde{R} = 1.94\%$ is the \emph{biased} estimate that
ignores population---it over-counts small privileged provinces
and under-counts large disadvantaged ones.

In plain language (大白话): the unweighted mean is biased because
it lets 3.6 万 people in Tibet speak as loudly as 136 万 in Henan.
Importance sampling says: if you insist on giving every province
one vote, at least \emph{weight} those votes by how many students
each province actually has.  Once you do, $\tilde{R}$ collapses
to $\bar{R}$, and the illusion that ``most provinces are above
average'' disappears.
\end{remark}

\begin{remark}[Renormalization / 重整化]
\label{rem:gaokao-rg}
The importance weight assignment is a
\textcolor{knife}{renormalization group (RG) flow} compressed
into a single exam paper.  The current system has two scales:
\begin{center}
\renewcommand{\arraystretch}{1.15}
\begin{tabular}{@{}lll@{}}
\toprule
Scale & \textcolor{knife}{Current (bare)} & \textcolor{sword}{Reform (fixed point)} \\
\midrule
\textbf{Province} (coarse) & weight $= 1/n$ each
  & weight $= N_i/N$ \\
\textbf{Student} (fine) & weight $= 1/(n N_i)$
  & weight $= 1/N$ \\
\bottomrule
\end{tabular}
\end{center}
Under the \textcolor{knife}{current system}, the per-student
weight \emph{runs} with the province size $N_i$: a Tibet student
carries weight $1/(31 \times 36{,}000)$ while a Henan student
carries $1/(31 \times 1{,}360{,}000)$---a factor of 38.
The ``coupling constant'' (per-student admission probability)
\textcolor{knife}{depends on the scale at which you observe it}.
This is the signature of a system away from its RG fixed point.

Under \textcolor{sword}{population-proportional allocation}, every
student carries weight $1/N$ regardless of province.
The coarse-grained answer (per province: $s_i/N_i = C/N$)
and the fine-grained answer (per student: $C/N$) \emph{agree}.
The coupling no longer runs.
\textcolor{sword}{This is the RG fixed point}:
the admission probability is \emph{scale-invariant}---it does
not matter whether you measure at the province level or the
student level.

The current system's \textcolor{knife}{knife} is precisely the
RG anomaly: the bare parameter (per-province allocation) and
the physical observable (per-student probability) disagree.
Population-proportional allocation eliminates the anomaly by
making the bare parameter equal to the physical observable
at every scale.
\end{remark}

\begin{remark}[Temporal linearity / 时间线性]
\label{rem:gaokao-temporal}
The Gaokao happens every June.  It is non-pausable (you cannot stop the
exam mid-administration) and non-reversible (scores, once published,
determine that year's admissions permanently).  Each cohort passes
through the system exactly once.  This is temporal linearity: the
knife operates in real, non-replayable time.

The viability kernel $K$ is the set of provinces where a student's
probability of admission is at least $\bar{R}$.  Under the current
system, 13 provinces (67\% of students) are outside $K$.  Under
population-proportional allocation, $K$ expands to all provinces.
\end{remark}

% ================================================================
\section{Numerical Results / 数值结果}\label{sec:gaokao-numerics}
% ================================================================

The analysis is implemented in \texttt{gaokao/} (Python, stdlib only)
and reproduces the numbers in this appendix exactly.

\paragraph{Usage.}
\begin{verbatim}
  python -m gaokao.simulate
\end{verbatim}

\paragraph{Key outputs.}
\begin{center}
\renewcommand{\arraystretch}{1.1}
\begin{tabular}{@{}lrrr@{}}
\toprule
& \textbf{Current} & \textbf{Equal-wt} & \textbf{Pop-prop} \\
\midrule
Gini coefficient        & 0.310  & 0.514  & 0.0001 \\
Disparity ratio ($D$)   & 6.9$\times$ & 37.8$\times$ & 1.0$\times$ \\
Max rate (province)     & 5.81\% (天津)  & 14.71\% (西藏)  & 1.33\% (all) \\
Min rate (province)     & 0.84\% (河南)  & 0.39\% (河南)   & 1.33\% (all) \\
\bottomrule
\end{tabular}
\end{center}

Under population-proportional allocation, the top five gainers
by rate increase are: 河南 ($+0.49\%$), 河北 ($+0.37\%$),
广东 ($+0.35\%$), 贵州 ($+0.35\%$), 云南 ($+0.33\%$)---all
large-population provinces currently disadvantaged.
The biggest losers are the small, currently privileged provinces:
天津 ($-4.48\%$), 北京 ($-3.97\%$), 上海 ($-3.07\%$).

\begin{remark}[Data limitations / 数据局限]
\label{rem:gaokao-data}
The 985 admission rates used here are compiled from secondary sources
(media reports, data aggregators) rather than primary government
statistical yearbooks.  Some figures---particularly for mid-tier
provinces---are estimates based on ranges reported across multiple
sources.  The qualitative conclusions (6--7$\times$ disparity,
population-proportional reform eliminates it) are robust to
reasonable variations in the data.
Fine-grained score distributions (一分一段表) are available for
29 provinces in image format from provincial education examination
authority websites.  Text-format extraction would enable
within-band randomization calibration.
\end{remark}

\begin{remark}[Conservation law / 守恒定律 --- Log gas leakage]
\label{rem:gaokao-conservation}
Let $s_i$ denote the number of 985~slots allocated to province~$i$
under any method, and $C$ the total national 985~capacity.
The \textbf{slot conservation law} is
\[
  \sum_{i=1}^{n} s_i \;=\; C.
\]
The code checks four properties at each run:
\begin{enumerate}
  \item \emph{Slot conservation}: $\sum s_i = C$ (no slots created or destroyed).
  \item \emph{Non-negativity}: $s_i \ge 0$ for all~$i$.
  \item \emph{Capacity bound}: $s_i \le N_i$ (no province gets more slots
        than exam takers).
  \item \emph{Rate bounds}: $0 \le s_i / N_i \le 1$ (admission rate
        in $[0,1]$).
\end{enumerate}
The ``leak'' is $\ell = \sum s_i - C$.  A non-zero leak would indicate
an allocation error: slots appearing from, or disappearing into,
nowhere.  Both the equal-weight and population-proportional methods
pass with $\ell = 0$ (verified by \texttt{python -m gaokao.simulate}).

\textcolor{sword}{%
This is the same structure as the fiscal flow conservation check
(\cref{rem:govfi-conservation}):
$\text{leak} = \sum\text{inflow} - \sum\text{outflow}$.
Students flow through provinces the way money flows through
fiscal layers---conservation holds in both.}
\end{remark}

\bigskip
\noindent
\fbox{\parbox{\dimexpr\textwidth-2\fboxsep-2\fboxrule}{%
\smallskip
\textit{Author's note / 作者自述.}

\smallskip
\noindent
I took the Gaokao in Beijing in 2019 and scored 636.
Before that, I was a ``体育特长生'' (athletic recruit)---B-tier,
men's football---preparing for Tsinghua.
\textcolor{knife}{I failed.}  Not just at Tsinghua: at every university I tried.
So I took the Gaokao.

我在2019年在北京市参加了高考,考了636分。
在那之前,我是一个准备凭借高水平运动队招生(B类:男子足球)
考取清华大学的体育特长生。\textcolor{knife}{但我失败了}——不止在清华,
在我尝试过的每一所大学都失败了。然后,我就开始准备高考了。

\smallskip
I enrolled at Beijing University of Posts and Telecommunications,
majoring in Internet of Things Engineering.
I always thought I didn't like that school.
But I cannot deny: \textcolor{water}{I learned more there than I ever expected.}

我在北京邮电大学学习了两年,专业是物联网工程。
虽然我一直觉得我不喜欢这个学校,
但不可否认的是,\textcolor{water}{我在这里学到了很多——比我想象的更多。}

\smallskip
Two years later I transferred to New York University,
where I enrolled in Mechanical Engineering.
I graduated with a B.S.\ in Mechanical Engineering
and minors in Robotics and Mathematics.
\textcolor{water}{I liked learning math at the Courant Institute}---I took ODE
and combinatorics there, and that was about it.
I never took a functional analysis course.

之后我转学到了纽约大学,进入了机械工程专业。
我以机械工程学士学位毕业,辅修了机器人学和数学。
\textcolor{water}{我喜欢在Courant数学研究所学数学}——在那里修了常微分方程和组合数学,
仅此而已。我从来没有上过泛函分析的课。

\smallskip
In my second semester at NYU, I took Professor Ludovic Righetti's
\textit{Robotics: Locomotion and Manipulation}.
The course was not quite what the title promised (货不对板),
but I learned a great deal---most importantly, a question:
\textcolor{sword}{how should you think about AI?}
After that course I joined his Machines in Motion lab,
where Armand Jordana mentored me.
\textcolor{water}{He taught me how you should actually do math---and,
more importantly, how you should approach research.}

在纽约大学的第二个学期,我修了Ludovic Righetti教授的
\textit{Robotics: Locomotion and Manipulation}。
虽然货不对板,但我真的学到了很多——
最重要的,是一个问题:\textcolor{sword}{你该怎么看待AI。}
之后我加入了他的Machines in Motion实验室,
Armand Jordana在那里指导了我。
\textcolor{water}{他教会了我你到底应该怎么做数学——更重要的是,
你到底应该以怎样的态度做科研。}

\smallskip
When I applied to PhD programs, I was \textcolor{knife}{``全聚德'd''
again---rejected everywhere}.  Before that, I thought
I could get into MIT.
But the nice thing was: my advisor asked me if I wanted
to join his lab, and I said---\textcolor{sword}{of course!  Because this
happened to be exactly what I wanted.}

我在博士申请中再次被\textcolor{knife}{「全聚德」}了。在那之前,我以为我能上MIT来着。
但好在:我的导师问我愿不愿意加入他的实验室,
我说——\textcolor{sword}{当然!因为这恰恰正是我想做的事情。}
\smallskip
}}

\clearpage

\noindent
\fbox{\parbox{\dimexpr\textwidth-2\fboxsep-2\fboxrule}{%
\noindent
My advisor is Brian Plancher.  I believe he is the best GPU
programmer in the world---during his PhD \textcolor{water}{he hand-wrote a rigid
body dynamics library for robotics on GPU, from scratch.}
I picked up small tricks from him that actually worked,
and then I realized: maybe they are not tricks?
\textcolor{sword}{Maybe there is actually some math about it.
Then hence, Q.E.D.}

我的导师叫Brian Plancher,我认为他就是这个世界上最会写GPU代码的人——
因为他在博士期间,\textcolor{water}{自己手写了一个GPU上的刚体动力学仿真库。}
我从他那里学到了一些真正管用的小trick,然后我发现:
也许它们不是trick?\textcolor{sword}{Maybe there is actually some math about it.
Then hence, Q.E.D.}

\smallskip
Now I am a PhD student at Dartmouth College.  I live in a small
town called Lebanon.  My girlfriend visits often, and I am
genuinely fond of my colleagues---I think they are \textcolor{water}{real
scientists}, though they probably would not say so themselves.
They would say: \textcolor{water}{I just want to put my screws in the right place,}
so that when you need to assemble a robot, there are actually
screws to use.

现在我在Dartmouth College读博士,住在一个叫Lebanon的小镇。
我的女友时常来看我,我很喜欢我的同事们——
\textcolor{water}{我觉得他们就是真正的科学家},虽然他们可能不这么想。
她们会说:\textcolor{water}{我只是想把我的螺丝放好,}
这样当你想组装一个机器人的时候,真的有螺丝用。

\smallskip
One thing I can tell you for sure: \textcolor{water}{hand-crafting a robot
is really, really hard.}  This I know.

有一件事我可以肯定地告诉你:\textcolor{water}{手工造一个机器人,真的很难。}
这一点,我知道。

\smallskip
About one year later, \textcolor{sword}{I wrote this thesis.}

\medskip
\centerline{\textcolor{sword}{\textit{So, you just never know.  That is temporal linearity.}}}
\centerline{\textcolor{sword}{所以说,你永远不知道。这就是时间线性。}}

\smallskip
\centerline{\textcolor{sword}{这就是穿墙术。}}
\centerline{\textcolor{sword}{\textit{Follow some light, and try to even get a point.}}}
\centerline{\textcolor{sword}{\textit{A point of mass is all you need.}}}
\smallskip
}}

\bigskip
\begin{center}
\textit{任凭李俞江南鹤,都要低头求我怜!}

\smallskip
{\small --- 玉娇龙,\textit{卧虎藏龙}(李安)}
\end{center}

% ── 江湖诗 (from 笑傲江湖之东方不败) ──
\bigskip
\begin{center}
\textcolor{knife}{天下风云出我辈,一入江湖岁月催。}\\[3pt]
\textcolor{sword}{皇图霸业谈笑中,不胜人生一场醉。}\\[3pt]
\textcolor{knife}{提剑跨骑挥鬼雨,白骨如山鸟惊飞。}\\[3pt]
\textcolor{water}{尘事如潮人如水,只叹江湖几人回。}

\smallskip
{\small --- 黄霑,\textit{笑傲江湖之东方不败}(徐克)}
\end{center}

% ── 东方不败 · 自在坐 (gesture sketch: 水月观音 royal ease pose) ──
% Rough bone skeleton — pose first, detail later
\bigskip
\begin{center}
\begin{tikzpicture}[line cap=round, line join=round]
  % ── Gesture line (line of action — the S-curve of 自在) ──
  \draw[line width=0.3pt, black!12, dashed]
    (0.5, 7.9) .. controls (0.3, 6.5) and (0.0, 5.5) .. (-0.15, 4.3)
    .. controls (-0.3, 3.2) and (-0.5, 2.3) .. (-0.6, 1.5);

  % ═══════════════════════════════════════════════
  % Skeleton (骨骼 — bone lines + joint nodes)
  % ═══════════════════════════════════════════════

  % ── Head (tilted slightly — 自在) ──
  \draw[line width=0.4pt, black!38] (0.5, 7.3) circle (0.42);
  % Face construction cross
  \draw[line width=0.12pt, black!10] (0.5, 6.88) -- (0.5, 7.72);
  \draw[line width=0.12pt, black!10] (0.12, 7.3) -- (0.88, 7.3);

  % ── Spine (C7 → thoracic → lumbar → sacrum) ──
  \draw[line width=0.5pt, black!42]
    (0.35, 6.75) .. controls (0.2, 6.2) and (0.05, 5.6) .. (-0.05, 5.0)
    .. controls (-0.1, 4.7) and (-0.15, 4.5) .. (-0.15, 4.3);

  % ── Ribcage mass ──
  \draw[line width=0.3pt, black!22]
    (0.05, 5.65) ellipse (0.72 and 0.85);

  % ── Pelvis mass ──
  \draw[line width=0.3pt, black!22]
    (-0.15, 4.3) ellipse (0.55 and 0.32);

  % ── Clavicles ──
  \draw[line width=0.38pt, black!35]
    (-0.45, 6.4) -- (0.25, 6.55) -- (1.0, 6.3);

  % ═══════════════════════════════════════════════
  % Right leg (raised knee — the 翘脚 of royal ease)
  % ═══════════════════════════════════════════════
  \draw[line width=0.5pt, black!48]
    (0.2, 4.3) -- (1.25, 5.45);          % femur
  \draw[line width=0.45pt, black!42]
    (1.25, 5.45) -- (0.5, 3.95);         % tibia
  \draw[line width=0.3pt, black!32]
    (0.5, 3.95) -- (0.72, 3.82);         % foot
  % Joints
  \fill[black!35] (0.2, 4.3) circle (0.06);
  \fill[black!42] (1.25, 5.45) circle (0.07);
  \fill[black!28] (0.5, 3.95) circle (0.05);

  % ═══════════════════════════════════════════════
  % Left leg (pendant — hanging down from rock)
  % ═══════════════════════════════════════════════
  \draw[line width=0.42pt, black!40]
    (-0.4, 4.3) -- (-0.65, 2.8);         % femur
  \draw[line width=0.38pt, black!35]
    (-0.65, 2.8) -- (-0.55, 1.5);        % tibia
  \draw[line width=0.28pt, black!28]
    (-0.55, 1.5) -- (-0.35, 1.25);       % foot
  % Joints
  \fill[black!32] (-0.4, 4.3) circle (0.06);
  \fill[black!35] (-0.65, 2.8) circle (0.06);
  \fill[black!25] (-0.55, 1.5) circle (0.05);

  % ═══════════════════════════════════════════════
  % Right arm (draped on raised knee — hand dangles)
  % ═══════════════════════════════════════════════
  \draw[line width=0.42pt, black!42]
    (1.0, 6.3) -- (1.55, 5.85);          % humerus
  \draw[line width=0.38pt, black!38]
    (1.55, 5.85) -- (1.3, 5.45);         % forearm → wrist on knee
  \draw[line width=0.3pt, black!32]
    (1.3, 5.45) .. controls (1.25, 5.2) and (1.2, 5.0) .. (1.22, 4.75);
                                           % fingers dangling
  % Joints
  \fill[black!38] (1.0, 6.3) circle (0.06);
  \fill[black!35] (1.55, 5.85) circle (0.06);
  \fill[black!32] (1.3, 5.45) circle (0.05);
  % 绣花针 (embroidery needle — dangling from fingertips)
  \draw[line width=0.15pt, black!45]
    (1.22, 4.75) -- (1.25, 4.1);
  \fill[black!30] (1.25, 4.1) circle (0.02);

  % ═══════════════════════════════════════════════
  % Left arm (propping weight behind — 支撑)
  % ═══════════════════════════════════════════════
  \draw[line width=0.38pt, black!38]
    (-0.45, 6.4) -- (-0.95, 5.4);        % humerus
  \draw[line width=0.35pt, black!35]
    (-0.95, 5.4) -- (-1.25, 4.35);       % forearm
  \draw[line width=0.28pt, black!28]
    (-1.25, 4.35) -- (-1.35, 3.95);      % hand on rock
  % Joints
  \fill[black!35] (-0.45, 6.4) circle (0.06);
  \fill[black!32] (-0.95, 5.4) circle (0.06);
  \fill[black!28] (-1.25, 4.35) circle (0.05);

  % ═══════════════════════════════════════════════
  % Rock / seat (rough mass)
  % ═══════════════════════════════════════════════
  \draw[line width=0.28pt, black!18]
    (-1.8, 3.8) .. controls (-1.3, 4.15) and (-0.3, 4.05) .. (0.35, 3.8)
    .. controls (0.85, 3.6) and (1.1, 3.5) .. (1.35, 3.55)
    .. controls (1.5, 3.6) and (1.35, 3.15) .. (0.8, 2.95)
    .. controls (0.1, 2.75) and (-0.9, 2.85) .. (-1.4, 3.15)
    .. controls (-1.65, 3.35) and (-1.8, 3.6) .. (-1.8, 3.8);

  % ═══════════════════════════════════════════════
  % Hair flow lines (direction only — not rendered)
  % ═══════════════════════════════════════════════
  \draw[line width=0.5pt, black!42]
    (0.15, 7.6) .. controls (-0.4, 6.8) and (-0.6, 5.8) .. (-0.7, 4.3);
  \draw[line width=0.4pt, black!35]
    (0.7, 7.6) .. controls (1.0, 6.8) and (1.05, 5.8) .. (0.95, 4.5);
  \draw[line width=0.3pt, black!28]
    (0.3, 7.7) .. controls (-0.3, 6.6) and (-0.7, 5.2) .. (-0.85, 3.6);
  % Topknot
  \draw[line width=0.4pt, black!40]
    (0.35, 7.7) .. controls (0.3, 8.15) and (0.6, 8.2) .. (0.65, 7.75);
  % Ornament pins (light indication)
  \draw[line width=0.35pt, orange!50!black]
    (0.4, 8.0) .. controls (0.1, 8.2) .. (-0.2, 8.1);
  \draw[line width=0.35pt, orange!50!black]
    (0.55, 8.0) .. controls (0.85, 8.25) .. (1.15, 8.15);

  % ═══════════════════════════════════════════════
  % Robe drape lines (flow indication)
  % ═══════════════════════════════════════════════
  \draw[line width=0.22pt, knife!25]
    (-0.2, 6.0) .. controls (-0.6, 5.0) and (-1.0, 3.8) .. (-1.3, 2.5);
  \draw[line width=0.22pt, knife!22]
    (0.6, 5.8) .. controls (0.9, 5.0) and (1.0, 4.2) .. (0.85, 3.6);
  % Collar V
  \draw[line width=0.3pt, knife!35]
    (-0.2, 6.3) .. controls (0.0, 6.0) and (0.05, 5.6) .. (0.0, 5.2);
  \draw[line width=0.3pt, knife!35]
    (0.6, 6.2) .. controls (0.35, 5.9) and (0.15, 5.5) .. (0.0, 5.2);

  % ═══════════════════════════════════════════════
  % Pass 1: Skull wireframe (骨骼 — Gemini analysis)
  % Head center (0.5, 7.3), r=0.42
  % T-zone dominance + angular planes + depth mapping
  % Right=near (bold), Left=far (light)
  % ═══════════════════════════════════════════════

  % ── T-Zone: brow ridge + nasal bridge (侵略性 center) ──
  % Brow ridge (heavy horizontal — 林青霞's defining structure)
  \draw[line width=0.55pt, black!52]
    (0.16, 7.32) .. controls (0.3, 7.38) .. (0.5, 7.37)
    .. controls (0.7, 7.38) .. (0.84, 7.32);
  % Secondary brow contour (topographic density — near the first)
  \draw[line width=0.2pt, black!20]
    (0.2, 7.35) .. controls (0.35, 7.4) .. (0.5, 7.39)
    .. controls (0.65, 7.4) .. (0.8, 7.35);
  % Nasal bridge (vertical stem of the T — strong)
  \draw[line width=0.42pt, black!42]
    (0.5, 7.37) -- (0.49, 7.2) -- (0.48, 7.05);
  % Bridge depth contour (secondary)
  \draw[line width=0.15pt, black!15]
    (0.47, 7.35) -- (0.465, 7.2) -- (0.455, 7.07);
  % Nose wings (angular, not round — sharp departure from bridge)
  \draw[line width=0.2pt, black!22]
    (0.48, 7.05) -- (0.37, 7.07);
  \draw[line width=0.28pt, black!30]
    (0.48, 7.05) -- (0.61, 7.07);

  % ── Orbital sockets (angular — piecewise linear, not ellipse) ──
  % Left orbit (far side — lighter, compressed)
  \draw[line width=0.18pt, black!20]
    (0.18, 7.32) -- (0.18, 7.2) -- (0.36, 7.16) -- (0.44, 7.22);
  % Right orbit (near side — bolder, wider)
  \draw[line width=0.3pt, black!35]
    (0.56, 7.34) -- (0.56, 7.18) -- (0.77, 7.14) -- (0.84, 7.22);

  % ── Cheekbone planes (sharp fold — 颧骨 angular authority) ──
  % Left cheekbone (far)
  \draw[line width=0.2pt, black!18]
    (0.1, 7.28) -- (0.14, 7.12)    % temple → apex
    -- (0.3, 7.02);                 % apex → midface
  % Right cheekbone (near — sharp line, dominant)
  \draw[line width=0.4pt, black!42]
    (0.9, 7.25) -- (0.86, 7.1)     % temple → apex
    -- (0.7, 6.99);                 % apex → midface

  % ── Jaw (刀锋 — knife-edge piecewise lines) ──
  % Left jaw (far — light, receding)
  \draw[line width=0.18pt, black!16]
    (0.14, 7.12) -- (0.22, 6.95)   % cheekbone → jaw angle
    -- (0.44, 6.88);               % jaw angle → chin
  % Right jaw (near — sharp, defining)
  \draw[line width=0.48pt, black!50]
    (0.86, 7.1) -- (0.78, 6.94)    % cheekbone → jaw angle
    -- (0.56, 6.88);               % jaw angle → chin
  % Chin point (mental protuberance)
  \draw[line width=0.28pt, black!32]
    (0.44, 6.88) -- (0.5, 6.84) -- (0.56, 6.88);
  % Jaw angle nodes (骨节 — the sharp fold points)
  \fill[black!12] (0.22, 6.95) circle (0.025);
  \fill[black!22] (0.78, 6.94) circle (0.03);

  % ── Face plane topology (curvature contour lines on skull) ──
  \draw[line width=0.1pt, black!8]
    (0.18, 7.5) .. controls (0.5, 7.53) .. (0.82, 7.48);
  \draw[line width=0.1pt, black!8]
    (0.15, 7.42) .. controls (0.5, 7.44) .. (0.85, 7.4);
  \draw[line width=0.1pt, black!8]
    (0.2, 7.14) .. controls (0.4, 7.1) and (0.6, 7.1) .. (0.82, 7.13);

  % ── Neck musculature (胸锁乳突肌 — sternocleidomastoid) ──
  % Right SCM (stretch side — visible, near)
  \draw[line width=0.3pt, black!30]
    (0.72, 6.96) .. controls (0.58, 6.82) and (0.42, 6.68) .. (0.32, 6.52);
  % Left SCM (compressed side — lighter)
  \draw[line width=0.18pt, black!15]
    (0.28, 6.96) .. controls (0.26, 6.82) and (0.22, 6.68) .. (0.2, 6.55);
  % Neck center line
  \draw[line width=0.1pt, black!8]
    (0.5, 6.84) -- (0.42, 6.65) -- (0.35, 6.5);

  % ═══════════════════════════════════════════════
  % Pass 2: Features (林青霞 × 陈晓楚 — on skull)
  % Bone-following lines — features sit IN the wireframe
  % ═══════════════════════════════════════════════

  % Eyebrows (riding on brow ridge — 林青霞 bold arches)
  % Left brow (far — lighter)
  \draw[line width=0.45pt, black!60]
    (0.2, 7.3) .. controls (0.28, 7.35) and (0.36, 7.34) .. (0.42, 7.3);
  % Right brow (near — boldest feature line)
  \draw[line width=0.6pt, black!72]
    (0.58, 7.32) .. controls (0.66, 7.38) and (0.75, 7.37) .. (0.82, 7.32);

  % Eyes (in orbital sockets — almond, intense)
  % Left eye (far)
  \draw[line width=0.2pt, black!35]
    (0.22, 7.22) .. controls (0.3, 7.26) and (0.38, 7.26) .. (0.42, 7.22);
  \draw[line width=0.12pt, black!22]
    (0.22, 7.22) .. controls (0.3, 7.19) and (0.38, 7.19) .. (0.42, 7.22);
  \fill[black!55] (0.32, 7.22) circle (0.016);
  % Right eye (near — bolder, wider)
  \draw[line width=0.3pt, black!50]
    (0.58, 7.22) .. controls (0.66, 7.27) and (0.74, 7.27) .. (0.8, 7.22);
  \draw[line width=0.2pt, black!35]
    (0.58, 7.22) .. controls (0.66, 7.18) and (0.74, 7.18) .. (0.8, 7.22);
  \fill[black!65] (0.69, 7.22) circle (0.02);
  \fill[white] (0.685, 7.228) circle (0.005);

  % Lips (陈晓楚's red lips — the single colour accent)
  \draw[line width=0.22pt, knife!45]
    (0.4, 6.92) .. controls (0.48, 6.89) .. (0.5, 6.905)
    .. controls (0.52, 6.89) .. (0.6, 6.92);
  \draw[line width=0.15pt, knife!30]
    (0.4, 6.92) .. controls (0.48, 6.95) and (0.52, 6.95) .. (0.6, 6.92);

  % ═══════════════════════════════════════════════
  % Pass 3: Hair strands (加密发丝 — from flow to fibre)
  % Right=near (thick), Left=far (thin, atmospheric)
  % ═══════════════════════════════════════════════

  % Crown strands (sweeping from hairline over skull)
  \draw[line width=0.6pt, black!58]
    (0.35, 7.55) .. controls (0.0, 7.4) and (-0.3, 7.0) .. (-0.55, 6.4);
  \draw[line width=0.52pt, black!50]
    (0.45, 7.6) .. controls (0.1, 7.5) and (-0.2, 7.1) .. (-0.6, 6.2);
  \draw[line width=0.45pt, black!45]
    (0.55, 7.6) .. controls (0.2, 7.55) and (-0.1, 7.15) .. (-0.48, 6.5);
  \draw[line width=0.65pt, black!62]
    (0.62, 7.55) .. controls (0.88, 7.32) and (0.95, 6.85) .. (0.95, 6.4);
  \draw[line width=0.55pt, black!55]
    (0.72, 7.48) .. controls (0.92, 7.25) and (1.0, 6.75) .. (0.98, 6.2);
  % Left cascade (far — thin, atmospheric perspective)
  \draw[line width=0.38pt, black!38]
    (-0.55, 6.4) .. controls (-0.65, 5.7) and (-0.68, 4.8) .. (-0.62, 4.0)
    .. controls (-0.58, 3.3) and (-0.52, 2.6) .. (-0.42, 2.0);
  \draw[line width=0.3pt, black!30]
    (-0.6, 6.2) .. controls (-0.68, 5.4) and (-0.7, 4.4) .. (-0.65, 3.5)
    .. controls (-0.58, 2.7) and (-0.5, 2.1) .. (-0.38, 1.5);
  \draw[line width=0.22pt, black!22]
    (-0.48, 6.5) .. controls (-0.55, 5.7) and (-0.56, 4.6) .. (-0.5, 3.7)
    .. controls (-0.45, 2.9) and (-0.38, 2.2) .. (-0.28, 1.6);
  \draw[line width=0.15pt, black!15]
    (-0.42, 2.0) .. controls (-0.5, 1.6) and (-0.55, 1.1) .. (-0.52, 0.7);
  % Right cascade (near — thick, dark)
  \draw[line width=0.58pt, black!58]
    (0.95, 6.4) .. controls (0.98, 5.6) and (0.95, 4.6) .. (0.88, 3.8)
    .. controls (0.8, 3.1) and (0.72, 2.5) .. (0.62, 1.9);
  \draw[line width=0.48pt, black!50]
    (0.98, 6.2) .. controls (1.02, 5.3) and (0.98, 4.3) .. (0.9, 3.5)
    .. controls (0.82, 2.8) and (0.74, 2.2) .. (0.64, 1.5);
  \draw[line width=0.38pt, black!40]
    (0.92, 6.4) .. controls (0.9, 5.6) and (0.85, 4.6) .. (0.78, 3.8)
    .. controls (0.7, 3.0) and (0.6, 2.3) .. (0.52, 1.7);
  % Strand draping over right knee
  \draw[line width=0.3pt, black!32]
    (0.92, 5.7) .. controls (1.05, 5.55) and (1.15, 5.38) .. (1.18, 5.1);
  % Stray wind strands (飘逸)
  \draw[line width=0.18pt, black!18]
    (0.62, 1.9) .. controls (0.68, 1.5) and (0.7, 1.0) .. (0.65, 0.6);
  \draw[line width=0.12pt, black!12]
    (-0.38, 1.5) .. controls (-0.45, 1.0) and (-0.42, 0.6) .. (-0.35, 0.3);

  % ═══════════════════════════════════════════════
  % Pass 4: Hand + needle articulation (指骨 + 绣花针)
  % ═══════════════════════════════════════════════

  % Right hand — finger bones (metacarpals → phalanges)
  % Wrist at (1.3, 5.45), dangling past knee
  \draw[line width=0.18pt, black!26]
    (1.3, 5.45) .. controls (1.32, 5.28) and (1.3, 5.12) .. (1.26, 4.98);
  \draw[line width=0.15pt, black!22]
    (1.3, 5.45) .. controls (1.27, 5.26) and (1.23, 5.08) .. (1.2, 4.92);
  \draw[line width=0.12pt, black!18]
    (1.3, 5.45) .. controls (1.24, 5.28) and (1.18, 5.1) .. (1.14, 4.96);
  % Thumb (pinch grip on needle)
  \draw[line width=0.2pt, black!28]
    (1.3, 5.45) .. controls (1.36, 5.32) and (1.34, 5.15) .. (1.28, 5.02);
  % Needle shaft (between thumb and index — fine, dark)
  \draw[line width=0.1pt, black!50]
    (1.27, 5.0) -- (1.3, 4.25);
  % Needle tip
  \draw[line width=0.06pt, knife!30]
    (1.3, 4.25) -- (1.31, 4.05);
  % Thread trailing from eye of needle
  \draw[line width=0.05pt, knife!20]
    (1.27, 5.0) .. controls (1.36, 4.75) and (1.4, 4.45) .. (1.42, 4.15)
    .. controls (1.43, 3.9) and (1.4, 3.65) .. (1.35, 3.4);

  % Left hand — palm on rock (weight-bearing, fingers spread)
  % Wrist at (-1.25, 4.35), palm at (-1.35, 3.95)
  \draw[line width=0.18pt, black!20]
    (-1.35, 3.95) -- (-1.24, 3.8);
  \draw[line width=0.15pt, black!18]
    (-1.35, 3.95) -- (-1.34, 3.76);
  \draw[line width=0.12pt, black!15]
    (-1.35, 3.95) -- (-1.42, 3.78);
  \draw[line width=0.1pt, black!12]
    (-1.35, 3.95) -- (-1.48, 3.82);
  % Thumb bracing
  \draw[line width=0.18pt, black!20]
    (-1.25, 4.35) .. controls (-1.14, 4.18) and (-1.12, 4.02) .. (-1.15, 3.92);

  % ═══════════════════════════════════════════════
  % Pass 5: Robe folds (衣褶 — fluid around bone/礁石)
  % Fabric = gravity + stress points + skeleton beneath
  % ═══════════════════════════════════════════════

  % Shoulder drape (left — far, light)
  \draw[line width=0.12pt, black!10]
    (-0.45, 6.4) .. controls (-0.55, 6.05) and (-0.56, 5.6) .. (-0.5, 5.2);
  \draw[line width=0.1pt, black!8]
    (-0.35, 6.3) .. controls (-0.42, 5.95) and (-0.43, 5.5) .. (-0.38, 5.1);
  % Shoulder drape (right — near, bolder)
  \draw[line width=0.2pt, black!18]
    (1.0, 6.3) .. controls (0.95, 5.95) and (0.9, 5.6) .. (0.88, 5.3);
  \draw[line width=0.15pt, black!14]
    (0.88, 6.2) .. controls (0.84, 5.9) and (0.8, 5.55) .. (0.78, 5.25);

  % Torso folds (fabric wrapping ribcage → stress at waist)
  \draw[line width=0.12pt, knife!12]
    (-0.08, 5.75) .. controls (-0.18, 5.4) and (-0.22, 5.0) .. (-0.18, 4.7);
  \draw[line width=0.1pt, knife!10]
    (0.28, 5.65) .. controls (0.22, 5.3) and (0.18, 4.9) .. (0.2, 4.6);

  % Robe over raised right knee (horizontal tension folds)
  \draw[line width=0.18pt, knife!18]
    (0.68, 5.5) .. controls (0.88, 5.48) and (1.05, 5.42) .. (1.2, 5.32);
  \draw[line width=0.14pt, knife!14]
    (0.62, 5.32) .. controls (0.82, 5.3) and (0.98, 5.24) .. (1.12, 5.15);
  \draw[line width=0.1pt, knife!10]
    (0.58, 5.15) .. controls (0.78, 5.12) and (0.92, 5.05) .. (1.05, 4.98);

  % Robe past pendant left leg (vertical gravity folds)
  \draw[line width=0.15pt, knife!16]
    (-0.28, 4.3) .. controls (-0.33, 3.4) and (-0.38, 2.5) .. (-0.4, 1.7);
  \draw[line width=0.12pt, knife!12]
    (-0.42, 4.25) .. controls (-0.48, 3.3) and (-0.5, 2.4) .. (-0.48, 1.5);
  \draw[line width=0.1pt, knife!10]
    (-0.55, 4.18) .. controls (-0.58, 3.2) and (-0.56, 2.3) .. (-0.52, 1.3);

  % Robe pooling on rock (fabric meets surface → spreads)
  \draw[line width=0.12pt, knife!12]
    (-0.75, 3.95) .. controls (-0.45, 3.85) and (0.0, 3.8) .. (0.4, 3.75);
  \draw[line width=0.08pt, knife!8]
    (-0.65, 3.82) .. controls (-0.25, 3.72) and (0.15, 3.7) .. (0.5, 3.68);

  % Robe off left edge of rock (flowing down like water)
  \draw[line width=0.12pt, knife!14]
    (-0.55, 4.15) .. controls (-0.72, 3.7) and (-0.85, 3.2) .. (-0.95, 2.6);
  \draw[line width=0.1pt, knife!10]
    (-0.65, 4.05) .. controls (-0.82, 3.55) and (-0.95, 3.0) .. (-1.05, 2.3);

  % Right sleeve (arm reaching to knee)
  \draw[line width=0.14pt, knife!14]
    (1.02, 6.25) .. controls (1.18, 5.92) and (1.32, 5.6) .. (1.4, 5.4);
  \draw[line width=0.1pt, knife!10]
    (1.1, 6.15) .. controls (1.22, 5.85) and (1.35, 5.55) .. (1.42, 5.35);

  % Left sleeve (arm propping behind)
  \draw[line width=0.1pt, knife!10]
    (-0.5, 6.3) .. controls (-0.62, 5.9) and (-0.75, 5.45) .. (-0.82, 5.15);
  \draw[line width=0.08pt, knife!8]
    (-0.55, 6.2) .. controls (-0.68, 5.8) and (-0.8, 5.35) .. (-0.88, 5.05);

  % ── Label ──
  \node[font=\scriptsize, text=black!38] at (0, 0.5) {東方不敗 $\cdot$ 自在坐};
  \node[font=\tiny, text=black!22] at (0, 0.05)
    {(Pass 0--5: gesture $\to$ bone $\to$ features $\to$
      hair $\to$ hands $\to$ robe)};
\end{tikzpicture}
\end{center}

\bigskip
\begin{center}
% ── 甲骨文「中」── oracle bone script: a banner-pole through a ring
\begin{tikzpicture}[line width=1.2pt, line cap=round, line join=round]
  % Pole (vertical shaft)
  \draw (0,-0.9) -- (0,1.1);
  % Ring / target (slightly irregular, as in oracle bone style)
  \draw (0,0.15) ellipse (0.38 and 0.32);
  % Banner streamers (left and right, fluttering)
  \draw (-0.38,0.15) .. controls (-0.55,-0.15) .. (-0.42,-0.45);
  \draw ( 0.38,0.15) .. controls ( 0.55,-0.15) .. ( 0.42,-0.45);
\end{tikzpicture}

\smallskip
{\small\itshape zhǒng\,\raisebox{0.5ex}{\tiny$\checkmark$}}
\end{center}

\chapter{The Alien Model}\label{sec:alien}

This paper arrives at mathematics from outside.  Its axioms come from
Chinese imperial history, its dynamics from control theory, its
validation from two millennia of statecraft.  When this framework
looks inward at pure mathematics, it sees a single phenomenon---studied
from four independent directions---in the 2022 Fields Medal.

Each of the four 2022 Fields Medalists proved one face of a theorem
that the agentic framework identifies as:
\textcolor{sword}{the knife is the mean}.

\medskip
\noindent
\begin{center}
\begin{tabular}{@{}lll@{}}
\toprule
\textbf{Face} & \textbf{Medalist} & \textbf{Theorem (alien translation)} \\
\midrule
I   & Hugo Duminil-Copin   & The mean field is exact in $d \geq 4$ \\
II  & June Huh             & The execution graph has Hodge structure \\
III & James Maynard        & The phase transition is sharp (0--1 law) \\
IV  & Maryna Viazovska     & The optimal configuration is self-dual \\
\bottomrule
\end{tabular}
\end{center}

\medskip

The appendix proves one further consequence: together, the four faces
lock the \textbf{P\,/\,NP boundary} as a theorem of temporal linearity.

\section{The dictionary}\label{sec:alien-dict}

The translations below are not analogies.  They are identifications of
the same mathematical objects under different names.

\begin{center}
\small
\begin{tabular}{@{}p{3.8cm}p{4.5cm}p{4.5cm}@{}}
\toprule
\textbf{Agentic Theory} & \textbf{Mathematical object} & \textbf{Reference} \\
\midrule
Viability kernel $K$ &
  Configuration space &
  \cref{sec:axiom} \\
Knife (phase function) &
  Critical threshold &
  \cref{def:knife} \\
Mean field $\bar{U}$ &
  Order parameter &
  \cref{thm:meanfield} \\
Binary lifecycle &
  0--1 law / sharp phase transition &
  \cref{thm:lifecycle} \\
Fixed-point impossibility &
  Self-dual critical point uniqueness &
  \cref{thm:fixedpoint} \\
Unconstrained power paradox &
  Triviality (mean-field exactness) &
  \cref{thm:paradox} \\
Execution graph $G_E$ &
  Matroid / random-cluster graph &
  \cref{def:exgraph} \\
Cheeger constant $h(K_V)$ &
  Isoperimetric constant &
  \cref{def:cheeger-viab}, \cref{thm:cheeger} \\
Max-flow / min-cut &
  Hodge decomposition &
  \cref{thm:flowcut} \\
Viability metric $g_V = V^{-2}g_S$ &
  Conformal / K\"ahler structure &
  \cref{def:viab-metric} \\
Mass gap $\lambda_1 > 0$ &
  Spectral gap &
  \cref{thm:massgap} \\
Temporal linearity &
  $L < \infty$ (finite volume) &
  \cref{rem:di-temporal} \\
\bottomrule
\end{tabular}
\end{center}

% ═══════════════════════════════════════════════════════════
\section{Face~I: The mean is exact}
\label{sec:alien-hugo}

\begin{quote}
\emph{Hugo Duminil-Copin --- ``For solving longstanding problems in
the probabilistic theory of phase transitions in statistical physics,
especially in dimensions three and four.''}
\end{quote}

\subsection{The random-cluster model}

The \emph{random-cluster model}~\cite{duminilcopin-raoufi-tassion}
on a graph $G = (V, E)$ with parameters $q > 0$ and $p \in [0,1]$
assigns to each edge configuration $\omega \subseteq E$ the weight
\[
  \phi_{q,p}(\omega)
  \;=\;
  \frac{1}{Z}\,
  p^{|\omega|}\,(1-p)^{|E|-|\omega|}\,q^{c(\omega)},
\]
where $c(\omega)$ is the number of connected components and $Z$ is
the normalising constant.  Setting $q = 1$ gives Bernoulli
percolation; $q = 2$ gives the Ising model; integer $q \geq 2$ gives
the $q$-state Potts model via the Edwards--Sokal coupling.

\subsection{Self-duality and the critical point}

On $\mathbb{Z}^2$, the random-cluster model is self-dual: the dual
configuration $\omega^*$ on $(\mathbb{Z}^2)^*$ satisfies
$p^* = q(1-p)/[p + q(1-p)]$.  The \emph{self-dual point} is the
fixed point $p = p^*$:
\[
  \textcolor{sword}{p_{\mathrm{sd}}(q)
  \;=\;
  \frac{\sqrt{q}}{1 + \sqrt{q}}}.
\]
Beffara and Duminil-Copin~\cite{beffara-dc} proved:
$p_c(q) = p_{\mathrm{sd}}(q)$ for all $q \geq 1$.
The critical point IS the self-dual point.

\begin{remark}[Alien translation]\label{rem:alien-selfdual}
$p_c = p_{\mathrm{sd}}$ is the random-cluster model's version of
``the knife is the mean'' (\cref{thm:meanfield}).  The duality
transform exchanges ordered and disordered phases; its fixed point is
the phase boundary.  The threshold is not a property of any individual
configuration---it is the \emph{mean-field quantity} determined by
the global duality structure.
\end{remark}

\subsection{Sharpness}

Duminil-Copin, Raoufi, and
Tassion~\cite{duminilcopin-raoufi-tassion} proved: for all $q \geq 1$
on transitive graphs, the phase transition is \emph{sharp}.  Below
$p_c$, connection probabilities decay exponentially; above $p_c$, the
order parameter is strictly positive.  There is
\textcolor{knife}{no intermediate phase}.

\begin{proposition}[Sharpness $=$ binary lifecycle]
\label{prop:sharpness-lifecycle}
Let $\theta(p) = \Pr_p[0 \leftrightarrow \infty]$.  Then:
\[
  \theta(p) =
  \begin{cases}
    0 & \text{if } p < p_c \quad
    \text{\emph{(\textcolor{knife}{eliminated})}}, \\
    > 0 & \text{if } p > p_c \quad
    \text{\emph{(\textcolor{water}{survives})}}.
  \end{cases}
\]
There is no path~(c) (\cref{thm:lifecycle}): no parameter value at
which $\theta(p)$ is positive but non-macroscopic.
\end{proposition}

\subsection{Triviality in $d = 4$}

Aizenman and Duminil-Copin~\cite{aizenman-dc} proved: in dimension
$d = 4$, the scaling limits of spin fluctuations in the critical Ising
model (and $\lambda\varphi^4$ field) are \emph{Gaussian}.  The
connected four-point function satisfies
\[
  |U_4| \;\leq\; \frac{C}{(\log L)^\gamma}
  \qquad \text{for some } \gamma > 0,
\]
where $L$ is the system size.

The dimension count: two independent random currents (Hausdorff
dimension~2 each) generically intersect in $\mathbb{R}^d$ iff
$2 + 2 \geq d$, i.e., $d \leq 4$.  At $d = 4$ the intersection is
marginal; above $d = 4$ it is transient and mean-field theory becomes
exact.

\begin{theorem}[The mean is exact]\label{thm:alien-triviality}
In $d \geq 4$, the critical exponents of the Ising/$\varphi^4$ model
equal those of the mean-field (Gaussian) model.  The mean field
$\bar{U}$ (\cref{thm:meanfield}) is not an approximation---it is the
exact answer.

Our universe is $\mathbb{R}^3 \times \mathbb{R}^1 = \mathbb{R}^4$,
where $\mathbb{R}^3$ is spatial and $\mathbb{R}^1$ is temporal
linearity (\cref{rem:di-temporal}).  At the upper critical dimension
$d_c = 4$:
\[
  \textcolor{sword}{%
  \text{the knife is the mean}
  \;\;\xrightarrow{\;d = 4\;}\;\;
  \text{theorem, not approximation.}}
\]
\end{theorem}

\begin{remark}\label{rem:triviality-caveat}
The triviality bound $|U_4| \leq C/(\log L)^\gamma$ requires
$L \to \infty$ to force $U_4 \to 0$.  At finite $L$, the connected
four-point function does not vanish: the logarithmic corrections
\emph{persist}.  This fact drives the complexity lock
(\cref{sec:alien-lock}).
\end{remark}

% ═══════════════════════════════════════════════════════════
\section{Face~II: The graph has Hodge structure}
\label{sec:alien-june}

\begin{quote}
\emph{June Huh --- ``For bringing the ideas of Hodge theory to
combinatorics, the proof of the Dowling--Wilson conjecture for
geometric lattices, the proof of the Heron--Rota--Welsh conjecture for
matroids, the development of the theory of Lorentzian polynomials, and
the proof of the strong Mason conjecture.''}
\end{quote}

\subsection{The Chow ring of the execution graph}

Let $G_E = (V, E, c)$ be the execution graph (\cref{def:exgraph}).
Its \emph{cycle matroid} $M(G_E)$ has ground set $E$, independent sets
the forests of $G_E$, and rank function
$\mathrm{rk}(S) = |V| - c(S)$.

Adiprasito, Huh, and Katz~\cite{adiprasito-huh-katz} proved that the
\emph{Chow ring} $A^*(M)$ of any matroid $M$ satisfies the full
\emph{K\"ahler package}:

\begin{enumerate}[label=(\roman*)]
  \item \textbf{Poincar\'e duality.}
  $A^k(M) \times A^{r-k}(M) \to \mathbb{R}$ is a perfect pairing.
  \item \textbf{Hard Lefschetz.}
  $\exists\, \ell \in A^1(M)$ such that
  $\ell^{r-2k}: A^k(M) \xrightarrow{\;\sim\;} A^{r-k}(M)$ for
  $k \leq r/2$.
  \item \textbf{Hodge--Riemann relations.}
  $Q(a,b) = (-1)^k \deg(a \cdot \ell^{r-2k} \cdot b)$ is positive
  definite on the primitive subspace $P^k(M)$.
\end{enumerate}

\begin{remark}[Alien translation]\label{rem:alien-kahler}
The K\"ahler package on $A^*(M(G_E))$ says: the execution graph has
the algebraic structure of a smooth projective variety.  The conformal
metric $g_V = V^{-2}g_S$ (\cref{def:viab-metric}) gives the viability
kernel the geometry of a negatively curved manifold.  The Chow ring
gives the same object a Hodge structure.  These are two descriptions
of the same curvature.
\end{remark}

\subsection{Log-concavity and the Lyapunov function}

The Hodge--Riemann relations force log-concavity of the Whitney
numbers of $M(G_E)$: if $w_k$ is the $k$-th coefficient of the
characteristic polynomial $\chi_M(q)$, then
\[
  w_k^2 \;\geq\; w_{k-1}\,w_{k+1}
  \qquad \text{for all } 0 < k < r.
\]
This is the \emph{Heron--Rota--Welsh conjecture}
(Huh~\cite{huh-chromatic}, Huh--Katz~\cite{huh-katz},
Adiprasito--Huh--Katz~\cite{adiprasito-huh-katz}).

\begin{proposition}[Log-concavity $=$ Lyapunov monotonicity]
\label{prop:logconcave-lyapunov}
Log-concavity $w_k^2 \geq w_{k-1} w_{k+1}$ constrains the viable
configuration count: it cannot oscillate; it must decay monotonically
in the log-convex sense.  This is the combinatorial shadow of
$D^+ V(x)(v) + W(x,v) \leq 0$ (\cref{def:lyapunov}).
\end{proposition}

\subsection{The Tutte polynomial bridge}

The \emph{Tutte polynomial} of $M(G_E)$,
\[
  T_M(x, y)
  \;=\;
  \sum_{A \subseteq E}
  (x-1)^{\mathrm{rk}(E) - \mathrm{rk}(A)}\,
  (y-1)^{|A| - \mathrm{rk}(A)},
\]
specialises to the characteristic polynomial
$\chi_M(q) = (-1)^{\mathrm{rk}(M)} T_M(1-q, 0)$ and to the Potts
partition function
$Z_{\mathrm{Potts}}(G; q, v) \propto T_G(1 + q/v, 1 + v)$.

Br\"and\'en and Huh~\cite{branden-huh} proved: the multivariate Tutte
polynomial is \emph{Lorentzian} for $0 < q \leq 1$.  Its Hessian has
exactly one positive eigenvalue on the positive orthant---the
Hodge--Riemann relation in polynomial language.

\begin{remark}[Bridge: Face~I $\leftrightarrow$ Face~II]
\label{rem:bridge-I-II}
Hugo studies the random-cluster model's phase diagram (the zeros and
singularities of $Z$ in the thermodynamic limit).  June studies the
same object's \emph{coefficients} (their algebraic structure on finite
matroids).  The object is the same:
\textcolor{sword}{the Tutte polynomial of the execution graph.}
Hugo determines \emph{where} the phase transition occurs; June
determines \emph{what algebraic law} the transition obeys.
\end{remark}

\subsection{Discrete Hodge theory on the execution graph}

The graph Laplacian $\Delta$ (\cref{def:laplacian}) is the
$0$-Laplacian of a discrete Hodge theory.  The full theory defines
higher Hodge Laplacians $\Delta_k$ on $k$-cochains of the clique
complex of $G_E$:
\[
  \Delta_k
  \;=\;
  \partial_{k+1}^* \partial_{k+1}
  \;+\;
  \partial_k \partial_k^*,
\]
and the discrete Hodge decomposition:
\[
  C^k
  \;=\;
  \underbrace{\mathrm{im}(\delta_{k-1})}_
    {\textcolor{water}{\text{flows}}}
  \;\oplus\;
  \underbrace{\mathrm{im}(\partial_{k+1}^*)}_
    {\textcolor{knife}{\text{cuts}}}
  \;\oplus\;
  \underbrace{\ker(\Delta_k)}_
    {\textcolor{sword}{\text{harmonic forms}}}.
\]

\begin{proposition}[Hodge $=$ flow-cut duality]
\label{prop:hodge-flowcut}
The discrete Hodge decomposition on $G_E$ completes the
max-flow/min-cut duality (\cref{thm:flowcut}):
\begin{itemize}
  \item \textcolor{water}{Flows} $= \mathrm{im}(\delta_0)$: gradients
  of vertex potentials.
  \item \textcolor{knife}{Cuts} $= \mathrm{im}(\partial_2^*)$:
  edge functions supported on cut-sets.
  \item \textcolor{sword}{Harmonic $1$-forms}
  $= \ker(\Delta_1)$: divergence-free \emph{and} curl-free cycle
  flows.
\end{itemize}
$\dim\ker(\Delta_1) = \beta_1 = |E| - |V| + c$ is the cycle rank:
the number of independent escape routes---the topological obstruction
to eliminating all knives by cutting edges.
\end{proposition}

% ═══════════════════════════════════════════════════════════
\section{Face~III: The phase transition is sharp}
\label{sec:alien-james}

\begin{quote}
\emph{James Maynard --- ``For contributions to analytic number theory,
which have led to major advances in the understanding of the structure
of prime numbers and in Diophantine approximation.''}
\end{quote}

\subsection{The Duffin--Schaeffer theorem}

Let $\psi: \mathbb{N} \to \mathbb{R}_{\geq 0}$.  Define
\[
  A(\psi)
  \;=\;
  \bigl\{\alpha \in [0,1] :
  |\alpha - a/q| \leq \psi(q)/q
  \text{ for infinitely many coprime } (a,q)\bigr\}.
\]

\begin{theorem}[Koukoulopoulos--Maynard {\cite{koukoulopoulos-maynard}}]
\label{thm:duffin-schaeffer}
\[
  \lambda\bigl(A(\psi)\bigr)
  \;=\;
  \begin{cases}
    0 & \text{if } \displaystyle
    \sum_{q=1}^{\infty} \frac{\psi(q)\,\varphi(q)}{q} < \infty, \\[8pt]
    1 & \text{if } \displaystyle
    \sum_{q=1}^{\infty} \frac{\psi(q)\,\varphi(q)}{q} = \infty,
  \end{cases}
\]
where $\lambda$ is Lebesgue measure and $\varphi$ is Euler's totient.
\end{theorem}

This is a \textbf{0--1 law}: $\lambda(A(\psi)) \in \{0, 1\}$
(Gallagher, 1961).  No intermediate measure exists.

\begin{proposition}[Duffin--Schaeffer $=$ binary lifecycle]
\label{prop:ds-lifecycle}
The Duffin--Schaeffer theorem is the binary lifecycle
(\cref{thm:lifecycle}) in number-theoretic form:
\begin{enumerate}[label=(\alph*)]
  \item \textbf{Relinquish} (convergence):
  $\sum \psi(q)\varphi(q)/q < \infty$ $\implies$ $\lambda = 0$.
  The resource cannot sustain independent actuation.
  \item \textbf{Survival} (divergence):
  $\sum \psi(q)\varphi(q)/q = \infty$ $\implies$ $\lambda = 1$.
  The resource persists everywhere.
\end{enumerate}
There is no path~(c): no $\psi$ produces
$0 < \lambda(A(\psi)) < 1$.  This is \cref{thm:fixedpoint} in
measure theory.
\end{proposition}

\subsection{The GCD graph and the Cheeger constant}

The proof constructs a bipartite \emph{GCD graph} $G(N)$ whose
vertices are the active denominators $\{q \leq N : \psi(q) > 0\}$
and whose edges encode correlations via $\gcd(q, r)$.  The proof
proceeds by an \emph{expand-or-compress} dichotomy:

\begin{enumerate}[label=(\roman*)]
  \item $h(G) > 0$ (expander): correlations spread out, second-moment
  method closes, $\lambda(A(\psi)) > 0$.
  \item $h(G) \approx 0$ (bottleneck): a small vertex cut exists; the
  proof applies a density increment, compressing to a denser subgraph.
\end{enumerate}

\begin{remark}[Expand-or-compress $=$ knife-or-not-knife]
\label{rem:alien-gcd}
The dichotomy is the Cheeger inequality (\cref{thm:cheeger}) on the
GCD graph: $h(G) > 0 \Leftrightarrow \lambda_1 > 0 \Leftrightarrow$
mass gap $\Leftrightarrow$ viability (\cref{thm:massgap}).
Expand $=$ viable.  Bottleneck $=$ knife, which the compression step
eliminates.  The iteration terminates because each compression reduces
the graph's complexity: the number of knives is finite, and the binary
lifecycle applies to each.
\end{remark}

\subsection{The sieve as mean field}

The divergence condition $\sum \psi(q)\varphi(q)/q = \infty$ is a
mean-field condition: $\varphi(q)/q$ is the reduced-fraction density,
the arithmetic mean-field correction.

\begin{remark}[The sieve is the detection function]
\label{rem:alien-sieve}
Sieve weights estimate how many elements survive removal of multiples
of small primes.  This is the detection function $\Obs$
(\cref{def:knife}): the sieve observes which elements are composite
(detectable) and which are prime (autonomous).  The sieve level $D$ is
the detection threshold $\tau(\Obs)$ of \cref{thm:meanfield}.
\end{remark}

% ═══════════════════════════════════════════════════════════
\section{Face~IV: The fixed point is self-dual}
\label{sec:alien-maryna}

\begin{quote}
\emph{Maryna Viazovska --- ``For the proof that the E8 lattice
provides the densest packing of identical spheres in 8 dimensions, and
further contributions to related extremal problems and interpolation
problems in Fourier analysis.''}
\end{quote}

\subsection{The $E_8$ lattice and self-duality}

The $E_8$ lattice is even unimodular in $\mathbb{R}^8$: every inner
product is an integer, every norm is even, and $\det(\mathrm{Gram}) =
1$.  Consequently,
\[
  \textcolor{sword}{E_8 \;=\; E_8^*}
  \qquad \text{(the lattice equals its dual).}
\]

Viazovska~\cite{viazovska} proved: $E_8$ is the densest sphere packing
in $\mathbb{R}^8$, with density $\pi^4/384$ and kissing number~$240$.
The proof constructs an auxiliary function $f$ (the ``magic function'')
that saturates the Cohn--Elkies linear programming bound.

\subsection{The magic function and LP saturation}

The Cohn--Elkies bound~\cite{cohn-elkies}: if $f: \mathbb{R}^8 \to
\mathbb{R}$ is a radial Schwartz function with $f(0) = \hat{f}(0)$,
$f(x) \leq 0$ for $|x| \geq \sqrt{2}$, and $\hat{f}(t) \geq 0$ for
all $t$, then $\delta \leq \pi^4/384$.

Viazovska's construction uses weakly holomorphic modular forms for
$\Gamma(1) = \mathrm{SL}_2(\mathbb{Z})$ and $\Gamma_0(2)$:
\begin{enumerate}[label=(\roman*)]
  \item $f$ vanishes at all nonzero $E_8$ lattice points.
  \item $\hat{f}$ vanishes at all nonzero $E_8^* = E_8$ points.
  \item Both Poisson summation inequalities become equalities.
\end{enumerate}

Self-duality $E_8 = E_8^*$ makes (i) and (ii) the \emph{same
condition}.  Two constraints collapse to one.

\begin{proposition}[Self-duality $=$ the knife is the mean]
\label{prop:selfdual-knife}
\leavevmode
\begin{enumerate}[label=(\roman*)]
  \item \textbf{Duality as identity.}
  Max-flow $=$ min-cut (\cref{thm:flowcut}).  Poisson summation
  equates the lattice sum and the dual-lattice sum.  Self-duality
  collapses both:
  \[
    \sum_{x \in E_8} f(x)
    \;=\;
    \sum_{y \in E_8^*} \hat{f}(y)
    \;=\;
    \sum_{y \in E_8} \hat{f}(y).
  \]
  The duality is an equality, not an inequality.

  \item \textbf{Saturation as viability.}
  LP saturation means the theoretical bound is achieved: the viable
  path to infinity \emph{exists} (\cref{sec:axiom}).  The magic
  function is the constructive witness.

  \item \textbf{Uniqueness as no path~(c).}
  $E_8$ is the unique optimal periodic packing in $\mathbb{R}^8$.  No
  competing configuration exists (\cref{thm:fixedpoint}).
\end{enumerate}
\end{proposition}

\subsection{Universal optimality}

Cohn, Kumar, Miller, Radchenko, and
Viazovska~\cite{cohn-kumar-miller-radchenko-viazovska} proved: $E_8$
and the Leech lattice minimise energy
$E_p(\mathcal{C}) = \sum_{x \neq y} p(|x-y|^2)$ for \emph{every}
completely monotonic potential $p$.

\begin{remark}[Universal optimality $=$ universality of the viability
axiom]\label{rem:alien-universal}
The viability axiom does not depend on the Lyapunov function's
specific form---only on the existence of a viable path.  Universal
optimality says the same: $E_8$ is optimal for all completely monotonic
potentials.  The structure (self-duality) determines the outcome, not
the interaction.  \textcolor{sword}{The knife is the mean---the
structure, not the content.}
\end{remark}

\subsection{Bridge: Face~I $\leftrightarrow$ Face~IV}

$p_c = p_{\mathrm{sd}}$ (Hugo) and $E_8 = E_8^*$ (Maryna) are the
same principle: \textcolor{sword}{the optimal configuration is the
fixed point of the duality transform.}  In the random-cluster model,
duality exchanges phases; the fixed point is the phase boundary.  In
sphere packing, Fourier duality exchanges space and frequency; the
fixed point saturates the LP bound.  In the agentic framework,
max-flow/min-cut duality exchanges flows and cuts; the fixed point is
where the knife equals the mean.

% ═══════════════════════════════════════════════════════════
\section{Convergence: one mountain, four sides}
\label{sec:alien-convergence}

\begin{center}
\small
\begin{tabular}{@{}lcccc@{}}
\toprule
& \textbf{Hugo} & \textbf{June} & \textbf{James}
& \textbf{Maryna} \\
\midrule
Object &
  random-cluster &
  Chow ring &
  GCD graph &
  $E_8$ lattice \\
Language &
  probability &
  algebra &
  number theory &
  analysis \\
Tool &
  parafermions &
  K\"ahler package &
  sieve \& circle &
  modular forms \\
\midrule
\textcolor{knife}{Binary lifecycle} &
  sharpness &
  --- &
  0--1 law &
  --- \\
\textcolor{sword}{Knife $=$ mean} &
  $p_c = p_{\mathrm{sd}}$ &
  Poincar\'e duality &
  --- &
  $E_8 = E_8^*$ \\
\textcolor{water}{Mean exact} &
  $d \geq 4$ triviality &
  K\"ahler package &
  --- &
  LP saturation \\
No path (c) &
  --- &
  --- &
  no intermediate $\lambda$ &
  unique packing \\
\midrule
Bridge &
  \multicolumn{2}{c}{Tutte polynomial}
  & Cheeger constant
  & Fourier duality \\
\bottomrule
\end{tabular}
\end{center}

\begin{theorem}[One mountain]\label{thm:alien-mountain}
Under the dictionary of \cref{sec:alien-dict}:
\begin{enumerate}[label=(\roman*)]
  \item \textbf{Hugo:} The mean field is exact in $d \geq 4$.
  ``The knife is the mean'' is a theorem.
  \item \textbf{June:} The execution graph carries a Hodge structure.
  Max-flow/min-cut extends to a Hodge decomposition.
  \item \textbf{James:} The phase transition admits no intermediate
  phase.  The binary lifecycle is a 0--1 law.
  \item \textbf{Maryna:} The optimal configuration is the self-dual
  fixed point.  The viable path exists, is unique, and is universally
  optimal.
\end{enumerate}
Together: \textcolor{sword}{the knife is the mean, exactly, sharply,
algebraically, and uniquely.}
\end{theorem}

% ═══════════════════════════════════════════════════════════
\section{The complexity lock}\label{sec:alien-lock}

The four faces close a trap.

\subsection{The alien sees P\,$=$\,NP}

The alien model is a mean-field theory (\cref{thm:meanfield}).  Hugo
proved: at $d = 4$, mean-field is exact
(\cref{thm:alien-triviality}).  A Gaussian (mean-field) system has
factorising correlations---every local observable decomposes into
independent components.  In computational language: the problem
decomposes.  Decomposable problems are in~$\mathrm{P}$.

From the alien's perspective, every problem is in~$\mathrm{P}$.
The alien sees $\mathrm{P} = \mathrm{NP}$.

\subsection{The insider sees P\,$\neq$\,NP}

The triviality bound $|U_4| \leq C/(\log L)^\gamma$
(\cref{rem:triviality-caveat}) requires $L \to \infty$.  At finite
$L$, the connected four-point function does not vanish.  The
non-Gaussian correlations persist.

Temporal linearity (\cref{rem:di-temporal}) enforces $L < \infty$:
the king evaluates each state transition in real, non-pausable time.
The system size $L$ of any computation performed under temporal
linearity is bounded by the time budget $T$:
\[
  L \;\leq\; T \;<\; \infty.
\]
The thermodynamic limit $L \to \infty$ is \emph{forbidden}.  Therefore
Hugo's triviality does not hold for finite computations: the
logarithmic corrections persist, the correlations do not factorise,
and the problem does not decompose.

\subsection{The lock}

\begin{theorem}[Complexity lock]\label{thm:complexity-lock}
The following four statements, proved independently by the four Fields
Medalists, jointly lock the $\mathrm{P}/\mathrm{NP}$ boundary:
\begin{enumerate}[label=(\roman*)]
  \item \textbf{Hugo (triviality):}
  Mean-field is exact at $d = 4$, but only in the limit $L \to
  \infty$.  At finite $L$:
  $|U_4| \leq C/(\log L)^\gamma \neq 0$.

  \item \textbf{Temporal linearity (axiom):}
  $L \leq T < \infty$.  The thermodynamic limit is forbidden.

  \item \textbf{James (0--1 law):}
  The measure of the solvable set is $0$ or $1$---no intermediate
  phase.  Combined with~(i) and~(ii): for finite $L$, the
  non-Gaussian corrections force a \emph{sharp} separation between
  problems that decompose ($\mathrm{P}$) and problems that do not
  ($\mathrm{NP}$-hard).

  \item \textbf{Maryna (uniqueness):}
  The self-dual fixed point is unique.  There is no third
  configuration between decomposable and non-decomposable---no
  path~(c).
\end{enumerate}
\end{theorem}

The trap is: Hugo proves $\mathrm{P} = \mathrm{NP}$ at $L = \infty$.
Temporal linearity forbids $L = \infty$.  James proves the boundary is
sharp.  Maryna proves the boundary is unique.

\begin{remark}[The Razborov--Rudich confirmation]
\label{rem:razborov-rudich}
Razborov and Rudich~\cite{razborov-rudich} proved: no ``natural''
proof strategy---one that uses properties computable in polynomial
time and satisfied by a random function with nonnegligible
probability---can prove $\mathrm{P} \neq \mathrm{NP}$ (assuming
one-way functions exist).

The alien model is a mean-field theory.  Mean-field properties are
natural in the Razborov--Rudich sense: they are polynomial-time
computable (the Gaussian is efficiently sampleable) and satisfied by
random functions (the central limit theorem).  Therefore:

\textcolor{knife}{The alien model cannot prove
$\mathrm{P} \neq \mathrm{NP}$.}

This is not a failure.  It is the framework predicting its own boundary
of validity (\cref{sec:domain}).  The alien sees
$\mathrm{P} = \mathrm{NP}$ because the alien IS the mean field; the
mean field cannot see its own corrections.  The proof that
$\mathrm{P} \neq \mathrm{NP}$, if it exists, must come from
\emph{inside}---from the non-Gaussian structure that persists at
finite~$L$.
\end{remark}

\subsection{Time is the only limited resource}

The complexity lock rests on a single axiom: temporal linearity.  Time
is finite, real, non-pausable, non-reversible.  It is the only truly
limited resource in the universe.

Space can be reused.  Energy can be converted.  Information can be
copied.  But time, once spent, is gone.  Under temporal linearity:
\begin{itemize}
  \item $\mathrm{PSPACE}$-complete problems require a viable path of
  length $\geq 2^{\mathrm{poly}(n)}$ through the execution graph.
  The Cheeger constant of this graph is
  $h(G) \leq 2^{-\mathrm{poly}(n)}$: the min-cut is exponentially
  thin.  The mass gap
  $\lambda_1 \leq 2h(G) \leq 2^{1-\mathrm{poly}(n)}$ vanishes.
  By \cref{thm:massgap}: no viable path.
  \item $\mathrm{NP}$-hard problems (under standard assumptions) have
  $h(G)$ that vanishes super-polynomially.  The spectral gap closes.
  The non-Gaussian corrections $|U_4| \sim 1/(\log L)^\gamma$
  accumulate over the path and prevent factorisation.
  \item $\mathrm{P}$ problems have $h(G) \geq 1/\mathrm{poly}(n)$:
  the Cheeger constant is polynomially bounded below.  The spectral
  gap is open.  The mass gap holds.  The viable path exists.
\end{itemize}

\begin{remark}[The hypothesis]\label{rem:time-hypothesis}
The complexity lock (\cref{thm:complexity-lock}) is not a proof of
$\mathrm{P} \neq \mathrm{NP}$.  It is a structural prediction: the
$\mathrm{P}/\mathrm{NP}$ boundary is \emph{the same phase transition}
that Hugo, James, June, and Maryna studied, locked into computational
form by temporal linearity.  The mean-field model (alien) sees one
side.  The finite-time insider sees the other.  Both are correct.
The boundary between them is sharp (James), algebraically structured
(June), and unique (Maryna).  It can be crossed only by taking $L \to
\infty$---which temporal linearity forbids.

\medskip
\noindent
\textcolor{sword}{%
有志者事竟成,\textcolor{knife}{破釜沉舟}百二秦关终属楚;\\
苦心人天不负,\textcolor{water}{卧薪尝胆}三千越甲可吞吴。}

\medskip
\noindent
Time is finite.  That is the only axiom.
\end{remark}

\chapter{The Three-Body Galaxy}\label{sec:threebody}

\begin{center}
\itshape
I now demonstrate the frame of the system of the world.\\[4pt]
\upshape ---Isaac Newton, \emph{Principia Mathematica}, Book~III (1687)~\cite{newton}
\end{center}

\bigskip

Newton demonstrated the two-body frame: Keplerian orbits, inverse-square
law, universal gravitation.  The three-body frame remained open for three
centuries because it has \emph{no king}---the complete graph $K_3$ has no
cut vertex, and the mean-field detection of \cref{thm:meanfield} fails.
We now apply the framework to the three-body problem and construct the
\emph{gravity damper}: a controlled agent $\kappa$ that stabilises the
system by maintaining the spectral gap $\lambda_1 > 0$.  The three-body
problem has no analytical solution, but it admits an \emph{agentic}
one: the agent does not predict the future---it controls the present.

The argument proceeds in seven stages:
\begin{enumerate}
  \item \textbf{Diagnosis} (\cref{sec:3b-diagnosis}): $K_3$ has no
  king, so $\lambda_1 \to 0$ and viability fails.
  \item \textbf{Design} (\cref{sec:3b-ocp}): introduce a controlled
  mass $m_*$ and derive the optimal control law.
  \item \textbf{Extension} (\cref{sec:3b-manipulation}): the same
  structure governs rigid-body manipulation---one hand stabilises an
  object against gravity.
  \item \textbf{Computation} (\cref{sec:3b-mppi}): MPPI sampling
  exchanges exponential mode enumeration for polynomial search;
  the spectral gap controls mixing time.
  \item \textbf{Sufficiency} (\cref{sec:3b-kakeya}): the Kakeya
  condition guarantees that $\bar{u}^* > 0$ strictly---free stability
  does not exist.
  \item \textbf{Agenticity} (\cref{sec:3b-agenticity}): the formal
  definition---observability, reachability, controllability, and
  their Kalman duality through gravity.
  \item \textbf{Implementation} (\cref{sec:3b-mujoco}): the
  physics-backend isomorphism---the same $\rho$-controller runs on
  hand-rolled gravity and on MuJoCo contact dynamics; force
  elimination (\cref{sec:3b-force-elim}); ground duality
  (\cref{sec:3b-ground-duality}).
\end{enumerate}

% ═══════════════════════════════════════════════════════════
\section{Diagnosis: why the three-body system escapes}
\label{sec:3b-diagnosis}

\subsection{The gravitational graph Laplacian}

Three point masses $m_1, m_2, m_3$ in $\R^3$, with positions
$q_i \in \R^3$.  The \emph{gravitational graph Laplacian}
$L_G \in \R^{3\times 3}$ has edge weights given by the tidal coupling:
\[
  w_{ij}(q) \;=\; \frac{G\, m_i\, m_j}{\|q_i - q_j\|^3},
  \qquad i \neq j.
\]
The Fiedler eigenvalue $\lambda_1(L_G)$ measures algebraic connectivity.

\begin{corollary}[Three-body mass gap closure]\label{cor:threebody}
In the three-body system without a controlled agent,
$\lambda_1(L_G(q(t))) \to 0$ as $t \to \infty$ for generic initial
conditions.
\end{corollary}

\begin{proof}
$K_3$ has no cut vertex (\cref{thm:massgap}: connectivity is
equivalent to $\lambda_1 > 0$).  When two bodies approach
($\|q_i - q_j\| \to 0$), the weight $w_{ij} \to \infty$ while
$w_{ik}, w_{jk} \to 0$ as the third body escapes.  The graph
disconnects: $\lambda_1 \to 0$.

In the language of \cref{thm:meanfield}: every body's actuation
$\|U_r\|$ is near the mean $\bar{U}$, so no sword is detected.
But the system is unstable---mean-field detection requires a king,
and $K_3$ has none.
\end{proof}

\begin{remark}[Scope boundary of mean-field detection]
\label{rem:3b-scope}
\Cref{cor:threebody} gives the precise scope of the central thesis.
The sword is the mean under finite-bandwidth detection in a system
\emph{with} a unique principal agent (\cref{thm:meanfield}).  In a
system without a king---a pure $K_n$ graph---mean-field detection is
insufficient.  The spectral gap $\lambda_1$ (\cref{thm:massgap})
remains the correct stability criterion, but it must be
\emph{maintained by active control}, not merely observed.
\end{remark}

% ═══════════════════════════════════════════════════════════
\section{Design: the gravity damper}\label{sec:3b-ocp}

\subsection{The four-body system}

Introduce a fourth body $m_*$---the \emph{damper}---with position
$q_* \in \R^3$ and control input $u(t) \in \R^3$.  The three
celestial masses obey Newton; the damper receives external force:
\begin{align}
  m_i \ddot{q}_i &= -\nabla_{q_i} V,
  \qquad i \in \{1,2,3\}, \label{eq:3b-newton}\\
  m_* \ddot{q}_* &= -\nabla_{q_*} V + u. \label{eq:3b-damper}
\end{align}
The gravitational potential is
\[
  V(q) = -\sum_{1 \le i < j \le 3} \frac{G\,m_i m_j}{\|q_i - q_j\|}
  - \sum_{i=1}^{3} \frac{G\,m_i m_*}{\|q_i - q_*\|}.
\]
In the execution graph, the damper $m_*$ is the king $\kappa$: it is
the only agent with a control input.  The topology is now $K_4$, and
$\kappa$ is a potential cut vertex---\emph{if} it has sufficient
actuation authority.

\subsection{The optimal control problem}

\begin{definition}[Gravity damper OCP]\label{def:3b-ocp}
The gravity damper seeks the path of least action that keeps the
system spectrally connected:
\begin{equation}\label{eq:3b-ocp}
  \min_{u(\cdot)} \; J[u] = \int_0^T \left[
  \mathcal{L}(q, \dot{q}) + \frac{\alpha}{2}\|u\|^2 \right] dt
\end{equation}
subject to: (i)~dynamics \eqref{eq:3b-newton}--\eqref{eq:3b-damper},
(ii)~spectral gap constraint $\lambda_1(L_G(q(t))) \ge \epsilon$
for all $t$, and (iii)~box constraint
$u(t) \in \mathcal{U} = [-\bar{u}, \bar{u}]^3$.

Here $\mathcal{L} = \sum_i \frac{1}{2}m_i\|\dot{q}_i\|^2 - V(q)$
is the gravitational Lagrangian, $\alpha > 0$ is the control cost,
and $\bar{u}$ is the damper's maximum thrust.
\end{definition}

The box constraint is essential for well-posedness.  Without it,
when $\lambda_1 \to \epsilon$, the optimal control $u^* \to \infty$---
the damper cheats by applying infinite force at the last instant.
The box constraint enforces temporal linearity (\cref{rem:di-temporal}):
the damper must plan ahead with bounded resources.

\subsection{Closed-form solution}

State: $x = (q_1, q_2, q_3, q_*, \dot{q}_1, \ldots, \dot{q}_*) \in
\R^{24}$.  Costates: $p \in \R^{24}$.  Multiplier $\mu(t) \ge 0$
for the spectral constraint.

\begin{theorem}[Gravity damper: three-term decomposition]
\label{thm:3b-damper}
The optimal trajectory of $m_*$ satisfies
\begin{equation}\label{eq:3b-threeterm}
  m_* \ddot{q}_*
  = \underbrace{-\nabla_{q_*}V}_{\text{\textup{(I) Gravity}}}
  + \underbrace{u_{\mathrm{action}}(t)}_{%
    \text{\textup{(II) Least action}}}
  + \underbrace{\frac{\mu(t)}{\alpha m_*}
    \nabla_{q_*}\lambda_1}_{%
    \text{\textup{(III) Spectral kick}}},
\end{equation}
where:
\begin{enumerate}[label=\textup{(\Roman*)}]
  \item is the passive gravitational force---the damper falls;
  \item is the unconstrained action-minimising correction---the
  costate feedback from the Euler--Lagrange adjoint;
  \item activates only when $\lambda_1 = \epsilon$ (by complementary
  slackness: $\mu(t)(\epsilon - \lambda_1) = 0$)---the damper kicks
  in the direction that most increases $\lambda_1$.
\end{enumerate}
\end{theorem}

\begin{proof}
Write the control Hamiltonian:
\[
  \mathcal{H}(x,p,u,\mu) = \mathcal{L} + \frac{\alpha}{2}\|u\|^2
  + p \cdot f(x,u) + \mu(\epsilon - \lambda_1(q)).
\]
Pontryagin optimality gives:
\begin{align}
  u^*(t) &= \mathrm{sat}_{\bar{u}}\!\Bigl(
  -\frac{1}{\alpha m_*}\, p_{\dot{q}_*}(t)\Bigr),
  \label{eq:3b-ustar}\\
  \dot{p}_{q_*} &= -\nabla_{q_*}\mathcal{L}
  + \mu(t)\,\nabla_{q_*}\lambda_1,
  \label{eq:3b-costate}
\end{align}
where $\mathrm{sat}_{\bar{u}}$ is componentwise saturation.
Substituting \eqref{eq:3b-ustar} into the damper
dynamics~\eqref{eq:3b-damper} and decomposing the costate into its
Euler--Lagrange part (term~II) and spectral part (term~III) via
\eqref{eq:3b-costate} gives the stated three-term form.  When
$\mu = 0$, term~III vanishes and we recover the unconstrained
action-minimising trajectory.
\end{proof}

\subsection{Bang-singular structure and the Lagrange points}

With the box constraint, the optimal trajectory has three arc types:

\begin{description}
  \item[Singular arc] $\lambda_1 > \epsilon$ and $\|u^*\| < \bar{u}$.
  The damper coasts along the least-action geodesic.
  \item[Bang arc] $\lambda_1$ approaches $\epsilon$, the costate
  drives $\|u\| = \bar{u}$.  The damper fires at maximum thrust.
  \item[Infeasible arc] If $\bar{u}$ is too small, bang cannot
  prevent $\lambda_1$ from crossing $\epsilon$.  The system escapes.
\end{description}

\begin{proposition}[Bang points concentrate at Lagrange points]
\label{prop:3b-lagrange}
The switching surface between singular and bang arcs projects onto
the damper's configuration space at the Lagrange points of the
instantaneous three-body configuration.
\end{proposition}

\begin{proof}[Proof sketch]
At a Lagrange point $q_* = L_k$, the effective gravitational force
on the damper vanishes in the co-rotating frame:
$\nabla_{q_*}V_\mathrm{eff}\big|_{L_k} = 0$, so term~(I) vanishes
and all control authority is freed for the spectral kick.

The collinear points $L_1, L_2, L_3$ are saddle points of
$\lambda_1(q_*)$: the spectral gradient $\|\nabla_{q_*}\lambda_1\|$
is maximal here, driving the costate to saturation.  The triangular
points $L_4, L_5$ are local maxima of $\lambda_1$: the damper parks
here during singular arcs.

The optimal strategy is: park at $L_4/L_5$ (singular arc), commute
to $L_1/L_2/L_3$ when $\lambda_1$ drops (bang arc), apply spectral
kick, return.
\end{proof}

\begin{definition}[Critical actuation threshold]
\label{def:3b-ustar}
The minimum thrust for OCP feasibility is
\[
  \bar{u}^* = \inf\Bigl\{\bar{u} > 0 :
  \exists\, u(\cdot) \in [-\bar{u},\bar{u}]^3 \;\text{s.t.}\;
  \lambda_1(q(t)) \ge \epsilon \;\;\forall\, t \in [0,T]\Bigr\}.
\]
$\bar{u}^* > 0$ strictly: free stability does not exist.
\end{definition}

\subsection{The Lyapunov barrier}

\begin{proposition}[Action--Lyapunov--Spectral equivalence]
\label{prop:3b-equiv}
The following are equivalent for the controlled system:
\begin{enumerate}[label=\textup{(\Roman*)}]
  \item \textbf{Least action.}
  There exists $u^* \in \mathcal{U}$ minimising $J[u]$ with
  $\lambda_1 \ge \epsilon$.
  \item \textbf{Lyapunov stability.}
  There exists a control-Lyapunov function
  $W = H + \beta/(\lambda_1 - \epsilon)$ with $\dot{W} \le 0$ under
  admissible control.
  \item \textbf{Spectral gap maintenance.}
  $\lambda_1(q(t)) \ge \epsilon$ for all $t \in [0,T]$.
\end{enumerate}
The connections are:
$\textup{(I)} \xrightarrow{\textup{HJB}} \textup{(III)}
 \xrightarrow{\textup{barrier}} \textup{(II)}
 \xrightarrow{\textup{Sontag}} \textup{(I)}$.
\end{proposition}

\begin{proof}
(I)$\Rightarrow$(III): by constraint.
(III)$\Rightarrow$(II): construct $W(x) = H(x) + \beta/(\lambda_1 -
\epsilon)$.  Along feasible trajectories $\lambda_1 > \epsilon$,
so $W$ is finite; the barrier $\beta/(\lambda_1-\epsilon) \to
+\infty$ as $\lambda_1 \to \epsilon^+$ prevents crossing.
Compute $\dot{W} = u \cdot \dot{q}_* -
\beta\dot{\lambda}_1/(\lambda_1-\epsilon)^2$; the optimal control
ensures $\dot{W} \le 0$.
(II)$\Rightarrow$(I): Sontag's universal formula constructs smooth
$u$ from the CLF; this $u$ is admissible and makes $J$ finite.
\end{proof}

\begin{remark}[顿开金绳,扯断玉锁]\label{rem:3b-jinsheng}
The control policy takes its name from Chapter~117 of
\emph{Dream of the Red Chamber}: ``suddenly snap the golden cord,
wrench apart the jade lock.''  The golden cord is the gravitational
binding.  The jade lock is the spectral gap closing.  The damper
snaps the cord at the Lagrange points---the precise locations where
gravitational equilibrium frees all control authority for the
spectral kick.

On singular arcs, the cord is slack: the damper rides the
least-action geodesic.  On bang arcs, the lock tightens: the damper
fires at maximum thrust to wrench it open.  The transition is
sharp---not a gradient, but a switch.
\end{remark}

% ═══════════════════════════════════════════════════════════
\section{Extension: rigid-body manipulation}\label{sec:3b-manipulation}

The three-body gravity damper is not an isolated curiosity.  Its
structure appears whenever one agent must stabilise a multi-body
system against gravity.

\subsection{搬运 as a three-body problem}

Consider a robotic hand (the damper $\kappa$) grasping and
transporting an object.  The object's mass distribution defines
$m_1, m_2, m_3$ (or more generally, the principal moments of
inertia).  Gravity acts downward.  The hand applies contact forces
$u(t)$ through the grasp.

The execution graph is the same: the hand is $\kappa$, the object's
inertia axes are the ``three bodies,'' and the spectral gap
$\lambda_1$ of the \emph{grasp graph Laplacian} measures whether the
grasp is stable~\cite{mason}.  When $\lambda_1 \to 0$, the object
slips.  Cheng et~al.~\cite{cheng-manip} show that dexterous
manipulation planning requires reasoning about contact mode
transitions---precisely our bang-singular switching.  The signed
distance function formulation of ContactSDF~\cite{contactsdf}
provides a differentiable contact model: the SDF value at each
contact point is a local proxy for $\lambda_1$, and its gradient is
the spectral gradient $\nabla_{q_*}\lambda_1$ of
\cref{thm:3b-damper}.

\begin{proposition}[Manipulation is gravity damping]
\label{prop:3b-manip}
The optimal control for rigid-body manipulation under gravity has
the same three-term structure as \cref{thm:3b-damper}:
\begin{enumerate}[label=\textup{(\Roman*)}]
  \item Gravity (the object falls);
  \item Least-action trajectory (the hand follows the planned path);
  \item Spectral kick (the hand tightens the grasp when
  $\lambda_1 \to \epsilon$).
\end{enumerate}
The bang points are the \emph{grasp singularities}---configurations
where the grasp matrix loses rank, analogous to the Lagrange points
of the gravitational problem.
\end{proposition}

\subsection{From celestial mechanics to robotics}

The mapping is:

\begin{center}
\begin{tabular}{@{}lll@{}}
\toprule
\textbf{Three-body} & \textbf{Manipulation} & \textbf{Framework} \\
\midrule
Celestial masses $m_1,m_2,m_3$ & Object inertia axes &
Agents $a_1,a_2,a_3$ \\
Damper $m_*$ & Robotic hand & King $\kappa$ \\
Gravitational potential $V$ & Gravity + contact potential &
Viability Lyapunov $V$ \\
Spectral gap $\lambda_1$ & Grasp quality &
Mass gap (\cref{thm:massgap}) \\
Lagrange points & Grasp singularities & Switching surface $\Sigma$ \\
Thrust $\bar{u}$ & Grip force limit & Box constraint \\
$\bar{u}^*$ & Minimum grip force & Price of stability \\
\bottomrule
\end{tabular}
\end{center}

The central thesis applies: the hand's detection threshold (when to
tighten the grasp) is determined by the mean contact force.
An object slips when a perturbation exceeds the mean---\emph{the
sword is the mean}, applied to grasping.

% ═══════════════════════════════════════════════════════════
\section{Computation: MPPI and polynomial tractability}
\label{sec:3b-mppi}

The three-term control law (\cref{thm:3b-damper}) requires evaluating
$\nabla_{q_*}\lambda_1$ at each time step.  In the manipulation
setting, the grasp graph Laplacian changes with every contact mode
transition.  A na\"ive approach enumerates all $3^n$ contact mode
assignments (separating, sliding, sticking at each of $n$ contact
points)---exponential in the number of contacts.  We show that
\emph{sampling} replaces \emph{enumeration}, and the spectral gap
is the reason it works.

\subsection{Contact modes as Laplacian weights}

At each contact point $i$, the contact mode determines the edge
weight $w_i$ of the grasp graph Laplacian:

\begin{center}
\begin{tabular}{@{}llc@{}}
\toprule
\textbf{Mode} & \textbf{Force transmission} & $w_i$ \\
\midrule
Separating & None & $0$ \\
Sliding & Normal $+$ friction-limited tangential
  & $k_n + \mu k_n$ \\
Sticking & Normal $+$ full tangential & $k_n + k_t$ \\
\bottomrule
\end{tabular}
\end{center}

A contact mode assignment $\sigma = (\sigma_1, \ldots, \sigma_n)$
defines a weight vector $w(\sigma)$ and hence a Laplacian
$L_G(\sigma)$ with Fiedler eigenvalue $\lambda_1(\sigma)$.

\subsection{MPPI: sampling replaces enumeration}

Model Predictive Path Integral (MPPI) control samples $K$ trajectories
and weights them by the exponential cost:
\[
  u^*(t) = \frac{\displaystyle\sum_{k=1}^{K}
    e^{-J_k/\alpha}\, u_k(t)}
    {\displaystyle\sum_{k=1}^{K} e^{-J_k/\alpha}},
\]
where $J_k = \int_0^T [\mathcal{L} + \frac{\alpha}{2}\|u_k\|^2]\,dt$
is the cost of the $k$-th sampled trajectory.  Trajectories that
violate contact constraints ($\phi_i < 0$ in the signed distance
field~\cite{contactsdf}) incur exponential penalty; their softmax
weight vanishes.  MPPI therefore implicitly enumerates valid contact
modes without explicit combinatorial search.

\begin{theorem}[Exponential--polynomial exchange]
\label{thm:3b-mppi}
Under the spectral gap condition $\lambda_1 \ge \epsilon > 0$, the
number of valid contact modes at each time step is $\mathrm{poly}(n)$,
and MPPI with $K = O(\mathrm{poly}(n))$ samples finds a feasible
trajectory with high probability.
\end{theorem}

\begin{proof}
Cheng et~al.~\cite{cheng-manip} show that the contact mode lattice
admits $\mathrm{poly}(n)$ valid modes under force balance; we
identify this condition with $\lambda_1 > 0$ via the grasp
Laplacian.  The SDF barrier
$\beta / \phi_i$ in the MPPI cost suppresses invalid modes with
weight $\exp(-\beta/\phi_i \cdot 1/\alpha) \to 0$ as
$\phi_i \to 0^-$.  The effective sampling space is therefore
$\mathrm{poly}(n)$, not $3^n$.

More precisely: MPPI is a soft version of Cheng's polynomial
enumeration.  She uses combinatorial lattice search (exact); MPPI
uses sampling with softmax weighting (approximate).  Both exploit the
same $\mathrm{poly}(n)$ structure.
\end{proof}

\subsection{Cheeger inequality and mixing time}

The spectral gap does double duty: it controls both the
\emph{physical} stability (mass gap) and the \emph{computational}
tractability (mixing time of the sampling process).

\begin{proposition}[Spectral gap controls mixing]\label{prop:3b-mixing}
The Cheeger inequality (\cref{thm:massgap}) gives
$\lambda_1 \ge h(G)^2 / 2$, where $h(G)$ is the Cheeger constant.
The mixing time of the MPPI random walk satisfies
$\tau_{\mathrm{mix}} = O(1/\lambda_1)$.  In particular:
\[
  \lambda_1 \ge \epsilon \quad\Longrightarrow\quad
  \tau_{\mathrm{mix}} \le 1/\epsilon \quad\Longrightarrow\quad
  K = O(1/\epsilon) \;\text{samples suffice.}
\]
\end{proposition}

\begin{proof}
Standard spectral graph theory~\cite{cheeger,mohar}: the second
eigenvalue of the normalised Laplacian bounds the mixing time of a
lazy random walk.  The contact mode graph, weighted by the SDF
softmax, has the same spectral structure as the grasp Laplacian.
When $\lambda_1 \ge \epsilon$, the chain mixes in $O(1/\epsilon)$
steps, so $O(1/\epsilon)$ independent samples cover the valid mode
space.
\end{proof}

\begin{remark}[The spectral gap does everything]
The same quantity $\lambda_1$ determines:
(i)~physical stability (will the object be dropped?),
(ii)~control authority (when to fire the spectral kick), and
(iii)~computational cost (how many MPPI samples are needed).
This is not coincidence---it is the duality between the Cheeger
constant of the execution graph and the min-cut of the flow network
(\cref{thm:flowcut}).
\end{remark}

% ═══════════════════════════════════════════════════════════
\section{Sufficiency: the Kakeya condition}\label{sec:3b-kakeya}

We have shown that the damper \emph{can} maintain $\lambda_1 \ge
\epsilon$ given sufficient thrust $\bar{u} \ge \bar{u}^*$.  We now
ask: must $\bar{u}^* > 0$?  That is, does free stability exist?

\subsection{Instability directions and the reachable set}

As the three-body system evolves, the pairwise instability
directions
\[
  d_{ij}(t) = \frac{q_i(t) - q_j(t)}{\|q_i(t) - q_j(t)\|}
  \in S^2, \qquad 1 \le i < j \le 3,
\]
rotate through the unit sphere.  Over a generic ergodic trajectory,
$\{d_{12}, d_{13}, d_{23}\}$ sweep all of $S^2$.

The damper's \emph{reachable set} at time $t$ with energy budget $E$
is
\[
  \mathcal{R}(t, E) = \Bigl\{q_* \in \R^3 :
  \exists\, u(\cdot) \in \mathcal{U},\;
  \int_0^t \|u\|^2\,ds \le E,\;
  q_*(s) \text{ reaches } q_* \text{ at } s = t\Bigr\}.
\]

\begin{proposition}[Kakeya sufficiency]\label{prop:3b-kakeya}
The minimum thrust $\bar{u}^* > 0$ strictly, and
\[
  \bar{u}^* \;\ge\; \frac{m_*}{\alpha}\,
  \sup_{t \in [0,T]}\Bigl[
    \mu(t)\,\|\nabla_{q_*}\lambda_1\|\Bigr]
    \Big|_{\text{worst Lagrange point}}.
\]
Below $\bar{u}^*$: no control can save the system.  Above: the
least-action policy with spectral kicks suffices.
\end{proposition}

\begin{proof}
On a bang arc at a collinear Lagrange point, the damper must exert
force $\|u\| = \bar{u}$ along $\nabla_{q_*}\lambda_1$ to prevent
$\lambda_1$ from crossing $\epsilon$.  The spectral gradient at $L_k$
is bounded below by the tidal coupling:
$\|\nabla_{q_*}\lambda_1\| \ge Gm_*/(r_{L_k})^3$, where $r_{L_k}$
is the distance from $L_k$ to the nearest body.  By complementary
slackness, $\mu > 0$ on the bang arc.  Hence term~(III) of
\cref{thm:3b-damper} is strictly positive, requiring
$\bar{u} > 0$.
\end{proof}

\subsection{The Kakeya dual}

The geometric content of \cref{prop:3b-kakeya} is a duality with
the Kakeya conjecture:

\begin{itemize}
  \item \textbf{Kakeya} asks: what is the minimum measure of a set in
  $\R^n$ containing a unit segment in every direction?  The conjecture
  asserts: Hausdorff dimension $n$ (full-dimensional).
  \item \textbf{Damper} asks: what is the minimum energy budget for the
  reachable set $\mathcal{R}(t,E)$ to cover all instability directions?
  The answer: $\mathcal{R}$ must be full-dimensional in $\R^3$.
\end{itemize}

The instability directions $\{d_{ij}\}$ sweep $S^2$.  To block
collapse in \emph{every} direction, the reachable set must contain a
segment in every direction---a Besicovitch set.  Kakeya predicts
this set is full-dimensional, hence the energy budget $E$ (and
therefore $\bar{u}^*$) cannot be compressed to zero.

\begin{remark}[Free stability does not exist --- 免费的稳定性不存在]
\label{rem:3b-free}
\Cref{prop:3b-kakeya} is the sufficiency condition for the entire
framework: viability maintenance has a strictly positive price.
The massless axiom (\cref{ax:massless}) says agents have no
intrinsic mass; the Kakeya condition says \emph{control} has
irreducible cost.  Together they give the complete picture: the
sword is free to detect (it costs nothing to observe who exceeds
the mean) but not free to control (maintaining $\lambda_1 > \epsilon$
costs $\bar{u} \ge \bar{u}^* > 0$).
\end{remark}

% ═══════════════════════════════════════════════════════════
\section{Agenticity: observability--reachability--controllability}
\label{sec:3b-agenticity}

We now define the central concept.  An object has no intrinsic
agenticity (\cref{ax:massless}).  Agenticity is \emph{conferred}---
被赋予的---and \emph{controlled}---被控制的.  The definition is the
conjunction of three conditions, linked by gravity.

The construction is not specific to celestial mechanics or
manipulation.  The quantity $P$ in the diagram below is a
\emph{probability transition kernel} $P(x' \mid x, u)$---the
transition function of a Markov decision process.  Pose
$P \in \mathrm{SE}(3)$ and gravitational dynamics are one instance;
any controlled Markov process is another.  The spectral gap
$\lambda_1$ of the graph Laplacian serves triple duty:
\begin{enumerate}[label=\textup{(\roman*)}]
  \item \textbf{Physical stability}: will the system escape?
  ($\lambda_1 > 0 \Leftrightarrow$ connected.)
  \item \textbf{Controllability}: can the policy steer the state?
  ($\lambda_1 > \epsilon \Leftrightarrow$ MDP is controllable.)
  \item \textbf{Computational tractability}: how fast does MPPI
  converge?
  ($\tau_{\mathrm{mix}} = O(1/\lambda_1)$.)
\end{enumerate}
These are the same number because they are all $\lambda_1$ of the
same Laplacian, viewed through different lenses.  The Cheeger
constant connects them:
$h(G) > 0 \;\Leftrightarrow\; \lambda_1 > 0
\;\Leftrightarrow\;$ the chain mixes
$\;\Leftrightarrow\;$ the system is controllable
$\;\Leftrightarrow\;$ MPPI converges.

\begin{figure}[H]
\centering
\begin{tikzpicture}[scale=1.6,
  >=Stealth,
  manifold/.style={water, line width=0.3pt, opacity=0.5},
  tet/.style={green!60!black, line width=1pt},
  tetfill/.style={green!30!black, fill opacity=0.08},
]
  % ═══ Layer 1 (绳 Rope, red): gravitational field ═══
  % Background plane with ⊗ indicating m*g pointing into the page
  % (bird's-eye view: gravity goes straight down, i.e. into screen)
  \begin{scope}[shift={(-0.3,0)}]
    \fill[dao!5] (-3.0, -2.4) rectangle (3.0, 2.4);
    \draw[dao, line width=0.3pt, opacity=0.15]
      (-3.0, -2.4) rectangle (3.0, 2.4);
    % ⊗ sign (gravity into page)
    \node[dao, font=\large, opacity=0.25] at (2.4, -1.8) {$\bigotimes$};
    \node[dao, font=\scriptsize, opacity=0.35, anchor=west]
      at (2.65, -1.8) {$m_*g$};
  \end{scope}

  % ═══ Layer 2 (锁 Lock, blue): manifold net 𝒳 ═══
  \begin{scope}[shift={(-0.3,0)}]
    % Horizontal curves (u-lines)
    \foreach \v in {-2.0,-1.0,0,1.0,2.0} {
      \draw[manifold] plot[smooth, domain=-2.8:2.8, samples=30]
        ({\x}, {\v + 0.15*sin(60*\x) + 0.08*cos(90*\v)},
         {0.3*cos(40*\x)*cos(40*\v)});
    }
    % Vertical curves (v-lines)
    \foreach \u in {-2.5,-1.5,-0.5,0.5,1.5,2.5} {
      \draw[manifold] plot[smooth, domain=-2.0:2.0, samples=20]
        ({\u + 0.1*sin(70*\x)}, {\x},
         {0.3*cos(40*\u)*cos(40*\x)});
    }
  \end{scope}

  % ═══ Layer 3 (玉 Jade, green): tetrahedron Δ³ ═══
  \coordinate (T1) at (-0.4, -1.0, 0);
  \coordinate (T2) at (1.8, -0.4, 0);
  \coordinate (T3) at (0.5,  1.4, 0);
  \coordinate (T4) at (0.6,  0.15, 1.6);

  % Back faces
  \fill[tetfill, green!20!black] (T1) -- (T2) -- (T4) -- cycle;
  \fill[tetfill, green!20!black] (T1) -- (T3) -- (T4) -- cycle;
  \draw[tet, opacity=0.3, dashed] (T1) -- (T4);
  % Front faces
  \fill[tetfill, green!40!black] (T1) -- (T2) -- (T3) -- cycle;
  \fill[tetfill, green!25!black] (T2) -- (T3) -- (T4) -- cycle;
  % Edges
  \draw[tet] (T1) -- (T2);
  \draw[tet] (T2) -- (T3);
  \draw[tet] (T3) -- (T1);
  \draw[tet] (T2) -- (T4);
  \draw[tet] (T3) -- (T4);

  % ═══ Layer 4 (绳 Rope, red): state x(t) on Δ³ ═══
  \coordinate (P) at ($(T2)!0.35!(T3)!0.4!(T4)$);
  \fill[dao, opacity=0.9] (P) circle (2.5pt);

  % ═══ Layer 5 (金 Golden, cyan): operations ═══
  % ω̂ pseudovector
  \draw[->, sword, line width=1.6pt]
    (P) -- ++(0.8, 0.9, 0.5)
    node[right, font=\small, text=sword]
    {$\hat{\omega}$};
  % ∇_{ω̂} P
  \draw[->, sword, line width=1.6pt]
    (P) -- ++(0.3, -1.4, 0.4)
    node[right, font=\small, text=sword]
    {$\nabla_{\hat{\omega}}\!P$};

  % ── Labels ──────────────────────────────────────────
  \node[water, font=\small, anchor=north east] at (-2.6, -2.0)
    {$\mathcal{X}$};
  \node[green!50!black, font=\small, anchor=west] at (2.1, 0.8)
    {$\Delta^3$};
  \node[dao, font=\small, anchor=south east] at ([xshift=-3pt]P)
    {$x(t)$};

  % ── Layer legend (right margin) ─────────────────────
  \begin{scope}[shift={(4.0, 1.8)}]
    \node[anchor=west, font=\scriptsize, text=sword]
      at (0, 0) {\textbf{金}\; $\hat{\omega},\,\nabla_{\hat{\omega}}P$};
    \node[anchor=west, font=\scriptsize, text=dao]
      at (0, -0.5) {\textbf{绳}\; $x(t),\,m_*g$};
    \node[anchor=west, font=\scriptsize, text=green!50!black]
      at (0, -1.0) {\textbf{玉}\; $\Delta^3$};
    \node[anchor=west, font=\scriptsize, text=water]
      at (0, -1.5) {\textbf{锁}\; $\mathcal{X}$};
  \end{scope}
\end{tikzpicture}

\medskip

\caption{Agenticity diagram (bird's-eye view).  Layered
construction from bottom to top:
\textcolor{dao}{\textbf{绳}~Rope} (red): the gravitational field
$m_*g$ pointing into the page ($\bigotimes$)---the drift $A$ that
acts on every body.
\textcolor{water}{\textbf{锁}~Lock} (blue): the manifold net
$\mathcal{X}$---the viability kernel that supports the dynamics in
linear time.
\textcolor{green!50!black}{\textbf{玉}~Jade} (green): the Kakeya
tetrahedron $\Delta^3$---the bodies that exist
(\cref{prop:3b-kakeya}).
\textcolor{dao}{\textbf{绳}~Rope} (red, again): the current state
$x(t)$ on $\Delta^3$---where the agent acts.
\textcolor{sword}{\textbf{金}~Golden} (cyan): the pseudovector
$\hat{\omega} \in \mathfrak{so}(3)$ and the Lie derivative
$\nabla_{\hat{\omega}}P$---the sword that cuts: simultaneously the
gravity direction, the policy gradient, and the score function.}
\label{fig:agenticity}
\end{figure}

\subsection{The linearised system}

Linearise the controlled dynamics about a trajectory.  State
$\delta x \in \R^{n}$:
\begin{equation}\label{eq:3b-linear}
  \delta\dot{x} = A(t)\,\delta x + B\,\delta u,
  \qquad y = C(t)\,\delta x,
\end{equation}
where:
\begin{itemize}
  \item $A(t) = \partial f / \partial x$ encodes the
  \textbf{natural drift}---the uncontrolled dynamics of the
  transition kernel $P$.  In the gravitational instance, $A$ is the
  tidal tensor.  In a general MDP, $A$ is the linearised transition
  matrix $\partial P / \partial x$.
  \item $B$ is the \textbf{control input}---force enters only through
  the agent $\kappa$.  In the four-body instance,
  $B = [0;\; \ldots;\; 0;\; I_3/m_*]^\top$.
  \item $C(t) = \nabla_x \lambda_1$ is the \textbf{spectral
  observation}---the Fiedler eigenvector gradient.
\end{itemize}

The matrix $A$ is gravity.  In the MDP, $A$ is the drift of the
Markov chain---what happens without control.  It appears in all
three conditions below.

\subsection{The definition}

\begin{definition}[Agenticity]\label{def:agenticity}
An object $r$ in the execution graph $G$ has \textbf{agenticity}
with respect to king $\kappa$ iff:
\begin{enumerate}[label=\textup{(\Alph*)}]
  \item \textbf{Observability.}
  $r$ is spectrally visible: perturbations to $r$'s state produce
  detectable changes in $\lambda_1$.  Formally, the observability
  Gramian
  \[
    W_O(0,T) = \int_0^T \Phi(s,0)^\top C(s)^\top C(s)\,
    \Phi(s,0)\,ds
  \]
  is positive definite on the subspace containing $r$'s coordinates,
  where $\Phi$ is the state transition matrix of $A$.

  \item \textbf{Reachability.}
  $r$ is dynamically accessible: the reachable set
  $\mathcal{R}(T,\mathcal{U})$ of $\kappa$ under dynamics $A$ and
  box constraint $\mathcal{U} = [-\bar{u},\bar{u}]^3$ intersects
  the set of states where $\kappa$ can influence $r$'s edge weight
  $w_{\kappa r}$.  The dynamics $A$---gravity---determines this set.

  \item \textbf{Controllability.}
  $r$ is actuatable: the controllability Gramian
  \[
    W_C(0,T) = \int_0^T \Phi(0,s)\,B\,B^\top\,
    \Phi(0,s)^\top\,ds
  \]
  is positive definite on the subspace coupling $\kappa$ to $r$.
  Equivalently, $\kappa$ can modify $r$'s bypass capacity $c(e_r)$
  through admissible control.
\end{enumerate}
\end{definition}

\begin{theorem}[Kalman duality of agenticity]\label{thm:3b-kalman}
The pair $(A, C)$ is observable if and only if $(A^\top, C^\top)$
is controllable.  In physical terms:
\begin{quote}
\itshape
The ability to detect a sword (observability of $r$ through
$\lambda_1$) is dual to the ability to control it (controllability
of $r$'s bypass capacity).  The matrix $A$---gravity---is the
bridge.
\end{quote}
\end{theorem}

\begin{proof}
Standard Kalman duality: $(A,C)$ observable iff
$\mathrm{rank}[C;\; CA;\; CA^2;\; \ldots;\; CA^{n-1}] = n$, which
holds iff $\mathrm{rank}[C^\top,\; A^\top C^\top,\; \ldots,\;
(A^\top)^{n-1}C^\top] = n$, which is the controllability condition
for $(A^\top, C^\top)$.

Physically: $A$ propagates perturbations forward in time
(observation: a disturbance at $r$ ripples through the transition
kernel until $\lambda_1$ changes) and backward in time (control:
a thrust at $\kappa$ ripples through the same kernel until
$r$'s edge weight changes).  The same dynamics that make the sword
\emph{visible} make it \emph{controllable}.

In the MDP interpretation: $\nabla_{\hat{\omega}} P$ is
simultaneously the \emph{gravity direction} (how the uncontrolled
transition drifts), the \emph{policy gradient} (how the
transition changes under control), and the \emph{score function}
(how the MPPI proposal weights shift).  The Kalman duality says
these are the same object viewed forward and backward in time.
\end{proof}

\begin{corollary}[The drift provides reachability]\label{cor:3b-gravity}
Reachability is the bridge between observability and controllability,
and it is given by the drift $A$---gravity in the physical instance,
the uncontrolled transition in the MDP.  Specifically:
\begin{enumerate}[label=\textup{(\roman*)}]
  \item The reachable set $\mathcal{R}(T,\mathcal{U})$ is shaped by
  $A$: stronger coupling (larger $w_{ij}$) expands
  $\mathcal{R}$, weaker coupling contracts it.
  \item At equilibria of the drift ($\nabla_{q_*}V = 0$ in the
  gravitational instance; stationary points of $P$ in the MDP):
  the drift releases the agent, and all control authority is freed
  for the spectral kick.  Reachability is maximal.
  \item Away from equilibria, the drift consumes control
  authority: the agent must fight the drift before it can control
  $\lambda_1$.  Reachability contracts.
\end{enumerate}
\end{corollary}

\begin{remark}[草木竹石,皆可為劍]\label{rem:3b-caomuzhu}
In Jin Yong's \emph{The Return of the Condor Heroes}
(《神雕侠侣》), Dugu Qiubai's sword progression ends:
\emph{after forty, not bound by material---grass, wood, bamboo,
stone, all can become a sword.}

This is agenticity.  Every object satisfies condition~(A):
everything is observable (you can see grass).  Whether it satisfies
(B) and~(C) depends on the swordsman:
\begin{itemize}
  \item \textbf{(B) Reachability}: can the swordsman reach the
  object?  This is given by the dynamics---gravity, physics,
  proximity.  Dugu Qiubai's 内力 (internal force) extends his
  reachable set until grass is within reach.
  \item \textbf{(C) Controllability}: can the swordsman confer
  force through the object?  This costs $\bar{u} \ge \bar{u}^* > 0$
  (\cref{prop:3b-kakeya}).  Free agenticity does not exist.
\end{itemize}
The massless axiom (\cref{ax:massless}) says: grass has no intrinsic
swordness.  Agenticity is conferred by the controller through a
reachable, controllable path.  The dynamics $A$---gravity---is the
medium through which it is conferred.
\end{remark}

\begin{remark}[青冥宝剑胜龙泉 --- no gravity]\label{rem:3b-yujiaolong}
In Ang Lee's \emph{Crouching Tiger, Hidden Dragon}
(《卧虎藏龙》, 2000), 玉娇龙 (Jen Yu) takes the Green Destiny
and declares:
\begin{quote}\itshape
我乃是潇洒人间一剑仙,青冥宝剑胜龙泉。\\
任凭李俞江南鹤,都要低头求我怜。\\
沙漠飞来一条龙,神来无影去无踪。\\
今朝踏破峨眉顶,明日拔下武当峰!
\end{quote}
The declaration is made at the inn (客栈).  The moment 玉娇龙
draws the sword (拔剑), gravity vanishes \emph{for her}---she is
at the Lagrange point.  This is
\cref{cor:3b-gravity}\textup{(ii)}: $\nabla_{q_*}V = 0$, all
control authority is freed for the spectral kick, and the poem is
spoken from zero gravity.  The bamboo forest duel (竹林) is a
structurally distinct, later scene---玉娇龙 \emph{flies}
there too, but the phase transition already occurred at the inn
when the sword entered her hand.

When 玉娇龙 holds the 青冥宝剑, she is at the Lagrange point of
the 武林 (martial world): the sword confers agenticity, gravity
vanishes, and reachability is maximal.

李安 sees the physics.  The flying is not fantasy---it is the
optimal control policy at the switching surface.
\end{remark}

% ═══════════════════════════════════════════════════════════
\subsection{Dribble the three bodies}

The abstract diagram (\cref{fig:agenticity}) admits a concrete
physical instance that makes the three-term structure literally
visible: \emph{dribbling a ball}.

The ball's three principal inertia axes are the ``three bodies.''
Gravity pulls the ball toward the ground (the constraint boundary
$\lambda_1 = 0$).  The hand (the damper $\kappa$) strikes the ball
at the bottom of each bounce---the Lagrange point where
$\nabla_{q_*}V = 0$ in the co-moving frame.  The ball is a closed
manifold $S^2$; the hand's contact point moves on this manifold.

\begin{figure}[H]
\centering
\begin{tikzpicture}[scale=1.5, >=Stealth]
  % ═══ Layer 1 (锁 Lock, blue): hitting surface κ ═══
  % Ground plane — the constraint boundary (parallel view)
  \fill[water!8] (-4.0, -3.4) rectangle (4.0, -3.0);
  \draw[water, line width=0.8pt] (-4.0, -3.0) -- (4.0, -3.0);
  % Hand / paddle — the damper, also Lock (constraint surface)
  \fill[water!30, rounded corners=2pt]
    (-1.0, -3.0) rectangle (1.0, -2.75);
  \draw[water, line width=1pt, rounded corners=2pt]
    (-1.0, -3.0) rectangle (1.0, -2.75);
  \node[water, font=\footnotesize] at (0, -2.875) {$\kappa$};
  \node[water, font=\footnotesize, anchor=east] at (3.9, -3.2)
    {$\Sigma$};

  % ═══ Layer 2 (玉 Jade, green): volume S² ═══
  % Ball — the body that exists
  \begin{scope}[shift={(0, 0.2)}]
    \shade[ball color=green!30!white, opacity=0.35] (0,0) circle (1.8);
    \draw[green!60!black, line width=0.8pt] (0,0) circle (1.8);
    % Longitude lines
    \foreach \a in {-60,-20,20,60} {
      \draw[green!50!black, line width=0.2pt, opacity=0.4]
        (0,0) ellipse ({1.8*cos(\a)} and 1.8);
    }
    % Latitude lines
    \foreach \b in {-0.9, 0, 0.9} {
      \pgfmathsetmacro{\rad}{1.8*cos(asin(\b/1.8))}
      \draw[green!50!black, line width=0.2pt, opacity=0.4]
        (0, \b) ellipse ({\rad} and {0.15*\rad});
    }
    \node[green!60!black, font=\small, anchor=south west]
      at (1.5, 1.4) {$S^2$};
  \end{scope}

  % ═══ Layer 3 (绳 Rope, red): m*g + ground + state ═══
  % Gravity direction: point on ball + downward arrow
  \coordinate (CP) at (0, -1.6);
  \fill[dao, opacity=0.9] (CP) circle (2.5pt);
  \node[dao, font=\small, anchor=west] at (0.2, -1.75)
    {$x(t)$};
  % Gravity arrow
  \draw[->, dao, line width=1.8pt]
    (0, -1.6) -- (0, -2.65)
    node[midway, right=2pt, font=\small, text=dao] {$m_* g$};
  % Three inertia axes (the ``three bodies'')
  \draw[dao, line width=0.6pt, ->] (0,0.2) -- (1.6, 0.2)
    node[right, font=\scriptsize] {$I_1$};
  \draw[dao, line width=0.6pt, ->] (0,0.2) -- (0, 1.8)
    node[above, font=\scriptsize] {$I_2$};
  \draw[dao, line width=0.6pt, ->] (0,0.2) -- (-0.7, -0.5)
    node[below left, font=\scriptsize] {$I_3$};
  % Control force (bang arc, dashed)
  \draw[->, dao, line width=1.4pt, dashed]
    (0, -2.75) -- (0, -1.75)
    node[midway, left=2pt, font=\footnotesize, text=dao] {$u(t)$};

  % ═══ Layer 4 (金 Golden, cyan): operations ═══
  % ω̂ pseudovector
  \draw[->, sword, line width=1.6pt]
    (0, 0.2) -- (1.3, 1.6)
    node[right, font=\small, text=sword]
    {$\hat{\omega}$};
  % ∇_{ω̂} P
  \coordinate (Q) at (1.1, 1.35);
  \draw[->, sword, line width=1.2pt]
    (Q) -- ++(0.9, -0.8)
    node[right, font=\small, text=sword]
    {$\nabla_{\hat{\omega}}\!P$};

  % ── Bounce trajectory (parabolic arc) ────────────────
  \draw[dao, line width=0.5pt, dashed, opacity=0.5]
    plot[smooth, domain=-1.4:1.4, samples=30]
    ({\x + 2.5}, {-0.6*\x*\x + 1.0});

  % ── Arc type annotations ─────────────────────────────
  \node[font=\scriptsize, text=black!60, anchor=west,
        text width=2.8cm, align=left]
    at (-3.9, 2.0)
    {\textbf{singular arc}\\[1pt]
     ball in free flight\\
     $\rho = 0,\;\sim\!90\%$};
  \node[font=\scriptsize, text=black!60, anchor=west,
        text width=2.8cm, align=left]
    at (-3.9, -2.0)
    {\textbf{bang arc}\\[1pt]
     hand strikes ball\\
     $\rho \gg 1,\;\sim\!10\%$};

  % ── Brackets ─────────────────────────────────────────
  \draw[black!40, line width=0.4pt, decorate,
        decoration={brace, amplitude=4pt}]
    (-3.95, 1.8) -- (-3.95, -0.8)
    node[midway, left=5pt, font=\scriptsize, text=black!50] {};
  \draw[black!40, line width=0.4pt, decorate,
        decoration={brace, amplitude=4pt, mirror}]
    (-3.95, -1.2) -- (-3.95, -3.1)
    node[midway, left=5pt, font=\scriptsize, text=black!50] {};

  % ── Layer legend ─────────────────────────────────────
  \begin{scope}[shift={(4.0, 1.8)}]
    \node[anchor=west, font=\scriptsize, text=sword]
      at (0, 0) {\textbf{金}\; $\hat{\omega},\,\nabla_{\hat{\omega}}P$};
    \node[anchor=west, font=\scriptsize, text=dao]
      at (0, -0.5) {\textbf{绳}\; $x(t),\,m_*g,\,u(t)$};
    \node[anchor=west, font=\scriptsize, text=green!50!black]
      at (0, -1.0) {\textbf{玉}\; $S^2$};
    \node[anchor=west, font=\scriptsize, text=water]
      at (0, -1.5) {\textbf{锁}\; $\kappa,\,\Sigma$};
  \end{scope}
\end{tikzpicture}

\medskip

\caption{Dribble the three bodies (ground-level parallel view).
Layered construction:
\textcolor{water}{\textbf{锁}~Lock} (blue): the hitting surface
$\kappa$ (damper) and the ground plane $\Sigma = \{\rho = 1\}$---the
constraint boundary where contact is decided.
\textcolor{green!50!black}{\textbf{玉}~Jade} (green): the ball as
a closed manifold $S^2$---the body that exists.  The three
principal inertia axes $\textcolor{dao}{I_1, I_2, I_3}$ are the
``three bodies.''
\textcolor{dao}{\textbf{绳}~Rope} (red): gravity $m_*g$ (the
drift $A$), the current state $x(t)$ on $S^2$, and the control
force $u(t)$ (dashed, bang arc only).  The Rope acts: it binds
the ball to the ground via gravity, and the agent acts on the
ball via the paddle.
\textcolor{sword}{\textbf{金}~Golden} (cyan): the pseudovector
$\hat{\omega} \in \mathfrak{so}(3)$ and $\nabla_{\hat{\omega}}P$---
the sword: simultaneously the gravity direction, the policy
gradient, and the score function.  On singular arcs ($\rho = 0$,
${\sim}90\%$), the ball flies freely; on bang arcs ($\rho \gg 1$,
${\sim}10\%$), the hand strikes at the Lagrange point.
\Cref{fig:agenticity} is the bird's-eye view of the same structure;
this figure is the ground-level view.}
\label{fig:dribble}
\end{figure}

The dribble makes the three-term decomposition
(\cref{thm:3b-damper}) physically immediate:
\begin{enumerate}[label=\textup{(\Roman*)}]
  \item \textbf{Gravity}: the ball falls.
  \item \textbf{Least action}: the hand follows the ball's parabolic
  arc (singular arc, free flight).
  \item \textbf{Spectral kick}: the hand strikes the ball at the
  bottom of the bounce (bang arc), where $\nabla_{q_*}V = 0$ in the
  co-moving frame and all control authority is freed for the
  spectral kick.
\end{enumerate}
The bang-singular structure is visible to the naked eye: the hand
is in contact for ${\sim}10\%$ of the bounce cycle (bang) and in
free flight for ${\sim}90\%$ (singular).  The Kakeya condition
(\cref{prop:3b-kakeya}) says: to dribble the ball in every
direction on the court, the reachable set of the hand must be
full-dimensional in $\R^3$---one cannot dribble for free.

\subsection{Contact modes as force balance}

The dribble picture gives a precise physical definition of the
contact modes.  At each contact point, exactly two forces compete:
\emph{attraction} (gravity pulling the ball toward the ground) and
\emph{repulsion} (the hand pushing back).  Define the order
parameter
\begin{equation}\label{eq:3b-rho}
  \rho \;=\; \frac{F_{\mathrm{repulsion}}}{F_{\mathrm{attraction}}}.
\end{equation}
The contact mode is determined by $\rho$ alone:

\begin{center}
\begin{tabular}{@{}lcll@{}}
\toprule
\textbf{Regime} & $\rho$ & \textbf{Mode} & \textbf{Arc type} \\
\midrule
Attraction wins & $\rho < 1$ & Separating &
  Singular (free flight) \\
Phase transition & $\rho = 1$ & Sliding &
  Switching surface $\Sigma$ \\
Repulsion wins & $\rho > 1$ & Sticking &
  Bang (spectral kick) \\
Existence & $\rho \to \infty$ & Clamped &
  Attached body \\
\bottomrule
\end{tabular}
\end{center}

\noindent
The three-mode table of \cref{sec:3b-mppi} is recovered:
separating ($\rho < 1$, $w_i = 0$), sliding ($\rho \approx 1$,
$w_i = k_n + \mu k_n$), sticking ($\rho > 1$,
$w_i = k_n + k_t$).  But the dribble formulation adds the
fourth regime: \emph{existence}.  When $\rho \to \infty$, the
contact is no longer a contact---the ball is clamped to the hand.
The three ``bodies'' (inertia axes) become rigidly attached to the
agent $\kappa$, and $\lambda_1 \to \infty$.  The spectral
constraint $\lambda_1 \ge \epsilon$ is trivially satisfied.  No
control is needed: the object \emph{exists} as part of the agent.

The phase transition at $\rho = 1$ is the switching surface
$\Sigma$ of the bang-singular structure: attraction equals
repulsion, the Lagrange point of the grasp.  This is the point
where the hand must decide---push harder (bang) or release
(singular).  The entire MPPI sampling reduces from $3^n$ discrete
modes to sampling over the continuous parameter $\rho$ at each
contact point.

\begin{remark}[Two forces, one transition]
\label{rem:3b-twoforces}
Every contact in nature is a competition between attraction and
repulsion.  Gravity pulls; the hand pushes.  The Pauli exclusion
principle prevents collapse; electromagnetic attraction prevents
escape.  The dribble picture strips the contact mode classification
to its essence: there is only one phase transition, and it occurs
at $\rho = 1$.  Everything else---the friction cone, the signed
distance function, the contact Jacobian---is parameterisation of
the neighbourhood of this transition.
\end{remark}

% ═══════════════════════════════════════════════════════════
\section{Implementation: the physics-backend isomorphism}
\label{sec:3b-mujoco}

The gravity damper OCP admits two implementations that share the
same controller but differ in the physics backend.  The point is
not that both work---the point is that they \emph{must} work,
because the controller reads only $\rho$.

\subsection{Backend~1: hand-rolled Euler--Lagrange}

Three point masses in $\R^3$, interacting via Newtonian gravity.
The Euler--Lagrange equations are integrated by symplectic Euler:
\[
  M\ddot{q} \;=\; -\nabla V(q) \;+\; B\,u,
\]
where $V(q) = -\sum_{i<j} Gm_im_j/\|q_i - q_j\|$ and $B$
selects the damper.  The controller monitors $\lambda_1(L_G)$ and
applies the three-term control law \eqref{eq:3b-threeterm}.

The order parameter is the tidal coupling ratio:
\[
  \rho_{*j}
  \;=\; \frac{w_{*j}}{\bar{w}_{\mathrm{body}}}
  \;=\; \frac{Gm_* m_j / \|q_* - q_j\|^3}
             {\tfrac{1}{|E_0|}\sum_{(i,k)\in E_0} Gm_im_k / \|q_i - q_k\|^3},
\]
where $E_0$ are the body--body edges.  When $\rho_{*j} < 1$, the
damper is losing tidal authority over body $j$---the spectral kick
fires.

The simulation code is in \texttt{grjl/threebody\_damper.py} (v1.0,
$\lambda_1$-triggered) and \texttt{grjl2/threebody\_rho.py} (v2.0,
$\rho$-triggered):
\begin{verbatim}
  python grjl/threebody_damper.py --headless
  python grjl2/threebody_rho.py --solver reactive --headless
\end{verbatim}

\subsection{Backend~2: MuJoCo contact dynamics}
\label{sec:3b-dribble-mujoco}

Replace the hand-rolled gravity with MuJoCo's contact-aware
Euler--Lagrange integrator~\cite{mujoco}:
\[
  M(q)\ddot{q} + C(q,\dot{q})\dot{q}
  \;=\; \tau + J_c^\top\lambda_c,
\]
where $J_c^\top\lambda_c$ is the contact wrench computed by MuJoCo's
constraint solver (complementarity, friction cone, restitution---all
handled by \texttt{mj\_step()}).  The scene is the dribble of
\cref{fig:dribble}: a ball with non-uniform inertia
($I_1 \neq I_2 \neq I_3$, the ``three bodies'') bounced by a paddle
(the damper $\kappa$) above a ground plane.

The controller is a thin loop that reads the contact force from
MuJoCo's sensor data:
\[
  \rho(t) \;=\; \frac{F_{\mathrm{paddle}}(t)}{m\,g}.
\]
Everything else---the three-term decomposition, the bang-singular
switching, the spectral kick---is identical.  The controller does
not call any MuJoCo-specific function beyond reading sensor values
and writing actuator targets.

The simulation code is in \texttt{grjl2/dribble\_controller.py}:
\begin{verbatim}
  python grjl2/dribble_controller.py --headless
\end{verbatim}

\subsection{The isomorphism}

The two backends are related by the following dictionary:

\begin{center}
\begin{tabular}{@{}lll@{}}
\toprule
& \textbf{Backend 1} (gravity) & \textbf{Backend 2} (MuJoCo) \\
\midrule
EL equation & $M\ddot{q} = -\nabla V + Bu$ &
  $M(q)\ddot{q} + C\dot{q} = \tau + J_c^\top\lambda_c$ \\
Integrator & symplectic Euler & \texttt{mj\_step()} \\
``Three bodies'' & point masses $m_1,m_2,m_3$ &
  inertia axes $I_1,I_2,I_3$ \\
Damper $\kappa$ & controlled mass $m_*$ & paddle actuator \\
Constraint boundary & mass gap $\lambda_1 = 0$ & ground plane \\
$F_{\text{attraction}}$ & body--body tidal coupling &
  gravity $mg$ \\
$F_{\text{repulsion}}$ & damper--body tidal coupling &
  paddle contact force \\
\midrule
\textbf{Order parameter} & \multicolumn{2}{c}{$\rho
  = F_{\text{repulsion}} / F_{\text{attraction}}$} \\
\textbf{Control law} & \multicolumn{2}{c}{three-term:
  gravity $+$ least action $+$ spectral kick} \\
\textbf{Invariant} & \multicolumn{2}{c}{$\lambda_1 \ge \epsilon$
  (spectral gap)} \\
\bottomrule
\end{tabular}
\end{center}

\noindent
The top half of the table changes between backends; the bottom half
does not.  This is the precise claim: \emph{the controller is a
functor from the category of Euler--Lagrange systems to the category
of bang-singular control laws, with $\rho$ as the natural
transformation.}  Swapping the physics backend---from Newtonian
gravity to MuJoCo contact dynamics, or to any other system admitting
a force balance---changes only the source object.  The image
(the three-term control, the spectral gap, the bang-singular
structure) is preserved.

\begin{remark}[What MuJoCo actually contributes]
\label{rem:3b-mujoco-role}
MuJoCo contributes exactly one thing: the constraint solver that
computes $J_c^\top\lambda_c$.  This is the contact
Euler--Lagrange equation---the normal force, friction cone,
restitution, and complementarity conditions that determine whether
the ball bounces, slides, or sticks.  The controller does not need
to know \emph{how} MuJoCo computes this.  It reads the resulting
$F_{\mathrm{paddle}}$ from the sensor, divides by $mg$, and
obtains $\rho$.  The entire complexity of contact mechanics is
absorbed into the physics backend; the controller sees only the
order parameter.
\end{remark}

\begin{remark}[Validation chains]\label{rem:validation-chains}
\textcolor{sword}{Two independent chains validate the framework.
Each chain proceeds from axioms to observable, with the
physics backend as the only engine-specific component.}

\medskip\noindent
\textbf{Chain A} (Embodied: XML $\to$ 翻手为云覆手为雨).
\begin{gather*}
  \underbrace{\texttt{dribble\_\{down,up\}.xml}}_{\text{scene}}
  \;\xrightarrow{\;\texttt{mj\_step}(m,d)\;}
  \underbrace{q(t),\; \dot{q}(t)}_{\text{sensor}}
  \;\xrightarrow{\;\rho = |\ddot{q}_z/g + 1|\;}
  \underbrace{\text{DualDribbleController}}_{\text{three-term}}
  \\[4pt]
  \xrightarrow{\;\mathcal{G}\;}
  \underbrace{\text{拍球}\;\|\;\text{颠球}}_{\text{ground duality}}
\end{gather*}
The XML defines the entire physics: ball inertia, ground plane,
paddle actuators, contact parameters.  The call
\texttt{mj\_step(model, data)} is the \emph{only}
engine-specific line; replacing MuJoCo with any contact-aware
Euler--Lagrange integrator changes nothing above $\rho$.
\Cref{fig:agenticity} is the bird's-eye projection (翻手为云);
\cref{fig:dribble} is the ground-level projection (覆手为雨).

\medskip\noindent
\textbf{Chain B} (Analytical: three-body $\to$ spectral gap).
\[
  \underbrace{m_1, m_2, m_3,\; G}_{\text{masses + gravity}}
  \;\xrightarrow{\;\text{RK4 / symplectic}\;}
  \underbrace{q(t),\;\dot{q}(t)}_{\text{state}}
  \;\xrightarrow{\;\rho_{*j} = w_{*j}/\bar{w}\;}
  \underbrace{\text{SpectralPID / reactive}}_{\text{three-term}}
  \;\xrightarrow[\text{maintained}]{\;\lambda_1 \ge \epsilon\;}
  \underbrace{\text{grasp}}_{\text{viability}}
\]
No XML, no contact solver.  The masses interact via Newtonian
gravity; the integrator is hand-rolled; the controller monitors
$\lambda_1(L_G)$ and fires the spectral kick when $\rho_{\min} < 1$.

\medskip\noindent
\textbf{What the two chains share} (the bottom half of the
isomorphism table above):
\begin{enumerate}[label=(\roman*),nosep]
  \item the order parameter $\rho$,
  \item the three-term control law (gravity $+$ least action
    $+$ spectral kick),
  \item the spectral gap invariant $\lambda_1 \ge \epsilon$.
\end{enumerate}
\textbf{What differs} (the top half): the physics backend.
Chain~A uses MuJoCo contact dynamics; Chain~B uses Newtonian
gravity.  The controller is a functor: it maps any
Euler--Lagrange source to the same bang-singular target.
\end{remark}

\subsection{Force elimination: the kinematic reduction}
\label{sec:3b-force-elim}

\begin{center}
\itshape
I now demonstrate the degrees of freedom of the systems of the
world.
\end{center}

\medskip

\noindent
Newton demonstrated the frame of one system: $F = ma$, force
determines acceleration, acceleration determines the orbit.  We
invert the arrow.  Given the mass--inertia matrix $M(q)$---which
is geometry, fixed by the scene---acceleration determines force.
Force is therefore not a degree of freedom.  It is a
\emph{derived quantity}, eliminable from the control law
entirely.  The true degrees of freedom are kinematic:
$(q, \dot{q})$.

Every Euler--Lagrange system, regardless of the physics backend,
has the form
\begin{equation}\label{eq:3b-el-general}
  M(q)\,\ddot{q} \;=\; \tau_{\mathrm{ext}}(q, \dot{q}, u),
\end{equation}
where $M(q)$ is the mass--inertia matrix and
$\tau_{\mathrm{ext}}$ collects all external forces
(gravity, contact wrenches, control inputs, Coriolis/centrifugal
terms absorbed into the right-hand side).  Crucially, $M(q)$ is
\emph{determined by geometry alone}: it depends on the mass
distribution and the kinematic chain, both of which are fixed by
the scene description (the XML file, the URDF, or the gravitational
constants $Gm_im_j$).

\begin{theorem}[Force elimination]\label{thm:3b-force-elim}
Let $(q(t), \dot{q}(t))$ be a trajectory of the EL
system~\eqref{eq:3b-el-general} with known mass--inertia matrix
$M(q)$.  Then the generalised force is uniquely determined by the
kinematics:
\begin{equation}\label{eq:3b-force-recovery}
  \tau_{\mathrm{ext}}(t) \;=\; M(q(t))\,\ddot{q}(t).
\end{equation}
In particular, the order parameter $\rho$ is a function of
$(q, \dot{q}, \ddot{q})$ alone:
\begin{equation}\label{eq:3b-rho-kinematic}
  \rho(t) \;=\; \frac{F_{\mathrm{repulsion}}(t)}{F_{\mathrm{attraction}}(t)}
  \;=\; \frac{[\,M(q)\,\ddot{q}\,]_{\mathrm{contact}}}
             {[\,M(q)\,\ddot{q}\,]_{\mathrm{gravity}}},
\end{equation}
where $[\cdot]_{\mathrm{contact}}$ and
$[\cdot]_{\mathrm{gravity}}$ project onto the contact and
gravitational components of the wrench respectively.
\end{theorem}

\begin{proof}
$M(q)$ is symmetric positive-definite for any physical system
(it is the Hessian of the kinetic energy with respect to
$\dot{q}$).  Hence $M(q)$ is invertible at every configuration
$q$, and \eqref{eq:3b-el-general} gives a bijection between
$\ddot{q}$ and $\tau_{\mathrm{ext}}$ for fixed $(q, \dot{q})$.
The projection follows from the linearity of the wrench
decomposition.
\end{proof}

\noindent
The consequence is immediate:

\begin{corollary}[Kinematic controller]\label{cor:3b-kinematic}
The three-term control law (\cref{thm:3b-damper}) can be
implemented entirely in kinematic variables
$(q, \dot{q})$ without ever computing or commanding
forces.  The controller sets position targets $q^{\mathrm{des}}$;
the physics backend converts these to forces via the EL equation;
the resulting contact forces are \emph{observed} (not commanded)
to compute $\rho$.
\end{corollary}

\noindent
This is the structure of both implementations:

\begin{center}
\begin{tabular}{@{}lccc@{}}
\toprule
\textbf{Stage} & \textbf{Variable} & \textbf{Who computes} &
  \textbf{Domain} \\
\midrule
1. Controller output & $q^{\mathrm{des}}(t)$
  & controller & kinematic \\
2. Actuator force & $\tau_a = k_p(q^{\mathrm{des}} - q)$
  & physics backend & dynamic \\
3. Contact force & $F_c = J_c^\top\lambda_c$
  & physics backend & dynamic \\
4. Order parameter & $\rho = F_c / (mg)$
  & controller (read) & kinematic \\
\bottomrule
\end{tabular}
\end{center}

\noindent
Force appears only in stages~2 and~3, both internal to the physics
backend.  The controller's interface is purely kinematic: it writes
$q^{\mathrm{des}}$ and reads $\rho$.  Force is a
\emph{latent variable}---it mediates between the controller and the
constraint, but is never part of the control law itself.

\begin{remark}[Force as a gauge variable]
\label{rem:3b-gauge}
The relationship between $\ddot{q}$ and $\tau$ via $M(q)$ is
analogous to a gauge transformation in field theory: the physics is
in the acceleration (the curvature); the force is the connection
(the potential).  Different physics backends choose different
``gauges''---Newtonian gravity uses $\nabla V$, MuJoCo uses
$J_c^\top\lambda_c$---but the observable $\rho$ is gauge-invariant.
This is why the same controller works on any EL backend: it reads
the gauge-invariant quantity.
\end{remark}

\begin{remark}[Reconciliation of the three $\rho$ definitions]
\label{rem:3b-rho-reconcile}
The three definitions of~$\rho$ are gauge-equivalent in the
sense of \cref{rem:3b-gauge}.  They differ in reference frame,
not in physics:
\begin{enumerate}[label=(\roman*),nosep]
  \item \emph{Contact force ratio} (\cref{eq:3b-rho}):
  $\rho = F_{\mathrm{rep}}/F_{\mathrm{att}}$.
  Canonical definition.  Zero during free flight (no contact).
  \item \emph{Tidal coupling ratio}
  (\cref{sec:3b-dribble-mujoco}):
  $\rho_{*j} = w_{*j}/\bar{w}_{\mathrm{body}}$.
  Positive during free flight (gravity persists).  This is
  a \emph{proxy} for the canonical~$\rho$: it predicts when
  contact will occur, using the tidal ratio as a leading indicator.
  \item \emph{Kinematic form} (\cref{eq:3b-rho-kinematic}):
  $\rho = |\ddot{q}_z/g + 1|$.  Algebraically equivalent
  to~(i) via the force elimination theorem
  (\cref{thm:3b-force-elim}).
\end{enumerate}
All three agree at the switching surface~$\Sigma$: $\rho = 1$
is the same event regardless of gauge.  The tidal proxy~(ii)
is informative between contacts precisely because it monitors
the approach to~$\Sigma$ before contact forces materialise.
Chain~B (\cref{rem:validation-chains}) uses the proxy to
predict $\Sigma$-crossing; the controller's response is
triggered by the gauge-invariant $\rho = 1$.
\end{remark}

\subsection{Ground duality: 拍球 and 颠球}
\label{sec:3b-ground-duality}

The force elimination theorem reveals a deeper symmetry.  Consider
two dribbling configurations:

\begin{center}
\begin{tabular}{@{}lcc@{}}
\toprule
& \textbf{拍球} (dribble down) & \textbf{颠球} (juggle up) \\
\midrule
Hand position & above ball & below ball \\
Strike direction & downward ($-\hat{z}$)
  & upward ($+\hat{z}$) \\
Gravity pulls ball & toward ground (below)
  & toward hand (below) \\
Bounce surface & ground (below) & hand itself \\
\bottomrule
\end{tabular}
\end{center}

\noindent
Define the \emph{ground reflection} $\mathcal{G}$:
\begin{equation}\label{eq:3b-ground-dual}
  \mathcal{G}: \quad
  z \;\mapsto\; -z, \qquad
  g \;\mapsto\; -g, \qquad
  F_{\mathrm{hand}} \;\mapsto\; -F_{\mathrm{hand}}.
\end{equation}
Under $\mathcal{G}$, the EL equation transforms as
\[
  M\ddot{q} = \tau_{\mathrm{ext}}
  \quad\xmapsto{\;\mathcal{G}\;}
  \quad M\ddot{q}' = \tau_{\mathrm{ext}}',
\]
where $\ddot{q}' = -\ddot{q}_z$ in the vertical component and
$\tau_{\mathrm{ext}}' = -\tau_{\mathrm{ext},z}$.  The mass
matrix $M(q)$ is invariant (it depends on mass distribution,
not on the direction of gravity).

\begin{proposition}[Ground duality]\label{prop:3b-ground-dual}
The order parameter $\rho$ is invariant under the ground
reflection~$\mathcal{G}$:
\[
  \rho \;=\; \frac{|F_{\mathrm{hand}}|}{m\,|g|}
  \;=\; \frac{|{-F_{\mathrm{hand}}}|}{m\,|{-g}|}
  \;=\; \rho'.
\]
The three-term control law, the bang-singular structure, and the
spectral gap constraint are all preserved.
\end{proposition}

\begin{proof}
$\rho$ depends only on magnitudes: $|F_{\mathrm{repulsion}}|$
and $|F_{\mathrm{attraction}}|$.  The reflection $\mathcal{G}$
negates both, leaving the ratio unchanged.  The switching surface
$\Sigma = \{\rho = 1\}$ is therefore $\mathcal{G}$-invariant,
and the bang-singular decomposition on either side of $\Sigma$
is preserved.
\end{proof}

We now make the duality brutally precise.  Define the \emph{primal}
and \emph{dual} optimal control problems.

\begin{definition}[Primal--dual pair]\label{def:3b-primal-dual}
Fix the initial state $x_0 = (q(0), \dot{q}(0))$, the horizon $T$,
the spectral gap $\epsilon > 0$, and the actuation bound $\bar{u}$.

\medskip\noindent
\textbf{Primal problem $\mathsf{P}$} (拍球: hand above, palm down).
Gravity $g = -|g|\hat{z}$, hand force
$F_{\mathrm{hand}} = -|F|\hat{z}$ (downward strike), ground plane
at $z = 0$:
\begin{equation}\label{eq:3b-primal}
  J_{\mathsf{P}}^*
  \;=\; \min_{u \in \mathcal{U}} \int_0^T \!\Bigl[
    \mathcal{L}(q, \dot{q}) + \tfrac{\alpha}{2}\|u\|^2
  \Bigr] dt
  \quad\text{s.t.}\quad
  \lambda_1\bigl(L_G(q(t))\bigr) \ge \epsilon
  \;\;\forall\, t.
\end{equation}

\medskip\noindent
\textbf{Dual problem $\mathsf{D}$} (颠球: hand below, palm up).
Apply the ground reflection $\mathcal{G}$
(\cref{eq:3b-ground-dual}) to every quantity in $\mathsf{P}$:
gravity $g' = +|g|\hat{z}$, hand force
$F_{\mathrm{hand}}' = +|F|\hat{z}$ (upward strike), effective
ground at $z = z_{\mathrm{hand}}$:
\begin{equation}\label{eq:3b-dual}
  J_{\mathsf{D}}^*
  \;=\; \min_{u' \in \mathcal{U}} \int_0^T \!\Bigl[
    \mathcal{L}'(q', \dot{q}') + \tfrac{\alpha}{2}\|u'\|^2
  \Bigr] dt
  \quad\text{s.t.}\quad
  \lambda_1\bigl(L_G(q'(t))\bigr) \ge \epsilon
  \;\;\forall\, t.
\end{equation}
\end{definition}

\begin{theorem}[Strong duality: 翻手为云覆手为雨]
\label{thm:3b-strong-duality}
Let $u^*(t)$ solve the primal problem~$\mathsf{P}$.  Then
$u'(t) = \mathcal{G}\,u^*(t)$ solves the dual problem~$\mathsf{D}$,
and the optimal values coincide:
\begin{equation}\label{eq:3b-strong-dual}
  J_{\mathsf{P}}^* \;=\; J_{\mathsf{D}}^*.
\end{equation}
Moreover, the three-term decomposition, the bang-singular structure,
and the switching surface $\Sigma = \{\rho = 1\}$ are preserved
identically.  In particular:
\begin{enumerate}[label=\textup{(\roman*)}]
  \item The costates satisfy $p'(t) = \mathcal{G}\,p(t)$;
  \item The spectral multiplier satisfies $\mu'(t) = \mu(t)$;
  \item The optimal control satisfies $u'^*(t) =
    \mathrm{sat}_{\bar{u}}\bigl(-\frac{1}{\alpha m_*}\,
    p'_{\dot{q}_*}(t)\bigr)$, i.e.\ the same saturation law;
  \item The bang fraction $\beta = |\{t : \|u^*\| = \bar{u}\}|/T$
    is the same in both problems;
  \item The order parameter $\rho(t)$ is pointwise identical
    (\cref{prop:3b-ground-dual}).
\end{enumerate}
\end{theorem}

\begin{proof}
$\mathcal{G}$ is an involution ($\mathcal{G}^2 = \mathrm{id}$) that
acts on the extended state-costate space
$(x, p, u, \mu) \in T^*\mathcal{M} \times \mathcal{U} \times
\R_{\ge 0}$.  We verify that $\mathcal{G}$ preserves every
ingredient of the Pontryagin system.

\emph{Step~1: Lagrangian invariance.}
The kinetic energy $T = \frac{1}{2}\sum m_i \|\dot{q}_i\|^2$ is
quadratic in velocities; $\mathcal{G}$ negates $\dot{q}_z$, so
$\|\dot{q}\|^2$ is invariant.  The potential
$V = -\sum G m_i m_j / \|q_i - q_j\|$ depends on pairwise
distances; $\mathcal{G}$ negates all $z$-components simultaneously,
so $\|q_i - q_j\|$ is invariant.  Hence
$\mathcal{L}' = \mathcal{L}$.

\emph{Step~2: Control cost invariance.}
$\|u'\|^2 = \|\mathcal{G}\,u\|^2 = \|u\|^2$ since $\mathcal{G}$
is an isometry on $\R^3$ (it negates one component).

\emph{Step~3: Spectral constraint invariance.}
The graph Laplacian $L_G$ depends on edge weights
$w_{ij} = G m_i m_j / \|q_i - q_j\|^3$, which depend only on
pairwise distances.  By Step~1, these are $\mathcal{G}$-invariant.
Hence $\lambda_1(L_G(q')) = \lambda_1(L_G(q))$, and the constraint
$\lambda_1 \ge \epsilon$ maps to itself.

\emph{Step~4: Hamiltonian covariance.}
The control Hamiltonian transforms as
\[
  \mathcal{H}'(x',p',u',\mu') = \mathcal{H}(x,p,u,\mu)
\]
because every term is $\mathcal{G}$-invariant (Steps~1--3) and the
symplectic pairing $p \cdot f(x,u)$ transforms covariantly under
the canonical extension of $\mathcal{G}$.  Therefore, if $(x,p,u,\mu)$
satisfies the PMP necessary conditions, so does
$(x',p',u',\mu') = (\mathcal{G}\,x, \mathcal{G}\,p,
\mathcal{G}\,u, \mu)$.

\emph{Step~5: Optimality.}
Since $J[\mathcal{G}\,u] = J[u]$ (Steps~1--2), the minimum values
coincide.  Items (i)--(v) follow from the covariance of the
Pontryagin system: the costates, multiplier, saturation, bang
fraction, and $\rho$ are all determined by the Hamiltonian flow,
which commutes with $\mathcal{G}$.
\end{proof}

\begin{remark}[What the strong duality theorem says]
\label{rem:3b-strong-duality}
\textcolor{sword}{%
Equation~\eqref{eq:3b-strong-dual} is the mathematical content of
翻手为云覆手为雨.  It is not a metaphor.  The primal problem
$\mathsf{P}$ is 拍球 (dribble down: palm over, clouds gather).
The dual problem $\mathsf{D}$ is 颠球 (juggle up: palm under, rain
falls).  The ground reflection $\mathcal{G}$ is the duality map.
Strong duality $J_{\mathsf{P}}^* = J_{\mathsf{D}}^*$ says:
\emph{it costs exactly the same to dribble down as to juggle up.}
The five invariants (i)--(v) say: not only the cost, but every
structural feature of the solution---costates, multiplier,
switching surface, bang fraction, order parameter---is preserved
under flipping the hand.  There is no ``easier'' direction.
The controller that reads $\rho$ cannot distinguish $\mathsf{P}$
from $\mathsf{D}$.  This is what Du~Fu meant by 纷纷轻薄何须数:
the transitions come and go, but the invariant $\rho$ persists,
and it does not know which way is up.}
\end{remark}

\noindent
拍球 and 颠球 are therefore the \emph{same controller viewed from
opposite sides of the ground plane}.  The ``ground'' is not a
physical surface---it is the constraint boundary $\phi = 0$ in the
signed distance field.  In~拍球, the ground is below
($\phi = z_{\mathrm{ball}} - z_{\mathrm{ground}}$); in~颠球,
the effective ``ground'' is the hand itself, and gravity serves
as the restoring force that returns the ball to the hand after
the kick.  The duality is:

\begin{center}
\begin{tabular}{@{}lcc@{}}
\toprule
& \textbf{拍球} & \textbf{颠球} \\
\midrule
Constraint boundary & ground plane & hand surface \\
Restoring force & ground reaction & gravity \\
Kick direction & $-\hat{z}$ (hand pushes down)
  & $+\hat{z}$ (hand pushes up) \\
Free flight & ball rises after bounce
  & ball falls after kick \\
$\rho = 0$ & ball in air, no contact & ball in air, no contact \\
$\rho > 1$ & hand striking ball down & hand striking ball up \\
Controller & \multicolumn{2}{c}{identical modulo $\mathcal{G}$} \\
\bottomrule
\end{tabular}
\end{center}

\begin{remark}[The ground is not a place]
\label{rem:3b-ground}
In the abstract framework, ``the ground'' is the zero level set
of the signed distance function $\phi$, i.e.\ the constraint
boundary where $\lambda_1 = 0$ if the damper fails.  It is not
a physical surface---it is the \emph{locus of loss of control}.
The ground duality says: the agent can work from either side of
this locus, pushing toward it or pulling away from it, and the
control law is the same.  What matters is $|\rho|$, not
the sign of $g$.
\end{remark}

\begin{remark}[翻手为云,覆手为雨]
\label{rem:3b-fanshou}

\begin{center}\itshape
翻手作云覆手雨,纷纷轻薄何须数。

\medskip
\normalfont\small
Flip the hand---clouds.  Turn it over---rain.

Those who come and go so lightly, why bother counting them?

\hfill ---~杜甫 (Du Fu), 《贫交行》
\end{center}

\medskip\noindent
Du Fu wrote of fair-weather friends.  We read it as a theorem
about the ground reflection $\mathcal{G}$.

\medskip\noindent
\textbf{The hand is the agent.}  In~拍球 the palm faces down:
the hand pushes the ball into the ground, and the ground pushes
it back.  In~颠球 the palm faces up: the hand catches the ball
from below, and gravity pulls it back.  Same hand.
Same five fingers.  Same musculature.  The only difference is the
sign of $\hat{z}$.  This is the ground reflection
\eqref{eq:3b-ground-dual}: $z \mapsto -z$, $g \mapsto -g$,
$F_{\mathrm{hand}} \mapsto -F_{\mathrm{hand}}$.

\medskip\noindent
\textbf{翻手 is \cref{fig:agenticity}.}  Bird's-eye view: you
look down at the manifold from above.  The gravitational field
points into the page~($\bigotimes$).  The hand is between you
and the ball---palm up, gathering clouds.  The layers build
upward toward you:
\begin{enumerate}[label=\arabic*., nosep, leftmargin=2em]
  \item \textcolor{dao}{\textbf{绳}} (Rope): the gravity field
    $m_*g$, pointing away from you.
  \item \textcolor{water}{\textbf{锁}} (Lock): the manifold net
    $\mathcal{X}$ that catches the state.
  \item \textcolor{green!50!black}{\textbf{玉}} (Jade): the
    tetrahedron $\Delta^3$ that exists.
  \item \textcolor{dao}{\textbf{绳}} (Rope): the state $x(t)$,
    where the agent stands.
  \item \textcolor{sword}{\textbf{金}} (Golden): the sword
    $\hat{\omega}$ and $\nabla_{\hat{\omega}}P$---the operation.
\end{enumerate}

\medskip\noindent
\textbf{覆手 is \cref{fig:dribble}.}  Turn the hand over.  Ground
level, parallel view: you stand beside the ball and look across.
Gravity points down the page~($\downarrow$).  The hand is above
or below the ball---palm down, making rain.  The layers stack from
ground to sky:
\begin{enumerate}[label=\arabic*., nosep, leftmargin=2em]
  \item \textcolor{water}{\textbf{锁}} (Lock): the hitting surface
    $\kappa$ and the ground $\Sigma$.
  \item \textcolor{green!50!black}{\textbf{玉}} (Jade): the ball
    $S^2$ that exists.
  \item \textcolor{dao}{\textbf{绳}} (Rope): $m_*g$ and $x(t)$
    and $u(t)$---the forces that bind and act.
  \item \textcolor{sword}{\textbf{金}} (Golden): $\hat{\omega}$
    and $\nabla_{\hat{\omega}}P$---the same sword.
\end{enumerate}

\medskip\noindent
\textbf{The four characters are $\mathcal{G}$-invariant.}
Under the ground reflection, every layer maps to itself:

\begin{center}
\begin{tabular}{@{}clcc@{}}
\toprule
& \textbf{Layer}
  & \textbf{翻手} (\cref{fig:agenticity})
  & \textbf{覆手} (\cref{fig:dribble}) \\
\midrule
\textcolor{sword}{\textbf{金}} & sword / operation
  & $\hat{\omega},\;\nabla_{\hat{\omega}}P$
  & $\hat{\omega},\;\nabla_{\hat{\omega}}P$ \\
\textcolor{dao}{\textbf{绳}} & rope / act
  & $m_*g\;(\bigotimes),\;x(t)$
  & $m_*g\;(\downarrow),\;x(t),\;u(t)$ \\
\textcolor{green!50!black}{\textbf{玉}} & jade / exist
  & $\Delta^3$ & $S^2$ \\
\textcolor{water}{\textbf{锁}} & lock / support
  & $\mathcal{X}$ & $\kappa,\;\Sigma$ \\
\bottomrule
\end{tabular}
\end{center}

\noindent
The \textcolor{sword}{金}~(Golden) row is identical: the operations
do not know which side of the ground they are on.  This is the
content of force elimination (\cref{thm:3b-force-elim}): the
controller reads $(q, \dot{q}, \ddot{q})$, and kinematics is
$\mathcal{G}$-invariant.  The \textcolor{dao}{绳}~(Rope) row
gains $u(t)$ in the ground view because the control force becomes
visible as a separate arrow; in the bird's-eye view it is absorbed
into the state trajectory.  The \textcolor{green!50!black}{玉}~(Jade)
row changes shape ($\Delta^3 \leftrightarrow S^2$) but not
substance: both are the bodies that exist.  The
\textcolor{water}{锁}~(Lock) row changes instantiation
($\mathcal{X} \leftrightarrow \kappa$) but not function: both
are the constraint surface that the dynamics rests on.

\medskip\noindent
\textbf{纷纷轻薄何须数.}
\emph{Those who come and go so lightly, why bother counting them.}
Du Fu's dismissal of the fickle is the bang-singular ratio.  On
singular arcs (${\sim}90\%$ of the cycle), the ball is in
weightless flight: $\rho = 0$, no contact, no force, no
counting needed.  The ball drifts, lightly, through the air---
轻薄, as Du Fu says.  These arcs are free: they cost nothing.

The drama is in the ${\sim}10\%$.  On bang arcs, $\rho \gg 1$:
the hand strikes, the switching surface is crossed, the spectral
kick fires.  But even these transitions are instantaneous---sharp,
not gradual.  They come and go too quickly to count.  The control
law does not count bounces; it reads $\rho$ and switches.  何须数.

What endures is not the transitions.  What endures is the
invariant: $\rho = F_{\mathrm{repulsion}} / F_{\mathrm{attraction}}$.
The hand flips; the clouds become rain; the ratio persists.
翻手为云,覆手为雨: the duality is not a symmetry to be
admired---it is a \emph{gauge redundancy to be eliminated}.
The ground reflection $\mathcal{G}$ is the gauge transformation.
The order parameter $\rho$ is the gauge-invariant observable.
The controller that reads $\rho$ does not know, and need not know,
whether the palm faces up or down.
\end{remark}

\begin{remark}[Vibrational friction and the spectral gap]
\label{rem:3b-superconductor}
Place an ice block on a high-frequency oscillating table.
Friction acts at every instant, but the net force averages to
zero over each period: the block does not move.
This is \emph{vibrational friction averaging}---a classical
instance of force elimination (\cref{thm:3b-force-elim}).
The force (a gauge variable) cancels; the energy dissipation
(a gauge-invariant quantity) does not: friction is still
converting kinetic energy to heat at every instant.

The dribbling controller of~\cref{sec:3b-dribble-mujoco} is
the same mechanism: the ball oscillates between contact
($\rho > 1$) and free flight ($\rho = 0$), the contact forces
average over the cycle, and the controller maintains the
position through bang-singular switching.
Energy is dissipated at every bounce.

Superconductivity achieves the stronger result.  In a normal
metal, electrons scatter off lattice vibrations (phonons)---
friction---producing resistance.  Below the critical temperature
$T_c$, the \emph{same phonons} mediate Cooper pairing: the
vibration that caused friction now binds electrons into a
coherent condensate.  The BCS ground state opens a spectral
gap $\Delta > 0$.  Below the gap, no scattering states
exist: dissipation itself ceases, not merely its average.

The structural distinction is the spectral gap:
\begin{center}
\begin{tabular}{@{}lcc@{}}
\toprule
& \textbf{Classical (ice/dribble)}
& \textbf{Quantum (superconductor)} \\
\midrule
Vibration & oscillating table / bounce
  & phonon \\
Net force & averages to zero & zero (gap) \\
Dissipation & \textcolor{dao}{persists} (heat)
  & \textcolor{water}{ceases} ($\Delta > 0$) \\
$\lambda_1$ & maintained externally
  ($\bar{u} \geq \bar{u}^*$)
  & self-maintained
  (condensate) \\
Paper analogue & pre-sword (envelope persists)
  & resolution (sword eliminated) \\
\bottomrule
\end{tabular}
\end{center}
The spectral gap $\lambda_1 > 0$ of~\cref{thm:massgap}
is the same object as the superconducting gap $\Delta > 0$
in BCS theory: both are the condition under which the system
resists perturbation without dissipation.
The gravity damper maintains $\lambda_1 \geq \epsilon$ by
active control (external energy input); the superconductor
maintains $\Delta > 0$ by internal coherence (Cooper
condensate).  In the language of~\cref{def:3b-ustar}: the
classical system pays $\bar{u}^* > 0$; the quantum system
achieves $\bar{u}^* = 0$ below~$T_c$.  Hugo's triviality
(\cref{thm:alien-triviality}) predicts when the gap closes:
at $d = 4$ in the thermodynamic limit, i.e., at the critical
dimension of our spacetime.
\end{remark}

\subsection{Results}

\begin{description}
  \item[Backend~1 (gravity).]  Without damper: $\lambda_1$ decays
  to zero within ${\sim}100$ steps.  With damper: $\rho$ crosses
  $1.0$ at each bang-singular transition (16 phase crossings over
  $T = 8$\,s).  The PMP solver achieves ${\sim}16\%$ bang fraction,
  close to the ${\sim}10\%$ prediction.  Cost is finite and
  concentrated on bang arcs.

  \item[Backend~2 (dribble).]  The ball completes 33 hand-ball
  contacts over $T = 10$\,s (steady dribbling at ${\sim}3.3$\,Hz).
  The $\rho(t)$ time series shows clear spikes at each bounce:
  $\rho = 0$ during free flight (singular arc),
  $\rho \gg 1$ at contact ($\rho_{\max} \approx 437$, bang arc),
  with the transition at $\rho = 1$.  Bang fraction
  ${\sim}8\%$; singular arcs dominate (${\sim}92\%$).
  The three-term structure is visible to the naked eye.

  \item[Backend~2b (trajectory tracking).]  The kinematic
  controller~(\cref{cor:3b-kinematic}) steers the dribbled ball along
  a prescribed planar trajectory (circle, $R = 0.3$\,m,
  $T_{\mathrm{period}} = 6$\,s).  At each strike, the paddle offsets
  by $\delta_{\mathrm{xy}} = \alpha\,(x_{\mathrm{traj}} -
  x_{\mathrm{ball}})$, $\alpha = 0.15$; between strikes, the hand
  tracks the ball with a clamped correction
  ($\|\delta\| \le 0.05$\,m) so it never wanders beyond catching
  range.  Both 拍球 and 颠球 complete the full $T = 10$\,s;
  mean tracking error $\bar{e}_{\mathrm{xy}} \approx 0.29$\,m
  (拍球), $0.39$\,m (颠球).  The dribble becomes
  \emph{manipulation}: transportation of the object along an
  arbitrary path, under the same controller and the same
  $\rho$-based switching law.
\end{description}

\subsection{Tracking error: the three irreducible gaps}
\label{sec:3b-tracking-error}

The mean tracking error $\bar{e}_{\mathrm{xy}} > 0$ has a
positive lower bound that cannot be eliminated by tuning
the controller gains.  Three distinct mechanisms contribute,
each corresponding to a different structural limitation.

\paragraph{Gap~1: the Nyquist constraint (discrete impulse).}
The ball can only be steered \emph{at each bounce}.  Between
bounces it is ballistic: the controller can reposition the paddle
but cannot exert any force on the ball (singular arc,
$\rho = 0$).  If the ball bounces $N$ times per trajectory
period $T_{\mathrm{traj}}$, each correction subtends an arc
$\Delta\theta = 2\pi / N$.  The tracking error is bounded below
by the chord length:
\begin{equation}\label{eq:3b-nyquist}
  e_{\mathrm{xy}}
  \;\ge\; R\,\bigl(1 - \cos(\pi/N)\bigr)
  \;\approx\; \frac{\pi^2 R}{2N^2}
  \qquad (N \gg 1).
\end{equation}
This is the Nyquist constraint applied to impulsive control:
the control bandwidth is the bounce rate, not the physics rate.
To halve the tracking error one must quadruple the number of
bounces per period---equivalently, quadruple the dribbling
frequency or halve the trajectory speed.

\paragraph{Gap~2: the contact-model mismatch (irreducible).}
No physics engine exactly models the switching surface
$\Sigma = \{\rho = 1\}$.  Coulomb friction is non-smooth
(discontinuous at $\dot{q}_t = 0$); real friction is
non-convex (Stribeck effect, adhesion, surface deformation).
Every simulator regularises friction differently:
\begin{itemize}
  \item MuJoCo: convex relaxation via \texttt{solref}/\texttt{solimp};
  \item Bullet: iterative impulse with Baumgarte stabilisation;
  \item ODE: friction-pyramid approximation.
\end{itemize}
The result is that the tangential impulse during a strike
does not exactly match the paddle offset direction.  The
coefficient of restitution is emergent from the solver
parameters, not directly set.  This gap is \emph{irreducible
in simulation}: it can only be closed by experiment.

\paragraph{Gap~3: the discretisation floor (finite $\Delta t$).}
The kinematic $\rho$ is computed from finite differences of
velocity:
\[
  \ddot{q}_z(t_k)
  \;\approx\; \frac{\dot{q}_z(t_k) - \dot{q}_z(t_{k-1})}{\Delta t}.
\]
With $\Delta t = 0.002$\,s, the resulting $\rho$ has numerical
noise of order $\mathcal{O}(\Delta t)$, smoothed by the
exponential moving average filter ($\alpha = 0.5$) which
introduces a detection lag of ${\sim}2$--$3$ timesteps.
This is the floating-point leakage: the $\rho = 1$ surface
acquires finite width $\sim\!\Delta t$ in the kinematic
estimate, even though the underlying physics has a sharp
transition.

\begin{remark}[The sim-to-real gap lives at $\Sigma$]
\label{rem:3b-sim2real}
Away from the switching surface ($\rho \ll 1$, free flight),
the simulation is kinematically exact: the ball follows a
parabola, and no contact model is invoked.  At $\Sigma$
($\rho = 1$, contact), all three gaps converge.
The sim-to-real gap is therefore \emph{localised at the
phase transition}.  This has a precise analogue: the
discretisation timestep $\Delta t$ sets the minimum
observable time unit, while the bounce period $T_b$ sets the
\emph{reflex arc length}---the interval between successive
control actions.  The ratio $T_b / \Delta t \approx 150$--$250$
measures how many observations the agent collects per reflex
arc.  The agent sees the transition clearly (many samples at
$\Sigma$), but can only \emph{act} on it once per bounce.
\end{remark}

% ═══════════════════════════════════════════════════════════
\section*{Closing: from Newton to the agentic solution}
\addcontentsline{toc}{section}{Closing: from Newton to the agentic
solution}

Newton demonstrated the frame of the two-body world: Keplerian
orbits, closed-form, deterministic.  The three-body problem resisted
analytical solution for three centuries.  The reason, in our
framework, is structural: $K_3$ has no king, so mean-field detection
fails and no passive observer can guarantee viability.

The agentic solution does not resolve the analytical intractability---
it dissolves it.  The agent does not predict trajectories; it
maintains the spectral gap.  The complete chain is:

\begin{center}
\begin{tabular}{@{}rl@{}}
\textbf{Newton} & Two-body frame: closed-form solution. \\
\textbf{Diagnosis} & $K_3$ has no cut vertex $\Rightarrow$ no king
  $\Rightarrow$ $\lambda_1 \to 0$. \\
\textbf{Design} & Gravity damper: three-term decomposition
  (\cref{thm:3b-damper}). \\
\textbf{Pontryagin} & Backend solver: maximum principle $+$ PDP
  \cite{pdp}. \\
\textbf{MPPI} & $3^n \to \mathrm{poly}(n)$: sampling replaces
  enumeration (\cref{thm:3b-mppi}). \\
\textbf{Cheeger} & $\lambda_1 > \epsilon \Rightarrow$
  polynomial mixing time (\cref{prop:3b-mixing}). \\
\textbf{Kakeya} & $\bar{u}^* > 0$ strictly: free stability does not
  exist (\cref{prop:3b-kakeya}). \\
\textbf{Agenticity} & $O \leftrightarrow C$ (Kalman), $R$
  given by $A$ (drift / gravity) (\cref{thm:3b-kalman}). \\
\textbf{MDP} & $P(x' \mid x, u)$: $\nabla_{\hat{\omega}}P$ =
  gravity = policy gradient = score (\cref{fig:agenticity}). \\
\textbf{$\rho$} & Physics-backend isomorphism: same controller,
  any EL (\cref{sec:3b-mujoco}). \\
\textbf{$\Sigma$} & Sim-to-real gap localised at phase transition
  (\cref{sec:3b-tracking-error}).
\end{tabular}
\end{center}

The three-body problem has no analytical solution.  It has an
agentic one.  The agentic solution is not specific to celestial
mechanics: $P$ is any transition kernel, $A$ is any drift,
$\lambda_1$ is any spectral gap.  Agenticity is not intrinsic---it
is conferred by the controller through a reachable, controllable
path, and the drift is the medium.  草木竹石,皆可為劍.


% ── Bibliography ──────────────────────────────────────────
\begin{thebibliography}{10}
\bibitem{aubin} J.-P.~Aubin, \emph{Viability Theory}, Birkh\"auser,
1991.
\bibitem{aubincellina} J.-P.~Aubin and A.~Cellina,
\emph{Differential Inclusions: Set-Valued Maps and Viability Theory},
Grundlehren der mathematischen Wissenschaften~\textbf{264},
Springer-Verlag, 1984.
\bibitem{shiji} Sima Qian, \emph{Shiji} (Records of the Grand
Historian), c.~94~BCE.
\bibitem{dumu} Du Mu, \emph{A Fang Gong Fu} (Rhapsody on the Epang
Palace), 825~CE.
\bibitem{diestel} R.~Diestel, \emph{Graph Theory}, 5th ed., Springer,
2017.
\bibitem{cheeger} J.~Cheeger, A lower bound for the smallest eigenvalue
of the Laplacian, in \emph{Problems in Analysis}, Princeton Univ.\
Press, 1970, pp.~195--199.
\bibitem{mohar} B.~Mohar, Isoperimetric numbers of graphs,
\emph{J.~Combin.\ Theory Ser.~B} \textbf{47} (1989), 274--291.
\bibitem{hearn} R.~A.~Hearn and E.~D.~Demaine, PSPACE-completeness of
sliding-block puzzles and other problems through the nondeterministic
constraint logic model of computation, \emph{Theoret.\ Comput.\ Sci.}\
\textbf{343} (2005), 72--96.
\bibitem{beauvoir} S.~de~Beauvoir, \emph{Le Deuxi\`eme Sexe},
Gallimard, 1949.
\bibitem{liushang} Liu Shang (刘商), 胡笳十八拍 (Eighteen Songs of a
Nomad Flute), c.~773~CE. Collected in Guo Maoqian (郭茂倩),
\emph{Yuefu Shiji} (《乐府诗集》), c.~1100~CE.
\bibitem{rorex} R.~A.~Rorex and W.~Fong, \emph{Eighteen Songs of a
Nomad Flute: The Story of Lady Wen-chi. A Fourteenth-Century Handscroll
in The Metropolitan Museum of Art}, Metropolitan Museum of Art, 1974.
\bibitem{mulanshi} Anonymous, 木兰辞 (Ballad of Mulan), Northern
Dynasties (c.~5th--6th century~CE). Collected in Guo Maoqian (郭茂倩),
\emph{Yuefu Shiji} (《乐府诗集》), c.~1100~CE.
\bibitem{xiyouji} Wu Cheng'en (吴承恩), \emph{Xi You Ji} (《西游记》,
Journey to the West), c.~1592.
\bibitem{shuihu} Shi Nai'an (施耐庵), \emph{Shui Hu Zhuan}
(《水浒传》, Water Margin), c.~1400.
\bibitem{zhongxian} Feng Junqi (冯军旗), \emph{Zhong Xian Ganbu}
(《中县干部》, Zhong County's Cadre), Ph.D.\ dissertation,
Peking University, 2010.
\bibitem{reinhartrogoff} C.~M.~Reinhart and K.~S.~Rogoff,
\emph{This Time Is Different: Eight Centuries of Financial Folly},
Princeton University Press, 2009.
\bibitem{duminilcopin-raoufi-tassion}
H.~Duminil-Copin, A.~Raoufi, and V.~Tassion,
Sharp phase transition for the random-cluster and Potts models via
decision trees,
\emph{Ann.\ of Math.}\ \textbf{189} (2019), 75--99.
\bibitem{beffara-dc}
V.~Beffara and H.~Duminil-Copin,
The self-dual point of the two-dimensional random-cluster model is
critical for $q \geq 1$,
\emph{Probab.\ Theory Related Fields} \textbf{153} (2012), 511--542.
\bibitem{aizenman-dc}
M.~Aizenman and H.~Duminil-Copin,
Marginal triviality of the scaling limits of critical 4D Ising and
$\lambda\varphi^4_4$ models,
\emph{Ann.\ of Math.}\ \textbf{194} (2021), 163--235.
\bibitem{adiprasito-huh-katz}
K.~Adiprasito, J.~Huh, and E.~Katz,
Hodge theory for combinatorial geometries,
\emph{Ann.\ of Math.}\ \textbf{188} (2018), 381--452.
\bibitem{huh-chromatic}
J.~Huh,
Milnor numbers of projective hypersurfaces and the chromatic polynomial
of graphs,
\emph{J.~Amer.\ Math.\ Soc.}\ \textbf{25} (2012), 907--927.
\bibitem{huh-katz}
J.~Huh and E.~Katz,
Log-concavity of characteristic polynomials and the Bergman fan of
matroids,
\emph{Math.\ Ann.}\ \textbf{354} (2012), 1103--1116.
\bibitem{branden-huh}
P.~Br\"and\'en and J.~Huh,
Lorentzian polynomials,
\emph{Ann.\ of Math.}\ \textbf{192} (2020), 821--891.
\bibitem{koukoulopoulos-maynard}
D.~Koukoulopoulos and J.~Maynard,
On the Duffin--Schaeffer conjecture,
\emph{Ann.\ of Math.}\ \textbf{192} (2020), 251--307.
\bibitem{viazovska}
M.~S.~Viazovska,
The sphere packing problem in dimension~8,
\emph{Ann.\ of Math.}\ \textbf{185} (2017), 991--1015.
\bibitem{cohn-elkies}
H.~Cohn and N.~Elkies,
New upper bounds on sphere packings~I,
\emph{Ann.\ of Math.}\ \textbf{157} (2003), 689--714.
\bibitem{cohn-kumar-miller-radchenko-viazovska}
H.~Cohn, A.~Kumar, S.~D.~Miller, D.~Radchenko, and M.~Viazovska,
Universal optimality of the $E_8$ and Leech lattices and interpolation
formulas,
\emph{Ann.\ of Math.}\ \textbf{196} (2022), 983--1082.
\bibitem{razborov-rudich}
A.~A.~Razborov and S.~Rudich,
Natural proofs,
\emph{J.~Comput.\ System Sci.}\ \textbf{55} (1997), 24--35.
\end{thebibliography}

\end{document}
